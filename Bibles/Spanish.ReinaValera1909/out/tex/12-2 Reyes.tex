\hypertarget{section}{%
\section{1}\label{section}}

\bibverse{1} Después de la muerte de Achâb rebelóse Moab contra Israel.
\bibverse{2} Y Ochôzías cayó por las celosías de una sala de la casa que
tenía en Samaria; y estando enfermo envió mensajeros, y díjoles: Id, y
consultad á Baal-zebub dios de Ecrón, si tengo de sanar de esta mi
enfermedad. \bibverse{3} Entonces el ángel de Jehová habló á Elías
Thisbita, diciendo: Levántate, y sube á encontrarte con los mensajeros
del rey de Samaria, y les dirás: ¿No hay Dios en Israel, que vosotros
vais á consultar á Baal-zebub dios de Ecrón? \bibverse{4} Por tanto así
ha dicho Jehová: Del lecho en que subiste no descenderás, antes morirás
ciertamente. Y Elías se fué. \bibverse{5} Y como los mensajeros se
volvieron al rey, él les dijo: ¿Por qué pues os habéis vuelto?
\bibverse{6} Y ellos le respondieron: Encontramos un varón que nos dijo:
Id, y volveos al rey que os envió, y decidle: Así ha dicho Jehová: ¿No
hay Dios en Israel, que tú envías á consultar á Baal-zebub dios de
Ecrón? Por tanto, del lecho en que subiste no descenderás, antes morirás
de cierto. \bibverse{7} Entonces él les dijo: ¿Qué hábito era el de
aquel varón que encontrasteis, y os dijo tales palabras? \bibverse{8} Y
ellos le respondieron: Un varón velloso, y ceñía sus lomos con un cinto
de cuero. Entonces él dijo: Elías Thisbita es. \bibverse{9} Y envió
luego á él un capitán de cincuenta con sus cincuenta, el cual subió á
él; y he aquí que él estaba sentado en la cumbre del monte. Y él le
dijo: Varón de Dios, el rey ha dicho que desciendas. \bibverse{10} Y
Elías respondió, y dijo al capitán de cincuenta: Si yo soy varón de
Dios, descienda fuego del cielo, y consúmate con tus cincuenta. Y
descendió fuego del cielo, que lo consumió á él y á sus cincuenta.
\bibverse{11} Volvió el rey á enviar á él otro capitán de cincuenta con
sus cincuenta; y hablóle, y dijo: Varón de Dios, el rey ha dicho así:
Desciende presto. \bibverse{12} Y respondióle Elías, y dijo: Si yo soy
varón de Dios, descienda fuego del cielo, y consúmate con tus cincuenta.
Y descendió fuego del cielo, que lo consumió á él y á sus cincuenta.
\bibverse{13} Y volvió á enviar el tercer capitán de cincuenta con sus
cincuenta: y subiendo aquel tercer capitán de cincuenta, hincóse de
rodillas delante de Elías, y rogóle, diciendo: Varón de Dios, ruégote
que sea de valor delante de tus ojos mi vida, y la vida de estos tus
cincuenta siervos. \bibverse{14} He aquí ha descendido fuego del cielo,
y ha consumido los dos primeros capitanes de cincuenta, con sus
cincuenta; sea ahora mi vida de valor delante de tus ojos. \bibverse{15}
Entonces el ángel de Jehová dijo á Elías: Desciende con él; no hayas de
él miedo. Y él se levantó, y descendió con él al rey. \bibverse{16} Y
díjole: Así ha dicho Jehová: Pues que enviaste mensajeros á consultar á
Baal-zebub dios de Ecrón, ¿no hay Dios en Israel para consultar en su
palabra? No descenderás, por tanto, del lecho en que subiste, antes
morirás de cierto. \bibverse{17} Y murió conforme á la palabra de Jehová
que había hablado Elías: y reinó en su lugar Joram, en el segundo año de
Joram, hijo de Josaphat rey de Judá; porque Ochôzías no tenía hijo.
\bibverse{18} Y lo demás de los hechos de Ochôzías, ¿no está escrito en
el libro de las crónicas de los reyes de Israel?

\hypertarget{section-1}{%
\section{2}\label{section-1}}

\bibverse{1} Y aconteció que, cuando quiso Jehová alzar á Elías en un
torbellino al cielo, Elías venía con Eliseo de Gilgal. \bibverse{2} Y
dijo Elías á Eliseo: Quédate ahora aquí, porque Jehová me ha enviado á
Beth-el. Y Eliseo dijo: Vive Jehová, y vive tu alma, que no te dejaré.
Descendieron pues á Beth-el. \bibverse{3} Y saliendo á Eliseo los hijos
de los profetas que estaban en Beth-el, dijéronle: ¿Sabes como Jehová
quitará hoy á tu señor de tu cabeza? Y él dijo: Sí, yo lo sé; callad.
\bibverse{4} Y Elías le volvió á decir: Eliseo, quédate aquí ahora,
porque Jehová me ha enviado á Jericó. Y él dijo: Vive Jehová, y vive tu
alma, que no te dejaré. Vinieron pues á Jericó. \bibverse{5} Y
llegáronse á Eliseo los hijos de los profetas que estaban en Jericó, y
dijéronle: ¿Sabes cómo Jehová quitará hoy á tu señor de tu cabeza? Y él
respondió: Sí, yo lo sé; callad. \bibverse{6} Y Elías le dijo: Ruégote
que te quedes aquí, porque Jehová me ha enviado al Jordán. Y él dijo:
Vive Jehová, y vive tu alma, que no te dejaré. Fueron pues ambos á dos.
\bibverse{7} Y vinieron cincuenta varones de los hijos de los profetas,
y paráronse enfrente á lo lejos: y ellos dos se pararon junto al Jordán.
\bibverse{8} Tomando entonces Elías su manto, doblólo, é hirió las
aguas, las cuales se apartaron á uno y á otro lado, y pasaron ambos en
seco. \bibverse{9} Y como hubieron pasado, Elías dijo á Eliseo: Pide lo
que quieres que haga por ti, antes que sea quitado de contigo. Y dijo
Eliseo: Ruégote que las dos partes de tu espíritu sean sobre mí.
\bibverse{10} Y él le dijo: Cosa difícil has pedido. Si me vieres cuando
fuere quitado de ti, te será así hecho; mas si no, no. \bibverse{11} Y
aconteció que, yendo ellos hablando, he aquí, un carro de fuego con
caballos de fuego apartó á los dos: y Elías subió al cielo en un
torbellino. \bibverse{12} Y viéndolo Eliseo, clamaba: ¡Padre mío, padre
mío, carro de Israel y su gente de á caballo! Y nunca más le vió, y
trabando de sus vestidos, rompiólos en dos partes. \bibverse{13} Alzó
luego el manto de Elías que se le había caído, y volvió, y paróse á la
orilla del Jordán. \bibverse{14} Y tomando el manto de Elías que se le
había caído, hirió las aguas, y dijo: ¿Dónde está Jehová, el Dios de
Elías? Y así que hubo del mismo modo herido las aguas, apartáronse á uno
y á otro lado, y pasó Eliseo. \bibverse{15} Y viéndole los hijos de los
profetas que estaban en Jericó de la otra parte, dijeron: El espíritu de
Elías reposó sobre Eliseo. Y viniéronle á recibir, é inclináronse á él
hasta la tierra. \bibverse{16} Y dijéronle: He aquí hay con tus siervos
cincuenta varones fuertes: vayan ahora y busquen á tu señor; quizá lo ha
levantado el espíritu de Jehová, y lo ha echado en algún monte ó en
algún valle. Y él les dijo: No enviéis. \bibverse{17} Mas ellos le
importunaron, hasta que avergonzándose dijo: Enviad. Entonces ellos
enviaron cincuenta hombres, los cuales lo buscaron tres días, mas no lo
hallaron. \bibverse{18} Y cuando volvieron á él, que se había quedado en
Jericó, él les dijo: ¿No os dije yo que no fueseis? \bibverse{19} Y los
hombres de la ciudad dijeron á Eliseo: He aquí el asiento de esta ciudad
es bueno, como mi señor ve; mas las aguas son malas, y la tierra
enferma. \bibverse{20} Entonces él dijo: Traedme una botija nueva, y
poned en ella sal. Y trajéronsela. \bibverse{21} Y saliendo él á los
manaderos de las aguas, echó dentro la sal, y dijo: Así ha dicho Jehová:
Yo sané estas aguas, y no habrá más en ellas muerte ni enfermedad.
\bibverse{22} Y fueron sanas las aguas hasta hoy, conforme á la palabra
que habló Eliseo. \bibverse{23} Después subió de allí á Beth-el; y
subiendo por el camino, salieron los muchachos de la ciudad, y se
burlaban de él, diciendo: ¡Calvo, sube! ¡calvo, sube! \bibverse{24} Y
mirando él atrás, viólos, y maldíjolos en el nombre de Jehová. Y
salieron dos osos del monte, y despedazaron de ellos cuarenta y dos
muchachos. \bibverse{25} De allí fué al monte de Carmelo, y de allí
volvió á Samaria.

\hypertarget{section-2}{%
\section{3}\label{section-2}}

\bibverse{1} Y joram hijo de Achâb comenzó á reinar en Samaria sobre
Israel el año dieciocho de Josaphat rey de Judá; y reinó doce años.
\bibverse{2} E hizo lo malo en ojos de Jehová, aunque no como su padre y
su madre; porque quitó las estatuas de Baal que su padre había hecho.
\bibverse{3} Mas allegóse á los pecados de Jeroboam, hijo de Nabat, que
hizo pecar á Israel; y no se apartó de ellos. \bibverse{4} Entonces Mesa
rey de Moab era propietario de ganados, y pagaba al rey de Israel cien
mil corderos y cien mil carneros con sus vellones. \bibverse{5} Mas
muerto Achâb, el rey de Moab se rebeló contra el rey de Israel.
\bibverse{6} Y salió entonces de Samaria el rey Joram, é inspeccionó á
todo Israel. \bibverse{7} Y fué y envió á decir á Josaphat rey de Judá:
El rey de Moab se ha rebelado contra mí: ¿irás tú conmigo á la guerra
contra Moab? Y él respondió: Iré, porque como yo, así tú; como mi
pueblo, así tu pueblo; como mis caballos, así también tus caballos.
\bibverse{8} Y dijo: ¿Por qué camino iremos? Y él respondió: Por el
camino del desierto de Idumea. \bibverse{9} Partieron pues el rey de
Israel, y el rey de Judá, y el rey de Idumea; y como anduvieron rodeando
por el desierto siete días de camino, faltóles el agua para el ejército,
y para las bestias que los seguían. \bibverse{10} Entonces el rey de
Israel dijo: ¡Ah! que ha llamado Jehová estos tres reyes para
entregarlos en manos de los Moabitas. \bibverse{11} Mas Josaphat dijo:
¿No hay aquí profeta de Jehová, para que consultemos á Jehová por él? Y
uno de los siervos del rey de Israel respondió y dijo: Aquí está Eliseo
hijo de Saphat, que daba agua á manos á Elías. \bibverse{12} Y Josaphat
dijo: Este tendrá palabra de Jehová. Y descendieron á él el rey de
Israel, y Josaphat, y el rey de Idumea. \bibverse{13} Entonces Eliseo
dijo al rey de Israel: ¿Qué tengo yo contigo? Ve á los profetas de tu
padre, y á los profetas de tu madre. Y el rey de Israel le respondió:
No: porque ha juntado Jehová estos tres reyes para entregarlos en manos
de los Moabitas. \bibverse{14} Y Eliseo dijo: Vive Jehová de los
ejércitos, en cuya presencia estoy, que si no tuviese respeto al rostro
de Josaphat rey de Judá, no mirara á ti, ni te viera. \bibverse{15} Mas
ahora traedme un tañedor. Y mientras el tañedor tocaba, la mano de
Jehová fué sobre Eliseo. \bibverse{16} Y dijo: Así ha dicho Jehová:
Haced en este valle muchas acequias. \bibverse{17} Porque Jehová ha
dicho así: No veréis viento, ni veréis lluvia, y este valle será lleno
de agua, y beberéis vosotros, y vuestras bestias, y vuestros ganados.
\bibverse{18} Y esto es cosa ligera en los ojos de Jehová; dará también
á los Moabitas en vuestras manos. \bibverse{19} Y vosotros heriréis á
toda ciudad fortalecida y á toda villa hermosa, y talaréis todo buen
árbol, y cegaréis todas las fuentes de aguas, y destruiréis con piedras
toda tierra fértil. \bibverse{20} Y aconteció que por la mañana, cuando
se ofrece el sacrificio, he aquí vinieron aguas por el camino de Idumea,
y la tierra fué llena de aguas. \bibverse{21} Y todos los de Moab, como
oyeron que los reyes subían á pelear contra ellos, juntáronse desde
todos los que ceñían talabarte arriba, y pusiéronse en la frontera.
\bibverse{22} Y como se levantaron por la mañana, y lució el sol sobre
las aguas, vieron los de Moab desde lejos las aguas rojas como sangre;
\bibverse{23} Y dijeron: ¡Sangre es esta de espada! Los reyes se han
revuelto, y cada uno ha muerto á su compañero. Ahora pues, ¡Moab, á la
presa! \bibverse{24} Mas cuando llegaron al campo de Israel,
levantáronse los Israelitas é hirieron á los de Moab, los cuales huyeron
delante de ellos: siguieron empero hiriendo todavía á los de Moab.
\bibverse{25} Y asolaron las ciudades, y en todas las heredades fértiles
echó cada uno su piedra, y las llenaron; cegaron también todas las
fuentes de las aguas, y derribaron todos los buenos árboles; hasta que
en Kir-hareseth solamente dejaron sus piedras; porque los honderos la
cercaron, y la hirieron. \bibverse{26} Y cuando el rey de Moab vió que
la batalla lo vencía, tomó consigo setecientos hombres que sacaban
espada, para romper contra el rey de Idumea: mas no pudieron.
\bibverse{27} Entonces arrebató á su primogénito que había de reinar en
su lugar, y sacrificóle en holocausto sobre el muro. Y hubo grande enojo
en Israel; y retiráronse de él, y volviéronse á su tierra.

\hypertarget{section-3}{%
\section{4}\label{section-3}}

\bibverse{1} Una mujer, de las mujeres de los hijos de los profetas,
clamó á Eliseo, diciendo: Tu siervo mi marido es muerto; y tú sabes que
tu siervo era temeroso de Jehová: y ha venido el acreedor para tomarse
dos hijos míos por siervos. \bibverse{2} Y Eliseo le dijo: ¿Qué te haré
yo? Declárame qué tienes en casa. Y ella dijo: Tu sierva ninguna cosa
tiene en casa, sino una botija de aceite. \bibverse{3} Y él le dijo: Ve,
y pide para ti vasos prestados de todos tus vecinos, vasos vacíos, no
pocos. \bibverse{4} Entra luego, y cierra la puerta tras ti y tras tus
hijos; y echa en todos los vasos, y en estando uno lleno, ponlo aparte.
\bibverse{5} Y partióse la mujer de él, y cerró la puerta tras sí y tras
sus hijos; y ellos le llegaban los vasos, y ella echaba del aceite.
\bibverse{6} Y como los vasos fueron llenos, dijo á un hijo suyo: Tráeme
aún otro vaso. Y él dijo: No hay más vasos. Entonces cesó el aceite.
\bibverse{7} Vino ella luego, y contólo al varón de Dios, el cual dijo:
Ve, y vende el aceite, y paga á tus acreedores; y tú y tus hijos vivid
de lo que quedare. \bibverse{8} Y aconteció también que un día pasaba
Eliseo por Sunem; y había allí una mujer principal, la cual le constriñó
á que comiese del pan: y cuando por allí pasaba, veníase á su casa á
comer del pan. \bibverse{9} Y ella dijo á su marido: He aquí ahora, yo
entiendo que éste que siempre pasa por nuestra casa, es varón de Dios
santo. \bibverse{10} Yo te ruego que hagas una pequeña cámara de
paredes, y pongamos en ella cama, y mesa, y silla, y candelero, para que
cuando viniere á nosotros, se recoja en ella. \bibverse{11} Y aconteció
que un día vino él por allí, y recogióse en aquella cámara, y durmió en
ella. \bibverse{12} Entonces dijo á Giezi su criado: Llama á esta
Sunamita. Y como él la llamó, pareció ella delante de él. \bibverse{13}
Y dijo él á Giezi: Dile: He aquí tú has estado solícita por nosotros con
todo este esmero: ¿qué quieres que haga por ti? ¿has menester que hable
por ti al rey, ó al general del ejército? Y ella respondió: Yo habito en
medio de mi pueblo. \bibverse{14} Y él dijo: ¿Qué pues haremos por ella?
Y Giezi respondió: He aquí ella no tiene hijo, y su marido es viejo.
\bibverse{15} Dijo entonces: Llámala. Y él la llamó, y ella se paró á la
puerta. \bibverse{16} Y él le dijo: A este tiempo según el tiempo de la
vida, abrazarás un hijo. Y ella dijo: No, señor mío, varón de Dios, no
hagas burla de tu sierva. \bibverse{17} Mas la mujer concibió, y parió
un hijo á aquel tiempo que Eliseo le había dicho, según el tiempo de la
vida. \bibverse{18} Y como el niño fué grande, aconteció que un día
salió á su padre, á los segadores. \bibverse{19} Y dijo á su padre: ¡Mi
cabeza, mi cabeza! Y él dijo á un criado: Llévalo á su madre.
\bibverse{20} Y habiéndole él tomado, y traídolo á su madre, estuvo
sentado sobre sus rodillas hasta medio día, y murióse. \bibverse{21}
Ella entonces subió, y púsolo sobre la cama del varón de Dios, y
cerrándole la puerta, salióse. \bibverse{22} Llamando luego á su marido,
díjole: Ruégote que envíes conmigo á alguno de los criados y una de las
asnas, para que yo vaya corriendo al varón de Dios, y vuelva.
\bibverse{23} Y él dijo: ¿Para qué has de ir á él hoy? No es nueva luna,
ni sábado. Y ella respondió: Paz. \bibverse{24} Después hizo enalbardar
una borrica, y dijo al mozo: Guía y anda; y no me hagas detener para que
suba, sino cuando yo te lo dijere. \bibverse{25} Partióse pues, y vino
al varón de Dios al monte del Carmelo. Y cuando el varón de Dios la vió
de lejos, dijo á su criado Giezi: He aquí la Sunamita: \bibverse{26}
Ruégote que vayas ahora corriendo á recibirla, y dile: ¿Tienes paz? ¿y
tu marido, y tu hijo? Y ella dijo: Paz. \bibverse{27} Y luego que llegó
al varón de Dios en el monte, asió de sus pies. Y llegóse Giezi para
quitarla; mas el varón de Dios le dijo: Déjala, porque su alma está en
amargura, y Jehová me ha encubierto el motivo, y no me lo ha revelado.
\bibverse{28} Y ella dijo: ¿Pedí yo hijo á mi señor? ¿No dije yo, que no
me burlases? \bibverse{29} Entonces dijo él á Giezi: Ciñe tus lomos, y
toma mi bordón en tu mano, y ve; y si alguno te encontrare, no lo
saludes; y si alguno te saludare, no le respondas: y pondrás mi bordón
sobre el rostro del niño. \bibverse{30} Y dijo la madre del niño: Vive
Jehová, y vive tu alma, que no te dejaré. \bibverse{31} El entonces se
levantó, y siguióla. Y Giezi había ido delante de ellos, y había puesto
el bordón sobre el rostro del niño, mas ni tenía voz ni sentido; y así
se había vuelto para encontrar á Eliseo; y declaróselo, diciendo: El
mozo no despierta. \bibverse{32} Y venido Eliseo á la casa, he aquí el
niño que estaba tendido muerto sobre su cama. \bibverse{33} Entrando él
entonces, cerró la puerta sobre ambos, y oró á Jehová. \bibverse{34}
Después subió, y echóse sobre el niño, poniendo su boca sobre la boca de
él, y sus ojos sobre sus ojos, y sus manos sobre las manos suyas; así se
tendió sobre él, y calentóse la carne del joven. \bibverse{35}
Volviéndose luego, paseóse por la casa á una parte y á otra, y después
subió, y tendióse sobre él; y el joven estornudó siete veces, y abrió
sus ojos. \bibverse{36} Entonces llamó él á Giezi, y díjole: Llama á
esta Sunamita. Y él la llamó. Y entrando ella, él le dijo: Toma tu hijo.
\bibverse{37} Y así que ella entró, echóse á sus pies, é inclinóse á
tierra: después tomó su hijo, y salióse. \bibverse{38} Y Eliseo se
volvió á Gilgal. Había entonces grande hambre en la tierra. Y los hijos
de los profetas estaban con él, por lo que dijo á su criado: Pon una
grande olla, y haz potaje para los hijos de los profetas. \bibverse{39}
Y salió uno al campo á coger hierbas, y halló una como parra montés, y
cogió de ella una faldada de calabazas silvestres: y volvió, y cortólas
en la olla del potaje: porque no sabía lo que era. \bibverse{40} Echóse
después para que comieran los hombres; pero sucedió que comiendo ellos
de aquel guisado, dieron voces, diciendo: ¡Varón de Dios, la muerte en
la olla! Y no lo pudieron comer. \bibverse{41} El entonces dijo: Traed
harina. Y esparcióla en la olla, y dijo: Echa de comer á la gente. Y no
hubo más mal en la olla. \bibverse{42} Vino entonces un hombre de
Baal-salisa, el cual trajo al varón de Dios panes de primicias, veinte
panes de cebada, y trigo nuevo en su espiga. Y él dijo: Da á la gente
para que coman. \bibverse{43} Y respondió su sirviente: ¿Cómo he de
poner esto delante de cien hombres? Mas él tornó á decir: Da á la gente
para que coman, porque así ha dicho Jehová: Comerán, y sobrará.
\bibverse{44} Entonces él lo puso delante de ellos, y comieron, y
sobróles, conforme á la palabra de Jehová.

\hypertarget{section-4}{%
\section{5}\label{section-4}}

\bibverse{1} Naamán, general del ejército del rey de Siria, era gran
varón delante de su señor, y en alta estima, porque por medio de él
había dado Jehová salvamento á la Siria. Era este hombre valeroso en
extremo, pero leproso. \bibverse{2} Y de Siria habían salido cuadrillas,
y habían llevado cautiva de la tierra de Israel una muchacha; la cual
sirviendo á la mujer de Naamán, \bibverse{3} Dijo á su señora: Si rogase
mi señor al profeta que está en Samaria, él lo sanaría de su lepra.
\bibverse{4} Y entrando Naamán á su señor, declaróselo, diciendo: Así y
así ha dicho una muchacha que es de la tierra de Israel. \bibverse{5} Y
díjole el rey de Siria: Anda, ve, y yo enviaré letras al rey de Israel.
Partió pues él, llevando consigo diez talentos de plata, y seis mil
piezas de oro, y diez mudas de vestidos. \bibverse{6} Tomó también
letras para el rey de Israel, que decían así: Luego en llegando á ti
estas letras, sabe por ellas que yo envío á ti mi siervo Naamán, para
que lo sanes de su lepra. \bibverse{7} Y luego que el rey de Israel leyó
las cartas, rasgó sus vestidos, y dijo: ¿Soy yo Dios, que mate y dé
vida, para que éste envíe á mí á que sane un hombre de su lepra?
Considerad ahora, y ved cómo busca ocasión contra mí. \bibverse{8} Y
como Eliseo, varón de Dios oyó que el rey de Israel había rasgado sus
vestidos, envió á decir al rey: ¿Por qué has rasgado tus vestidos? Venga
ahora á mí, y sabrá que hay profeta en Israel. \bibverse{9} Y vino
Naamán con sus caballos y con su carro, y paróse á las puertas de la
casa de Eliseo. \bibverse{10} Entonces Eliseo le envió un mensajero,
diciendo: Ve, y lávate siete veces en el Jordán, y tu carne se te
restaurará, y serás limpio. \bibverse{11} Y Naamán se fué enojado,
diciendo: He aquí yo decía para mí: Saldrá él luego, y estando en pie
invocará el nombre de Jehová su Dios, y alzará su mano, y tocará el
lugar, y sanará la lepra. \bibverse{12} Abana y Pharphar, ríos de
Damasco, ¿no son mejores que todas las aguas de Israel? Si me lavare en
ellos, ¿no seré también limpio? Y volvióse, y fuése enojado.
\bibverse{13} Mas sus criados se llegaron á él, y habláronle, diciendo:
Padre mío, si el profeta te mandara alguna gran cosa, ¿no la hicieras?
¿cuánto más, diciéndote: Lávate, y serás limpio? \bibverse{14} El
entonces descendió, y zambullóse siete veces en el Jordán, conforme á la
palabra del varón de Dios: y su carne se volvió como la carne de un
niño, y fué limpio. \bibverse{15} Y volvió al varón de Dios, él y toda
su compañía, y púsose delante de él, y dijo: He aquí ahora conozco que
no hay Dios en toda la tierra, sino en Israel. Ruégote que recibas algún
presente de tu siervo. \bibverse{16} Mas él dijo: Vive Jehová, delante
del cual estoy, que no lo tomaré. E importunándole que tomase, él nunca
quiso. \bibverse{17} Entonces Naamán dijo: Ruégote pues, ¿no se dará á
tu siervo una carga de un par de acémilas de aquesta tierra? porque de
aquí adelante tu siervo no sacrificará holocausto ni sacrificio á otros
dioses, sino á Jehová. \bibverse{18} En esto perdone Jehová á tu siervo:
que cuando mi señor entrare en el templo de Rimmón, y para adorar en él
se apoyare sobre mi mano, si yo también me inclinare en el templo de
Rimmón, si en el templo de Rimmón me inclino, Jehová perdone en esto á
tu siervo. \bibverse{19} Y él le dijo: Vete en paz. Partióse pues de él,
y caminó como el espacio de una milla. \bibverse{20} Entonces Giezi,
criado de Eliseo el varón de Dios, dijo entre sí: He aquí mi señor
estorbó á este Siro Naamán, no tomando de su mano las cosas que había
traído. Vive Jehová, que correré yo tras él, y tomaré de él alguna cosa.
\bibverse{21} Y siguió Giezi á Naamán: y como le vió Naamán que venía
corriendo tras él, apeóse del carro para recibirle, y dijo: ¿Va bien?
\bibverse{22} Y él dijo: Bien. Mi señor me envía á decir: He aquí
vinieron á mí en esta hora del monte de Ephraim dos mancebos de los
hijos de los profetas: ruégote que les des un talento de plata, y sendas
mudas de vestidos. \bibverse{23} Y Naamán dijo: Ruégote que tomes dos
talentos. Y él le constriñó, y ató dos talentos de plata en dos sacos, y
dos mudas de vestidos, y púsolo á cuestas á dos de sus criados, que lo
llevasen delante de él. \bibverse{24} Y llegado que hubo á un lugar
secreto, él lo tomó de mano de ellos, y guardólo en casa: luego mandó á
los hombres que se fuesen. \bibverse{25} Y él entró, y púsose delante de
su señor. Y Eliseo le dijo: ¿De dónde vienes, Giezi? Y él dijo: Tu
siervo no ha ido á ninguna parte. \bibverse{26} El entonces le dijo: ¿No
fué también mi corazón, cuando el hombre volvió de su carro á recibirte?
¿es tiempo de tomar plata, y de tomar vestidos, olivares, viñas, ovejas,
bueyes, siervos y siervas? \bibverse{27} La lepra de Naamán se te pegará
á ti, y á tu simiente para siempre. Y salió de delante de él leproso,
blanco como la nieve.

\hypertarget{section-5}{%
\section{6}\label{section-5}}

\bibverse{1} Los hijos de los profetas dijeron á Eliseo: He aquí, el
lugar en que moramos contigo nos es estrecho. \bibverse{2} Vamos ahora
al Jordán, y tomemos de allí cada uno una viga, y hagámonos allí lugar
en que habitemos. Y él dijo: Andad. \bibverse{3} Y dijo uno: Rogámoste
que quieras venir con tus siervos. Y él respondió: Yo iré. \bibverse{4}
Fuése pues con ellos; y como llegaron al Jordán, cortaron la madera.
\bibverse{5} Y aconteció que derribando uno un árbol, cayósele el hacha
en el agua; y dió voces, diciendo: ¡Ah, señor mío, que era emprestada!
\bibverse{6} Y el varón de Dios dijo: ¿Dónde cayó? Y él le mostró el
lugar. Entonces cortó él un palo, y echólo allí; é hizo nadar el hierro.
\bibverse{7} Y dijo: Tómalo. Y él tendió la mano, y tomólo. \bibverse{8}
Tenía el rey de Siria guerra contra Israel, y consultando con sus
siervos, dijo: En tal y tal lugar estará mi campamento. \bibverse{9} Y
el varón de Dios envió á decir al rey de Israel: Mira que no pases por
tal lugar, porque los Siros van allí. \bibverse{10} Entonces el rey de
Israel envió á aquel lugar que el varón de Dios había dicho y
amonestádole; y guardóse de allí, no una vez ni dos. \bibverse{11} Y el
corazón del rey de Siria fué turbado de esto; y llamando á sus siervos,
díjoles: ¿No me declararéis vosotros quién de los nuestros es del rey de
Israel? \bibverse{12} Entonces uno de los siervos dijo: No, rey señor
mío; sino que el profeta Eliseo está en Israel, el cual declara al rey
de Israel las palabras que tú hablas en tu más secreta cámara.
\bibverse{13} Y él dijo: Id, y mirad dónde está, para que yo envíe á
tomarlo. Y fuéle dicho: He aquí él está en Dothán. \bibverse{14}
Entonces envió el rey allá gente de á caballo, y carros, y un grande
ejército, los cuales vinieron de noche, y cercaron la ciudad.
\bibverse{15} Y levantándose de mañana el que servía al varón de Dios,
para salir, he aquí el ejército que tenía cercada la ciudad, con gente
de á caballo y carros. Entonces su criado le dijo: ¡Ah, señor mío! ¿qué
haremos? \bibverse{16} Y él le dijo: No hayas miedo: porque más son los
que están con nosotros que los que están con ellos. \bibverse{17} Y oró
Eliseo, y dijo: Ruégote, oh Jehová, que abras sus ojos para que vea.
Entonces Jehová abrió los ojos del mozo, y miró: y he aquí que el monte
estaba lleno de gente de á caballo, y de carros de fuego alrededor de
Eliseo. \bibverse{18} Y luego que los Siros descendieron á él, oró
Eliseo á Jehová, y dijo: Ruégote que hieras á esta gente con ceguedad. E
hiriólos con ceguedad, conforme al dicho de Eliseo. \bibverse{19}
Después les dijo Eliseo: No es este el camino, ni es esta la ciudad;
seguidme, que yo os guiaré al hombre que buscáis. Y guiólos á Samaria.
\bibverse{20} Y así que llegaron á Samaria, dijo Eliseo: Jehová, abre
los ojos de éstos, para que vean. Y Jehová abrió sus ojos, y miraron, y
halláronse en medio de Samaria. \bibverse{21} Y cuando el rey de Israel
los hubo visto, dijo á Eliseo: ¿Herirélos, padre mío? \bibverse{22} Y él
le respondió: No los hieras; ¿herirías tú á los que tomaste cautivos con
tu espada y con tu arco? Pon delante de ellos pan y agua, para que coman
y beban, y se vuelvan á sus señores. \bibverse{23} Entonces les fué
aparejada grande comida: y como hubieron comido y bebido, enviólos, y
ellos se volvieron á su señor. Y nunca más vinieron cuadrillas de Siria
á la tierra de Israel. \bibverse{24} Después de esto aconteció, que
Ben-adad rey de Siria juntó todo su ejército, y subió, y puso cerco á
Samaria. \bibverse{25} Y hubo grande hambre en Samaria, teniendo ellos
cerco sobre ella; tanto, que la cabeza de un asno era vendida por
ochenta piezas de plata, y la cuarta de un cabo de estiércol de palomas
por cinco piezas de plata. \bibverse{26} Y pasando el rey de Israel por
el muro, una mujer le dió voces, y dijo: Salva, rey señor mío.
\bibverse{27} Y él dijo: Si no te salva Jehová, ¿de dónde te tengo de
salvar yo? ¿del alfolí, ó del lagar? \bibverse{28} Y díjole el rey: ¿Qué
tienes? Y ella respondió: Esta mujer me dijo: Da acá tu hijo, y
comámoslo hoy, y mañana comeremos el mío. \bibverse{29} Cocimos pues mi
hijo, y le comimos. El día siguiente yo le dije: Da acá tu hijo, y
comámoslo. Mas ella ha escondido su hijo. \bibverse{30} Y como el rey
oyó las palabras de aquella mujer, rasgó sus vestidos, y pasó así por el
muro: y llegó á ver el pueblo el saco que traía interiormente sobre su
carne. \bibverse{31} Y él dijo: Así me haga Dios, y así me añada, si la
cabeza de Eliseo hijo de Saphat quedare sobre él hoy. \bibverse{32}
Estaba á la sazón Eliseo sentado en su casa, y con él estaban sentados
los ancianos: y el rey envió á él un hombre. Mas antes que el mensajero
viniese á él, dijo él á los ancianos: ¿No habéis visto cómo este hijo
del homicida me envía á quitar la cabeza? Mirad pues, y cuando viniere
el mensajero, cerrad la puerta, é impedidle la entrada: ¿no viene tras
él el ruido de los pies de su amo? \bibverse{33} Aun estaba él hablando
con ellos, y he aquí el mensajero que descendía á él; y dijo:
Ciertamente este mal de Jehová viene. ¿Para qué tengo de esperar más á
Jehová?

\hypertarget{section-6}{%
\section{7}\label{section-6}}

\bibverse{1} Dijo entonces Eliseo: Oid palabra de Jehová: Así dijo
Jehová: Mañana á estas horas valdrá el seah de flor de harina un siclo,
y dos seah de cebada un siclo, á la puerta de Samaria. \bibverse{2} Y un
príncipe sobre cuya mano el rey se apoyaba, respondió al varón de Dios,
y dijo: Si Jehová hiciese ahora ventanas en el cielo, ¿sería esto así? Y
él dijo: He aquí tú lo verás con tus ojos, mas no comerás de ello.
\bibverse{3} Y había cuatro hombres leprosos á la entrada de la puerta,
los cuales dijeron el uno al otro: ¿Para qué nos estamos aquí hasta que
muramos? \bibverse{4} Si tratáremos de entrar en la ciudad, por el
hambre que hay en la ciudad moriremos en ella; y si nos quedamos aquí,
también moriremos. Vamos pues ahora, y pasémonos al ejército de los
Siros: si ellos nos dieren la vida, viviremos; y si nos dieren la
muerte, moriremos. \bibverse{5} Levantáronse pues en el principio de la
noche, para irse al campo de los Siros; y llegando á las primeras
estancias de los Siros, no había allí hombre. \bibverse{6} Porque el
Señor había hecho que en el campo de los Siros se oyese estruendo de
carros, ruido de caballos, y estrépito de grande ejército; y dijéronse
los unos á los otros: He aquí el rey de Israel ha pagado contra nosotros
á los reyes de los Hetheos, y á los reyes de los Egipcios, para que
vengan contra nosotros. \bibverse{7} Y así se habían levantado y huído
al principio de la noche, dejando sus tiendas, sus caballos, sus asnos,
y el campo como se estaba; y habían huído por salvar las vidas.
\bibverse{8} Y como los leprosos llegaron á las primeras estancias,
entráronse en una tienda, y comieron y bebieron, y tomaron de allí
plata, y oro, y vestidos, y fueron, y escondiéronlo: y vueltos, entraron
en otra tienda, y de allí también tomaron, y fueron, y escondieron.
\bibverse{9} Y dijéronse el uno al otro: No hacemos bien: hoy es día de
buena nueva, y nosotros callamos: y si esperamos hasta la luz de la
mañana, nos alcanzará la maldad. Vamos pues ahora, entremos, y demos la
nueva en casa del rey. \bibverse{10} Y vinieron, y dieron voces á los
guardas de la puerta de la ciudad, y declaráronles, diciendo: Nosotros
fuimos al campo de los Siros, y he aquí que no había allí hombre, ni voz
de hombre, sino caballos atados, asnos también atados, y el campo como
se estaba. \bibverse{11} Y los porteros dieron voces, y declaráronlo
dentro, en el palacio del rey. \bibverse{12} Y levantóse el rey de
noche, y dijo á sus siervos: Yo os declararé lo que nos han hecho los
Siros. Ellos saben que tenemos hambre, y hanse salido de las tiendas y
escondídose en el campo, diciendo: Cuando hubieren salido de la ciudad,
los tomaremos vivos, y entraremos en la ciudad. \bibverse{13} Entonces
respondió uno de sus siervos, y dijo: Tomen ahora cinco de los caballos
que han quedado en la ciudad, (porque ellos también son como toda la
multitud de Israel que ha quedado en ella; también ellos son como toda
la multitud de Israel que ha perecido;) y enviemos, y veamos qué hay.
\bibverse{14} Tomaron pues dos caballos de un carro, y envió el rey tras
el campo de los Siros, diciendo: Id, y ved. \bibverse{15} Y ellos
fueron, y siguiéronlos hasta el Jordán: y he aquí, todo el camino estaba
lleno de vestidos y enseres que los Siros habían arrojado con la
premura. Y volvieron los mensajeros, é hiciéronlo saber al rey.
\bibverse{16} Entonces el pueblo salió, y saquearon el campo de los
Siros. Y fué vendido un seah de flor de harina por un siclo, y dos seah
de cebada por un siclo, conforme á la palabra de Jehová. \bibverse{17} Y
el rey puso á la puerta á aquel príncipe sobre cuya mano él se apoyaba:
y atropellóle el pueblo á la entrada, y murió, conforme á lo que había
dicho el varón de Dios, lo que habló cuando el rey descendió á él.
\bibverse{18} Aconteció pues de la manera que el varón de Dios había
hablado al rey, diciendo: Dos seah de cebada por un siclo, y el seah de
flor de harina será vendido por un siclo mañana á estas horas, á la
puerta de Samaria. \bibverse{19} A lo cual aquel príncipe había
respondido al varón de Dios, diciendo: Aunque Jehová hiciese ventanas en
el cielo, ¿pudiera ser eso? Y él dijo: He aquí tú lo verás con tus ojos,
mas no comerás de ello. \bibverse{20} Y vínole así; porque el pueblo le
atropelló á la entrada, y murió.

\hypertarget{section-7}{%
\section{8}\label{section-7}}

\bibverse{1} Y habló Eliseo á aquella mujer á cuyo hijo había hecho
vivir, diciendo: Levántate, vete tú y toda tu casa á vivir donde
pudieres; porque Jehová ha llamado el hambre, la cual vendrá también
sobre la tierra siete años. \bibverse{2} Entonces la mujer se levantó, é
hizo como el varón de Dios le dijo: y partióse ella con su familia, y
vivió en tierra de los Filisteos siete años. \bibverse{3} Y como fueron
pasados los siete años, la mujer volvió de la tierra de los Filisteos:
después salió para clamar al rey por su casa, y por sus tierras.
\bibverse{4} Y había el rey hablado con Giezi, criado del varón de Dios,
diciéndole: Ruégote que me cuentes todas las maravillas que ha hecho
Eliseo. \bibverse{5} Y contando él al rey cómo había hecho vivir á un
muerto, he aquí la mujer, á cuyo hijo había hecho vivir, que clamaba al
rey por su casa y por sus tierras. Entonces dijo Giezi: Rey señor mío,
esta es la mujer, y este es su hijo, al cual Eliseo hizo vivir.
\bibverse{6} Y preguntando el rey á la mujer, ella se lo contó. Entonces
el rey le dió un eunuco, diciéndole: Hazle volver todas las cosas que
eran suyas, y todos los frutos de las tierras desde el día que dejó el
país hasta ahora. \bibverse{7} Eliseo se fué luego á Damasco, y Ben-adad
rey de Siria estaba enfermo, al cual dieron aviso, diciendo: El varón de
Dios ha venido aquí. \bibverse{8} Y el rey dijo á Hazael: Toma en tu
mano un presente, y ve á recibir al varón de Dios, y consulta por él á
Jehová, diciendo: ¿Tengo de sanar de esta enfermedad? \bibverse{9} Tomó
pues Hazael en su mano un presente de todos los bienes de Damasco,
cuarenta camellos cargados, y saliólo á recibir: y llegó, y púsose
delante de él, y dijo: Tu hijo Ben-adad, rey de Siria, me ha enviado á
ti, diciendo: ¿Tengo de sanar de esta enfermedad? \bibverse{10} Y Eliseo
le dijo: Ve, dile: Seguramente vivirás. Empero Jehová me ha mostrado que
él ha de morir ciertamente. \bibverse{11} Y el varón de Dios le volvió
el rostro afirmadamente, y estúvose así una gran pieza; y lloró el varón
de Dios. \bibverse{12} Entonces díjole Hazael: ¿Por qué llora mi señor?
Y él respondió: Porque sé el mal que has de hacer á los hijos de Israel:
á sus fortalezas pegarás fuego, y á sus mancebos matarás á cuchillo, y
estrellarás á sus niños, y abrirás á sus preñadas. \bibverse{13} Y
Hazael dijo: ¿Por qué? ¿es tu siervo perro, que hará esta gran cosa? Y
respondió Eliseo: Jehová me ha mostrado que tú has de ser rey de Siria.
\bibverse{14} Y él se partió de Eliseo, y vino á su señor, el cual le
dijo: ¿Qué te ha dicho Eliseo? Y él respondió: Díjome que seguramente
vivirás. \bibverse{15} El día siguiente tomó un paño basto, y metiólo en
agua, y tendiólo sobre el rostro de Ben-adad, y murió: y reinó Hazael en
su lugar. \bibverse{16} En el quinto año de Joram hijo de Achâb rey de
Israel, y siendo Josaphat rey de Judá, comenzó á reinar Joram hijo de
Josaphat rey de Judá. \bibverse{17} De treinta y dos años era cuando
comenzó á reinar, y ocho años reinó en Jerusalem. \bibverse{18} Y anduvo
en el camino de los reyes de Israel, como hizo la casa de Achâb, porque
una hija de Achâb fué su mujer; é hizo lo malo en ojos de Jehová.
\bibverse{19} Con todo eso, Jehová no quiso cortar á Judá, por amor de
David su siervo, como le había prometido darle lámpara de sus hijos
perpetuamente. \bibverse{20} En su tiempo se rebeló Edom de debajo de la
mano de Judá, y pusieron rey sobre sí. \bibverse{21} Joram por tanto
pasó á Seir, y todos sus carros con él: y levantándose de noche hirió á
los Idumeos, los cuales le habían cercado, y á los capitanes de los
carros: y el pueblo huyó á sus estancias. \bibverse{22} Sustrájose no
obstante Edom de bajo la mano de Judá, hasta hoy. Rebelóse además Libna
en el mismo tiempo. \bibverse{23} Lo demás de los hechos de Joram, y
todas las cosas que hizo, ¿no está escrito en el libro de las crónicas
de los reyes de Judá? \bibverse{24} Y durmió Joram con sus padres, y fué
sepultado con sus padres en la ciudad de David: y reinó en lugar suyo
Ochôzías, su hijo. \bibverse{25} En el año doce de Joram hijo de Achâb
rey de Israel, comenzó á reinar Ochôzías hijo de Joram rey de Judá.
\bibverse{26} De veintidós años era Ochôzías cuando comenzó á reinar, y
reinó un año en Jerusalem. El nombre de su madre fué Athalía hija de
Omri rey de Israel. \bibverse{27} Y anduvo en el camino de la casa de
Achâb, é hizo lo malo en ojos de Jehová, como la casa de Achâb: porque
era yerno de la casa de Achâb. \bibverse{28} Y fué á la guerra con Joram
hijo de Achâb á Ramoth de Galaad, contra Hazael rey de Siria; y los
Siros hirieron á Joram. \bibverse{29} Y el rey Joram se volvió á
Jezreel, para curarse de las heridas que los Siros le hicieron delante
de Ramoth, cuando peleó contra Hazael rey de Siria. Y descendió Ochôzías
hijo de Joram rey de Judá, á visitar á Joram hijo de Achâb en Jezreel,
porque estaba enfermo.

\hypertarget{section-8}{%
\section{9}\label{section-8}}

\bibverse{1} Entonces el profeta Eliseo llamó á uno de los hijos de los
profetas, y díjole: Ciñe tus lomos, y toma esta alcuza de aceite en tu
mano, y ve á Ramoth de Galaad. \bibverse{2} Y cuando llegares allá,
verás allí á Jehú hijo de Josaphat hijo de Nimsi; y entrando, haz que se
levante de entre sus hermanos, y mételo en la recámara. \bibverse{3}
Toma luego la alcuza de aceite, y derrámala sobre su cabeza, y di: Así
dijo Jehová: Yo te he ungido por rey sobre Israel. Y abriendo la puerta,
echa á huir, y no esperes. \bibverse{4} Fué pues el mozo, el mozo del
profeta, á Ramoth de Galaad. \bibverse{5} Y como él entró, he aquí los
príncipes del ejército que estaban sentados. Y él dijo: Príncipe, una
palabra tengo que decirte. Y Jehú dijo: ¿A cuál de todos nosotros? Y él
dijo: A ti, príncipe. \bibverse{6} Y él se levantó, y entróse en casa; y
el otro derramó el aceite sobre su cabeza, y díjole: Así dijo Jehová
Dios de Israel: Yo te he ungido por rey sobre el pueblo de Jehová, sobre
Israel. \bibverse{7} Y herirás la casa de Achâb tu señor, para que yo
vengue la sangre de mis siervos los profetas, y la sangre de todos los
siervos de Jehová, de la mano de Jezabel. \bibverse{8} Y perecerá toda
la casa de Achâb, y talaré de Achâb todo meante á la pared, así al
guardado como al desamparado en Israel. \bibverse{9} Y yo pondré la casa
de Achâb como la casa de Jeroboam hijo de Nabat, y como la casa de Baasa
hijo de Ahía. \bibverse{10} Y á Jezabel comerán perros en el campo de
Jezreel, y no habrá quien la sepulte. En seguida abrió la puerta, y echó
á huir. \bibverse{11} Después salió Jehú á los siervos de su señor, y
dijéronle: ¿Hay paz? ¿para qué entró á ti aquel loco? Y él les dijo:
Vosotros conocéis al hombre y sus palabras. \bibverse{12} Y ellos
dijeron: Mentira; decláranoslo ahora. Y él dijo: Así y así me habló,
diciendo: Así ha dicho Jehová: Yo te he ungido por rey sobre Israel.
\bibverse{13} Entonces tomaron prestamente su ropa, y púsola cada uno
debajo de él en un trono alto, y tocaron corneta, y dijeron: Jehú es
rey. \bibverse{14} Así conjuró Jehú hijo de Josaphat hijo de Nimsi,
contra Joram. (Estaba Joram guardando á Ramoth de Galaad con todo
Israel, por causa de Hazael rey de Siria. \bibverse{15} Habíase empero
vuelto el rey Joram á Jezreel, para curarse de las heridas que los Siros
le habían hecho, peleando contra Hazael rey de Siria.) Y Jehú dijo: Si
es vuestra voluntad, ninguno escape de la ciudad, para ir á dar las
nuevas en Jezreel. \bibverse{16} Entonces Jehú cabalgó, y fuése á
Jezreel, porque Joram estaba allí enfermo. También Ochôzías rey de Judá
había descendido á visitar á Joram. \bibverse{17} Y el atalaya que
estaba en la torre de Jezreel, vió la cuadrilla de Jehú, que venía, y
dijo: Yo veo una cuadrilla. Y Joram dijo: Toma uno de á caballo, y envía
á reconocerlos, y que les diga: ¿Hay paz? \bibverse{18} Fué pues el de á
caballo á reconocerlos, y dijo: El rey dice así: ¿Hay paz? Y Jehú le
dijo: ¿Qué tienes tú que ver con la paz? vuélvete tras mí. El atalaya
dió luego aviso, diciendo: El mensajero llegó hasta ellos, y no vuelve.
\bibverse{19} Entonces envió otro de á caballo, el cual llegando á
ellos, dijo: El rey dice así: ¿Hay paz? Y Jehú respondió: ¿Qué tienes tú
que ver con la paz? vuélvete tras mí. \bibverse{20} El atalaya volvió á
decir: También éste llegó á ellos y no vuelve: mas el marchar del que
viene es como el marchar de Jehú hijo de Nimsi, porque viene
impetuosamente. \bibverse{21} Entonces Joram dijo: Unce. Y uncido que
fué su carro, salió Joram rey de Israel, y Ochôzías rey de Judá, cada
uno en su carro, y salieron á encontrar á Jehú, al cual hallaron en la
heredad de Naboth de Jezreel. \bibverse{22} Y en viendo Joram á Jehú,
dijo: ¿Hay paz, Jehú? Y él respondió: ¿Qué paz, con las fornicaciones de
Jezabel tu madre, y sus muchas hechicerías? \bibverse{23} Entonces Joram
volviendo la mano huyó, y dijo á Ochôzías: ¡Traición, Ochôzías!
\bibverse{24} Mas Jehú flechó su arco, é hirió á Joram entre las
espaldas, y la saeta salió por su corazón, y cayó en su carro.
\bibverse{25} Dijo luego Jehú á Bidkar su capitán: Tómalo y échalo á un
cabo de la heredad de Naboth de Jezreel. Acuérdate que cuando tú y yo
íbamos juntos con la gente de Achâb su padre, Jehová pronunció esta
sentencia sobre él, diciendo: \bibverse{26} Que yo he visto ayer las
sangres de Naboth, y las sangres de sus hijos, dijo Jehová; y tengo de
darte la paga en esta heredad, dijo Jehová. Tómalo pues ahora, y échalo
en la heredad, conforme á la palabra de Jehová. \bibverse{27} Y viendo
esto Ochôzías rey de Judá, huyó por el camino de la casa del huerto. Y
siguiólo Jehú, diciendo: Herid también á éste en el carro. Y le hirieron
á la subida de Gur, junto á Ibleam. Y él huyó á Megiddo, y murió allí.
\bibverse{28} Y sus siervos le llevaron en un carro á Jerusalem, y allá
le sepultaron con sus padres, en su sepulcro en la ciudad de David.
\bibverse{29} En el undécimo año de Joram hijo de Achâb, comenzó á
reinar Ochôzías sobre Judá. \bibverse{30} Vino después Jehú á Jezreel: y
como Jezabel lo oyó, adornó sus ojos con alcohol, y atavió su cabeza, y
asomóse á una ventana. \bibverse{31} Y como entraba Jehú por la puerta,
ella dijo: ¿Sucedió bien á Zimri, que mató á su señor? \bibverse{32}
Alzando él entonces su rostro hacia la ventana, dijo: ¿Quién es conmigo?
¿quién? Y miraron hacia él dos ó tres eunucos. \bibverse{33} Y él les
dijo: Echadla abajo. Y ellos la echaron: y parte de su sangre fué
salpicada en la pared, y en los caballos; y él la atropelló.
\bibverse{34} Entró luego, y después que comió y bebió, dijo: Id ahora á
ver aquella maldita, y sepultadla; que es hija de rey. \bibverse{35}
Empero cuando fueron para sepultarla, no hallaron de ella más que la
calavera, y los pies, y las palmas de las manos. \bibverse{36} Y
volvieron, y dijéronselo. Y él dijo: La palabra de Dios es ésta, la cual
él habló por mano de su siervo Elías Thisbita, diciendo: En la heredad
de Jezreel comerán los perros las carnes de Jezabel. \bibverse{37} Y el
cuerpo de Jezabel fué cual estiércol sobre la faz de la tierra en la
heredad de Jezreel; de manera que nadie pueda decir: Esta es Jezabel.

\hypertarget{section-9}{%
\section{10}\label{section-9}}

\bibverse{1} Y tenía Achâb en Samaria setenta hijos; y escribió letras
Jehú, y enviólas á Samaria á los principales de Jezreel, á los ancianos
y á los ayos de Achâb, diciendo: \bibverse{2} Luego en llegando estas
letras á vosotros los que tenéis los hijos de vuestro señor, y los que
tenéis carros y gente de á caballo, la ciudad pertrechada, y las armas,
\bibverse{3} Mirad cuál es el mejor y él más recto de los hijos de
vuestro señor, y ponedlo en el trono de su padre, y pelead por la casa
de vuestro señor. \bibverse{4} Mas ellos tuvieron gran temor, y dijeron:
He aquí dos reyes no pudieron resistirle, ¿cómo le resistiremos
nosotros? \bibverse{5} Y el mayordomo, y el presidente de la ciudad, y
los ancianos, y los ayos, enviaron á decir á Jehú: Siervos tuyos somos,
y haremos todo lo que nos mandares: no elegiremos por rey á ninguno; tú
harás lo que bien te pareciere. \bibverse{6} El entonces les escribió la
segunda vez, diciendo: Si sois míos, y queréis obedecerme, tomad las
cabezas de los varones hijos de vuestro señor, y venid mañana á estas
horas á mí á Jezreel. Y los hijos del rey, setenta varones, estaban con
los principales de la ciudad, que los criaban. \bibverse{7} Y como las
letras llegaron á ellos, tomaron á los hijos del rey, y degollaron
setenta varones, y pusieron sus cabezas en canastillos, y enviáronselas
á Jezreel. \bibverse{8} Y vino un mensajero que le dió las nuevas,
diciendo: Traído han las cabezas de los hijos del rey. Y él le dijo:
Ponedlas en dos montones á la entrada de la puerta hasta la mañana.
\bibverse{9} Venida la mañana, salió él, y estando en pie dijo á todo el
pueblo: Vosotros sois justos: he aquí yo he conspirado contra mi señor,
y lo he muerto: ¿mas quién ha muerto á todos estos? \bibverse{10} Sabed
ahora que de la palabra de Jehová que habló sobre la casa de Achâb, nada
caerá en tierra: y que Jehová ha hecho lo que dijo por su siervo Elías.
\bibverse{11} Mató entonces Jehú á todos los que habían quedado de la
casa de Achâb en Jezreel, y á todos sus príncipes, y á todos sus
familiares, y á sus sacerdotes, que no le quedó ninguno. \bibverse{12} Y
levantóse de allí, y vino á Samaria; y llegando él en el camino á una
casa de esquileo de pastores, \bibverse{13} Halló allí á los hermanos de
Ochôzías rey de Judá, y díjoles: ¿Quién sois vosotros? Y ellos dijeron:
Somos hermanos de Ochôzías, y hemos venido á saludar á los hijos del
rey, y á los hijos de la reina. \bibverse{14} Entonces él dijo:
Prendedlos vivos. Y después que los tomaron vivos, degolláronlos junto
al pozo de la casa de esquileo, cuarenta y dos varones, sin dejar
ninguno de ellos. \bibverse{15} Partiéndose luego de allí encontróse con
Jonadab hijo de Rechâb; y después que lo hubo saludado, díjole: ¿Es
recto tu corazón, como el mío es recto con el tuyo? Y Jonadab dijo: Lo
es. Pues que lo es, dame la mano. Y él le dió su mano. Hízolo luego
subir consigo en el carro. \bibverse{16} Y díjole: Ven conmigo, y verás
mi celo por Jehová. Pusiéronlo pues en su carro. \bibverse{17} Y luego
que hubo Jehú llegado á Samaria, mató á todos los que habían quedado de
Achâb en Samaria, hasta extirparlos, conforme á la palabra de Jehová,
que había hablado por Elías. \bibverse{18} Y juntó Jehú todo el pueblo,
y díjoles: Achâb sirvió poco á Baal; mas Jehú lo servirá mucho.
\bibverse{19} Llamadme pues luego á todos los profetas de Baal, á todos
sus siervos, y á todos sus sacerdotes; que no falte uno, porque tengo un
gran sacrifico para Baal; cualquiera que faltare, no vivirá. Esto hacía
Jehú con astucia, para destruir á los que honraban á Baal. \bibverse{20}
Y dijo Jehú: Santificad un día solemne á Baal. Y ellos convocaron.
\bibverse{21} Y envió Jehú por todo Israel, y vinieron todos los siervos
de Baal, que no faltó ninguno que no viniese. Y entraron en el templo de
Baal, y el templo de Baal se llenó de cabo á cabo. \bibverse{22}
Entonces dijo al que tenía el cargo de las vestiduras: Saca vestiduras
para todos lo siervos de Baal. Y él les sacó vestimentas. \bibverse{23}
Y entró Jehú con Jonadab hijo de Rechâb en el templo de Baal, y dijo á
los siervos de Baal: Mirad y ved que por dicha no haya aquí entre
vosotros alguno de los siervos de Jehová, sino solos los siervos de
Baal. \bibverse{24} Y como ellos entraron para hacer sacrificios y
holocaustos, Jehú puso fuera ochenta hombres, y díjoles: Cualquiera que
dejare vivo alguno de aquellos hombres que yo he puesto en vuestras
manos, su vida será por la del otro. \bibverse{25} Y después que
acabaron ellos de hacer el holocausto, Jehú dijo á los de su guardia y á
los capitanes: Entrad, y matadlos; que no escape ninguno. Y los hirieron
á cuchillo: y dejáronlos tendidos los de la guardia y los capitanes, y
fueron hasta la ciudad del templo de Baal. \bibverse{26} Y sacaron las
estatuas de la casa de Baal, y quemáronlas. \bibverse{27} Y quebraron la
estatua de Baal, y derribaron la casa de Baal, é hiciéronla necesaria,
hasta hoy. \bibverse{28} Así extinguió Jehú á Baal de Israel.
\bibverse{29} Con todo eso Jehú no se apartó de los pecados de Jeroboam
hijo de Nabat, que hizo pecar á Israel; á saber, de en pos de los
becerros de oro que estaban en Beth-el y en Dan. \bibverse{30} Y Jehová
dijo á Jehú: Por cuanto has hecho bien ejecutando lo recto delante de
mis ojos, é hiciste á la casa de Achâb conforme á todo lo que estaba en
mi corazón, tus hijos se sentarán sobre el trono de Israel hasta la
cuarta generación. \bibverse{31} Mas Jehú no cuidó de andar en la ley de
Jehová Dios de Israel con todo su corazón, ni se apartó de los pecados
de Jeroboam, el que había hecho pecar á Israel. \bibverse{32} En
aquellos días comenzó Jehová á talar en Israel: é hiriólos Hazael en
todos los términos de Israel, \bibverse{33} Desde el Jordán al
nacimiento del sol, toda la tierra de Galaad, de Gad, de Rubén, y de
Manasés, desde Aroer que está junto al arroyo de Arnón, á Galaad y á
Basán. \bibverse{34} Lo demás de los hechos de Jehú, y todas las cosas
que hizo, y toda su valentía, ¿no está escrito en el libro de las
crónicas de los reyes de Israel? \bibverse{35} Y durmió Jehú con sus
padres, y sepultáronlo en Samaria: y reinó en su lugar Joachâz su hijo.
\bibverse{36} El tiempo que reinó Jehú sobre Israel en Samaria, fué
veintiocho años.

\hypertarget{section-10}{%
\section{11}\label{section-10}}

\bibverse{1} Y athalía madre de Ochôzías, viendo que su hijo era muerto,
levantóse, y destruyó toda la simiente real. \bibverse{2} Pero tomando
Josaba hija del rey Joram, hermana de Ochôzías, á Joas hijo de Ochôzías,
sacólo furtivamente de entre los hijos del rey, que se mataban, y
ocultólo de delante de Athalía, á él y á su ama, en la cámara de las
camas, y así no lo mataron. \bibverse{3} Y estuvo con ella escondido en
la casa de Jehová seis años: y Athalía fué reina sobre el país.
\bibverse{4} Mas al séptimo año envió Joiada, y tomó centuriones,
capitanes, y gente de la guardia, y metiólos consigo en la casa de
Jehová: é hizo con ellos liga, juramentándolos en la casa de Jehová; y
mostróles al hijo del rey. \bibverse{5} Y mandóles, diciendo: Esto es lo
que habéis de hacer: la tercera parte de vosotros, los que entrarán el
sábado, tendrán la guardia de la casa del rey; \bibverse{6} Y la otra
tercera parte estará á la puerta del sur, y la otra tercera parte á la
puerta del postigo de los de la guardia: así guardaréis la casa, para
que no sea allanada. \bibverse{7} Y las dos partes de vosotros, es á
saber, todos los que salen el sábado, tendréis la guarda de la casa de
Jehová junto al rey. \bibverse{8} Y estaréis alrededor del rey de todas
partes, teniendo cada uno sus armas en las manos, y cualquiera que
entrare dentro de estos órdenes, sea muerto. Y habéis de estar con el
rey cuando saliere, y cuando entrare. \bibverse{9} Los centuriones pues,
hicieron todo como el sacerdote Joiada les mandó: y tomando cada uno los
suyos, es á saber, los que habían de entrar el sábado, y los que habían
salido el sábado, viniéronse á Joiada el sacerdote. \bibverse{10} Y el
sacerdote dió á los centuriones las picas y los escudos que habían sido
del rey David, que estaban en la casa de Jehová. \bibverse{11} Y los de
la guardia se pusieron en orden, teniendo cada uno sus armas en sus
manos, desde el lado derecho de la casa hasta el lado izquierdo, junto
al altar y el templo, en derredor del rey. \bibverse{12} Sacando luego
Joiada al hijo del rey, púsole la corona y el testimonio, é hiciéronle
rey ungiéndole; y batiendo las manos dijeron: ¡Viva el rey!
\bibverse{13} Y oyendo Athalía el estruendo del pueblo que corría, entró
al pueblo en el templo de Jehová; \bibverse{14} Y como miró, he aquí el
rey que estaba junto á la columna, conforme á la costumbre, y los
príncipes y los trompetas junto al rey; y que todo el pueblo del país
hacía alegrías, y que tocaban las trompetas. Entonces Athalía, rasgando
sus vestidos, clamó á voz en grito: ¡Traición, traición! \bibverse{15}
Mas el sacerdote Joiada mandó á los centuriones que gobernaban el
ejército, y díjoles: Sacadla fuera del recinto del templo, y al que la
siguiere, matadlo á cuchillo. (Porque el sacerdote dijo que no la
matasen en el templo de Jehová.) \bibverse{16} Diéronle pues lugar, y
como iba el camino por donde entran los de á caballo á la casa del rey,
allí la mataron. \bibverse{17} Entonces Joiada hizo alianza entre Jehová
y el rey y el pueblo, que serían pueblo de Jehová: y asimismo entre el
rey y el pueblo. \bibverse{18} Y todo el pueblo de la tierra entró en el
templo de Baal, y derribáronlo: asimismo despedazaron enteramente sus
altares y sus imágenes, y mataron á Mathán sacerdote de Baal delante de
los altares. Y el sacerdote puso guarnición sobre la casa de Jehová.
\bibverse{19} Después tomó los centuriones, y capitanes, y los de la
guardia, y á todo el pueblo de la tierra, y llevaron al rey desde la
casa de Jehová, y vinieron por el camino de la puerta de los de la
guardia á la casa del rey; y sentóse el rey sobre el trono de los reyes.
\bibverse{20} Y todo el pueblo de la tierra hizo alegrías, y la ciudad
estuvo en reposo, habiendo sido Athalía muerta á cuchillo junto á la
casa del rey. \bibverse{21} Era Joas de siete años cuando comenzó á
reinar.

\hypertarget{section-11}{%
\section{12}\label{section-11}}

\bibverse{1} En el séptimo año de Jehú comenzó á reinar Joas, y reinó
cuarenta años en Jerusalem. El nombre de su madre fué Sibia, de
Beer-seba. \bibverse{2} Y Joas hizo lo recto en ojos de Jehová todo el
tiempo que le dirigió el sacerdote Joiada. \bibverse{3} Con todo eso los
altos no se quitaron; que aun sacrificaba y quemaba el pueblo perfumes
en los altos. \bibverse{4} Y Joas dijo á los sacerdotes: Todo el dinero
de las santificaciones que se suele traer á la casa de Jehová, el dinero
de los que pasan en cuenta, el dinero por las personas, cada cual según
su tasa, y todo el dinero que cada uno de su propia voluntad mete en la
casa de Jehová, \bibverse{5} Recíbanlo los sacerdotes, cada uno de sus
familiares, y reparen los portillos del templo donde quiera que se
hallare abertura. \bibverse{6} Pero el año veintitrés del rey Joas, no
habían aún reparado los sacerdotes las aberturas del templo.
\bibverse{7} Llamando entonces el rey Joas al pontífice Joiada y á los
sacerdotes, díjoles: ¿Por qué no reparáis las aberturas del templo?
Ahora pues, no toméis más el dinero de vuestros familiares, sino dadlo
para reparar las roturas del templo. \bibverse{8} Y los sacerdotes
consintieron en no tomar más dinero del pueblo, ni tener cargo de
reparar las aberturas del templo. \bibverse{9} Mas el pontífice Joiada
tomó un arca, é hízole en la tapa un agujero, y púsola junto al altar, á
la mano derecha como se entra en el templo de Jehová; y los sacerdotes
que guardaban la puerta, ponían allí todo el dinero que se metía en la
casa de Jehová. \bibverse{10} Y cuando veían que había mucho dinero en
el arca, venía el notario del rey y el gran sacerdote, y contaban el
dinero que hallaban en el templo de Jehová, y guardábanlo. \bibverse{11}
Y daban el dinero suficiente en mano de los que hacían la obra, y de los
que tenían el cargo de la casa de Jehová; y ellos lo expendían en pagar
los carpinteros y maestros que reparaban la casa de Jehová,
\bibverse{12} Y los albañiles y canteros; y en comprar la madera y
piedra de cantería para reparar las aberturas de la casa de Jehová; y en
todo lo que se gastaba en la casa para repararla. \bibverse{13} Mas de
aquel dinero que se traía á la casa de Jehová, no se hacían tazas de
plata, ni salterios, ni jofainas, ni trompetas; ni ningún otro vaso de
oro ni de plata se hacía para el templo de Jehová: \bibverse{14} Porque
lo daban á los que hacían la obra, y con él reparaban la casa de Jehová.
\bibverse{15} Y no se tomaba cuenta á los hombres en cuyas manos el
dinero era entregado, para que ellos lo diesen á los que hacían la obra:
porque lo hacían ellos fielmente. \bibverse{16} El dinero por el delito,
y el dinero por los pecados, no se metía en la casa de Jehová; porque
era de los sacerdotes. \bibverse{17} Entonces subió Hazael rey de Siria,
y peleó contra Gath, y tomóla: y puso Hazael su rostro para subir contra
Jerusalem; \bibverse{18} Por lo que tomó Joas rey de Judá todas las
ofrendas que había dedicado Josaphat, y Joram y Ochôzías sus padres,
reyes de Judá, y las que él había dedicado, y todo el oro que se halló
en los tesoros de la casa de Jehová, y en la casa del rey, y enviólo á
Hazael rey de Siria: y él se partió de Jerusalem. \bibverse{19} Lo demás
de los hechos de Joas, y todas las cosas que hizo, ¿no está escrito en
el libro de las crónicas de los reyes de Judá? \bibverse{20} Y
levantáronse sus siervos, y conspiraron en conjuración, y mataron á Joas
en la casa de Millo, descendiendo él á Silla; \bibverse{21} Pues
Josachâr hijo de Simaath, y Jozabad hijo de Somer, sus siervos,
hiriéronle, y murió. Y sepultáronlo con sus padres en la ciudad de
David, y reinó en su lugar Amasías su hijo.

\hypertarget{section-12}{%
\section{13}\label{section-12}}

\bibverse{1} En el año veintitrés de Joas hijo de Ochôzías, rey de Judá,
comenzó á reinar Joachâz hijo de Jehú sobre Israel en Samaria; y reinó
diecisiete años. \bibverse{2} E hizo lo malo en ojos de Jehová, y siguió
los pecados de Jeroboam hijo de Nabat, el que hizo pecar á Israel; y no
se apartó de ellos. \bibverse{3} Y encendióse el furor de Jehová contra
Israel, y entrególos en mano de Hazael rey de Siria, y en mano de
Ben-adad hijo de Hazael, por largo tiempo. \bibverse{4} Mas Joachâz oró
á la faz de Jehová, y Jehová lo oyó: porque miró la aflicción de Israel,
pues el rey de Siria los afligía. \bibverse{5} (Y dió Jehová salvador á
Israel, y salieron de bajo la mano de los Siros; y habitaron los hijos
de Israel en sus estancias, como antes. \bibverse{6} Con todo eso no se
apartaron de los pecados de la casa de Jeroboam, el que hizo pecar á
Israel: en ellos anduvieron; y también el bosque permaneció en Samaria.)
\bibverse{7} Porque no le había quedado gente á Joachâz, sino cincuenta
hombres de á caballo, y diez carros, y diez mil hombres de á pié; pues
el rey de Siria los había destruído, y los había puesto como polvo para
hollar. \bibverse{8} Lo demás de los hechos de Joachâz, y todo lo que
hizo, y sus valentías, ¿no está escrito en el libro de las crónicas de
los reyes de Israel? \bibverse{9} Y durmió Joachâz con sus padres, y
sepultáronlo en Samaria: y reinó en su lugar Joas su hijo. \bibverse{10}
El año treinta y siete de Joas rey de Judá, comenzó á reinar Joas hijo
de Joachâz sobre Israel en Samaria; y reinó dieciséis años.
\bibverse{11} E hizo lo malo en ojos de Jehová: no se apartó de todos
los pecados de Jeroboam hijo de Nabat, el que hizo pecar á Israel; en
ellos anduvo. \bibverse{12} Lo demás de los hechos de Joas, y todas las
cosas que hizo, y su esfuerzo con que guerreó contra Amasías rey de
Judá, ¿no está escrito en el libro de las crónicas de los reyes de
Israel? \bibverse{13} Y durmió Joas con sus padres, y sentóse Jeroboam
sobre su trono: y Joas fué sepultado en Samaria con los reyes de Israel.
\bibverse{14} Estaba Eliseo enfermo de aquella su enfermedad de que
murió. Y descendió á él Joas rey de Israel, y llorando delante de él,
dijo: ¡Padre mío, padre mío, carro de Israel y su gente de á caballo!
\bibverse{15} Y díjole Eliseo: Toma un arco y unas saetas. Tomóse él
entonces un arco y unas saetas. \bibverse{16} Y dijo Eliseo al rey de
Israel: Pon tu mano sobre el arco. Y puso él su mano sobre el arco.
Entonces puso Eliseo sus manos sobre las manos del rey, \bibverse{17} Y
dijo: Abre la ventana de hacia el oriente. Y como él la abrió dijo
Eliseo: Tira. Y tirando él, dijo Eliseo: Saeta de salud de Jehová, y
saeta de salud contra Siria: porque herirás á los Siros en Aphec, hasta
consumirlos. \bibverse{18} Y tornóle á decir: Toma las saetas. Y luego
que el rey de Israel las hubo tomado, díjole: Hiere la tierra. Y él
hirió tres veces, y cesó. \bibverse{19} Entonces el varón de Dios,
enojado con él, le dijo: A herir cinco ó seis veces, herirías á Siria,
hasta no quedar ninguno: empero ahora tres veces herirás á Siria.
\bibverse{20} Y murió Eliseo, y sepultáronlo. Entrado el año vinieron
partidas de Moabitas á la tierra. \bibverse{21} Y aconteció que al
sepultar unos un hombre, súbitamente vieron una partida, y arrojaron al
hombre en el sepulcro de Eliseo: y cuando llegó á tocar el muerto los
huesos de Eliseo, revivió, y levantóse sobre sus pies. \bibverse{22}
Hazael pues, rey de Siria, afligió á Israel todo el tiempo de Joachâz.
\bibverse{23} Mas Jehová tuvo misericordia de ellos, y compadecióse de
ellos, y mirólos, por amor de su pacto con Abraham, Isaac y Jacob; y no
quiso destruirlos ni echarlos de delante de sí hasta ahora.
\bibverse{24} Y murió Hazael rey de Siria, y reinó en su lugar Ben-adad
su hijo. \bibverse{25} Y volvió Joas hijo de Joachâz, y tomó de mano de
Ben-adad hijo de Hazael, las ciudades que él había tomado de mano de
Joachâz su padre en guerra. Tres veces lo batió Joas, y restituyó las
ciudades á Israel.

\hypertarget{section-13}{%
\section{14}\label{section-13}}

\bibverse{1} En el año segundo de Joas hijo de Joachâz rey de Israel,
comenzó á reinar Amasías hijo de Joas rey de Judá. \bibverse{2} Cuando
comenzó á reinar era de veinticinco años, y veintinueve años reinó en
Jerusalem: el nombre de su madre fué Joaddan, de Jerusalem. \bibverse{3}
Y él hizo lo recto en ojos de Jehová, aunque no como David su padre:
hizo conforme á todas las cosas que había hecho Joas su padre.
\bibverse{4} Con todo eso los altos no fueron quitados; que el pueblo
aun sacrificaba y quemaba perfumes en los altos. \bibverse{5} Y luego
que el reino fué confirmado en su mano, hirió á sus siervos, los que
habían muerto al rey su padre. \bibverse{6} Mas no mató á los hijos de
los que le mataron, conforme á lo que está escrito en el libro de la ley
de Moisés, donde Jehová mandó, diciendo: No matarán á los padres por los
hijos, ni á los hijos por los padres: mas cada uno morirá por su pecado.
\bibverse{7} Este hirió asimismo diez mil Idumeos en el valle de las
Salinas, y tomó á Sela por guerra, y llamóla Jocteel, hasta hoy.
\bibverse{8} Entonces Amasías envió embajadores á Joas, hijo de Joachâz
hijo de Jehú, rey de Israel, diciendo: Ven, y veámonos de rostro.
\bibverse{9} Y Joas rey de Israel envió á Amasías rey de Judá esta
respuesta: El cardillo que está en el Líbano envió á decir al cedro que
está en el Líbano: Da tu hija por mujer á mi hijo. Y pasaron las bestias
fieras que están en el Líbano, y hollaron el cardillo. \bibverse{10}
Ciertamente has herido á Edom, y tu corazón te ha envanecido: gloríate
pues, mas estáte en tu casa. ¿Y por qué te entrometerás en un mal, para
que caigas tú, y Judá contigo? \bibverse{11} Mas Amasías no dió oídos;
por lo que subió Joas rey de Israel, y viéronse de rostro él y Amasías
rey de Judá, en Beth-semes, que es de Judá. \bibverse{12} Y Judá cayó
delante de Israel, y huyeron cada uno á sus estancias. \bibverse{13}
Además Joas rey de Israel tomó á Amasías rey de Judá, hijo de Joas hijo
de Ochôzías, en Beth-semes: y vino á Jerusalem, y rompió el muro de
Jerusalem desde la puerta de Ephraim hasta la puerta de la esquina,
cuatrocientos codos. \bibverse{14} Y tomó todo el oro y la plata, y
todos los vasos que fueron hallados en la casa de Jehová, y en los
tesoros de la casa del rey, y los hijos en rehenes, y volvióse á
Samaria. \bibverse{15} Lo demás de los hechos de Joas que ejecutó, y sus
hazañas, y cómo peleó contra Amasías rey de Judá, ¿no está escrito en el
libro de las crónicas de los reyes de Israel? \bibverse{16} Y durmió
Joas con sus padres, y fué sepultado en Samaria con los reyes de Israel;
y reinó en su lugar Jeroboam su hijo. \bibverse{17} Y Amasías hijo de
Joas rey de Judá, vivió después de la muerte de Joas hijo de Joachâz rey
de Israel, quince años. \bibverse{18} Lo demás de los hechos de Amasías,
¿no está escrito en el libro de las crónicas de los reyes de Judá?
\bibverse{19} E hicieron conspiración contra él en Jerusalem, y él huyó
á Lachîs; mas enviaron tras él á Lachîs, y allá lo mataron.
\bibverse{20} Trajéronlo luego sobre caballos, y sepultáronlo en
Jerusalem con sus padres, en la ciudad de David. \bibverse{21} Entonces
todo el pueblo de Judá tomó á Azarías, que era de diez y seis años, é
hiciéronlo rey en lugar de Amasías su padre. \bibverse{22} Edificó él á
Elath, y la restituyó á Judá, después que el rey durmió con sus padres.
\bibverse{23} El año quince de Amasías hijo de Joas rey de Judá, comenzó
á reinar Jeroboam hijo de Joas sobre Israel en Samaria; y reinó cuarenta
y un años. \bibverse{24} E hizo lo malo en ojos de Jehová, y no se
apartó de todos los pecados de Jeroboam hijo de Nabat, el que hizo pecar
á Israel. \bibverse{25} El restituyó los términos de Israel desde la
entrada de Amath hasta la mar de la llanura, conforme á la palabra de
Jehová Dios de Israel, la cual había él hablado por su siervo Jonás hijo
de Amittai, profeta que fué de Gath-hepher. \bibverse{26} Por cuanto
Jehová miró la muy amarga aflicción de Israel; que no había guardado ni
desamparado, ni quien diese ayuda á Israel; \bibverse{27} Y Jehová no
había determinado raer el nombre de Israel de debajo del cielo: por
tanto, los salvó por mano de Jeroboam hijo de Joas. \bibverse{28} Y lo
demás de los hechos de Jeroboam, y todas las cosas que hizo, y su
valentía, y todas las guerras que hizo, y cómo restituyó á Judá en
Israel á Damasco y á Hamath, ¿no está escrito en el libro de las
crónicas de los reyes de Israel? \bibverse{29} Y durmió Jeroboam con sus
padres, los reyes de Israel, y reinó en su lugar Zachârías su hijo.

\hypertarget{section-14}{%
\section{15}\label{section-14}}

\bibverse{1} En el año veintisiete de Jeroboam, rey de Israel, comenzó á
reinar Azarías hijo de Amasías rey de Judá. \bibverse{2} Cuando comenzó
á reinar era de dieciséis años, y cincuenta y dos años reinó en
Jerusalem; el nombre de su madre fué Jecolía, de Jerusalem. \bibverse{3}
E hizo lo recto en ojos de Jehová, conforme á todas las cosas que su
padre Amasías había hecho. \bibverse{4} Con todo eso los altos no se
quitaron; que el pueblo sacrificaba aún y quemaba perfumes en los altos.
\bibverse{5} Mas Jehová hirió al rey con lepra, y fué leproso hasta el
día de su muerte, y habitó en casa separada, y Jotham hijo del rey tenía
el cargo del palacio, gobernando al pueblo de la tierra. \bibverse{6} Lo
demás de los hechos de Azarías, y todas las cosas que hizo, ¿no está
escrito en el libro de las crónicas de los reyes de Judá? \bibverse{7} Y
durmió Azarías con sus padres, y sepultáronlo con sus padres en la
ciudad de David: y reinó en su lugar Jotham su hijo. \bibverse{8} En el
año treinta y ocho de Azarías rey de Judá, reinó Zachârías hijo de
Jeroboam sobre Israel seis meses. \bibverse{9} E hizo lo malo en ojos de
Jehová, como habían hecho sus padres: no se apartó de los pecados de
Jeroboam hijo de Nabat, el que hizo pecar á Israel. \bibverse{10} Contra
él se conjuró Sallum hijo de Jabes, y lo hirió en presencia de su
pueblo, y matólo, y reinó en su lugar. \bibverse{11} Lo demás de los
hechos de Zachârías, he aquí está escrito en el libro de las crónicas de
los reyes de Israel. \bibverse{12} Y esta fué la palabra de Jehová que
había hablado á Jehú, diciendo: Tus hijos hasta la cuarta generación se
sentarán en el trono de Israel. Y fué así. \bibverse{13} Sallum hijo de
Jabes comenzó á reinar en el año treinta y nueve de Uzzía rey de Judá, y
reinó el tiempo de un mes en Samaria; \bibverse{14} Pues subió Manahem
hijo de Gadi, de Thirsa, y vino á Samaria, é hirió á Sallum hijo de
Jabes en Samaria, y matólo, y reinó en su lugar. \bibverse{15} Lo demás
de los hechos de Sallum, y su conjuración con que conspiró, he aquí está
escrito en el libro de las crónicas de los reyes de Israel.
\bibverse{16} Entonces hirió Manahem á Tiphsa, y á todos los que estaban
en ella, y también sus términos desde Thirsa; é hirióla porque no le
habían abierto; y abrió á todas sus preñadas. \bibverse{17} En el año
treinta y nueve de Azarías rey de Judá, reinó Manahem hijo de Gadi sobre
Israel diez años, en Samaria. \bibverse{18} E hizo lo malo en ojos de
Jehová: no se apartó en todo su tiempo de los pecados de Jeroboam hijo
de Nabat, el que hizo pecar á Israel. \bibverse{19} Y vino Phul rey de
Asiria á la tierra; y dió Manahem á Phul mil talentos de plata porque le
ayudara á confirmarse en el reino. \bibverse{20} E impuso Manahem este
dinero sobre Israel, sobre todos los poderosos y opulentos: de cada uno
cincuenta siclos de plata, para dar al rey de Asiria, y el rey de Asiria
se volvió, y no se detuvo allí en la tierra. \bibverse{21} Lo demás de
los hechos de Manahem, y todas las cosas que hizo, ¿no está escrito en
el libro de las crónicas de los reyes de Israel? \bibverse{22} Y durmió
Manahem con sus padres, y reinó en su lugar Pekaía su hijo.
\bibverse{23} En el año cincuenta de Azarías rey de Judá, reinó Pekaía
hijo de Manahem sobre Israel en Samaria, dos años. \bibverse{24} E hizo
lo malo en ojos de Jehová: no se apartó de los pecados de Jeroboam hijo
de Nabat, el que hizo pecar á Israel. \bibverse{25} Y conspiró contra él
Peka hijo de Remalías, capitán suyo, é hiriólo en Samaria, en el palacio
de la casa real, en compañía de Argob y de Ariph, y con cincuenta
hombres de los hijos de los Galaaditas; y matólo, y reinó en su lugar.
\bibverse{26} Lo demás de los hechos de Pekaía, y todas las cosas que
hizo, he aquí está escrito en el libro de las crónicas de los reyes de
Israel. \bibverse{27} En el año cincuenta y dos de Azarías rey de Judá,
reinó Peka hijo de Remalías sobre Israel en Samaria; y reinó veinte
años. \bibverse{28} E hizo lo malo en ojos de Jehová; no se apartó de
los pecados de Jeroboam hijo de Nabat, el que hizo pecar á Israel.
\bibverse{29} En los días de Peka rey de Israel, vino Tiglath-pileser
rey de los Asirios, y tomó á Ahión, Abel-beth-maachâ, y Janoa, y Cedes,
y Asor, y Galaad, y Galilea, y toda la tierra de Nephtalí; y
trasportólos á Asiria. \bibverse{30} Y Oseas hijo de Ela hizo
conjuración contra Peka hijo de Remalías, é hiriólo, y matólo, y reinó
en su lugar, á los veinte años de Jotham hijo de Uzzía. \bibverse{31} Lo
demás de los hechos de Peka, y todo lo que hizo, he aquí está escrito en
el libro de las crónicas de los reyes de Israel. \bibverse{32} En el
segundo año de Peka hijo de Remalías rey de Israel, comenzó á reinar
Jotham hijo de Uzzía rey de Judá. \bibverse{33} Cuando comenzó á reinar
era de veinticinco años, y reinó dieciséis años en Jerusalem. El nombre
de su madre fué Jerusa hija de Sadoc. \bibverse{34} Y él hizo lo recto
en ojos de Jehová; hizo conforme á todas las cosas que había hecho su
padre Uzzía. \bibverse{35} Con todo eso los altos no fueron quitados;
que el pueblo sacrificaba aún, y quemaba perfumes en los altos. Edificó
él la puerta más alta de la casa de Jehová. \bibverse{36} Lo demás de
los hechos de Jotham, y todas las cosas que hizo, ¿no está escrito en el
libro de las crónicas de los reyes de Judá? \bibverse{37} En aquel
tiempo comenzó Jehová á enviar contra Judá á Resín rey de Siria, y á
Peka hijo de Remalías. \bibverse{38} Y durmió Jotham con sus padres, y
fué sepultado con sus padres en la ciudad de David su padre: y reinó en
su lugar Achâz su hijo.

\hypertarget{section-15}{%
\section{16}\label{section-15}}

\bibverse{1} En el año diecisiete de Peka hijo de Remalías, comenzó á
reinar Achâz hijo de Jotham rey de Judá. \bibverse{2} Cuando comenzó á
reinar Achâz, era de veinte años, y reinó en Jerusalem dieciséis años: y
no hizo lo recto en ojos de Jehová su Dios, como David su padre;
\bibverse{3} Antes anduvo en el camino de los reyes de Israel, y aun
hizo pasar por el fuego á su hijo, según las abominaciones de las gentes
que Jehová echó de delante de los hijos de Israel. \bibverse{4} Asimismo
sacrificó, y quemó perfumes en los altos, y sobre los collados, y debajo
de todo árbol umbroso. \bibverse{5} Entonces Resín rey de Siria, y Peka
hijo de Remalías rey de Israel, subieron á Jerusalem para hacer guerra,
y cercar á Achâz; mas no pudieron tomarla. \bibverse{6} En aquel tiempo
Resín rey de Siria restituyó Elath á Siria, y echó á los Judíos de
Elath; y los Siros vinieron á Elath, y habitaron allí hasta hoy.
\bibverse{7} Entonces Achâz envió embajadores á Tiglath-pileser rey de
Asiria, diciendo: Yo soy tu siervo y tu hijo: sube, y defiéndeme de mano
del rey de Siria, y de mano del rey de Israel, que se han levantado
contra mí. \bibverse{8} Y tomando Achâz la plata y el oro que se halló
en la casa de Jehová, y en los tesoros de la casa real, envió al rey de
Asiria un presente. \bibverse{9} Y atendióle el rey de Asiria; pues
subió el rey de Asiria contra Damasco, y tomóla, y trasportó los
moradores á Kir, y mató á Resín. \bibverse{10} Y fué el rey Achâz á
encontrar á Tiglath-pileser rey de Asiria en Damasco; y visto que hubo
el rey Achâz el altar que estaba en Damasco, envió á Urías sacerdote el
diseño y la descripción del altar, conforme á toda su hechura.
\bibverse{11} Y Urías el sacerdote edificó el altar; conforme á todo lo
que el rey Achâz había enviado de Damasco, así lo hizo el sacerdote
Urías, entre tanto que el rey Achâz venía de Damasco. \bibverse{12} Y
luego que vino el rey de Damasco, y hubo visto el altar, acercóse el rey
á él, y sacrificó en él; \bibverse{13} Y encendió su holocausto, y su
presente, y derramó sus libaciones, y esparció la sangre de sus
pacíficos junto al altar. \bibverse{14} Y el altar de bronce que estaba
delante de Jehová, hízolo acercar delante de la frontera de la casa,
entre el altar y el templo de Jehová, y púsolo al lado del altar hacia
el aquilón. \bibverse{15} Y mandó el rey Achâz al sacerdote Urías,
diciendo: En el gran altar encenderás el holocausto de la mañana y el
presente de la tarde, y el holocausto del rey y su presente, y asimismo
el holocausto de todo el pueblo de la tierra y su presente y sus
libaciones: y esparcirás sobre él toda la sangre de holocausto, y toda
la sangre de sacrificio: y el altar de bronce será mío para preguntar en
él. \bibverse{16} E hizo el sacerdote Urías conforme á todas las cosas
que el rey Achâz le mandó. \bibverse{17} Y cortó el rey Achâz las cintas
de las basas, y quitóles las fuentes; quitó también el mar de sobre los
bueyes de bronce que estaban debajo de él, y púsolo sobre el solado de
piedra. \bibverse{18} Asimismo la tienda del sábado que habían edificado
en la casa, y el pasadizo de afuera del rey, mudólos del templo de
Jehová, por causa del rey de Asiria. \bibverse{19} Lo demás de los
hechos de Achâz que puso por obra, ¿no está todo escrito en el libro de
las crónicas de los reyes de Judá? \bibverse{20} Y durmió el rey Achâz
con sus padres y fué sepultado con sus padres en la ciudad de David: y
reinó en su lugar Ezechîas su hijo.

\hypertarget{section-16}{%
\section{17}\label{section-16}}

\bibverse{1} En el año duodécimo de Achâz rey de Judá, comenzó á reinar
Oseas hijo de Ela en Samaria sobre Israel; y reinó nueve años.
\bibverse{2} E hizo lo malo en ojos de Jehová, aunque no como los reyes
de Israel que antes de él habían sido. \bibverse{3} Contra éste subió
Salmanasar rey de los Asirios; y Oseas fué hecho su siervo, y pagábale
tributo. \bibverse{4} Mas el rey de Asiria halló que Oseas hacía
conjuración: porque había enviado embajadores á So, rey de Egipto, y no
pagaba tributo al rey de Asiria, como cada año: por lo que el rey de
Asiria le detuvo, y le aprisionó en la casa de la cárcel. \bibverse{5} Y
el rey de Asiria partió contra todo el país, y subió contra Samaria, y
estuvo sobre ella tres años. \bibverse{6} En el año nueve de Oseas tomó
el rey de Asiria á Samaria, y trasportó á Israel á Asiria, y púsolos en
Hala, y en Habor, junto al río de Gozán, y en las ciudades de los Medos.
\bibverse{7} Porque como los hijos de Israel pecasen contra Jehová su
Dios, que los sacó de tierra de Egipto de bajo la mano de Faraón rey de
Egipto, y temiesen á dioses ajenos, \bibverse{8} Y anduviesen en los
estatutos de las gentes que Jehová había lanzado delante de los hijos de
Israel, y en los de los reyes de Israel, que hicieron; \bibverse{9} Y
como los hijos de Israel paliasen cosas no rectas contra Jehová su Dios,
edificándose altos en todas sus ciudades, desde las torres de las
atalayas hasta las ciudades fuertes, \bibverse{10} Y se levantasen
estatuas y bosques en todo collado alto, y debajo de todo árbol umbroso,
\bibverse{11} Y quemasen allí perfumes en todos los altos, á la manera
de las gentes que había Jehová traspuesto delante de ellos, é hiciesen
cosas muy malas para provocar á ira á Jehová, \bibverse{12} Pues servían
á los ídolos, de los cuales Jehová les había dicho: Vosotros no habéis
de hacer esto; \bibverse{13} Jehová protestaba entonces contra Israel y
contra Judá, por mano de todos los profetas, y de todos los videntes,
diciendo: Volveos de vuestros malos caminos, y guardad mis mandamientos
y mis ordenanzas, conforme á todas las leyes que yo prescribí á vuestros
padres, y que os he enviado por mano de mis siervos los profetas.
\bibverse{14} Mas ellos no obedecieron, antes endurecieron su cerviz,
como la cerviz de sus padres, los cuales no creyeron en Jehová su Dios.
\bibverse{15} Y desecharon sus estatutos, y su pacto que él había
concertado con sus padres, y sus testimonios que él había protestado
contra ellos; y siguieron la vanidad, y se hicieron vanos, y fueron en
pos de las gentes que estaban alrededor de ellos, de las cuales les
había Jehová mandado que no hiciesen á la manera de ellas: \bibverse{16}
Y dejaron todos los mandamientos de Jehová su Dios, é hiciéronse
vaciadizos dos becerros, y también bosques, y adoraron á todo el
ejército del cielo, y sirvieron á Baal: \bibverse{17} E hicieron pasar á
sus hijos y á sus hijas por fuego; y diéronse á adivinaciones y agüeros,
y entregáronse á hacer lo malo en ojos de Jehová, provocándole á ira.
\bibverse{18} Jehová por tanto se airó en gran manera contra Israel, y
quitólos de delante de su rostro; que no quedó sino sólo la tribu de
Judá. \bibverse{19} Mas ni aun Judá guardó los mandamientos de Jehová su
Dios; antes anduvieron en los estatutos de Israel, los cuales habían
ellos hecho. \bibverse{20} Y desechó Jehová toda la simiente de Israel,
y afligiólos, y entrególos en manos de saqueadores, hasta echarlos de su
presencia. \bibverse{21} Porque cortó á Israel de la casa de David, y
ellos se hicieron rey á Jeroboam hijo de Nabat; y Jeroboam rempujó á
Israel de en pos de Jehová, é hízoles cometer gran pecado. \bibverse{22}
Y los hijos de Israel anduvieron en todos los pecados de Jeroboam que él
hizo, sin apartarse de ellos; \bibverse{23} Hasta tanto que Jehová quitó
á Israel de delante de su rostro, como lo había él dicho por mano de
todos los profetas sus siervos: é Israel fué trasportado de su tierra á
Asiria, hasta hoy. \bibverse{24} Y trajo el rey de Asiria gente de
Babilonia, y de Cutha, y de Ava, y de Hamath, y de Sepharvaim, y púsolos
en las ciudades de Samaria, en lugar de los hijos de Israel; y poseyeron
á Samaria, y habitaron en sus ciudades. \bibverse{25} Y aconteció al
principio, cuando comenzaron á habitar allí, que no temiendo ellos á
Jehová, envió Jehová contra ellos leones que los mataban. \bibverse{26}
Entonces dijeron ellos al rey de Asiria: Las gentes que tú traspasaste y
pusiste en las ciudades de Samaria, no saben la costumbre del Dios de
aquella tierra, y él ha echado leones en ellos, y he aquí los matan,
porque no saben la costumbre del Dios de la tierra. \bibverse{27} Y el
rey de Asiria mandó, diciendo: Llevad allí á alguno de los sacerdotes
que trajisteis de allá, y vayan y habiten allí, y enséñenles la
costumbre del Dios del país. \bibverse{28} Y vino uno de los sacerdotes
que habían trasportado de Samaria, y habitó en Beth-el, y enseñóles cómo
habían de temer á Jehová. \bibverse{29} Mas cada nación se hizo sus
dioses, y pusiéronlos en los templos de los altos que habían hecho los
de Samaria; cada nación en su ciudad donde habitaba. \bibverse{30} Los
de Babilonia hicieron á Succoth-benoth, y los de Cutha hicieron á
Nergal, y los de Hamath hicieron á Asima; \bibverse{31} Los Heveos
hicieron á Nibhaz y á Tharthac; y los de Sepharvaim quemaban sus hijos
al fuego á Adra-melech y á Anamelech, dioses de Sepharvaim.
\bibverse{32} Y temían á Jehová; é hicieron del pueblo bajo sacerdotes
de los altos, quienes sacrificaban para ellos en los templos de los
altos. \bibverse{33} Temían á Jehová, y honraban á sus dioses, según la
costumbre de las gentes de donde habían sido trasladados. \bibverse{34}
Hasta hoy hacen como primero; que ni temen á Jehová, ni guardan sus
estatutos, ni sus ordenanzas, ni hacen según la ley y los mandamientos
que prescribió Jehová á los hijos de Jacob, al cual puso el nombre de
Israel; \bibverse{35} Con los cuales había Jehová hecho pacto, y les
mandó, diciendo: No temeréis á otros dioses, ni los adoraréis, ni les
serviréis, ni les sacrificaréis: \bibverse{36} Mas á Jehová, que os sacó
de tierra de Egipto con grande poder y brazo extendido, á éste temeréis,
y á éste adoraréis, y á éste haréis sacrificio. \bibverse{37} Los
estatutos y derechos y ley y mandamientos que os dió por escrito,
cuidaréis siempre de ponerlos por obra, y no temeréis dioses ajenos.
\bibverse{38} Y no olvidaréis el pacto que hice con vosotros; ni
temeréis dioses ajenos: \bibverse{39} Mas temed á Jehová vuestro Dios, y
él os librará de mano de todos vuestros enemigos. \bibverse{40} Empero
ellos no escucharon; antes hicieron según su costumbre antigua.
\bibverse{41} Así temieron á Jehová aquellas gentes, y juntamente
sirvieron á sus ídolos: y también sus hijos y sus nietos, según que
hicieron sus padres, así hacen hasta hoy.

\hypertarget{section-17}{%
\section{18}\label{section-17}}

\bibverse{1} En el tercer año de Oseas hijo de Ela rey de Israel,
comenzó á reinar Ezechîas hijo de Achâz rey de Judá. \bibverse{2} Cuando
comenzó á reinar era de veinticinco años, y reinó en Jerusalem
veintinueve años. El nombre de su madre fué Abi hija de Zachârías.
\bibverse{3} Hizo lo recto en ojos de Jehová, conforme á todas las cosas
que había hecho David su padre. \bibverse{4} El quitó los altos, y
quebró las imágenes, y taló los bosques, é hizo pedazos la serpiente de
bronce que había hecho Moisés, porque hasta entonces le quemaban
perfumes los hijos de Israel; y llamóle por nombre Nehustán.
\bibverse{5} En Jehová Dios de Israel puso su esperanza: después ni
antes de él no hubo otro como él en todos los reyes de Judá.
\bibverse{6} Porque se llegó á Jehová, y no se apartó de él, sino que
guardó los mandamientos que Jehová prescribió á Moisés. \bibverse{7} Y
Jehová fué con él; y en todas las cosas á que salía prosperaba. El se
rebeló contra el rey de Asiria, y no le sirvió. \bibverse{8} Hirió
también á los Filisteos hasta Gaza y sus términos, desde las torres de
las atalayas hasta la ciudad fortalecida. \bibverse{9} En el cuarto año
del rey Ezechîas, que era el año séptimo de Oseas hijo de Ela rey de
Israel, subió Salmanasar rey de los Asirios contra Samaria, y cercóla.
\bibverse{10} Y tomáronla al cabo de tres años; esto es, en el sexto año
de Ezechîas, el cual era el año nono de Oseas rey de Israel, fué Samaria
tomada. \bibverse{11} Y el rey de Asiria traspuso á Israel á Asiria, y
púsolos en Hala, y en Habor, junto al río de Gozán, y en las ciudades de
los Medos: \bibverse{12} Por cuanto no habían atendido la voz de Jehová
su Dios, antes habían quebrantado su pacto; y todas las cosas que Moisés
siervo de Jehová había mandado, ni las habían escuchado, ni puesto por
obra. \bibverse{13} Y á los catorce años del rey Ezechîas, subió
Sennachêrib rey de Asiria contra todas las ciudades fuertes de Judá, y
tomólas. \bibverse{14} Entonces Ezechîas rey de Judá envió á decir al
rey de Asiria en Lachîs: Yo he pecado: vuélvete de mí, y llevaré todo lo
que me impusieres. Y el rey de Asiria impuso á Ezechîas rey de Judá
trescientos talentos de plata, y treinta talentos de oro. \bibverse{15}
Dió por tanto Ezechîas toda la plata que fué hallada en la casa de
Jehová, y en los tesoros de la casa real. \bibverse{16} Entonces
descompuso Ezechîas las puertas del templo de Jehová, y los quiciales
que el mismo rey Ezechîas había cubierto de oro, y diólo al rey de
Asiria. \bibverse{17} Después el rey de Asiria envió al rey Ezechîas,
desde Lachîs contra Jerusalem, á Thartán y á Rabsaris y á Rabsaces, con
un grande ejército: y subieron, y vinieron á Jerusalem. Y habiendo
subido, vinieron y pararon junto al conducto del estanque de arriba, que
es en el camino de la heredad del batanero. \bibverse{18} Llamaron luego
al rey, y salió á ellos Eliacim hijo de Hilcías, que era mayordomo, y
Sebna escriba, y Joah hijo de Asaph, canciller. \bibverse{19} Y díjoles
Rabsaces: Decid ahora á Ezechîas: Así dice el gran rey de Asiria: ¿Qué
confianza es esta en que tú estás? \bibverse{20} Dices, (por cierto
palabras de labios): Consejo tengo y esfuerzo para la guerra. Mas ¿en
qué confías, que te has rebelado contra mí? \bibverse{21} He aquí tú
confías ahora en este bordón de caña cascada, en Egipto, en el que si
alguno se apoyare, entrarále por la mano, y se le pasará. Tal es Faraón
rey de Egipto, para todos los que en él confían. \bibverse{22} Y si me
decís: Nosotros confiamos en Jehová nuestro Dios: ¿no es aquél cuyos
altos y altares ha quitado Ezechîas, y ha dicho á Judá y á Jerusalem:
Delante de este altar adoraréis en Jerusalem? \bibverse{23} Por tanto,
ahora yo te ruego que des rehenes á mi señor, el rey de Asiria, y yo te
daré dos mil caballos, si tú pudieres dar jinetes para ellos.
\bibverse{24} ¿Cómo pues harás volver el rostro de un capitán el menor
de los siervos de mi señor, aunque estés confiado en Egipto por sus
carros y su gente de á caballo? \bibverse{25} Además, ¿he venido yo
ahora sin Jehová á este lugar, para destruirlo? Jehová me ha dicho: Sube
á esta tierra, y destrúyela. \bibverse{26} Entonces dijo Eliacim hijo de
Hilcías, y Sebna y Joah, á Rabsaces: Ruégote que hables á tus siervos
siriaco, porque nosotros lo entendemos, y no hables con nosotros judaico
á oídos del pueblo que está sobre el muro. \bibverse{27} Y Rabsaces les
dijo: ¿Hame enviado mi señor á ti y á tu señor para decir estas
palabras, y no antes á los hombres que están sobre el muro, para comer
su estiércol, y beber el agua de sus pies con vosotros? \bibverse{28}
Paróse luego Rabsaces, y clamó á gran voz en judaico, y habló, diciendo:
Oid la palabra del gran rey, el rey de Asiria. \bibverse{29} Así ha
dicho el rey: No os engañe Ezechîas, porque no os podrá librar de mi
mano. \bibverse{30} Y no os haga Ezechîas confiar en Jehová, diciendo:
De cierto nos librará Jehová, y esta ciudad no será entregada en mano
del rey de Asiria. \bibverse{31} No oigáis á Ezechîas, porque así dice
el rey de Asiria: Haced conmigo paz, y salid á mí, y cada uno comerá de
su vid, y de su higuera, y cada uno beberá las aguas de su pozo;
\bibverse{32} Hasta que yo venga, y os lleve á una tierra como la
vuestra, tierra de grano y de vino, tierra de pan y de viñas, tierra de
olivas, de aceite, y de miel; y viviréis, y no moriréis. No oigáis á
Ezechîas, porque os engaña cuando dice: Jehová nos librará.
\bibverse{33} ¿Acaso alguno de los dioses de las gentes ha librado su
tierra de la mano del rey de Asiria? \bibverse{34} ¿Dónde está el dios
de Hamath, y de Arphad? ¿dónde el dios de Sepharvaim, de Hena, y de
Hiva? ¿pudieron éstos librar á Samaria de mi mano? \bibverse{35} ¿Qué
dios de todos los dioses de las provincias ha librado á su provincia de
mi mano, para que libre Jehová de mi mano á Jerusalem? \bibverse{36} Y
el pueblo calló, que no le respondieron palabra: porque había
mandamiento del rey, el cual había dicho: No le respondáis.
\bibverse{37} Entonces Eliacim hijo de Hilcías, que era mayordomo, y
Sebna el escriba, y Joah hijo de Asaph, canciller, vinieron á Ezechîas,
rotos sus vestidos, y recitáronle las palabras de Rabsaces.

\hypertarget{section-18}{%
\section{19}\label{section-18}}

\bibverse{1} Y como el rey Ezechîas lo oyó, rasgó sus vestidos, y
cubrióse de saco, y entróse en la casa de Jehová. \bibverse{2} Y envió á
Eliacim el mayordomo, y á Sebna escriba, y á los ancianos de los
sacerdotes, vestidos de sacos á Isaías profeta hijo de Amós,
\bibverse{3} Que le dijesen: Así ha dicho Ezechîas: Este día es día de
angustia, y de reprensión, y de blasfemia; porque los hijos han venido
hasta la rotura, y la que pare no tiene fuerzas. \bibverse{4} Quizá oirá
Jehová tu Dios todas las palabras de Rabsaces, al cual el rey de los
Asirios su señor ha enviado para injuriar al Dios vivo, y á vituperar
con palabras, las cuales Jehová tu Dios ha oído: por tanto, eleva
oración por las reliquias que aun se hallan. \bibverse{5} Vinieron pues
los siervos del rey Ezechîas á Isaías. \bibverse{6} E Isaías les
respondió: Así diréis á vuestro señor: Así ha dicho Jehová; No temas por
las palabras que has oído, con las cuales me han blasfemado los siervos
del rey de Asiria. \bibverse{7} He aquí pondré yo en él un espíritu, y
oirá rumor, y volveráse á su tierra: y yo haré que en su tierra caiga á
cuchillo. \bibverse{8} Y regresando Rabsaces, halló al rey de Asiria
combatiendo á Libna; porque había oído que se había partido de Lachîs.
\bibverse{9} Y oyó decir de Thiraca rey de Ethiopía: He aquí es salido
para hacerte guerra. Entonces volvió él, y envió embajadores á Ezechîas,
diciendo: \bibverse{10} Así diréis á Ezechîas rey de Judá: No te engañe
tu Dios en quien tú confías, para decir: Jerusalem no será entregada en
mano del rey de Asiria. \bibverse{11} He aquí tú has oído lo que han
hecho los reyes de Asiria á todas las tierras, destruyéndolas; ¿y has tú
de escapar? \bibverse{12} ¿Libráronlas los dioses de las gentes, que mis
padres destruyeron, es á saber, Gozán, y Harán, y Reseph, y los hijos de
Edén que estaban en Thalasar? \bibverse{13} ¿Dónde está el rey de
Hamath, el rey de Arphad, el rey de la ciudad de Sepharvaim, de Hena, y
de Hiva? \bibverse{14} Y tomó Ezechîas las letras de mano de los
embajadores; y después que las hubo leído, subió á la casa de Jehová, y
extendiólas Ezechîas delante de Jehová. \bibverse{15} Y oró Ezechîas
delante de Jehová, diciendo: Jehová Dios de Israel, que habitas entre
los querubines, tú solo eres Dios de todos los reinos de la tierra; tú
hiciste el cielo y la tierra. \bibverse{16} Inclina, oh Jehová, tu oído,
y oye; abre, oh Jehová, tus ojos, y mira: y oye las palabras de
Sennachêrib, que ha enviado á blasfemar al Dios viviente. \bibverse{17}
Es verdad, oh Jehová, que los reyes de Asiria han destruído las gentes y
sus tierras; \bibverse{18} Y que pusieron en el fuego á sus dioses, por
cuanto ellos no eran dioses, sino obra de manos de hombres, madera ó
piedra, y así los destruyeron. \bibverse{19} Ahora pues, oh Jehová Dios
nuestro, sálvanos, te suplico, de su mano, para que sepan todos los
reinos de la tierra que tú solo, Jehová, eres Dios. \bibverse{20}
Entonces Isaías hijo de Amós envió á decir á Ezechîas: Así ha dicho
Jehová, Dios de Israel: Lo que me rogaste acerca de Sennachêrib rey de
Asiria, he oído. \bibverse{21} Esta es la palabra que Jehová ha hablado
contra él: Hate menospreciado, hate escarnecido la virgen hija de Sión;
ha movido su cabeza detrás de ti la hija de Jerusalem. \bibverse{22} ¿A
quién has injuriado y á quién has blasfemado? ¿y contra quién has
hablado alto, y has alzado en alto tus ojos? Contra el Santo de Israel.
\bibverse{23} Por mano de tus mensajeros has proferido injuria contra el
Señor, y has dicho: Con la multitud de mis carros he subido á las
cumbres de los montes, á las cuestas del Líbano; y cortaré sus altos
cedros, sus hayas escogidas; y entraré á la morada de su término, al
monte de su Carmel. \bibverse{24} Yo he cavado y bebido las aguas
ajenas, y he secado con las plantas de mis pies todos los ríos de
lugares bloqueados. \bibverse{25} ¿Nunca has oído que mucho tiempo ha yo
lo hice, y de días antiguos lo he formado? Y ahora lo he hecho venir, y
fué para desolación de ciudades fuertes en montones de ruinas.
\bibverse{26} Y sus moradores, cortos de manos, quebrantados y confusos,
fueron cual hierba del campo, como legumbre verde, y heno de los
tejados, que antes que venga á madurez es seco. \bibverse{27} Yo he
sabido tu asentarte, tu salir y tu entrar, y tu furor contra mí.
\bibverse{28} Por cuanto te has airado contra mí, y tu estruendo ha
subido á mis oídos, yo por tanto pondré mi anzuelo en tus narices, y mi
bocado en tus labios, y te haré volver por el camino por donde viniste.
\bibverse{29} Y esto te será por señal Ezechîas: Este año comerás lo que
nacerá de suyo, y el segundo año lo que nacerá de suyo; y el tercer año
haréis sementera, y segaréis, y plantaréis viñas, y comeréis el fruto de
ellas. \bibverse{30} Y lo que hubiere escapado, lo que habrá quedado de
la casa de Judá, tornará á echar raíz abajo, y hará fruto arriba.
\bibverse{31} Porque saldrán de Jerusalem reliquias, y los que
escaparán, del monte de Sión: el celo de Jehová de los ejércitos hará
esto. \bibverse{32} Por tanto, Jehová dice así del rey de Asiria: No
entrará en esta ciudad, ni echará saeta en ella; ni vendrá delante de
ella escudo, ni será echado contra ella baluarte. \bibverse{33} Por el
camino que vino se volverá, y no entrará en esta ciudad, dice Jehová.
\bibverse{34} Porque yo ampararé á esta ciudad para salvarla, por amor
de mí, y por amor de David mi siervo. \bibverse{35} Y aconteció que la
misma noche salió el ángel de Jehová, é hirió en el campo de los Asirios
ciento ochenta y cinco mil; y como se levantaron por la mañana, he aquí
los cuerpos de los muertos. \bibverse{36} Entonces Sennachêrib, rey de
Asiria se partió, y se fué y tornó á Nínive, donde se estuvo.
\bibverse{37} Y aconteció que, estando él adorando en el templo de
Nisroch su dios, Adramelech y Saresar sus hijos lo hirieron á cuchillo;
y huyéronse á tierra de Ararat. Y reinó en su lugar Esar-hadón su hijo.

\hypertarget{section-19}{%
\section{20}\label{section-19}}

\bibverse{1} En aquellos días cayó Ezechîas enfermo de muerte, y vino á
él Isaías profeta hijo de Amós, y díjole: Jehová dice así: Dispón de tu
casa, porque has de morir, y no vivirás. \bibverse{2} Entonces volvió él
su rostro á la pared, y oró á Jehová, y dijo: \bibverse{3} Ruégote, oh
Jehová, ruégote hagas memoria de que he andado delante de ti en verdad é
íntegro corazón, y que he hecho las cosas que te agradan. Y lloró
Ezechîas con gran lloro. \bibverse{4} Y antes que Isaías saliese hasta
la mitad del patio, fué palabra de Jehová á Isaías, diciendo:
\bibverse{5} Vuelve, y di á Ezechîas, príncipe de mi pueblo: Así dice
Jehová, el Dios de David tu padre: Yo he oído tu oración, y he visto tus
lágrimas: he aquí yo te sano; al tercer día subirás á la casa de Jehová.
\bibverse{6} Y añadiré á tus días quince años, y te libraré á ti y á
esta ciudad de mano del rey de Asiria; y ampararé esta ciudad por amor
de mí, y por amor de David mi siervo. \bibverse{7} Y dijo Isaías: Tomad
masa de higos. Y tomándola, pusieron sobre la llaga, y sanó.
\bibverse{8} Y Ezechîas había dicho á Isaías: ¿Qué señal tendré de que
Jehová me sanará, y que subiré á la casa de Jehová al tercer día?
\bibverse{9} Y respondió Isaías: Esta señal tendrás de Jehová, de que
hará Jehová esto que ha dicho: ¿Avanzará la sombra diez grados, ó
retrocederá diez grados? \bibverse{10} Y Ezechîas respondió: Fácil cosa
es que la sombra decline diez grados: pero, que la sombra vuelva atrás
diez grados. \bibverse{11} Entonces el profeta Isaías clamó á Jehová; é
hizo volver la sombra por los grados que había descendido en el reloj de
Achâz, diez grados atrás. \bibverse{12} En aquel tiempo Berodach-baladán
hijo de Baladán, rey de Babilonia, envió letras y presentes á Ezechîas,
porque había oído que Ezechîas había caído enfermo. \bibverse{13} Y
Ezechîas los oyó, y mostróles toda la casa de las cosas preciosas,
plata, oro, y especiería, y preciosos ungüentos; y la casa de sus armas,
y todo lo que había en sus tesoros: ninguna cosa quedó que Ezechîas no
les mostrase, así en su casa como en todo su señorío. \bibverse{14}
Entonces el profeta Isaías vino al rey Ezechîas, y díjole: ¿Qué dijeron
aquellos varones, y de dónde vinieron á ti? Y Ezechîas le respondió: De
lejanas tierras han venido, de Babilonia. \bibverse{15} Y él le volvió á
decir: ¿Qué vieron en tu casa? Y Ezechîas respondió: Vieron todo lo que
había en mi casa; nada quedó en mis tesoros que no les mostrase.
\bibverse{16} Entonces Isaías dijo á Ezechîas: Oye palabra de Jehová:
\bibverse{17} He aquí vienen días, en que todo lo que está en tu casa, y
todo lo que tus padres han atesorado hasta hoy, será llevado á
Babilonia, sin quedar nada, dijo Jehová. \bibverse{18} Y de tus hijos
que saldrán de ti, que habrás engendrado, tomarán; y serán eunucos en el
palacio del rey de Babilonia. \bibverse{19} Entonces Ezechîas dijo á
Isaías: La palabra de Jehová que has hablado, es buena. Después dijo:
¿Mas no habrá paz y verdad en mis días? \bibverse{20} Lo demás de los
hechos de Ezechîas, y todo su vigor, y cómo hizo el estanque, y el
conducto, y metió las aguas en la ciudad, ¿no está escrito en el libro
de las crónicas de los reyes de Judá? \bibverse{21} Y durmió Ezechîas
con sus padres, y reinó en su lugar Manasés su hijo.

\hypertarget{section-20}{%
\section{21}\label{section-20}}

\bibverse{1} De doce años era Manasés cuando comenzó á reinar, y reinó
en Jerusalem cincuenta y cinco años: el nombre de su madre fué Hepsiba.
\bibverse{2} E hizo lo malo en ojos de Jehová, según las abominaciones
de las gentes que Jehová había echado delante de los hijos de Israel.
\bibverse{3} Porque él volvió á edificar los altos que Ezechîas su padre
había derribado, y levantó altares á Baal, é hizo bosque, como había
hecho Achâb rey de Israel: y adoró á todo el ejército del cielo, y
sirvió á aquellas cosas. \bibverse{4} Asimismo edificó altares en la
casa de Jehová, de la cual Jehová había dicho: Yo pondré mi nombre en
Jerusalem. \bibverse{5} Y edificó altares para todo el ejército del
cielo en los dos atrios de la casa de Jehová. \bibverse{6} Y pasó á su
hijo por fuego, y miró en tiempos, y fué agorero, é instituyó pythones y
adivinos, multiplicando así el hacer lo malo en ojos de Jehová, para
provocarlo á ira. \bibverse{7} Y puso una entalladura del bosque que él
había hecho, en la casa de la cual había Jehová dicho á David y á
Salomón su hijo: Yo pondré mi nombre para siempre en esta casa, y en
Jerusalem, á la cual escogí de todas las tribus de Israel: \bibverse{8}
Y no volveré á hacer que el pie de Israel sea movido de la tierra que dí
á sus padres, con tal que guarden y hagan conforme á todas las cosas que
yo les he mandado, y conforme á toda la ley que mi siervo Moisés les
mandó. \bibverse{9} Mas ellos no escucharon; y Manasés los indujo á que
hiciesen más mal que las gentes que Jehová destruyó delante de los hijos
de Israel. \bibverse{10} Y habló Jehová por mano de sus siervos los
profetas, diciendo: \bibverse{11} Por cuanto Manasés rey de Judá ha
hecho estas abominaciones, y ha hecho más mal que todo lo que hicieron
los Amorrheos que fueron antes de él, y también ha hecho pecar á Judá en
sus ídolos; \bibverse{12} Por tanto, así ha dicho Jehová el Dios de
Israel: He aquí yo traigo tal mal sobre Jerusalem y sobre Judá, que el
que lo oyere, le retiñirán ambos oídos. \bibverse{13} Y extenderé sobre
Jerusalem el cordel de Samaria, y el plomo de la casa de Achâb: y yo
limpiaré á Jerusalem como se limpia una escudilla, que después que la
han limpiado, la vuelven sobre su haz. \bibverse{14} Y desampararé las
reliquias de mi heredad, y entregarlas he en manos de sus enemigos; y
serán para saco y para robo á todos sus adversarios; \bibverse{15} Por
cuanto han hecho lo malo en mis ojos, y me han provocado á ira, desde el
día que sus padres salieron de Egipto hasta hoy. \bibverse{16} Fuera de
esto, derramó Manasés mucha sangre inocente en gran manera, hasta
henchir á Jerusalem de cabo á cabo: además de su pecado con que hizo
pecar á Judá, para que hiciese lo malo en ojos de Jehová. \bibverse{17}
Lo demás de los hechos de Manasés, y todas las cosas que hizo, y su
pecado que cometió, ¿no está todo escrito en el libro de las crónicas de
los reyes de Judá? \bibverse{18} Y durmió Manasés con sus padres, y fué
sepultado en el huerto de su casa, en el huerto de Uzza; y reinó en su
lugar Amón su hijo. \bibverse{19} De veinte y dos años era Amón cuando
comenzó á reinar, y reinó dos años en Jerusalem. El nombre de su madre
fué Mesalemeth hija de Harus de Jotba. \bibverse{20} E hizo lo malo en
ojos de Jehová, como había hecho Manasés su padre. \bibverse{21} Y
anduvo en todos los caminos en que su padre anduvo, y sirvió á las
inmundicias á las cuales había servido su padre, y á ellas adoró;
\bibverse{22} Y dejó á Jehová el Dios de sus padres, y no anduvo en el
camino de Jehová. \bibverse{23} Y los siervos de Amón conspiraron contra
él, y mataron al rey en su casa. \bibverse{24} Entonces el pueblo de la
tierra hirió á todos los que habían conspirado contra el rey Amón; y
puso el pueblo de la tierra por rey en su lugar á Josías su hijo.
\bibverse{25} Lo demás de los hechos de Amón, que efectuara, ¿no está
todo escrito en el libro de las crónicas de los reyes de Judá?
\bibverse{26} Y fué sepultado en su sepulcro en el huerto de Uzza, y
reinó en su lugar Josías su hijo.

\hypertarget{section-21}{%
\section{22}\label{section-21}}

\bibverse{1} Cuando Josías comenzó á reinar era de ocho años, y reinó en
Jerusalem treinta y un años. El nombre de su madre fué Idida hija de
Adaía de Boscath. \bibverse{2} E hizo lo recto en ojos de Jehová, y
anduvo en todo el camino de David su padre, sin apartarse á diestra ni á
siniestra. \bibverse{3} Y á los dieciocho años del rey Josías, fué que
envió el rey á Saphán hijo de Azalía, hijo de Mesullam, escriba, á la
casa de Jehová, diciendo: \bibverse{4} Ve á Hilcías, sumo sacerdote:
dile que recoja el dinero que se ha metido en la casa de Jehová, que han
juntado del pueblo los guardianes de la puerta, \bibverse{5} Y que lo
pongan en manos de los que hacen la obra, que tienen cargo de la casa de
Jehová, y que lo entreguen á los que hacen la obra de la casa de Jehová,
para reparar las aberturas de la casa: \bibverse{6} A los carpinteros, á
los maestros y albañiles, para comprar madera y piedra de cantería para
reparar la casa; \bibverse{7} Y que no se les cuente el dinero cuyo
manejo se les confiare, porque ellos proceden con fidelidad.
\bibverse{8} Entonces dijo el sumo sacerdote Hilcías á Saphán escriba:
El libro de la ley he hallado en la casa de Jehová. E Hilcías dió el
libro á Saphán, y leyólo. \bibverse{9} Viniendo luego Saphán escriba al
rey, dió al rey la respuesta, y dijo: Tus siervos han juntado el dinero
que se halló en el templo, y lo han entregado en poder de los que hacen
la obra, que tienen cargo de la casa de Jehová. \bibverse{10} Asimismo
Saphán escriba declaró al rey, diciendo: Hilcías el sacerdote me ha dado
un libro. Y leyólo Saphán delante del rey. \bibverse{11} Y cuando el rey
hubo oído las palabras del libro de la ley, rasgó sus vestidos.
\bibverse{12} Luego mandó el rey á Hilcías el sacerdote, y á Ahicam hijo
de Saphán, y á Achbor hijo de Michâía, y á Saphán escriba, y á Asaía
siervo del rey, diciendo: \bibverse{13} Id, y preguntad á Jehová por mí,
y por el pueblo, y por todo Judá, acerca de las palabras de este libro
que se ha hallado: porque grande ira de Jehová es la que ha sido
encendida contra nosotros, por cuanto nuestros padres no escucharon las
palabras de este libro, para hacer conforme á todo lo que nos fué
escrito. \bibverse{14} Entonces fué Hilcías el sacerdote, y Ahicam y
Achbor y Saphán y Asaía, á Hulda profetisa, mujer de Sallum hijo de
Ticva hijo de Araas, guarda de las vestiduras, la cual moraba en
Jerusalem en la segunda parte de la ciudad, y hablaron con ella.
\bibverse{15} Y ella les dijo: Así ha dicho Jehová el Dios de Israel:
Decid al varón que os envió á mí: \bibverse{16} Así dijo Jehová: He aquí
yo traigo mal sobre este lugar, y sobre los que en él moran, á saber,
todas las palabras del libro que ha leído el rey de Judá: \bibverse{17}
Por cuanto me dejaron á mí, y quemaron perfumes á dioses ajenos,
provocándome á ira en toda obra de sus manos; y mi furor se ha encendido
contra este lugar, y no se apagará. \bibverse{18} Mas al rey de Judá que
os ha enviado para que preguntaseis á Jehová, diréis así: Así ha dicho
Jehová el Dios de Israel: Por cuanto oíste las palabras del libro,
\bibverse{19} Y tu corazón se enterneció, y te humillaste delante de
Jehová, cuando oíste lo que yo he pronunciado contra este lugar y contra
sus moradores, que vendrían á ser asolados y malditos, y rasgaste tus
vestidos, y lloraste en mi presencia, también yo te he oído, dice
Jehová. \bibverse{20} Por tanto, he aquí yo te recogeré con tus padres,
y tú serás recogido á tu sepulcro en paz, y no verán tus ojos todo el
mal que yo traigo sobre este lugar. Y ellos dieron al rey la respuesta.

\hypertarget{section-22}{%
\section{23}\label{section-22}}

\bibverse{1} Entonces el rey envió, y juntaron á él todos los ancianos
de Judá y de Jerusalem. \bibverse{2} Y subió el rey á la casa de Jehová
con todos los varones de Judá, y con todos los moradores de Jerusalem,
con los sacerdotes y profetas y con todo el pueblo, desde el más chico
hasta el más grande; y leyó, oyéndolo ellos, todas las palabras del
libro del pacto que había sido hallado en la casa de Jehová.
\bibverse{3} Y poniéndose el rey en pie junto á la columna, hizo alianza
delante de Jehová, de que irían en pos de Jehová, y guardarían sus
mandamientos, y sus testimonios, y sus estatutos, con todo el corazón y
con toda el alma, y que cumplirían las palabras de la alianza que
estaban escritas en aquel libro. Y todo el pueblo confirmó el pacto.
\bibverse{4} Entonces mandó el rey al sumo sacerdote Hilcías, y á los
sacerdotes de segundo orden, y á los guardianes de la puerta, que
sacasen del templo de Jehová todos los vasos que habían sido hechos para
Baal, y para el bosque, y para toda la milicia del cielo; y quemólos
fuera de Jerusalem en el campo de Cedrón, é hizo llevar las cenizas de
ellos á Beth-el. \bibverse{5} Y quitó á los Camoreos, que habían puesto
los reyes de Judá para que quemasen perfumes en los altos en las
ciudades de Judá, y en los alrededores de Jerusalem; y asimismo á los
que quemaban perfumes á Baal, al sol y á la luna, y á los signos, y á
todo el ejército del cielo. \bibverse{6} Hizo también sacar el bosque
fuera de la casa de Jehová, fuera de Jerusalem, al torrente de Cedrón, y
quemólo en el torrente de Cedrón, y tornólo en polvo, y echó el polvo de
él sobre los sepulcros de los hijos del pueblo. \bibverse{7} Además
derribó las casas de los sodomitas que estaban en la casa de Jehová, en
las cuales tejían las mujeres pabellones para el bosque. \bibverse{8} E
hizo venir todos los sacerdotes de las ciudades de Judá, y profanó los
altos donde los sacerdotes quemaban perfumes, desde Gabaa hasta
Beer-seba; y derribó los altares de las puertas que estaban á la entrada
de la puerta de Josué, gobernador de la ciudad, que estaban á la mano
izquierda, á la puerta de la ciudad. \bibverse{9} Empero los sacerdotes
de los altos no subían al altar de Jehová en Jerusalem, mas comían panes
sin levadura entre sus hermanos. \bibverse{10} Asimismo profanó á
Topheth, que está en el valle del hijo de Hinnom, porque ninguno pasase
su hijo ó su hija por fuego á Moloch. \bibverse{11} Quitó también los
caballos que los reyes de Judá habían dedicado al sol á la entrada del
templo de Jehová, junto á la cámara de Nathan-melech eunuco, el cual
tenía cargo de los ejidos; y quemó al fuego los carros del sol.
\bibverse{12} Derribó además el rey los altares que estaban sobre la
techumbre de la sala de Achâz, que los reyes de Judá habían hecho, y los
altares que había hecho Manasés en los dos atrios de la casa de Jehová;
y de allí corrió y arrojó el polvo en el torrente de Cedrón.
\bibverse{13} Asimismo profanó el rey los altos que estaban delante de
Jerusalem, á la mano derecha del monte de la destrucción, los cuales
Salomón rey de Israel había edificado á Astharoth, abominación de los
Sidonios, y á Chêmos abominación de Moab, y á Milcom abominación de los
hijos de Ammón. \bibverse{14} Y quebró las estatuas, y taló los bosques,
é hinchió el lugar de ellos de huesos de hombres. \bibverse{15}
Igualmente el altar que estaba en Beth-el, y el alto que había hecho
Jeroboam hijo de Nabat, el que hizo pecar á Israel, aquel altar y el
alto destruyó; y quemó el alto, y lo tornó en polvo, y puso fuego al
bosque. \bibverse{16} Y volvióse Josías, y viendo los sepulcros que
estaban allí en el monte, envió y sacó los huesos de los sepulcros, y
quemólos sobre el altar para contaminarlo, conforme á la palabra de
Jehová que había profetizado el varón de Dios, el cual había anunciado
estos negocios. \bibverse{17} Y después dijo: ¿Qué título es este que
veo? Y los de la ciudad le respondieron: Este es el sepulcro del varón
de Dios que vino de Judá, y profetizó estas cosas que tú has hecho sobre
el altar de Beth-el. \bibverse{18} Y él dijo: Dejadlo; ninguno mueva sus
huesos: y así fueron preservados sus huesos, y los huesos del profeta
que había venido de Samaria. \bibverse{19} Y todas las casas de los
altos que estaban en las ciudades de Samaria, las cuales habían hecho
los reyes de Israel para provocar á ira, quitólas también Josías, é hizo
de ellas como había hecho en Beth-el. \bibverse{20} Mató además sobre
los altares á todos los sacerdotes de los altos que allí estaban, y
quemó sobre ellos huesos de hombres, y volvióse á Jerusalem.
\bibverse{21} Entonces mandó el rey á todo el pueblo, diciendo: Haced la
pascua á Jehová vuestro Dios, conforme á lo que está escrito en el libro
de esta alianza. \bibverse{22} No fué hecha tal pascua desde los tiempos
de los jueces que gobernaron á Israel, ni en todos los tiempos de los
reyes de Israel, y de los reyes de Judá. \bibverse{23} A los diez y ocho
años del rey Josías fué hecha aquella pascua á Jehová en Jerusalem.
\bibverse{24} Asimismo barrió Josías los pythones, adivinos, y
terapheos, y todas las abominaciones que se veían en la tierra de Judá y
en Jerusalem, para cumplir las palabras de la ley que estaban escritas
en el libro que el sacerdote Hilcías había hallado en la casa de Jehová.
\bibverse{25} No hubo tal rey antes de él que se convirtiese á Jehová de
todo su corazón, y de toda su alma, y de todas su fuerzas, conforme á
toda la ley de Moisés; ni después de él nació otro tal. \bibverse{26}
Con todo eso, no se volvió Jehová del ardor de su grande ira, con que se
había encendido su enojo contra Judá, por todas las provocaciones con
que Manasés le había irritado. \bibverse{27} Y dijo Jehová: También he
de quitar de mi presencia á Judá, como quité á Israel, y abominaré á
esta ciudad que había escogido, á Jerusalem, y á la casa de la cual
había yo dicho: Mi nombre será allí. \bibverse{28} Lo demás de los
hechos de Josías, y todas las cosas que hizo, ¿no está todo escrito en
el libro de las crónicas de los reyes de Judá? \bibverse{29} En aquellos
días Faraón Nechâo rey de Egipto subió contra el rey de Asiria al río
Eufrates, y salió contra él el rey Josías; pero aquél, así que le vió,
matólo en Megiddo. \bibverse{30} Y sus siervos lo pusieron en un carro,
y trajéronlo muerto de Megiddo á Jerusalem, y sepultáronlo en su
sepulcro. Entonces el pueblo de la tierra tomó á Joachâz hijo de Josías,
y ungiéronle y pusiéronlo por rey en lugar de su padre. \bibverse{31} De
veintitrés años era Joachâz cuando comenzó á reinar, y reinó tres meses
en Jerusalem. El nombre de su madre fué Amutal, hija de Jeremías de
Libna. \bibverse{32} Y él hizo lo malo en ojos de Jehová, conforme á
todas las cosas que sus padres habían hecho. \bibverse{33} Y echólo
preso Faraón Nechâo en Ribla en la provincia de Hamath, reinando él en
Jerusalem; é impuso sobre la tierra una multa de cien talentos de plata,
y uno de oro. \bibverse{34} Entonces Faraón Nechâo puso por rey á
Eliacim hijo de Josías, en lugar de Josías su padre, y mudóle el nombre
en el de Joacim; y tomó á Joachâz, y llevólo á Egipto, y murió allí.
\bibverse{35} Y Joacim pagó á Faraón la plata y el oro; mas hizo
apreciar la tierra para dar el dinero conforme al mandamiento de Faraón,
sacando la plata y oro del pueblo de la tierra, de cada uno según la
estimación de su hacienda, para dar á Faraón Nechâo. \bibverse{36} De
veinticinco años era Joacim cuando comenzó á Reinar, y once años reinó
en Jerusalem. El nombre de su madre fué Zebuda hija de Pedaia, de Ruma.
\bibverse{37} E hizo lo malo en ojos de Jehová, conforme á todas las
cosas que sus padres habían hecho.

\hypertarget{section-23}{%
\section{24}\label{section-23}}

\bibverse{1} En su tiempo subió Nabucodonosor rey de Babilonia, al cual
sirvió Joacim tres años; volvióse luego, y se rebeló contra él.
\bibverse{2} Jehová empero envió contra él tropas de Caldeos, y tropas
de Siros, y tropas de Moabitas, y tropas de Ammonitas; los cuales envió
contra Judá para que la destruyesen, conforme á la palabra de Jehová que
había hablado por sus siervos los profetas. \bibverse{3} Ciertamente
vino esto contra Judá por dicho de Jehová, para quitarla de su
presencia, por los pecados de Manasés, conforme á todo lo que hizo;
\bibverse{4} Asimismo por la sangre inocente que derramó, pues hinchió á
Jerusalem de sangre inocente: Jehová por tanto, no quiso perdonar.
\bibverse{5} Lo demás de los hechos de Joacim, y todas las cosas que
hizo, ¿no está escrito en el libro de las crónicas de los reyes de Judá?
\bibverse{6} Y durmió Joacim con sus padres, y reinó en su lugar Joachîn
su hijo. \bibverse{7} Y nunca más el rey de Egipto salió de su tierra:
porque el rey de Babilonia le tomó todo lo que era suyo, desde el río de
Egipto hasta el río de Eufrates. \bibverse{8} De dieciocho años era
Joachîn cuando comenzó á reinar, y reinó en Jerusalem tres meses. El
nombre de su madre fué Neusta hija de Elnathán, de Jerusalem.
\bibverse{9} E hizo lo malo en ojos de Jehová, conforme á todas las
cosas que había hecho su padre. \bibverse{10} En aquel tiempo subieron
los siervos de Nabucodonosor rey de Babilonia contra Jerusalem, y la
ciudad fué cercada. \bibverse{11} Vino también Nabucodonosor rey de
Babilonia contra la ciudad, cuando sus siervos la tenían cercada.
\bibverse{12} Entonces salió Joachîn rey de Judá al rey de Babilonia,
él, y su madre, y sus siervos, y sus príncipes, y sus eunucos: y
prendiólo el rey de Babilonia en el octavo año de su reinado.
\bibverse{13} Y sacó de allí todos los tesoros de la casa de Jehová, y
los tesoros de la casa real, y quebró en piezas todos los vasos de oro
que había hecho Salomón rey de Israel en la casa de Jehová, como Jehová
había dicho. \bibverse{14} Y llevó en cautiverio á toda Jerusalem, á
todos los príncipes, y á todos los hombres valientes, hasta diez mil
cautivos, y á todos los oficiales y herreros; que no quedó nadie,
excepto los pobres del pueblo de la tierra. \bibverse{15} Asimismo
trasportó á Joachîn á Babilonia, y á la madre del rey, y á las mujeres
del rey, y á sus eunucos, y á los poderosos de la tierra; cautivos los
llevó de Jerusalem á Babilonia. \bibverse{16} A todos los hombre de
guerra, que fueron siete mil, y á los oficiales y herreros, que fueron
mil, y á todos los valientes para hacer la guerra, llevó cautivos el rey
de Babilonia. \bibverse{17} Y el rey de Babilonia puso por rey en lugar
de Joachîn á Mathanías su tío, y mudóle el nombre en el de Sedecías.
\bibverse{18} De veintiún años era Sedecías cuando comenzó á reinar, y
reinó en Jerusalem once años. El nombre de su madre fué Amutal hija de
Jeremías, de Libna. \bibverse{19} E hizo lo malo en ojos de Jehová,
conforme á todo lo que había hecho Joacim. \bibverse{20} Fué pues la ira
de Jehová contra Jerusalem y Judá, hasta que los echó de su presencia. Y
Sedecías se rebeló contra el rey de Babilonia.

\hypertarget{section-24}{%
\section{25}\label{section-24}}

\bibverse{1} Y aconteció á los nueve años de su reinado, en el mes
décimo, á los diez del mes, que Nabucodonosor rey de Babilonia vino con
todo su ejército contra Jerusalem, y cercóla; y levantaron contra ella
ingenios alrededor. \bibverse{2} Y estuvo la ciudad cercada hasta el
undécimo año del rey Sedecías. \bibverse{3} A los nueve del mes
prevaleció el hambre en la ciudad, que no hubo pan para el pueblo de la
tierra. \bibverse{4} Abierta ya la ciudad, huyeron de noche todos los
hombres de guerra por el camino de la puerta que estaba entre los dos
muros, junto á los huertos del rey, estando los Caldeos alrededor de la
ciudad; y el rey se fué camino de la campiña. \bibverse{5} Y el ejército
de los Caldeos siguió al rey, y tomólo en las llanuras de Jericó,
habiéndose esparcido de él todo su ejército. \bibverse{6} Tomado pues el
rey, trajéronle al rey de Babilonia á Ribla, y profirieron contra él
sentencia. \bibverse{7} Y degollaron á los hijos de Sedecías en
presencia suya; y á Sedecías sacaron los ojos, y atado con cadenas
lleváronlo á Babilonia. \bibverse{8} En el mes quinto, á los siete del
mes, siendo el año diecinueve de Nabucodonosor rey de Babilonia, vino á
Jerusalem Nabuzaradán, capitán de los de la guardia, siervo del rey de
Babilonia. \bibverse{9} Y quemó la casa de Jehová, y la casa del rey, y
todas las casas de Jerusalem; y todas las casas de los príncipes quemó á
fuego. \bibverse{10} Y todo el ejército de los Caldeos que estaba con el
capitán de la guardia, derribó los muros de Jerusalem alrededor.
\bibverse{11} Y á los del pueblo que habían quedado en la ciudad, y á
los que se habían juntado al rey de Babilonia, y á los que habían
quedado del vulgo, trasportólos Nabuzaradán, capitán de los de la
guardia. \bibverse{12} Mas de los pobres de la tierra dejó Nabuzaradán,
capitán de los de la guardia, para que labrasen las viñas y las tierras.
\bibverse{13} Y quebraron los Caldeos las columnas de bronce que estaban
en la casa de Jehová, y las basas, y el mar de bronce que estaba en la
casa de Jehová, y llevaron el metal de ello á Babilonia. \bibverse{14}
Lleváronse también los calderos, y las paletas, y las tenazas, y los
cucharones, y todos los vasos de metal con que ministraban.
\bibverse{15} Incensarios, cuencos, los que de oro, en oro, y los que de
plata, en plata, todo lo llevó el capitán de los de la guardia;
\bibverse{16} Las dos columnas, un mar, y las basas que Salomón había
hecho para la casa de Jehová: no había peso de todos estos vasos.
\bibverse{17} La altura de la una columna era diez y ocho codos y tenía
encima un capitel de bronce, y la altura del capitel era de tres codos;
y sobre el capitel había un enredado y granadas alrededor, todo de
bronce: y semejante obra había en la otra columna con el enredado.
\bibverse{18} Tomó entonces el capitán de los de la guardia á Saraías
primer sacerdote, y á Sophonías segundo sacerdote, y tres guardas de la
vajilla; \bibverse{19} Y de la ciudad tomó un eunuco, el cual era
maestre de campo, y cinco varones de los continuos del rey, que se
hallaron en la ciudad; y al principal escriba del ejército, que hacía la
reseña de la gente del país; y sesenta varones del pueblo de la tierra,
que se hallaron en la ciudad. \bibverse{20} Estos tomó Nabuzaradán,
capitán de los de la guardia, y llevólos á Ribla al rey de Babilonia.
\bibverse{21} Y el rey de Babilonia los hirió y mató en Ribla, en tierra
de Hamath. Así fué trasportado Judá de sobre su tierra. \bibverse{22} Y
al pueblo que Nabucodonosor rey de Babilonia dejó en tierra de Judá,
puso por gobernador á Gedalías, hijo de Ahicam hijo de Saphán.
\bibverse{23} Y oyendo todos los príncipes del ejército, ellos y su
gente, que el rey de Babilonia había puesto por gobernador á Gedalías,
viniéronse á él en Mizpa, es á saber, Ismael hijo de Nathanías, y
Johanán hijo de Carea, y Saraía hijo de Tanhumet Netofatita, y Jaazanías
hijo de Maachâti, ellos con los suyos. \bibverse{24} Entonces Gedalías
les hizo juramento, á ellos y á los suyos, y díjoles: No temáis de ser
siervos de los Caldeos; habitad en la tierra, y servid al rey de
Babilonia, y os irá bien. \bibverse{25} Mas en el mes séptimo vino
Ismael hijo de Nathanías, hijo de Elisama, de la estirpe real, y con él
diez varones, é hirieron á Gedalías, y murió: y también á los Judíos y
Caldeos que estaban con él en Mizpa. \bibverse{26} Y levantándose todo
el pueblo, desde el menor hasta el mayor, con los capitanes del
ejército, fuéronse á Egipto por temor de los Caldeos. \bibverse{27} Y
aconteció á los treinta y siete años de la trasportación de Joachîn rey
de Judá, en el mes duodécimo, á los veinte y siete del mes, que
Evil-merodach rey de Babilonia, en el primer año de su reinado, levantó
la cabeza de Joachîn rey de Judá, sacándolo de la casa de la cárcel;
\bibverse{28} Y hablóle bien, y puso su asiento sobre el asiento de los
reyes que con él estaban en Babilonia. \bibverse{29} Y mudóle los
vestidos de su prisión, y comió siempre delante de él todos los días de
su vida. \bibverse{30} Y fuéle diariamente dada su comida de parte del
rey de continuo, todos los días de su vida.
