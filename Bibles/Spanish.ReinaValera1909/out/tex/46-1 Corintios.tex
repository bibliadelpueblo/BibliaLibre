\hypertarget{bendiciones}{%
\subsection{Bendiciones}\label{bendiciones}}

\hypertarget{section}{%
\section{1}\label{section}}

\bibleverse{1} Pablo, llamado á ser apóstol de Jesucristo por la
voluntad de Dios, y Sóstenes el hermano, \bibleverse{2} A la iglesia de
Dios que está en Corinto, santificados en Cristo Jesús, llamados santos,
y á todos los que invocan el nombre de nuestro Señor Jesucristo en
cualquier lugar, Señor de ellos y nuestro: \bibleverse{3} Gracia y paz
de Dios nuestro Padre, y del Señor Jesucristo.

\hypertarget{acciuxf3n-de-gracias-del-apuxf3stol-por-la-rica-gracia-de-dios-que-cayuxf3-sobre-los-corintios-esperanza-segura-para-el-futuro}{%
\subsection{Acción de gracias del apóstol por la rica gracia de Dios que
cayó sobre los corintios; esperanza segura para el
futuro}\label{acciuxf3n-de-gracias-del-apuxf3stol-por-la-rica-gracia-de-dios-que-cayuxf3-sobre-los-corintios-esperanza-segura-para-el-futuro}}

\bibleverse{4} Gracias doy á mi Dios siempre por vosotros, por la gracia
de Dios que os es dada en Cristo Jesús; \bibleverse{5} Que en todas las
cosas sois enriquecidos en él, en toda lengua y en toda ciencia;
\bibleverse{6} Así como el testimonio de Cristo ha sido confirmado en
vosotros: \bibleverse{7} De tal manera que nada os falte en ningún don,
esperando la manifestación de nuestro Señor Jesucristo: \footnote{\textbf{1:7}
  Tit 2,13; 2Pe 3,13-14} \bibleverse{8} El cual también os confirmará
hasta el fin, para que seáis sin falta en el día de nuestro Señor
Jesucristo. \footnote{\textbf{1:8} Fil 1,6; 1Tes 3,13} \bibleverse{9}
Fiel es Dios, por el cual sois llamados á la participación de su Hijo
Jesucristo nuestro Señor. \footnote{\textbf{1:9} 1Tes 5,24}

\hypertarget{contiendas-en-la-iglesia}{%
\subsection{Contiendas en la iglesia}\label{contiendas-en-la-iglesia}}

\bibleverse{10} Os ruego pues, hermanos, por el nombre de nuestro Señor
Jesucristo, que habléis todos una misma cosa, y que no haya entre
vosotros disensiones, antes seáis perfectamente unidos en una misma
mente y en un mismo parecer. \footnote{\textbf{1:10} 1Cor 11,18; Rom
  15,5; Fil 2,2} \bibleverse{11} Porque me ha sido declarado de
vosotros, hermanos míos, por los que son de Cloé, que hay entre vosotros
contiendas; \bibleverse{12} Quiero decir, que cada uno de vosotros dice:
Yo cierto soy de Pablo; pues yo de Apolos; y yo de Cefas; y yo de
Cristo. \footnote{\textbf{1:12} 1Cor 3,4; Hech 18,24-27; Juan 1,42}
\bibleverse{13} ¿Está dividido Cristo? ¿Fué crucificado Pablo por
vosotros? ¿ó habéis sido bautizados en el nombre de Pablo?
\bibleverse{14} Doy gracias á Dios, que á ninguno de vosotros he
bautizado, sino á Crispo y á Gayo; \bibleverse{15} Para que ninguno diga
que habéis sido bautizados en mi nombre. \bibleverse{16} Y también
bauticé la familia de Estéfanas: mas no sé si he bautizado algún otro.
\footnote{\textbf{1:16} 1Cor 16,15} \bibleverse{17} Porque no me envió
Cristo á bautizar, sino á predicar el evangelio: no en sabiduría de
palabras, porque no sea hecha vana la cruz de Cristo.

\hypertarget{la-palabra-de-la-cruz-es-un-poder-divino-opuesto-a-la-sabiduruxeda-mundial-y-respetado-por-el-mundo-como-una-locura}{%
\subsection{La palabra de la cruz es un poder divino, opuesto a la
sabiduría mundial y respetado por el mundo como una
locura}\label{la-palabra-de-la-cruz-es-un-poder-divino-opuesto-a-la-sabiduruxeda-mundial-y-respetado-por-el-mundo-como-una-locura}}

\bibleverse{18} Porque la palabra de la cruz es locura á los que se
pierden; mas á los que se salvan, es á saber, á nosotros, es potencia de
Dios. \bibleverse{19} Porque está escrito: Destruiré la sabiduría de los
sabios, y desecharé la inteligencia de los entendidos.

\bibleverse{20} ¿Qué es del sabio? ¿qué del escriba? ¿qué del
escudriñador de este siglo? ¿no ha enloquecido Dios la sabiduría del
mundo? \footnote{\textbf{1:20} Rom 1,22; Mat 11,25} \bibleverse{21}
Porque por no haber el mundo conocido en la sabiduría de Dios á Dios por
sabiduría, agradó á Dios salvar á los creyentes por la locura de la
predicación. \bibleverse{22} Porque los Judíos piden señales, y los
Griegos buscan sabiduría: \bibleverse{23} Mas nosotros predicamos á
Cristo crucificado, á los Judíos ciertamente tropezadero, y á los
Gentiles locura; \footnote{\textbf{1:23} 1Cor 2,14; Gal 5,11; Hech 17,32}
\bibleverse{24} Empero á los llamados, así Judíos como Griegos, Cristo
potencia de Dios, y sabiduría de Dios. \footnote{\textbf{1:24} Col 2,3}
\bibleverse{25} Porque lo loco de Dios es más sabio que los hombres; y
lo flaco de Dios es más fuerte que los hombres.

\hypertarget{prueba-de-la-existencia-real-de-la-comunidad-cristiana-llamada-por-dios-en-corinto}{%
\subsection{Prueba de la existencia real de la comunidad cristiana
llamada por Dios en
Corinto}\label{prueba-de-la-existencia-real-de-la-comunidad-cristiana-llamada-por-dios-en-corinto}}

\bibleverse{26} Porque mirad, hermanos, vuestra vocación, que no sois
muchos sabios según la carne, no muchos poderosos, no muchos nobles;
\footnote{\textbf{1:26} Juan 7,48; Sant 2,1-5} \bibleverse{27} Antes lo
necio del mundo escogió Dios, para avergonzar á los sabios; y lo flaco
del mundo escogió Dios, para avergonzar lo fuerte; \bibleverse{28} Y lo
vil del mundo y lo menospreciado escogió Dios, y lo que no es, para
deshacer lo que es: \bibleverse{29} Para que ninguna carne se jacte en
su presencia. \bibleverse{30} Mas de él sois vosotros en Cristo Jesús,
el cual nos ha sido hecho por Dios sabiduría, y justificación, y
santificación, y redención: \footnote{\textbf{1:30} Jer 23,5-6; 2Cor
  5,21; Juan 17,19; Mat 20,28} \bibleverse{31} Para que, como está
escrito: El que se gloría, gloríese en el Señor. \footnote{\textbf{1:31}
  2Cor 10,17}

\hypertarget{la-manera-de-predicar-de-pablo-cuando-se-funduxf3-la-iglesia-era-poco-exigente-y-carecuxeda-de-sabiduruxeda-mundana}{%
\subsection{La manera de predicar de Pablo cuando se fundó la iglesia
era poco exigente y carecía de sabiduría
mundana}\label{la-manera-de-predicar-de-pablo-cuando-se-funduxf3-la-iglesia-era-poco-exigente-y-carecuxeda-de-sabiduruxeda-mundana}}

\hypertarget{section-1}{%
\section{2}\label{section-1}}

\bibleverse{1} Así que, hermanos, cuando fuí á vosotros, no fuí con
altivez de palabra, ó de sabiduría, á anunciaros el testimonio de
Cristo. \bibleverse{2} Porque no me propuse saber algo entre vosotros,
sino á Jesucristo, y á éste crucificado. \footnote{\textbf{2:2} Gal 6,14}
\bibleverse{3} Y estuve yo con vosotros con flaqueza, y mucho temor y
temblor; \footnote{\textbf{2:3} Hech 18,9; 2Cor 10,1; Gal 4,13}
\bibleverse{4} Y ni mi palabra ni mi predicación fué con palabras
persuasivas de humana sabiduría, mas con demostración del Espíritu y de
poder; \footnote{\textbf{2:4} Mat 10,20} \bibleverse{5} Para que vuestra
fe no esté fundada en sabiduría de hombres, mas en poder de Dios.
\footnote{\textbf{2:5} 1Tes 1,5}

\hypertarget{la-misteriosa-sabiduruxeda-de-dios-para-los-perfectos}{%
\subsection{La misteriosa sabiduría de Dios para los
perfectos}\label{la-misteriosa-sabiduruxeda-de-dios-para-los-perfectos}}

\bibleverse{6} Empero hablamos sabiduría entre perfectos; y sabiduría,
no de este siglo, ni de los príncipes de este siglo, que se deshacen:
\bibleverse{7} Mas hablamos sabiduría de Dios en misterio, la sabiduría
oculta, la cual Dios predestinó antes de los siglos para nuestra gloria:
\footnote{\textbf{2:7} Rom 16,25; Mat 11,24} \bibleverse{8} La que
ninguno de los príncipes de este siglo conoció; porque si la hubieran
conocido, nunca hubieran crucificado al Señor de gloria: \bibleverse{9}
Antes, como está escrito: Cosas que ojo no vió, ni oreja oyó, ni han
subido en corazón de hombre, son las que ha Dios preparado para aquellos
que le aman.

\hypertarget{la-exploraciuxf3n-y-absorciuxf3n-de-esta-sabiduruxeda-solo-es-posible-para-personas-espirituales}{%
\subsection{La exploración y absorción de esta sabiduría solo es posible
para personas
espirituales}\label{la-exploraciuxf3n-y-absorciuxf3n-de-esta-sabiduruxeda-solo-es-posible-para-personas-espirituales}}

\bibleverse{10} Empero Dios nos lo reveló á nosotros por el Espíritu:
porque el Espíritu todo lo escudriña, aun lo profundo de Dios.
\bibleverse{11} Porque ¿quién de los hombres sabe las cosas del hombre,
sino el espíritu del hombre que está en él? Así tampoco nadie conoció
las cosas de Dios, sino el Espíritu de Dios. \bibleverse{12} Y nosotros
hemos recibido, no el espíritu del mundo, sino el Espíritu que es de
Dios, para que conozcamos lo que Dios nos ha dado; \footnote{\textbf{2:12}
  Juan 14,16-17} \bibleverse{13} Lo cual también hablamos, no con doctas
palabras de humana sabiduría, mas con doctrina del Espíritu, acomodando
lo espiritual á lo espiritual. \bibleverse{14} Mas el hombre animal no
percibe las cosas que son del Espíritu de Dios, porque le son locura: y
no las puede entender, porque se han de examinar espiritualmente.
\bibleverse{15} Empero el espiritual juzga todas las cosas; mas él no es
juzgado de nadie. \bibleverse{16} Porque ¿quién conoció la mente del
Señor? ¿quién le instruyó? Mas nosotros tenemos la mente de Cristo.
\footnote{\textbf{2:16} Rom 11,34}

\hypertarget{hasta-ahora-pablo-no-ha-podido-proclamar-plena-sabiduruxeda-a-los-corintios-debido-a-su-inmadurez-que-ha-sido-demostrada-por-la-picarduxeda-del-partido}{%
\subsection{Hasta ahora Pablo no ha podido proclamar plena sabiduría a
los corintios debido a su inmadurez, que ha sido demostrada por la
picardía del
partido}\label{hasta-ahora-pablo-no-ha-podido-proclamar-plena-sabiduruxeda-a-los-corintios-debido-a-su-inmadurez-que-ha-sido-demostrada-por-la-picarduxeda-del-partido}}

\hypertarget{section-2}{%
\section{3}\label{section-2}}

\bibleverse{1} De manera que yo, hermanos, no pude hablaros como á
espirituales, sino como á carnales, como á niños en Cristo. \footnote{\textbf{3:1}
  Juan 16,12} \bibleverse{2} Os dí á beber leche, y no vianda: porque
aun no podíais, ni aun podéis ahora; \footnote{\textbf{3:2} 1Pe 2,2}
\bibleverse{3} Porque todavía sois carnales: pues habiendo entre
vosotros celos, y contiendas, y disensiones, ¿no sois carnales, y andáis
como hombres? \footnote{\textbf{3:3} 1Cor 1,10-11; 1Cor 11,18}
\bibleverse{4} Porque diciendo el uno: Yo cierto soy de Pablo; y el
otro: Yo de Apolos; ¿no sois carnales? \footnote{\textbf{3:4} 1Cor 1,12}

\hypertarget{son-siervos-y-colaboradores-de-dios}{%
\subsection{Son siervos y colaboradores de
Dios}\label{son-siervos-y-colaboradores-de-dios}}

\bibleverse{5} ¿Qué pues es Pablo? ¿y qué es Apolos? Ministros por los
cuales habéis creído; y eso según que á cada uno ha concedido el Señor.
\bibleverse{6} Yo planté, Apolos regó: mas Dios ha dado el crecimiento.
\bibleverse{7} Así que, ni el que planta es algo, ni el que riega; sino
Dios, que da el crecimiento. \bibleverse{8} Y el que planta y el que
riega son una misma cosa; aunque cada uno recibirá su recompensa
conforme á su labor. \footnote{\textbf{3:8} 1Cor 4,5} \bibleverse{9}
Porque nosotros, coadjutores somos de Dios; y vosotros labranza de Dios
sois, edificio de Dios sois. \footnote{\textbf{3:9} Mat 13,3-9; Efes
  2,20}

\hypertarget{cada-maestro-procura-que-su-obra-consista-en-el-fuego-del-juicio-divino-de-un-duxeda}{%
\subsection{¡Cada maestro procura que su obra consista en el fuego del
juicio divino de un
día!}\label{cada-maestro-procura-que-su-obra-consista-en-el-fuego-del-juicio-divino-de-un-duxeda}}

\bibleverse{10} Conforme á la gracia de Dios que me ha sido dada, yo
como perito arquitecto puse el fundamento, y otro edifica encima: empero
cada uno vea cómo sobreedifica. \bibleverse{11} Porque nadie puede poner
otro fundamento que el que está puesto, el cual es Jesucristo.
\footnote{\textbf{3:11} 1Pe 2,4-6} \bibleverse{12} Y si alguno edificare
sobre este fundamento oro, plata, piedras preciosas, madera, heno,
hojarasca; \bibleverse{13} La obra de cada uno será manifestada: porque
el día la declarará; porque por el fuego será manifestada; y la obra de
cada uno cuál sea, el fuego hará la prueba. \bibleverse{14} Si
permaneciere la obra de alguno que sobreedificó, recibirá recompensa.
\bibleverse{15} Si la obra de alguno fuere quemada, será perdida: él
empero será salvo, mas así como por fuego.

\bibleverse{16} ¿No sabéis que sois templo de Dios, y que el Espíritu de
Dios mora en vosotros? \footnote{\textbf{3:16} 1Cor 6,19; 2Cor 6,16}
\bibleverse{17} Si alguno violare el templo de Dios, Dios destruirá al
tal: porque el templo de Dios, el cual sois vosotros, santo es.

\bibleverse{18} Nadie se engañe á sí mismo: si alguno entre vosotros
parece ser sabio en este siglo, hágase simple, para ser sabio.
\bibleverse{19} Porque la sabiduría de este mundo es necedad para con
Dios; pues escrito está: El que prende á los sabios en la astucia de
ellos. \bibleverse{20} Y otra vez: El Señor conoce los pensamientos de
los sabios, que son vanos. \bibleverse{21} Así que, ninguno se gloríe en
los hombres; porque todo es vuestro; \bibleverse{22} Sea Pablo, sea
Apolos, sea Cefas, sea el mundo, sea la vida, sea la muerte, sea lo
presente, sea los por venir; todo es vuestro; \bibleverse{23} Y vosotros
de Cristo; y Cristo de Dios. \footnote{\textbf{3:23} 1Cor 11,3}

\hypertarget{pablo-sabe-que-es-responsable-solo-ante-el-seuxf1or}{%
\subsection{Pablo sabe que es responsable solo ante el
Señor}\label{pablo-sabe-que-es-responsable-solo-ante-el-seuxf1or}}

\hypertarget{section-3}{%
\section{4}\label{section-3}}

\bibleverse{1} Téngannos los hombres por ministros de Cristo, y
dispensadores de los misterios de Dios. \footnote{\textbf{4:1} Tit 1,7}
\bibleverse{2} Mas ahora se requiere en los dispensadores, que cada uno
sea hallado fiel. \footnote{\textbf{4:2} Luc 12,42} \bibleverse{3} Yo en
muy poco tengo el ser juzgado de vosotros, ó de juicio humano; y ni aun
yo me juzgo. \bibleverse{4} Porque aunque de nada tengo mala conciencia,
no por eso soy justificado; mas el que me juzga, el Señor es.
\bibleverse{5} Así que, no juzguéis nada antes de tiempo, hasta que
venga el Señor, el cual también aclarará lo oculto de las tinieblas, y
manifestará los intentos de los corazones: y entonces cada uno tendrá de
Dios la alabanza.

\hypertarget{pablo-reprocha-a-los-corintios-su-arrogancia-hacia-el-sufrimiento-de-los-apuxf3stoles}{%
\subsection{Pablo reprocha a los corintios su arrogancia hacia el
sufrimiento de los
apóstoles}\label{pablo-reprocha-a-los-corintios-su-arrogancia-hacia-el-sufrimiento-de-los-apuxf3stoles}}

\bibleverse{6} Esto empero, hermanos, he pasado por ejemplo en mí y en
Apolos por amor de vosotros; para que en nosotros aprendáis á no saber
más de lo que está escrito, hinchándoos por causa de otro el uno contra
el otro. \footnote{\textbf{4:6} Rom 12,3} \bibleverse{7} Porque ¿quién
te distingue? ¿ó qué tienes que no hayas recibido? Y si lo recibiste,
¿de qué te glorías como si no hubieras recibido?

\bibleverse{8} Ya estáis hartos, ya estáis ricos, sin nosotros reináis;
y ojalá reinéis, para que nosotros reinemos también juntamente con
vosotros. \bibleverse{9} Porque á lo que pienso, Dios nos ha mostrado á
nosotros los apóstoles por los postreros, como á sentenciados á muerte:
porque somos hechos espectáculo al mundo, y á los ángeles, y á los
hombres. \footnote{\textbf{4:9} Rom 8,36; Heb 10,33} \bibleverse{10}
Nosotros necios por amor de Cristo, y vosotros prudentes en Cristo;
nosotros flacos, y vosotros fuertes; vosotros nobles, y nosotros viles.
\footnote{\textbf{4:10} 1Cor 3,18} \bibleverse{11} Hasta esta hora
hambreamos, y tenemos sed, y estamos desnudos, y somos heridos de
golpes, y andamos vagabundos; \footnote{\textbf{4:11} 2Cor 11,23-27}
\bibleverse{12} Y trabajamos, obrando con nuestras manos: nos maldicen,
y bendecimos: padecemos persecución, y sufrimos: \footnote{\textbf{4:12}
  1Cor 9,15; Hech 18,3; Mat 5,44; Rom 12,14} \bibleverse{13} Somos
blasfemados, y rogamos: hemos venido á ser como la hez del mundo, el
desecho de todos hasta ahora.

\hypertarget{la-referencia-de-pablo-a-su-relaciuxf3n-personal-con-la-iglesia}{%
\subsection{La referencia de Pablo a su relación personal con la
iglesia}\label{la-referencia-de-pablo-a-su-relaciuxf3n-personal-con-la-iglesia}}

\bibleverse{14} No escribo esto para avergonzaros: mas amonéstoos como á
mis hijos amados. \bibleverse{15} Porque aunque tengáis diez mil ayos en
Cristo, no tendréis muchos padres; que en Cristo Jesús yo os engendré
por el evangelio. \footnote{\textbf{4:15} 1Cor 9,2; Gal 4,19}
\bibleverse{16} Por tanto, os ruego que me imitéis. \footnote{\textbf{4:16}
  1Cor 11,1} \bibleverse{17} Por lo cual os he enviado á Timoteo, que es
mi hijo amado y fiel en el Señor, el cual os amonestará de mis caminos
cuáles sean en Cristo, de la manera que enseño en todas partes en todas
las iglesias. \footnote{\textbf{4:17} Hech 16,1-3} \bibleverse{18} Mas
algunos están envanecidos, como si nunca hubiese yo de ir á vosotros.
\bibleverse{19} Empero iré presto á vosotros, si el Señor quisiere; y
entenderé, no las palabras de los que andan hinchados, sino la virtud.
\bibleverse{20} Porque el reino de Dios no consiste en palabras, sino en
virtud. \bibleverse{21} ¿Qué queréis? ¿iré á vosotros con vara, ó con
caridad y espíritu de mansedumbre? \footnote{\textbf{4:21} 2Cor 10,2}

\hypertarget{grave-reprimenda-por-la-tolerancia-mostrada-por-la-comunidad-a-un-fornicario}{%
\subsection{Grave reprimenda por la tolerancia mostrada por la comunidad
a un
fornicario}\label{grave-reprimenda-por-la-tolerancia-mostrada-por-la-comunidad-a-un-fornicario}}

\hypertarget{section-4}{%
\section{5}\label{section-4}}

\bibleverse{1} De cierto se oye que hay entre vosotros fornicación, y
tal fornicación cual ni aun se nombra entre los Gentiles; tanto que
alguno tenga la mujer de su padre. \bibleverse{2} Y vosotros estáis
hinchados, y no más bien tuvisteis duelo, para que fuese quitado de en
medio de vosotros el que hizo tal obra. \bibleverse{3} Y ciertamente,
como ausente con el cuerpo, mas presente en espíritu, ya como presente
he juzgado al que esto así ha cometido: \footnote{\textbf{5:3} Col 2,5}
\bibleverse{4} En el nombre del Señor nuestro Jesucristo, juntados
vosotros y mi espíritu, con la facultad de nuestro Señor Jesucristo,
\footnote{\textbf{5:4} Mat 16,19; Mat 18,18; 2Cor 13,10} \bibleverse{5}
El tal sea entregado á Satanás para muerte de la carne, porque el
espíritu sea salvo en el día del Señor Jesús. \footnote{\textbf{5:5}
  1Tim 1,20}

\hypertarget{amonestaciuxf3n-general-a-la-pureza-moral-con-referencia-a-la-muerte-en-sacrificio-de-jesuxfas-el-cordero-pascual}{%
\subsection{Amonestación general a la pureza moral con referencia a la
muerte en sacrificio de Jesús, ``el cordero
pascual''}\label{amonestaciuxf3n-general-a-la-pureza-moral-con-referencia-a-la-muerte-en-sacrificio-de-jesuxfas-el-cordero-pascual}}

\bibleverse{6} No es buena vuestra jactancia. ¿No sabéis que un poco de
levadura leuda toda la masa? \footnote{\textbf{5:6} Gal 5,9}
\bibleverse{7} Limpiad pues la vieja levadura, para que seáis nueva
masa, como sois sin levadura: porque nuestra pascua, que es Cristo, fué
sacrificada por nosotros. \footnote{\textbf{5:7} Éxod 12,3-20; Éxod
  13,7; Is 53,7; 1Pe 1,19} \bibleverse{8} Así que hagamos fiesta, no en
la vieja levadura, ni en la levadura de malicia y de maldad, sino en
ázimos de sinceridad y de verdad.

\hypertarget{correcciuxf3n-de-un-malentendido-corintio-sobre-la-advertencia-contra-los-fornicadores}{%
\subsection{Corrección de un malentendido corintio sobre la advertencia
contra los
fornicadores}\label{correcciuxf3n-de-un-malentendido-corintio-sobre-la-advertencia-contra-los-fornicadores}}

\bibleverse{9} Os he escrito por carta, que no os envolváis con los
fornicarios: \bibleverse{10} No absolutamente con los fornicarios de
este mundo, ó con los avaros, ó con los ladrones, ó con los idólatras;
pues en tal caso os sería menester salir del mundo. \bibleverse{11} Mas
ahora os he escrito, que no os envolváis, es á saber, que si alguno
llamándose hermano fuere fornicario, ó avaro, ó idólatra, ó maldiciente,
ó borracho, ó ladrón, con el tal ni aun comáis. \bibleverse{12} Porque
¿qué me va á mí en juzgar á los que están fuera? ¿No juzgáis vosotros á
los que están dentro? \bibleverse{13} Porque á los que están fuera, Dios
juzgará: quitad pues á ese malo de entre vosotros. \footnote{\textbf{5:13}
  Deut 13,6; Mat 18,17}

\hypertarget{denuncia-de-litigio-en-tribunales-paganos-y-litigio-en-general}{%
\subsection{Denuncia de litigio en tribunales paganos y litigio en
general}\label{denuncia-de-litigio-en-tribunales-paganos-y-litigio-en-general}}

\hypertarget{section-5}{%
\section{6}\label{section-5}}

\bibleverse{1} ¿Osa alguno de vosotros, teniendo algo con otro, ir á
juicio delante de los injustos, y no delante de los santos?
\bibleverse{2} ¿O no sabéis que los santos han de juzgar al mundo? Y si
el mundo ha de ser juzgado por vosotros, ¿sois indignos de juzgar cosas
muy pequeñas? \bibleverse{3} ¿O no sabéis que hemos de juzgar á los
ángeles? ¿cuánto más las cosas de este siglo? \bibleverse{4} Por tanto,
si hubiereis de tener juicios de cosas de este siglo, poned para juzgar
á los que son de menor estima en la iglesia. \bibleverse{5} Para
avergonzaros lo digo. ¿Pues qué, no hay entre vosotros sabio, ni aun uno
que pueda juzgar entre sus hermanos; \bibleverse{6} Sino que el hermano
con el hermano pleitea en juicio, y esto ante los infieles?
\bibleverse{7} Así que, por cierto es ya una falta en vosotros que
tengáis pleitos entre vosotros mismos. ¿Por qué no sufrís antes la
injuria? ¿por qué no sufrís antes ser defraudados? \footnote{\textbf{6:7}
  Mat 5,38-41; 1Tes 5,15; 1Pe 3,9} \bibleverse{8} Empero vosotros hacéis
la injuria, y defraudáis, y esto á los hermanos.

\bibleverse{9} ¿No sabéis que los injustos no poseerán el reino de Dios?
No erréis, que ni los fornicarios, ni los idólatras, ni los adúlteros,
ni los afeminados, ni los que se echan con varones, \bibleverse{10} Ni
los ladrones, ni los avaros, ni los borrachos, ni los maldicientes, ni
los robadores, heredarán el reino de Dios. \bibleverse{11} Y esto erais
algunos: mas ya sois lavados, mas ya sois santificados, mas ya sois
justificados en el nombre del Señor Jesús, y por el Espíritu de nuestro
Dios. \footnote{\textbf{6:11} 1Cor 1,2; Rom 3,26; Tit 3,3-7}

\hypertarget{los-pecados-de-fornicaciuxf3n-no-tienen-nada-que-ver-con-la-libertad-cristiana-advertencia-de-fornicaciuxf3n}{%
\subsection{Los pecados de fornicación no tienen nada que ver con la
libertad cristiana; Advertencia de
fornicación}\label{los-pecados-de-fornicaciuxf3n-no-tienen-nada-que-ver-con-la-libertad-cristiana-advertencia-de-fornicaciuxf3n}}

\bibleverse{12} Todas las cosas me son lícitas, mas no todas convienen:
todas las cosas me son lícitas, mas yo no me meteré debajo de potestad
de nada. \footnote{\textbf{6:12} 1Cor 10,23} \bibleverse{13} Las viandas
para el vientre, y el vientre para las viandas; empero y á él y á ellas
deshará Dios. Mas el cuerpo no es para la fornicación, sino para el
Señor; y el Señor para el cuerpo: \footnote{\textbf{6:13} 1Cor 1,3-5}
\bibleverse{14} Y Dios que levantó al Señor, también á nosotros nos
levantará con su poder. \footnote{\textbf{6:14} 1Cor 15,20; 2Cor 4,14}
\bibleverse{15} ¿No sabéis que vuestros cuerpos son miembros de Cristo?
¿Quitaré pues los miembros de Cristo, y los haré miembros de una ramera?
Lejos sea. \bibleverse{16} ¿O no sabéis que el que se junta con una
ramera, es hecho con ella un cuerpo? porque serán, dice, los dos en una
carne. \bibleverse{17} Empero el que se junta con el Señor, un espíritu
es. \footnote{\textbf{6:17} Juan 17,21-22} \bibleverse{18} Huid la
fornicación. Cualquier otro pecado que el hombre hiciere, fuera del
cuerpo es; mas el que fornica, contra su propio cuerpo peca.
\bibleverse{19} ¿O ignoráis que vuestro cuerpo es templo del Espíritu
Santo, el cual está en vosotros, el cual tenéis de Dios, y que no sois
vuestros? \bibleverse{20} Porque comprados sois por precio: glorificad
pues á Dios en vuestro cuerpo y en vuestro espíritu, los cuales son de
Dios. \footnote{\textbf{6:20} 1Cor 7,23; 1Pe 1,18-19; Fil 1,20}

\hypertarget{el-valor-y-las-necesidades-del-matrimonio-y-la-vida-conyugal-en-general}{%
\subsection{El valor y las necesidades del matrimonio y la vida conyugal
en
general}\label{el-valor-y-las-necesidades-del-matrimonio-y-la-vida-conyugal-en-general}}

\hypertarget{section-6}{%
\section{7}\label{section-6}}

\bibleverse{1} Cuanto á las cosas de que me escribisteis, bien es al
hombre no tocar mujer. \bibleverse{2} Mas á causa de las fornicaciones,
cada uno tenga su mujer, y cada una tenga su marido. \bibleverse{3} El
marido pague á la mujer la debida benevolencia; y asimismo la mujer al
marido. \bibleverse{4} La mujer no tiene potestad de su propio cuerpo,
sino el marido: é igualmente tampoco el marido tiene potestad de su
propio cuerpo, sino la mujer. \bibleverse{5} No os defraudéis el uno al
otro, á no ser por algún tiempo de mutuo consentimiento, para ocuparos
en la oración: y volved á juntaros en uno, porque no os tiente Satanás á
causa de vuestra incontinencia.

\bibleverse{6} Mas esto digo por permisión, no por mandamiento.
\bibleverse{7} Quisiera más bien que todos los hombres fuesen como yo:
empero cada uno tiene su propio don de Dios; uno á la verdad así, y otro
así.

\hypertarget{sobre-el-comportamiento-de-las-personas-solteras-y-sobre-el-divorcio-en-los-matrimonios-cristianos}{%
\subsection{Sobre el comportamiento de las personas solteras y sobre el
divorcio en los matrimonios
cristianos}\label{sobre-el-comportamiento-de-las-personas-solteras-y-sobre-el-divorcio-en-los-matrimonios-cristianos}}

\bibleverse{8} Digo pues á los solteros y á las viudas, que bueno les es
si se quedaren como yo. \bibleverse{9} Y si no tienen don de
continencia, cásense; que mejor es casarse que quemarse. \footnote{\textbf{7:9}
  1Tim 5,14} \bibleverse{10} Mas á los que están juntos en matrimonio,
denuncio, no yo, sino el Señor: Que la mujer no se aparte del marido;
\footnote{\textbf{7:10} Mat 5,32} \bibleverse{11} Y si se apartare, que
se quede sin casar, ó reconcíliese con su marido; y que el marido no
despida á su mujer.

\hypertarget{comportamiento-en-el-matrimonio-mixto}{%
\subsection{Comportamiento en el matrimonio
mixto}\label{comportamiento-en-el-matrimonio-mixto}}

\bibleverse{12} Y á los demás yo digo, no el Señor: Si algún hermano
tiene mujer infiel, y ella consiente en habitar con él, no la despida.
\bibleverse{13} Y la mujer que tiene marido infiel, y él consiente en
habitar con ella, no lo deje. \bibleverse{14} Porque el marido infiel es
santificado en la mujer, y la mujer infiel en el marido: pues de otra
manera vuestros hijos serían inmundos; empero ahora son santos.
\footnote{\textbf{7:14} Rom 11,16} \bibleverse{15} Pero si el infiel se
aparta, apártese: que no es el hermano ó la hermana sujeto á servidumbre
en semejante caso; antes á paz nos llamó Dios. \footnote{\textbf{7:15}
  Rom 14,19} \bibleverse{16} Porque ¿de dónde sabes, oh mujer, si quizá
harás salvo á tu marido? ¿ó de dónde sabes, oh marido, si quizá harás
salva á tu mujer? \footnote{\textbf{7:16} 1Pe 3,1}

\hypertarget{regla-general-sobre-la-posiciuxf3n-del-cristiano-a-las-condiciones-externas-existentes-todo-creyente-permanece-en-la-posiciuxf3n-en-la-que-fue-llamado}{%
\subsection{Regla general sobre la posición del cristiano a las
condiciones externas existentes: ¡Todo creyente permanece en la posición
en la que fue
llamado!}\label{regla-general-sobre-la-posiciuxf3n-del-cristiano-a-las-condiciones-externas-existentes-todo-creyente-permanece-en-la-posiciuxf3n-en-la-que-fue-llamado}}

\bibleverse{17} Empero cada uno como el Señor le repartió, y como Dios
llamó á cada uno, así ande: y así enseño en todas las iglesias.

\bibleverse{18} ¿Es llamado alguno circuncidado? quédese circunciso. ¿Es
llamado alguno incircuncidado? que no se circuncide. \bibleverse{19} La
circuncisión nada es, y la incircuncisión nada es; sino la observancia
de las mandamientos de Dios. \bibleverse{20} Cada uno en la vocación en
que fué llamado, en ella se quede. \bibleverse{21} ¿Eres llamado siendo
siervo? no se te dé cuidado: mas también si puedes hacerte libre,
procúralo más. \bibleverse{22} Porque el que en el Señor es llamado
siendo siervo, liberto es del Señor: asimismo también el que es llamado
siendo libre, siervo es de Cristo. \footnote{\textbf{7:22} Efes 6,6;
  Filem 1,16} \bibleverse{23} Por precio sois comprados; no os hagáis
siervos de los hombres. \footnote{\textbf{7:23} 1Cor 6,20}
\bibleverse{24} Cada uno, hermanos, en lo que es llamado, en esto se
quede para con Dios.

\hypertarget{sobre-el-celibato-de-ambos-sexos-consejos-para-casarse-con-mujeres-solteras-y-volver-a-casarse-con-viudas}{%
\subsection{Sobre el celibato de ambos sexos; Consejos para casarse con
mujeres solteras y volver a casarse con
viudas}\label{sobre-el-celibato-de-ambos-sexos-consejos-para-casarse-con-mujeres-solteras-y-volver-a-casarse-con-viudas}}

\bibleverse{25} Empero de las vírgenes no tengo mandamiento del Señor;
mas doy mi parecer, como quien ha alcanzado misericordia del Señor para
ser fiel. \bibleverse{26} Tengo, pues, esto por bueno á causa de la
necesidad que apremia, que bueno es al hombre estarse así. \footnote{\textbf{7:26}
  1Cor 10,11} \bibleverse{27} ¿Estás ligado á mujer? no procures
soltarte. ¿Estás suelto de mujer? no procures mujer. \bibleverse{28} Mas
también si tomares mujer, no pecaste; y si la doncella se casare, no
pecó: pero aflicción de carne tendrán los tales: mas yo os dejo.
\bibleverse{29} Esto empero digo, hermanos, que el tiempo es corto: lo
que resta es, que los que tienen mujeres sean como los que no las
tienen; \bibleverse{30} Y los que lloran, como los que no lloran; y los
que se huelgan, como los que no se huelgan; y los que compran, como los
que no poseen; \bibleverse{31} Y los que usan de este mundo, como los
que no usan: porque la apariencia de este mundo se pasa. \footnote{\textbf{7:31}
  1Jn 2,15-17}

\bibleverse{32} Quisiera, pues, que estuvieseis sin congoja. El soltero
tiene cuidado de las cosas que son del Señor, cómo ha de agradar al
Señor: \bibleverse{33} Empero el que se casó tiene cuidado de las cosas
que son del mundo, cómo ha de agradar á su mujer. \bibleverse{34} Hay
asimismo diferencia entre la casada y la doncella: la doncella tiene
cuidado de las cosas del Señor, para ser santa así en el cuerpo como en
el espíritu: mas la casada tiene cuidado de las cosas del mundo, cómo ha
de agradar á su marido. \bibleverse{35} Esto empero digo para vuestro
provecho; no para echaros lazo, sino para lo honesto y decente, y para
que sin impedimento os lleguéis al Señor.

\bibleverse{36} Mas, si á alguno parece cosa fea en su hija virgen, que
pase ya de edad, y que así conviene que se haga, haga lo que quisiere,
no peca; cásese. \bibleverse{37} Pero el que está firme en su corazón, y
no tiene necesidad, sino que tiene libertad de su voluntad, y determinó
en su corazón esto, el guardar su hija virgen, bien hace.
\bibleverse{38} Así que, el que la da en casamiento, bien hace; y el que
no la da en casamiento, hace mejor.

\hypertarget{sobre-el-nuevo-matrimonio-de-las-viudas}{%
\subsection{Sobre el nuevo matrimonio de las
viudas}\label{sobre-el-nuevo-matrimonio-de-las-viudas}}

\bibleverse{39} La mujer casada está atada á la ley, mientras vive su
marido; mas si su marido muriere, libre es: cásese con quien quisiere,
con tal que sea en el Señor. \footnote{\textbf{7:39} Rom 7,2}
\bibleverse{40} Empero más venturosa será si se quedare así, según mi
consejo; y pienso que también yo tengo Espíritu de Dios.

\hypertarget{el-conocimiento-en-suxed-mismo-tiene-menos-valor-que-el-amor}{%
\subsection{El conocimiento en sí mismo tiene menos valor que el
amor}\label{el-conocimiento-en-suxed-mismo-tiene-menos-valor-que-el-amor}}

\hypertarget{section-7}{%
\section{8}\label{section-7}}

\bibleverse{1} Y por lo que hace á lo sacrificado á los ídolos, sabemos
que todos tenemos ciencia. La ciencia hincha, mas la caridad edifica.
\bibleverse{2} Y si alguno se imagina que sabe algo, aun no sabe nada
como debe saber. \footnote{\textbf{8:2} Gal 6,3} \bibleverse{3} Mas si
alguno ama á Dios, el tal es conocido de él. \footnote{\textbf{8:3} Gal
  4,9; 1Cor 13,12}

\hypertarget{no-todo-el-mundo-tiene-un-conocimiento-perfecto}{%
\subsection{No todo el mundo tiene un conocimiento
perfecto}\label{no-todo-el-mundo-tiene-un-conocimiento-perfecto}}

\bibleverse{4} Acerca, pues, de las viandas que son sacrificadas á los
ídolos, sabemos que el ídolo nada es en el mundo, y que no hay más de un
Dios. \footnote{\textbf{8:4} Deut 6,4} \bibleverse{5} Porque aunque haya
algunos que se llamen dioses, ó en el cielo, ó en la tierra (como hay
muchos dioses y muchos señores), \footnote{\textbf{8:5} 1Cor 10,19-20;
  Sal 136,2-3; Rom 8,38-39} \bibleverse{6} Nosotros empero no tenemos
más de un Dios, el Padre, del cual son todas las cosas, y nosotros en
él: y un Señor Jesucristo, por el cual son todas las cosas, y nosotros
por él. \footnote{\textbf{8:6} 1Cor 12,5-6; Col 1,16; Efes 4,5-6; Mal
  2,10; Juan 1,3}

\bibleverse{7} Mas no en todos hay esta ciencia: porque algunos con
conciencia del ídolo hasta aquí, comen como sacrificado á ídolos; y su
conciencia, siendo flaca, es contaminada. \footnote{\textbf{8:7} 1Cor
  10,28}

\hypertarget{para-el-uso-de-la-libertad-cristiana-la-consideraciuxf3n-amorosa-por-los-duxe9biles-es-decisiva}{%
\subsection{Para el uso de la libertad cristiana, la consideración
amorosa por los débiles es
decisiva}\label{para-el-uso-de-la-libertad-cristiana-la-consideraciuxf3n-amorosa-por-los-duxe9biles-es-decisiva}}

\bibleverse{8} Si bien la vianda no nos hace más aceptos á Dios: porque
ni que comamos, seremos más ricos; ni que no comamos, seremos más
pobres. \footnote{\textbf{8:8} Rom 14,17} \bibleverse{9} Mas mirad que
esta vuestra libertad no sea tropezadero á los que son flacos.
\footnote{\textbf{8:9} Gal 5,13} \bibleverse{10} Porque si te ve alguno,
á ti que tienes ciencia, que estás sentado á la mesa en el lugar de los
ídolos, ¿la conciencia de aquel que es flaco, no será adelantada á comer
de lo sacrificado á los ídolos? \bibleverse{11} Y por tu ciencia se
perderá el hermano flaco por el cual Cristo murió. \footnote{\textbf{8:11}
  Rom 14,15} \bibleverse{12} De esta manera, pues, pecando contra los
hermanos, é hiriendo su flaca conciencia, contra Cristo pecáis.
\bibleverse{13} Por lo cual, si la comida es á mi hermano ocasión de
caer, jamás comeré carne por no escandalizar á mi hermano.

\hypertarget{explicaciuxf3n-y-justificaciuxf3n-de-los-derechos-debidos-a-pablo-como-apuxf3stol}{%
\subsection{Explicación y justificación de los derechos debidos a Pablo
como
apóstol}\label{explicaciuxf3n-y-justificaciuxf3n-de-los-derechos-debidos-a-pablo-como-apuxf3stol}}

\hypertarget{section-8}{%
\section{9}\label{section-8}}

\bibleverse{1} ¿No soy apóstol? ¿no soy libre? ¿no he visto á Jesús el
Señor nuestro? ¿no sois vosotros mi obra en el Señor? \footnote{\textbf{9:1}
  1Cor 15,8; Hech 9,3-5; Hech 9,15} \bibleverse{2} Si á los otros no soy
apóstol, á vosotros ciertamente lo soy: porque el sello de mi apostolado
sois vosotros en el Señor. \footnote{\textbf{9:2} 1Cor 4,15; 2Cor 3,2-3}

\bibleverse{3} Esta es mi respuesta á los que me preguntan.
\bibleverse{4} Qué, ¿no tenemos potestad de comer y de beber?
\footnote{\textbf{9:4} Luc 10,8} \bibleverse{5} ¿No tenemos potestad de
traer con nosotros una hermana mujer también como los otros apóstoles, y
los hermanos del Señor, y Cefas? \footnote{\textbf{9:5} Juan 1,42; Mat
  8,14} \bibleverse{6} ¿O sólo yo y Bernabé no tenemos potestad de no
trabajar? \footnote{\textbf{9:6} Hech 4,36; 2Tes 3,7-9} \bibleverse{7}
¿Quién jamás peleó á sus expensas? ¿quién planta viña, y no come de su
fruto? ¿ó quién apacienta el ganado, y no come de la leche del ganado?

\bibleverse{8} ¿Digo esto según los hombres? ¿no dice esto también la
ley? \bibleverse{9} Porque en la ley de Moisés está escrito: No pondrás
bozal al buey que trilla. ¿Tiene Dios cuidado de los bueyes?
\bibleverse{10} ¿O dícelo enteramente por nosotros? Pues por nosotros
está escrito; porque con esperanza ha de arar el que ara; y el que
trilla, con esperanza de recibir el fruto. \bibleverse{11} Si nosotros
os sembramos lo espiritual, ¿es gran cosa si segáremos lo vuestro
carnal? \footnote{\textbf{9:11} Rom 15,27} \bibleverse{12} Si otros
tienen en vosotros esta potestad, ¿no más bien nosotros? Mas no hemos
usado de esta potestad: antes lo sufrimos todo, por no poner ningún
obstáculo al evangelio de Cristo. \footnote{\textbf{9:12} Hech 20,33-35;
  2Cor 11,9}

\hypertarget{explique-las-razones-por-las-que-pablo-renuncia-a-sus-derechos}{%
\subsection{Explique las razones por las que Pablo renuncia a sus
derechos}\label{explique-las-razones-por-las-que-pablo-renuncia-a-sus-derechos}}

\bibleverse{13} ¿No sabéis que los que trabajan en el santuario, comen
del santuario; y que los que sirven al altar, del altar participan?
\footnote{\textbf{9:13} Núm 18,18-19; Núm 18,31; Deut 18,1-3}
\bibleverse{14} Así también ordenó el Señor á los que anuncian el
evangelio, que vivan del evangelio. \footnote{\textbf{9:14} Gal 6,6; Luc
  10,7}

\bibleverse{15} Mas yo de nada de esto me aproveché: ni tampoco he
escrito esto para que se haga así conmigo; porque tengo por mejor morir,
antes que nadie haga vana esta mi gloria. \footnote{\textbf{9:15} Hech
  18,3} \bibleverse{16} Pues bien que anuncio el evangelio, no tengo por
qué gloriarme; porque me es impuesta necesidad; y ¡ay de mí si no
anunciare el evangelio! \footnote{\textbf{9:16} Jer 20,9}
\bibleverse{17} Por lo cual, si lo hago de voluntad, premio tendré; mas
si por fuerza, la dispensación me ha sido encargada. \footnote{\textbf{9:17}
  1Cor 4,1} \bibleverse{18} ¿Cuál, pues, es mi merced? Que predicando el
evangelio, ponga el evangelio de Cristo de balde, para no usar mal de mi
potestad en el evangelio.

\hypertarget{pablo-aunque-exteriormente-es-completamente-libre-es-sin-embargo-un-servidor-de-todos-los-hombres}{%
\subsection{Pablo, aunque exteriormente es completamente libre, es sin
embargo un servidor de todos los
hombres}\label{pablo-aunque-exteriormente-es-completamente-libre-es-sin-embargo-un-servidor-de-todos-los-hombres}}

\bibleverse{19} Por lo cual, siendo libre para con todos, me he hecho
siervo de todos por ganar á más. \bibleverse{20} Heme hecho á los Judíos
como Judío, por ganar á los Judíos; á los que están sujetos á la ley
(aunque yo no sea sujeto á la ley) como sujeto á la ley, por ganar á los
que están sujetos á la ley; \footnote{\textbf{9:20} 1Cor 10,33; Hech
  16,3; Hech 21,20-26} \bibleverse{21} A los que son sin ley, como si yo
fuera sin ley, (no estando yo sin ley de Dios, mas en la ley de Cristo)
por ganar á los que estaban sin ley. \footnote{\textbf{9:21} Gal 2,3}
\bibleverse{22} Me he hecho á los flacos flaco, por ganar á los flacos:
á todos me he hecho todo, para que de todo punto salve á algunos.
\footnote{\textbf{9:22} Rom 11,14} \bibleverse{23} Y esto hago por causa
del evangelio, por hacerme juntamente participante de él.

\hypertarget{el-apuxf3stol-como-competidor-por-el-premio-celestial}{%
\subsection{El apóstol como competidor por el premio
celestial}\label{el-apuxf3stol-como-competidor-por-el-premio-celestial}}

\bibleverse{24} ¿No sabéis que los que corren en el estadio, todos á la
verdad corren, mas uno lleva el premio? Corred de tal manera que lo
obtengáis. \bibleverse{25} Y todo aquel que lucha, de todo se abstiene:
y ellos, á la verdad, para recibir una corona corruptible; mas nosotros,
incorruptible. \footnote{\textbf{9:25} 2Tim 2,4-5; 1Pe 5,4}
\bibleverse{26} Así que, yo de esta manera corro, no como á cosa
incierta; de esta manera peleo, no como quien hiere el aire:
\bibleverse{27} Antes hiero mi cuerpo, y lo pongo en servidumbre; no sea
que, habiendo predicado á otros, yo mismo venga á ser reprobado.

\hypertarget{das-durch-guxf6ttliche-gnadenerweise-in-der-wuxfcste-gesegnete-und-zur-rettung-ins-heilige-land-berufene-israel}{%
\subsection{Das durch göttliche Gnadenerweise in der Wüste gesegnete und
zur Rettung ins heilige Land berufene
Israel}\label{das-durch-guxf6ttliche-gnadenerweise-in-der-wuxfcste-gesegnete-und-zur-rettung-ins-heilige-land-berufene-israel}}

\hypertarget{section-9}{%
\section{10}\label{section-9}}

\bibleverse{1} Porque no quiero, hermanos, que ignoréis que nuestros
padres todos estuvieron bajo la nube, y todos pasaron la mar;
\footnote{\textbf{10:1} Éxod 13,21; Éxod 14,22} \bibleverse{2} Y todos
en Moisés fueron bautizados en la nube y en la mar; \footnote{\textbf{10:2}
  Éxod 16,4; Éxod 16,35; Deut 8,3} \bibleverse{3} Y todos comieron la
misma vianda espiritual; \bibleverse{4} Y todos bebieron la misma bebida
espiritual; porque bebían de la piedra espiritual que los seguía, y la
piedra era Cristo: \footnote{\textbf{10:4} Éxod 17,6}

\hypertarget{a-pesar-de-esto-debido-a-que-voluntariamente-sirvieron-su-lujuria-por-la-carne-fueron-rechazados-como-un-ejemplo-de-advertencia-para-nosotros}{%
\subsection{A pesar de esto, debido a que voluntariamente sirvieron su
lujuria por la carne, fueron rechazados como un ejemplo de advertencia
para
nosotros}\label{a-pesar-de-esto-debido-a-que-voluntariamente-sirvieron-su-lujuria-por-la-carne-fueron-rechazados-como-un-ejemplo-de-advertencia-para-nosotros}}

\bibleverse{5} Mas de muchos de ellos no se agradó Dios; por lo cual
fueron postrados en el desierto. \footnote{\textbf{10:5} Núm 14,22-32}

\bibleverse{6} Empero estas cosas fueron en figura de nosotros, para que
no codiciemos cosas malas, como ellos codiciaron. \footnote{\textbf{10:6}
  Núm 11,4} \bibleverse{7} Ni seáis honradores de ídolos, como algunos
de ellos; según está escrito: Sentóse el pueblo á comer y á beber, y se
levantaron á jugar. \bibleverse{8} Ni forniquemos, como algunos de ellos
fornicaron, y cayeron en un día veinte y tres mil. \bibleverse{9} Ni
tentemos á Cristo, como también algunos de ellos le tentaron, y
perecieron por las serpientes. \footnote{\textbf{10:9} Núm 21,4-6}
\bibleverse{10} Ni murmuréis, como algunos de ellos murmuraron, y
perecieron por el destructor. \footnote{\textbf{10:10} Núm 14,2; Núm
  14,35-36; Heb 3,11; Heb 3,17} \bibleverse{11} Y estas cosas les
acontecieron en figura; y son escritas para nuestra admonición, en
quienes los fines de los siglos han parado. \footnote{\textbf{10:11} 1Pe
  4,7} \bibleverse{12} Así que, el que piensa estar firme, mire no
caiga.

\bibleverse{13} No os ha tomado tentación, sino humana: mas fiel es
Dios, que no os dejará ser tentados más de lo que podéis llevar; antes
dará también juntamente con la tentación la salida, para que podáis
aguantar.

\hypertarget{la-participaciuxf3n-en-idolatruxeda-y-comidas-de-sacrificio-es-incompatible-con-la-celebraciuxf3n-de-la-cena-del-seuxf1or-cristiano-y-por-lo-tanto-debe-evitarse}{%
\subsection{La participación en idolatría y comidas de sacrificio es
incompatible con la celebración de la Cena del Señor cristiano y, por lo
tanto, debe
evitarse}\label{la-participaciuxf3n-en-idolatruxeda-y-comidas-de-sacrificio-es-incompatible-con-la-celebraciuxf3n-de-la-cena-del-seuxf1or-cristiano-y-por-lo-tanto-debe-evitarse}}

\bibleverse{14} Por tanto, amados míos, huid de la idolatría.
\footnote{\textbf{10:14} 1Jn 5,21} \bibleverse{15} Como á sabios hablo;
juzgad vosotros lo que digo. \bibleverse{16} La copa de bendición que
bendecimos, ¿no es la comunión de la sangre de Cristo? El pan que
partimos, ¿no es la comunión del cuerpo de Cristo? \bibleverse{17}
Porque un pan, es que muchos somos un cuerpo; pues todos participamos de
aquel un pan. \footnote{\textbf{10:17} 1Cor 12,27; Rom 12,5}
\bibleverse{18} Mirad á Israel según la carne: los que comen de los
sacrificios ¿no son partícipes con el altar? \footnote{\textbf{10:18}
  Lev 7,6}

\bibleverse{19} ¿Qué pues digo? ¿Que el ídolo es algo? ¿ó que sea algo
lo que es sacrificado á los ídolos? \footnote{\textbf{10:19} 1Cor 8,4}
\bibleverse{20} Antes digo que lo que los Gentiles sacrifican, á los
demonios lo sacrifican, y no á Dios: y no querría que vosotros fueseis
partícipes con los demonios. \bibleverse{21} No podéis beber la copa del
Señor, y la copa de los demonios: no podéis ser partícipes de la mesa
del Señor, y de la mesa de los demonios. \bibleverse{22} ¿O provocaremos
á celo al Señor? ¿Somos más fuertes que él?

\hypertarget{cuuxe1ndo-es-seguro-el-consumo-de-carne-sacrificada-a-los-uxeddolos-restricciuxf3n-de-la-libertad-cristiana-por-consideraciuxf3n-al-amor-fraternal}{%
\subsection{¿Cuándo es seguro el consumo de carne sacrificada a los
ídolos? Restricción de la libertad cristiana por consideración al amor
fraternal}\label{cuuxe1ndo-es-seguro-el-consumo-de-carne-sacrificada-a-los-uxeddolos-restricciuxf3n-de-la-libertad-cristiana-por-consideraciuxf3n-al-amor-fraternal}}

\bibleverse{23} Todo me es lícito, mas no todo conviene: todo me es
lícito, mas no todo edifica. \footnote{\textbf{10:23} 1Cor 6,12}
\bibleverse{24} Ninguno busque su propio bien, sino el del otro.
\footnote{\textbf{10:24} Rom 15,2; Fil 2,4} \bibleverse{25} De todo lo
que se vende en la carnicería, comed, sin preguntar nada por causa de la
conciencia; \footnote{\textbf{10:25} Rom 14,2-10; Rom 14,22}
\bibleverse{26} Porque del Señor es la tierra y lo que la hinche.
\bibleverse{27} Y si algún infiel os llama, y queréis ir, de todo lo que
se os pone delante comed, sin preguntar nada por causa de la conciencia.
\bibleverse{28} Mas si alguien os dijere: Esto fué sacrificado á los
ídolos: no lo comáis, por causa de aquel que lo declaró, y por causa de
la conciencia: porque del Señor es la tierra y lo que la hinche.
\bibleverse{29} La conciencia, digo, no tuya, sino del otro. Pues ¿por
qué ha de ser juzgada mi libertad por otra conciencia? \bibleverse{30} Y
si yo con agradecimiento participo, ¿por qué he de ser blasfemado por lo
que doy gracias? \footnote{\textbf{10:30} 1Tim 4,4}

\hypertarget{amonestaciuxf3n-final-para-el-correcto-caminar-cristiano-en-todo-momento}{%
\subsection{Amonestación final para el correcto caminar cristiano en
todo
momento}\label{amonestaciuxf3n-final-para-el-correcto-caminar-cristiano-en-todo-momento}}

\bibleverse{31} Si pues coméis, ó bebéis, ó hacéis otra cosa, hacedlo
todo á gloria de Dios. \footnote{\textbf{10:31} Col 3,17}
\bibleverse{32} Sed sin ofensa á Judíos, y á Gentiles, y á la iglesia de
Dios: \footnote{\textbf{10:32} Rom 14,13} \bibleverse{33} Como también
yo en todas las cosas complazco á todos, no procurando mi propio
beneficio, sino el de muchos, para que sean salvos. \footnote{\textbf{10:33}
  1Cor 9,20-22}

\hypertarget{section-10}{%
\section{11}\label{section-10}}

\bibleverse{1} Sed imitadores de mí, así como yo de Cristo.

\hypertarget{sobre-el-comportamiento-decente-de-los-hombres-y-el-velo-de-las-mujeres-durante-la-oraciuxf3n-y-el-culto}{%
\subsection{Sobre el comportamiento decente de los hombres y el velo de
las mujeres durante la oración y el
culto}\label{sobre-el-comportamiento-decente-de-los-hombres-y-el-velo-de-las-mujeres-durante-la-oraciuxf3n-y-el-culto}}

\bibleverse{2} Y os alabo, hermanos, que en todo os acordáis de mí, y
retenéis las instrucciones mías, de la manera que os enseñé.
\bibleverse{3} Mas quiero que sepáis, que Cristo es la cabeza de todo
varón; y el varón es la cabeza de la mujer; y Dios la cabeza de Cristo.
\footnote{\textbf{11:3} Gén 3,16; Efes 5,23; 1Cor 3,23} \bibleverse{4}
Todo varón que ora ó profetiza cubierta la cabeza, afrenta su cabeza.
\bibleverse{5} Mas toda mujer que ora ó profetiza no cubierta su cabeza,
afrenta su cabeza; porque lo mismo es que si se rayese. \bibleverse{6}
Porque si la mujer no se cubre, trasquílese también: y si es deshonesto
á la mujer trasquilarse ó raerse, cúbrase. \bibleverse{7} Porque el
varón no ha de cubrir la cabeza, porque es imagen y gloria de Dios: mas
la mujer es gloria del varón. \bibleverse{8} Porque el varón no es de la
mujer, sino la mujer del varón. \bibleverse{9} Porque tampoco el varón
fué criado por causa de la mujer, sino la mujer por causa del varón.
\footnote{\textbf{11:9} Gén 2,18} \bibleverse{10} Por lo cual, la mujer
debe tener señal de potestad sobre su cabeza, por causa de los ángeles.

\hypertarget{rechazo-del-desduxe9n-por-la-mujer-y-todas-las-discusiones-sobre-el-tema}{%
\subsection{Rechazo del desdén por la mujer y todas las discusiones
sobre el
tema}\label{rechazo-del-desduxe9n-por-la-mujer-y-todas-las-discusiones-sobre-el-tema}}

\bibleverse{11} Mas ni el varón sin la mujer, ni la mujer sin el varón,
en el Señor. \bibleverse{12} Porque como la mujer es del varón, así
también el varón es por la mujer: empero todo de Dios. \bibleverse{13}
Juzgad vosotros mismos: ¿es honesto orar la mujer á Dios no cubierta?
\bibleverse{14} La misma naturaleza ¿no os enseña que al hombre sea
deshonesto criar cabello? \bibleverse{15} Por el contrario, á la mujer
criar el cabello le es honroso; porque en lugar de velo le es dado el
cabello. \bibleverse{16} Con todo eso, si alguno parece ser contencioso,
nosotros no tenemos tal costumbre, ni las iglesias de Dios.

\hypertarget{seria-reprimenda-por-los-agravios-en-las-comidas-comunes-e-instrucciones-para-la-celebraciuxf3n-digna-de-la-cena-del-seuxf1or}{%
\subsection{Seria reprimenda por los agravios en las comidas comunes e
instrucciones para la celebración digna de la Cena del
Señor}\label{seria-reprimenda-por-los-agravios-en-las-comidas-comunes-e-instrucciones-para-la-celebraciuxf3n-digna-de-la-cena-del-seuxf1or}}

\bibleverse{17} Esto empero os denuncio, que no alabo, que no por mejor
sino por peor os juntáis. \bibleverse{18} Porque lo primero, cuando os
juntáis en la iglesia, oigo que hay entre vosotros disensiones; y en
parte lo creo. \bibleverse{19} Porque preciso es que haya entre vosotros
aun herejías, para que los que son probados se manifiesten entre
vosotros. \footnote{\textbf{11:19} Mat 18,7; 1Jn 2,19} \bibleverse{20}
Cuando pues os juntáis en uno, esto no es comer la cena del Señor:
\bibleverse{21} Porque cada uno toma antes para comer su propia cena; y
el uno tiene hambre, y el otro está embriagado. \bibleverse{22} Pues
qué, ¿no tenéis casas en que comáis y bebáis? ¿ó menospreciáis la
iglesia de Dios, y avergonzáis á los que no tienen? ¿Qué os diré? ¿os
alabaré? En esto no os alabo. \footnote{\textbf{11:22} Sant 2,5; Sant
  1,2-6}

\hypertarget{la-celebraciuxf3n-correcta-de-la-cena-del-seuxf1or-y-las-malas-consecuencias-de-un-disfrute-indigno-recordatorio-final}{%
\subsection{La celebración correcta de la Cena del Señor y las malas
consecuencias de un disfrute indigno; recordatorio
final}\label{la-celebraciuxf3n-correcta-de-la-cena-del-seuxf1or-y-las-malas-consecuencias-de-un-disfrute-indigno-recordatorio-final}}

\bibleverse{23} Porque yo recibí del Señor lo que también os he
enseñado: Que el Señor Jesús, la noche que fué entregado, tomó pan;
\footnote{\textbf{11:23} Mat 26,26-28; Mar 14,22-24; Luc 22,19-20}
\bibleverse{24} Y habiendo dado gracias, lo partió, y dijo: Tomad,
comed: esto es mi cuerpo que por vosotros es partido: haced esto en
memoria de mí. \bibleverse{25} Asimismo tomó también la copa, después de
haber cenado, diciendo: Esta copa es el nuevo pacto en mi sangre: haced
esto todas las veces que bebiereis, en memoria de mí. \bibleverse{26}
Porque todas las veces que comiereis este pan, y bebiereis esta copa, la
muerte del Señor anunciáis hasta que venga. \footnote{\textbf{11:26} Mat
  26,29}

\bibleverse{27} De manera que, cualquiera que comiere este pan ó bebiere
esta copa del Señor indignamente, será culpado del cuerpo y de la sangre
del Señor. \footnote{\textbf{11:27} 1Cor 11,21-22} \bibleverse{28} Por
tanto, pruébese cada uno á sí mismo, y coma así de aquel pan, y beba de
aquella copa. \footnote{\textbf{11:28} Mat 26,22} \bibleverse{29} Porque
el que come y bebe indignamente, juicio come y bebe para sí, no
discerniendo el cuerpo del Señor. \footnote{\textbf{11:29} 1Cor 10,16-17}
\bibleverse{30} Por lo cual hay muchos enfermos y debilitados entre
vosotros; y muchos duermen. \bibleverse{31} Que si nos examinásemos á
nosotros mismos, cierto no seríamos juzgados. \bibleverse{32} Mas siendo
juzgados, somos castigados del Señor, para que no seamos condenados con
el mundo. \footnote{\textbf{11:32} Prov 3,11-12} \bibleverse{33} Así,
que, hermanos míos, cuando os juntáis á comer, esperaos unos á otros.
\bibleverse{34} Si alguno tuviere hambre, coma en su casa, porque no os
juntéis para juicio. Las demás cosas ordenaré cuando llegare.

\hypertarget{la-marca-de-los-dones-espirituales-divinamente-forjados}{%
\subsection{La marca de los dones espirituales divinamente
forjados}\label{la-marca-de-los-dones-espirituales-divinamente-forjados}}

\hypertarget{section-11}{%
\section{12}\label{section-11}}

\bibleverse{1} Y acerca de los dones espirituales, no quiero, hermanos,
que ignoréis. \bibleverse{2} Sabéis que cuando erais Gentiles, ibais,
como erais llevados, á los ídolos mudos. \bibleverse{3} Por tanto os
hago saber, que nadie que hable por Espíritu de Dios, llama anatema á
Jesús; y nadie puede llamar á Jesús Señor, sino por Espíritu Santo.
\footnote{\textbf{12:3} Mar 9,39; 1Jn 4,2; 1Jn 1,4-3}

\hypertarget{diversidad-de-dones-espirituales-pero-solo-un-espuxedritu-activo-y-un-propuxf3sito}{%
\subsection{Diversidad de dones espirituales, pero solo un espíritu
activo y un
propósito}\label{diversidad-de-dones-espirituales-pero-solo-un-espuxedritu-activo-y-un-propuxf3sito}}

\bibleverse{4} Empero hay repartimiento de dones; mas el mismo Espíritu
es. \footnote{\textbf{12:4} Rom 12,6; Efes 4,4-999} \bibleverse{5} Y hay
repartimiento de ministerios; mas el mismo Señor es. \footnote{\textbf{12:5}
  1Cor 12,28} \bibleverse{6} Y hay repartimiento de operaciones; mas el
mismo Dios es el que obra todas las cosas en todos. \bibleverse{7}
Empero á cada uno le es dada manifestación del Espíritu para provecho.
\bibleverse{8} Porque á la verdad, á éste es dada por el Espíritu
palabra de sabiduría; á otro, palabra de ciencia según el mismo
Espíritu; \bibleverse{9} A otro, fe por el mismo Espíritu; y á otro,
dones de sanidades por el mismo Espíritu; \bibleverse{10} A otro,
operaciones de milagros; y á otro, profecía; y á otro, discreción de
espíritus; y á otro, géneros de lenguas; y á otro, interpretación de
lenguas. \footnote{\textbf{12:10} 1Cor 14,-1; Hech 2,4} \bibleverse{11}
Mas todas estas cosas obra uno y el mismo Espíritu, repartiendo
particularmente á cada uno como quiere. \footnote{\textbf{12:11} Rom
  12,3; Efes 4,7}

\hypertarget{ilustrado-por-la-paruxe1bola-del-cuerpo-humano-y-sus-muchos-miembros}{%
\subsection{Ilustrado por la parábola del cuerpo humano y sus muchos
miembros}\label{ilustrado-por-la-paruxe1bola-del-cuerpo-humano-y-sus-muchos-miembros}}

\bibleverse{12} Porque de la manera que el cuerpo es uno, y tiene muchos
miembros, empero todos los miembros del cuerpo, siendo muchos, son un
cuerpo, así también Cristo. \bibleverse{13} Porque por un Espíritu somos
todos bautizados en un cuerpo, ora Judíos ó Griegos, ora siervos ó
libres; y todos hemos bebido de un mismo Espíritu. \footnote{\textbf{12:13}
  Gal 3,28}

\bibleverse{14} Pues ni tampoco el cuerpo es un miembro, sino muchos.
\bibleverse{15} Si dijere el pie: Porque no soy mano, no soy del cuerpo:
¿por eso no será del cuerpo? \bibleverse{16} Y si dijere la oreja:
Porque no soy ojo, no soy del cuerpo: ¿por eso no será del cuerpo?
\bibleverse{17} Si todo el cuerpo fuese ojo, ¿dónde estaría el oído? Si
todo fuese oído, ¿dónde estaría el olfato? \bibleverse{18} Mas ahora
Dios ha colocado los miembros cada uno de ellos en el cuerpo, como
quiso. \bibleverse{19} Que si todos fueran un miembro, ¿dónde estuviera
el cuerpo? \bibleverse{20} Mas ahora muchos miembros son á la verdad,
empero un cuerpo. \bibleverse{21} Ni el ojo puede decir á la mano: No te
he menester: ni asimismo la cabeza á los pies: No tengo necesidad de
vosotros. \bibleverse{22} Antes, mucho más los miembros del cuerpo que
parecen más flacos, son necesarios; \bibleverse{23} Y á aquellos del
cuerpo que estimamos ser más viles, á éstos vestimos más honrosamente; y
los que en nosotros son menos honestos, tienen más compostura.
\bibleverse{24} Porque los que en nosotros son más honestos, no tienen
necesidad: mas Dios ordenó el cuerpo, dando más abundante honor al que
le faltaba; \bibleverse{25} Para que no haya desavenencia en el cuerpo,
sino que los miembros todos se interesen los unos por los otros.
\bibleverse{26} Por manera que si un miembro padece, todos los miembros
á una se duelen; y si un miembro es honrado, todos los miembros á una se
gozan.

\hypertarget{aplicaciuxf3n-de-la-imagen-a-la-estructura-divina-de-la-iglesia}{%
\subsection{Aplicación de la imagen a la estructura divina de la
iglesia}\label{aplicaciuxf3n-de-la-imagen-a-la-estructura-divina-de-la-iglesia}}

\bibleverse{27} Pues vosotros sois el cuerpo de Cristo, y miembros en
parte. \bibleverse{28} Y á unos puso Dios en la iglesia, primeramente
apóstoles, luego profetas, lo tercero doctores; luego facultades; luego
dones de sanidades, ayudas, gobernaciones, géneros de lenguas.
\footnote{\textbf{12:28} Efes 4,11-12} \bibleverse{29} ¿Son todos
apóstoles? ¿son todos profetas? ¿todos doctores? ¿todos facultades?
\bibleverse{30} ¿Tienen todos dones de sanidad? ¿hablan todos lenguas?
¿interpretan todos?

\hypertarget{sin-amor-incluso-los-dones-espirituales-muxe1s-elevados-no-valen-nada}{%
\subsection{Sin amor, incluso los dones espirituales más elevados no
valen
nada}\label{sin-amor-incluso-los-dones-espirituales-muxe1s-elevados-no-valen-nada}}

\bibleverse{31} Empero procurad los mejores dones: mas aun yo os muestro
un camino más excelente.

\hypertarget{section-12}{%
\section{13}\label{section-12}}

\bibleverse{1} Si yo hablase lenguas humanas y angélicas, y no tengo
caridad, vengo á ser como metal que resuena, ó címbalo que retiñe.
\bibleverse{2} Y si tuviese profecía, y entendiese todos los misterios y
toda ciencia; y si tuviese toda la fe, de tal manera que traspasase los
montes, y no tengo caridad, nada soy. \footnote{\textbf{13:2} Mat 7,22;
  Mat 17,20} \bibleverse{3} Y si repartiese toda mi hacienda para dar de
comer á pobres, y si entregase mi cuerpo para ser quemado, y no tengo
caridad, de nada me sirve. \footnote{\textbf{13:3} Mat 6,2}

\hypertarget{la-esencia-del-amor}{%
\subsection{La esencia del amor}\label{la-esencia-del-amor}}

\bibleverse{4} La caridad es sufrida, es benigna; la caridad no tiene
envidia, la caridad no hace sinrazón, no se ensancha; \bibleverse{5} No
es injuriosa, no busca lo suyo, no se irrita, no piensa el mal;
\footnote{\textbf{13:5} Fil 2,4} \bibleverse{6} No se huelga de la
injusticia, mas se huelga de la verdad; \footnote{\textbf{13:6} Rom 12,9}
\bibleverse{7} Todo lo sufre, todo lo cree, todo lo espera, todo lo
soporta. \footnote{\textbf{13:7} Mat 18,21-22; Prov 10,12; Rom 15,1}

\hypertarget{la-perfecciuxf3n-del-amor-eterno-contra-el-fragmento-de-otras-gracias}{%
\subsection{La perfección del amor eterno contra el fragmento de otras
gracias}\label{la-perfecciuxf3n-del-amor-eterno-contra-el-fragmento-de-otras-gracias}}

\bibleverse{8} La caridad nunca deja de ser: mas las profecías se han de
acabar, y cesarán las lenguas, y la ciencia ha de ser quitada;
\bibleverse{9} Porque en parte conocemos, y en parte profetizamos;
\bibleverse{10} Mas cuando venga lo que es perfecto, entonces lo que es
en parte será quitado. \bibleverse{11} Cuando yo era niño, hablaba como
niño, pensaba como niño, juzgaba como niño; mas cuando ya fuí hombre
hecho, dejé lo que era de niño. \bibleverse{12} Ahora vemos por espejo,
en obscuridad; mas entonces veremos cara á cara: ahora conozco en parte;
mas entonces conoceré como soy conocido. \bibleverse{13} Y ahora
permanecen la fe, la esperanza, y la caridad, estas tres: empero la
mayor de ellas es la caridad. \footnote{\textbf{13:13} 1Tes 1,3; 1Jn
  4,16}

\hypertarget{section-13}{%
\section{14}\label{section-13}}

\bibleverse{1} Seguid la caridad; y procurad los dones espirituales, mas
sobre todo que profeticéis.

\hypertarget{la-diferencia-entre-el-habla-profuxe9tica-y-el-hablar-en-lenguas}{%
\subsection{La diferencia entre el habla profética y el hablar en
lenguas}\label{la-diferencia-entre-el-habla-profuxe9tica-y-el-hablar-en-lenguas}}

\bibleverse{2} Porque el que habla en lenguas, no habla á los hombres,
sino á Dios; porque nadie le entiende, aunque en espíritu hable
misterios. \bibleverse{3} Mas el que profetiza, habla á los hombres para
edificación, y exhortación, y consolación. \bibleverse{4} El que habla
lengua extraña, á sí mismo se edifica; mas el que profetiza, edifica á
la iglesia. \bibleverse{5} Así que, quisiera que todos vosotros
hablaseis lenguas, empero más que profetizaseis: porque mayor es el que
profetiza que el que habla lenguas, si también no interpretare, para que
la iglesia tome edificación. \footnote{\textbf{14:5} Núm 11,29; 1Cor
  12,10}

\bibleverse{6} Ahora pues, hermanos, si yo fuere á vosotros hablando
lenguas, ¿qué os aprovecharé, si no os hablare, ó con revelación, ó con
ciencia, ó con profecía, ó con doctrina? \footnote{\textbf{14:6} 1Cor
  12,8}

\hypertarget{la-inutilidad-e-inadecuaciuxf3n-de-todo-sonido-y-habla-incomprensibles}{%
\subsection{La inutilidad e inadecuación de todo sonido y habla
incomprensibles}\label{la-inutilidad-e-inadecuaciuxf3n-de-todo-sonido-y-habla-incomprensibles}}

\bibleverse{7} Ciertamente las cosas inanimadas que hacen sonidos, como
la flauta ó la vihuela, si no dieren distinción de voces, ¿cómo se sabrá
lo que se tañe con la flauta, ó con la vihuela? \bibleverse{8} Y si la
trompeta diere sonido incierto, ¿quién se apercibirá á la batalla?
\bibleverse{9} Así también vosotros, si por la lengua no diereis palabra
bien significante, ¿cómo se entenderá lo que se dice? porque hablaréis
al aire. \bibleverse{10} Tantos géneros de voces, por ejemplo, hay en el
mundo, y nada hay mudo; \bibleverse{11} Mas si yo ignorare el valor de
la voz, seré bárbaro al que habla, y el que habla será bárbaro para mí.
\bibleverse{12} Así también vosotros; pues que anheláis espirituales
dones, procurad ser excelentes para la edificación de la iglesia.

\bibleverse{13} Por lo cual, el que habla lengua extraña, pida que la
interprete. \footnote{\textbf{14:13} 1Cor 12,10} \bibleverse{14} Porque
si yo orare en lengua desconocida, mi espíritu ora; mas mi entendimiento
es sin fruto.

\bibleverse{15} ¿Qué pues? Oraré con el espíritu, mas oraré también con
entendimiento; cantaré con el espíritu, mas cantaré también con
entendimiento. \bibleverse{16} Porque si bendijeres con el espíritu, el
que ocupa lugar de un mero particular, ¿cómo dirá amén á tu acción de
gracias? pues no sabe lo que has dicho. \bibleverse{17} Porque tú, á la
verdad, bien haces gracias; mas el otro no es edificado. \bibleverse{18}
Doy gracias á Dios que hablo lenguas más que todos vosotros:
\bibleverse{19} Pero en la iglesia más quiero hablar cinco palabras con
mi sentido, para que enseñe también á los otros, que diez mil palabras
en lengua desconocida.

\hypertarget{el-antiguo-testamento-y-el-mundo-exterior-no-cristiano-tambiuxe9n-condenan-este-incomprensible-discurso}{%
\subsection{El Antiguo Testamento y el mundo exterior no cristiano
también condenan este incomprensible
discurso}\label{el-antiguo-testamento-y-el-mundo-exterior-no-cristiano-tambiuxe9n-condenan-este-incomprensible-discurso}}

\bibleverse{20} Hermanos, no seáis niños en el sentido, sino sed niños
en la malicia: empero perfectos en el sentido. \footnote{\textbf{14:20}
  Efes 4,14} \bibleverse{21} En la ley está escrito: En otras lenguas y
en otros labios hablaré á este pueblo; y ni aun así me oirán, dice el
Señor. \bibleverse{22} Así que, las lenguas por señal son, no á los
fieles, sino á los infieles: mas la profecía, no á los infieles, sino á
los fieles. \bibleverse{23} De manera que, si toda la iglesia se juntare
en uno, y todos hablan lenguas, y entran indoctos ó infieles, ¿no dirán
que estáis locos? \bibleverse{24} Mas si todos profetizan, y entra algún
infiel ó indocto, de todos es convencido, de todos es juzgado;
\bibleverse{25} Lo oculto de su corazón se hace manifiesto: y así,
postrándose sobre el rostro, adorará á Dios, declarando que
verdaderamente Dios está en vosotros.

\hypertarget{orden-de-los-altavoces}{%
\subsection{Orden de los altavoces}\label{orden-de-los-altavoces}}

\bibleverse{26} ¿Qué hay pues, hermanos? Cuando os juntáis, cada uno de
vosotros tiene salmo, tiene doctrina, tiene lengua, tiene revelación,
tiene interpretación: hágase todo para edificación. \footnote{\textbf{14:26}
  1Cor 12,8-10; Efes 4,12} \bibleverse{27} Si hablare alguno en lengua
extraña, sea esto por dos, ó á lo más tres, y por turno; mas uno
interprete. \bibleverse{28} Y si no hubiere intérprete, calle en la
iglesia, y hable á sí mismo y á Dios. \bibleverse{29} Asimismo, los
profetas hablen dos ó tres, y los demás juzguen. \bibleverse{30} Y si á
otro que estuviere sentado, fuere revelado, calle el primero.
\bibleverse{31} Porque podéis todos profetizar uno por uno, para que
todos aprendan, y todos sean exhortados. \bibleverse{32} Y los espíritus
de los que profetizaren, sujétense á los profetas; \bibleverse{33}
Porque Dios no es Dios de disensión, sino de paz; como en todas las
iglesias de los santos. \footnote{\textbf{14:33} 1Cor 14,40}

\hypertarget{contra-los-discursos-inapropiados-de-mujeres-en-reuniones}{%
\subsection{Contra los discursos inapropiados de mujeres en
reuniones}\label{contra-los-discursos-inapropiados-de-mujeres-en-reuniones}}

\bibleverse{34} Vuestras mujeres callen en las congregaciones; porque no
les es permitido hablar, sino que estén sujetas, como también la ley
dice. \footnote{\textbf{14:34} 1Tim 2,11-12; Gén 3,16} \bibleverse{35} Y
si quieren aprender alguna cosa, pregunten en casa á sus maridos; porque
deshonesta cosa es hablar una mujer en la congregación. \bibleverse{36}
Qué, ¿ha salido de vosotros la palabra de Dios? ¿ó á vosotros solos ha
llegado?

\bibleverse{37} Si alguno á su parecer, es profeta, ó espiritual,
reconozca lo que os escribo, porque son mandamientos del Señor.
\footnote{\textbf{14:37} 1Jn 4,6} \bibleverse{38} Mas el que ignora,
ignore.

\bibleverse{39} Así que, hermanos, procurad profetizar; y no impidáis el
hablar lenguas. \bibleverse{40} Empero hágase todo decentemente y con
orden.

\hypertarget{de-los-hechos-y-testigos-por-los-que-se-certifica-la-resurrecciuxf3n-de-cristo}{%
\subsection{De los hechos y testigos por los que se certifica la
resurrección de
Cristo}\label{de-los-hechos-y-testigos-por-los-que-se-certifica-la-resurrecciuxf3n-de-cristo}}

\hypertarget{section-14}{%
\section{15}\label{section-14}}

\bibleverse{1} Además os declaro, hermanos, el evangelio que os he
predicado, el cual también recibisteis, en el cual también perseveráis;
\bibleverse{2} Por el cual asimismo, si retenéis la palabra que os he
predicado, sois salvos, si no creísteis en vano.

\bibleverse{3} Porque primeramente os he enseñado lo que asimismo
recibí: Que Cristo fué muerto por nuestros pecados, conforme á las
Escrituras; \footnote{\textbf{15:3} Is 53,8-9} \bibleverse{4} Y que fué
sepultado, y que resucitó al tercer día, conforme á las Escrituras;
\footnote{\textbf{15:4} Luc 24,27; Luc 24,44-46} \bibleverse{5} Y que
apareció á Cefas, y después á los doce. \footnote{\textbf{15:5} Juan
  20,19; Juan 20,26; Luc 23,34} \bibleverse{6} Después apareció á más de
quinientos hermanos juntos; de los cuales muchos viven aún, y otros son
muertos. \bibleverse{7} Después apareció á Jacobo; después á todos los
apóstoles. \bibleverse{8} Y el postrero de todos, como á un abortivo, me
apareció á mí. \footnote{\textbf{15:8} 1Cor 9,1; Hech 9,3-6}
\bibleverse{9} Porque yo soy el más pequeño de los apóstoles, que no soy
digno de ser llamado apóstol, porque perseguí la iglesia de Dios.
\footnote{\textbf{15:9} Hech 8,3; Efes 3,8} \bibleverse{10} Empero por
la gracia de Dios soy lo que soy: y su gracia no ha sido en vano para
conmigo; antes he trabajado más que todos ellos: pero no yo, sino la
gracia de Dios que fué conmigo. \footnote{\textbf{15:10} 2Cor 11,5; 2Cor
  11,23} \bibleverse{11} Porque, ó sea yo ó sean ellos, así predicamos,
y así habéis creído.

\hypertarget{la-fe-y-la-firme-esperanza-de-todos-los-cristianos-se-basan-en-la-resurrecciuxf3n-de-cristo-de-entre-los-muertos}{%
\subsection{La fe y la firme esperanza de todos los cristianos se basan
en la resurrección de Cristo de entre los
muertos}\label{la-fe-y-la-firme-esperanza-de-todos-los-cristianos-se-basan-en-la-resurrecciuxf3n-de-cristo-de-entre-los-muertos}}

\bibleverse{12} Y si Cristo es predicado que resucitó de los muertos,
¿cómo dicen algunos entre vosotros que no hay resurrección de muertos?
\bibleverse{13} Porque si no hay resurrección de muertos, Cristo tampoco
resucitó: \bibleverse{14} Y si Cristo no resucitó, vana es entonces
nuestra predicación, vana es también vuestra fe. \bibleverse{15} Y aun
somos hallados falsos testigos de Dios; porque hemos testificado de Dios
que él haya levantado á Cristo; al cual no levantó, si en verdad los
muertos no resucitan. \bibleverse{16} Porque si los muertos no
resucitan, tampoco Cristo resucitó: \bibleverse{17} Y si Cristo no
resucitó, vuestra fe es vana; aun estáis en vuestros pecados.
\bibleverse{18} Entonces también los que durmieron en Cristo son
perdidos. \bibleverse{19} Si en esta vida solamente esperamos en Cristo,
los más miserables somos de todos los hombres.

\hypertarget{exposiciuxf3n-de-las-consecuencias-de-la-resurrecciuxf3n-de-cristo-los-procesos-en-los-que-la-resurrecciuxf3n-tiene-lugar-hasta-su-finalizaciuxf3n}{%
\subsection{Exposición de las consecuencias de la resurrección de
Cristo; los procesos en los que la resurrección tiene lugar hasta su
finalización}\label{exposiciuxf3n-de-las-consecuencias-de-la-resurrecciuxf3n-de-cristo-los-procesos-en-los-que-la-resurrecciuxf3n-tiene-lugar-hasta-su-finalizaciuxf3n}}

\bibleverse{20} Mas ahora Cristo ha resucitado de los muertos; primicias
de los que durmieron es hecho. \footnote{\textbf{15:20} 1Cor 6,14; Col
  1,18} \bibleverse{21} Porque por cuanto la muerte entró por un hombre,
también por un hombre la resurrección de los muertos. \footnote{\textbf{15:21}
  Gén 3,17-19; Rom 5,18} \bibleverse{22} Porque así como en Adam todos
mueren, así también en Cristo todos serán vivificados. \bibleverse{23}
Mas cada uno en su orden: Cristo las primicias; luego los que son de
Cristo, en su venida. \footnote{\textbf{15:23} 1Tes 4,16-17}
\bibleverse{24} Luego el fin; cuando entregará el reino á Dios y al
Padre, cuando habrá quitado todo imperio, y toda potencia y potestad.
\footnote{\textbf{15:24} Rom 8,38} \bibleverse{25} Porque es menester
que él reine, hasta poner á todos sus enemigos debajo de sus pies.
\footnote{\textbf{15:25} Mat 22,44} \bibleverse{26} Y el postrer enemigo
que será deshecho, será la muerte. \footnote{\textbf{15:26} Apoc 20,14;
  Apoc 21,4} \bibleverse{27} Porque todas las cosas sujetó debajo de sus
pies. Y cuando dice: Todas las cosas son sujetadas á él, claro está
exceptuado aquel que sujetó á él todas las cosas. \bibleverse{28} Mas
luego que todas las cosas le fueren sujetas, entonces también el mismo
Hijo se sujetará al que le sujetó á él todas las cosas, para que Dios
sea todas las cosas en todos.

\hypertarget{mucho-de-lo-que-los-cristianos-hacen-y-sufren-solo-es-justificado-y-comprensible-cuando-creen-en-la-resurrecciuxf3n}{%
\subsection{Mucho de lo que los cristianos hacen y sufren solo es
justificado y comprensible cuando creen en la
resurrección}\label{mucho-de-lo-que-los-cristianos-hacen-y-sufren-solo-es-justificado-y-comprensible-cuando-creen-en-la-resurrecciuxf3n}}

\bibleverse{29} De otro modo, ¿qué harán los que se bautizan por los
muertos, si en ninguna manera los muertos resucitan? ¿Por qué pues se
bautizan por los muertos? \bibleverse{30} ¿Y por qué nosotros peligramos
á toda hora? \footnote{\textbf{15:30} Rom 8,36; Gal 5,11}
\bibleverse{31} Sí, por la gloria que en orden á vosotros tengo en
Cristo Jesús Señor nuestro, cada día muero. \footnote{\textbf{15:31}
  2Cor 4,10} \bibleverse{32} Si como hombre batallé en Efeso contra las
bestias, ¿qué me aprovecha? Si los muertos no resucitan, comamos y
bebamos, que mañana moriremos. \bibleverse{33} No erréis: las malas
conversaciones corrompen las buenas costumbres. \bibleverse{34} Velad
debidamente, y no pequéis; porque algunos no conocen á Dios: para
vergüenza vuestra hablo. \footnote{\textbf{15:34} 1Tes 5,8}

\hypertarget{la-imagen-de-la-semilla}{%
\subsection{La imagen de la semilla}\label{la-imagen-de-la-semilla}}

\bibleverse{35} Mas dirá alguno: ¿Cómo resucitarán los muertos? ¿Con qué
cuerpo vendrán? \bibleverse{36} Necio, lo que tú siembras no se
vivifica, si no muriere antes. \bibleverse{37} Y lo que siembras, no
siembras el cuerpo que ha de salir, sino el grano desnudo, acaso de
trigo, ó de otro grano: \bibleverse{38} Mas Dios le da el cuerpo como
quiso, y á cada simiente su propio cuerpo. \footnote{\textbf{15:38} Gén
  1,11}

\hypertarget{toda-la-creaciuxf3n-muestra-la-mayor-diversidad-de-materia-forma-y-naturaleza-de-las-cosas}{%
\subsection{Toda la creación muestra la mayor diversidad de materia,
forma y naturaleza de las
cosas}\label{toda-la-creaciuxf3n-muestra-la-mayor-diversidad-de-materia-forma-y-naturaleza-de-las-cosas}}

\bibleverse{39} Toda carne no es la misma carne; mas una carne
ciertamente es la de los hombres, y otra carne la de los animales, y
otra la de los peces, y otra la de las aves. \bibleverse{40} Y cuerpos
hay celestiales, y cuerpos terrestres; mas ciertamente una es la gloria
de los celestiales, y otra la de los terrestres. \bibleverse{41} Otra es
la gloria del sol, y otra la gloria de la luna, y otra la gloria de las
estrellas: porque una estrella es diferente de otra en gloria.

\bibleverse{42} Así también es la resurrección de los muertos. Se
siembra en corrupción, se levantará en incorrupción; \bibleverse{43} Se
siembra en vergüenza, se levantará con gloria; se siembra en flaqueza,
se levantará con potencia; \bibleverse{44} Se siembra cuerpo animal,
resucitará espiritual cuerpo. Hay cuerpo animal, y hay cuerpo
espiritual.

\hypertarget{la-realidad-de-un-cuerpo-celestial-incorruptible}{%
\subsection{La realidad de un cuerpo celestial,
incorruptible}\label{la-realidad-de-un-cuerpo-celestial-incorruptible}}

\bibleverse{45} Así también está escrito: Fué hecho el primer hombre
Adam en ánima viviente; el postrer Adam en espíritu vivificante.
\footnote{\textbf{15:45} 2Cor 3,17} \bibleverse{46} Mas lo espiritual no
es primero, sino lo animal; luego lo espiritual. \bibleverse{47} El
primer hombre, es de la tierra, terreno: el segundo hombre, que es el
Señor, es del cielo. \bibleverse{48} Cual el terreno, tales también los
terrenos; y cual el celestial, tales también los celestiales.
\bibleverse{49} Y como trajimos la imagen del terreno, traeremos también
la imagen del celestial.

\hypertarget{la-transformaciuxf3n-final-en-la-consumaciuxf3n-de-los-creyentes}{%
\subsection{La transformación final en la consumación de los
creyentes}\label{la-transformaciuxf3n-final-en-la-consumaciuxf3n-de-los-creyentes}}

\bibleverse{50} Esto empero digo, hermanos: que la carne y la sangre no
pueden heredar el reino de Dios; ni la corrupción hereda la
incorrupción.

\bibleverse{51} He aquí, os digo un misterio: Todos ciertamente no
dormiremos, mas todos seremos transformados, \footnote{\textbf{15:51}
  1Tes 4,15-17} \bibleverse{52} En un momento, en un abrir de ojo, á la
final trompeta; porque será tocada la trompeta, y los muertos serán
levantados sin corrupción, y nosotros seremos transformados. \footnote{\textbf{15:52}
  Mat 24,31} \bibleverse{53} Porque es menester que esto corruptible sea
vestido de incorrupción, y esto mortal sea vestido de inmortalidad.
\footnote{\textbf{15:53} 2Cor 5,4} \bibleverse{54} Y cuando esto
corruptible fuere vestido de incorrupción, y esto mortal fuere vestido
de inmortalidad, entonces se efectuará la palabra que está escrita:
Sorbida es la muerte con victoria. \bibleverse{55} ¿Dónde está, oh
muerte, tu aguijón? ¿dónde, oh sepulcro, tu victoria?

\bibleverse{56} Ya que el aguijón de la muerte es el pecado, y la
potencia del pecado, la ley. \bibleverse{57} Mas á Dios gracias, que nos
da la victoria por el Señor nuestro Jesucristo. \footnote{\textbf{15:57}
  1Jn 5,4} \bibleverse{58} Así que, hermanos míos amados, estad firmes y
constantes, creciendo en la obra del Señor siempre, sabiendo que vuestro
trabajo en el Señor no es vano. \footnote{\textbf{15:58} 2Cró 15,7}

\hypertarget{invitaciuxf3n-a-participar-en-la-recaudaciuxf3n-de-fondos-para-jerusaluxe9n}{%
\subsection{Invitación a participar en la recaudación de fondos para
Jerusalén}\label{invitaciuxf3n-a-participar-en-la-recaudaciuxf3n-de-fondos-para-jerusaluxe9n}}

\hypertarget{section-15}{%
\section{16}\label{section-15}}

\bibleverse{1} Cuanto á la colecta para los santos, haced vosotros
también de la manera que ordené en las iglesias de Galacia. \footnote{\textbf{16:1}
  2Cor 8,9; Gal 2,10} \bibleverse{2} Cada primer día de la semana cada
uno de vosotros aparte en su casa, guardando lo que por la bondad de
Dios pudiere; para que cuando yo llegare, no se hagan entonces colectas.
\footnote{\textbf{16:2} Hech 20,7} \bibleverse{3} Y cuando habré
llegado, los que aprobareis por cartas, á éstos enviaré que lleven
vuestro beneficio á Jerusalem. \bibleverse{4} Y si fuere digno el
negocio de que yo también vaya, irán conmigo.

\hypertarget{los-planes-de-viaje-de-pablo-y-las-noticias-de-la-venida-de-timoteo-y-apolos}{%
\subsection{Los planes de viaje de Pablo y las noticias de la venida de
Timoteo y
Apolos}\label{los-planes-de-viaje-de-pablo-y-las-noticias-de-la-venida-de-timoteo-y-apolos}}

\bibleverse{5} Y á vosotros iré, cuando hubiere pasado por Macedonia,
porque por Macedonia tengo de pasar: \footnote{\textbf{16:5} Hech 19,21}
\bibleverse{6} Y podrá ser que me quede con vosotros, ó invernaré
también, para que vosotros me llevéis á donde hubiere de ir.
\bibleverse{7} Porque no os quiero ahora ver de paso; porque espero
estar con vosotros algún tiempo, si el Señor lo permitiere.
\bibleverse{8} Empero estaré en Efeso hasta Pentecostés; \footnote{\textbf{16:8}
  Hech 19,1; Hech 19,10} \bibleverse{9} Porque se me ha abierto puerta
grande y eficaz, y muchos son los adversarios. \footnote{\textbf{16:9}
  2Cor 2,12; Col 4,3}

\bibleverse{10} Y si llegare Timoteo, mirad que esté con vosotros
seguramente; porque la obra del Señor hace también como yo. \footnote{\textbf{16:10}
  1Cor 4,17; Fil 2,19-22} \bibleverse{11} Por tanto, nadie le tenga en
poco; antes, llevadlo en paz, para que venga á mí: porque lo espero con
los hermanos.

\bibleverse{12} Acerca del hermano Apolos, mucho le he rogado que fuese
á vosotros con los hermanos; mas en ninguna manera tuvo voluntad de ir
por ahora; pero irá cuando tuviere oportunidad.

\hypertarget{advertencias-finales-recomendaciones-personales-saludos-y-bendiciones}{%
\subsection{Advertencias finales, recomendaciones personales, saludos y
bendiciones}\label{advertencias-finales-recomendaciones-personales-saludos-y-bendiciones}}

\bibleverse{13} Velad, estad firmes en la fe; portaos varonilmente, y
esforzaos. \footnote{\textbf{16:13} Efes 6,10} \bibleverse{14} Todas
vuestras cosas sean hechas con caridad.

\bibleverse{15} Y os ruego, hermanos, (ya sabéis que la casa de
Estéfanas es las primicias de Acaya, y que se han dedicado al ministerio
de los santos,) \bibleverse{16} Que vosotros os sujetéis á los tales, y
á todos los que ayudan y trabajan. \footnote{\textbf{16:16} Fil 2,29}
\bibleverse{17} Huélgome de la venida de Estéfanas y de Fortunato y de
Achâico: porque éstos suplieron lo que á vosotros faltaba.
\bibleverse{18} Porque recrearon mi espíritu y el vuestro: reconoced
pues á los tales.

\bibleverse{19} Las iglesias de Asia os saludan. Os saludan mucho en el
Señor Aquila y Priscila, con la iglesia que está en su casa. \footnote{\textbf{16:19}
  Hech 18,2; Rom 16,3; Rom 16,5} \bibleverse{20} Os saludan todos los
hermanos. Saludaos los unos á los otros con ósculo santo.

\bibleverse{21} La salutación de mí, Pablo, de mi mano. \bibleverse{22}
El que no amare al Señor Jesucristo, sea anatema. Maranatha.
\bibleverse{23} La gracia del Señor Jesucristo sea con vosotros.
\bibleverse{24} Mi amor en Cristo Jesús sea con todos vosotros. Amén. La
primera á los Corintios fué enviada de Filipos con Estéfanas, y
Fortunato, y Achâico, y Timoteo.
