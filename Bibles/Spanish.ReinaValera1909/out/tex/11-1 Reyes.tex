\hypertarget{section}{%
\section{1}\label{section}}

\bibverse{1} Como el rey David era viejo, y entrado en días, cubríanle
de vestidos, mas no se calentaba. \bibverse{2} Dijéronle por tanto sus
siervos: Busquen á mi señor el rey una moza virgen, para que esté
delante del rey, y lo abrigue, y duerma á su lado, y calentará á mi
señor el rey. \bibverse{3} Y buscaron una moza hermosa por todo el
término de Israel, y hallaron á Abisag Sunamita, y trajéronla al rey.
\bibverse{4} Y la moza era hermosa, la cual calentaba al rey, y le
servía: mas el rey nunca la conoció. \bibverse{5} Entonces Adonía hijo
de Haggith se levantó, diciendo: Yo reinaré. E hízose de carros y gente
de á caballo, y cincuenta hombres que corriesen delante de él.
\bibverse{6} Y su padre nunca lo entristeció en todos sus días con
decirle: ¿Por qué haces así? Y también éste era de hermoso parecer; y
habíalo engendrado después de Absalom. \bibverse{7} Y tenía tratos con
Joab hijo de Sarvia, y con Abiathar sacerdote, los cuales ayudaban á
Adonía. \bibverse{8} Mas Sadoc sacerdote, y Benaía hijo de Joiada, y
Nathán profeta, y Semei, y Reihi, y todos los grandes de David, no
seguían á Adonía. \bibverse{9} Y matando Adonía ovejas y vacas y
animales engordados junto á la peña de Zoheleth, que está cerca de la
fuente de Rogel, convidó á todos sus hermanos los hijos del rey, y á
todos los varones de Judá, siervos del rey: \bibverse{10} Mas no convidó
á Nathán profeta, ni á Benaía, ni á los grandes, ni á Salomón su
hermano. \bibverse{11} Y habló Nathán á Bath-sheba madre de Salomón,
diciendo: ¿No has oído que reina Adonía hijo de Haggith, sin saberlo
David nuestro señor? \bibverse{12} Ven pues ahora, y toma mi consejo,
para que guardes tu vida, y la vida de tu hijo Salomón. \bibverse{13}
Ve, y entra al rey David, y dile: Rey señor mío, ¿no has tú jurado á tu
sierva, diciendo: Salomón tu hijo reinará después de mí, y él se sentará
en mi trono? ¿por qué pues reina Adonía? \bibverse{14} Y estando tú aún
hablando con el rey, yo entraré tras ti, y acabaré tus razones.
\bibverse{15} Entonces Bath-sheba entró al rey á la cámara: y el rey era
muy viejo; y Abisag Sunamita servía al rey. \bibverse{16} Y Bath-sheba
se inclinó, é hizo reverencia al rey. Y el rey dijo: ¿Qué tienes?
\bibverse{17} Y ella le respondió: Señor mío, tú juraste á tu sierva por
Jehová tu Dios, diciendo: Salomón tu hijo reinará después de mí, y él se
sentará en mi trono; \bibverse{18} Y he aquí ahora Adonía reina: y tú,
mi señor rey, ahora no lo supiste. \bibverse{19} Ha matado bueyes, y
animales engordados, y muchas ovejas, y ha convidado á todos los hijos
del rey, y á Abiathar sacerdote, y á Joab general del ejército; mas á
Salomón tu siervo no ha convidado. \bibverse{20} Entre tanto, rey señor
mío, los ojos de todo Israel están sobre ti, para que les declares quién
se ha de sentar en el trono de mi señor el rey después de él.
\bibverse{21} De otra suerte acontecerá, cuando mi señor el rey durmiere
con sus padres, que yo y mi hijo Salomón seremos tenidos por culpables.
\bibverse{22} Y estando aún hablando ella con el rey, he aquí Nathán
profeta, que vino. \bibverse{23} Y dieron aviso al rey, diciendo: He
aquí Nathán profeta: el cual como entró al rey, postróse delante del rey
inclinando su rostro á tierra. \bibverse{24} Y dijo Nathán: Rey señor
mío, ¿has tú dicho: Adonía reinará después de mí, y él se sentará en mi
trono? \bibverse{25} Porque hoy ha descendido, y ha matado bueyes, y
animales engordados, y muchas ovejas, y ha convidado á todos los hijos
del rey, y á los capitanes del ejército, y también á Abiathar sacerdote;
y he aquí, están comiendo y bebiendo delante de él, y han dicho: ¡Viva
el rey Adonía! \bibverse{26} Mas ni á mí tu siervo, ni á Sadoc
sacerdote, ni á Benaía hijo de Joiada, ni á Salomón tu siervo, ha
convidado. \bibverse{27} ¿Es este negocio ordenado por mi señor el rey,
sin haber declarado á tu siervo quién se había de sentar en el trono de
mi señor el rey después de él? \bibverse{28} Entonces el rey David
respondió, y dijo: Llamadme á Bath-sheba. Y ella entró á la presencia
del rey, y púsose delante del rey. \bibverse{29} Y el rey juró,
diciendo: Vive Jehová, que ha redimido mi alma de toda angustia,
\bibverse{30} Que como yo te he jurado por Jehová Dios de Israel,
diciendo: Tu hijo Salomón reinará después de mí, y él se sentará en mi
trono en lugar mío; que así lo haré hoy. \bibverse{31} Entonces
Bath-sheba se inclinó al rey, su rostro á tierra, y haciendo reverencia
al rey, dijo: Viva mi señor el rey David para siempre. \bibverse{32} Y
el rey David dijo: Llamadme á Sadoc sacerdote, y á Nathán profeta, y á
Benaía hijo de Joiada. Y ellos entraron á la presencia del rey.
\bibverse{33} Y el rey les dijo: Tomad con vosotros los siervos de
vuestro señor, y haced subir á Salomón mi hijo en mi mula, y llevadlo á
Gihón: \bibverse{34} Y allí lo ungirán Sadoc sacerdote y Nathán profeta
por rey sobre Israel; y tocaréis trompeta, diciendo: ¡Viva el rey
Salomón! \bibverse{35} Después iréis vosotros detrás de él, y vendrá y
se sentará en mi trono, y él reinará por mí; porque á él he ordenado
para que sea príncipe sobre Israel y sobre Judá. \bibverse{36} Entonces
Benaía hijo de Joiada respondió al rey, y dijo: Amén. Así lo diga
Jehová, Dios de mi señor el rey. \bibverse{37} De la manera que Jehová
ha sido con mi señor el rey, así sea con Salomón; y él haga mayor su
trono que el trono de mi señor el rey David. \bibverse{38} Y descendió
Sadoc sacerdote, y Nathán profeta, y Benaía hijo de Joiada, y los
Ceretheos y los Peletheos, é hicieron subir á Salomón en la mula del rey
David, y lleváronlo á Gihón. \bibverse{39} Y tomando Sadoc sacerdote el
cuerno del aceite del tabernáculo, ungió á Salomón: y tocaron trompeta,
y dijo todo el pueblo: ¡Viva el rey Salomón! \bibverse{40} Después subió
todo el pueblo en pos de él, y cantaba la gente con flautas, y hacían
grandes alegrías, que parecía que la tierra se hundía con el clamor de
ellos. \bibverse{41} Y oyólo Adonía, y todos los convidados que con él
estaban, cuando ya habían acabado de comer. Y oyendo Joab el sonido de
la trompeta, dijo: ¿Por qué se alborota la ciudad con estruendo?
\bibverse{42} Estando aún él hablando, he aquí Jonathán hijo de Abiathar
sacerdote vino, al cual dijo Adonía: Entra, porque tú eres hombre de
esfuerzo, y traerás buenas nuevas. \bibverse{43} Y Jonathán respondió, y
dijo á Adonía: Ciertamente nuestro señor el rey David ha hecho rey á
Salomón: \bibverse{44} Y el rey ha enviado con él á Sadoc sacerdote y á
Nathán profeta, y á Benaía hijo de Joiada, y también á los Ceretheos y á
los Peletheos, los cuales le hicieron subir en la mula del rey;
\bibverse{45} Y Sadoc sacerdote y Nathán profeta lo han ungido en Gihón
por rey: y de allá han subido con alegrías, y la ciudad está llena de
estruendo. Este es el alboroto que habéis oído. \bibverse{46} Y también
Salomón se ha sentado en el trono del reino. \bibverse{47} Y aun los
siervos del rey han venido á bendecir á nuestro señor el rey David,
diciendo: Dios haga bueno el nombre de Salomón más que tu nombre, y haga
mayor su trono que el tuyo. Y el rey adoró en la cama. \bibverse{48} Y
también el rey habló así: Bendito sea Jehová Dios de Israel, que ha dado
hoy quien se siente en mi trono, viéndolo mis ojos. \bibverse{49} Ellos
entonces se estremecieron, y levantáronse todos los convidados que
estaban con Adonía, y fuése cada uno por su camino. \bibverse{50} Mas
Adonía, temiendo de la presencia de Salomón, levantóse y fuése, y cogió
los cornijales del altar. \bibverse{51} Y fué hecho saber á Salomón,
diciendo: He aquí que Adonía tiene miedo del rey Salomón: pues ha cogido
los cornijales del altar, diciendo: Júreme hoy el rey Salomón que no
matará á cuchillo á su siervo. \bibverse{52} Y Salomón dijo: Si él fuere
virtuoso, ni uno de sus cabellos caerá en tierra: mas si se hallare mal
en él, morirá. \bibverse{53} Y envió el rey Salomón, y trajéronlo del
altar; y él vino, é inclinóse al rey Salomón. Y Salomón le dijo: Vete á
tu casa.

\hypertarget{section-1}{%
\section{2}\label{section-1}}

\bibverse{1} Y llegáronse los días de David para morir, y mandó á
Salomón su hijo, diciendo: \bibverse{2} Yo voy el camino de toda la
tierra: esfuérzate, y sé varón. \bibverse{3} Guarda la ordenanza de
Jehová tu Dios, andando en sus caminos, y observando sus estatutos y
mandamientos, y sus derechos y sus testimonios, de la manera que está
escrito en la ley de Moisés, para que seas dichoso en todo lo que
hicieres, y en todo aquello á que te tornares; \bibverse{4} Para que
confirme Jehová la palabra que me habló, diciendo: Si tus hijos
guardaren su camino, andando delante de mí con verdad, de todo su
corazón, y de toda su alma, jamás, dice, faltará á ti varón del trono de
Israel. \bibverse{5} Y ya sabes tú lo que me ha hecho Joab hijo de
Sarvia, lo que hizo á dos generales del ejército de Israel, á Abner hijo
de Ner, y á Amasa hijo de Jether, los cuales él mató, derramando en paz
la sangre de guerra, y poniendo la sangre de guerra en su talabarte que
tenía sobre sus lomos, y en sus zapatos que tenía en sus pies.
\bibverse{6} Tú pues harás conforme á tu sabiduría; no dejarás descender
sus canas á la huesa en paz. \bibverse{7} Mas á los hijos de Barzillai
Galaadita harás misericordia, que sean de los convidados á tu mesa;
porque ellos vinieron así á mí, cuando iba huyendo de Absalom tu
hermano. \bibverse{8} También tienes contigo á Semei hijo de Gera, hijo
de Benjamín, de Bahurim, el cual me maldijo con una maldición fuerte el
día que yo iba á Mahanaim. Mas él mismo descendió á recibirme al Jordán,
y yo le juré por Jehová, diciendo: Yo no te mataré á cuchillo.
\bibverse{9} Empero ahora no lo absolverás: que hombre sabio eres, y
sabes cómo te has de haber con él: y harás descender sus canas con
sangre á la sepultura. \bibverse{10} Y David durmió con sus padres, y
fué sepultado en la ciudad de David. \bibverse{11} Los días que reinó
David sobre Israel fueron cuarenta años: siete años reinó en Hebrón, y
treinta y tres años reinó en Jerusalem. \bibverse{12} Y se sentó Salomón
en el trono de David su padre, y fué su reino firme en gran manera.
\bibverse{13} Entonces Adonía hijo de Haggith vino á Bath-sheba madre de
Salomón; y ella dijo: ¿Es tu venida de paz? Y él respondió: Sí, de paz.
\bibverse{14} En seguida dijo: Una palabra tengo que decirte. Y ella
dijo: Di. \bibverse{15} Y él dijo: Tú sabes que el reino era mío, y que
todo Israel había puesto en mí su rostro, para que yo reinara: mas el
reino fué traspasado, y vino á mi hermano; porque por Jehová era suyo.
\bibverse{16} Y ahora yo te hago una petición: no me hagas volver mi
rostro. Y ella le dijo: Habla. \bibverse{17} El entonces dijo: Yo te
ruego que hables al rey Salomón, (porque él no te hará volver tu
rostro,) para que me dé á Abisag Sunamita por mujer. \bibverse{18} Y
Bath-sheba dijo: Bien; yo hablaré por ti al rey. \bibverse{19} Y vino
Bath-sheba al rey Salomón para hablarle por Adonía. Y el rey se levantó
á recibirla, é inclinóse á ella, y volvió á sentarse en su trono, é hizo
poner una silla á la madre del rey, la cual se sentó á su diestra.
\bibverse{20} Y ella dijo: Una pequeña petición pretendo de ti; no me
hagas volver mi rostro. Y el rey le dijo: Pide, madre mía, que yo no te
haré volver el rostro. \bibverse{21} Y ella dijo: Dése Abisag Sunamita
por mujer á tu hermano Adonía. \bibverse{22} Y el rey Salomón respondió,
y dijo á su madre: ¿Por qué pides á Abisag Sunamita para Adonía? Demanda
también para él el reino, porque él es mi hermano mayor; y tiene también
á Abiathar sacerdote, y á Joab hijo de Sarvia. \bibverse{23} Y el rey
Salomón juró por Jehová, diciendo: Así me haga Dios y así me añada, que
contra su vida ha hablado Adonía esta palabra. \bibverse{24} Ahora pues,
vive Jehová, que me ha confirmado y me ha puesto sobre el trono de David
mi padre, y que me ha hecho casa, como me había dicho, que Adonía morirá
hoy. \bibverse{25} Entonces el rey Salomón envió por mano de Benaía hijo
de Joiada, el cual dió sobre él, y murió. \bibverse{26} Y á Abiathar
sacerdote dijo el rey: Vete á Anathoth á tus heredades, que tú eres
digno de muerte; mas no te mataré hoy, por cuanto has llevado el arca
del Señor Jehová delante de David mi padre, y además has sido trabajado
en todas las cosas en que fué trabajado mi padre. \bibverse{27} Así echó
Salomón á Abiathar del sacerdocio de Jehová, para que se cumpliese la
palabra de Jehová que había dicho sobre la casa de Eli en Silo.
\bibverse{28} Y vino la noticia hasta Joab: porque también Joab se había
adherido á Adonía, si bien no se había adherido á Absalom. Y huyó Joab
al tabernáculo de Jehová, y asióse á los cornijales del altar.
\bibverse{29} Y fué hecho saber á Salomón que Joab había huído al
tabernáculo de Jehová, y que estaba junto al altar. Entonces envió
Salomón á Benaía hijo de Joiada, diciendo: Ve, y da sobre él.
\bibverse{30} Y entró Benaía al tabernáculo de Jehová, y díjole: El rey
ha dicho que salgas. Y él dijo: No, sino aquí moriré. Y Benaía volvió
con esta respuesta al rey, diciendo: Así habló Joab, y así me respondió.
\bibverse{31} Y el rey le dijo: Haz como él ha dicho; mátale y
entiérralo, y quita de mí y de la casa de mi padre la sangre que Joab ha
derramado injustamente. \bibverse{32} Y Jehová hará tornar su sangre
sobre su cabeza: que él ha muerto dos varones más justos y mejores que
él, á los cuales mató á cuchillo sin que mi padre David supiese nada: á
Abner hijo de Ner, general del ejército de Israel, y á Amasa hijo de
Jether, general de ejército de Judá. \bibverse{33} La sangre pues de
ellos recaerá sobre la cabeza de Joab, y sobre la cabeza de su simiente
para siempre: mas sobre David y sobre su simiente, y sobre su casa y
sobre su trono, habrá perpetuamente paz de parte de Jehová.
\bibverse{34} Entonces Benaía hijo de Joiada subió, y dió sobre él, y
matólo; y fué sepultado en su casa en el desierto. \bibverse{35} Y el
rey puso en su lugar á Benaía hijo de Joiada sobre el ejército: y á
Sadoc puso el rey por sacerdote en lugar de Abiathar. \bibverse{36}
Después envió el rey, é hizo venir á Semei, y díjole: Edifícate una casa
en Jerusalem, y mora ahí, y no salgas de allá á una parte ni á otra;
\bibverse{37} Porque sabe de cierto que el día que salieres, y pasares
el torrente de Cedrón, sin duda morirás, y tu sangre será sobre tu
cabeza. \bibverse{38} Y Semei dijo al rey: La palabra es buena; como el
rey mi señor ha dicho, así lo hará tu siervo. Y habitó Semei en
Jerusalem muchos días. \bibverse{39} Pero pasados tres años, aconteció
que se le huyeron á Semei dos siervos á Achîs, hijo de Maachâ, rey de
Gath. Y dieron aviso á Semei, diciendo: He aquí que tus siervos están en
Gath. \bibverse{40} Levantóse entonces Semei, y enalbardó su asno, y fué
á Gath, á Achîs, á procurar sus siervos. Fué pues Semei, y volvió sus
siervos de Gath. \bibverse{41} Díjose luego á Salomón como Semei había
ido de Jerusalem hasta Gath, y que había vuelto. \bibverse{42} Entonces
el rey envió, é hizo venir á Semei, y díjole: ¿No te conjuré yo por
Jehová, y te protesté, diciendo: El día que salieres, y fueres acá ó
acullá, sabe de cierto que has de morir? Y tú me dijiste: La palabra es
buena, yo la obedezco. \bibverse{43} ¿Por qué pues no guardaste el
juramento de Jehová, y el mandamiento que yo te impuse? \bibverse{44}
Dijo además el rey á Semei: Tú sabes todo el mal, el cual tu corazón
bien sabe, que cometiste contra mi padre David; Jehová pues, ha tornado
el mal sobre tu cabeza. \bibverse{45} Y el rey Salomón será bendito, y
el trono de David será firme perpetuamente delante de Jehová.
\bibverse{46} Entonces el rey mandó á Benaía hijo de Joiada, el cual
salió é hirióle; y murió. Y el reino fué confirmado en la mano de
Salomón.

\hypertarget{section-2}{%
\section{3}\label{section-2}}

\bibverse{1} Y salomón hizo parentesco con Faraón rey de Egipto, porque
tomó la hija de Faraón, y trájola á la ciudad de David, entre tanto que
acababa de edificar su casa, y la casa de Jehová, y los muros de
Jerusalem alrededor. \bibverse{2} Hasta entonces el pueblo sacrificaba
en los altos; porque no había casa edificada al nombre de Jehová hasta
aquellos tiempos. \bibverse{3} Mas Salomón amó á Jehová, andando en los
estatutos de su padre David: solamente sacrificaba y quemaba perfumes en
los altos. \bibverse{4} E iba el rey á Gabaón, porque aquél era el alto
principal, y sacrificaba allí: mil holocaustos sacrificaba Salomón sobre
aquel altar. \bibverse{5} Y aparecióse Jehová á Salomón en Gabaón una
noche en sueños, y díjole Dios: Pide lo que quisieres que yo te dé.
\bibverse{6} Y Salomón dijo: Tú hiciste gran misericordia á tu siervo
David mi padre, según que él anduvo delante de ti en verdad, en
justicia, y con rectitud de corazón para contigo: y tú le has guardado
esta tu grande misericordia, que le diste hijo que se sentase en su
trono, como sucede en este día. \bibverse{7} Ahora pues, Jehová Dios
mío, tú has puesto á mí tu siervo por rey en lugar de David mi padre: y
yo soy mozo pequeño, que no sé cómo entrar ni salir. \bibverse{8} Y tu
siervo está en medio de tu pueblo al cual tú escogiste; un pueblo
grande, que no se puede contar ni numerar por su multitud. \bibverse{9}
Da pues á tu siervo corazón dócil para juzgar á tu pueblo, para
discernir entre lo bueno y lo malo: porque ¿quién podrá gobernar este tu
pueblo tan grande? \bibverse{10} Y agradó delante de Adonai que Salomón
pidiese esto. \bibverse{11} Y díjole Dios: Porque has demandado esto, y
no pediste para ti muchos días, ni pediste para ti riquezas, ni pediste
la vida de tus enemigos, mas demandaste para ti inteligencia para oir
juicio; \bibverse{12} He aquí lo he hecho conforme á tus palabras: he
aquí que te he dado corazón sabio y entendido, tanto que no haya habido
antes de ti otro como tú, ni después de ti se levantará otro como tú.
\bibverse{13} Y aun también te he dado las cosas que no pediste,
riquezas y gloria: tal, que entre los reyes ninguno haya como tú en
todos tus días. \bibverse{14} Y si anduvieres en mis caminos, guardando
mis estatutos y mis mandamientos, como anduvo David tu padre, yo
alargaré tus días. \bibverse{15} Y como Salomón despertó, vió que era
sueño: y vino á Jerusalem, y presentóse delante del arca del pacto de
Jehová, y sacrificó holocaustos, é hizo pacíficos; hizo también banquete
á todos sus siervos. \bibverse{16} En aquella sazón vinieron dos mujeres
rameras al rey, y presentáronse delante de él. \bibverse{17} Y dijo la
una mujer: ¡Ah, señor mío! yo y esta mujer morábamos en una misma casa,
y yo parí estando con ella en la casa. \bibverse{18} Y aconteció al
tercer día después que yo parí, que ésta parió también, y morábamos
nosotras juntas; ninguno de fuera estaba en casa, sino nosotras dos en
la casa. \bibverse{19} Y una noche el hijo de esta mujer murió, porque
ella se acostó sobre él. \bibverse{20} Y levantóse á media noche, y tomó
á mi hijo de junto á mí, estando yo tu sierva durmiendo, y púsolo á su
lado, y púsome á mi lado su hijo muerto. \bibverse{21} Y como yo me
levanté por la mañana para dar el pecho á mi hijo, he aquí que estaba
muerto: mas observéle por la mañana, y vi que no era mi hijo, que yo
había parido. \bibverse{22} Entonces la otra mujer dijo: No; mi hijo es
el que vive, y tu hijo es el muerto. Y la otra volvió á decir: No; tu
hijo es el muerto, y mi hijo es el que vive. Así hablaban delante del
rey. \bibverse{23} El rey entonces dijo: Esta dice: Mi hijo es el que
vive, y tu hijo es el muerto: y la otra dice, No, mas el tuyo es el
muerto, y mi hijo es el que vive. \bibverse{24} Y dijo el rey: Traedme
un cuchillo. Y trajeron al rey un cuchillo. \bibverse{25} En seguida el
rey dijo: Partid por medio el niño vivo, y dad la mitad á la una, y la
otra mitad á la otra. \bibverse{26} Entonces la mujer cuyo era el hijo
vivo, habló al rey (porque sus entrañas se le conmovieron por su hijo),
y dijo: ¡Ah, señor mío! dad á ésta el niño vivo, y no lo matéis. Mas la
otra dijo: Ni á mí ni á ti; partidlo. \bibverse{27} Entonces el rey
respondió, y dijo: Dad á aquélla el hijo vivo, y no lo matéis: ella es
su madre. \bibverse{28} Y todo Israel oyó aquel juicio que había dado el
rey: y temieron al rey, porque vieron que había en él sabiduría de Dios
para juzgar.

\hypertarget{section-3}{%
\section{4}\label{section-3}}

\bibverse{1} Fué pues el rey Salomón rey sobre todo Israel. \bibverse{2}
Y estos fueron los príncipes que tuvo: Azarías hijo de Sadoc, sacerdote;
\bibverse{3} Elioreph y Ahía, hijos de Sisa, escribas; Josaphat hijo de
Ahilud, canciller; \bibverse{4} Benaía hijo de Joiada era sobre el
ejército; y Sadoc y Abiathar eran los sacerdotes; \bibverse{5} Azaría
hijo de Nathán era sobre los gobernadores; Zabud hijo de Nathán era
principal oficial, amigo del rey; \bibverse{6} Y Ahisar era mayordomo; y
Adoniram hijo de Abda era sobre el tributo. \bibverse{7} Y tenía Salomón
doce gobernadores sobre todo Israel, los cuales mantenían al rey y á su
casa. Cada uno de ellos estaba obligado á abastecer por un mes en el
año. \bibverse{8} Y estos son los nombres de ellos: el hijo de Hur en el
monte de Ephraim; \bibverse{9} El hijo de Decar, en Maccas, y en
Saalbim, y en Beth-semes, y en Elón, y en Beth-hanan; \bibverse{10} El
hijo de Hesed, en Aruboth; éste tenía también á Sochô y toda la tierra
de Ephet. \bibverse{11} El hijo de Abinadab, en todos los términos de
Dor: éste tenía por mujer á Thaphat hija de Salomón; \bibverse{12} Baana
hijo de Ahilud, en Taanach y Megiddo, y en toda Beth-san, que es cerca
de Zaretán, por bajo de Jezreel, desde Beth-san hasta Abel-mehola, y
hasta la otra parte de Jocmeam; \bibverse{13} El hijo de Geber, en
Ramoth de Galaad; éste tenía también las ciudades de Jair hijo de
Manasés, las cuales estaban en Galaad; tenía también la provincia de
Argob, que era en Basán, sesenta grandes ciudades con muro y cerraduras
de bronce; \bibverse{14} Ahinadab hijo de Iddo, en Mahanaim;
\bibverse{15} Ahimaas en Nephtalí; éste tomó también por mujer á
Basemath hija de Salomón. \bibverse{16} Baana hijo de Husai, en Aser y
en Aloth; \bibverse{17} Josaphat hijo de Pharua, en Issachâr;
\bibverse{18} Semei hijo de Ela, en Benjamín; \bibverse{19} Geber hijo
de Uri, en la tierra de Galaad, la tierra de Sehón rey de los Amorrheos,
y de Og rey de Basán; éste era el único gobernador en aquella tierra.
\bibverse{20} Judá é Israel eran muchos, como la arena que está junto á
la mar en multitud, comiendo y bebiendo y alegrándose. \bibverse{21} Y
Salomón señoreaba sobre todos los reinos, desde el río de la tierra de
los Filisteos hasta el término de Egipto: y traían presentes, y
sirvieron á Salomón todos los días que vivió. \bibverse{22} Y la
despensa de Salomón era cada día treinta coros de flor de harina, y
sesenta coros de harina. \bibverse{23} Diez bueyes engordados, y veinte
bueyes de pasto, y cien ovejas; sin los ciervos, cabras, búfalos, y aves
engordadas. \bibverse{24} Porque él señoreaba en toda la región que
estaba de la otra parte del río, desde Tiphsa hasta Gaza, sobre todos
los reyes de la otra parte del río; y tuvo paz por todos lados en
derredor suyo. \bibverse{25} Y Judá é Israel vivían seguros, cada uno
debajo de su parra y debajo de su higuera, desde Dan hasta Beer-seba,
todos los días de Salomón. \bibverse{26} Tenía además de esto Salomón
cuarenta mil caballos en sus caballerizas para sus carros, y doce mil
jinetes. \bibverse{27} Y estos gobernadores mantenían al rey Salomón, y
á todos los que á la mesa del rey Salomón venían, cada uno un mes; y
hacían que nada faltase. \bibverse{28} Hacían también traer cebada y
paja para los caballos y para las bestias de carga, al lugar donde él
estaba, cada uno conforme al cargo que tenía. \bibverse{29} Y dió Dios á
Salomón sabiduría, y prudencia muy grande, y anchura de corazón como la
arena que está á la orilla del mar. \bibverse{30} Que fué mayor la
sabiduría de Salomón que la de todos los orientales, y que toda la
sabiduría de los Egipcios. \bibverse{31} Y aun fué más sabio que todos
los hombres; más que Ethán Ezrahita, y que Emán y Calchôl y Darda, hijos
de Mahol: y fué nombrado entre todas las naciones de alrededor.
\bibverse{32} Y propuso tres mil parábolas; y sus versos fueron mil y
cinco. \bibverse{33} También disertó de los árboles, desde el cedro del
Líbano hasta el hisopo que nace en la pared. Asimismo disertó de los
animales, de las aves, de los reptiles, y de los peces. \bibverse{34} Y
venían de todos los pueblos á oir la sabiduría de Salomón, y de todos
los reyes de la tierra, donde había llegado la fama de su sabiduría.

\hypertarget{section-4}{%
\section{5}\label{section-4}}

\bibverse{1} Hiram rey de Tiro envió también sus siervos á Salomón,
luego que oyó que lo habían ungido por rey en lugar de su padre: porque
Hiram había siempre amado á David. \bibverse{2} Entonces Salomón envió á
decir á Hiram: \bibverse{3} Tú sabes como mi padre David no pudo
edificar casa al nombre de Jehová su Dios, por las guerras que le
cercaron, hasta que Jehová puso sus enemigos bajo las plantas de sus
pies. \bibverse{4} Ahora Jehová mi Dios me ha dado reposo por todas
partes; que ni hay adversarios, ni mal encuentro. \bibverse{5} Yo por
tanto he determinado ahora edificar casa al nombre de Jehová mi Dios,
como Jehová lo habló á David mi padre, diciendo: Tu hijo, que yo pondré
en lugar tuyo en tu trono, él edificará casa á mi nombre. \bibverse{6}
Manda pues ahora que me corten cedros del Líbano; y mis siervos estarán
con los tuyos, y yo te daré por tus siervos el salario que tú dijeres:
porque tú sabes bien que ninguno hay entre nosotros que sepa labrar la
madera como los Sidonios. \bibverse{7} Y como Hiram oyó las palabras de
Salomón, holgóse en gran manera, y dijo: Bendito sea hoy Jehová, que dió
hijo sabio á David sobre este pueblo tan grande. \bibverse{8} Y envió
Hiram á decir á Salomón: He oído lo que me mandaste á decir: yo haré
todo lo que te pluguiere acerca de la madera de cedro, y la madera de
haya. \bibverse{9} Mis siervos la llevarán desde el Líbano á la mar; y
yo la pondré en balsas por la mar hasta el lugar que tú me señalares, y
allí se desatará, y tú la tomarás: y tú harás mi voluntad en dar de
comer á mi familia. \bibverse{10} Dió pues Hiram á Salomón madera de
cedro y madera de haya todo lo que quiso. \bibverse{11} Y Salomón daba á
Hiram veinte mil coros de trigo para el sustento de su familia, y veinte
coros de aceite limpio: esto daba Salomón á Hiram cada un año.
\bibverse{12} Dió pues Jehová á Salomón sabiduría como le había dicho: y
hubo paz entre Hiram y Salomón, é hicieron alianza entre ambos.
\bibverse{13} Y el rey Salomón impuso tributo á todo Israel, y el
tributo fué de treinta mil hombres: \bibverse{14} Los cuales enviaba al
Líbano de diez mil en diez mil, cada mes por su turno, viniendo así á
estar un mes en el Líbano, y dos meses en sus casas: y Adoniram estaba
sobre aquel tributo. \bibverse{15} Tenía también Salomón setenta mil que
llevaban las cargas, y ochenta mil cortadores en el monte; \bibverse{16}
Sin los principales oficiales de Salomón que estaban sobre la obra, tres
mil y trescientos, los cuales tenían cargo del pueblo que hacía la obra.
\bibverse{17} Y mandó el rey que trajesen grandes piedras, piedras de
precio, para los cimientos de la casa, y piedras labradas. \bibverse{18}
Y los albañiles de Salomón y los de Hiram, y los aparejadores, cortaron
y aparejaron la madera y la cantería para labrar la casa.

\hypertarget{section-5}{%
\section{6}\label{section-5}}

\bibverse{1} Y fué en el año cuatrocientos ochenta después que los hijos
de Israel salieron de Egipto, en el cuarto año del principio del reino
de Salomón sobre Israel, en el mes de Ziph, que es el mes segundo, que
él comenzó á edificar la casa de Jehová. \bibverse{2} La casa que el rey
Salomón edificó á Jehová, tuvo sesenta codos de largo y veinte de ancho,
y treinta codos de alto. \bibverse{3} Y el pórtico delante del templo de
la casa, de veinte codos de largo, según la anchura de la casa, y su
ancho era de diez codos delante de la casa. \bibverse{4} E hizo á la
casa ventanas anchas por de dentro, y estrechas por de fuera.
\bibverse{5} Edificó también junto al muro de la casa aposentos
alrededor, contra las paredes de la casa en derredor del templo y del
oráculo: é hizo cámaras alrededor. \bibverse{6} El aposento de abajo era
de cinco codos de ancho, y el de en medio de seis codos de ancho, y el
tercero de siete codos de ancho: porque por de fuera había hecho
disminuciones á la casa en derredor, para no trabar las vigas de las
paredes de la casa. \bibverse{7} Y la casa cuando se edificó,
fabricáronla de piedras que traían ya acabadas; de tal manera que cuando
la edificaban, ni martillos ni hachas se oyeron en la casa, ni ningún
otro instrumento de hierro. \bibverse{8} La puerta del aposento de en
medio estaba al lado derecho de la casa: y subíase por un caracol al de
en medio, y del aposento de en medio al tercero. \bibverse{9} Labró pues
la casa, y acabóla; y cubrió la casa con artesonados de cedro.
\bibverse{10} Y edificó asimismo el aposento en derredor de toda la
casa, de altura de cinco codos, el cual se apoyaba en la casa con
maderas de cedro. \bibverse{11} Y fué palabra de Jehová á Salomón,
diciendo: \bibverse{12} Esta casa que tú edificas, si anduvieres en mis
estatutos, é hicieres mis derechos, y guardares todos mis mandamientos
andando en ellos, yo tendré firme contigo mi palabra que hablé á David
tu padre; \bibverse{13} Y habitaré en medio de los hijos de Israel, y no
dejaré á mi pueblo Israel. \bibverse{14} Así que, Salomón labró la casa,
y acabóla. \bibverse{15} Y aparejó las paredes de la casa por de dentro
con tablas de cedro, vistiéndola de madera por dentro, desde el solado
de la casa hasta las paredes de la techumbre: cubrió también el
pavimento con madera de haya. \bibverse{16} Asimismo hizo al cabo de la
casa un edificio de veinte codos de tablas de cedro, desde el solado
hasta lo más alto; y fabricóse en la casa un oráculo, que es el lugar
santísimo. \bibverse{17} Y la casa, á saber, el templo de dentro, tenía
cuarenta codos. \bibverse{18} Y la casa estaba cubierta de cedro por de
dentro, y tenía entalladuras de calabazas silvestres y de botones de
flores. Todo era cedro; ninguna piedra se veía. \bibverse{19} Y adornó
el oráculo por de dentro en medio de la casa, para poner allí el arca
del pacto de Jehová. \bibverse{20} Y el oráculo estaba en la parte de
adentro, el cual tenía veinte codos de largo, y otros veinte de ancho, y
otros veinte de altura; y vistiólo de oro purísimo: asimismo cubrió el
altar de cedro. \bibverse{21} De suerte que vistió Salomón de oro puro
la casa por de dentro, y cerró la entrada del oráculo con cadenas de
oro, y vistiólo de oro. \bibverse{22} Cubrió pues de oro toda la casa
hasta el cabo; y asimismo vistió de oro todo el altar que estaba delante
del oráculo. \bibverse{23} Hizo también en el oráculo dos querubines de
madera de oliva, cada uno de altura de diez codos. \bibverse{24} La una
ala del querubín tenía cinco codos, y la otra ala del querubín otros
cinco codos: así que había diez codos desde la punta de la una ala hasta
la punta de la otra. \bibverse{25} Asimismo el otro querubín tenía diez
codos; porque ambos querubines eran de un tamaño y de una hechura.
\bibverse{26} La altura del uno era de diez codos, y asimismo el otro.
\bibverse{27} Y puso estos querubines dentro de la casa de adentro: los
cuales querubines extendían sus alas, de modo que el ala del uno tocaba
á la pared, y el ala del otro querubín tocaba á la otra pared, y las
otras dos alas se tocaban la una á la otra en la mitad de la casa.
\bibverse{28} Y vistió de oro los querubines. \bibverse{29} Y esculpió
todas las paredes de la casa alrededor de diversas figuras, de
querubines, de palmas, y de botones de flores, por de dentro y por de
fuera. \bibverse{30} Y cubrió de oro el piso de la casa, de dentro y de
fuera. \bibverse{31} Y á la entrada del oráculo hizo puertas de madera
de oliva; y el umbral y los postes eran de cinco esquinas. \bibverse{32}
Las dos puertas eran de madera de oliva; y entalló en ellas figuras de
querubines y de palmas y de botones de flores, y cubriólas de oro:
cubrió también de oro los querubines y las palmas. \bibverse{33}
Igualmente hizo á la puerta del templo postes de madera de oliva
cuadrados. \bibverse{34} Pero las dos puertas eran de madera de haya; y
los dos lados de la una puerta eran redondos, y los otros dos lados de
la otra puerta también redondos. \bibverse{35} Y entalló en ellas
querubines y palmas y botones de flores, y cubriólas de oro ajustado á
las entalladuras. \bibverse{36} Y edificó el atrio interior de tres
órdenes de piedras labradas, y de un orden de vigas de cedro.
\bibverse{37} En el cuarto año, en el mes de Ziph, se echaron los
cimientos de la casa de Jehová: \bibverse{38} Y en el undécimo año, en
el mes de Bul, que es el mes octavo, fué acabada la casa con todas sus
pertenencias, y con todo lo necesario. Edificóla pues, en siete años.

\hypertarget{section-6}{%
\section{7}\label{section-6}}

\bibverse{1} Después edificó Salomón su propia casa en trece años, y
acabóla toda. \bibverse{2} Asimismo edificó la casa del bosque del
Líbano, la cual tenía cien codos de longitud, y cincuenta codos de
anchura, y treinta codos de altura, sobre cuatro órdenes de columnas de
cedro, con vigas de cedro sobre las columnas. \bibverse{3} Y estaba
cubierta de tablas de cedro arriba sobre las vigas, que se apoyaban en
cuarenta y cinco columnas: cada hilera tenía quince columnas.
\bibverse{4} Y había tres órdenes de ventanas, una ventana contra la
otra en tres órdenes. \bibverse{5} Y todas las puertas y postes eran
cuadrados: y las unas ventanas estaban frente á las otras en tres
órdenes. \bibverse{6} También hizo un pórtico de columnas, que tenía de
largo cincuenta codos, y treinta codos de ancho; y aqueste pórtico
estaba delante de aquellas otras, con sus columnas y maderos
correspondientes. \bibverse{7} Hizo asimismo el pórtico del trono en que
había de juzgar, el pórtico del juicio, y vistiólo de cedro de suelo á
suelo. \bibverse{8} Y en la casa en que él moraba, había otro atrio
dentro del pórtico, de obra semejante á esta. Edificó también Salomón
una casa para la hija de Faraón, que había tomado por mujer, de la misma
obra de aquel pórtico. \bibverse{9} Todas aquellas obras fueron de
piedras de precio, cortadas y aserradas con sierras según las medidas,
así por de dentro como por de fuera, desde el cimiento hasta los
remates, y asimismo por de fuera hasta el gran atrio. \bibverse{10} El
cimiento era de piedras de precio, de piedras grandes, de piedras de
diez codos, y de piedras de ocho codos. \bibverse{11} De allí arriba
eran también piedras de precio, labradas conforme á sus medidas, y obra
de cedro. \bibverse{12} Y en el gran atrio alrededor había tres órdenes
de piedras labradas, y un orden de vigas de cedro: y así el atrio
interior de la casa de Jehová, y el atrio de la casa. \bibverse{13} Y
envió el rey Salomón, é hizo venir de Tiro á Hiram, \bibverse{14} Hijo
de una viuda de la tribu de Nephtalí, y su padre había sido de Tiro:
trabajaba él en bronce, lleno de sabiduría y de inteligencia y saber en
toda obra de metal. Este pues vino al rey Salomón, é hizo toda su obra.
\bibverse{15} Y vació dos columnas de bronce, la altura de cada cual era
de diez y ocho codos: y rodeaba á una y á otra columna un hilo de doce
codos. \bibverse{16} Hizo también dos capiteles de fundición de bronce,
para que fuesen puestos sobre las cabezas de las columnas: la altura de
un capitel era de cinco codos, y la del otro capitel de cinco codos.
\bibverse{17} Había trenzas á manera de red, y unas cintas á manera de
cadenas, para los capiteles que se habían de poner sobre las cabezas de
las columnas: siete para cada capitel. \bibverse{18} Y cuando hubo hecho
las columnas, hizo también dos órdenes de granadas alrededor en el un
enredado, para cubrir los capiteles que estaban en las cabezas de las
columnas con las granadas: y de la misma forma hizo en el otro capitel.
\bibverse{19} Los capiteles que estaban sobre las columnas en el
pórtico, tenían labor de flores por cuatro codos. \bibverse{20} Tenían
también los capiteles de sobre las dos columnas, doscientas granadas en
dos órdenes alrededor en cada capitel, encima del vientre del capitel,
el cual vientre estaba delante del enredado. \bibverse{21} Estas
columnas erigió en el pórtico del templo: y cuando hubo alzado la
columna de la mano derecha, púsole por nombre Jachîn: y alzando la
columna de la mano izquierda, llamó su nombre Boaz. \bibverse{22} Y puso
en las cabezas de las columnas labor en forma de azucenas; y así se
acabó la obra de las columnas. \bibverse{23} Hizo asimismo un mar de
fundición, de diez codos del un lado al otro, perfectamente redondo: su
altura era de cinco codos, y ceñíalo alrededor un cordón de treinta
codos. \bibverse{24} Y cercaban aquel mar por debajo de su labio en
derredor unas bolas como calabazas, diez en cada codo, que ceñían el mar
alrededor en dos órdenes, las cuales habían sido fundidas cuando él fué
fundido. \bibverse{25} Y estaba asentado sobre doce bueyes: tres miraban
al norte, y tres miraban al poniente, y tres miraban al mediodía, y tres
miraban al oriente; sobre éstos se apoyaba el mar, y las traseras de
ellos estaban hacia la parte de adentro. \bibverse{26} El grueso del mar
era de un palmo, y su labio era labrado como el labio de un cáliz, ó de
flor de lis: y cabían en él dos mil batos. \bibverse{27} Hizo también
diez basas de bronce, siendo la longitud de cada basa de cuatro codos, y
la anchura de cuatro codos, y de tres codos la altura. \bibverse{28} La
obra de las basas era esta: tenían unas cintas, las cuales estaban entre
molduras: \bibverse{29} Y sobre aquellas cintas que estaban entre las
molduras, figuras de leones, y de bueyes, y de querubines; y sobre las
molduras de la basa, así encima como debajo de los leones y de los
bueyes, había unas añadiduras de bajo relieve. \bibverse{30} Cada basa
tenía cuatro ruedas de bronce con mesas de bronce; y en sus cuatro
esquinas había unos hombrillos, los cuales nacían de fundición á cada un
lado de aquellas añadiduras, para estar debajo de la fuente.
\bibverse{31} Y la boca del pie de la fuente entraba un codo en el
remate que salía para arriba de la basa; y era su boca redonda, de la
hechura del mismo remate, y éste de codo y medio. Había también sobre la
boca entalladuras con sus cintas, las cuales eran cuadradas, no
redondas. \bibverse{32} Las cuatro ruedas estaban debajo de las cintas,
y los ejes de las ruedas nacían en la misma basa. La altura de cada
rueda era de un codo y medio. \bibverse{33} Y la hechura de las ruedas
era como la hechura de las ruedas de un carro: sus ejes, sus rayos, y
sus cubos, y sus cinchos, todo era de fundición. \bibverse{34} Asimismo
los cuatro hombrillos á las cuatro esquinas de cada basa: y los
hombrillos eran de la misma basa. \bibverse{35} Y en lo alto de la basa
había medio codo de altura redondo por todas partes: y encima de la basa
sus molduras y cintas, las cuales eran de ella misma. \bibverse{36} E
hizo en las tablas de las molduras, y en las cintas, entalladuras de
querubines, y de leones, y de palmas, con proporción en el espacio de
cada una, y alrededor otros adornos. \bibverse{37} De esta forma hizo
diez basas fundidas de una misma manera, de una misma medida, y de una
misma entalladura. \bibverse{38} Hizo también diez fuentes de bronce:
cada fuente contenía cuarenta batos, y cada una era de cuatro codos; y
asentó una fuente sobre cada una de las diez basas. \bibverse{39} Y puso
las cinco basas á la mano derecha de la casa, y las otras cinco á la
mano izquierda: y asentó el mar al lado derecho de la casa, al oriente,
hacia el mediodía. \bibverse{40} Asimismo hizo Hiram fuentes, y tenazas,
y cuencos. Así acabó toda la obra que hizo á Salomón para la casa de
Jehová: \bibverse{41} Es á saber, dos columnas, y los vasos redondos de
los capiteles que estaban en lo alto de las dos columnas; y dos redes
que cubrían los dos vasos redondos de los capiteles que estaban sobre la
cabeza de las columnas; \bibverse{42} Y cuatrocientas granadas para las
dos redes, dos órdenes de granadas en cada red, para cubrir los dos
vasos redondos que estaban sobre las cabezas de las columnas;
\bibverse{43} Y las diez basas, y las diez fuentes sobre las basas;
\bibverse{44} Y un mar, y doce bueyes debajo del mar; \bibverse{45} Y
calderos, y paletas, y cuencos; y todos los vasos que Hiram hizo al rey
Salomón, para la casa de Jehová, de metal acicalado. \bibverse{46} Todo
lo hizo fundir el rey en la llanura del Jordán, en tierra arcillosa,
entre Succoth y Sarthán. \bibverse{47} Y dejó Salomón sin inquirir el
peso del metal de todos los vasos, por la grande multitud de ellos.
\bibverse{48} Entonces hizo Salomón todos los vasos que pertenecían á la
casa de Jehová: un altar de oro, y una mesa sobre la cual estaban los
panes de la proposición, también de oro; \bibverse{49} Y cinco
candeleros de oro purísimo á la mano derecha, y otros cinco á la
izquierda, delante del oráculo; con las flores, y las lámparas, y
despabiladeras de oro; \bibverse{50} Asimismo los cántaros, vasos,
tazas, cucharillas, é incensarios, de oro purísimo; también de oro los
quiciales de las puertas de la casa de adentro, del lugar santísimo, y
los de las puertas del templo. \bibverse{51} Así se acabó toda la obra
que dispuso hacer el rey Salomón para la casa de Jehová. Y metió Salomón
lo que David su padre había dedicado, es á saber, plata, y oro, y vasos,
y púsolo todo en guarda en las tesorerías de la casa de Jehová.

\hypertarget{section-7}{%
\section{8}\label{section-7}}

\bibverse{1} Entonces juntó Salomón los ancianos de Israel, y á todas
las cabezas de las tribus, y á los príncipes de las familias de los
hijos de Israel, al rey Salomón en Jerusalem para traer el arca del
pacto de Jehová de la ciudad de David, que es Sión. \bibverse{2} Y se
juntaron al rey Salomón todos los varones de Israel en el mes de
Ethanim, que es el mes séptimo, en el día solemne. \bibverse{3} Y
vinieron todos los ancianos de Israel, y los sacerdotes tomaron el arca.
\bibverse{4} Y llevaron el arca de Jehová, y el tabernáculo del
testimonio, y todos los vasos sagrados que estaban en el tabernáculo;
los cuales llevaban los sacerdotes y Levitas. \bibverse{5} Y el rey
Salomón, y toda la congregación de Israel que á él se había juntado,
estaban con él delante del arca, sacrificando ovejas y bueyes, que por
la multitud no se podían contar ni numerar. \bibverse{6} Y los
sacerdotes metieron el arca del pacto de Jehová en su lugar, en el
oráculo de la casa, en el lugar santísimo, debajo de las alas de los
querubines. \bibverse{7} Porque los querubines tenían extendidas las
alas sobre el lugar del arca, y así cubrían los querubines el arca y sus
varas por encima. \bibverse{8} E hicieron salir las varas; que las
cabezas de las varas se dejaban ver desde el santuario delante del
oráculo, mas no se veían desde afuera: y así se quedaron hasta hoy.
\bibverse{9} En el arca ninguna cosa había más de las dos tablas de
piedra que había allí puesto Moisés en Horeb, donde Jehová hizo la
alianza con los hijos de Israel, cuando salieron de la tierra de Egipto.
\bibverse{10} Y como los sacerdotes salieron del santuario, la nube
hinchió la casa de Jehová. \bibverse{11} Y los sacerdotes no pudieron
estar para ministrar por causa de la nube; porque la gloria de Jehová
había henchido la casa de Jehová. \bibverse{12} Entonces dijo Salomón:
Jehová ha dicho que él habitaría en la oscuridad. \bibverse{13} Yo he
edificado casa por morada para ti, asiento en que tú habites para
siempre. \bibverse{14} Y volviendo el rey su rostro, bendijo á toda la
congregación de Israel; y toda la congregación de Israel estaba en pie.
\bibverse{15} Y dijo: Bendito sea Jehová Dios de Israel, que habló de su
boca á David mi padre, y con su mano lo ha cumplido, diciendo:
\bibverse{16} Desde el día que saqué mi pueblo Israel de Egipto, no he
escogido ciudad de todas las tribus de Israel para edificar casa en la
cual estuviese mi nombre, aunque escogí á David para que presidiese en
mi pueblo Israel. \bibverse{17} Y David mi padre tuvo en el corazón
edificar casa al nombre de Jehová Dios de Israel. \bibverse{18} Mas
Jehová dijo á David mi padre: Cuanto á haber tú tenido en el corazón
edificar casa á mi nombre, bien has hecho en tener tal voluntad;
\bibverse{19} Empero tú no edificarás la casa, sino tu hijo que saldrá
de tus lomos, él edificará casa á mi nombre. \bibverse{20} Y Jehová ha
verificado su palabra que había dicho; que me he levantado yo en lugar
de David mi padre, y heme sentado en el trono de Israel, como Jehová
había dicho, y he edificado la casa al nombre de Jehová Dios de Israel.
\bibverse{21} Y he puesto en ella lugar para el arca, en la cual está el
pacto de Jehová, que él hizo con nuestros padres cuando los sacó de la
tierra de Egipto. \bibverse{22} Púsose luego Salomón delante del altar
de Jehová, en presencia de toda la congregación de Israel, y extendiendo
sus manos al cielo, \bibverse{23} Dijo: Jehová Dios de Israel, no hay
Dios como tú, ni arriba en los cielos ni abajo en la tierra, que guardas
el pacto y la misericordia á tus siervos, los que andan delante de ti de
todo su corazón; \bibverse{24} Que has guardado á tu siervo David mi
padre lo que le dijiste: dijístelo con tu boca, y con tu mano lo has
cumplido, como aparece este día. \bibverse{25} Ahora pues, Jehová Dios
de Israel, cumple á tu siervo David mi padre lo que le prometiste,
diciendo: No faltará varón de ti delante de mí, que se siente en el
trono de Israel, con tal que tus hijos guarden su camino, que anden
delante de mí como tú has delante de mí andado. \bibverse{26} Ahora
pues, oh Dios de Israel, verifíquese tu palabra que dijiste á tu siervo
David mi padre. \bibverse{27} Empero ¿es verdad que Dios haya de morar
sobre la tierra? He aquí que los cielos, los cielos de los cielos, no te
pueden contener: ¿cuánto menos esta casa que yo he edificado?
\bibverse{28} Con todo, tú atenderás á la oración de tu siervo, y á su
plegaria, oh Jehová Dios mío, oyendo propicio el clamor y oración que tu
siervo hace hoy delante de ti: \bibverse{29} Que estén tus ojos abiertos
de noche y de día sobre esta casa, sobre este lugar del cual has dicho:
Mi nombre estará allí; y que oigas la oración que tu siervo hará en este
lugar. \bibverse{30} Oye pues la oración de tu siervo, y de tu pueblo
Israel; cuando oraren en este lugar, también tú lo oirás en el lugar de
tu habitación, desde los cielos: que oigas y perdones. \bibverse{31}
Cuando alguno hubiere pecado contra su prójimo, y le tomaren juramento
haciéndole jurar, y viniere el juramento delante de tu altar en esta
casa; \bibverse{32} Tú oirás desde el cielo, y obrarás, y juzgarás á tus
siervos, condenando al impío, tornando su proceder sobre su cabeza, y
justificando al justo para darle conforme á su justicia. \bibverse{33}
Cuando tu pueblo Israel hubiere caído delante de sus enemigos, por haber
pecado contra ti, y á ti se volvieren, y confesaren tu nombre, y oraren,
y te rogaren y suplicaren en esta casa; \bibverse{34} Oyelos tú en los
cielos, y perdona el pecado de tu pueblo Israel, y vuélvelos á la tierra
que diste á sus padres. \bibverse{35} Cuando el cielo se cerrare, y no
lloviere, por haber ellos pecado contra ti, y te rogaren en este lugar,
y confesaren tu nombre, y se volvieren del pecado, cuando los hubieres
afligido; \bibverse{36} Tú oirás en los cielos, y perdonarás el pecado
de tus siervos y de tu pueblo Israel, enseñándoles el buen camino en que
anden; y darás lluvias sobre tu tierra, la cual diste á tu pueblo por
heredad. \bibverse{37} Cuando en la tierra hubiere hambre, ó
pestilencia, ó tizoncillo, ó niebla, ó langosta, ó pulgón: si sus
enemigos los tuvieren cercados en la tierra de su domicilio; cualquiera
plaga ó enfermedad que sea; \bibverse{38} Toda oración y toda súplica
que hiciere cualquier hombre, ó todo tu pueblo Israel, cuando cualquiera
sintiere la plaga de su corazón, y extendiere sus manos á esta casa;
\bibverse{39} Tú oirás en los cielos, en la habitación de tu morada, y
perdonarás, y obrarás, y darás á cada uno conforme á sus caminos, cuyo
corazón tú conoces; (porque sólo tú conoces el corazón de todos los
hijos de los hombres;) \bibverse{40} Para que te teman todos los días
que vivieren sobre la haz de la tierra que tú diste á nuestros padres.
\bibverse{41} Asimismo el extranjero, que no es de tu pueblo Israel, que
hubiere venido de lejanas tierras á causa de tu nombre, \bibverse{42}
(Porque oirán de tu grande nombre, y de tu mano fuerte, y de tu brazo
extendido;) y viniere á orar á esta casa; \bibverse{43} Tú oirás en los
cielos, en la habitación de tu morada, y harás conforme á todo aquello
por lo cual el extranjero hubiere á ti clamado: para que todos los
pueblos de la tierra conozcan tu nombre, y te teman, como tu pueblo
Israel, y entiendan que tu nombre es invocado sobre esta casa que yo
edifiqué. \bibverse{44} Si tu pueblo saliere en batalla contra sus
enemigos por el camino que tú los enviares, y oraren á Jehová hacia la
ciudad que tú elegiste, y hacia la casa que yo edifiqué á tu nombre,
\bibverse{45} Tú oirás en los cielos su oración y su súplica, y les
harás derecho. \bibverse{46} Si hubieren pecado contra ti, (porque no
hay hombre que no peque) y tú estuvieres airado contra ellos, y los
entregares delante del enemigo, para que los cautiven y lleven á tierra
enemiga, sea lejos ó cerca, \bibverse{47} Y ellos volvieren en sí en la
tierra donde fueren cautivos; si se convirtieren, y oraren á ti en la
tierra de los que los cautivaron, y dijeren: Pecamos, hemos hecho lo
malo, hemos cometido impiedad; \bibverse{48} Y si se convirtieren á ti
de todo su corazón y de toda su alma, en la tierra de sus enemigos que
los hubieren llevado cautivos, y oraren á ti hacia su tierra, que tú
diste á sus padres, hacia la ciudad que tú elegiste y la casa que yo he
edificado á tu nombre; \bibverse{49} Tú oirás en los cielos, en la
habitación de tu morada, su oración y su súplica, y les harás derecho;
\bibverse{50} Y perdonarás á tu pueblo que había pecado contra ti, y
todas sus infracciones con que se habrán contra ti rebelado; y harás que
hayan de ellos misericordia los que los hubieren llevado cautivos:
\bibverse{51} Porque ellos son tu pueblo y tu heredad, que tú sacaste de
Egipto, de en medio del horno de hierro. \bibverse{52} Que tus ojos
estén abiertos á la oración de tu siervo, y á la plegaria de tu pueblo
Israel, para oirlos en todo aquello por lo que te invocaren:
\bibverse{53} Pues que tú los apartaste para ti por tu heredad de todos
los pueblos de la tierra, como lo dijiste por mano de Moisés tu siervo,
cuando sacaste á nuestros padres de Egipto, oh Señor Jehová.
\bibverse{54} Y fué, que como acabó Salomón de hacer á Jehová toda esta
oración y súplica, levantóse de estar de rodillas delante del altar de
Jehová con sus manos extendidas al cielo; \bibverse{55} Y puesto en pie,
bendijo á toda la congregación de Israel, diciendo en voz alta:
\bibverse{56} Bendito sea Jehová, que ha dado reposo á su pueblo Israel,
conforme á todo lo que él había dicho; ninguna palabra de todas sus
promesas que expresó por Moisés su siervo, ha faltado. \bibverse{57} Sea
con nosotros Jehová nuestro Dios, como fué con nuestros padres; y no nos
desampare, ni nos deje; \bibverse{58} Incline nuestro corazón hacia sí,
para que andemos en todos sus caminos, y guardemos sus mandamientos y
sus estatutos y sus derechos, los cuales mandó á nuestros padres.
\bibverse{59} Y que estas mis palabras con que he orado delante de
Jehová, estén cerca de Jehová nuestro Dios de día y de noche, para que
él proteja la causa de su siervo, y de su pueblo Israel, cada cosa en su
tiempo; \bibverse{60} A fin de que todos los pueblos de la tierra sepan
que Jehová es Dios, y que no hay otro. \bibverse{61} Sea pues perfecto
vuestro corazón para con Jehová nuestro Dios, andando en sus estatutos,
y guardando sus mandamientos, como el día de hoy. \bibverse{62} Entonces
el rey, y todo Israel con él, sacrificaron víctimas delante de Jehová.
\bibverse{63} Y sacrificó Salomón por sacrificios pacíficos, los cuales
ofreció á Jehová, veinte y dos mil bueyes, y ciento veinte mil ovejas.
Así dedicaron el rey y todos los hijos de Israel la casa de Jehová.
\bibverse{64} Aquel mismo día santificó el rey el medio del atrio que
estaba delante de la casa de Jehová: porque ofreció allí los
holocaustos, y los presentes, y los sebos de los pacíficos; por cuanto
el altar de bronce que estaba delante de Jehová era pequeño, y no
cupieran en él los holocaustos, y los presentes, y los sebos de los
pacíficos. \bibverse{65} En aquel tiempo Salomón hizo fiesta, y con él
todo Israel, una grande congregación, desde como entran en Hamath hasta
el río de Egipto, delante de Jehová nuestro Dios, por siete días y otros
siete días, esto es, por catorce días. \bibverse{66} Y el octavo día
despidió al pueblo: y ellos bendiciendo al rey, se fueron á sus
estancias alegres y gozosos de corazón por todos los beneficios que
Jehová había hecho á David su siervo, y á su pueblo Israel.

\hypertarget{section-8}{%
\section{9}\label{section-8}}

\bibverse{1} Y como Salomón hubo acabado la obra de la casa de Jehová, y
la casa real, y todo lo que Salomón quiso hacer, \bibverse{2} Jehová
apareció á Salomón la segunda vez, como le había aparecido en Gabaón.
\bibverse{3} Y díjole Jehová: Yo he oído tu oración y tu ruego, que has
hecho en mi presencia. Yo he santificado esta casa que tú has edificado,
para poner mi nombre en ella para siempre; y en ella estarán mis ojos y
mi corazón todos los días. \bibverse{4} Y si tú anduvieres delante de
mí, como anduvo David tu padre, en integridad de corazón y en equidad,
haciendo todas las cosas que yo te he mandado, y guardando mis estatutos
y mis derechos, \bibverse{5} Yo afirmaré el trono de tu reino sobre
Israel para siempre, como hablé á David tu padre, diciendo: No faltará
de ti varón en el trono de Israel. \bibverse{6} Mas si obstinadamente os
apartareis de mí vosotros y vuestros hijos, y no guardareis mis
mandamientos y mis estatutos que yo he puesto delante de vosotros, sino
que fuereis y sirviereis á dioses ajenos, y los adorareis; \bibverse{7}
Yo cortaré á Israel de sobre la haz de la tierra que les he entregado; y
esta casa que he santificado á mi nombre, yo la echaré de delante de mí,
é Israel será por proverbio y fábula á todos los pueblos; \bibverse{8} Y
esta casa que estaba en estima, cualquiera que pasare por ella se
pasmará, y silbará, y dirá: ¿Por qué ha hecho así Jehová á esta tierra,
y á esta casa? \bibverse{9} Y dirán: Por cuanto dejaron á Jehová su
Dios, que había sacado á sus padres de tierra de Egipto, y echaron mano
á dioses ajenos, y los adoraron, y los sirvieron: por eso ha traído
Jehová sobre ellos todo aqueste mal. \bibverse{10} Y aconteció al cabo
de veinte años, en que Salomón había edificado las dos casas, la casa de
Jehová y la casa real, \bibverse{11} (Para las cuales Hiram rey de Tiro,
había traído á Salomón madera de cedro y de haya, y cuanto oro él
quiso), que el rey Salomón dió á Hiram veinte ciudades en tierra de
Galilea. \bibverse{12} Y salió Hiram de Tiro para ver las ciudades que
Salomón le había dado, y no le contentaron. \bibverse{13} Y dijo: ¿Qué
ciudades son estas que me has dado, hermano? Y púsoles por nombre, la
tierra de Cabul, hasta hoy. \bibverse{14} Y había Hiram enviado al rey
ciento y veinte talentos de oro. \bibverse{15} Y esta es la razón del
tributo que el rey Salomón impuso para edificar la casa de Jehová, y su
casa, y á Millo, y el muro de Jerusalem, y á Hasor, y Megiddo, y Gezer.
\bibverse{16} Faraón el rey de Egipto había subido y tomado á Gezer, y
quemádola, y había muerto los Cananeos que habitaban la ciudad, y dádola
en don á su hija la mujer de Salomón. \bibverse{17} Restauró pues
Salomón á Gezer, y á la baja Beth-oron, \bibverse{18} Y á Baalath, y á
Tadmor en tierra del desierto; \bibverse{19} Asimismo todas las ciudades
donde Salomón tenía municiones, y las ciudades de los carros, y las
ciudades de la gente de á caballo, y todo lo que Salomón deseó edificar
en Jerusalem, en el Líbano, y en toda la tierra de su señorío.
\bibverse{20} A todos los pueblos que quedaron de los Amorrheos,
Hetheos, Pherezeos, Heveos, Jebuseos, que no fueron de los hijos de
Israel; \bibverse{21} A sus hijos que quedaron en la tierra después de
ellos, que los hijos de Israel no pudieron acabar, hizo Salomón que
sirviesen con tributo hasta hoy. \bibverse{22} Mas á ninguno de los
hijos de Israel impuso Salomón servicio, sino que eran hombres de
guerra, ó sus criados, ó sus príncipes, ó sus capitanes, ó comandantes
de sus carros, ó su gente de á caballo. \bibverse{23} Y los que Salomón
había hecho jefes y prepósitos sobre las obras, eran quinientos y
cincuenta, los cuales estaban sobre el pueblo que trabajaba en aquella
obra. \bibverse{24} Y subió la hija de Faraón de la ciudad de David á su
casa que Salomón le había edificado: entonces edificó él á Millo.
\bibverse{25} Y ofrecía Salomón tres veces cada un año holocaustos y
pacíficos sobre el altar que él edificó á Jehová, y quemaba perfumes
sobre el que estaba delante de Jehová, después que la casa fué acabada.
\bibverse{26} Hizo también el rey Salomón navíos en Ezión-geber, que es
junto á Elath en la ribera del mar Bermejo, en la tierra de Edom.
\bibverse{27} Y envió Hiram en ellos á sus siervos, marineros y diestros
en la mar, con los siervos de Salomón: \bibverse{28} Los cuales fueron á
Ophir, y tomaron de allí oro, cuatrocientos y veinte talentos, y
trajéronlo al rey Salomón.

\hypertarget{section-9}{%
\section{10}\label{section-9}}

\bibverse{1} Y oyendo la reina de Seba la fama de Salomón en el nombre
de Jehová, vino á probarle con preguntas. \bibverse{2} Y vino á
Jerusalem con muy grande comitiva, con camellos cargados de especias, y
oro en grande abundancia, y piedras preciosas: y como vino á Salomón,
propúsole todo lo que en su corazón tenía. \bibverse{3} Y Salomón le
declaró todas sus palabras: ninguna cosa se le escondió al rey, que no
le declarase. \bibverse{4} Y cuando la reina de Seba vió toda la
sabiduría de Salomón, y la casa que había edificado, \bibverse{5}
Asimismo la comida de su mesa, el asiento de sus siervos, el estado y
vestidos de los que le servían, sus maestresalas, y sus holocaustos que
sacrificaba en la casa de Jehová, quedóse enajenada. \bibverse{6} Y dijo
al rey: Verdad es lo que oí en mi tierra de tus cosas y de tu sabiduría;
\bibverse{7} Mas yo no lo creía, hasta que he venido, y mis ojos han
visto, que ni aun la mitad fué lo que se me dijo: es mayor tu sabiduría
y bien que la fama que yo había oído. \bibverse{8} Bienaventurados tus
varones, dichosos estos tus siervos, que están continuamente delante de
ti, y oyen tu sabiduría. \bibverse{9} Jehová tu Dios sea bendito, que se
agradó de ti para ponerte en el trono de Israel; porque Jehová ha amado
siempre á Israel, y te ha puesto por rey, para que hagas derecho y
justicia. \bibverse{10} Y dió ella al rey ciento y veinte talentos de
oro, y muy mucha especiería, y piedras preciosas: nunca vino tan grande
copia de especias, como la reina de Seba dió al rey Salomón.
\bibverse{11} La flota de Hiram que había traído el oro de Ophir, traía
también de Ophir muy mucha madera de brasil, y piedras preciosas.
\bibverse{12} Y de la madera de brasil hizo el rey balaustres para la
casa de Jehová, y para las casas reales, arpas también y salterios para
los cantores: nunca vino tanta madera de brasil, ni se ha visto hasta
hoy. \bibverse{13} Y el rey Salomón dió á la reina de Seba todo lo que
quiso, y todo lo que pidió, además de lo que Salomón le dió como de mano
del rey Salomón. Y ella se volvió, y se fué á su tierra con sus criados.
\bibverse{14} El peso del oro que Salomón tenía de renta cada un año,
era seiscientos sesenta y seis talentos de oro; \bibverse{15} Sin lo de
los mercaderes, y de la contratación de especias, y de todos los reyes
de Arabia, y de los principales de la tierra. \bibverse{16} Hizo también
el rey Salomón doscientos paveses de oro extendido: seiscientos siclos
de oro gastó en cada pavés. \bibverse{17} Asimismo trescientos escudos
de oro extendido, en cada uno de los cuales gastó tres libras de oro: y
púsolos el rey en la casa del bosque del Líbano. \bibverse{18} Hizo
también el rey un gran trono de marfil, el cual cubrió de oro purísimo.
\bibverse{19} Seis gradas tenía el trono, y lo alto de él era redondo
por el respaldo: y de la una parte y de la otra tenía apoyos cerca del
asiento, junto á los cuales estaban colocados dos leones. \bibverse{20}
Estaban también doce leones puestos allí sobre las seis gradas, de la
una parte y de la otra: en ningún otro reino se había hecho trono
semejante. \bibverse{21} Y todos los vasos de beber del rey Salomón eran
de oro, y asimismo toda la vajilla de la casa del bosque del Líbano era
de oro fino: no había plata; en tiempo de Salomón no era de estima.
\bibverse{22} Porque el rey tenía la flota que salía á la mar, á
Tharsis, con la flota de Hiram: una vez en cada tres años venía la flota
de Tharsis, y traía oro, plata, marfil, simios y pavos. \bibverse{23}
Así excedía el rey Salomón á todos los reyes de la tierra en riquezas y
en sabiduría. \bibverse{24} Toda la tierra procuraba ver la cara de
Salomón, para oir su sabiduría, la cual Dios había puesto en su corazón.
\bibverse{25} Y todos le llevaban cada año sus presentes: vasos de oro,
vasos de plata, vestidos, armas, aromas, caballos y acémilas.
\bibverse{26} Y juntó Salomón carros y gente de á caballo; y tenía mil
cuatrocientos carros, y doce mil jinetes, los cuales puso en las
ciudades de los carros, y con el rey en Jerusalem. \bibverse{27} Y puso
el rey en Jerusalem plata como piedras, y cedros como los cabrahigos que
están por los campos en abundancia. \bibverse{28} Y sacaban caballos y
lienzos á Salomón de Egipto: porque la compañía de los mercaderes del
rey compraban caballos y lienzos. \bibverse{29} Y venía y salía de
Egipto, el carro por seiscientas piezas de plata, y el caballo por
ciento y cincuenta; y así los sacaban por mano de ellos todos los reyes
de los Hetheos, y de Siria.

\hypertarget{section-10}{%
\section{11}\label{section-10}}

\bibverse{1} Empero el rey Salomón amó, á más de la hija de Faraón,
muchas mujeres extranjeras: á las de Moab, á las de Ammón, á las de
Idumea, á las de Sidón, y á las Hetheas; \bibverse{2} Gentes de las
cuales Jehová había dicho á los hijos de Israel: No entraréis á ellas,
ni ellas entrarán á vosotros; porque ciertamente harán inclinar vuestros
corazones tras sus dioses. A éstas pues se juntó Salomón con amor.
\bibverse{3} Y tuvo setecientas mujeres reinas, y trescientas
concubinas; y sus mujeres torcieron su corazón. \bibverse{4} Y ya que
Salomón era viejo, sus mujeres inclinaron su corazón tras dioses ajenos;
y su corazón no era perfecto con Jehová su Dios, como el corazón de su
padre David. \bibverse{5} Porque Salomón siguió á Astaroth, diosa de los
Sidonios, y á Milcom, abominación de los Ammonitas. \bibverse{6} E hizo
Salomón lo malo en los ojos de Jehová, y no fué cumplidamente tras
Jehová como David su padre. \bibverse{7} Entonces edificó Salomón un
alto á Chêmos, abominación de Moab, en el monte que está enfrente de
Jerusalem; y á Moloch, abominación de los hijos de Ammón. \bibverse{8} Y
así hizo para todas sus mujeres extranjeras, las cuales quemaban
perfumes, y sacrificaban á sus dioses. \bibverse{9} Y enojóse Jehová
contra Salomón, por cuanto estaba su corazón desviado de Jehová Dios de
Israel, que le había aparecido dos veces, \bibverse{10} Y le había
mandado acerca de esto, que no siguiese dioses ajenos: mas él no guardó
lo que le mandó Jehová. \bibverse{11} Y dijo Jehová á Salomón: Por
cuanto ha habido esto en ti, y no has guardado mi pacto y mis estatutos
que yo te mandé, romperé el reino de ti, y lo entregaré á tu siervo.
\bibverse{12} Empero no lo haré en tus días, por amor de David tu padre:
romperélo de la mano de tu hijo. \bibverse{13} Sin embargo no romperé
todo el reino, sino que daré una tribu á tu hijo, por amor de David mi
siervo, y por amor de Jerusalem que yo he elegido. \bibverse{14} Y
Jehová suscitó un adversario á Salomón, á Adad, Idumeo, de la sangre
real, el cual estaba en Edom. \bibverse{15} Porque cuando David estaba
en Edom, y subió Joab el general del ejército á enterrar los muertos, y
mató á todos los varones de Edom, \bibverse{16} (Porque seis meses
habitó allí Joab, y todo Israel, hasta que hubo acabado á todo el sexo
masculino en Edom;) \bibverse{17} Entonces huyó Adad, y con él algunos
varones Idumeos de los siervos de su padre, y fuése á Egipto; era
entonces Adad muchacho pequeño. \bibverse{18} Y levantáronse de Madián,
y vinieron á Parán; y tomando consigo hombres de Parán, viniéronse á
Egipto, á Faraón rey de Egipto, el cual le dió casa, y le señaló
alimentos, y aun le dió tierra. \bibverse{19} Y halló Adad grande gracia
delante de Faraón, el cual le dió por mujer á la hermana de su esposa, á
la hermana de la reina Thaphnes. \bibverse{20} Y la hermana de Thaphnes
le parió á su hijo Genubath, al cual destetó Thaphnes dentro de la casa
de Faraón; y estaba Genubath en casa de Faraón entre los hijos de
Faraón. \bibverse{21} Y oyendo Adad en Egipto que David había dormido
con sus padres, y que era muerto Joab general del ejército, Adad dijo á
Faraón: Déjame ir á mi tierra. \bibverse{22} Y respondióle Faraón: ¿Por
qué? ¿qué te falta conmigo, que procuras irte á tu tierra? Y él
respondió: Nada; con todo, ruégote que me dejes ir. \bibverse{23}
Despertóle también Dios por adversario á Rezón, hijo de Eliada, el cual
había huído de su amo Adad-ezer, rey de Soba. \bibverse{24} Y había
juntado gente contra él, y habíase hecho capitán de una compañía, cuando
David deshizo á los de Soba. Después se fueron á Damasco, y habitaron
allí, é hiciéronle rey en Damasco. \bibverse{25} Y fué adversario á
Israel todos los días de Salomón; y fué otro mal con el de Adad, porque
aborreció á Israel, y reinó sobre la Siria. \bibverse{26} Asimismo
Jeroboam hijo de Nabat, Ephrateo de Sereda, siervo de Salomón, (su madre
se llamaba Serva, mujer viuda) alzó su mano contra el rey. \bibverse{27}
Y la causa por qué éste alzó mano contra el rey, fué esta: Salomón
edificando á Millo, cerró el portillo de la ciudad de David su padre.
\bibverse{28} Y el varón Jeroboam era valiente y esforzado; y viendo
Salomón al mancebo que era hombre activo, encomendóle todo el cargo de
la casa de José. \bibverse{29} Aconteció pues en aquel tiempo, que
saliendo Jeroboam de Jerusalem, topóle en el camino el profeta Ahías
Silonita; y él estaba cubierto con una capa nueva; y estaban ellos dos
solos en el campo. \bibverse{30} Y trabando Ahías de la capa nueva que
tenía sobre sí, rompióla en doce pedazos, \bibverse{31} Y dijo á
Jeroboam: Toma para ti los diez pedazos; porque así dijo Jehová Dios de
Israel: He aquí que yo rompo el reino de la mano de Salomón, y á ti daré
diez tribus; \bibverse{32} (Y él tendrá una tribu, por amor de David mi
siervo, y por amor de Jerusalem, ciudad que yo he elegido de todas las
tribus de Israel:) \bibverse{33} Por cuanto me han dejado, y han adorado
á Astharoth diosa de los Sidonios, y á Chêmos dios de Moab, y á Moloch
dios de los hijos de Ammón; y no han andado en mis caminos, para hacer
lo recto delante de mis ojos, y mis estatutos, y mis derechos, como hizo
David su padre. \bibverse{34} Empero no quitaré nada de su reino de sus
manos, sino que lo retendré por caudillo todos los días de su vida, por
amor de David mi siervo, al cual yo elegí, y él guardó mis mandamientos
y mis estatutos: \bibverse{35} Mas yo quitaré el reino de la mano de su
hijo, y darélo á ti, las diez tribus. \bibverse{36} Y á su hijo daré una
tribu, para que mi siervo David tenga lámpara todos los días delante de
mí en Jerusalem, ciudad que yo me elegí para poner en ella mi nombre.
\bibverse{37} Yo pues te tomaré á ti, y tú reinarás en todas las cosas
que deseare tu alma, y serás rey sobre Israel. \bibverse{38} Y será que,
si prestares oído á todas las cosas que te mandare, y anduvieres en mis
caminos, é hicieres lo recto delante de mis ojos, guardando mis
estatutos y mis mandamientos, como hizo David mi siervo, yo seré
contigo, y te edificaré casa firme, como la edifiqué á David, y yo te
entregaré á Israel. \bibverse{39} Y yo afligiré la simiente de David á
causa de esto, mas no para siempre. \bibverse{40} Procuró por tanto
Salomón de matar á Jeroboam, pero levantándose Jeroboam, huyó á Egipto,
á Sisac rey de Egipto, y estuvo en Egipto hasta la muerte de Salomón.
\bibverse{41} Lo demás de los hechos de Salomón, y todas las cosas que
hizo, y su sabiduría, ¿no están escritas en el libro de los hechos de
Salomón? \bibverse{42} Y los días que Salomón reinó en Jerusalem sobre
todo Israel, fueron cuarenta años. \bibverse{43} Y durmió Salomón con
sus padres, y fué sepultado en la ciudad de su padre David: y reinó en
su lugar Roboam su hijo.

\hypertarget{section-11}{%
\section{12}\label{section-11}}

\bibverse{1} Y fué Roboam á Sichêm; porque todo Israel había venido á
Sichêm para hacerlo rey. \bibverse{2} Y aconteció, que como lo oyó
Jeroboam hijo de Nabat, que estaba en Egipto, porque había huído de
delante del rey Salomón, y habitaba en Egipto; \bibverse{3} Enviaron y
llamáronle. Vino pues Jeroboam y toda la congregación de Israel, y
hablaron á Roboam, diciendo: \bibverse{4} Tu padre agravó nuestro yugo,
mas ahora tú disminuye algo de la dura servidumbre de tu padre, y del
yugo pesado que puso sobre nosotros, y te serviremos. \bibverse{5} Y él
les dijo: Idos, y de aquí á tres días volved á mí. Y el pueblo se fué.
\bibverse{6} Entonces el rey Roboam tomó consejo con los ancianos que
habían estado delante de Salomón su padre cuando vivía, y dijo: ¿Cómo
aconsejáis vosotros que responda á este pueblo? \bibverse{7} Y ellos le
hablaron, diciendo: Si tú fueres hoy siervo de este pueblo, y lo
sirvieres, y respondiéndole buenas palabras les hablares, ellos te
servirán para siempre. \bibverse{8} Mas él, dejado el consejo de los
viejos que ellos le habían dado, tomó consejo con los mancebos que se
habían criado con él, y estaban delante de él. \bibverse{9} Y díjoles:
¿Cómo aconsejáis vosotros que respondamos á este pueblo, que me ha
hablado, diciendo: Disminuye algo del yugo que tu padre puso sobre
nosotros? \bibverse{10} Entonces los mancebos que se habían criado con
él, le respondieron, diciendo: Así hablarás á este pueblo que te ha
dicho estas palabras: Tu padre agravó nuestro yugo; mas tú disminúyenos
algo: así les hablarás: El menor dedo de los míos es más grueso que los
lomos de mi padre. \bibverse{11} Ahora pues, mi padre os cargó de pesado
yugo, mas yo añadiré á vuestro yugo; mi padre os hirió con azotes, mas
yo os heriré con escorpiones. \bibverse{12} Y al tercer día vino
Jeroboam con todo el pueblo á Roboam; según el rey lo había mandado,
diciendo: Volved á mí al tercer día. \bibverse{13} Y el rey respondió al
pueblo duramente, dejado el consejo de los ancianos que ellos le habían
dado; \bibverse{14} Y hablóles conforme al consejo de los mancebos,
diciendo: Mi padre agravó vuestro yugo, pero yo añadiré á vuestro yugo;
mi padre os hirió con azotes, mas yo os heriré con escorpiones.
\bibverse{15} Y no oyó el rey al pueblo; porque era ordenación de
Jehová, para confirmar su palabra, que Jehová había hablado por medio de
Ahías Silonita á Jeroboam hijo de Nabat. \bibverse{16} Y cuando todo el
pueblo vió que el rey no les había oído, respondióle estas palabras,
diciendo: ¿Qué parte tenemos nosotros con David? No tenemos heredad en
el hijo de Isaí. ¡Israel, á tus estancias! ¡Provee ahora en tu casa,
David! Entonces Israel se fué á sus estancias. \bibverse{17} Mas reinó
Roboam sobre los hijos de Israel que moraban en las ciudades de Judá.
\bibverse{18} Y el rey Roboam envió á Adoram, que estaba sobre los
tributos; pero apedreóle todo Israel, y murió. Entonces el rey Roboam se
esforzó á subir en un carro, y huir á Jerusalem. \bibverse{19} Así se
apartó Israel de la casa de David hasta hoy. \bibverse{20} Y aconteció,
que oyendo todo Israel que Jeroboam había vuelto, enviaron y llamáronle
á la congregación, é hiciéronle rey sobre todo Israel, sin quedar tribu
alguna que siguiese la casa de David, sino sólo la tribu de Judá.
\bibverse{21} Y como Roboam vino á Jerusalem, juntó toda la casa de Judá
y la tribu de Benjamín, ciento y ochenta mil hombres escogidos de
guerra, para hacer guerra á la casa de Israel, y reducir el reino á
Roboam hijo de Salomón. \bibverse{22} Mas fué palabra de Jehová á
Semeías varón de Dios, diciendo: \bibverse{23} Habla á Roboam hijo de
Salomón, rey de Judá, y á toda la casa de Judá y de Benjamín, y á los
demás del pueblo, diciendo: \bibverse{24} Así ha dicho Jehová: No
vayáis, ni peleéis contra vuestros hermanos los hijos de Israel; volveos
cada uno á su casa; porque este negocio yo lo he hecho. Y ellos oyeron
la palabra de Dios, y volviéronse, y fuéronse, conforme á la palabra de
Jehová. \bibverse{25} Y reedificó Jeroboam á Sichêm en el monte de
Ephraim, y habitó en ella; y saliendo de allí, reedificó á Penuel.
\bibverse{26} Y dijo Jeroboam en su corazón: Ahora se volverá el reino á
la casa de David, \bibverse{27} Si este pueblo subiere á sacrificar á la
casa de Jehová en Jerusalem: porque el corazón de este pueblo se
convertirá á su señor Roboam rey de Judá, y me matarán á mí, y se
tornarán á Roboam rey de Judá. \bibverse{28} Y habido consejo, hizo el
rey dos becerros de oro, y dijo al pueblo: Harto habéis subido á
Jerusalem: he aquí tus dioses, oh Israel, que te hicieron subir de la
tierra de Egipto. \bibverse{29} Y puso el uno en Beth-el, y el otro puso
en Dan. \bibverse{30} Y esto fué ocasión de pecado; porque el pueblo iba
á adorar delante del uno, hasta Dan. \bibverse{31} Hizo también casa de
altos, é hizo sacerdotes de la clase del pueblo, que no eran de los
hijos de Leví. \bibverse{32} Entonces instituyó Jeroboam solemnidad en
el mes octavo, á los quince del mes, conforme á la solemnidad que se
celebraba en Judá; y sacrificó sobre altar. Así hizo en Beth-el,
sacrificando á los becerros que había hecho. Ordenó también en Beth-el
sacerdotes de los altos que él había fabricado. \bibverse{33} Sacrificó
pues sobre el altar que él había hecho en Beth-el, á los quince del mes
octavo, el mes que él había inventado de su corazón; é hizo fiesta á los
hijos de Israel, y subió al altar para quemar perfumes.

\hypertarget{section-12}{%
\section{13}\label{section-12}}

\bibverse{1} Y he aquí que un varón de Dios por palabra de Jehová vino
de Judá á Beth-el; y estando Jeroboam al altar para quemar perfumes,
\bibverse{2} El clamó contra el altar por palabra de Jehová, y dijo:
Altar, altar, así ha dicho Jehová: He aquí que á la casa de David nacerá
un hijo, llamado Josías, el cual sacrificará sobre ti á los sacerdotes
de los altos que queman sobre ti perfumes; y sobre ti quemarán huesos de
hombres. \bibverse{3} Y aquel mismo día dió una señal, diciendo: Esta es
la señal de que Jehová ha hablado: he aquí que el altar se quebrará, y
la ceniza que sobre él está se derramará. \bibverse{4} Y como el rey
Jeroboam oyó la palabra del varón de Dios, que había clamado contra el
altar de Beth-el, extendiendo su mano desde el altar, dijo: ¡Prendedle!
Mas la mano que había extendido contra él, se le secó, que no la pudo
tornar á sí. \bibverse{5} Y el altar se rompió, y derramóse la ceniza
del altar, conforme á la señal que el varón de Dios había dado por
palabra de Jehová. \bibverse{6} Entonces respondiendo el rey, dijo al
varón de Dios: Te pido que ruegues á la faz de Jehová tu Dios, y ora por
mí, que mi mano me sea restituída. Y el varón de Dios oró á la faz de
Jehová, y la mano del rey se le recuperó, y tornóse como antes.
\bibverse{7} Y el rey dijo al varón de Dios: Ven conmigo á casa, y
comerás, y yo te daré un presente. \bibverse{8} Mas el varón de Dios
dijo al rey: Si me dieses la mitad de tu casa, no iría contigo, ni
comería pan ni bebería agua en este lugar; \bibverse{9} Porque así me
está mandado por palabra de Jehová, diciendo: No comas pan, ni bebas
agua, ni vuelvas por el camino que fueres. \bibverse{10} Fuése pues por
otro camino, y no volvió por el camino por donde había venido á Beth-el.
\bibverse{11} Moraba á la sazón en Beth-el un viejo profeta, al cual
vino su hijo, y contóle todo lo que el varón de Dios había hecho aquel
día en Beth-el: contáronle también á su padre las palabras que había
hablado al rey. \bibverse{12} Y su padre les dijo: ¿Por qué camino fué?
Y sus hijos le mostraron el camino por donde se había tornado el varón
de Dios, que había venido de Judá. \bibverse{13} Y él dijo á sus hijos:
Enalbardadme el asno. Y ellos le enalbardaron el asno, y subió en él.
\bibverse{14} Y yendo tras el varón de Dios, hallóle que estaba sentado
debajo de un alcornoque: y díjole: ¿Eres tú el varón de Dios que viniste
de Judá? Y él dijo: Yo soy. \bibverse{15} Díjole entonces: Ven conmigo á
casa, y come del pan. \bibverse{16} Mas él respondió: No podré volver
contigo, ni iré contigo; ni tampoco comeré pan ni beberé agua contigo en
este lugar; \bibverse{17} Porque por palabra de Dios me ha sido dicho:
No comas pan ni bebas agua allí, ni vuelvas por el camino que fueres.
\bibverse{18} Y el otro le dijo: Yo también soy profeta como tú, y un
ángel me ha hablado por palabra de Jehová, diciendo: Vuélvele contigo á
tu casa, para que coma pan y beba agua. Empero mintióle. \bibverse{19}
Entonces volvió con él, y comió del pan en su casa, y bebió del agua.
\bibverse{20} Y aconteció que, estando ellos á la mesa, fué palabra de
Jehová al profeta que le había hecho volver; \bibverse{21} Y clamó al
varón de Dios que había venido de Judá, diciendo: Así dijo Jehová: Por
cuanto has sido rebelde al dicho de Jehová, y no guardaste el
mandamiento que Jehová tu Dios te había prescrito, \bibverse{22} Sino
que volviste, y comiste del pan y bebiste del agua en el lugar donde
Jehová te había dicho no comieses pan ni bebieses agua, no entrará tu
cuerpo en el sepulcro de tus padres. \bibverse{23} Y como hubo comido
del pan y bebido, el profeta que le había hecho volver le enalbardó un
asno; \bibverse{24} Y yéndose, topóle un león en el camino, y matóle; y
su cuerpo estaba echado en el camino, y el asno estaba junto á él, y el
león también estaba junto al cuerpo. \bibverse{25} Y he aquí unos que
pasaban, y vieron el cuerpo que estaba echado en el camino, y el león
que estaba junto al cuerpo: y vinieron, y dijéronlo en la ciudad donde
el viejo profeta habitaba. \bibverse{26} Y oyéndolo el profeta que le
había vuelto del camino, dijo: El varón de Dios es, que fué rebelde al
dicho de Jehová: por tanto Jehová le ha entregado al león, que le ha
quebrantado y muerto, conforme á la palabra de Jehová que él le dijo.
\bibverse{27} Y habló á sus hijos, y díjoles: Enalbardadme un asno. Y
ellos se lo enalbardaron. \bibverse{28} Y él fué, y halló su cuerpo
tendido en el camino, y el asno y el león estaban junto al cuerpo: el
león no había comido el cuerpo, ni dañado al asno. \bibverse{29} Y
tomando el profeta el cuerpo del varón de Dios, púsolo sobre el asno, y
llevóselo. Y el profeta viejo vino á la ciudad, para endecharle y
enterrarle. \bibverse{30} Y puso su cuerpo en su sepulcro; y
endecháronle, diciendo: ¡Ay, hermano mío! \bibverse{31} Y después que le
hubieron enterrado, habló á sus hijos, diciendo: Cuando yo muriere,
enterradme en el sepulcro en que está sepultado el varón de Dios; poned
mis huesos junto á los suyos. \bibverse{32} Porque sin duda vendrá lo
que él dijo á voces por palabra de Jehová contra el altar que está en
Beth-el, y contra todas las casas de los altos que están en las ciudades
de Samaria. \bibverse{33} Después de esto no se tornó Jeroboam de su mal
camino: antes volvió á hacer sacerdotes de los altos de la clase del
pueblo, y quien quería se consagraba, y era de los sacerdotes de los
altos. \bibverse{34} Y esto fué causa de pecado á la casa de Jeroboam;
por lo cual fué cortada y raída de sobre la haz de la tierra.

\hypertarget{section-13}{%
\section{14}\label{section-13}}

\bibverse{1} En aquel tiempo Abías hijo de Jeroboam cayó enfermo.
\bibverse{2} Y dijo Jeroboam á su mujer: Levántate ahora, disfrázate,
porque no te conozcan que eres la mujer de Jeroboam, y ve á Silo; que
allá está Ahías profeta, el que me dijo que yo había de ser rey sobre
este pueblo. \bibverse{3} Y toma en tu mano diez panes, y turrones, y
una botija de miel, y ve á él; que te declare lo que ha de ser de este
mozo. \bibverse{4} Y la mujer de Jeroboam hízolo así; y levantóse, y fué
á Silo, y vino á casa de Ahías. Y no podía ya ver Ahías, que sus ojos se
habían oscurecido á causa de su vejez. \bibverse{5} Mas Jehová había
dicho á Ahías: He aquí que la mujer de Jeroboam vendrá á consultarte por
su hijo, que está enfermo: así y así le has de responder; pues será que
cuando ella viniere, vendrá disimulada. \bibverse{6} Y como Ahías oyó el
sonido de sus pies cuando entraba por la puerta, dijo: Entra, mujer de
Jeroboam; ¿por qué te finges otra? empero yo soy enviado á ti con
revelación dura. \bibverse{7} Ve, y di á Jeroboam: Así dijo Jehová Dios
de Israel: Por cuanto yo te levanté de en medio del pueblo, y te hice
príncipe sobre mi pueblo Israel, \bibverse{8} Y rompí el reino de la
casa de David, y te lo entregué á ti; y tú no has sido como David mi
siervo, que guardó mis mandamientos y anduvo en pos de mí con todo su
corazón, haciendo solamente lo derecho delante de mis ojos; \bibverse{9}
Antes hiciste lo malo sobre todos los que han sido antes de ti: que
fuiste y te hiciste dioses ajenos y de fundición para enojarme, y á mí
me echaste tras tus espaldas: \bibverse{10} Por tanto, he aquí que yo
traigo mal sobre la casa de Jeroboam, y yo talaré de Jeroboam todo
meante á la pared, así el guardado como el desamparado en Israel; y
barreré la posteridad de la casa de Jeroboam, como es barrido el
estiércol, hasta que sea acabada. \bibverse{11} El que muriere de los de
Jeroboam en la ciudad, le comerán los perros; y el que muriere en el
campo, comerlo han las aves del cielo; porque Jehová lo ha dicho.
\bibverse{12} Y tú levántate, y vete á tu casa; que en entrando tu pie
en la ciudad, morirá el mozo. \bibverse{13} Y todo Israel lo endechará,
y le enterrarán; porque sólo él de los de Jeroboam entrará en sepultura;
por cuanto se ha hallado en él alguna cosa buena de Jehová Dios de
Israel, en la casa de Jeroboam. \bibverse{14} Y Jehová se levantará un
rey sobre Israel, el cual talará la casa de Jeroboam en este día; ¿y
qué, si ahora? \bibverse{15} Y Jehová sacudirá á Israel, al modo que la
caña se agita en las aguas: y él arrancará á Israel de esta buena tierra
que había dado á sus padres, y esparcirálos de la otra parte del río,
por cuanto han hecho sus bosques, enojando á Jehová. \bibverse{16} Y él
entregará á Israel por los pecados de Jeroboam, el cual pecó, y ha hecho
pecar á Israel. \bibverse{17} Entonces la mujer de Jeroboam se levantó,
y se fué, y vino á Thirsa: y entrando ella por el umbral de la casa, el
mozo murió. \bibverse{18} Y enterráronlo, y endechólo todo Israel,
conforme á la palabra de Jehová, que él había hablado por mano de su
siervo Ahías profeta. \bibverse{19} Los otros hechos de Jeroboam, qué
guerras hizo, y cómo reinó, todo está escrito en el libro de las
historias de los reyes de Israel. \bibverse{20} El tiempo que reinó
Jeroboam fueron veintidós años; y habiendo dormido con sus padres, reinó
en su lugar Nadab su hijo. \bibverse{21} Y Roboam hijo de Salomón reinó
en Judá. De cuarenta y un años era Roboam cuando comenzó á reinar, y
diecisiete años reinó en Jerusalem, ciudad que Jehová eligió de todas
las tribus de Israel, para poner allí su nombre. El nombre de su madre
fué Naama, Ammonita. \bibverse{22} Y Judá hizo lo malo en los ojos de
Jehová, y enojáronle más que todo lo que sus padres habían hecho en sus
pecados que cometieron. \bibverse{23} Porque ellos también se edificaron
altos, estatuas, y bosques, en todo collado alto, y debajo de todo árbol
frondoso: \bibverse{24} Y hubo también sodomitas en la tierra, é
hicieron conforme á todas las abominaciones de las gentes que Jehová
había echado delante de los hijos de Israel. \bibverse{25} Al quinto año
del rey Roboam subió Sisac rey de Egipto contra Jerusalem. \bibverse{26}
Y tomó los tesoros de la casa de Jehová, y los tesoros de la casa real,
y saqueólo todo: llevóse también todos los escudos de oro que Salomón
había hecho. \bibverse{27} Y en lugar de ellos hizo el rey Roboam
escudos de metal, y diólos en manos de los capitanes de los de la
guardia, quienes custodiaban la puerta de la casa real. \bibverse{28} Y
cuando el rey entraba en la casa de Jehová, los de la guardia los
llevaban; y poníanlos después en la cámara de los de la guardia.
\bibverse{29} Lo demás de los hechos de Roboam, y todas las cosas que
hizo, ¿no están escritas en las crónicas de los reyes de Judá?
\bibverse{30} Y hubo guerra entre Roboam y Jeroboam todos los días.
\bibverse{31} Y durmió Roboam con sus padres, y fué sepultado con sus
padres en la ciudad de David. El nombre de su madre fué Naama, Ammonita.
Y reinó en su lugar Abiam su hijo.

\hypertarget{section-14}{%
\section{15}\label{section-14}}

\bibverse{1} En el año dieciocho del rey Jeroboam hijo de Nabat, Abiam
comenzó á reinar sobre Judá. \bibverse{2} Reinó tres años en Jerusalem.
El nombre de su madre fué Maachâ, hija de Abisalom. \bibverse{3} Y
anduvo en todos los pecados de su padre, que había éste hecho antes de
él; y no fué su corazón perfecto con Jehová su Dios, como el corazón de
David su padre. \bibverse{4} Mas por amor de David, dióle Jehová su Dios
lámpara en Jerusalem, levantándole á su hijo después de él, y
sosteniendo á Jerusalem: \bibverse{5} Por cuanto David había hecho lo
recto ante los ojos de Jehová, y de ninguna cosa que le mandase se había
apartado en todos los días de su vida, excepto el negocio de Uría
Hetheo. \bibverse{6} Y hubo guerra entre Roboam y Jeroboam todos los
días de su vida. \bibverse{7} Lo demás de los hechos de Abiam, y todas
las cosas que hizo, ¿no están escritas en el libro de las crónicas de
los reyes de Judá? Y hubo guerra entre Abiam y Jeroboam. \bibverse{8} Y
durmió Abiam con sus padres, y sepultáronlo en la ciudad de David: y
reinó Asa su hijo en su lugar. \bibverse{9} En el año veinte de Jeroboam
rey de Israel, Asa comenzó á reinar sobre Judá. \bibverse{10} Y reinó
cuarenta y un años en Jerusalem; el nombre de su madre fué Maachâ, hija
de Abisalom. \bibverse{11} Y Asa hizo lo recto ante los ojos de Jehová,
como David su padre. \bibverse{12} Porque quitó los sodomitas de la
tierra, y quitó todas las suciedades que sus padres habían hecho.
\bibverse{13} Y también privó á su madre Maachâ de ser princesa, porque
había hecho un ídolo en un bosque. Además deshizo Asa el ídolo de su
madre, y quemólo junto al torrente de Cedrón. \bibverse{14} Empero los
altos no se quitaron: con todo, el corazón de Asa fué perfecto para con
Jehová toda su vida. \bibverse{15} También metió en la casa de Jehová lo
que su padre había dedicado, y lo que él dedicó: oro, y plata, y vasos.
\bibverse{16} Y hubo guerra entre Asa y Baasa rey de Israel, todo el
tiempo de ambos. \bibverse{17} Y subió Baasa rey de Israel contra Judá,
y edificó á Rama, para no dejar salir ni entrar á ninguno de Asa, rey de
Judá. \bibverse{18} Entonces tomando Asa toda la plata y oro que había
quedado en los tesoros de la casa de Jehová, y los tesoros de la casa
real, entrególos en las manos de sus siervos, y enviólos el rey Asa á
Ben-adad, hijo de Tabrimón, hijo de Hezión, rey de Siria, el cual
residía en Damasco, diciendo: \bibverse{19} Alianza hay entre mí y ti, y
entre mi padre y el tuyo: he aquí yo te envío un presente de plata y
oro: ve, y rompe tu alianza con Baasa rey de Israel, para que me deje.
\bibverse{20} Y Ben-adad consintió con el rey Asa, y envió los príncipes
de los ejércitos que tenía contra las ciudades de Israel, é hirió á
Ahión, y á Dan, y á Abel-beth-maachâ, y á toda Cinneroth, con toda la
tierra de Nephtalí. \bibverse{21} Y oyendo esto Baasa, dejó de edificar
á Rama, y estúvose en Thirsa. \bibverse{22} Entonces el rey Asa convocó
á todo Judá, sin exceptuar ninguno; y quitaron de Rama la piedra y la
madera con que Baasa edificaba, y edificó el rey Asa con ello á Gabaa de
Benjamín, y á Mizpa. \bibverse{23} Lo demás de todos los hechos de Asa,
y toda su fortaleza, y todas las cosas que hizo, y las ciudades que
edificó, ¿no está todo escrito en el libro de las crónicas de los reyes
de Judá? Mas en el tiempo de su vejez enfermó de sus pies. \bibverse{24}
Y durmió Asa con sus padres, y fué sepultado con sus padres en la ciudad
de David su padre: y reinó en su lugar Josaphat su hijo. \bibverse{25} Y
Nadab, hijo de Jeroboam, comenzó á reinar sobre Israel en el segundo año
de Asa rey de Judá; y reinó sobre Israel dos años. \bibverse{26} E hizo
lo malo ante los ojos de Jehová, andando en el camino de su padre, y en
sus pecados con que hizo pecar á Israel. \bibverse{27} Y Baasa hijo de
Ahía, el cual era de la casa de Issachâr, hizo conspiración contra él: é
hiriólo Baasa en Gibbethón, que era de los Filisteos: porque Nadab y
todo Israel tenían cercado á Gibbethón. \bibverse{28} Matólo pues Baasa
en el tercer año de Asa rey de Judá, y reinó en lugar suyo.
\bibverse{29} Y como él vino al reino, hirió toda la casa de Jeroboam,
sin dejar alma viviente de los de Jeroboam, hasta raerlo, conforme á la
palabra de Jehová que él habló por su siervo Ahías Silonita;
\bibverse{30} Por los pecados de Jeroboam que él había cometido, y con
los cuales hizo pecar á Israel; y por su provocación con que provocó á
enojo á Jehová Dios de Israel. \bibverse{31} Lo demás de los hechos de
Nadab, y todas las cosas que hizo, ¿no está todo escrito en el libro de
las crónicas de los reyes de Israel? \bibverse{32} Y hubo guerra entre
Asa y Baasa rey de Israel, todo el tiempo de ambos. \bibverse{33} En el
tercer año de Asa rey de Judá, comenzó á reinar Baasa hijo de Ahía sobre
todo Israel en Thirsa; y reinó veinticuatro años. \bibverse{34} E hizo
lo malo á los ojos de Jehová, y anduvo en el camino de Jeroboam, y en su
pecado con que hizo pecar á Israel.

\hypertarget{section-15}{%
\section{16}\label{section-15}}

\bibverse{1} Y fué palabra de Jehová á Jehú hijo de Hanani contra Baasa,
diciendo: \bibverse{2} Pues que yo te levanté del polvo, y te puse por
príncipe sobre mi pueblo Israel, y tú has andado en el camino de
Jeroboam, y has hecho pecar á mi pueblo Israel, provocándome á ira con
sus pecados; \bibverse{3} He aquí yo barreré la posteridad de Baasa, y
la posteridad de su casa: y pondré tu casa como la casa de Jeroboam hijo
de Nabat. \bibverse{4} El que de Baasa fuere muerto en la ciudad, le
comerán los perros; y el que de él fuere muerto en el campo, comerlo han
las aves del cielo. \bibverse{5} Lo demás de los hechos de Baasa, y las
cosas que hizo, y su fortaleza, ¿no está todo escrito en el libro de las
crónicas de los reyes de Israel? \bibverse{6} Y durmió Baasa con sus
padres, y fué sepultado en Thirsa; y reinó en su lugar Ela su hijo.
\bibverse{7} Empero la palabra de Jehová por mano de Jehú profeta, hijo
de Hanani, había sido contra Baasa y también contra su casa, con motivo
de todo lo malo que hizo á los ojos de Jehová, provocándole á ira con
las obras de sus manos, para que fuese hecha como la casa de Jeroboam; y
porque lo había herido. \bibverse{8} En el año veintiséis de Asa rey de
Judá, comenzó á reinar Ela hijo de Baasa sobre Israel en Thirsa; y reinó
dos años. \bibverse{9} E hizo conjuración contra él su siervo Zimri,
comandante de la mitad de los carros. Y estando él en Thirsa, bebiendo y
embriagado en casa de Arsa su mayordomo en Thirsa, \bibverse{10} Vino
Zimri, y lo hirió y mató, en el año veintisiete de Asa rey de Judá; y
reinó en lugar suyo. \bibverse{11} Y luego que llegó á reinar y estuvo
sentado en su trono, hirió toda la casa de Baasa, sin dejar en ella
meante á la pared, ni sus parientes ni amigos. \bibverse{12} Así rayó
Zimri toda la casa de Baasa, conforme á la palabra de Jehová, que había
proferido contra Baasa por medio del profeta Jehú; \bibverse{13} Por
todos los pecados de Baasa, y los pecados de Ela su hijo, con que ellos
pecaron é hicieron pecar á Israel, provocando á enojo á Jehová Dios de
Israel con sus vanidades. \bibverse{14} Los demás hechos de Ela, y todas
las cosas que hizo, ¿no está todo escrito en el libro de las crónicas de
los reyes de Israel? \bibverse{15} En el año veintisiete de Asa rey de
Judá, comenzó á reinar Zimri, y reinó siete días en Thirsa; y el pueblo
había asentado campo sobre Gibbethón, ciudad de los Filisteos.
\bibverse{16} Y el pueblo que estaba en el campo oyó decir: Zimri ha
hecho conjuración, y ha muerto al rey. Entonces todo Israel levantó el
mismo día por rey sobre Israel á Omri, general del ejército, en el
campo. \bibverse{17} Y subió Omri de Gibbethón, y con él todo Israel, y
cercaron á Thirsa. \bibverse{18} Mas viendo Zimri tomada la ciudad,
metióse en el palacio de la casa real, y pegó fuego á la casa consigo:
así murió, \bibverse{19} Por sus pecados que él había cometido, haciendo
lo malo á los ojos de Jehová, y andando en los caminos de Jeroboam, y en
su pecado que cometió, haciendo pecar á Israel. \bibverse{20} Los demás
hechos de Zimri, y su conspiración que formó, ¿no está todo escrito en
el libro de las crónicas de los reyes de Israel? \bibverse{21} Entonces
el pueblo de Israel fué dividido en dos partes: la mitad del pueblo
seguía á Thibni hijo de Gineth, para hacerlo rey: y la otra mitad seguía
á Omri. \bibverse{22} Mas el pueblo que seguía á Omri, pudo más que el
que seguía á Thibni hijo de Gineth; y Thibni murió, y Omri fué rey.
\bibverse{23} En el año treinta y uno de Asa rey de Judá, comenzó á
reinar Omri sobre Israel, y reinó doce años: en Thirsa reinó seis años.
\bibverse{24} Y compró él de Semer el monte de Samaria por dos talentos
de plata, y edificó en el monte: y llamó el nombre de la ciudad que
edificó, Samaria, del nombre de Semer, señor que fué de aquel monte.
\bibverse{25} Y Omri hizo lo malo á los ojos de Jehová, é hizo peor que
todos los que habían sido antes de él: \bibverse{26} Pues anduvo en
todos los caminos de Jeroboam hijo de Nabat, y en su pecado con que hizo
pecar á Israel, provocando á ira á Jehová Dios de Israel con sus ídolos.
\bibverse{27} Lo demás de los hechos de Omri, y todas las cosas que
hizo, y sus valentías que ejecutó, ¿no está todo escrito en el libro de
las crónicas de los reyes de Israel? \bibverse{28} Y Omri durmió con sus
padres, y fué sepultado en Samaria; y reinó en lugar suyo Achâb, su
hijo. \bibverse{29} Y comenzó á reinar Achâb hijo de Omri sobre Israel
el año treinta y ocho de Asa rey de Judá. \bibverse{30} Y reinó Achâb
hijo de Omri sobre Israel en Samaria veintidós años. Y Achâb hijo de
Omri hizo lo malo á los ojos de Jehová sobre todos los que fueron antes
de él; \bibverse{31} Porque le fué ligera cosa andar en los pecados de
Jeroboam hijo de Nabat, y tomó por mujer á Jezabel hija de Ethbaal rey
de los Sidonios, y fué y sirvió á Baal, y lo adoró. \bibverse{32} E hizo
altar á Baal, en el templo de Baal que él edificó en Samaria.
\bibverse{33} Hizo también Achâb un bosque; y añadió Achâb haciendo
provocar á ira á Jehová Dios de Israel, más que todos los reyes de
Israel que antes de él habían sido. \bibverse{34} En su tiempo Hiel de
Beth-el reedificó á Jericó. En Abiram su primogénito echó el cimiento, y
en Segub su hijo postrero puso sus puertas; conforme á la palabra de
Jehová que había hablado por Josué hijo de Nun.

\hypertarget{section-16}{%
\section{17}\label{section-16}}

\bibverse{1} Entonces Elías Thisbita, que era de los moradores de
Galaad, dijo á Achâb: Vive Jehová Dios de Israel, delante del cual
estoy, que no habrá lluvia ni rocío en estos años, sino por mi palabra.
\bibverse{2} Y fué á él palabra de Jehová, diciendo: \bibverse{3}
Apártate de aquí, y vuélvete al oriente, y escóndete en el arroyo de
Cherith, que está delante del Jordán; \bibverse{4} Y beberás del arroyo;
y yo he mandado á los cuervos que te den allí de comer. \bibverse{5} Y
él fué, é hizo conforme á la palabra de Jehová; pues se fué y asentó
junto al arroyo de Cherith, que está antes del Jordán. \bibverse{6} Y
los cuervos le traían pan y carne por la mañana, y pan y carne á la
tarde; y bebía del arroyo. \bibverse{7} Pasados algunos días, secóse el
arroyo; porque no había llovido sobre la tierra. \bibverse{8} Y fué á él
palabra de Jehová, diciendo: \bibverse{9} Levántate, vete á Sarepta de
Sidón, y allí morarás: he aquí yo he mandado allí á una mujer viuda que
te sustente. \bibverse{10} Entonces él se levantó, y se fué á Sarepta. Y
como llegó á la puerta de la ciudad, he aquí una mujer viuda que estaba
allí cogiendo serojas; y él la llamó, y díjole: Ruégote que me traigas
una poca de agua en un vaso, para que beba. \bibverse{11} Y yendo ella
para traérsela, él la volvió á llamar, y díjole: Ruégote que me traigas
también un bocado de pan en tu mano. \bibverse{12} Y ella respondió:
Vive Jehová Dios tuyo, que no tengo pan cocido; que solamente un puñado
de harina tengo en la tinaja, y un poco de aceite en una botija: y ahora
cogía dos serojas, para entrarme y aderezarlo para mí y para mi hijo, y
que lo comamos, y nos muramos. \bibverse{13} Y Elías le dijo: No hayas
temor; ve, haz como has dicho: empero hazme á mí primero de ello una
pequeña torta cocida debajo de la ceniza, y tráemela; y después harás
para ti y para tu hijo. \bibverse{14} Porque Jehová Dios de Israel ha
dicho así: La tinaja de la harina no escaseará, ni se disminuirá la
botija del aceite, hasta aquel día que Jehová dará lluvia sobre la haz
de la tierra. \bibverse{15} Entonces ella fué, é hizo como le dijo
Elías; y comió él, y ella y su casa, muchos días. \bibverse{16} Y la
tinaja de la harina no escaseó, ni menguó la botija del aceite, conforme
á la palabra de Jehová que había dicho por Elías. \bibverse{17} Después
de estas cosas aconteció que cayó enfermo el hijo del ama de la casa, y
la enfermedad fué tan grave, que no quedó en él resuello. \bibverse{18}
Y ella dijo á Elías: ¿Qué tengo yo contigo, varón de Dios? ¿has venido á
mí para traer en memoria mis iniquidades, y para hacerme morir mi hijo?
\bibverse{19} Y él le dijo: Dame acá tu hijo. Entonces él lo tomó de su
regazo, y llevólo á la cámara donde él estaba, y púsole sobre su cama;
\bibverse{20} Y clamando á Jehová, dijo: Jehová Dios mío, ¿aun á la
viuda en cuya casa yo estoy hospedado has afligido, matándole su hijo?
\bibverse{21} Y midióse sobre el niño tres veces, y clamó á Jehová, y
dijo: Jehová Dios mío, ruégote que vuelva el alma de este niño á sus
entrañas. \bibverse{22} Y Jehová oyó la voz de Elías, y el alma del niño
volvió á sus entrañas, y revivió. \bibverse{23} Tomando luego Elías al
niño, trájolo de la cámara á la casa, y diólo á su madre, y díjole
Elías: Mira, tu hijo vive. \bibverse{24} Entonces la mujer dijo á Elías:
Ahora conozco que tú eres varón de Dios, y que la palabra de Jehová es
verdad en tu boca.

\hypertarget{section-17}{%
\section{18}\label{section-17}}

\bibverse{1} Pasados muchos días, fué palabra de Jehová á Elías en el
tercer año, diciendo: Ve, muéstrate á Achâb, y yo daré lluvia sobre la
haz de la tierra. \bibverse{2} Fué pues Elías á mostrarse á Achâb. Había
á la sazón grande hambre en Samaria. \bibverse{3} Y Achâb llamó á Abdías
su mayordomo, el cual Abdías era en grande manera temeroso de Jehová;
\bibverse{4} Porque cuando Jezabel destruía á los profetas de Jehová,
Abdías tomó cien profetas, los cuales escondió de cincuenta en cincuenta
por cuevas, y sustentólos á pan y agua. \bibverse{5} Y dijo Achâb á
Abdías: Ve por el país á todas las fuentes de aguas, y á todos los
arroyos; que acaso hallaremos grama con que conservemos la vida á los
caballos y á las acémilas, para que no nos quedemos sin bestias.
\bibverse{6} Y partieron entre sí el país para recorrerlo: Achâb fué de
por sí por un camino, y Abdías fué separadamente por otro. \bibverse{7}
Y yendo Abdías por el camino, topóse con Elías; y como le conoció,
postróse sobre su rostro, y dijo: ¿No eres tú mi señor Elías?
\bibverse{8} Y él respondió: Yo soy; ve, di á tu amo: He aquí Elías.
\bibverse{9} Pero él dijo: ¿En qué he pecado, para que tú entregues tu
siervo en mano de Achâb para que me mate? \bibverse{10} Vive Jehová tu
Dios, que no ha habido nación ni reino donde mi señor no haya enviado á
buscarte; y respondiendo ellos, No está aquí, él ha conjurado á reinos y
naciones si no te han hallado. \bibverse{11} ¿Y ahora tú dices: Ve, di á
tu amo: Aquí está Elías? \bibverse{12} Y acontecerá que, luego que yo me
haya partido de ti, el espíritu de Jehová te llevará donde yo no sepa; y
viniendo yo, y dando las nuevas á Achâb, y no hallándote él, me matará;
y tu siervo teme á Jehová desde su mocedad. \bibverse{13} ¿No ha sido
dicho á mi señor lo que hice, cuando Jezabel mataba á los profetas de
Jehová: que escondí cien varones de los profetas de Jehová de cincuenta
en cincuenta en cuevas, y los mantuve á pan y agua? \bibverse{14} ¿Y
ahora dices tú: Ve, di á tu amo: Aquí está Elías: para que él me mate?
\bibverse{15} Y díjole Elías: Vive Jehová de los ejércitos, delante del
cual estoy, que hoy me mostraré á él. \bibverse{16} Entonces Abdías fué
á encontrarse con Achâb, y dióle el aviso; y Achâb vino á encontrarse
con Elías. \bibverse{17} Y como Achâb vió á Elías, díjole Achâb: ¿Eres
tú el que alborotas á Israel? \bibverse{18} Y él respondió: Yo no he
alborotado á Israel, sino tú y la casa de tu padre, dejando los
mandamientos de Jehová, y siguiendo á los Baales. \bibverse{19} Envía
pues ahora y júntame á todo Israel en el monte de Carmelo, y los
cuatrocientos y cincuenta profetas de Baal, y los cuatrocientos profetas
de los bosques, que comen de la mesa de Jezabel. \bibverse{20} Entonces
Achâb envió á todos los hijos de Israel, y juntó los profetas en el
monte de Carmelo. \bibverse{21} Y acercándose Elías á todo el pueblo,
dijo: ¿Hasta cuándo claudicaréis vosotros entre dos pensamientos? Si
Jehová es Dios, seguidle; y si Baal, id en pos de él. Y el pueblo no
respondió palabra. \bibverse{22} Y Elías tornó á decir al pueblo: Sólo
yo he quedado profeta de Jehová; mas de los profetas de Baal hay
cuatrocientos y cincuenta hombres. \bibverse{23} Dénsenos pues dos
bueyes, y escójanse ellos el uno, y córtenlo en pedazos, y pónganlo
sobre leña, mas no pongan fuego debajo; y yo aprestaré el otro buey, y
pondrélo sobre leña, y ningún fuego pondré debajo. \bibverse{24} Invocad
luego vosotros en el nombre de vuestros dioses, y yo invocaré en el
nombre de Jehová: y el Dios que respondiere por fuego, ése sea Dios. Y
todo el pueblo respondió, diciendo: Bien dicho. \bibverse{25} Entonces
Elías dijo á los profetas de Baal: Escogeos el un buey, y haced primero,
pues que vosotros sois los más: é invocad en el nombre de vuestros
dioses, mas no pongáis fuego debajo. \bibverse{26} Y ellos tomaron el
buey que les fué dado, y aprestáronlo, é invocaron en el nombre de Baal
desde la mañana hasta el medio día, diciendo: ¡Baal, respóndenos! Mas no
había voz, ni quien respondiese; entre tanto, ellos andaban saltando
cerca del altar que habían hecho. \bibverse{27} Y aconteció al medio
día, que Elías se burlaba de ellos, diciendo: Gritad en alta voz, que
dios es: quizá está conversando, ó tiene algún empeño, ó va de camino;
acaso duerme, y despertará. \bibverse{28} Y ellos clamaban á grandes
voces, y sajábanse con cuchillos y con lancetas conforme á su costumbre,
hasta chorrear la sangre sobre ellos. \bibverse{29} Y como pasó el medio
día, y ellos profetizaran hasta el tiempo del sacrificio del presente, y
no había voz, ni quien respondiese ni escuchase; \bibverse{30} Elías
dijo entonces á todo el pueblo: Acercaos á mí. Y todo el pueblo se llegó
á él: y él reparó el altar de Jehová que estaba arruinado. \bibverse{31}
Y tomando Elías doce piedras, conforme al número de las tribus de los
hijos de Jacob, al cual había sido palabra de Jehová, diciendo, Israel
será tu nombre; \bibverse{32} Edificó con las piedras un altar en el
nombre de Jehová: después hizo una reguera alrededor del altar, cuanto
cupieran dos satos de simiente. \bibverse{33} Compuso luego la leña, y
cortó el buey en pedazos, y púsolo sobre la leña. \bibverse{34} Y dijo:
Henchid cuatro cántaros de agua, y derramadla sobre el holocausto y
sobre la leña. Y dijo: Hacedlo otra vez; y otra vez lo hicieron. Dijo
aún: Hacedlo la tercera vez; é hiciéronlo la tercera vez. \bibverse{35}
De manera que las aguas corrían alrededor del altar; y había también
henchido de agua la reguera. \bibverse{36} Y como llegó la hora de
ofrecerse el holocausto, llegóse el profeta Elías, y dijo: Jehová Dios
de Abraham, de Isaac, y de Israel, sea hoy manifiesto que tú eres Dios
en Israel, y que yo soy tu siervo, y que por mandato tuyo he hecho todas
estas cosas. \bibverse{37} Respóndeme, Jehová, respóndeme; para que
conozca este pueblo que tú, oh Jehová, eres el Dios, y que tú volviste
atrás el corazón de ellos. \bibverse{38} Entonces cayó fuego de Jehová,
el cual consumió el holocausto, y la leña, y las piedras, y el polvo, y
aun lamió las aguas que estaban en la reguera. \bibverse{39} Y viéndolo
todo el pueblo, cayeron sobre sus rostros, y dijeron: ¡Jehová es el
Dios! ¡Jehová es el Dios! \bibverse{40} Y díjoles Elías: Prended á los
profetas de Baal, que no escape ninguno. Y ellos los prendieron; y
llevólos Elías al arroyo de Cisón, y allí los degolló. \bibverse{41} Y
entonces Elías dijo á Achâb: Sube, come y bebe; porque una grande lluvia
suena. \bibverse{42} Y Achâb subió á comer y á beber. Y Elías subió á la
cumbre del Carmelo; y postrándose en tierra, puso su rostro entre las
rodillas. \bibverse{43} Y dijo á su criado: Sube ahora, y mira hacia la
mar. Y él subió, y miró, y dijo: No hay nada. Y él le volvió á decir:
Vuelve siete veces. \bibverse{44} Y á la séptima vez dijo: Yo veo una
pequeña nube como la palma de la mano de un hombre, que sube de la mar.
Y él dijo: Ve, y di á Achâb: Unce y desciende, porque la lluvia no te
ataje. \bibverse{45} Y aconteció, estando en esto, que los cielos se
oscurecieron con nubes y viento; y hubo una gran lluvia. Y subiendo
Achâb, vino á Jezreel. \bibverse{46} Y la mano de Jehová fué sobre
Elías, el cual ciñó sus lomos, y vino corriendo delante de Achâb hasta
llegar á Jezreel.

\hypertarget{section-18}{%
\section{19}\label{section-18}}

\bibverse{1} Y achab dió la nueva á Jezabel de todo lo que Elías había
hecho, de como había muerto á cuchillo á todos los profetas.
\bibverse{2} Entonces envió Jezabel á Elías un mensajero, diciendo: Así
me hagan los dioses, y así me añadan, si mañana á estas horas yo no haya
puesto tu persona como la de uno de ellos. \bibverse{3} Viendo pues el
peligro, levantóse y fuése por salvar su vida, y vino á Beer-seba, que
es en Judá, y dejó allí su criado. \bibverse{4} Y él se fué por el
desierto un día de camino, y vino y sentóse debajo de un enebro; y
deseando morirse, dijo: Baste ya, oh Jehová, quita mi alma; que no soy
yo mejor que mis padres. \bibverse{5} Y echándose debajo del enebro,
quedóse dormido: y he aquí luego un ángel que le tocó, y le dijo:
Levántate, come. \bibverse{6} Entonces él miró, y he aquí á su cabecera
una torta cocida sobre las ascuas, y un vaso de agua: y comió y bebió, y
volvióse á dormir. \bibverse{7} Y volviendo el ángel de Jehová la
segunda vez, tocóle, diciendo: Levántate, come: porque gran camino te
resta. \bibverse{8} Levantóse pues, y comió y bebió; y caminó con la
fortaleza de aquella comida cuarenta días y cuarenta noches, hasta el
monte de Dios, Horeb. \bibverse{9} Y allí se metió en una cueva, donde
tuvo la noche. Y fué á él palabra de Jehová, el cual le dijo: ¿Qué haces
aquí, Elías? \bibverse{10} Y él respondió: Sentido he un vivo celo por
Jehová Dios de los ejércitos; porque los hijos de Israel han dejado tu
alianza, han derribado tus altares, y han muerto á cuchillo tus
profetas: y yo solo he quedado, y me buscan para quitarme la vida.
\bibverse{11} Y él le dijo: Sal fuera, y ponte en el monte delante de
Jehová. Y he aquí Jehová que pasaba, y un grande y poderoso viento que
rompía los montes, y quebraba las peñas delante de Jehová: mas Jehová no
estaba en el viento. Y tras el viento un terremoto: mas Jehová no estaba
en el terremoto. \bibverse{12} Y tras el terremoto un fuego: mas Jehová
no estaba en el fuego. Y tras el fuego un silbo apacible y delicado.
\bibverse{13} Y cuando lo oyó Elías, cubrió su rostro con su manto, y
salió, y paróse á la puerta de la cueva. Y he aquí llegó una voz á él,
diciendo: ¿Qué haces aquí, Elías? \bibverse{14} Y él respondió: Sentido
he un vivo celo por Jehová Dios de los ejércitos; porque los hijos de
Israel han dejado tu alianza, han derribado tus altares, y han muerto á
cuchillo tus profetas: y yo solo he quedado, y me buscan para quitarme
la vida. \bibverse{15} Y díjole Jehová: Ve, vuélvete por tu camino, por
el desierto de Damasco: y llegarás, y ungirás á Hazael por rey de Siria;
\bibverse{16} Y á Jehú hijo de Nimsi, ungirás por rey sobre Israel; y á
Eliseo hijo de Saphat, de Abelmehula, ungirás para que sea profeta en
lugar de ti. \bibverse{17} Y será, que el que escapare del cuchillo de
Hazael, Jehú lo matará; y el que escapare del cuchillo de Jehú, Eliseo
lo matará. \bibverse{18} Y yo haré que queden en Israel siete mil; todas
rodillas que no se encorvaron á Baal, y bocas todas que no lo besaron.
\bibverse{19} Y partiéndose él de allí, halló á Eliseo hijo de Saphat,
que araba con doce yuntas delante de sí; y él era uno de los doce
gañanes. Y pasando Elías por delante de él, echó sobre él su manto.
\bibverse{20} Entonces dejando él los bueyes, vino corriendo en pos de
Elías, y dijo: Ruégote que me dejes besar mi padre y mi madre, y luego
te seguiré. Y él le dijo: Ve, vuelve: ¿qué te he hecho yo? \bibverse{21}
Y volvióse de en pos de él, y tomó un par de bueyes, y matólos, y con el
arado de los bueyes coció la carne de ellos, y dióla al pueblo que
comiesen. Después se levantó, y fué tras Elías, y servíale.

\hypertarget{section-19}{%
\section{20}\label{section-19}}

\bibverse{1} Entonces Ben-adad rey de Siria juntó á todo su ejército, y
con él treinta y dos reyes, con caballos y carros: y subió, y puso cerco
á Samaria, y combatióla. \bibverse{2} Y envió mensajeros á la ciudad á
Achâb rey de Israel, diciendo: \bibverse{3} Así ha dicho Ben-adad: Tu
plata y tu oro es mío, y tus mujeres y tus hijos hermosos son míos.
\bibverse{4} Y el rey de Israel respondió, y dijo: Como tú dices, rey
señor mío, yo soy tuyo, y todo lo que tengo. \bibverse{5} Y volviendo
los mensajeros otra vez, dijeron: Así dijo Ben-adad: Yo te envié á
decir: Tu plata y tu oro, y tus mujeres y tus hijos me darás.
\bibverse{6} Además mañana á estas horas enviaré yo á ti mis siervos,
los cuales escudriñarán tu casa, y las casas de tus siervos; y tomarán
con sus manos, y llevarán todo lo precioso que tuvieres. \bibverse{7}
Entonces el rey de Israel llamó á todos los ancianos de la tierra, y
díjoles: Entended, y ved ahora cómo éste no busca sino mal: pues que ha
enviado á mí por mis mujeres y mis hijos, y por mi plata y por mi oro; y
yo no se lo he negado. \bibverse{8} Y todos los ancianos y todo el
pueblo le respondieron: No le obedezcas, ni hagas lo que te pide.
\bibverse{9} Entonces él respondió á los embajadores de Ben-adad: Decid
al rey mi señor: Haré todo lo que mandaste á tu siervo al principio; mas
esto no lo puedo hacer. Y los embajadores fueron, y diéronle la
respuesta. \bibverse{10} Y Ben-adad tornó á enviarle á decir: Así me
hagan los dioses, y así me añadan, que el polvo de Samaria no bastará á
los puños de todo el pueblo que me sigue. \bibverse{11} Y el rey de
Israel respondió, y dijo: Decidle, que no se alabe el que se ciñe, como
el que ya se desciñe. \bibverse{12} Y como él oyó esta palabra, estando
bebiendo con los reyes en las tiendas, dijo á sus siervos: Poned. Y
ellos pusieron contra la ciudad. \bibverse{13} Y he aquí un profeta se
llegó á Achâb rey de Israel, y le dijo: Así ha dicho Jehová: ¿Has visto
esta grande multitud? he aquí yo te la entregaré hoy en tu mano, para
que conozcas que yo soy Jehová. \bibverse{14} Y respondió Achâb: ¿Por
mano de quién? Y él dijo: Así ha dicho Jehová: Por mano de los criados
de los príncipes de las provincias. Y dijo Achâb: ¿Quién comenzará la
batalla? Y él respondió: Tú. \bibverse{15} Entonces él reconoció los
criados de los príncipes de las provincias, los cuales fueron doscientos
treinta y dos. Luego reconoció todo el pueblo, todos los hijos de
Israel, que fueron siete mil. \bibverse{16} Y salieron á medio día. Y
estaba Ben-adad bebiendo, borracho en las tiendas, él y los reyes, los
treinta y dos reyes que habían venido en su ayuda. \bibverse{17} Y los
criados de los príncipes de las provincias salieron los primeros. Y
había Ben-adad enviado quien le dió aviso, diciendo: Han salido hombres
de Samaria. \bibverse{18} El entonces dijo: Si han salido por paz,
tomadlos vivos; y si han salido para pelear, tomadlos vivos.
\bibverse{19} Salieron pues de la ciudad los criados de los príncipes de
las provincias, y en pos de ellos el ejército. \bibverse{20} E hirió
cada uno al que venía contra sí: y huyeron los Siros, siguiéndolos los
de Israel. Y el rey de Siria, Ben-adad, se escapó en un caballo con
alguna gente de caballería. \bibverse{21} Y salió el rey de Israel, é
hirió la gente de á caballo, y los carros; y deshizo los Siros con
grande estrago. \bibverse{22} Llegándose luego el profeta al rey de
Israel, le dijo: Ve, fortalécete, y considera y mira lo que has de
hacer; porque pasado el año, el rey de Siria ha de venir contra ti.
\bibverse{23} Y los siervos del rey de Siria le dijeron: Sus dioses son
dioses de los montes, por eso nos han vencido; mas si peleáremos con
ellos en la llanura, se verá si no los vencemos. \bibverse{24} Haz pues
así: Saca á los reyes cada uno de su puesto, y pon capitanes en lugar de
ellos. \bibverse{25} Y tú, fórmate otro ejército como el ejército que
perdiste, caballos por caballos, y carros por carros; luego pelearemos
con ellos en campo raso, y veremos si no los vencemos. Y él les dió
oído, é hízolo así. \bibverse{26} Pasado el año, Ben-adad reconoció los
Siros, y vino á Aphec á pelear contra Israel. \bibverse{27} Y los hijos
de Israel fueron también inspeccionados, y tomando provisiones fuéronles
al encuentro; y asentaron campo lo hijos de Israel delante de ellos,
como dos rebañuelos de cabras; y los Siros henchían la tierra.
\bibverse{28} Llegándose entonces el varón de Dios al rey de Israel,
hablóle diciendo: Así dijo Jehová: Por cuanto los Siros han dicho,
Jehová es Dios de los montes, no Dios de los valles, yo entregaré toda
esta grande multitud en tu mano, para que conozcáis que yo soy Jehová.
\bibverse{29} Siete días tuvieron asentado campo los unos delante de los
otros, y al séptimo día se dió la batalla: y mataron los hijos de Israel
de los Siros en un día cien mil hombres de á pie. \bibverse{30} Los
demás huyeron á Aphec, á la ciudad: y el muro cayó sobre veinte y siete
mil hombres que habían quedado. También Ben-adad vino huyendo á la
ciudad, y escondíase de cámara en cámara. \bibverse{31} Entonces sus
siervos le dijeron: He aquí, hemos oído de los reyes de la casa de
Israel que son reyes clementes: pongamos pues ahora sacos en nuestros
lomos, y sogas en nuestras cabezas, y salgamos al rey de Israel: por
ventura te salvará la vida. \bibverse{32} Ciñeron pues sus lomos de
sacos, y sogas á sus cabezas, y vinieron al rey de Israel, y dijéronle:
Tu siervo Ben-adad dice: Ruégote que viva mi alma. Y él respondió: Si él
vive aún, mi hermano es. \bibverse{33} Esto tomaron aquellos hombres por
buen agüero, y presto tomaron esta palabra de su boca, y dijeron: ¡Tu
hermano Ben-adad! Y él dijo: Id, y traedle. Ben-adad entonces se
presentó á Achâb, y él le hizo subir en un carro. \bibverse{34} Y díjole
Ben-adad: Las ciudades que mi padre tomó al tuyo, yo las restituiré; y
haz plazas en Damasco para ti, como mi padre las hizo en Samaria. Y yo,
dijo Achâb, te dejaré partir con esta alianza. Hizo pues con él alianza,
y dejóle ir. \bibverse{35} Entonces un varón de los hijos de los
profetas dijo á su compañero por palabra de Dios: Hiéreme ahora. Mas el
otro varón no quiso herirle. \bibverse{36} Y él le dijo: Por cuanto no
has obedecido á la palabra de Jehová, he aquí en apartándote de mí, te
herirá un león. Y como se apartó de él, topóle un león, é hirióle.
\bibverse{37} Encontróse luego con otro hombre, y díjole: Hiéreme ahora.
Y el hombre le dió un golpe, é hízole una herida. \bibverse{38} Y el
profeta se fué, y púsose delante del rey en el camino, y disfrazóse con
un velo sobre los ojos. \bibverse{39} Y como el rey pasaba, él dió voces
al rey, y dijo: Tu siervo salió entre la tropa: y he aquí apartándose
uno, trájome un hombre, diciendo: Guarda á este hombre, y si llegare á
faltar, tu vida será por la suya, ó pagarás un talento de plata.
\bibverse{40} Y como tu siervo estaba ocupado á una parte y á otra, él
desapareció. Entonces el rey de Israel le dijo: Esa será tu sentencia:
tú la has pronunciado. \bibverse{41} Pero él se quitó de presto el velo
de sobre sus ojos, y el rey de Israel conoció que era de los profetas.
\bibverse{42} Y él le dijo: Así ha dicho Jehová: Por cuanto soltaste de
la mano el hombre de mi anatema, tu vida será por la suya, y tu pueblo
por el suyo. \bibverse{43} Y el rey de Israel se fué á su casa triste y
enojado, y llegó á Samaria.

\hypertarget{section-20}{%
\section{21}\label{section-20}}

\bibverse{1} Pasados estos negocios, aconteció que Naboth de Jezreel
tenía en Jezreel una viña junto al palacio de Achâb rey de Samaria.
\bibverse{2} Y Achâb habló á Naboth, diciendo: Dame tu viña para un
huerto de legumbres, porque está cercana, junto á mi casa, y yo te daré
por ella otra viña mejor que esta; ó si mejor te pareciere, te pagaré su
valor en dinero. \bibverse{3} Y Naboth respondió á Achâb: Guárdeme
Jehová de que yo te dé á ti la heredad de mis padres. \bibverse{4} Y
vínose Achâb á su casa triste y enojado, por la palabra que Naboth de
Jezreel le había respondido, diciendo: No te daré la heredad de mis
padres. Y acostóse en su cama, y volvió su rostro, y no comió pan.
\bibverse{5} Y vino á él su mujer Jezabel, y díjole: ¿Por qué está tan
triste tu espíritu, y no comes pan? \bibverse{6} Y él respondió: Porque
hablé con Naboth de Jezreel, y díjele que me diera su viña por dinero, ó
que, si más quería, le daría otra viña por ella; y él respondió: Yo no
te daré mi viña. \bibverse{7} Y su mujer Jezabel le dijo: ¿Eres tú ahora
rey sobre Israel? Levántate, y come pan, y alégrate: yo te daré la viña
de Naboth de Jezreel. \bibverse{8} Entonces ella escribió cartas en
nombre de Achâb, y sellólas con su anillo y enviólas á los ancianos y á
los principales que moraban en su ciudad con Naboth. \bibverse{9} Y las
cartas que escribió decían así: Proclamad ayuno, y poned á Naboth á la
cabecera del pueblo; \bibverse{10} Y poned dos hombres perversos delante
de él, que atestigüen contra él, y digan: Tú has blasfemado á Dios y al
rey. Y entonces sacadlo, y apedreadlo, y muera. \bibverse{11} Y los de
su ciudad, los ancianos y los principales que moraban en su ciudad, lo
hicieron como Jezabel les mandó, conforme á lo escrito en las cartas que
ella les había enviado. \bibverse{12} Y promulgaron ayuno, y asentaron á
Naboth á la cabecera del pueblo. \bibverse{13} Vinieron entonces dos
hombres perversos, y sentáronse delante de él: y aquellos hombres de
Belial atestiguaron contra Naboth delante del pueblo, diciendo: Naboth
ha blasfemado á Dios y al rey. Y sacáronlo fuera de la ciudad, y
apedreáronlo con piedras, y murió. \bibverse{14} Después enviaron á
decir á Jezabel: Naboth ha sido apedreado y muerto. \bibverse{15} Y como
Jezabel oyó que Naboth había sido apedreado y muerto, dijo á Achâb:
Levántate y posee la viña de Naboth de Jezreel, que no te la quiso dar
por dinero; porque Naboth no vive, sino que es muerto. \bibverse{16} Y
oyendo Achâb que Naboth era muerto, levantóse para descender á la viña
de Naboth de Jezreel, para tomar posesión de ella. \bibverse{17}
Entonces fué palabra de Jehová á Elías Thisbita, diciendo: \bibverse{18}
Levántate, desciende á encontrarte con Achâb rey de Israel, que está en
Samaria: he aquí él está en la viña de Naboth, á la cual ha descendido
para tomar posesión de ella. \bibverse{19} Y hablarle has, diciendo: Así
ha dicho Jehová: ¿No mataste y también has poseído? Y tornarás á
hablarle, diciendo: Así ha dicho Jehová: En el mismo lugar donde
lamieron los perros la sangre de Naboth, los perros lamerán también tu
sangre, la tuya misma. \bibverse{20} Y Achâb dijo á Elías: ¿Me has
hallado, enemigo mío? Y él respondió: Hete encontrado, porque te has
vendido á mal hacer delante de Jehová. \bibverse{21} He aquí yo traigo
mal sobre ti, y barreré tu posteridad, y talaré de Achâb todo meante á
la pared, al guardado y al desamparado en Israel: \bibverse{22} Y yo
pondré tu casa como la casa de Jeroboam hijo de Nabat, y como la casa de
Baasa hijo de Ahía; por la provocación con que me provocaste á ira, y
con que has hecho pecar á Israel. \bibverse{23} De Jezabel también ha
hablado Jehová, diciendo: Los perros comerán á Jezabel en la barbacana
de Jezreel. \bibverse{24} El que de Achâb fuere muerto en la ciudad,
perros le comerán: y el que fuere muerto en el campo, comerlo han las
aves del cielo. \bibverse{25} (A la verdad ninguno fué como Achâb, que
se vendiese á hacer lo malo á los ojos de Jehová; porque Jezabel su
mujer lo incitaba. \bibverse{26} El fué en grande manera abominable,
caminando en pos de los ídolos, conforme á todo lo que hicieron los
Amorrheos, á los cuales lanzó Jehová delante de los hijos de Israel.)
\bibverse{27} Y acaeció cuando Achâb oyó estas palabras, que rasgó sus
vestidos, y puso saco sobre su carne, y ayunó, y durmió en saco, y
anduvo humillado. \bibverse{28} Entonces fué palabra de Jehová á Elías
Thisbita, diciendo: \bibverse{29} ¿No has visto como Achâb se ha
humillado delante de mí? Pues por cuanto se ha humillado delante de mí,
no traeré el mal en sus días: en los días de su hijo traeré el mal sobre
su casa.

\hypertarget{section-21}{%
\section{22}\label{section-21}}

\bibverse{1} Tres años pasaron sin guerra entre los Siros é Israel.
\bibverse{2} Y aconteció al tercer año, que Josaphat rey de Judá
descendió al rey de Israel. \bibverse{3} Y el rey de Israel dijo á sus
siervos: ¿No sabéis que es nuestra Ramoth de Galaad? y nosotros callamos
en orden á tomarla de mano del rey de Siria. \bibverse{4} Y dijo á
Josaphat: ¿Quieres venir conmigo á pelear contra Ramoth de Galaad? Y
Josaphat respondió al rey de Israel: Como yo, así tú; y como mi pueblo,
así tu pueblo; y como mis caballos, tus caballos. \bibverse{5} Y dijo
luego Josaphat al rey de Israel: Yo te ruego que consultes hoy la
palabra de Jehová. \bibverse{6} Entonces el rey de Israel juntó los
profetas, como cuatrocientos hombres, á los cuales dijo: ¿Iré á la
guerra contra Ramoth de Galaad, ó la dejaré? Y ellos dijeron: Sube;
porque el Señor la entregará en mano del rey. \bibverse{7} Y dijo
Josaphat: ¿Hay aún aquí algún profeta de Jehová, por el cual
consultemos? \bibverse{8} Y el rey de Israel respondió á Josaphat: Aun
hay un varón por el cual podríamos consultar á Jehová, Michêas, hijo de
Imla: mas yo le aborrezco, porque nunca me profetiza bien, sino
solamente mal. Y Josaphat dijo: No hable el rey así. \bibverse{9}
Entonces el rey de Israel llamó á un eunuco, y díjole: trae presto á
Michêas hijo de Imla. \bibverse{10} Y el rey de Israel y Josaphat rey de
Judá estaban sentados cada uno en su silla, vestidos de sus ropas
reales, en la plaza junto á la entrada de la puerta de Samaria; y todos
los profetas profetizaban delante de ellos. \bibverse{11} Y Sedechîas
hijo de Chânaana se había hecho unos cuernos de hierro, y dijo: Así ha
dicho Jehová: Con éstos acornearás á los Siros hasta acabarlos.
\bibverse{12} Y todos los profetas profetizaban de la misma manera,
diciendo: Sube á Ramoth de Galaad, y serás prosperado; que Jehová la
dará en mano del rey. \bibverse{13} Y el mensajero que había ido á
llamar á Michêas, hablóle, diciendo: He aquí las palabras de los
profetas á una boca anuncian al rey bien: sea ahora tu palabra conforme
á la palabra de alguno de ellos, y anuncia bien. \bibverse{14} Y Michêas
respondió: Vive Jehová, que lo que Jehová me hablare, eso diré.
\bibverse{15} Vino pues al rey, y el rey le dijo: Michêas, ¿iremos á
pelear contra Ramoth de Galaad, ó la dejaremos? Y él le respondió: Sube,
que serás prosperado, y Jehová la entregará en mano del rey.
\bibverse{16} Y el rey le dijo: ¿Hasta cuántas veces he de conjurarte
que no me digas sino la verdad en el nombre de Jehová? \bibverse{17}
Entonces él dijo: Yo ví á todo Israel esparcido por los montes, como
ovejas que no tienen pastor: y Jehová dijo: Estos no tienen señor;
vuélvase cada uno á su casa en paz. \bibverse{18} Y el rey de Israel
dijo á Josaphat: ¿No te lo había yo dicho? Ninguna cosa buena
profetizará él acerca de mí, sino solamente mal. \bibverse{19} Entonces
él dijo: Oye pues palabra de Jehová: Yo vi á Jehová sentado en su trono,
y todo el ejército de los cielos estaba junto á él, á su diestra y á su
siniestra. \bibverse{20} Y Jehová dijo: ¿Quién inducirá á Achâb, para
que suba y caiga en Ramoth de Galaad? Y uno decía de una manera; y otro
decía de otra. \bibverse{21} Y salió un espíritu, y púsose delante de
Jehová, y dijo: Yo le induciré. Y Jehová le dijo: ¿De qué manera?
\bibverse{22} Y él dijo: Yo saldré, y seré espíritu de mentira en boca
de todos sus profetas. Y él dijo: Inducirlo has, y aun saldrás con ello;
sal pues, y hazlo así. \bibverse{23} Y ahora, he aquí Jehová ha puesto
espíritu de mentira en la boca de todos estos tus profetas, y Jehová ha
decretado el mal acerca de ti. \bibverse{24} Llegándose entonces
Sedechîas hijo de Chânaana, hirió á Michêas en la mejilla, diciendo:
¿Por dónde se fué de mí el espíritu de Jehová para hablarte á ti?
\bibverse{25} Y Michêas respondió: He aquí tú lo verás en aquel día,
cuando te irás metiendo de cámara en cámara por esconderte.
\bibverse{26} Entonces el rey de Israel dijo: Toma á Michêas, y vuélvelo
á Amón gobernador de la ciudad, y á Joas hijo del rey; \bibverse{27} Y
dirás: Así ha dicho el rey: Echad á éste en la cárcel, y mantenedle con
pan de angustia y con agua de aflicción, hasta que yo vuelva en paz.
\bibverse{28} Y dijo Michêas: Si llegares á volver en paz, Jehová no ha
hablado por mí. En seguida dijo: Oid, pueblos todos. \bibverse{29} Subió
pues el rey de Israel con Josaphat rey de Judá á Ramoth de Galaad.
\bibverse{30} Y el rey de Israel dijo á Josaphat: Yo me disfrazaré, y
entraré en la batalla: y tú vístete tus vestidos. Y el rey de Israel se
disfrazó, y entró en la batalla. \bibverse{31} Mas el rey de Siria había
mandado á sus treinta y dos capitanes de los carros, diciendo: No
peleéis vosotros ni con grande ni con chico, sino sólo contra el rey de
Israel. \bibverse{32} Y como los capitanes de los carros vieron á
Josaphat, dijeron: Ciertamente éste es el rey de Israel; y viniéronse á
él para pelear con él; mas el rey Josaphat dió voces. \bibverse{33}
Viendo entonces los capitanes de los carros que no era el rey de Israel,
apartáronse de él. \bibverse{34} Y un hombre disparando su arco á la
ventura, hirió al rey de Israel por entre las junturas de la armadura;
por lo que dijo él á su carretero: Toma la vuelta, y sácame del campo,
que estoy herido. \bibverse{35} Mas la batalla había arreciado aquel
día, y el rey estuvo en su carro delante de los Siros, y á la tarde
murió: y la sangre de la herida corría por el seno del carro.
\bibverse{36} Y á puesta del sol salió un pregón por el campo, diciendo:
¡Cada uno á su ciudad, y cada cual á su tierra! \bibverse{37} Y murió
pues el rey, y fué traído á Samaria; y sepultaron al rey en Samaria.
\bibverse{38} Y lavaron el carro en el estanque de Samaria; lavaron
también sus armas; y los perros lamieron su sangre, conforme á la
palabra de Jehová que había hablado. \bibverse{39} Lo demás de los
hechos de Achâb, y todas las cosas que ejecutó, y la casa de marfil que
hizo, y todas las ciudades que edificó, ¿no está escrito en el libro de
las crónicas de los reyes de Israel? \bibverse{40} Y durmió Achâb con
sus padres, y reinó en su lugar Ochôzías su hijo. \bibverse{41} Y
Josaphat hijo de Asa comenzó á reinar sobre Judá en el cuarto año de
Achâb rey de Israel. \bibverse{42} Y era Josaphat de treinta y cinco
años cuando comenzó á reinar, y reinó veinticinco años en Jerusalem. El
nombre de su madre fué Azuba hija de Silai. \bibverse{43} Y anduvo en
todo el camino de Asa su padre, sin declinar de él, haciendo lo recto en
los ojos de Jehová. \bibverse{44} Con todo eso los altos no fueron
quitados; que el pueblo sacrificaba aún, y quemaba perfumes en los
altos. \bibverse{45} Y Josaphat hizo paz con el rey de Israel.
\bibverse{46} Lo demás de los hechos de Josaphat, y sus hazañas, y las
guerras que hizo, ¿no está escrito en el libro de las crónicas de los
reyes de Judá? \bibverse{47} Barrió también de la tierra el resto de los
sodomitas que habían quedado en el tiempo de su padre Asa. \bibverse{48}
No había entonces rey en Edom; presidente había en lugar de rey.
\bibverse{49} Había Josaphat hecho navíos en Tharsis, los cuales habían
de ir á Ophir por oro; mas no fueron, porque se rompieron en
Ezion-geber. \bibverse{50} Entonces Ochôzías hijo de Achâb dijo á
Josaphat: Vayan mis siervos con los tuyos en los navíos. Mas Josaphat no
quiso. \bibverse{51} Y durmió Josaphat con sus padres, y fué sepultado
con sus padres en la ciudad de David su padre; y en su lugar reinó Joram
su hijo. \bibverse{52} Y Ochôzías hijo de Achâb comenzó á reinar sobre
Israel en Samaria, el año diecisiete de Josaphat rey de Judá; y reinó
dos años sobre Israel. \bibverse{53} E hizo lo malo en los ojos de
Jehová, y anduvo en el camino de su padre, y en el camino de su madre, y
en el camino de Jeroboam hijo de Nabat, que hizo pecar á Israel: Porque
sirvió á Baal, y lo adoró, y provocó á ira á Jehová Dios de Israel,
conforme á todas las cosas que su padre había hecho.
