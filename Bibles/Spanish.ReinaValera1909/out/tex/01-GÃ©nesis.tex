\hypertarget{la-creaciuxf3n-del-mundo}{%
\subsection{La creación del mundo}\label{la-creaciuxf3n-del-mundo}}

\hypertarget{section}{%
\section{1}\label{section}}

\bibverse{1} EN el principio crió Dios los cielos y la tierra.
\footnote{\textbf{1:1} Hech 17,24; Apoc 4,11; Heb 11,3; Juan 1,1-3}
\bibverse{2} Y la tierra estaba desordenada y vacía, y las tinieblas
estaban sobre la haz del abismo, y el Espíritu de Dios se movía sobre la
haz de las aguas.

\hypertarget{la-creaciuxf3n-de-la-luz}{%
\subsection{La creación de la luz}\label{la-creaciuxf3n-de-la-luz}}

\bibverse{3} Y dijo Dios: Sea la luz: y fué la luz. \bibverse{4} Y vió
Dios que la luz era buena: y apartó Dios la luz de las tinieblas.
\bibverse{5} Y llamó Dios á la luz Día, y á las tinieblas llamó Noche: y
fué la tarde y la mañana un día.

\hypertarget{la-creaciuxf3n-de-la-expansion-de-los-cielos}{%
\subsection{La creación de la expansion de los
Cielos}\label{la-creaciuxf3n-de-la-expansion-de-los-cielos}}

\bibverse{6} Y dijo Dios: Haya expansión en medio de las aguas, y separe
las aguas de las aguas. \bibverse{7} E hizo Dios la expansión, y apartó
las aguas que estaban debajo de la expansión, de las aguas que estaban
sobre la expansión: y fué así. \footnote{\textbf{1:7} Sal 19,2}
\bibverse{8} Y llamó Dios á la expansión Cielos: y fué la tarde y la
mañana el día segundo.

\hypertarget{la-separacion-de-la-tierra-y-el-mar-y-la-creaciuxf3n-de-las-plantas}{%
\subsection{La separacion de la tierra y el mar y la creación de las
plantas}\label{la-separacion-de-la-tierra-y-el-mar-y-la-creaciuxf3n-de-las-plantas}}

\bibverse{9} Y dijo Dios: Júntense las aguas que están debajo de los
cielos en un lugar, y descúbrase la seca: y fué así. \bibverse{10} Y
llamó Dios á la seca Tierra, y á la reunión de las aguas llamó Mares: y
vió Dios que era bueno. \bibverse{11} Y dijo Dios: Produzca la tierra
hierba verde, hierba que dé simiente; árbol de fruto que dé fruto según
su género, que su simiente esté en él, sobre la tierra: y fué así.
\bibverse{12} Y produjo la tierra hierba verde, hierba que da simiente
según su naturaleza, y árbol que da fruto, cuya simiente está en él,
según su género: y vió Dios que era bueno. \bibverse{13} Y fué la tarde
y la mañana el día tercero.

\hypertarget{la-creaciuxf3n-de-las-estrellas}{%
\subsection{La creación de las
estrellas}\label{la-creaciuxf3n-de-las-estrellas}}

\bibverse{14} Y dijo Dios: Sean lumbreras en la expansión de los cielos
para apartar el día y la noche: y sean por señales, y para las
estaciones, y para días y años; \footnote{\textbf{1:14} Sal 74,16}
\bibverse{15} Y sean por lumbreras en la expansión de los cielos para
alumbrar sobre la tierra: y fué así. \bibverse{16} E hizo Dios las dos
grandes lumbreras; la lumbrera mayor para que señorease en el día, y la
lumbrera menor para que señorease en la noche: hizo también las
estrellas. \bibverse{17} Y púsolas Dios en la expansión de los cielos,
para alumbrar sobre la tierra, \bibverse{18} Y para señorear en el día y
en la noche, y para apartar la luz y las tinieblas: y vió Dios que era
bueno. \bibverse{19} Y fué la tarde y la mañana el día cuarto.

\hypertarget{la-creaciuxf3n-de-los-animales-acuuxe1ticos-y-de-los-aves}{%
\subsection{La creación de los animales acuáticos y de los
aves}\label{la-creaciuxf3n-de-los-animales-acuuxe1ticos-y-de-los-aves}}

\bibverse{20} Y dijo Dios: Produzcan las aguas reptil de ánima viviente,
y aves que vuelen sobre la tierra, en la abierta expansión de los
cielos. \bibverse{21} Y crió Dios las grandes ballenas, y toda cosa viva
que anda arrastrando, que las aguas produjeron según su género, y toda
ave alada según su especie: y vió Dios que era bueno. \bibverse{22} Y
Dios los bendijo diciendo: Fructificad y multiplicad, y henchid las
aguas en los mares, y las aves se multipliquen en la tierra.
\bibverse{23} Y fué la tarde y la mañana el día quinto.

\hypertarget{la-creaciuxf3n-de-los-animales-terrestres-y-del-hombre}{%
\subsection{La creación de los animales terrestres y del
hombre}\label{la-creaciuxf3n-de-los-animales-terrestres-y-del-hombre}}

\bibverse{24} Y dijo Dios: Produzca la tierra seres vivientes según su
género, bestias y serpientes y animales de la tierra según su especie: y
fué así. \bibverse{25} E hizo Dios animales de la tierra según su
género, y ganado según su género, y todo animal que anda arrastrando
sobre la tierra según su especie: y vió Dios que era bueno.

\bibverse{26} Y dijo Dios: Hagamos al hombre á nuestra imagen, conforme
á nuestra semejanza; y señoree en los peces de la mar, y en las aves de
los cielos, y en las bestias, y en toda la tierra, y en todo animal que
anda arrastrando sobre la tierra. \footnote{\textbf{1:26} Sal 8,6-9}
\bibverse{27} Y crió Dios al hombre á su imagen, á imagen de Dios lo
crió; varón y hembra los crió. \footnote{\textbf{1:27} Efes 4,24; Gén
  2,7; Gén 2,22; Mat 19,4} \bibverse{28} Y los bendijo Dios; y díjoles
Dios: Fructificad y multiplicad, y henchid la tierra, y sojuzgadla, y
señoread en los peces de la mar, y en las aves de los cielos, y en todas
las bestias que se mueven sobre la tierra. \footnote{\textbf{1:28} Hech
  17,26} \bibverse{29} Y dijo Dios: He aquí que os he dado toda hierba
que da simiente, que está sobre la haz de toda la tierra; y todo árbol
en que hay fruto de árbol que da simiente, seros ha para comer.
\bibverse{30} Y á toda bestia de la tierra, y á todas las aves de los
cielos, y á todo lo que se mueve sobre la tierra, en que hay vida, toda
hierba verde les será para comer: y fué así.

\bibverse{31} Y vió Dios todo lo que había hecho, y he aquí que era
bueno en gran manera. Y fué la tarde y la mañana el día sexto.

\hypertarget{el-dia-de-reposo}{%
\subsection{El dia de reposo}\label{el-dia-de-reposo}}

\hypertarget{section-1}{%
\section{2}\label{section-1}}

\bibverse{1} Y FUERON acabados los cielos y la tierra, y todo su
ornamento. \bibverse{2} Y acabó Dios en el día séptimo su obra que hizo,
y reposó el día séptimo de toda su obra que había hecho. \bibverse{3} Y
bendijo Dios al día séptimo, y santificólo, porque en él reposó de toda
su obra que había Dios criado y hecho. \footnote{\textbf{2:3} Éxod
  20,8-11}

\hypertarget{la-creacion-del-hombre-y-de-la-mujer-en-el-parauxedso}{%
\subsection{La creacion del hombre y de la mujer en el
paraíso}\label{la-creacion-del-hombre-y-de-la-mujer-en-el-parauxedso}}

\bibverse{4} Estos son los orígenes de los cielos y de la tierra cuando
fueron criados, el día que Jehová Dios hizo la tierra y los cielos,
\bibverse{5} Y toda planta del campo antes que fuese en la tierra, y
toda hierba del campo antes que naciese: porque aun no había Jehová Dios
hecho llover sobre la tierra, ni había hombre para que labrase la
tierra; \bibverse{6} Mas subía de la tierra un vapor, que regaba toda la
faz de la tierra. \bibverse{7} Formó, pues, Jehová Dios al hombre del
polvo de la tierra, y alentó en su nariz soplo de vida; y fué el hombre
en alma viviente.

\bibverse{8} Y había Jehová Dios plantado un huerto en Edén al oriente,
y puso allí al hombre que había formado. \bibverse{9} Y había Jehová
Dios hecho nacer de la tierra todo árbol delicioso á la vista, y bueno
para comer: también el árbol de vida en medio del huerto, y el árbol de
ciencia del bien y del mal. \footnote{\textbf{2:9} Gén 3,22; Gén 3,24;
  Apoc 2,7; Apoc 22,2}

\hypertarget{el-ruxedo-en-el-parauxedso-y-sus-ramales}{%
\subsection{El río en el paraíso y sus
ramales}\label{el-ruxedo-en-el-parauxedso-y-sus-ramales}}

\bibverse{10} Y salía de Edén un río para regar el huerto, y de allí se
repartía en cuatro ramales. \bibverse{11} El nombre del uno era Pisón:
éste es el que cerca toda la tierra de Havilah, donde hay oro:
\bibverse{12} Y el oro de aquella tierra es bueno: hay allí también
bdelio y piedra cornerina. \bibverse{13} El nombre del segundo río es
Gihón: éste es el que rodea toda la tierra de Etiopía. \bibverse{14} Y
el nombre del tercer río es Hiddekel: éste es el que va delante de
Asiria. Y el cuarto río es el Eufrates.

\hypertarget{el-mandamiento-de-dios-por-adam}{%
\subsection{El mandamiento de Dios por
Adam}\label{el-mandamiento-de-dios-por-adam}}

\bibverse{15} Tomó, pues, Jehová Dios al hombre, y le puso en el huerto
de Edén, para que lo labrara y lo guardase. \bibverse{16} Y mandó Jehová
Dios al hombre, diciendo: De todo árbol del huerto comerás;
\bibverse{17} Mas del árbol de ciencia del bien y del mal no comerás de
él; porque el día que de él comieres, morirás.

\hypertarget{la-creaciuxf3n-de-la-mujer-y-la-fundaciuxf3n-del-matrimonio}{%
\subsection{La creación de la mujer y la fundación del
matrimonio}\label{la-creaciuxf3n-de-la-mujer-y-la-fundaciuxf3n-del-matrimonio}}

\bibverse{18} Y dijo Jehová Dios: No es bueno que el hombre esté solo;
haréle ayuda idónea para él. \footnote{\textbf{2:18} Prov 31,10-31}
\bibverse{19} Formó, pues, Jehová Dios de la tierra toda bestia del
campo, y toda ave de los cielos, y trájolas á Adam, para que viese cómo
les había de llamar; y todo lo que Adam llamó á los animales vivientes,
ese es su nombre. \bibverse{20} Y puso Adam nombres á toda bestia y ave
de los cielos y á todo animal del campo: mas para Adam no halló ayuda
que estuviese idónea para él.

\bibverse{21} Y Jehová Dios hizo caer sueño sobre Adam, y se quedó
dormido: entonces tomó una de sus costillas, y cerró la carne en su
lugar; \bibverse{22} Y de la costilla que Jehová Dios tomó del hombre,
hizo una mujer, y trájola al hombre. \bibverse{23} Y dijo Adam: Esto es
ahora hueso de mis huesos, y carne de mi carne: ésta será llamada
Varona, porque del varón fué tomada. \bibverse{24} Por tanto, dejará el
hombre á su padre y á su madre, y allegarse ha á su mujer, y serán una
sola carne. \footnote{\textbf{2:24} Mat 19,5-6; Efes 5,28-31}
\bibverse{25} Y estaban ambos desnudos, Adam y su mujer, y no se
avergonzaban.

\hypertarget{la-tentacion-y-la-cauxedda-del-hombre}{%
\subsection{La tentacion y la caída del
hombre}\label{la-tentacion-y-la-cauxedda-del-hombre}}

\hypertarget{section-2}{%
\section{3}\label{section-2}}

\bibverse{1} EMPERO la serpiente era astuta, más que todos los animales
del campo que Jehová Dios había hecho; la cual dijo á la mujer: ¿Conque
Dios os ha dicho: No comáis de todo árbol del huerto?

\bibverse{2} Y la mujer respondió á la serpiente: Del fruto de los
árboles del huerto comemos; \footnote{\textbf{3:2} Gén 2,16}
\bibverse{3} Mas del fruto del árbol que está en medio del huerto dijo
Dios: No comeréis de él, ni le tocaréis, porque no muráis. \footnote{\textbf{3:3}
  Gén 2,17}

\bibverse{4} Entonces la serpiente dijo á la mujer: No moriréis;
\footnote{\textbf{3:4} Juan 8,44} \bibverse{5} Mas sabe Dios que el día
que comiereis de él, serán abiertos vuestros ojos, y seréis como dioses
sabiendo el bien y el mal.

\bibverse{6} Y vió la mujer que el árbol era bueno para comer, y que era
agradable á los ojos, y árbol codiciable para alcanzar la sabiduría; y
tomó de su fruto, y comió; y dió también á su marido, el cual comió así
como ella. \bibverse{7} Y fueron abiertos los ojos de entrambos, y
conocieron que estaban desnudos: entonces cosieron hojas de higuera, y
se hicieron delantales. \footnote{\textbf{3:7} Gén 2,25}

\hypertarget{el-interrogatorio-y-el-castigo}{%
\subsection{El interrogatorio y el
castigo}\label{el-interrogatorio-y-el-castigo}}

\bibverse{8} Y oyeron la voz de Jehová Dios que se paseaba en el huerto
al aire del día: y escondióse el hombre y su mujer de la presencia de
Jehová Dios entre los árboles del huerto. \footnote{\textbf{3:8} Jer
  23,24}

\bibverse{9} Y llamó Jehová Dios al hombre, y le dijo: ¿Dónde estás tú?

\bibverse{10} Y él respondió: Oí tu voz en el huerto, y tuve miedo,
porque estaba desnudo; y escondíme.

\bibverse{11} Y díjole: ¿Quién te enseñó que estabas desnudo? ¿Has
comido del árbol de que yo te mandé no comieses?

\bibverse{12} Y el hombre respondió: La mujer que me diste por compañera
me dió del árbol, y yo comí.

\bibverse{13} Entonces Jehová Dios dijo á la mujer: ¿Qué es lo que has
hecho? Y dijo la mujer: La serpiente me engañó, y comí. \footnote{\textbf{3:13}
  2Cor 11,3}

\bibverse{14} Y Jehová Dios dijo á la serpiente: Por cuanto esto
hiciste, maldita serás entre todas las bestias y entre todos los
animales del campo; sobre tu pecho andarás, y polvo comerás todos los
días de tu vida: \footnote{\textbf{3:14} Is 65,25} \bibverse{15} Y
enemistad pondré entre ti y la mujer, y entre tu simiente y la simiente
suya; ésta te herirá en la cabeza, y tú le herirás en el calcañar.
\footnote{\textbf{3:15} Gal 4,4; 1Jn 3,8; Heb 2,14; Rom 16,20; Juan
  14,30; Apoc 12,17}

\bibverse{16} A la mujer dijo: Multiplicaré en gran manera tus dolores y
tus preñeces; con dolor parirás los hijos; y á tu marido será tu deseo,
y él se enseñoreará de ti. \footnote{\textbf{3:16} Efes 5,22; 1Tim
  2,11-12}

\bibverse{17} Y al hombre dijo: Por cuanto obedeciste á la voz de tu
mujer, y comiste del árbol de que te mandé diciendo, No comerás de él;
maldita será la tierra por amor de ti; con dolor comerás de ella todos
los días de tu vida; \bibverse{18} Espinos y cardos te producirá, y
comerás hierba del campo; \bibverse{19} En el sudor de tu rostro comerás
el pan hasta que vuelvas á la tierra; porque de ella fuiste tomado: pues
polvo eres, y al polvo serás tornado. \footnote{\textbf{3:19} 2Tes 3,10;
  Ecl 12,7}

\hypertarget{la-expulsiuxf3n-del-parauxedso}{%
\subsection{La expulsión del
paraíso}\label{la-expulsiuxf3n-del-parauxedso}}

\bibverse{20} Y llamó el hombre el nombre de su mujer, Eva; por cuanto
ella era madre de todos los vivientes. \bibverse{21} Y Jehová Dios hizo
al hombre y á su mujer túnicas de pieles, y vistiólos.

\bibverse{22} Y dijo Jehová Dios: He aquí el hombre es como uno de Nos
sabiendo el bien y el mal: ahora, pues, porque no alargue su mano, y
tome también del árbol de la vida, y coma, y viva para siempre:
\bibverse{23} Y sacólo Jehová del huerto de Edén, para que labrase la
tierra de que fué tomado. \bibverse{24} Echó, pues, fuera al hombre, y
puso al oriente del huerto de Edén querubines, y una espada encendida
que se revolvía á todos lados, para guardar el camino del árbol de la
vida. \footnote{\textbf{3:24} Ezeq 10,1}

\hypertarget{cauxedn-y-abel}{%
\subsection{Caín y Abel}\label{cauxedn-y-abel}}

\hypertarget{section-3}{%
\section{4}\label{section-3}}

\bibverse{1} Y CONOCIÓ Adam á su mujer Eva, la cual concibió y parió á
Caín, y dijo: Adquirido he varón por Jehová. \bibverse{2} Y después
parió á su hermano Abel. Y fué Abel pastor de ovejas, y Caín fué
labrador de la tierra. \bibverse{3} Y aconteció andando el tiempo, que
Caín trajo del fruto de la tierra una ofrenda á Jehová. \bibverse{4} Y
Abel trajo también de los primogénitos de sus ovejas, y de su grosura. Y
miró Jehová con agrado á Abel y á su ofrenda; \bibverse{5} Mas no miró
propicio á Caín y á la ofrenda suya. Y ensañóse Caín en gran manera, y
decayó su semblante. \bibverse{6} Entonces Jehová dijo á Caín: ¿Por qué
te has ensañado, y por qué se ha inmutado tu rostro? \bibverse{7} Si
bien hicieres, ¿no serás ensalzado? y si no hicieres bien, el pecado
está á la puerta: con todo esto, á ti será su deseo, y tú te
enseñorearás de él. \footnote{\textbf{4:7} Gal 5,17; Rom 6,12}
\bibverse{8} Y habló Caín á su hermano Abel: y aconteció que estando
ellos en el campo, Caín se levantó contra su hermano Abel, y le mató.
\footnote{\textbf{4:8} 1Jn 3,12; 1Jn 1,3-15}

\hypertarget{el-castigo-de-cauxedn}{%
\subsection{El castigo de Caín}\label{el-castigo-de-cauxedn}}

\bibverse{9} Y Jehová dijo á Caín: ¿Dónde está Abel tu hermano? Y él
respondió: No sé; ¿soy yo guarda de mi hermano?

\bibverse{10} Y él le dijo: ¿Qué has hecho? La voz de la sangre de tu
hermano clama á mí desde la tierra. \footnote{\textbf{4:10} Sal 9,13;
  Mat 23,35; Heb 12,24} \bibverse{11} Ahora pues, maldito seas tú de la
tierra que abrió su boca para recibir la sangre de tu hermano de tu
mano: \bibverse{12} Cuando labrares la tierra, no te volverá á dar su
fuerza: errante y extranjero serás en la tierra.

\bibverse{13} Y dijo Caín á Jehová: Grande es mi iniquidad para ser
perdonada. \bibverse{14} He aquí me echas hoy de la faz de la tierra, y
de tu presencia me esconderé; y seré errante y extranjero en la tierra;
y sucederá que cualquiera que me hallare, me matará.

\bibverse{15} Y respondióle Jehová: Cierto que cualquiera que matare á
Caín, siete veces será castigado. Entonces Jehová puso señal en Caín,
para que no lo hiriese cualquiera que le hallara.

\bibverse{16} Y salió Caín de delante de Jehová, y habitó en tierra de
Nod, al oriente de Edén.

\hypertarget{los-hijos-de-cauxedn}{%
\subsection{Los hijos de Caín}\label{los-hijos-de-cauxedn}}

\bibverse{17} Y conoció Caín á su mujer, la cual concibió y parió á
Henoch: y edificó una ciudad, y llamó el nombre de la ciudad del nombre
de su hijo, Henoch. \bibverse{18} Y á Henoch nació Irad, é Irad engendró
á Mehujael, y Mehujael engendró á Methusael, y Methusael engendró á
Lamech. \bibverse{19} Y tomó para sí Lamech dos mujeres; el nombre de la
una fué Ada, y el nombre de la otra Zilla. \bibverse{20} Y Ada parió á
Jabal, el cual fué padre de los que habitan en tiendas, y crían ganados.
\bibverse{21} Y el nombre de su hermano fué Jubal, el cual fué padre de
todos los que manejan arpa y órgano. \bibverse{22} Y Zilla también parió
á Tubal-Caín, acicalador de toda obra de metal y de hierro: y la hermana
de Tubal-Caín fué Naama. \bibverse{23} Y dijo Lamech á sus mujeres: Ada
y Zilla, oid mi voz; mujeres de Lamech, escuchad mi dicho: que varón
mataré por mi herida, y mancebo por mi golpe: \bibverse{24} Si siete
veces será vengado Caín, Lamech en verdad setenta veces siete lo será.
\footnote{\textbf{4:24} Gén 4,15; Mat 18,21-22}

\hypertarget{el-nacimiento-de-seth}{%
\subsection{El nacimiento de Seth}\label{el-nacimiento-de-seth}}

\bibverse{25} Y conoció de nuevo Adam á su mujer, la cual parió un hijo,
y llamó su nombre Seth: Porque Dios (dijo ella) me ha sustituído otra
simiente en lugar de Abel, á quien mató Caín. \bibverse{26} Y á Seth
también le nació un hijo, y llamó su nombre Enós. Entonces los hombres
comenzaron á llamarse del nombre de Jehová.

\hypertarget{la-descendiencia-de-seth-hasta-nouxe9}{%
\subsection{La descendiencia de Seth hasta
Noé}\label{la-descendiencia-de-seth-hasta-nouxe9}}

\hypertarget{section-4}{%
\section{5}\label{section-4}}

\bibverse{1} ESTE es el libro de las generaciones de Adam. El día en que
crió Dios al hombre, á la semejanza de Dios lo hizo; \footnote{\textbf{5:1}
  Gén 1,27; Luc 3,38} \bibverse{2} Varón y hembra los crió; y los
bendijo, y llamó el nombre de ellos Adam, el día en que fueron criados.
\bibverse{3} Y vivió Adam ciento y treinta años, y engendró un hijo á su
semejanza, conforme á su imagen, y llamó su nombre Seth. \bibverse{4} Y
fueron los días de Adam, después que engendró á Seth, ochocientos años:
y engendró hijos é hijas. \bibverse{5} Y fueron todos los días que vivió
Adam novecientos y treinta años, y murió.

\bibverse{6} Y vivió Seth ciento y cinco años, y engendró á Enós.
\bibverse{7} Y vivió Seth, después que engendró á Enós, ochocientos y
siete años: y engendró hijos é hijas. \bibverse{8} Y fueron todos los
días de Seth novecientos y doce años; y murió.

\bibverse{9} Y vivió Enós noventa años, y engendró á Cainán.
\bibverse{10} Y vivió Enós después que engendró á Cainán, ochocientos y
quince años: y engendró hijos é hijas. \bibverse{11} Y fueron todos los
días de Enós novecientos y cinco años; y murió.

\bibverse{12} Y vivió Cainán setenta años, y engendró á Mahalaleel.
\bibverse{13} Y vivió Cainán, después que engendró á Mahalaleel,
ochocientos y cuarenta años: y engendró hijos é hijas. \bibverse{14} Y
fueron todos los días de Cainán novecientos y diez años; y murió.

\bibverse{15} Y vivió Mahalaleel sesenta y cinco años, y engendró á
Jared. \bibverse{16} Y vivió Mahalaleel, después que engendró á Jared,
ochocientos y treinta años: y engendró hijos é hijas. \bibverse{17} Y
fueron todos los días de Mahalaleel ochocientos noventa y cinco años; y
murió.

\bibverse{18} Y vivió Jared ciento sesenta y dos años, y engendró á
Henoch. \bibverse{19} Y vivió Jared, después que engendró á Henoch,
ochocientos años: y engendró hijos é hijas. \bibverse{20} Y fueron todos
los días de Jared novecientos sesenta y dos años; y murió.

\bibverse{21} Y vivió Henoch sesenta y cinco años, y engendró á
Mathusalam. \bibverse{22} Y caminó Henoch con Dios, después que engendró
á Mathusalam, trescientos años: y engendró hijos é hijas. \footnote{\textbf{5:22}
  Gén 6,9; Jds 1,14} \bibverse{23} Y fueron todos los días de Henoch
trescientos sesenta y cinco años. \bibverse{24} Caminó, pues, Henoch con
Dios, y desapareció, porque le llevó Dios.

\bibverse{25} Y vivió Mathusalam ciento ochenta y siete años, y engendró
á Lamech. \bibverse{26} Y vivió Mathusalam, después que engendró á
Lamech, setecientos ochenta y dos años: y engendró hijos é hijas.
\bibverse{27} Fueron, pues, todos los días de Mathusalam, novecientos
sesenta y nueve años; y murió.

\bibverse{28} Y vivió Lamech ciento ochenta y dos años, y engendró un
hijo: \bibverse{29} Y llamó su nombre Noé, diciendo: Este nos aliviará
de nuestras obras, y del trabajo de nuestras manos, á causa de la tierra
que Jehová maldijo. \footnote{\textbf{5:29} Gén 3,17-19} \bibverse{30} Y
vivió Lamech, después que engendró á Noé, quinientos noventa y cinco
años: y engendró hijos é hijas. \bibverse{31} Y fueron todos los días de
Lamech setecientos setenta y siete años; y murió.

\bibverse{32} Y siendo Noé de quinientos años, engendró á Sem, Châm, y á
Japhet.

\hypertarget{los-matrimonios-de-los-hijos-de-dios-con-las-hijas-de-los-hombres}{%
\subsection{Los matrimonios de los hijos de Dios con las hijas de los
hombres}\label{los-matrimonios-de-los-hijos-de-dios-con-las-hijas-de-los-hombres}}

\hypertarget{section-5}{%
\section{6}\label{section-5}}

\bibverse{1} Y ACAECIÓ que, cuando comenzaron los hombres á
multiplicarse sobre la faz de la tierra, y les nacieron hijas,
\bibverse{2} Viendo los hijos de Dios que las hijas de los hombres eran
hermosas, tomáronse mujeres, escogiendo entre todas. \bibverse{3} Y dijo
Jehová: No contenderá mi espíritu con el hombre para siempre, porque
ciertamente él es carne: mas serán sus días ciento y veinte años.
\footnote{\textbf{6:3} 1Pe 3,20} \bibverse{4} Había gigantes en la
tierra en aquellos días, y también después que entraron los hijos de
Dios á las hijas de los hombres, y les engendraron hijos: éstos fueron
los valientes que desde la antigüedad fueron varones de nombre.

\hypertarget{la-maldad-de-los-hombres.-anuncio-del-diluvio}{%
\subsection{La maldad de los hombres. Anuncio del
diluvio}\label{la-maldad-de-los-hombres.-anuncio-del-diluvio}}

\bibverse{5} Y vió Jehová que la malicia de los hombres era mucha en la
tierra, y que todo designio de los pensamientos del corazón de ellos era
de continuo solamente el mal. \bibverse{6} Y arrepintióse Jehová de
haber hecho hombre en la tierra, y pesóle en su corazón. \footnote{\textbf{6:6}
  Jer 18,10; Núm 23,19; Sal 18,27} \bibverse{7} Y dijo Jehová: Raeré los
hombres que he criado de sobre la faz de la tierra, desde el hombre
hasta la bestia, y hasta el reptil y las aves del cielo: porque me
arrepiento de haberlos hecho. \bibverse{8} Empero Noé halló gracia en
los ojos de Jehová.

\hypertarget{llamado-de-nouxe9-y-la-construcciuxf3n-del-arca}{%
\subsection{Llamado de Noé y la construcción del
arca}\label{llamado-de-nouxe9-y-la-construcciuxf3n-del-arca}}

\bibverse{9} Estas son las generaciones de Noé: Noé, varón justo,
perfecto fué en sus generaciones; con Dios caminó Noé. \bibverse{10} Y
engendró Noé tres hijos: á Sem, á Châm, y á Japhet. \bibverse{11} Y
corrompióse la tierra delante de Dios, y estaba la tierra llena de
violencia. \bibverse{12} Y miró Dios la tierra, y he aquí que estaba
corrompida; porque toda carne había corrompido su camino sobre la
tierra. \footnote{\textbf{6:12} Sal 14,2-3}

\bibverse{13} Y dijo Dios á Noé: El fin de toda carne ha venido delante
de mí; porque la tierra está llena de violencia á causa de ellos; y he
aquí que yo los destruiré con la tierra. \footnote{\textbf{6:13} Am 8,2}
\bibverse{14} Hazte un arca de madera de Gopher: harás aposentos en el
arca, y la embetunarás con brea por dentro y por fuera. \bibverse{15} Y
de esta manera la harás: de trescientos codos la longitud del arca, de
cincuenta codos su anchura, y de treinta codos su altura. \bibverse{16}
Una ventana harás al arca, y la acabarás á un codo de elevación por la
parte de arriba: y pondrás la puerta del arca á su lado; y le harás piso
bajo, segundo y tercero. \bibverse{17} Y yo, he aquí que yo traigo un
diluvio de aguas sobre la tierra, para destruir toda carne en que haya
espíritu de vida debajo del cielo; todo lo que hay en la tierra morirá.
\bibverse{18} Mas estableceré mi pacto contigo, y entrarás en el arca
tú, y tus hijos y tu mujer, y las mujeres de tus hijos contigo.
\bibverse{19} Y de todo lo que vive, de toda carne, dos de cada especie
meterás en el arca, para que tengan vida contigo; macho y hembra serán.
\bibverse{20} De las aves según su especie, y de las bestias según su
especie, de todo reptil de la tierra según su especie, dos de cada
especie entrarán contigo para que hayan vida. \bibverse{21} Y toma
contigo de toda vianda que se come, y allégala á ti; servirá de alimento
para ti y para ellos. \bibverse{22} E hízolo así Noé; hizo conforme á
todo lo que Dios le mandó.

\hypertarget{el-diluvio.-nouxe9-entra-la-arca}{%
\subsection{El diluvio. Noé entra la
arca}\label{el-diluvio.-nouxe9-entra-la-arca}}

\hypertarget{section-6}{%
\section{7}\label{section-6}}

\bibverse{1} Y JEHOVÁ dijo á Noé: Entra tú y toda tu casa en el arca;
porque á ti he visto justo delante de mí en esta generación.
\bibverse{2} De todo animal limpio te tomarás de siete en siete, macho y
su hembra; mas de los animales que no son limpios, dos, macho y su
hembra. \footnote{\textbf{7:2} Gén 8,20; Lev 11,-1} \bibverse{3} También
de las aves de los cielos de siete en siete, macho y hembra; para
guardar en vida la casta sobre la faz de toda la tierra. \bibverse{4}
Porque pasados aún siete días, yo haré llover sobre la tierra cuarenta
días y cuarenta noches; y raeré toda sustancia que hice de sobre la faz
de la tierra.

\bibverse{5} E hizo Noé conforme á todo lo que le mandó Jehová.

\bibverse{6} Y siendo Noé de seiscientos años, el diluvio de las aguas
fué sobre la tierra. \bibverse{7} Y vino Noé, y sus hijos, y su mujer, y
las mujeres de sus hijos con él al arca, por las aguas del diluvio.
\footnote{\textbf{7:7} 1Pe 3,20} \bibverse{8} De los animales limpios, y
de los animales que no eran limpios, y de las aves, y de todo lo que
anda arrastrando sobre la tierra, \bibverse{9} De dos en dos entraron á
Noé en el arca: macho y hembra, como mandó Dios á Noé. \bibverse{10} Y
sucedió que al séptimo día las aguas del diluvio fueron sobre la tierra.

\hypertarget{el-aumento-del-diluvio}{%
\subsection{El aumento del diluvio}\label{el-aumento-del-diluvio}}

\bibverse{11} El año seiscientos de la vida de Noé, en el mes segundo, á
diecisiete días del mes, aquel día fueron rotas todas las fuentes del
grande abismo, y las cataratas de los cielos fueron abiertas;
\bibverse{12} Y hubo lluvia sobre la tierra cuarenta días y cuarenta
noches.

\bibverse{13} En este mismo día entró Noé, y Sem y Châm y Japhet, hijos
de Noé, la mujer de Noé, y las tres mujeres de sus hijos con él en el
arca; \bibverse{14} Ellos, y todos los animales silvestres según sus
especies, y todos los animales mansos según sus especies, y todo reptil
que anda arrastrando sobre la tierra según su especie, y toda ave según
su especie, todo pájaro, toda especie de volátil. \bibverse{15} Y
vinieron á Noé al arca, de dos en dos de toda carne en que había
espíritu de vida. \bibverse{16} Y los que vinieron, macho y hembra de
toda carne vinieron, como le había mandado Dios: y Jehová le cerró la
puerta. \footnote{\textbf{7:16} Gén 6,19}

\bibverse{17} Y fué el diluvio cuarenta días sobre la tierra; y las
aguas crecieron, y alzaron el arca, y se elevó sobre la tierra.
\bibverse{18} Y prevalecieron las aguas, y crecieron en gran manera
sobre la tierra; y andaba el arca sobre la faz de las aguas.
\bibverse{19} Y las aguas prevalecieron mucho en extremo sobre la
tierra; y todos los montes altos que había debajo de todos los cielos,
fueron cubiertos. \bibverse{20} Quince codos en alto prevalecieron las
aguas; y fueron cubiertos los montes. \bibverse{21} Y murió toda carne
que se mueve sobre la tierra, así de aves como de ganados, y de bestias,
y de todo reptil que anda arrastrando sobre la tierra, y todo hombre:
\bibverse{22} Todo lo que tenía aliento de espíritu de vida en sus
narices, de todo lo que había en la tierra, murió. \bibverse{23} Así fué
destruída toda sustancia que vivía sobre la faz de la tierra, desde el
hombre hasta la bestia, y los reptiles, y las aves del cielo; y fueron
raídos de la tierra; y quedó solamente Noé, y lo que con él estaba en el
arca. \bibverse{24} Y prevalecieron las aguas sobre la tierra ciento y
cincuenta días.

\hypertarget{el-fin-del-diluvio}{%
\subsection{El fin del diluvio}\label{el-fin-del-diluvio}}

\hypertarget{section-7}{%
\section{8}\label{section-7}}

\bibverse{1} Y ACORDÓSE Dios de Noé, y de todos los animales, y de todas
las bestias que estaban con él en el arca; é hizo pasar Dios un viento
sobre la tierra, y disminuyeron las aguas. \bibverse{2} Y se cerraron
las fuentes del abismo, y las cataratas de los cielos; y la lluvia de
los cielos fué detenida. \footnote{\textbf{8:2} Gén 7,11-12}
\bibverse{3} Y tornáronse las aguas de sobre la tierra, yendo y
volviendo: y decrecieron las aguas al cabo de ciento y cincuenta días.
\bibverse{4} Y reposó el arca en el mes séptimo, á diecisiete días del
mes, sobre los montes de Armenia. \bibverse{5} Y las aguas fueron
decreciendo hasta el mes décimo: en el décimo, al primero del mes, se
descubrieron las cimas de los montes.

\bibverse{6} Y sucedió que, al cabo de cuarenta días, abrió Noé la
ventana del arca que había hecho, \bibverse{7} Y envió al cuervo, el
cual salió, y estuvo yendo y tornando hasta que las aguas se secaron de
sobre la tierra. \bibverse{8} Envió también de sí á la paloma, para ver
si las aguas se habían retirado de sobre la faz de la tierra;
\bibverse{9} Y no halló la paloma donde sentar la planta de su pie, y
volvióse á él al arca, porque las aguas estaban aún sobre la faz de toda
la tierra: entonces él extendió su mano, y cogiéndola, hízola entrar
consigo en el arca. \bibverse{10} Y esperó aún otros siete días, y
volvió á enviar la paloma fuera del arca. \bibverse{11} Y la paloma
volvió á él á la hora de la tarde; y he aquí que traía una hoja de oliva
tomada en su pico: y entendió Noé que las aguas se habían retirado de
sobre la tierra. \bibverse{12} Y esperó aún otros siete días, y envió la
paloma, la cual no volvió ya más á él.

\bibverse{13} Y sucedió que en el año seiscientos y uno de Noé, en el
mes primero, al primero del mes, las aguas se enjugaron de sobre la
tierra; y quitó Noé la cubierta del arca, y miró, y he aquí que la faz
de la tierra estaba enjuta. \bibverse{14} Y en el mes segundo, á los
veintisiete días del mes, se secó la tierra.

\bibverse{15} Y habló Dios á Noé diciendo: \bibverse{16} Sal del arca
tú, y tu mujer, y tus hijos, y las mujeres de tus hijos contigo.
\bibverse{17} Todos los animales que están contigo de toda carne, de
aves y de bestias y de todo reptil que anda arrastrando sobre la tierra,
sacarás contigo; y vayan por la tierra, y fructifiquen, y multiplíquense
sobre la tierra.

\bibverse{18} Entonces salió Noé, y sus hijos, y su mujer, y las mujeres
de sus hijos con él. \footnote{\textbf{8:18} 2Pe 2,5} \bibverse{19}
Todos los animales, y todo reptil y toda ave, todo lo que se mueve sobre
la tierra según sus especies, salieron del arca.

\hypertarget{el-holocausto-de-nouxe9-y-la-promesa-de-dios}{%
\subsection{El holocausto de Noé y la promesa de
Dios}\label{el-holocausto-de-nouxe9-y-la-promesa-de-dios}}

\bibverse{20} Y edificó Noé un altar á Jehová, y tomó de todo animal
limpio y de toda ave limpia, y ofreció holocausto en el altar.
\bibverse{21} Y percibió Jehová olor de suavidad; y dijo Jehová en su
corazón: No tornaré más á maldecir la tierra por causa del hombre;
porque el intento del corazón del hombre es malo desde su juventud: ni
volveré más á destruir todo viviente, como he hecho. \footnote{\textbf{8:21}
  Gén 6,5; Sal 14,3; Job 14,4; Mat 15,19; Rom 3,23; Is 54,9}
\bibverse{22} Todavía serán todos los tiempos de la tierra; la sementera
y la siega, y el frío y calor, verano é invierno, y día y noche, no
cesarán. \footnote{\textbf{8:22} Jer 33,20; Jer 33,25}

\hypertarget{renovaciuxf3n-de-la-bendiciuxf3n-de-la-creaciuxf3n}{%
\subsection{Renovación de la bendición de la
creación}\label{renovaciuxf3n-de-la-bendiciuxf3n-de-la-creaciuxf3n}}

\hypertarget{section-8}{%
\section{9}\label{section-8}}

\bibverse{1} Y BENDIJO Dios á Noé y á sus hijos, y díjoles: Fructificad,
y multiplicad, y henchid la tierra: \footnote{\textbf{9:1} Gén 1,28}
\bibverse{2} Y vuestro temor y vuestro pavor será sobre todo animal de
la tierra, y sobre toda ave de los cielos, en todo lo que se moverá en
la tierra, y en todos los peces del mar: en vuestra mano son entregados.
\bibverse{3} Todo lo que se mueve y vive, os será para mantenimiento:
así como las legumbres y hierbas, os lo he dado todo. \bibverse{4}
Empero carne con su vida, que es su sangre, no comeréis. \footnote{\textbf{9:4}
  Lev 3,17} \bibverse{5} Porque ciertamente demandaré la sangre de
vuestras vidas; de mano de todo animal la demandaré, y de mano del
hombre; de mano del varón su hermano demandaré la vida del hombre.
\footnote{\textbf{9:5} Éxod 21,28-29; Gén 4,11} \bibverse{6} El que
derramare sangre del hombre, por el hombre su sangre será derramada;
porque á imagen de Dios es hecho el hombre. \footnote{\textbf{9:6} Éxod
  21,12; Lev 24,17; Mat 26,52; Apoc 13,10; Gén 1,27} \bibverse{7} Mas
vosotros fructificad, y multiplicaos; procread abundantemente en la
tierra, y multiplicaos en ella.

\hypertarget{el-pacto-entro-dios-y-nouxe9-y-la-creaciuxf3n}{%
\subsection{El pacto entro Dios y Noé y la
creación}\label{el-pacto-entro-dios-y-nouxe9-y-la-creaciuxf3n}}

\bibverse{8} Y habló Dios á Noé y á sus hijos con él, diciendo:
\bibverse{9} Yo, he aquí que yo establezco mi pacto con vosotros, y con
vuestra simiente después de vosotros; \bibverse{10} Y con toda alma
viviente que está con vosotros, de aves, de animales, y de toda bestia
de la tierra que está con vosotros; desde todos los que salieron del
arca hasta todo animal de la tierra. \footnote{\textbf{9:10} Os 2,10}
\bibverse{11} Estableceré mi pacto con vosotros, y no fenecerá ya más
toda carne con aguas de diluvio; ni habrá más diluvio para destruir la
tierra. \footnote{\textbf{9:11} Gén 8,21-22} \bibverse{12} Y dijo Dios:
Esta será la señal del pacto que yo establezco entre mí y vosotros y
toda alma viviente que está con vosotros, por siglos perpetuos:
\bibverse{13} Mi arco pondré en las nubes, el cual será por señal de
convenio entre mí y la tierra. \bibverse{14} Y será que cuando haré
venir nubes sobre la tierra, se dejará ver entonces mi arco en las
nubes. \bibverse{15} Y acordarme he del pacto mío, que hay entre mí y
vosotros y toda alma viviente de toda carne; y no serán más las aguas
por diluvio para destruir toda carne. \bibverse{16} Y estará el arco en
las nubes, y verlo he para acordarme del pacto perpetuo entre Dios y
toda alma viviente, con toda carne que hay sobre la tierra.
\bibverse{17} Dijo, pues, Dios á Noé: Esta será la señal del pacto que
he establecido entre mí y toda carne que está sobre la tierra.

\hypertarget{la-embriaguez-de-nouxe9}{%
\subsection{La embriaguez de Noé}\label{la-embriaguez-de-nouxe9}}

\bibverse{18} Y los hijos de Noé que salieron del arca fueron Sem, Châm
y Japhet: y Châm es el padre de Canaán. \bibverse{19} Estos tres son los
hijos de Noé; y de ellos fué llena toda la tierra.

\bibverse{20} Y comenzó Noé á labrar la tierra, y plantó una viña:
\bibverse{21} Y bebió del vino, y se embriagó, y estaba descubierto en
medio de su tienda. \bibverse{22} Y Châm, padre de Canaán, vió la
desnudez de su padre, y díjolo á sus dos hermanos á la parte de afuera.
\footnote{\textbf{9:22} Prov 30,17; Sir 3,12} \bibverse{23} Entonces Sem
y Japhet tomaron la ropa, y la pusieron sobre sus propios hombros, y
andando hacia atrás, cubrieron la desnudez de su padre, teniendo vueltos
sus rostros, y así no vieron la desnudez de su padre. \bibverse{24} Y
despertó Noé de su vino, y supo lo que había hecho con él su hijo el más
joven; \bibverse{25} Y dijo: Maldito sea Canaán; siervo de siervos será
á sus hermanos.

\bibverse{26} Dijo más: Bendito Jehová el Dios de Sem, y séale Canaán
siervo. \bibverse{27} Engrandezca Dios á Japhet, y habite en las tiendas
de Sem, y séale Canaán siervo. \footnote{\textbf{9:27} Efes 3,6}

\bibverse{28} Y vivió Noé después del diluvio trescientos y cincuenta
años. \bibverse{29} Y fueron todos los días de Noé novecientos y
cincuenta años; y murió.

\hypertarget{los-descendientes-de-los-hijos-de-nouxe9}{%
\subsection{Los descendientes de los hijos de
Noé}\label{los-descendientes-de-los-hijos-de-nouxe9}}

\hypertarget{section-9}{%
\section{10}\label{section-9}}

\bibverse{1} ESTAS son las generaciones de los hijos de Noé: Sem, Châm y
Japhet, á los cuales nacieron hijos después del diluvio.

\bibverse{2} Los hijos de Japhet: Gomer, y Magog, y Madai, y Javán, y
Tubal, y Meshech, y Tiras. \bibverse{3} Y los hijos de Gomer: Ashkenaz,
y Riphat, y Togorma. \bibverse{4} Y los hijos de Javán: Elisa, y Tarsis,
Kittim, y Dodanim. \bibverse{5} Por éstos fueron repartidas las islas de
las gentes en sus tierras, cada cual según su lengua, conforme á sus
familias en sus naciones.

\bibverse{6} Los hijos de Châm: Cush, y Mizraim, y Phut, y Canaán.
\bibverse{7} Y los hijos de Cush: Seba, Havila, y Sabta, y Raama, y
Sabtecha. Y los hijos de Raama: Sheba y Dedán. \bibverse{8} Y Cush
engendró á Nimrod: éste comenzó á ser poderoso en la tierra.
\bibverse{9} Este fué vigoroso cazador delante de Jehová; por lo cual se
dice: Así como Nimrod, vigoroso cazador delante de Jehová. \bibverse{10}
Y fué la cabecera de su reino Babel, y Erech, y Accad, y Calneh, en la
tierra de Shinar. \bibverse{11} De aquesta tierra salió Assur, y edificó
á Nínive, y á Rehoboth, y á Calah, \footnote{\textbf{10:11} Jon 1,2}
\bibverse{12} Y á Ressen entre Nínive y Calah; la cual es ciudad grande.
\bibverse{13} Y Mizraim engendró á Ludim, y á Anamim, y á Lehabim, y á
Naphtuhim, \bibverse{14} Y á Pathrusim, y á Casluim, de donde salieron
los Filisteos, y á Caphtorim.

\bibverse{15} Y Canaán engendró á Sidón, su primogénito, y á Heth,
\bibverse{16} Y al Jebuseo, y al Amorrheo, y al Gergeseo, \bibverse{17}
Y al Heveo, y al Araceo, y al Sineo, \bibverse{18} Y al Aradio, y al
Samareo, y al Amatheo: y después se derramaron las familias de los
Cananeos. \bibverse{19} Y fué el término de los Cananeos desde Sidón,
viniendo á Gerar hasta Gaza, hasta entrar en Sodoma y Gomorra, Adma, y
Zeboim hasta Lasa. \bibverse{20} Estos son los hijos de Châm por sus
familias, por sus lenguas, en sus tierras, en sus naciones.

\bibverse{21} También le nacieron hijos á Sem, padre de todos los hijos
de Heber, y hermano mayor de Japhet. \bibverse{22} Y los hijos de Sem:
Elam, y Assur, y Arphaxad, y Lud, y Aram. \bibverse{23} Y los hijos de
Aram: Uz, y Hul, y Gether, y Mas. \bibverse{24} Y Arphaxad engendró á
Sala, y Sala engendró á Heber. \bibverse{25} Y á Heber nacieron dos
hijos: el nombre del uno fué Peleg, porque en sus días fué repartida la
tierra; y el nombre de su hermano, Joctán. \footnote{\textbf{10:25} Gén
  11,8} \bibverse{26} Y Joctán engendró á Almodad, y á Sheleph, y
Hazarmaveth, y á Jera, \bibverse{27} Y á Hadoram, y á Uzal, y á Dicla,
\bibverse{28} Y á Obal, y á Abimael, y á Seba, \bibverse{29} Y á Ophir,
y á Havila, y á Jobad: todos estos fueron hijos de Joctán. \bibverse{30}
Y fué su habitación desde Mesa viniendo de Sephar, monte á la parte del
oriente. \bibverse{31} Estos fueron los hijos de Sem por sus familias,
por sus lenguas, en sus tierras, en sus naciones.

\bibverse{32} Estas son las familias de Noé por sus descendencias, en
sus naciones; y de éstos fueron divididas las gentes en la tierra
después del diluvio.

\hypertarget{la-torre-de-babel}{%
\subsection{La torre de Babel}\label{la-torre-de-babel}}

\hypertarget{section-10}{%
\section{11}\label{section-10}}

\bibverse{1} ERA entonces toda la tierra de una lengua y unas mismas
palabras. \bibverse{2} Y aconteció que, como se partieron de oriente,
hallaron una vega en la tierra de Shinar, y asentaron allí. \bibverse{3}
Y dijeron los unos á los otros: Vaya, hagamos ladrillo y cozámoslo con
fuego. Y fuéles el ladrillo en lugar de piedra, y el betún en lugar de
mezcla. \bibverse{4} Y dijeron: Vamos, edifiquémonos una ciudad y una
torre, cuya cúspide llegue al cielo; y hagámonos un nombre, por si
fuéremos esparcidos sobre la faz de toda la tierra.

\bibverse{5} Y descendió Jehová para ver la ciudad y la torre que
edificaban los hijos de los hombres. \footnote{\textbf{11:5} Gén 18,21;
  Sal 18,10; Sal 14,2} \bibverse{6} Y dijo Jehová: He aquí el pueblo es
uno, y todos éstos tienen un lenguaje: y han comenzado á obrar, y nada
les retraerá ahora de lo que han pensando hacer. \bibverse{7} Ahora
pues, descendamos, y confundamos allí sus lenguas, para que ninguno
entienda el habla de su compañero. \bibverse{8} Así los esparció Jehová
desde allí sobre la faz de toda la tierra, y dejaron de edificar la
ciudad. \bibverse{9} Por esto fué llamado el nombre de ella Babel,
porque allí confundió Jehová el lenguaje de toda la tierra, y desde allí
los esparció sobre la faz de toda la tierra.

\hypertarget{los-descendientes-de-sem}{%
\subsection{Los descendientes de Sem}\label{los-descendientes-de-sem}}

\bibverse{10} Estas son las generaciones de Sem: Sem, de edad de cien
años, engendró á Arphaxad, dos años después del diluvio. \footnote{\textbf{11:10}
  Gén 10,22; Luc 3,36} \bibverse{11} Y vivió Sem, después que engendró á
Arphaxad, quinientos años, y engendró hijos é hijas.

\bibverse{12} Y Arphaxad vivió treinta y cinco años, y engendró á Sala.
\bibverse{13} Y vivió Arphaxad, después que engendró á Sala,
cuatrocientos y tres años, y engendró hijos é hijas.

\bibverse{14} Y vivió Sala treinta años, y engendró á Heber.
\bibverse{15} Y vivió Sala, después que engendró á Heber, cuatrocientos
y tres años, y engendró hijos é hijas.

\bibverse{16} Y vivió Heber treinta y cuatro años, y engendró á Peleg.
\bibverse{17} Y vivió Heber, después que engendró á Peleg, cuatrocientos
y treinta años, y engendró hijos é hijas.

\bibverse{18} Y vivió Peleg treinta años, y engendró á Reu.
\bibverse{19} Y vivió Peleg, después que engendró á Reu, doscientos y
nueve años, y engendró hijos é hijas.

\bibverse{20} Y Reu vivió treinta y dos años, y engendró á Serug.
\bibverse{21} Y vivió Reu, después que engendró á Serug, doscientos y
siete años, y engendró hijos é hijas.

\bibverse{22} Y vivió Serug treinta años, y engendró á Nachôr.
\bibverse{23} Y vivió Serug, después que engendró á Nachôr, doscientos
años, y engendró hijos é hijas.

\bibverse{24} Y vivió Nachôr veintinueve años, y engendró á Thare.
\bibverse{25} Y vivió Nachôr, después que engendró á Thare, ciento
diecinueve años, y engendró hijos é hijas.

\bibverse{26} Y vivió Thare setenta años, y engendró á Abram, y á
Nachôr, y á Harán.

\hypertarget{los-descendientes-de-thare}{%
\subsection{Los descendientes de
Thare}\label{los-descendientes-de-thare}}

\bibverse{27} Estas son las generaciones de Thare: Thare engendró á
Abram, y á Nachôr, y á Harán; y Harán engendró á Lot. \bibverse{28} Y
murió Harán antes que su padre Thare en la tierra de su naturaleza, en
Ur de los Caldeos. \bibverse{29} Y tomaron Abram y Nachôr para sí
mujeres: el nombre de la mujer de Abram fué Sarai, y el nombre de la
mujer de Nachôr, Milca, hija de Harán, padre de Milca y de Isca.
\bibverse{30} Mas Sarai fué estéril, y no tenía hijo. \footnote{\textbf{11:30}
  Jos 24,2; Neh 9,7} \bibverse{31} Y tomó Thare á Abram su hijo, y á Lot
hijo de Harán, hijo de su hijo, y á Sarai su nuera, mujer de Abram su
hijo: y salió con ellos de Ur de los Caldeos, para ir á la tierra de
Canaán: y vinieron hasta Harán, y asentaron allí. \bibverse{32} Y fueron
los días de Thare doscientos y cinco años; y murió Thare en Harán.

\hypertarget{el-llamado-de-abram}{%
\subsection{El llamado de Abram}\label{el-llamado-de-abram}}

\hypertarget{section-11}{%
\section{12}\label{section-11}}

\bibverse{1} EMPERO Jehová había dicho á Abram: Vete de tu tierra y de
tu parentela, y de la casa de tu padre, á la tierra que te mostraré;
\bibverse{2} Y haré de ti una nación grande, y bendecirte he, y
engrandeceré tu nombre, y serás bendición: \footnote{\textbf{12:2} Gén
  24,1; Gén 24,35; Sal 72,17} \bibverse{3} Y bendeciré á los que te
bendijeren, y á los que te maldijeren maldeciré: y serán benditas en ti
todas las familias de la tierra. \footnote{\textbf{12:3} Éxod 23,22; Gén
  18,18; Gén 22,18; Gén 26,4; Gén 28,14; Hech 3,25; Gal 3,8}

\hypertarget{la-inmigraciuxf3n-de-abram-a-canauxe1n}{%
\subsection{La Inmigración de Abram a
Canaán}\label{la-inmigraciuxf3n-de-abram-a-canauxe1n}}

\bibverse{4} Y fuése Abram, como Jehová le dijo; y fué con él Lot: y era
Abram de edad de setenta y cinco años cuando salió de Harán.
\bibverse{5} Y tomó Abram á Sarai su mujer, y á Lot hijo de su hermano,
y toda su hacienda que habían ganado, y las almas que habían adquirido
en Harán, y salieron para ir á tierra de Canaán; y á tierra de Canaán
llegaron. \bibverse{6} Y pasó Abram por aquella tierra hasta el lugar de
Sichêm, hasta el valle de Moreh: y el Cananeo estaba entonces en la
tierra.

\bibverse{7} Y apareció Jehová á Abram, y le dijo: A tu simiente daré
esta tierra. Y edificó allí un altar á Jehová, que le había aparecido.
\footnote{\textbf{12:7} Gén 13,15; Gén 15,18; Gén 17,8; Gén 24,7; Gén
  26,3-4; Gén 28,13; Gén 35,12; Éxod 6,4; Éxod 6,8; Éxod 32,13; Jos
  21,43; Hech 7,5}

\bibverse{8} Y pasóse de allí á un monte al oriente de Bethel, y tendió
su tienda, teniendo á Bethel al occidente y Hai al oriente: y edificó
allí altar á Jehová, é invocó el nombre de Jehová. \footnote{\textbf{12:8}
  Gén 4,26} \bibverse{9} Y movió Abram de allí, caminando y yendo hacia
el Mediodía.

\hypertarget{abram-y-sarai-en-egipto}{%
\subsection{Abram y Sarai en Egipto}\label{abram-y-sarai-en-egipto}}

\bibverse{10} Y hubo hambre en la tierra, y descendió Abram á Egipto
para peregrinar allá; porque era grande el hambre en la tierra.
\footnote{\textbf{12:10} Gén 20,1; Gén 26,1-11} \bibverse{11} Y
aconteció que cuando estaba para entrar en Egipto, dijo á Sarai su
mujer: He aquí, ahora conozco que eres mujer hermosa de vista;
\bibverse{12} Y será que cuando te habrán visto los Egipcios, dirán: Su
mujer es: y me matarán á mí, y á ti te reservarán la vida. \bibverse{13}
Ahora pues, di que eres mi hermana, para que yo haya bien por causa
tuya, y viva mi alma por amor de ti.

\bibverse{14} Y aconteció que, como entró Abram en Egipto, los Egipcios
vieron la mujer que era hermosa en gran manera. \bibverse{15} Viéronla
también los príncipes de Faraón, y se la alabaron; y fué llevada la
mujer á casa de Faraón: \bibverse{16} E hizo bien á Abram por causa de
ella; y tuvo ovejas, y vacas, y asnos, y siervos, y criadas, y asnas y
camellos. \bibverse{17} Mas Jehová hirió á Faraón y á su casa con
grandes plagas, por causa de Sarai mujer de Abram. \bibverse{18}
Entonces Faraón llamó á Abram, y le dijo: ¿Qué es esto que has hecho
conmigo? ¿Por qué no me declaraste que era tu mujer? \bibverse{19} ¿Por
qué dijiste: Es mi hermana? poniéndome en ocasión de tomarla para mí por
mujer? Ahora pues, he aquí tu mujer, tómala y vete.

\bibverse{20} Entonces Faraón dió orden á sus gentes acerca de Abram; y
le acompañaron, y á su mujer con todo lo que tenía.

\hypertarget{el-regreso-de-abram}{%
\subsection{El regreso de Abram}\label{el-regreso-de-abram}}

\hypertarget{section-12}{%
\section{13}\label{section-12}}

\bibverse{1} SUBIÓ, pues, Abram de Egipto hacia el Mediodía, él, y su
mujer, con todo lo que tenía, y con él Lot. \bibverse{2} Y Abram era
riquísimo en ganado, en plata y oro. \footnote{\textbf{13:2} Prov 10,22}
\bibverse{3} Y volvió por sus jornadas de la parte del Mediodía hacia
Bethel, hasta el lugar donde había estado antes su tienda entre Bethel y
Hai; \bibverse{4} Al lugar del altar que había hecho allí antes: é
invocó allí Abram el nombre de Jehová.

\hypertarget{abram-se-separa-de-lot}{%
\subsection{Abram se separa de Lot}\label{abram-se-separa-de-lot}}

\bibverse{5} Y asimismo Lot, que andaba con Abram, tenía ovejas, y
vacas, y tiendas. \bibverse{6} Y la tierra no podía darles para que
habitasen juntos: porque su hacienda era mucha, y no podían morar en un
mismo lugar. \bibverse{7} Y hubo contienda entre los pastores del ganado
de Abram y los pastores del ganado de Lot: y el Cananeo y el Pherezeo
habitaban entonces en la tierra. \bibverse{8} Entonces Abram dijo á Lot:
No haya ahora altercado entre mí y ti, entre mis pastores y los tuyos,
porque somos hermanos. \footnote{\textbf{13:8} Sal 133,1} \bibverse{9}
¿No está toda la tierra delante de ti? Yo te ruego que te apartes de mí.
Si fueres á la mano izquierda, yo iré á la derecha: y si tú á la
derecha, yo iré á la izquierda.

\hypertarget{salida-de-lot-por-el-valle-del-jorduxe1n}{%
\subsection{Salida de Lot por el valle del
Jordán}\label{salida-de-lot-por-el-valle-del-jorduxe1n}}

\bibverse{10} Y alzó Lot sus ojos, y vió toda la llanura del Jordán, que
toda ella era de riego, antes que destruyese Jehová á Sodoma y á
Gomorra, como el huerto de Jehová, como la tierra de Egipto entrando en
Zoar. \bibverse{11} Entonces Lot escogió para sí toda la llanura del
Jordán: y partióse Lot de Oriente, y apartáronse el uno del otro.
\bibverse{12} Abram asentó en la tierra de Canaán, y Lot asentó en las
ciudades de la llanura, y fué poniendo sus tiendas hasta Sodoma.
\bibverse{13} Mas los hombres de Sodoma eran malos y pecadores para con
Jehová en gran manera.

\hypertarget{dios-promete-a-abram-el-pauxeds-de-canuxe1n}{%
\subsection{Dios promete a Abram el país de
Canán}\label{dios-promete-a-abram-el-pauxeds-de-canuxe1n}}

\bibverse{14} Y Jehová dijo á Abram, después que Lot se apartó de él:
Alza ahora tus ojos, y mira desde el lugar donde estás hacia el Aquilón,
y al Mediodía, y al Oriente y al Occidente; \bibverse{15} Porque toda la
tierra que ves, la daré á ti y á tu simiente para siempre. \footnote{\textbf{13:15}
  Gén 12,7} \bibverse{16} Y haré tu simiente como el polvo de la tierra:
que si alguno podrá contar el polvo de la tierra, también tu simiente
será contada. \footnote{\textbf{13:16} Gén 28,14; Núm 23,10}
\bibverse{17} Levántate, ve por la tierra á lo largo de ella y á su
ancho; porque á ti la tengo de dar.

\bibverse{18} Abram, pues, removiendo su tienda, vino y moró en el
alcornocal de Mamre, que es en Hebrón, y edificó allí altar á Jehová.
\footnote{\textbf{13:18} Gén 14,24}

\hypertarget{guerra-del-rey-kedorlaomer-en-el-valle-del-jordan}{%
\subsection{Guerra del rey Kedorlaomer en el valle del
Jordan}\label{guerra-del-rey-kedorlaomer-en-el-valle-del-jordan}}

\hypertarget{section-13}{%
\section{14}\label{section-13}}

\bibverse{1} Y ACONTECIÓ en los días de Amraphel, rey de Shinar, Arioch,
rey de Elazar, Chêdorlaomer, rey de Elá, y Tidal, rey de naciones,
\bibverse{2} Que éstos hicieron guerra contra Bera, rey de Sodoma, y
contra Birsha, rey de Gomorra, y contra Shinab, rey de Adma, y contra
Shemeber, rey de Zeboim, y contra el rey de Bela, la cual es Zoar.
\bibverse{3} Todos estos se juntaron en el valle de Siddim, que es el
mar salado. \bibverse{4} Doce años habían servido á Chêdorlaomer, y al
décimotercio año se rebelaron. \bibverse{5} Y en el año décimocuarto
vino Chêdorlaomer, y los reyes que estaban de su parte, y derrotaron á
los Raphaitas en Ashteroth-carnaim, á los Zuzitas en Ham, y á los Emitas
en Shave-Kiriataim. \bibverse{6} Y á los Horeos en el monte de Seir,
hasta la llanura de Parán, que está junto al desierto. \bibverse{7} Y
volvieron, y vinieron á Emmisphat, que es Cades, y devastaron todas las
haciendas de los Amalecitas, y también al Amorrheo, que habitaba en
Hazezón-tamar. \bibverse{8} Y salió el rey de Sodoma, y el rey de
Gomorra, y el rey de Adma, y el rey de Zeboim, y el rey de Bela, que es
Zoar, y ordenaron contra ellos batalla en el valle de Siddim;
\bibverse{9} Es á saber, contra Chêdorlaomer, rey de Elam, y Tidal, rey
de naciones, y Amraphel, rey de Shinar, y Arioch, rey de Elasar; cuatro
reyes contra cinco. \bibverse{10} Y el valle de Siddim estaba lleno de
pozos de betún: y huyeron el rey de Sodoma y el de Gomorra, y cayeron
allí; y los demás huyeron al monte. \bibverse{11} Y tomaron toda la
riqueza de Sodoma y de Gomorra, y todas sus vituallas, y se fueron.
\bibverse{12} Tomaron también á Lot, hijo del hermano de Abram, que
moraba en Sodoma, y su hacienda, y se fueron. \footnote{\textbf{14:12}
  Gén 13,10-12}

\hypertarget{ayuda-de-abram-por-lot}{%
\subsection{Ayuda de Abram por Lot}\label{ayuda-de-abram-por-lot}}

\bibverse{13} Y vino uno de los que escaparon, y denunciólo á Abram el
Hebreo, que habitaba en el valle de Mamre Amorrheo, hermano de Eschôl y
hermano de Aner, los cuales estaban confederados con Abram.
\bibverse{14} Y oyó Abram que su hermano estaba prisionero, y armó sus
criados, los criados de su casa, trescientos dieciocho, y siguiólos
hasta Dan. \bibverse{15} Y derramóse sobre ellos de noche él y sus
siervos, é hiriólos, y fuélos siguiendo hasta Hobah, que está á la
izquierda de Damasco. \bibverse{16} Y recobró todos los bienes, y
también á Lot su hermano y su hacienda, y también las mujeres y gente.

\hypertarget{abram-encuentra-melchuxeesedec-rey-de-salem}{%
\subsection{Abram encuentra Melchîsedec, rey de
Salem}\label{abram-encuentra-melchuxeesedec-rey-de-salem}}

\bibverse{17} Y salió el rey de Sodoma á recibirlo, cuando volvía de la
derrota de Chêdorlaomer y de los reyes que con él estaban, al valle de
Shave, que es el valle del Rey. \bibverse{18} Entonces Melchîsedec, rey
de Salem, sacó pan y vino; el cual era sacerdote del Dios alto;
\bibverse{19} Y bendíjole, y dijo: Bendito sea Abram del Dios alto,
poseedor de los cielos y de la tierra; \bibverse{20} Y bendito sea el
Dios alto, que entregó tus enemigos en tu mano. Y dióle Abram los
diezmos de todo.

\hypertarget{la-humildad-de-abram-con-el-rey-de-sodoma}{%
\subsection{La humildad de Abram con el rey de
Sodoma}\label{la-humildad-de-abram-con-el-rey-de-sodoma}}

\bibverse{21} Entonces el rey de Sodoma dijo á Abram: Dame las personas,
y toma para ti la hacienda.

\bibverse{22} Y respondió Abram al rey de Sodoma: He alzado mi mano á
Jehová Dios alto, poseedor de los cielos y de la tierra, \bibverse{23}
Que desde un hilo hasta la correa de un calzado, nada tomaré de todo lo
que es tuyo, porque no digas: Yo enriquecí á Abram: \bibverse{24}
Sacando solamente lo que comieron los mancebos, y la porción de los
varones que fueron conmigo, Aner, Eschôl, y Mamre; los cuales tomarán su
parte.

\hypertarget{dios-promete-a-abram-un-hijo}{%
\subsection{Dios promete a Abram un
hijo}\label{dios-promete-a-abram-un-hijo}}

\hypertarget{section-14}{%
\section{15}\label{section-14}}

\bibverse{1} DESPUÉS de estas cosas fué la palabra de Jehová á Abram en
visión, diciendo: No temas, Abram; yo soy tu escudo, y tu galardón
sobremanera grande. \footnote{\textbf{15:1} Sal 3,4; Sal 84,12; Sal
  119,114}

\bibverse{2} Y respondió Abram: Señor Jehová ¿qué me has de dar, siendo
así que ando sin hijo, y el mayordomo de mi casa es ese Damasceno
Eliezer? \bibverse{3} Dijo más Abram: Mira que no me has dado prole, y
he aquí que es mi heredero uno nacido en mi casa.

\bibverse{4} Y luego la palabra de Jehová fué á él diciendo: No te
heredará éste, sino el que saldrá de tus entrañas será el que te herede.
\bibverse{5} Y sacóle fuera, y dijo: Mira ahora á los cielos, y cuenta
las estrellas, si las puedes contar. Y le dijo: Así será tu simiente.
\bibverse{6} Y creyó á Jehová, y contóselo por justicia. \footnote{\textbf{15:6}
  Rom 4,3-5; Rom 4,18-22; Sant 2,23}

\hypertarget{dios-confirme-su-promesa}{%
\subsection{Dios confirme su promesa}\label{dios-confirme-su-promesa}}

\bibverse{7} Y díjole: Yo soy Jehová, que te saqué de Ur de los Caldeos,
para darte á heredar esta tierra. \footnote{\textbf{15:7} Gén 11,31}
\bibverse{8} Y él respondió: Señor Jehová, ¿en qué conoceré que la tengo
de heredar? \footnote{\textbf{15:8} 2Re 20,8; Luc 1,18}

\bibverse{9} Y le dijo: Apártame una becerra de tres años, y una cabra
de tres años, y un carnero de tres años, una tórtola también, y un
palomino. \bibverse{10} Y tomó él todas estas cosas, y partiólas por la
mitad, y puso cada mitad una enfrente de otra; mas no partió las aves.
\bibverse{11} Y descendían aves sobre los cuerpos muertos, y ojeábalas
Abram.

\bibverse{12} Mas á la caída del sol sobrecogió el sueño á Abram, y he
aquí que el pavor de una grande obscuridad cayó sobre él. \footnote{\textbf{15:12}
  Job 4,13-14} \bibverse{13} Entonces dijo á Abram: Ten por cierto que
tu simiente será peregrina en tierra no suya, y servirá á los de allí, y
serán por ellos afligidos cuatrocientos años. \footnote{\textbf{15:13}
  Éxod 12,40; Hech 7,6} \bibverse{14} Mas también á la gente á quien
servirán, juzgaré yo; y después de esto saldrán con grande riqueza.
\footnote{\textbf{15:14} Éxod 3,21-22} \bibverse{15} Y tú vendrás á tus
padres en paz, y serás sepultado en buena vejez. \bibverse{16} Y en la
cuarta generación volverán acá: porque aun no está cumplida la maldad
del Amorrheo hasta aquí. \bibverse{17} Y sucedió que puesto el sol, y ya
obscurecido, dejóse ver un horno humeando, y una antorcha de fuego que
pasó por entre los animales divididos. \bibverse{18} En aquel día hizo
Jehová un pacto con Abram diciendo: A tu simiente daré esta tierra desde
el río de Egipto hasta el río grande, el río Eufrates; \bibverse{19} Los
Cineos, y los Ceneceos, y los Cedmoneos, \footnote{\textbf{15:19} Gén
  10,15-18} \bibverse{20} Y los Hetheos, y los Pherezeos, y los
Raphaitas, \footnote{\textbf{15:20} Núm 13,33} \bibverse{21} Y los
Amorrheos, y los Cananeos, y los Gergeseos, y los Jebuseos.

\hypertarget{sarai-da-a-su-sierva-agar-como-mujer-uxe1-abram}{%
\subsection{Sarai da a su sierva Agar como mujer á
Abram}\label{sarai-da-a-su-sierva-agar-como-mujer-uxe1-abram}}

\hypertarget{section-15}{%
\section{16}\label{section-15}}

\bibverse{1} Y SARAI, mujer de Abram, no le paría: y ella tenía una
sierva egipcia, que se llamaba Agar. \bibverse{2} Dijo, pues, Sarai á
Abram: Ya ves que Jehová me ha hecho estéril: ruégote que entres á mi
sierva; quizá tendré hijos de ella. Y atendió Abram al dicho de Sarai.
\footnote{\textbf{16:2} Gén 30,3; Gén 30,9; 1Cor 7,2} \bibverse{3} Y
Sarai, mujer de Abram, tomó á Agar su sierva egipcia, al cabo de diez
años que había habitado Abram en la tierra de Canaán, y dióla á Abram su
marido por mujer. \bibverse{4} Y él cohabitó con Agar, la cual concibió:
y cuando vió que había concebido, miraba con desprecio á su señora.
\bibverse{5} Entonces Sarai dijo á Abram: Mi afrenta sea sobre ti: yo
puse mi sierva en tu seno, y viéndose embarazada, me mira con desprecio;
juzgue Jehová entre mí y ti.

\bibverse{6} Y respondió Abram á Sarai: He ahí tu sierva en tu mano, haz
con ella lo que bien te pareciere. Y como Sarai la afligiese, huyóse de
su presencia.

\hypertarget{dios-se-aparece-a-agar-uxe1-una-fuente-de-agua-en-el-desierto}{%
\subsection{Dios se aparece a Agar á una fuente de agua en el
desierto}\label{dios-se-aparece-a-agar-uxe1-una-fuente-de-agua-en-el-desierto}}

\bibverse{7} Y hallóla el ángel de Jehová junto á una fuente de agua en
el desierto, junto á la fuente que está en el camino del Sur.
\bibverse{8} Y le dijo: Agar, sierva de Sarai, ¿de dónde vienes tú, y á
dónde vas? Y ella respondió: Huyo de delante de Sarai, mi señora.

\bibverse{9} Y díjole el ángel de Jehová: Vuélvete á tu señora, y ponte
sumisa bajo de su mano. \bibverse{10} Díjole también el ángel de Jehová:
Multiplicaré tanto tu linaje, que no será contado á causa de la
muchedumbre. \bibverse{11} Díjole aún el ángel de Jehová: He aquí que
has concebido, y parirás un hijo, y llamarás su nombre Ismael, porque
oído ha Jehová tu aflicción. \bibverse{12} Y él será hombre fiero; su
mano contra todos, y las manos de todos contra él, y delante de todos
sus hermanos habitará. \footnote{\textbf{16:12} Gén 25,18}

\bibverse{13} Entonces llamó el nombre de Jehová que con ella hablaba:
Tú eres el Dios de la vista; porque dijo: ¿No he visto también aquí al
que me ve? \bibverse{14} Por lo cual llamó al pozo, Pozo del Viviente
que me ve. He aquí está entre Cades y Bered.

\bibverse{15} Y parió Agar á Abram un hijo, y llamó Abram el nombre de
su hijo que le parió Agar, Ismael. \bibverse{16} Y era Abram de edad de
ochenta y seis años, cuando parió Agar á Ismael.

\hypertarget{dios-confirme-su-pacto-con-abram}{%
\subsection{Dios confirme su pacto con
Abram}\label{dios-confirme-su-pacto-con-abram}}

\hypertarget{section-16}{%
\section{17}\label{section-16}}

\bibverse{1} Y SIENDO Abram de edad de noventa y nueve años, aparecióle
Jehová, y le dijo: Yo soy el Dios Todopoderoso; anda delante de mí, y sé
perfecto. \footnote{\textbf{17:1} Gén 35,11; Éxod 6,3; Gén 48,15}
\bibverse{2} Y pondré mi pacto entre mí y ti, y multiplicarte he mucho
en gran manera.

\bibverse{3} Entonces Abram cayó sobre su rostro, y Dios habló con él
diciendo: \bibverse{4} Yo, he aquí mi pacto contigo: Serás padre de
muchedumbre de gentes: \bibverse{5} Y no se llamará más tu nombre Abram,
sino que será tu nombre Abraham, porque te he puesto por padre de
muchedumbre de gentes. \bibverse{6} Y multiplicarte he mucho en gran
manera, y te pondré en gentes, y reyes saldrán de ti. \bibverse{7} Y
estableceré mi pacto entre mí y ti, y tu simiente después de ti en sus
generaciones, por alianza perpetua, para serte á ti por Dios, y á tu
simiente después de ti. \bibverse{8} Y te daré á ti, y á tu simiente
después de ti, la tierra de tus peregrinaciones, toda la tierra de
Canaán en heredad perpetua; y seré el Dios de ellos. \footnote{\textbf{17:8}
  Gén 23,4; Gén 35,27; Heb 11,9-16}

\hypertarget{la-circuncision}{%
\subsection{La circuncision}\label{la-circuncision}}

\bibverse{9} Dijo de nuevo Dios á Abraham: Tú empero guardarás mi pacto,
tú y tu simiente después de ti por sus generaciones. \bibverse{10} Este
será mi pacto, que guardaréis entre mí y vosotros y tu simiente después
de ti: Será circuncidado todo varón de entre vosotros. \bibverse{11}
Circuncidaréis, pues, la carne de vuestro prepucio, y será por señal del
pacto entre mí y vosotros. \bibverse{12} Y de edad de ocho días será
circuncidado todo varón entre vosotros por vuestras generaciones: el
nacido en casa, y el comprado á dinero de cualquier extranjero, que no
fuere de tu simiente. \bibverse{13} Debe ser circuncidado el nacido en
tu casa, y el comprado por tu dinero: y estará mi pacto en vuestra carne
para alianza perpetua. \bibverse{14} Y el varón incircunciso que no
hubiere circuncidado la carne de su prepucio, aquella persona será
borrada de su pueblo; ha violado mi pacto.

\hypertarget{dios-promete-abraham-un-hijo-de-sara}{%
\subsection{Dios promete Abraham un hijo de
Sara}\label{dios-promete-abraham-un-hijo-de-sara}}

\bibverse{15} Dijo también Dios á Abraham: A Sarai tu mujer no la
llamarás Sarai, mas Sara será su nombre. \bibverse{16} Y bendecirla he,
y también te daré de ella hijo; sí, la bendeciré, y vendrá á ser madre
de naciones; reyes de pueblos serán de ella.

\bibverse{17} Entonces Abraham cayó sobre su rostro, y rióse, y dijo en
su corazón: ¿A hombre de cien años ha de nacer hijo? ¿y Sara, ya de
noventa años, ha de parir? \footnote{\textbf{17:17} Gén 18,12; Gén 21,6;
  Luc 1,18} \bibverse{18} Y dijo Abraham á Dios: Ojalá Ismael viva
delante de ti.

\bibverse{19} Y respondió Dios: Ciertamente Sara tu mujer te parirá un
hijo, y llamarás su nombre Isaac; y confirmaré mi pacto con él por
alianza perpetua para su simiente después de él. \bibverse{20} Y en
cuanto á Ismael, también te he oído: he aquí que le bendeciré, y le haré
fructificar y multiplicar mucho en gran manera: doce príncipes
engendrará, y ponerlo he por gran gente. \footnote{\textbf{17:20} Gén
  16,10; Gén 21,13; Gén 21,18; Gén 25,16} \bibverse{21} Mas yo
estableceré mi pacto con Isaac, al cual te parirá Sara por este tiempo
el año siguiente.

\bibverse{22} Y acabó de hablar con él, y subió Dios de con Abraham.

\hypertarget{abraham-realiza-la-circuncisiuxf3n}{%
\subsection{Abraham realiza la
circuncisión}\label{abraham-realiza-la-circuncisiuxf3n}}

\bibverse{23} Entonces tomó Abraham á Ismael su hijo, y á todos los
siervos nacidos en su casa, y á todos los comprados por su dinero, á
todo varón entre los domésticos de la casa de Abraham, y circuncidó la
carne del prepucio de ellos en aquel mismo día, como Dios le había
dicho. \bibverse{24} Era Abraham de edad de noventa y nueve años cuando
circuncidó la carne de su prepucio. \bibverse{25} E Ismael su hijo era
de trece años, cuando fué circuncidada la carne de su prepucio.
\bibverse{26} En el mismo día fué circuncidado Abraham é Ismael su hijo.
\bibverse{27} Y todos los varones de su casa, el siervo nacido en casa,
y el comprado por dinero del extranjero, fueron circuncidados con él.

\hypertarget{dios-visita-a-abraham}{%
\subsection{Dios visita a Abraham}\label{dios-visita-a-abraham}}

\hypertarget{section-17}{%
\section{18}\label{section-17}}

\bibverse{1} Y APARECIÓLE Jehová en el valle de Mamre, estando él
sentado á la puerta de su tienda en el calor del día. \bibverse{2} Y
alzó sus ojos y miró, y he aquí tres varones que estaban junto á él: y
cuando los vió, salió corriendo de la puerta de su tienda á recibirlos,
é inclinóse hacia la tierra, \footnote{\textbf{18:2} Heb 13,2}
\bibverse{3} Y dijo: Señor, si ahora he hallado gracia en tus ojos,
ruégote que no pases de tu siervo. \bibverse{4} Que se traiga ahora un
poco de agua, y lavad vuestros pies; y recostaos debajo de un árbol,
\bibverse{5} Y traeré un bocado de pan, y sustentad vuestro corazón;
después pasaréis: porque por eso habéis pasado cerca de vuestro siervo.
Y ellos dijeron: Haz así como has dicho.

\bibverse{6} Entonces Abraham fué de priesa á la tienda á Sara, y le
dijo: Toma presto tres medidas de flor de harina, amasa y haz panes
cocidos debajo del rescoldo. \bibverse{7} Y corrió Abraham á las vacas,
y tomó un becerro tierno y bueno, y diólo al mozo, y dióse éste priesa á
aderezarlo. \bibverse{8} Tomó también manteca y leche, y el becerro que
había aderezado, y púsolo delante de ellos; y él estaba junto á ellos
debajo del árbol; y comieron.

\bibverse{9} Y le dijeron: ¿Dónde está Sara tu mujer? Y él respondió:
Aquí en la tienda.

\bibverse{10} Entonces dijo: De cierto volveré á ti según el tiempo de
la vida, y he aquí, tendrá un hijo Sara tu mujer. Y Sara escuchaba á la
puerta de la tienda, que estaba detrás de él.

\bibverse{11} Y Abraham y Sara eran viejos, entrados en días: á Sara
había cesado ya la costumbre de las mujeres. \bibverse{12} Rióse, pues,
Sara entre sí, diciendo: ¿Después que he envejecido tendré deleite,
siendo también mi señor ya viejo? \footnote{\textbf{18:12} Gén 17,17;
  1Pe 3,6}

\bibverse{13} Entonces Jehová dijo á Abraham: ¿Por qué se ha reído Sara
diciendo: Será cierto que he de parir siendo ya vieja? \bibverse{14}
¿Hay para Dios alguna cosa difícil? Al tiempo señalado volveré á ti,
según el tiempo de la vida, y Sara tendrá un hijo.

\bibverse{15} Entonces Sara negó diciendo: No me reí; porque tuvo miedo.
Y él dijo: No es así, sino que te has reído.

\hypertarget{la-intercesiuxf3n-de-abraham-por-sodoma}{%
\subsection{La intercesión de Abraham por
Sodoma}\label{la-intercesiuxf3n-de-abraham-por-sodoma}}

\bibverse{16} Y los varones se levantaron de allí, y miraron hacia
Sodoma: y Abraham iba con ellos acompañándolos. \bibverse{17} Y Jehová
dijo: ¿Encubriré yo á Abraham lo que voy á hacer, \bibverse{18} Habiendo
de ser Abraham en una nación grande y fuerte, y habiendo de ser benditas
en él todas las gentes de la tierra? \footnote{\textbf{18:18} Gén 12,3}
\bibverse{19} Porque yo lo he conocido, sé que mandará á sus hijos y á
su casa después de sí, que guarden el camino de Jehová, haciendo
justicia y juicio, para que haga venir Jehová sobre Abraham lo que ha
hablado acerca de él. \footnote{\textbf{18:19} Deut 6,7; Deut 32,46}
\bibverse{20} Entonces Jehová le dijo: Por cuanto el clamor de Sodoma y
Gomorra se aumenta más y más, y el pecado de ellos se ha agravado en
extremo, \footnote{\textbf{18:20} Gén 19,13} \bibverse{21} Descenderé
ahora, y veré si han consumado su obra según el clamor que ha venido
hasta mí; y si no, saberlo he. \footnote{\textbf{18:21} Gén 11,5; Sal
  34,16-17}

\bibverse{22} Y apartáronse de allí los varones, y fueron hacia Sodoma:
mas Abraham estaba aún delante de Jehová. \footnote{\textbf{18:22} Gén
  19,1} \bibverse{23} Y acercóse Abraham y dijo: ¿Destruirás también al
justo con el impío? \footnote{\textbf{18:23} Núm 16,22; 2Sam 24,17}
\bibverse{24} Quizá hay cincuenta justos dentro de la ciudad:
¿destruirás también y no perdonarás al lugar por cincuenta justos que
estén dentro de él? \bibverse{25} Lejos de ti el hacer tal, que hagas
morir al justo con el impío, y que sea el justo tratado como el impío;
nunca tal hagas. El juez de toda la tierra, ¿no ha de hacer lo que es
justo?

\bibverse{26} Entonces respondió Jehová: Si hallare en Sodoma cincuenta
justos dentro de la ciudad, perdonaré á todo este lugar por amor de
ellos. \footnote{\textbf{18:26} Is 65,8; Mat 24,22; Ezeq 22,30}
\bibverse{27} Y Abraham replicó y dijo: He aquí ahora que he comenzado á
hablar á mi Señor, aunque soy polvo y ceniza: \bibverse{28} Quizá
faltarán de cincuenta justos cinco: ¿destruirás por aquellos cinco toda
la ciudad? Y dijo: No la destruiré, si hallare allí cuarenta y cinco.

\bibverse{29} Y volvió á hablarle, y dijo: Quizá se hallarán allí
cuarenta. Y respondió: No lo haré por amor de los cuarenta.

\bibverse{30} Y dijo: No se enoje ahora mi Señor, si hablare: quizá se
hallarán allí treinta. Y respondió: No lo haré si hallare allí treinta.

\bibverse{31} Y dijo: He aquí ahora que he emprendido el hablar á mi
Señor: quizá se hallarán allí veinte. No la destruiré, respondió, por
amor de los veinte.

\bibverse{32} Y volvió á decir: No se enoje ahora mi Señor, si hablare
solamente una vez: quizá se hallarán allí diez. No la destruiré,
respondió, por amor de los diez.

\bibverse{33} Y fuése Jehová, luego que acabó de hablar á Abraham: y
Abraham se volvió á su lugar.

\hypertarget{la-cauxedda-de-sodoma-y-gomorrha}{%
\subsection{La caída de Sodoma y
Gomorrha}\label{la-cauxedda-de-sodoma-y-gomorrha}}

\hypertarget{section-18}{%
\section{19}\label{section-18}}

\bibverse{1} LLEGARON, pues, los dos ángeles á Sodoma á la caída de la
tarde: y Lot estaba sentado á la puerta de Sodoma. Y viéndolos Lot,
levantóse á recibirlos, é inclinóse hacia el suelo; \footnote{\textbf{19:1}
  Gén 18,22} \bibverse{2} Y dijo: Ahora, pues, mis señores, os ruego que
vengáis á casa de vuestro siervo y os hospedéis, y lavaréis vuestros
pies: y por la mañana os levantaréis, y seguiréis vuestro camino. Y
ellos respondieron: No, que en la plaza nos quedaremos esta noche.

\bibverse{3} Mas él porfió con ellos mucho, y se vinieron con él, y
entraron en su casa; é hízoles banquete, y coció panes sin levadura, y
comieron. \bibverse{4} Y antes que se acostasen, cercaron la casa los
hombres de la ciudad, los varones de Sodoma, todo el pueblo junto, desde
el más joven hasta el más viejo; \bibverse{5} Y llamaron á Lot, y le
dijeron: ¿Dónde están los varones que vinieron á ti esta noche?
sácanoslos, para que los conozcamos.

\bibverse{6} Entonces Lot salió á ellos á la puerta, y cerró las puertas
tras sí, \bibverse{7} Y dijo: Os ruego, hermanos míos, que no hagáis tal
maldad. \bibverse{8} He aquí ahora yo tengo dos hijas que no han
conocido varón; os las sacaré afuera, y haced de ellas como bien os
pareciere: solamente á estos varones no hagáis nada, pues que vinieron á
la sombra de mi tejado.

\bibverse{9} Y ellos respondieron: Quita allá: y añadieron: Vino éste
aquí para habitar como un extraño, ¿y habrá de erigirse en juez? Ahora
te haremos más mal que á ellos. Y hacían gran violencia al varón, á Lot,
y se acercaron para romper las puertas. \bibverse{10} Entonces los
varones alargaron la mano, y metieron á Lot en casa con ellos, y
cerraron las puertas. \bibverse{11} Y á los hombres que estaban á la
puerta de la casa desde el menor hasta el mayor, hirieron con ceguera;
mas ellos se fatigaban por hallar la puerta. \footnote{\textbf{19:11}
  2Re 6,18}

\hypertarget{la-salvacion-de-lot}{%
\subsection{La salvacion de Lot}\label{la-salvacion-de-lot}}

\bibverse{12} Y dijeron los varones á Lot: ¿Tienes aquí alguno más?
Yernos, y tus hijos y tus hijas, y todo lo que tienes en la ciudad,
sácalo de este lugar: \bibverse{13} Porque vamos á destruir este lugar,
por cuanto el clamor de ellos ha subido de punto delante de Jehová; por
tanto Jehová nos ha enviado para destruirlo.

\bibverse{14} Entonces salió Lot, y habló á sus yernos, los que habían
de tomar sus hijas, y les dijo: Levantaos, salid de este lugar; porque
Jehová va á destruir esta ciudad. Mas pareció á sus yernos como que se
burlaba. \footnote{\textbf{19:14} Núm 16,21}

\bibverse{15} Y al rayar el alba, los ángeles daban prisa á Lot,
diciendo: Levántate, toma tu mujer, y tus dos hijas que se hallan aquí,
porque no perezcas en el castigo de la ciudad. \bibverse{16} Y
deteniéndose él, los varones asieron de su mano, y de la mano de su
mujer, y de las manos de sus dos hijas, según la misericordia de Jehová
para con él; y le sacaron, y le pusieron fuera de la ciudad.
\bibverse{17} Y fué que cuando los hubo sacado fuera, dijo: Escapa por
tu vida; no mires tras ti, ni pares en toda esta llanura; escapa al
monte, no sea que perezcas.

\bibverse{18} Y Lot les dijo: No, yo os ruego, señores míos;
\bibverse{19} He aquí ahora ha hallado tu siervo gracia en tus ojos, y
has engrandecido tu misericordia que has hecho conmigo dándome la vida;
mas yo no podré escapar al monte, no sea caso que me alcance el mal, y
muera. \bibverse{20} He aquí ahora esta ciudad está cerca para huir
allá, la cual es pequeña; escaparé ahora allá, (¿no es ella pequeña?) y
vivirá mi alma.

\bibverse{21} Y le respondió: He aquí he recibido también tu súplica
sobre esto, y no destruiré la ciudad de que has hablado. \bibverse{22}
Date priesa, escápate allá; porque nada podré hacer hasta que allí hayas
llegado. Por esto fué llamado el nombre de la ciudad, Zoar.

\bibverse{23} El sol salía sobre la tierra, cuando Lot llegó á Zoar.
\bibverse{24} Entonces llovió Jehová sobre Sodoma y sobre Gomorra azufre
y fuego de parte de Jehová desde los cielos; \footnote{\textbf{19:24}
  Deut 29,22; Sal 11,6; Am 4,11; Luc 17,29; 2Pe 2,6; Is 1,9-10; Is 13,19}
\bibverse{25} Y destruyó las ciudades, y toda aquella llanura, con todos
los moradores de aquellas ciudades, y el fruto de la tierra.
\bibverse{26} Entonces la mujer de Lot miró atrás, á espaldas de él, y
se volvió estatua de sal.

\bibverse{27} Y subió Abraham por la mañana al lugar donde había estado
delante de Jehová: \bibverse{28} Y miró hacia Sodoma y Gomorra, y hacia
toda la tierra de aquella llanura miró; y he aquí que el humo subía de
la tierra como el humo de un horno.

\bibverse{29} Así fué que, cuando destruyó Dios las ciudades de la
llanura, acordóse Dios de Abraham, y envió fuera á Lot de en medio de la
destrucción, al asolar las ciudades donde Lot estaba.

\hypertarget{el-pecado-de-las-hijas-de-lot-el-nacimiento-de-los-padres-de-los-moabitas-y-ammonitas}{%
\subsection{El pecado de las hijas de Lot; El nacimiento de los padres
de los Moabitas y
Ammonitas}\label{el-pecado-de-las-hijas-de-lot-el-nacimiento-de-los-padres-de-los-moabitas-y-ammonitas}}

\bibverse{30} Empero Lot subió de Zoar, y asentó en el monte, y sus dos
hijas con él; porque tuvo miedo de quedar en Zoar, y se alojó en una
cueva él y sus dos hijas. \bibverse{31} Entonces la mayor dijo á la
menor: Nuestro padre es viejo, y no queda varón en la tierra que entre á
nosotras conforme á la costumbre de toda la tierra: \bibverse{32} Ven,
demos á beber vino á nuestro padre, y durmamos con él, y conservaremos
de nuestro padre generación. \footnote{\textbf{19:32} Lev 18,7}
\bibverse{33} Y dieron á beber vino á su padre aquella noche: y entró la
mayor, y durmió con su padre; mas él no sintió cuándo se acostó ella, ni
cuándo se levantó. \bibverse{34} El día siguiente dijo la mayor á la
menor: He aquí yo dormí la noche pasada con mi padre; démosle á beber
vino también esta noche, y entra y duerme con él, para que conservemos
de nuestro padre generación. \bibverse{35} Y dieron á beber vino á su
padre también aquella noche: y levantóse la menor, y durmió con él; pero
no echó de ver cuándo se acostó ella, ni cuándo se levantó.
\bibverse{36} Y concibieron las dos hijas de Lot, de su padre.
\bibverse{37} Y parió la mayor un hijo, y llamó su nombre Moab, el cual
es padre de los Moabitas hasta hoy. \bibverse{38} La menor también parió
un hijo, y llamó su nombre Ben-ammí, el cual es padre de los Ammonitas
hasta hoy. \footnote{\textbf{19:38} Deut 2,19}

\hypertarget{abraham-donde-abimelech-en-gerar}{%
\subsection{Abraham donde Abimelech en
Gerar}\label{abraham-donde-abimelech-en-gerar}}

\hypertarget{section-19}{%
\section{20}\label{section-19}}

\bibverse{1} DE allí partió Abraham á la tierra del Mediodía, y asentó
entre Cades y Shur, y habitó como forastero en Gerar. \footnote{\textbf{20:1}
  Gén 12,9-10; Gén 26,1} \bibverse{2} Y dijo Abraham de Sara su mujer:
Mi hermana es. Y Abimelech, rey de Gerar, envió y tomó á Sara.
\bibverse{3} Empero Dios vino á Abimelech en sueños de noche, y le dijo:
He aquí muerto eres á causa de la mujer que has tomado, la cual es
casada con marido.

\bibverse{4} Mas Abimelech no había llegado á ella, y dijo: Señor,
¿matarás también la gente justa? \bibverse{5} ¿No me dijo él: Mi hermana
es; y ella también dijo: Es mi hermano? Con sencillez de mi corazón, y
con limpieza de mis manos he hecho esto.

\bibverse{6} Y díjole Dios en sueños: Yo también sé que con integridad
de tu corazón has hecho esto; y yo también te detuve de pecar contra mí,
y así no te permití que la tocases. \bibverse{7} Ahora, pues, vuelve la
mujer á su marido; porque es profeta, y orará por ti, y vivirás. Y si tú
no la volvieres, sabe que de cierto morirás, con todo lo que fuere tuyo.
\footnote{\textbf{20:7} Sal 105,15}

\bibverse{8} Entonces Abimelech se levantó de mañana, y llamó á todos
sus siervos, y dijo todas estas palabras en los oídos de ellos; y
temieron los hombres en gran manera. \bibverse{9} Después llamó
Abimelech á Abraham, y le dijo: ¿Qué nos has hecho? ¿y en qué pequé yo
contra ti, que has atraído sobre mí y sobre mi reino tan gran pecado? lo
que no debiste hacer has hecho conmigo. \bibverse{10} Y dijo más
Abimelech á Abraham: ¿Qué viste para que hicieses esto?

\bibverse{11} Y Abraham respondió: Porque dije para mí: Cierto no hay
temor de Dios en este lugar, y me matarán por causa de mi mujer.
\bibverse{12} Y á la verdad también es mi hermana, hija de mi padre, mas
no hija de mi madre, y toméla por mujer. \bibverse{13} Y fué que, cuando
Dios me hizo salir errante de la casa de mi padre, yo le dije: Esta es
la merced que tú me harás, que en todos los lugares donde llegáremos,
digas de mí: Mi hermano es.

\bibverse{14} Entonces Abimelech tomó ovejas y vacas, y siervos y
siervas, y diólo á Abraham, y devolvióle á Sara su mujer. \bibverse{15}
Y dijo Abimelech: He aquí mi tierra está delante de ti, habita donde
bien te pareciere. \bibverse{16} Y á Sara dijo: He aquí he dado mil
monedas de plata á tu hermano; mira que él te es por velo de ojos para
todos los que están contigo, y para con todos: así fué reprendida.

\bibverse{17} Entonces Abraham oró á Dios; y Dios sanó á Abimelech y á
su mujer, y á sus siervas, y parieron. \bibverse{18} Porque había del
todo cerrado Jehová toda matriz de la casa de Abimelech, á causa de Sara
mujer de Abraham.

\hypertarget{nacimiento-de-isaac}{%
\subsection{Nacimiento de Isaac}\label{nacimiento-de-isaac}}

\hypertarget{section-20}{%
\section{21}\label{section-20}}

\bibverse{1} Y VISITÓ Jehová á Sara, como había dicho, é hizo Jehová con
Sara como había hablado. \bibverse{2} Y concibió y parió Sara á Abraham
un hijo en su vejez, en el tiempo que Dios le había dicho. \footnote{\textbf{21:2}
  Heb 11,11} \bibverse{3} Y llamó Abraham el nombre de su hijo que le
nació, que le parió Sara, Isaac. \footnote{\textbf{21:3} Gén 17,19}
\bibverse{4} Y circuncidó Abraham á su hijo Isaac de ocho días, como
Dios le había mandado. \footnote{\textbf{21:4} Gén 17,11-12}
\bibverse{5} Y era Abraham de cien años, cuando le nació Isaac su hijo.
\footnote{\textbf{21:5} Gén 17,17; Rom 4,19} \bibverse{6} Entonces dijo
Sara: Dios me ha hecho reir, y cualquiera que lo oyere, se reirá
conmigo. \footnote{\textbf{21:6} Gén 18,12} \bibverse{7} Y añadió:
¿Quién dijera á Abraham que Sara había de dar de mamar á hijos? pues que
le he parido un hijo á su vejez. \bibverse{8} Y creció el niño, y fué
destetado; é hizo Abraham gran banquete el día que fué destetado Isaac.

\hypertarget{el-repudio-y-la-salvaciuxf3n-de-ismael}{%
\subsection{El repudio y la salvación de
Ismael}\label{el-repudio-y-la-salvaciuxf3n-de-ismael}}

\bibverse{9} Y vió Sara al hijo de Agar la Egipcia, el cual había ésta
parido á Abraham, que se burlaba. \bibverse{10} Por tanto dijo á
Abraham: Echa á esta sierva y á su hijo; que el hijo de esta sierva no
ha de heredar con mi hijo, con Isaac.

\bibverse{11} Este dicho pareció grave en gran manera á Abraham á causa
de su hijo. \bibverse{12} Entonces dijo Dios á Abraham: No te parezca
grave á causa del muchacho y de tu sierva; en todo lo que te dijere
Sara, oye su voz, porque en Isaac te será llamada descendencia.
\footnote{\textbf{21:12} Rom 9,7-8; Heb 11,18} \bibverse{13} Y también
al hijo de la sierva pondré en gente, porque es tu simiente. \footnote{\textbf{21:13}
  Gén 17,20} \bibverse{14} Entonces Abraham se levantó muy de mañana, y
tomó pan, y un odre de agua, y diólo á Agar, poniéndolo sobre su hombro,
y entrególe el muchacho, y despidióla. Y ella partió, y andaba errante
por el desierto de Beer-seba. \bibverse{15} Y faltó el agua del odre, y
echó al muchacho debajo de un árbol; \bibverse{16} Y fuése y sentóse
enfrente, alejándose como un tiro de arco; porque decía: No veré cuando
el muchacho morirá: y sentóse enfrente, y alzó su voz y lloró.
\bibverse{17} Y oyó Dios la voz del muchacho; y el ángel de Dios llamó á
Agar desde el cielo, y le dijo: ¿Qué tienes, Agar? No temas; porque Dios
ha oído la voz del muchacho en donde está.

\bibverse{18} Levántate, alza al muchacho, y ásele de tu mano, porque en
gran gente lo tengo de poner.

\bibverse{19} Entonces abrió Dios sus ojos, y vió una fuente de agua; y
fué, y llenó el odre de agua, y dió de beber al muchacho.

\bibverse{20} Y fué Dios con el muchacho; y creció, y habitó en el
desierto, y fué tirador de arco. \bibverse{21} Y habitó en el desierto
de Parán; y su madre le tomó mujer de la tierra de Egipto. \footnote{\textbf{21:21}
  Gén 16,3}

\hypertarget{el-pacto-entre-abraham-y-abimelech}{%
\subsection{El pacto entre Abraham y
Abimelech}\label{el-pacto-entre-abraham-y-abimelech}}

\bibverse{22} Y aconteció en aquel mismo tiempo que habló Abimelech, y
Phicol, príncipe de su ejército, á Abraham diciendo: Dios es contigo en
todo cuanto haces: \footnote{\textbf{21:22} Gén 26,26} \bibverse{23}
Ahora pues, júrame aquí por Dios, que no faltarás á mí, ni á mi hijo, ni
á mi nieto; sino que conforme á la bondad que yo hice contigo, harás tú
conmigo, y con la tierra donde has peregrinado. \footnote{\textbf{21:23}
  Gén 20,15}

\bibverse{24} Y respondió Abraham: Yo juraré. \bibverse{25} Y Abraham
reconvino á Abimelech á causa de un pozo de agua, que los siervos de
Abimelech le habían quitado. \bibverse{26} Y respondió Abimelech: No sé
quién haya hecho esto, ni tampoco tú me lo hiciste saber, ni yo lo he
oído hasta hoy.

\bibverse{27} Y tomó Abraham ovejas y vacas, y dió á Abimelech; é
hicieron ambos alianza. \bibverse{28} Y puso Abraham siete corderas del
rebaño aparte. \bibverse{29} Y dijo Abimelech á Abraham: ¿Qué significan
esas siete corderas que has puesto aparte?

\bibverse{30} Y él respondió: Que estas siete corderas tomarás de mi
mano, para que me sean en testimonio de que yo cavé este pozo.
\bibverse{31} Por esto llamó á aquel lugar Beer-seba; porque allí
juraron ambos. \footnote{\textbf{21:31} Gén 26,33} \bibverse{32} Así
hicieron alianza en Beer-seba: y levantóse Abimelech, y Phicol, príncipe
de su ejército, y se volvieron á tierra de los Filisteos. \bibverse{33}
Y plantó Abraham un bosque en Beer-seba, é invocó allí el nombre de
Jehová Dios eterno. \bibverse{34} Y moró Abraham en tierra de los
Filisteos muchos días.

\hypertarget{el-orden-de-dios-para-sacrificar-isaac}{%
\subsection{El orden de Dios para sacrificar
Isaac}\label{el-orden-de-dios-para-sacrificar-isaac}}

\hypertarget{section-21}{%
\section{22}\label{section-21}}

\bibverse{1} Y ACONTECIÓ después de estas cosas, que tentó Dios á
Abraham, y le dijo: Abraham. Y él respondió: Heme aquí. \footnote{\textbf{22:1}
  Heb 11,17; Sant 1,12}

\bibverse{2} Y dijo: Toma ahora tu hijo, tu único, Isaac, á quien amas,
y vete á tierra de Moriah, y ofrécelo allí en holocausto sobre uno de
los montes que yo te diré. \footnote{\textbf{22:2} 2Cró 3,1}

\hypertarget{la-obedencia-de-abraham}{%
\subsection{La obedencia de Abraham}\label{la-obedencia-de-abraham}}

\bibverse{3} Y Abraham se levantó muy de mañana, y enalbardó su asno, y
tomó consigo dos mozos suyos, y á Isaac su hijo: y cortó leña para el
holocausto, y levantóse, y fué al lugar que Dios le dijo. \bibverse{4}
Al tercer día alzó Abraham sus ojos, y vió el lugar de lejos.
\bibverse{5} Entonces dijo Abraham á sus mozos: Esperaos aquí con el
asno, y yo y el muchacho iremos hasta allí, y adoraremos, y volveremos á
vosotros. \bibverse{6} Y tomó Abraham la leña del holocausto, y púsola
sobre Isaac su hijo: y él tomó en su mano el fuego y el cuchillo; y
fueron ambos juntos. \bibverse{7} Entonces habló Isaac á Abraham su
padre, y dijo: Padre mío. Y él respondió: Heme aquí, mi hijo. Y él dijo:
He aquí el fuego y la leña; mas ¿dónde está el cordero para el
holocausto? \bibverse{8} Y respondió Abraham: Dios se proveerá de
cordero para el holocausto, hijo mío. E iban juntos.

\hypertarget{la-preparacion-del-holocausto-y-la-intervenciuxf3n-de-dios}{%
\subsection{La preparacion del holocausto y la intervención de
Dios}\label{la-preparacion-del-holocausto-y-la-intervenciuxf3n-de-dios}}

\bibverse{9} Y como llegaron al lugar que Dios le había dicho, edificó
allí Abraham un altar, y compuso la leña, y ató á Isaac su hijo, y
púsole en el altar sobre la leña. \bibverse{10} Y extendió Abraham su
mano, y tomó el cuchillo, para degollar á su hijo. \footnote{\textbf{22:10}
  Sant 2,21}

\bibverse{11} Entonces el ángel de Jehová le dió voces del cielo, y
dijo: Abraham, Abraham. Y él respondió: Heme aquí.

\bibverse{12} Y dijo: No extiendas tu mano sobre el muchacho, ni le
hagas nada; que ya conozco que temes á Dios, pues que no me rehusaste tu
hijo, tu único.

\bibverse{13} Entonces alzó Abraham sus ojos, y miró, y he aquí un
carnero á sus espaldas trabado en un zarzal por sus cuernos: y fué
Abraham, y tomó el carnero, y ofrecióle en holocausto en lugar de su
hijo. \bibverse{14} Y llamó Abraham el nombre de aquel lugar, Jehová
proveerá. Por tanto se dice hoy: En el monte de Jehová será provisto.

\hypertarget{la-aprobaciuxf3n-de-dios-y-promesas-por-abraham}{%
\subsection{La aprobación de Dios y promesas por
Abraham}\label{la-aprobaciuxf3n-de-dios-y-promesas-por-abraham}}

\bibverse{15} Y llamó el ángel de Jehová á Abraham segunda vez desde el
cielo, \bibverse{16} Y dijo: Por mí mismo he jurado, dice Jehová, que
por cuanto has hecho esto, y no me has rehusado tu hijo, tu único;
\footnote{\textbf{22:16} Heb 6,13} \bibverse{17} Bendiciendo te
bendeciré, y multiplicando multiplicaré tu simiente como las estrellas
del cielo, y como la arena que está á la orilla del mar; y tu simiente
poseerá las puertas de sus enemigos: \footnote{\textbf{22:17} Gén 13,16;
  Gén 15,5; Heb 11,12; Gén 24,60} \bibverse{18} En tu simiente serán
benditas todas las gentes de la tierra, por cuanto obedeciste á mi voz.
\footnote{\textbf{22:18} Gén 12,3; Gal 3,16}

\bibverse{19} Y tornóse Abraham á sus mozos, y levantáronse y se fueron
juntos á Beer-seba; y habitó Abraham en Beer-seba.

\hypertarget{los-descendientes-de-nahor-el-hermano-de-abraham}{%
\subsection{Los descendientes de Nahor, el hermano de
Abraham}\label{los-descendientes-de-nahor-el-hermano-de-abraham}}

\bibverse{20} Y aconteció después de estas cosas, que fué dada nueva á
Abraham, diciendo: He aquí que también Milca ha parido hijos á Nachôr tu
hermano: \bibverse{21} A Huz su primogénito, y á Buz su hermano, y á
Kemuel padre de Aram, \bibverse{22} Y á Chêsed, y á Hazo, y á Pildas, y
á Jidlaph, y á Bethuel. \bibverse{23} Y Bethuel engendró á Rebeca. Estos
ocho parió Milca á Nachôr, hermano de Abraham. \footnote{\textbf{22:23}
  Gén 24,15} \bibverse{24} Y su concubina, que se llamaba Reúma, parió
también á Teba, y á Gaham, y á Taas, y á Maachâ.

\hypertarget{muerte-y-sepultura-de-sara}{%
\subsection{Muerte y sepultura de
Sara}\label{muerte-y-sepultura-de-sara}}

\hypertarget{section-22}{%
\section{23}\label{section-22}}

\bibverse{1} Y FUÉ la vida de Sara ciento veintisiete años: tantos
fueron los años de la vida de Sara. \bibverse{2} Y murió Sara en
Kiriath-arba, que es Hebrón, en la tierra de Canaán: y vino Abraham á
hacer el duelo á Sara, y á llorarla. \bibverse{3} Y levantóse Abraham de
delante de su muerto, y habló á los hijos de Heth, diciendo:
\bibverse{4} Peregrino y advenedizo soy entre vosotros; dadme heredad de
sepultura con vosotros, y sepultaré mi muerto de delante de mí.

\bibverse{5} Y respondieron los hijos de Heth á Abraham, y dijéronle:
\bibverse{6} Oyenos, señor mío, eres un príncipe de Dios entre nosotros;
en lo mejor de nuestras sepulturas sepulta á tu muerto; ninguno de
nosotros te impedirá su sepultura, para que entierres tu muerto.

\bibverse{7} Y Abraham se levantó, é inclinóse al pueblo de aquella
tierra, á los hijos de Heth; \bibverse{8} Y habló con ellos, diciendo:
Si tenéis voluntad que yo sepulte mi muerto de delante de mí, oidme, é
interceded por mí con Ephrón, hijo de Zohar, \bibverse{9} Para que me dé
la cueva de Macpela, que tiene al cabo de su heredad: que por su justo
precio me la dé, para posesión de sepultura en medio de vosotros.

\bibverse{10} Este Ephrón hallábase entre los hijos de Heth: y respondió
Ephrón Hetheo á Abraham, en oídos de los hijos de Heth, de todos los que
entraban por la puerta de su ciudad, diciendo: \bibverse{11} No, señor
mío, óyeme: te doy la heredad, y te doy también la cueva que está en
ella; delante de los hijos de mi pueblo te la doy; sepulta tu muerto.

\bibverse{12} Y Abraham se inclinó delante del pueblo de la tierra.
\bibverse{13} Y respondió á Ephrón en oídos del pueblo de la tierra,
diciendo: Antes, si te place, ruégote que me oigas; yo daré el precio de
la heredad, tómalo de mí, y sepultaré en ella mi muerto.

\bibverse{14} Y respondió Ephrón á Abraham, diciéndole: \bibverse{15}
Señor mío, escúchame: la tierra vale cuatrocientos siclos de plata: ¿qué
es esto entre mí y ti? entierra pues tu muerto.

\bibverse{16} Entonces Abraham se convino con Ephrón, y pesó Abraham á
Ephrón el dinero que dijo, oyéndolo los hijos de Heth, cuatrocientos
siclos de plata, de buena ley entre mercaderes.

\bibverse{17} Y quedó la heredad de Ephrón que estaba en Macpela
enfrente de Mamre, la heredad y la cueva que estaba en ella, y todos los
árboles que había en la heredad, y en todo su término al derredor,
\bibverse{18} Por de Abraham en posesión, á vista de los hijos de Heth,
y de todos los que entraban por la puerta de la ciudad. \bibverse{19} Y
después de esto sepultó Abraham á Sara su mujer en la cueva de la
heredad de Macpela enfrente de Mamre, que es Hebrón en la tierra de
Canaán. \bibverse{20} Y quedó la heredad y la cueva que en ella había,
por de Abraham, en posesión de sepultura adquirida de los hijos de Heth.
\footnote{\textbf{23:20} Gén 25,9-10; Gén 47,30; Gén 49,29-30; Gén 50,13}

\hypertarget{abraham-envuxeda-a-su-criado-para-buscar-una-esposa-por-isaac}{%
\subsection{Abraham envía a su criado para buscar una esposa por
Isaac}\label{abraham-envuxeda-a-su-criado-para-buscar-una-esposa-por-isaac}}

\hypertarget{section-23}{%
\section{24}\label{section-23}}

\bibverse{1} Y ABRAHAM era viejo, y bien entrado en días; y Jehová había
bendecido á Abraham en todo. \footnote{\textbf{24:1} Gén 12,2; Sal
  112,2-3} \bibverse{2} Y dijo Abraham á un criado suyo, el más viejo de
su casa, que era el que gobernaba en todo lo que tenía: Pon ahora tu
mano debajo de mi muslo, \bibverse{3} Y te juramentaré por Jehová, Dios
de los cielos y Dios de la tierra, que no has de tomar mujer para mi
hijo de las hijas de los Cananeos, entre los cuales yo habito;
\footnote{\textbf{24:3} Gén 28,1; Éxod 34,16} \bibverse{4} Sino que irás
á mi tierra y á mi parentela, y tomarás mujer para mi hijo Isaac.

\bibverse{5} Y el criado le respondió: Quizá la mujer no querrá venir en
pos de mí á esta tierra: ¿volveré, pues, tu hijo á la tierra de donde
saliste?

\bibverse{6} Y Abraham le dijo: Guárdate que no vuelvas á mi hijo allá.
\bibverse{7} Jehová, Dios de los cielos, que me tomó de la casa de mi
padre y de la tierra de mi parentela, y me habló y me juró, diciendo: A
tu simiente daré esta tierra; él enviará su ángel delante de ti, y tú
tomarás de allá mujer para mi hijo. \bibverse{8} Y si la mujer no
quisiere venir en pos de ti, serás libre de este mi juramento; solamente
que no vuelvas allá á mi hijo. \bibverse{9} Entonces el criado puso su
mano debajo del muslo de Abraham su señor, y juróle sobre este negocio.

\hypertarget{la-viaje-del-criado-por-haran}{%
\subsection{La viaje del criado por
Haran}\label{la-viaje-del-criado-por-haran}}

\bibverse{10} Y el criado tomó diez camellos de los camellos de su
señor, y fuése, pues tenía á su disposición todos los bienes de su
señor: y puesto en camino, llegó á Mesopotamia, á la ciudad de Nachôr.
\footnote{\textbf{24:10} Gén 11,31; Gén 27,43} \bibverse{11} E hizo
arrodillar los camellos fuera de la ciudad, junto á un pozo de agua, á
la hora de la tarde, á la hora en que salen las mozas por agua.
\bibverse{12} Y dijo: Jehová, Dios de mi señor Abraham, dame, te ruego,
el tener hoy buen encuentro, y haz misericordia con mi señor Abraham.
\bibverse{13} He aquí yo estoy junto á la fuente de agua, y las hijas de
los varones de esta ciudad salen por agua: \bibverse{14} Sea, pues, que
la moza á quien yo dijere: Baja tu cántaro, te ruego, para que yo beba;
y ella respondiere: Bebe, y también daré de beber á tus camellos: que
sea ésta la que tú has destinado para tu siervo Isaac; y en esto
conoceré que habrás hecho misericordia con mi señor.

\bibverse{15} Y aconteció que antes que él acabase de hablar, he aquí
Rebeca, que había nacido á Bethuel, hijo de Milca, mujer de Nachôr
hermano de Abraham, la cual salía con su cántaro sobre su hombro.
\bibverse{16} Y la moza era de muy hermoso aspecto, virgen, á la que
varón no había conocido; la cual descendió á la fuente, y llenó su
cántaro, y se volvía. \bibverse{17} Entonces el criado corrió hacia
ella, y dijo: Ruégote que me des á beber un poco de agua de tu cántaro.

\bibverse{18} Y ella respondió: Bebe, señor mío: y dióse prisa á bajar
su cántaro sobre su mano, y le dió á beber. \bibverse{19} Y cuando acabó
de darle á beber, dijo: También para tus camellos sacaré agua, hasta que
acaben de beber. \bibverse{20} Y dióse prisa, y vació su cántaro en la
pila, y corrió otra vez al pozo para sacar agua, y sacó para todos sus
camellos.

\bibverse{21} Y el hombre estaba maravillado de ella, callando, para
saber si Jehová había prosperado ó no su viaje.

\hypertarget{el-criado-llega-a-la-casa-de-nachuxf4r}{%
\subsection{El criado llega a la casa de
Nachôr}\label{el-criado-llega-a-la-casa-de-nachuxf4r}}

\bibverse{22} Y fué que como los camellos acabaron de beber, presentóle
el hombre un pendiente de oro que pesaba medio siclo, y dos brazaletes
que pesaban diez: \bibverse{23} Y dijo: ¿De quién eres hija? Ruégote me
digas, ¿hay lugar en casa de tu padre donde posemos?

\bibverse{24} Y ella respondió: Soy hija de Bethuel, hijo de Milca, el
cual parió ella á Nachôr. \bibverse{25} Y añadió: También hay en nuestra
casa paja y mucho forraje, y lugar para posar.

\bibverse{26} El hombre entonces se inclinó, y adoró á Jehová.
\bibverse{27} Y dijo: Bendito sea Jehová, Dios de mi amo Abraham, que no
apartó su misericordia y su verdad de mi amo, guiándome Jehová en el
camino á casa de los hermanos de mi amo.

\bibverse{28} Y la moza corrió, é hizo saber en casa de su madre estas
cosas. \bibverse{29} Y Rebeca tenía un hermano que se llamaba Labán, el
cual corrió afuera al hombre, á la fuente; \bibverse{30} Y fué que como
vió el pendiente y los brazaletes en las manos de su hermana, que decía,
Así me habló aquel hombre; vino á él: y he aquí que estaba junto á los
camellos á la fuente. \bibverse{31} Y díjole: Ven, bendito de Jehová;
¿por qué estás fuera? yo he limpiado la casa, y el lugar para los
camellos.

\bibverse{32} Entonces el hombre vino á casa, y Labán desató los
camellos; y dióles paja y forraje, y agua para lavar los piés de él, y
los piés de los hombres que con él venían. \bibverse{33} Y pusiéronle
delante qué comer; mas él dijo: No comeré hasta que haya dicho mi
mensaje. Y él le dijo: Habla.

\hypertarget{el-cortejo-por-la-novia}{%
\subsection{El cortejo por la novia}\label{el-cortejo-por-la-novia}}

\bibverse{34} Entonces dijo: Yo soy criado de Abraham; \bibverse{35} Y
Jehová ha bendecido mucho á mi amo, y él se ha engrandecido: y le ha
dado ovejas y vacas, plata y oro, siervos y siervas, camellos y asnos.
\bibverse{36} Y Sara, mujer de mi amo, parió en su vejez un hijo á mi
señor, quien le ha dado todo cuanto tiene. \bibverse{37} Y mi amo me
hizo jurar, diciendo: No tomarás mujer para mi hijo de las hijas de los
Cananeos, en cuya tierra habito; \bibverse{38} Sino que irás á la casa
de mi padre, y á mi parentela, y tomarás mujer para mi hijo.
\bibverse{39} Y yo dije: Quizás la mujer no querrá seguirme.
\bibverse{40} Entonces él me respondió: Jehová, en cuya presencia he
andado, enviará su ángel contigo, y prosperará tu camino; y tomarás
mujer para mi hijo de mi linaje y de la casa de mi padre: \footnote{\textbf{24:40}
  Gén 17,1} \bibverse{41} Entonces serás libre de mi juramento, cuando
hubieres llegado á mi linaje; y si no te la dieren, serás libre de mi
juramento. \bibverse{42} Llegué, pues, hoy á la fuente, y dije: Jehová,
Dios de mi señor Abraham, si tú prosperas ahora mi camino por el cual
ando; \bibverse{43} He aquí yo estoy junto á la fuente de agua; sea,
pues, que la doncella que saliere por agua, á la cual dijere: Dame á
beber, te ruego, un poco de agua de tu cántaro; \bibverse{44} Y ella me
respondiere, Bebe tú, y también para tus camellos sacaré agua: ésta sea
la mujer que destinó Jehová para el hijo de mi señor. \bibverse{45} Y
antes que acabase de hablar en mi corazón, he aquí Rebeca, que salía con
su cántaro sobre su hombro; y descendió á la fuente, y sacó agua; y le
dije: Ruégote que me des á beber. \bibverse{46} Y prestamente bajó su
cántaro de encima de sí, y dijo: Bebe, y también á tus camellos daré á
beber. Y bebí, y dió también de beber á mis camellos. \bibverse{47}
Entonces preguntéle, y dije: ¿De quién eres hija? Y ella respondió: Hija
de Bethuel, hijo de Nachôr, que le parió Milca. Entonces púsele un
pendiente sobre su nariz, y brazaletes sobre sus manos: \bibverse{48} E
inclinéme, y adoré á Jehová, y bendije á Jehová, Dios de mi señor
Abraham, que me había guiado por camino de verdad para tomar la hija del
hermano de mi señor para su hijo. \bibverse{49} Ahora pues, si vosotros
hacéis misericordia y verdad con mi señor, declarádmelo; y si no,
declarádmelo; y echaré á la diestra ó á la siniestra.

\hypertarget{la-despedida-de-rebeca}{%
\subsection{La despedida de Rebeca}\label{la-despedida-de-rebeca}}

\bibverse{50} Entonces Labán y Bethuel respondieron y dijeron: De Jehová
ha salido esto; no podemos hablarte malo ni bueno. \bibverse{51} He ahí
Rebeca delante de ti; tómala y vete, y sea mujer del hijo de tu señor,
como lo ha dicho Jehová.

\bibverse{52} Y fué, que como el criado de Abraham oyó sus palabras,
inclinóse á tierra á Jehová. \bibverse{53} Y sacó el criado vasos de
plata, y vasos de oro y vestidos, y dió á Rebeca: también dió cosas
preciosas á su hermano y á su madre. \bibverse{54} Y comieron y bebieron
él y los varones que venían con él, y durmieron; y levantándose de
mañana, dijo: Enviadme á mi señor.

\bibverse{55} Entonces respondió su hermano y su madre: Espere la moza
con nosotros á lo menos diez días, y después irá.

\bibverse{56} Y él les dijo: No me detengáis, pues que Jehová ha
prosperado mi camino; despachadme para que me vaya á mi señor.

\bibverse{57} Ellos respondieron entonces: Llamemos la moza y
preguntémosle. \bibverse{58} Y llamaron á Rebeca, y dijéronle: ¿Irás tú
con este varón? Y ella respondió: Sí, iré.

\bibverse{59} Entonces dejaron ir á Rebeca su hermana, y á su nodriza, y
al criado de Abraham y á sus hombres. \bibverse{60} Y bendijeron á
Rebeca, y dijéronle: Nuestra hermana eres; seas en millares de millares,
y tu generación posea la puerta de sus enemigos.

\bibverse{61} Levantóse entonces Rebeca y sus mozas, y subieron sobre
los camellos, y siguieron al hombre; y el criado tomó á Rebeca, y fuése.

\hypertarget{la-llegada-de-la-novia-al-novio}{%
\subsection{La llegada de la novia al
novio}\label{la-llegada-de-la-novia-al-novio}}

\bibverse{62} Y venía Isaac del pozo del Viviente que me ve; porque él
habitaba en la tierra del Mediodía; \footnote{\textbf{24:62} Gén 16,14;
  Gén 25,11} \bibverse{63} Y había salido Isaac á orar al campo, á la
hora de la tarde; y alzando sus ojos miró, y he aquí los camellos que
venían. \bibverse{64} Rebeca también alzó sus ojos, y vió á Isaac, y
descendió del camello; \bibverse{65} Porque había preguntado al criado:
¿Quién es este varón que viene por el campo hacia nosotros? Y el siervo
había respondido: Este es mi señor. Ella entonces tomó el velo, y
cubrióse.

\bibverse{66} Entonces el criado contó á Isaac todo lo que había hecho.
\bibverse{67} E introdújola Isaac á la tienda de su madre Sara, y tomó á
Rebeca por mujer; y amóla: y consolóse Isaac después de la muerte de su
madre.

\hypertarget{segundo-matrimonio-de-abraham-su-muerte-y-entierro}{%
\subsection{Segundo matrimonio de Abraham; su muerte y
entierro}\label{segundo-matrimonio-de-abraham-su-muerte-y-entierro}}

\hypertarget{section-24}{%
\section{25}\label{section-24}}

\bibverse{1} Y ABRAHAM tomó otra mujer, cuyo nombre fué Cetura;
\bibverse{2} La cual le parió á Zimram, y á Joksan, y á Medan, y á
Midiam, y á Ishbak, y á Sua. \bibverse{3} Y Joksan engendró á Seba, y á
Dedán: é hijos de Dedán fueron Assurim, y Letusim, y Leummim.
\bibverse{4} E hijos de Midiam: Epha, y Epher, y Enech, y Abida, y
Eldaa. Todos estos fueron hijos de Cetura. \bibverse{5} Y Abraham dió
todo cuanto tenía á Isaac. \bibverse{6} Y á los hijos de sus concubinas
dió Abraham dones, y enviólos de junto Isaac su hijo, mientras él vivía,
hacia el oriente, á la tierra oriental. \bibverse{7} Y estos fueron los
días de vida que vivió Abraham: ciento setenta y cinco años.
\bibverse{8} Y exhaló el espíritu, y murió Abraham en buena vejez,
anciano y lleno de días, y fué unido á su pueblo. \footnote{\textbf{25:8}
  Gén 15,15; Job 5,26} \bibverse{9} Y sepultáronlo Isaac é Ismael sus
hijos en la cueva de Macpela, en la heredad de Ephrón, hijo de Zoar
Hetheo, que está enfrente de Mamre; \bibverse{10} Heredad que compró
Abraham de los hijos de Heth; allí fué Abraham sepultado, y Sara su
mujer. \bibverse{11} Y sucedió, después de muerto Abraham, que Dios
bendijo á Isaac su hijo: y habitó Isaac junto al pozo del Viviente que
me ve. \footnote{\textbf{25:11} Gén 24,62}

\hypertarget{los-descendientes-de-ismael}{%
\subsection{Los descendientes de
Ismael}\label{los-descendientes-de-ismael}}

\bibverse{12} Y estas son las generaciones de Ismael, hijo de Abraham,
que le parió Agar Egipcia, sierva de Sara: \footnote{\textbf{25:12} Gén
  21,13} \bibverse{13} Estos, pues, son los nombres de los hijos de
Ismael, por sus nombres, por sus linajes: El primogénito de Ismael,
Nabaioth; luego Cedar, y Abdeel, y Mibsam, \bibverse{14} Y Misma, y
Duma, y Massa, \bibverse{15} Hadad, y Tema, y Jetur, y Naphis, y Cedema.
\bibverse{16} Estos son los hijos de Ismael, y estos sus nombres, por
sus villas y por sus campamentos; doce príncipes por sus familias.
\footnote{\textbf{25:16} Gén 17,20} \bibverse{17} Y estos fueron los
años de la vida de Ismael, ciento treinta y siete años: y exhaló el
espíritu Ismael, y murió; y fué unido á su pueblo. \bibverse{18} Y
habitaron desde Havila hasta Shur, que está enfrente de Egipto viniendo
á Asiria; y murió en presencia de todos sus hermanos.

\hypertarget{el-nacimiento-de-esau-y-jacob}{%
\subsection{El nacimiento de Esau y
Jacob}\label{el-nacimiento-de-esau-y-jacob}}

\bibverse{19} Y estas son las generaciones de Isaac, hijo de Abraham.
Abraham engendró á Isaac: \bibverse{20} Y era Isaac de cuarenta años
cuando tomó por mujer á Rebeca, hija de Bethuel Arameo de Padan-aram,
hermana de Labán Arameo. \bibverse{21} Y oró Isaac á Jehová por su
mujer, que era estéril; y aceptólo Jehová, y concibió Rebeca su mujer.
\bibverse{22} Y los hijos se combatían dentro de ella; y dijo: Si es así
¿para qué vivo yo? Y fué á consultar á Jehová. \bibverse{23} Y
respondióle Jehová: Dos gentes hay en tu seno, y dos pueblos serán
divididos desde tus entrañas: y el un pueblo será más fuerte que el otro
pueblo, y el mayor servirá al menor. \footnote{\textbf{25:23} Gén 27,29;
  Mal 1,2; Rom 9,10-12}

\bibverse{24} Y como se cumplieron sus días para parir, he aquí mellizos
en su vientre. \bibverse{25} Y salió el primero rubio, y todo él velludo
como una pelliza; y llamaron su nombre Esaú. \bibverse{26} Y después
salió su hermano, trabada su mano al calcañar de Esaú: y fué llamado su
nombre Jacob. Y era Isaac de edad de sesenta años cuando ella los parió.

\bibverse{27} Y crecieron los niños, y Esaú fué diestro en la caza,
hombre del campo: Jacob empero era varón quieto, que habitaba en
tiendas. \bibverse{28} Y amó Isaac á Esaú, porque comía de su caza; mas
Rebeca amaba á Jacob.

\hypertarget{jacob-compra-la-primogenitura-de-esauxfa}{%
\subsection{Jacob compra la primogenitura de
Esaú}\label{jacob-compra-la-primogenitura-de-esauxfa}}

\bibverse{29} Y guisó Jacob un potaje; y volviendo Esaú del campo
cansado, \bibverse{30} Dijo á Jacob: Ruégote que me des á comer de eso
bermejo, pues estoy muy cansado. Por tanto fué llamado su nombre Edom.

\bibverse{31} Y Jacob respondió: Véndeme en este día tu primogenitura.

\bibverse{32} Entonces dijo Esaú: He aquí yo me voy á morir; ¿para qué,
pues, me servirá la primogenitura?

\bibverse{33} Y dijo Jacob: Júramelo en este día. Y él le juró, y vendió
á Jacob su primogenitura.

\bibverse{34} Entonces Jacob dió á Esaú pan y del guisado de las
lentejas; y él comió y bebió, y levantóse, y fuése. Así menospreció Esaú
la primogenitura.

\hypertarget{isaac-se-muda-a-gerar-cuando-hay-hambre}{%
\subsection{Isaac se muda a Gerar cuando hay
hambre}\label{isaac-se-muda-a-gerar-cuando-hay-hambre}}

\hypertarget{section-25}{%
\section{26}\label{section-25}}

\bibverse{1} Y HUBO hambre en la tierra, además de la primera hambre que
fué en los días de Abraham: y fuése Isaac á Abimelech rey de los
Filisteos, en Gerar. \footnote{\textbf{26:1} Gén 12,10; Gén 20,2}
\bibverse{2} Y apareciósele Jehová, y díjole: No desciendas á Egipto:
habita en la tierra que yo te diré; \bibverse{3} Habita en esta tierra,
y seré contigo, y te bendeciré; porque á ti y á tu simiente daré todas
estas tierras, y confirmaré el juramento que juré á Abraham tu padre:
\bibverse{4} Y multiplicaré tu simiente como las estrellas del cielo, y
daré á tu simiente todas estas tierras; y todas las gentes de la tierra
serán benditas en tu simiente: \footnote{\textbf{26:4} Gén 15,5; Gén
  12,3} \bibverse{5} Por cuanto oyó Abraham mi voz, y guardó mi
precepto, mis mandamientos, mis estatutos y mis leyes.

\bibverse{6} Habitó, pues, Isaac en Gerar. \bibverse{7} Y los hombres de
aquel lugar le preguntaron acerca de su mujer; y él respondió: Es mi
hermana; porque tuvo miedo de decir: Es mi mujer; que tal vez, dijo, los
hombres del lugar me matarían por causa de Rebeca; porque era de hermoso
aspecto. \bibverse{8} Y sucedió que, después que él estuvo allí muchos
días, Abimelech, rey de los Filisteos, mirando por una ventana, vió á
Isaac que jugaba con Rebeca su mujer. \bibverse{9} Y llamó Abimelech á
Isaac, y dijo: He aquí ella es de cierto tu mujer: ¿cómo, pues, dijiste:
Es mi hermana? E Isaac le respondió: Porque dije: Quizá moriré por causa
de ella.

\bibverse{10} Y Abimelech dijo: ¿Por qué nos has hecho esto? Por poco
hubiera dormido alguno del pueblo con tu mujer, y hubieras traído sobre
nosotros el pecado.

\bibverse{11} Entonces Abimelech mandó á todo el pueblo, diciendo: El
que tocare á este hombre ó á su mujer, de cierto morirá.

\hypertarget{la-creciente-riqueza-de-isaac-disputas-de-fuentes}{%
\subsection{La creciente riqueza de Isaac; Disputas de
fuentes;}\label{la-creciente-riqueza-de-isaac-disputas-de-fuentes}}

\bibverse{12} Y sembró Isaac en aquella tierra, y halló aquel año ciento
por uno: y bendíjole Jehová. \footnote{\textbf{26:12} Prov 10,22}
\bibverse{13} Y el varón se engrandeció, y fué adelantando y
engrandeciéndose, hasta hacerse muy poderoso: \bibverse{14} Y tuvo hato
de ovejas, y hato de vacas, y grande apero; y los Filisteos le tuvieron
envidia. \bibverse{15} Y todos los pozos que habían abierto los criados
de Abraham su padre en sus días, los Filisteos los habían cegado y
llenado de tierra. \bibverse{16} Y dijo Abimelech á Isaac: Apártate de
nosotros, porque mucho más poderoso que nosotros te has hecho.

\bibverse{17} E Isaac se fué de allí; y asentó sus tiendas en el valle
de Gerar, y habitó allí.

\bibverse{18} Y volvió á abrir Isaac los pozos de agua que habían
abierto en los días de Abraham su padre, y que los Filisteos habían
cegado, muerto Abraham; y llamólos por los nombres que su padre los
había llamado. \bibverse{19} Y los siervos de Isaac cavaron en el valle,
y hallaron allí un pozo de aguas vivas. \bibverse{20} Y los pastores de
Gerar riñeron con los pastores de Isaac, diciendo: El agua es nuestra:
por eso llamó el nombre del pozo Esek, porque habían altercado con él.
\bibverse{21} Y abrieron otro pozo, y también riñeron sobre él: y llamó
su nombre Sitnah. \bibverse{22} Y apartóse de allí, y abrió otro pozo, y
no riñeron sobre él: y llamó su nombre Rehoboth, y dijo: Porque ahora
nos ha hecho ensanchar Jehová, y fructificaremos en la tierra.

\bibverse{23} Y de allí subió á Beer-seba. \bibverse{24} Y apareciósele
Jehová aquella noche, y dijo: Yo soy el Dios de Abraham tu padre; no
temas, que yo soy contigo, y yo te bendeciré, y multiplicaré tu simiente
por amor de Abraham mi siervo.

\bibverse{25} Y edificó allí un altar, é invocó el nombre de Jehová, y
tendió allí su tienda: y abrieron allí los siervos de Isaac un pozo.
\footnote{\textbf{26:25} Gén 12,8}

\hypertarget{el-pacto-entre-isaac-y-abimelech-en-beerseba}{%
\subsection{El pacto entre Isaac y Abimelech en
Beerseba}\label{el-pacto-entre-isaac-y-abimelech-en-beerseba}}

\bibverse{26} Y Abimelech vino á él desde Gerar, y Ahuzzath, amigo suyo,
y Phicol, capitán de su ejército. \footnote{\textbf{26:26} Gén 21,22}
\bibverse{27} Y díjoles Isaac: ¿Por qué venís á mí, pues que me habéis
aborrecido, y me echasteis de entre vosotros?

\bibverse{28} Y ellos respondieron: Hemos visto que Jehová es contigo; y
dijimos: Haya ahora juramento entre nosotros, entre nosotros y ti, y
haremos alianza contigo: \bibverse{29} Que no nos hagas mal, como
nosotros no te hemos tocado, y como solamente te hemos hecho bien, y te
enviamos en paz: tú ahora, bendito de Jehová.

\bibverse{30} Entonces él les hizo banquete, y comieron y bebieron.
\bibverse{31} Y se levantaron de madrugada, y juraron el uno al otro; é
Isaac los despidió, y ellos se partieron de él en paz. \bibverse{32} Y
en aquel día sucedió que vinieron los criados de Isaac, y diéronle
nuevas acerca del pozo que habían abierto, y le dijeron: Agua hemos
hallado. \bibverse{33} Y llamólo Seba: por cuya causa el nombre de
aquella ciudad es Beer-seba hasta este día. \footnote{\textbf{26:33} Gén
  21,31}

\hypertarget{esauxfa-se-casa-con-dos-mujeres-hititas-en-contra-de-la-voluntad-de-sus-padres}{%
\subsection{Esaú se casa con dos mujeres hititas en contra de la
voluntad de sus
padres}\label{esauxfa-se-casa-con-dos-mujeres-hititas-en-contra-de-la-voluntad-de-sus-padres}}

\bibverse{34} Y cuando Esaú fué de cuarenta años, tomó por mujer á
Judith hija de Beeri Hetheo, y á Basemat hija de Elón Hetheo:
\footnote{\textbf{26:34} Gén 36,2-3} \bibverse{35} Y fueron amargura de
espíritu á Isaac y á Rebeca.

\hypertarget{isaac-se-prepara-para-bendecir-esau}{%
\subsection{Isaac se prepara para bendecir
Esau}\label{isaac-se-prepara-para-bendecir-esau}}

\hypertarget{section-26}{%
\section{27}\label{section-26}}

\bibverse{1} Y ACONTECIÓ que cuando hubo Isaac envejecido, y sus ojos se
ofuscaron quedando sin vista, llamó á Esaú, su hijo el mayor, y díjole:
Mi hijo. Y él respondió: Heme aquí.

\bibverse{2} Y él dijo: He aquí ya soy viejo, no sé el día de mi muerte:
\bibverse{3} Toma, pues, ahora tus armas, tu aljaba y tu arco, y sal al
campo, y cógeme caza; \bibverse{4} Y hazme un guisado, como yo gusto, y
tráemelo, y comeré; para que te bendiga mi alma antes que muera.
\footnote{\textbf{27:4} Heb 11,20}

\hypertarget{la-intervenciuxf3n-engauxf1osa-de-rebeca}{%
\subsection{La intervención engañosa de
Rebeca}\label{la-intervenciuxf3n-engauxf1osa-de-rebeca}}

\bibverse{5} Y Rebeca estaba oyendo, cuando hablaba Isaac á Esaú su
hijo: y fuése Esaú al campo para coger la caza que había de traer.
\bibverse{6} Entonces Rebeca habló á Jacob su hijo, diciendo: He aquí yo
he oído á tu padre que hablaba con Esaú tu hermano, diciendo:
\bibverse{7} Tráeme caza, y hazme un guisado, para que coma, y te
bendiga delante de Jehová antes que yo muera. \bibverse{8} Ahora pues,
hijo mío, obedece á mi voz en lo que te mando; \bibverse{9} Ve ahora al
ganado, y tráeme de allí dos buenos cabritos de las cabras, y haré de
ellos viandas para tu padre, como él gusta; \bibverse{10} Y tú las
llevarás á tu padre, y comerá, para que te bendiga antes de su muerte.

\bibverse{11} Y Jacob dijo á Rebeca su madre: He aquí, Esaú mi hermano
es hombre velloso, y yo lampiño: \bibverse{12} Quizá me tentará mi
padre, y me tendrá por burlador, y traeré sobre mí maldición y no
bendición.

\bibverse{13} Y su madre respondió: Hijo mío, sobre mí tu maldición:
solamente obedece á mi voz, y ve y tráemelos.

\bibverse{14} Entonces él fué, y tomó, y trájolos á su madre: y su madre
hizo guisados, como su padre gustaba. \bibverse{15} Y tomó Rebeca los
vestidos de Esaú su hijo mayor, los preciosos, que ella tenía en casa, y
vistió á Jacob su hijo menor: \bibverse{16} E hízole vestir sobre sus
manos, y sobre la cerviz donde no tenía vello, las pieles de los
cabritos de las cabras; \bibverse{17} Y entregó los guisados y el pan
que había aderezado, en mano de Jacob su hijo.

\hypertarget{jacob-recibe-la-bendiciuxf3n-del-primoguxe9nito}{%
\subsection{Jacob recibe la bendición del
primogénito}\label{jacob-recibe-la-bendiciuxf3n-del-primoguxe9nito}}

\bibverse{18} Y él fué á su padre, y dijo: Padre mío: y él respondió:
Heme aquí, ¿quién eres, hijo mío?

\bibverse{19} Y Jacob dijo á su padre: Yo soy Esaú tu primogénito; he
hecho como me dijiste: levántate ahora, y siéntate, y come de mi caza,
para que me bendiga tu alma.

\bibverse{20} Entonces Isaac dijo á su hijo: ¿Cómo es que la hallaste
tan presto, hijo mío? Y él respondió: Porque Jehová tu Dios hizo que se
encontrase delante de mí.

\bibverse{21} E Isaac dijo á Jacob: Acércate ahora, y te palparé, hijo
mío, por si eres mi hijo Esaú, ó no.

\bibverse{22} Y llegóse Jacob á su padre Isaac; y él le palpó, y dijo:
La voz es la voz de Jacob, mas las manos, las manos de Esaú.
\bibverse{23} Y no le conoció, porque sus manos eran vellosas como las
manos de Esaú: y le bendijo. \bibverse{24} Y dijo: ¿Eres tú mi hijo
Esaú? Y él respondió: Yo soy.

\bibverse{25} Y dijo: Acércamela, y comeré de la caza de mi hijo, para
que te bendiga mi alma; y él se la acercó, y comió: trájole también
vino, y bebió.

\bibverse{26} Y díjole Isaac su padre: Acércate ahora, y bésame, hijo
mío. \bibverse{27} Y él se llegó, y le besó; y olió Isaac el olor de sus
vestidos, y le bendijo, y dijo: Mira, el olor de mi hijo como el olor
del campo que Jehová ha bendecido: \bibverse{28} Dios, pues, te dé del
rocío del cielo, y de las grosuras de la tierra, y abundancia de trigo y
de mosto. \bibverse{29} Sírvante pueblos, y naciones se inclinen á ti:
sé señor de tus hermanos, e inclínense á ti los hijos de tu madre:
malditos los que te maldijeren, y benditos los que te bendijeren.
\footnote{\textbf{27:29} Gén 25,23; Gén 12,3}

\hypertarget{el-regreso-de-esauxfa-su-lamento-y-la-bendiciuxf3n-que-le-dio-su-padre}{%
\subsection{El regreso de Esaú, su lamento y la bendición que le dio su
padre}\label{el-regreso-de-esauxfa-su-lamento-y-la-bendiciuxf3n-que-le-dio-su-padre}}

\bibverse{30} Y aconteció, luego que hubo Isaac acabado de bendecir á
Jacob, y apenas había salido Jacob de delante de Isaac su padre, que
Esaú su hermano vino de su caza. \bibverse{31} E hizo él también
guisados, y trajo á su padre, y díjole: Levántese mi padre, y coma de la
caza de su hijo, para que me bendiga tu alma.

\bibverse{32} Entonces Isaac su padre le dijo: ¿Quién eres tú? Y él
dijo: Yo soy tu hijo, tu primogénito, Esaú.

\bibverse{33} Y estremecióse Isaac con grande estremecimiento, y dijo:
¿Quién es el que vino aquí, que cogió caza, y me trajo, y comí de todo
antes que vinieses? Yo le bendije, y será bendito.

\bibverse{34} Como Esaú oyó las palabras de su padre, clamó con una muy
grande y muy amarga exclamación, y le dijo: Bendíceme también á mí,
padre mío.

\bibverse{35} Y él dijo: Vino tu hermano con engaño, y tomó tu
bendición.

\bibverse{36} Y él respondió: Bien llamaron su nombre Jacob, que ya me
ha engañado dos veces; alzóse con mi primogenitura, y he aquí ahora ha
tomado mi bendición. Y dijo: ¿No has guardado bendición para mí?
\footnote{\textbf{27:36} Gén 25,26; Gén 25,33}

\bibverse{37} Isaac respondió y dijo á Esaú: He aquí yo le he puesto por
señor tuyo, y le he dado por siervos á todos sus hermanos: de trigo y de
vino le he provisto: ¿qué, pues, te haré á ti ahora, hijo mío?

\bibverse{38} Y Esaú respondió á su padre: ¿No tienes más que una sola
bendición, padre mío? bendíceme también á mí, padre mío. Y alzó Esaú su
voz, y lloró.

\bibverse{39} Entonces Isaac su padre habló y díjole: He aquí será tu
habitación en grosuras de la tierra, y del rocío de los cielos de
arriba; \bibverse{40} Y por tu espada vivirás, y á tu hermano servirás:
y sucederá cuando te enseñorees, que descargarás su yugo de tu cerviz.

\hypertarget{esauxfa-busca-matar-a-su-hermano}{%
\subsection{Esaú busca matar a su
hermano}\label{esauxfa-busca-matar-a-su-hermano}}

\bibverse{41} Y aborreció Esaú á Jacob por la bendición con que le había
bendecido, y dijo en su corazón: Llegarán los días del luto de mi padre,
y yo mataré á Jacob mi hermano.

\bibverse{42} Y fueron dichas á Rebeca las palabras de Esaú su hijo
mayor: y ella envió y llamó á Jacob su hijo menor, y díjole: He aquí,
Esaú tu hermano se consuela acerca de ti con la idea de matarte.
\bibverse{43} Ahora pues, hijo mío, obedece á mi voz; levántate, y
húyete á Labán mi hermano, á Harán; \footnote{\textbf{27:43} Gén 24,10}
\bibverse{44} Y mora con él algunos días, hasta que el enojo de tu
hermano se mitigue; \bibverse{45} Hasta que se aplaque la ira de tu
hermano contra ti, y se olvide de lo que le has hecho: yo enviaré
entonces, y te traeré de allá: ¿por qué seré privada de vosotros ambos
en un día?

\bibverse{46} Y dijo Rebeca á Isaac: Fastidio tengo de mi vida, á causa
de las hijas de Heth. Si Jacob toma mujer de las hijas de Heth, como
éstas, de las hijas de esta tierra, ¿para qué quiero la vida?

\hypertarget{jacob-huye-a-padan-aram}{%
\subsection{Jacob huye a Padan-aram}\label{jacob-huye-a-padan-aram}}

\hypertarget{section-27}{%
\section{28}\label{section-27}}

\bibverse{1} ENTONCES Isaac llamó á Jacob, y bendíjolo, y mandóle
diciendo: No tomes mujer de las hijas de Canaán. \footnote{\textbf{28:1}
  Gén 24,3} \bibverse{2} Levántate, ve á Padan-aram, á casa de Bethuel,
padre de tu madre, y toma allí mujer de las hijas de Labán, hermano de
tu madre. \footnote{\textbf{28:2} Gén 22,23; Gén 24,29} \bibverse{3} Y
el Dios omnipotente te bendiga, y te haga fructificar, y te multiplique,
hasta venir á ser congregación de pueblos; \bibverse{4} Y te dé la
bendición de Abraham, y á tu simiente contigo, para que heredes la
tierra de tus peregrinaciones, que Dios dió á Abraham. \footnote{\textbf{28:4}
  Gén 12,2}

\bibverse{5} Así envió Isaac á Jacob, el cual fué á Padan-aram, á Labán,
hijo de Bethuel Arameo, hermano de Rebeca, madre de Jacob y de Esaú.

\hypertarget{el-nuevo-matrimonio-de-esauxfa-con-una-hija-de-ismael}{%
\subsection{El nuevo matrimonio de Esaú con una hija de
Ismael}\label{el-nuevo-matrimonio-de-esauxfa-con-una-hija-de-ismael}}

\bibverse{6} Y vió Esaú cómo Isaac había bendecido á Jacob, y le había
enviado á Padan-aram, para tomar para sí mujer de allí; y que cuando le
bendijo, le había mandado, diciendo: No tomarás mujer de las hijas de
Canaán; \bibverse{7} Y que Jacob había obedecido á su padre y á su
madre, y se había ido á Padan-aram. \bibverse{8} Vió asimismo Esaú que
las hijas de Canaán parecían mal á Isaac su padre; \bibverse{9} Y fuése
Esaú á Ismael, y tomó para sí por mujer á Mahaleth, hija de Ismael, hijo
de Abraham, hermana de Nabaioth, además de sus otras mujeres.

\hypertarget{el-sueuxf1o-de-jacob-en-betel-de-la-escalera-al-cielo}{%
\subsection{El sueño de Jacob en Betel de la escalera al
cielo}\label{el-sueuxf1o-de-jacob-en-betel-de-la-escalera-al-cielo}}

\bibverse{10} Y salió Jacob de Beer-seba, y fué á Harán; \bibverse{11} Y
encontró con un lugar, y durmió allí, porque ya el sol se había puesto:
y tomó de las piedras de aquel paraje y puso á su cabecera, y acostóse
en aquel lugar. \bibverse{12} Y soñó, y he aquí una escala que estaba
apoyada en tierra, y su cabeza tocaba en el cielo: y he aquí ángeles de
Dios que subían y descendían por ella. \footnote{\textbf{28:12} Juan
  1,51} \bibverse{13} Y he aquí, Jehová estaba en lo alto de ella, el
cual dijo: Yo soy Jehová, el Dios de Abraham tu padre, y el Dios de
Isaac: la tierra en que estás acostado te la daré á ti y á tu simiente.
\footnote{\textbf{28:13} Gén 12,7} \bibverse{14} Y será tu simiente como
el polvo de la tierra, y te extenderás al occidente, y al oriente, y al
aquilón, y al mediodía; y todas las familias de la tierra serán benditas
en ti y en tu simiente. \footnote{\textbf{28:14} Gén 13,16; Gén 12,3}
\bibverse{15} Y he aquí, yo soy contigo, y te guardaré por donde quiera
que fueres, y te volveré á esta tierra; porque no te dejaré hasta tanto
que haya hecho lo que te he dicho.

\hypertarget{jacob-consagra-una-piedra-conmemorativa-como-el-comienzo-de-una-casa-de-dios-en-betel}{%
\subsection{Jacob consagra una piedra conmemorativa como el comienzo de
una casa de Dios en
Betel}\label{jacob-consagra-una-piedra-conmemorativa-como-el-comienzo-de-una-casa-de-dios-en-betel}}

\bibverse{16} Y despertó Jacob de su sueño, y dijo: Ciertamente Jehová
está en este lugar, y yo no lo sabía. \bibverse{17} Y tuvo miedo, y
dijo: ¡Cuán terrible es este lugar! No es otra cosa que casa de Dios, y
puerta del cielo.

\bibverse{18} Y levantóse Jacob de mañana, y tomó la piedra que había
puesto de cabecera, y alzóla por título, y derramó aceite encima de
ella. \bibverse{19} Y llamó el nombre de aquel lugar Beth-el, bien que
Luz era el nombre de la ciudad primero. \footnote{\textbf{28:19} Gén
  35,14-15} \bibverse{20} E hizo Jacob voto, diciendo: Si fuere Dios
conmigo, y me guardare en este viaje que voy, y me diere pan para comer
y vestido para vestir, \bibverse{21} Y si tornare en paz á casa de mi
padre, Jehová será mi Dios, \bibverse{22} Y esta piedra que he puesto
por título, será casa de Dios: y de todo lo que me dieres, el diezmo lo
he de apartar para ti.

\hypertarget{jacob-al-pozo-de-haran}{%
\subsection{Jacob al pozo de Haran}\label{jacob-al-pozo-de-haran}}

\hypertarget{section-28}{%
\section{29}\label{section-28}}

\bibverse{1} Y SIGUIÓ Jacob su camino, y fué á la tierra de los
orientales. \bibverse{2} Y miró, y vió un pozo en el campo: y he aquí
tres rebaños de ovejas que yacían cerca de él; porque de aquel pozo
abrevaban los ganados: y había una gran piedra sobre la boca del pozo.
\bibverse{3} Y juntábanse allí todos los rebaños; y revolvían la piedra
de sobre la boca del pozo, y abrevaban las ovejas; y volvían la piedra
sobre la boca del pozo á su lugar. \bibverse{4} Y díjoles Jacob:
Hermanos míos, ¿de dónde sois? Y ellos respondieron: De Harán somos.

\bibverse{5} Y él les dijo: ¿Conocéis á Labán, hijo de Nachôr? Y ellos
dijeron: Sí, le conocemos.

\bibverse{6} Y él les dijo: ¿Tiene paz? Y ellos dijeron: Paz; y he aquí
Rachêl su hija viene con el ganado.

\bibverse{7} Y él dijo: He aquí el día es aún grande; no es tiempo
todavía de recoger el ganado; abrevad las ovejas, é id á apacentarlas.

\bibverse{8} Y ellos respondieron: No podemos, hasta que se junten todos
los ganados, y remuevan la piedra de sobre la boca del pozo, para que
abrevemos las ovejas.

\hypertarget{el-saludo-de-jacob-con-rachuxeal-y-su-admisiuxf3n-a-labuxe1n}{%
\subsection{El saludo de Jacob con Rachêl y su admisión a
Labán}\label{el-saludo-de-jacob-con-rachuxeal-y-su-admisiuxf3n-a-labuxe1n}}

\bibverse{9} Estando aún él hablando con ellos, Rachêl vino con el
ganado de su padre, porque ella era la pastora. \bibverse{10} Y sucedió
que, como Jacob vió á Rachêl, hija de Labán hermano de su madre, y á las
ovejas de Labán el hermano de su madre, llegóse Jacob, y removió la
piedra de sobre la boca del pozo, y abrevó el ganado de Labán hermano de
su madre. \bibverse{11} Y Jacob besó á Rachêl, y alzó su voz, y lloró.
\bibverse{12} Y Jacob dijo á Rachêl como él era hermano de su padre, y
como era hijo de Rebeca: y ella corrió, y dió las nuevas á su padre.

\bibverse{13} Y así que oyó Labán las nuevas de Jacob, hijo de su
hermana, corrió á recibirlo, y abrazólo, y besólo, y trájole á su casa:
y él contó á Labán todas estas cosas. \bibverse{14} Y Labán le dijo:
Ciertamente hueso mío y carne mía eres. Y estuvo con él el tiempo de un
mes.

\hypertarget{jacob-entra-en-servicio-con-labuxe1n}{%
\subsection{Jacob entra en servicio con
Labán}\label{jacob-entra-en-servicio-con-labuxe1n}}

\bibverse{15} Entonces dijo Labán á Jacob: ¿Por ser tú mi hermano, me
has de servir de balde? declárame qué será tu salario.

\bibverse{16} Y Labán tenía dos hijas: el nombre de la mayor era Lea, y
el nombre de la menor, Rachêl. \bibverse{17} Y los ojos de Lea eran
tiernos, pero Rachêl era de lindo semblante y de hermoso parecer.
\bibverse{18} Y Jacob amó á Rachêl, y dijo: Yo te serviré siete años por
Rachêl tu hija menor.

\bibverse{19} Y Labán respondió: Mejor es que te la dé á ti, que no que
la dé á otro hombre: estáte conmigo.

\bibverse{20} Así sirvió Jacob por Rachêl siete años: y pareciéronle
como pocos días, porque la amaba.

\bibverse{21} Y dijo Jacob á Labán: Dame mi mujer, porque mi tiempo es
cumplido, para que cohabite con ella.

\bibverse{22} Entonces Labán juntó á todos los varones de aquel lugar, é
hizo banquete. \bibverse{23} Y sucedió que á la noche tomó á Lea su
hija, y se la trajo: y él entró á ella. \bibverse{24} Y dió Labán su
sierva Zilpa á su hija Lea por criada. \bibverse{25} Y venida la mañana,
he aquí que era Lea: y él dijo á Labán: ¿Qué es esto que me has hecho?
¿no te he servido por Rachêl? ¿por qué, pues, me has engañado?

\bibverse{26} Y Labán respondió: No se hace así en nuestro lugar, que se
dé la menor antes de la mayor. \bibverse{27} Cumple la semana de ésta, y
se te dará también la otra, por el servicio que hicieres conmigo otros
siete años.

\bibverse{28} E hizo Jacob así, y cumplió la semana de aquélla: y él le
dió á Rachêl su hija por mujer. \bibverse{29} Y dió Labán á Rachêl su
hija por criada á su sierva Bilha. \bibverse{30} Y entró también á
Rachêl: y amóla también más que á Lea: y sirvió con él aún otros siete
años. \footnote{\textbf{29:30} Lev 18,18}

\hypertarget{los-primer-cuatro-hijos-de-lea}{%
\subsection{Los primer cuatro hijos de
Lea}\label{los-primer-cuatro-hijos-de-lea}}

\bibverse{31} Y vió Jehová que Lea era aborrecida, y abrió su matriz:
pero Rachêl era estéril. \bibverse{32} Y concibió Lea, y parió un hijo,
y llamó su nombre Rubén, porque dijo: Ya que ha mirado Jehová mi
aflicción; ahora por tanto me amará mi marido. \bibverse{33} Y concibió
otra vez, y parió un hijo, y dijo: Por cuanto oyó Jehová que yo era
aborrecida, me ha dado también éste. Y llamó su nombre Simeón.
\bibverse{34} Y concibió otra vez, y parió un hijo, y dijo: Ahora esta
vez se unirá mi marido conmigo, porque le he parido tres hijos: por
tanto, llamó su nombre Leví. \bibverse{35} Y concibió otra vez, y parió
un hijo, y dijo: Esta vez alabaré á Jehová: por esto llamó su nombre
Judá: y dejó de parir.

\hypertarget{los-dos-hijos-de-bilha-la-sierva-de-rachuxeal}{%
\subsection{Los dos hijos de Bilha, la sierva de
Rachêl}\label{los-dos-hijos-de-bilha-la-sierva-de-rachuxeal}}

\hypertarget{section-29}{%
\section{30}\label{section-29}}

\bibverse{1} Y VIENDO Rachêl que no daba hijos á Jacob, tuvo envidia de
su hermana, y decía á Jacob: Dame hijos, ó si no, me muero.

\bibverse{2} Y Jacob se enojaba contra Rachêl, y decía: ¿Soy yo en lugar
de Dios, que te impidió el fruto de tu vientre?

\bibverse{3} Y ella dijo: He aquí mi sierva Bilha; entra á ella, y
parirá sobre mis rodillas, y yo también tendré hijos de ella.
\footnote{\textbf{30:3} Gén 16,2} \bibverse{4} Así le dió á Bilha su
sierva por mujer; y Jacob entró á ella. \bibverse{5} Y concibió Bilha, y
parió á Jacob un hijo. \bibverse{6} Y dijo Rachêl: Juzgóme Dios, y
también oyó mi voz, y dióme un hijo. Por tanto llamó su nombre Dan.
\bibverse{7} Y concibió otra vez Bilha, la sierva de Rachêl, y parió el
hijo segundo á Jacob. \bibverse{8} Y dijo Rachêl: Con luchas de Dios he
contendido con mi hermana, y he vencido. Y llamó su nombre Nephtalí.

\hypertarget{los-dos-hijos-de-silpa-la-sierva-de-lea}{%
\subsection{Los dos hijos de Silpa, la sierva de
Lea}\label{los-dos-hijos-de-silpa-la-sierva-de-lea}}

\bibverse{9} Y viendo Lea que había dejado de parir, tomó á Zilpa su
sierva, y dióla á Jacob por mujer. \bibverse{10} Y Zilpa, sierva de Lea,
parió á Jacob un hijo. \bibverse{11} Y dijo Lea: Vino la ventura. Y
llamó su nombre Gad. \bibverse{12} Y Zilpa, la sierva de Lea, parió otro
hijo á Jacob. \bibverse{13} Y dijo Lea: Para dicha mía; porque las
mujeres me dirán dichosa: y llamó su nombre Aser.

\hypertarget{los-ultimos-niuxf1os-de-lea}{%
\subsection{Los ultimos niños de
Lea}\label{los-ultimos-niuxf1os-de-lea}}

\bibverse{14} Y fué Rubén en tiempo de la siega de los trigos, y halló
mandrágoras en el campo, y trájolas á Lea su madre: y dijo Rachêl á Lea:
Ruégote que me des de las mandrágoras de tu hijo.

\bibverse{15} Y ella respondió: ¿Es poco que hayas tomado mi marido,
sino que también te has de llevar las mandrágoras de mi hijo? y dijo
Rachêl: Pues dormirá contigo esta noche por las mandrágoras de tu hijo.

\bibverse{16} Y cuando Jacob volvía del campo á la tarde, salió Lea á
él, y le dijo: A mí has de entrar, porque á la verdad te he alquilado
por las mandrágoras de mi hijo. Y durmió con ella aquella noche.

\bibverse{17} Y oyó Dios á Lea: y concibió, y parió á Jacob el quinto
hijo. \bibverse{18} Y dijo Lea: Dios me ha dado mi recompensa, por
cuanto dí mi sierva á mi marido: por eso llamó su nombre Issachâr.
\bibverse{19} Y concibió Lea otra vez, y parió el sexto hijo á Jacob.
\bibverse{20} Y dijo Lea: Dios me ha dado una buena dote: ahora morará
conmigo mi marido, porque le he parido seis hijos: y llamó su nombre
Zabulón. \bibverse{21} Y después parió una hija, y llamó su nombre Dina.

\hypertarget{rachuxeal-cresce-madre-de-josuxe9}{%
\subsection{Rachêl cresce madre de
José}\label{rachuxeal-cresce-madre-de-josuxe9}}

\bibverse{22} Y acordóse Dios de Rachêl, y oyóla Dios, y abrió su
matriz. \footnote{\textbf{30:22} 1Sam 1,19} \bibverse{23} Y concibió, y
parió un hijo: y dijo: Quitado ha Dios mi afrenta: \footnote{\textbf{30:23}
  Is 4,1; Luc 1,25} \bibverse{24} Y llamó su nombre José, diciendo:
Añádame Jehová otro hijo.

\hypertarget{el-nuevo-pacto-de-servicio-de-jacob-con-labuxe1n}{%
\subsection{El nuevo pacto de servicio de Jacob con
Labán}\label{el-nuevo-pacto-de-servicio-de-jacob-con-labuxe1n}}

\bibverse{25} Y aconteció, cuando Rachêl hubo parido á José, que Jacob
dijo á Labán: Envíame, é iré á mi lugar, y á mi tierra. \bibverse{26}
Dame mis mujeres y mis hijos, por las cuales he servido contigo, y
déjame ir; pues tú sabes los servicios que te he hecho. \footnote{\textbf{30:26}
  Gén 29,20; Gén 29,30}

\bibverse{27} Y Labán le respondió: Halle yo ahora gracia en tus ojos, y
quédate; experimentado he que Jehová me ha bendecido por tu causa.
\footnote{\textbf{30:27} Gén 39,5} \bibverse{28} Y dijo: Señálame tu
salario, que yo lo daré.

\bibverse{29} Y él respondió: Tú sabes cómo te he servido, y cómo ha
estado tu ganado conmigo; \bibverse{30} Porque poco tenías antes de mi
venida, y ha crecido en gran número; y Jehová te ha bendecido con mi
llegada: y ahora ¿cuándo tengo de hacer yo también por mi propia casa?

\bibverse{31} Y él dijo: ¿Qué te daré? Y respondió Jacob: No me des
nada: si hicieres por mí esto, volveré á apacentar tus ovejas.

\bibverse{32} Yo pasaré hoy por todas tus ovejas, poniendo aparte todas
las reses manchadas y de color vario, y todas las reses de color oscuro
entre las ovejas, y las manchadas y de color vario entre las cabras; y
esto será mi salario. \bibverse{33} Así responderá por mí mi justicia
mañana, cuando me viniere mi salario delante de ti: toda la que no fuere
pintada ni manchada en las cabras y de color oscuro en las ovejas mías,
se me ha de tener por de hurto.

\bibverse{34} Y dijo Labán: Mira, ojalá fuese como tú dices.

\hypertarget{jacob-obtuvo-una-gran-propiedad-de-ganado-a-travuxe9s-de-la-astucia}{%
\subsection{Jacob obtuvo una gran propiedad de ganado a través de la
astucia}\label{jacob-obtuvo-una-gran-propiedad-de-ganado-a-travuxe9s-de-la-astucia}}

\bibverse{35} Y apartó aquel día los machos de cabrío rayados y
manchados; y todas las cabras manchadas y de color vario, y toda res que
tenía en sí algo de blanco, y todas las de color oscuro entre las
ovejas, y púsolas en manos de sus hijos; \bibverse{36} Y puso tres días
de camino entre sí y Jacob: y Jacob apacentaba las otras ovejas de
Labán.

\bibverse{37} Y tomóse Jacob varas de álamo verdes, y de avellano, y de
castaño, y descortezó en ellas mondaduras blancas, descubriendo así lo
blanco de las varas. \bibverse{38} Y puso las varas que había mondado en
las pilas, delante del ganado, en los abrevaderos del agua donde venían
á beber las ovejas, las cuales se recalentaban viniendo á beber.
\bibverse{39} Y concebían las ovejas delante de las varas, y parían
borregos listados, pintados y salpicados de diversos colores.
\bibverse{40} Y apartaba Jacob los corderos, y poníalos con su rebaño,
los listados, y todo lo que era oscuro en el hato de Labán. Y ponía su
hato aparte, y no lo ponía con las ovejas de Labán. \bibverse{41} Y
sucedía que cuantas veces se recalentaban las tempranas, Jacob ponía las
varas delante de las ovejas en las pilas, para que concibiesen á la
vista de las varas. \bibverse{42} Y cuando venían las ovejas tardías, no
las ponía: así eran las tardías para Labán, y las tempranas para Jacob.
\bibverse{43} Y acreció el varón muy mucho, y tuvo muchas ovejas, y
siervas y siervos, y camellos y asnos. \footnote{\textbf{30:43} Gén
  12,16}

\hypertarget{las-razones-de-la-fuga-de-jacob}{%
\subsection{Las razones de la fuga de
Jacob}\label{las-razones-de-la-fuga-de-jacob}}

\hypertarget{section-30}{%
\section{31}\label{section-30}}

\bibverse{1} Y OÍA él las palabras de los hijos de Labán, que decían:
Jacob ha tomado todo lo que era de nuestro padre; y de lo que era de
nuestro padre ha adquirido toda esta grandeza. \footnote{\textbf{31:1}
  Gén 30,35} \bibverse{2} Miraba también Jacob el semblante de Labán, y
veía que no era para con él como ayer y antes de ayer. \bibverse{3}
También Jehová dijo á Jacob: Vuélvete á la tierra de tus padres, y á tu
parentela; que yo seré contigo. \footnote{\textbf{31:3} Gén 28,15}

\hypertarget{la-consulta-de-jacob-con-sus-esposas}{%
\subsection{La consulta de Jacob con sus
esposas}\label{la-consulta-de-jacob-con-sus-esposas}}

\bibverse{4} Y envió Jacob, y llamó á Rachêl y á Lea al campo á sus
ovejas, \bibverse{5} Y díjoles: Veo que el semblante de vuestro padre no
es para conmigo como ayer y antes de ayer: mas el Dios de mi padre ha
sido conmigo. \bibverse{6} Y vosotras sabéis que con todas mis fuerzas
he servido á vuestro padre: \bibverse{7} Y vuestro padre me ha engañado,
y me ha mudado el salario diez veces: pero Dios no le ha permitido que
me hiciese mal. \bibverse{8} Si él decía así: Los pintados serán tu
salario; entonces todas las ovejas parían pintados: y si decía así: Los
listados serán tu salario; entonces todas las ovejas parían listados.
\footnote{\textbf{31:8} Gén 30,32; Gén 30,39} \bibverse{9} Así quitó
Dios el ganado de vuestro padre, y diómelo á mí. \bibverse{10} Y sucedió
que al tiempo que las ovejas se recalentaban, alcé yo mis ojos y vi en
sueños, y he aquí los machos que cubrían á las hembras eran listados,
pintados y abigarrados. \bibverse{11} Y díjome el ángel de Dios en
sueños: Jacob. Y yo dije: Heme aquí. \bibverse{12} Y él dijo: Alza ahora
tus ojos, y verás todos los machos que cubren á las ovejas listados,
pintados y abigarrados; porque yo he visto todo lo que Labán te ha
hecho. \bibverse{13} Yo soy el Dios de Beth-el, donde tú ungiste el
título, y donde me hiciste un voto. Levántate ahora, y sal de esta
tierra, y vuélvete á la tierra de tu naturaleza.

\bibverse{14} Y respondió Rachêl y Lea, y dijéronle: ¿Tenemos ya parte
ni heredad en la casa de nuestro padre? \bibverse{15} ¿No nos tiene ya
como por extrañas, pues que nos vendió, y aun se ha comido del todo
nuestro precio? \footnote{\textbf{31:15} Gén 29,18; Gén 29,27}
\bibverse{16} Porque toda la riqueza que Dios ha quitado á nuestro
padre, nuestra es y de nuestros hijos: ahora pues, haz todo lo que Dios
te ha dicho.

\hypertarget{la-fuga-de-jacob-y-la-persecuciuxf3n-de-labuxe1n}{%
\subsection{La fuga de Jacob y la persecución de
Labán}\label{la-fuga-de-jacob-y-la-persecuciuxf3n-de-labuxe1n}}

\bibverse{17} Entonces se levantó Jacob, y subió sus hijos y sus mujeres
sobre los camellos. \bibverse{18} Y puso en camino todo su ganado, y
toda su hacienda que había adquirido, el ganado de su ganancia que había
obtenido en Padan-aram, para volverse á Isaac su padre en la tierra de
Canaán. \bibverse{19} Y Labán había ido á trasquilar sus ovejas: y
Rachêl hurtó los ídolos de su padre.

\bibverse{20} Y recató Jacob el corazón de Labán Arameo, en no hacerle
saber que se huía. \bibverse{21} Huyó, pues, con todo lo que tenía; y
levantóse, y pasó el río, y puso su rostro al monte de Galaad.

\bibverse{22} Y fué dicho á Labán al tercero día como Jacob se había
huído. \bibverse{23} Entonces tomó á sus hermanos consigo, y fué tras él
camino de siete días, y alcanzóle en el monte de Galaad. \bibverse{24} Y
vino Dios á Labán Arameo en sueños aquella noche, y le dijo: Guárdate
que no hables á Jacob descomedidamente. \footnote{\textbf{31:24} Gén
  20,3; Prov 16,7}

\hypertarget{discurso-de-castigo-de-labuxe1n-y-registro-de-la-casa}{%
\subsection{Discurso de castigo de Labán y registro de la
casa}\label{discurso-de-castigo-de-labuxe1n-y-registro-de-la-casa}}

\bibverse{25} Alcanzó pues Labán á Jacob, y éste había fijado su tienda
en el monte: y Labán plantóla con sus hermanos en el monte de Galaad.
\bibverse{26} Y dijo Labán á Jacob: ¿Qué has hecho, que me hurtaste el
corazón, y has traído á mis hijas como prisioneras de guerra?
\bibverse{27} ¿Por qué te escondiste para huir, y me hurtaste, y no me
diste noticia, para que yo te enviara con alegría y con cantares, con
tamborín y vihuela? \bibverse{28} Que aun no me dejaste besar mis hijos
y mis hijas. Ahora locamente has hecho. \bibverse{29} Poder hay en mi
mano para haceros mal: mas el Dios de vuestro padre me habló anoche
diciendo: Guárdate que no hables á Jacob descomedidamente. \bibverse{30}
Y ya que te ibas, porque tenías deseo de la casa de tu padre, ¿por qué
me hurtaste mis dioses?

\bibverse{31} Y Jacob respondió, y dijo á Labán: Porque tuve miedo; pues
dije, que quizá me quitarías por fuerza tus hijas. \bibverse{32} En
quien hallares tus dioses, no viva: delante de nuestros hermanos
reconoce lo que yo tuviere tuyo, y llévatelo. Jacob no sabía que Rachêl
los había hurtado.

\bibverse{33} Y entró Labán en la tienda de Jacob, y en la tienda de
Lea, y en la tienda de las dos siervas, y no los halló, y salió de la
tienda de Lea, y vino á la tienda de Rachêl. \bibverse{34} Y tomó Rachêl
los ídolos, y púsolos en una albarda de un camello, y sentóse sobre
ellos: y tentó Labán toda la tienda, y no los halló. \bibverse{35} Y
ella dijo á su padre: No se enoje mi señor, porque no me puedo levantar
delante de ti; pues estoy con la costumbre de las mujeres. Y él buscó,
pero no halló los ídolos.

\hypertarget{discurso-de-acusaciuxf3n-de-jacob}{%
\subsection{Discurso de acusación de
Jacob}\label{discurso-de-acusaciuxf3n-de-jacob}}

\bibverse{36} Entonces Jacob se enojó, y regañó con Labán; y respondió
Jacob y dijo á Labán: ¿Qué prevaricación es la mía? ¿cuál es mi pecado,
que con tanto ardor has venido en seguimiento mío? \bibverse{37} Pues
que has tentado todos mis muebles, ¿qué has hallado de todas las alhajas
de tu casa? Ponlo aquí delante de mis hermanos y tuyos, y juzguen entre
nosotros ambos.

\bibverse{38} Estos veinte años he estado contigo: tus ovejas y tus
cabras nunca abortaron, ni yo comí carnero de tus ovejas. \bibverse{39}
Nunca te traje lo arrebatado por las fieras; yo pagaba el daño; lo
hurtado así de día como de noche, de mi mano lo requerías. \footnote{\textbf{31:39}
  Éxod 22,11-12} \bibverse{40} De día me consumía el calor, y de noche
la helada, y el sueño se huía de mis ojos. \bibverse{41} Así he estado
veinte años en tu casa: catorce años te serví por tus dos hijas, y seis
años por tu ganado; y has mudado mi salario diez veces. \bibverse{42} Si
el Dios de mi padre, el Dios de Abraham, y el temor de Isaac, no fuera
conmigo, de cierto me enviarías ahora vacío: vió Dios mi aflicción y el
trabajo de mis manos, y reprendióte anoche. \footnote{\textbf{31:42} Gén
  31,54; Gén 31,24}

\hypertarget{la-respuesta-de-labuxe1n-el-tratado-de-paz-entre-uxe9l-y-jacob}{%
\subsection{La respuesta de Labán; el tratado de paz entre él y
Jacob}\label{la-respuesta-de-labuxe1n-el-tratado-de-paz-entre-uxe9l-y-jacob}}

\bibverse{43} Y respondió Labán, y dijo á Jacob: Las hijas son hijas
mías, y los hijos, hijos míos son, y las ovejas son mis ovejas, y todo
lo que tú ves es mío: ¿y qué puedo yo hacer hoy á estas mis hijas, ó á
sus hijos que ellas han parido? \bibverse{44} Ven pues ahora, y hagamos
alianza yo y tú; y sea en testimonio entre mí y entre ti.

\bibverse{45} Entonces Jacob tomó una piedra, y levantóla por título.
\bibverse{46} Y dijo Jacob á sus hermanos: Coged piedras. Y tomaron
piedras é hicieron un majano; y comieron allí sobre aquel majano.
\bibverse{47} Y llamólo Labán Jegar Sahadutha: y lo llamó Jacob Galaad.
\bibverse{48} Porque Labán dijo: Este majano es testigo hoy entre mí y
entre ti; por eso fué llamado su nombre Galaad; \footnote{\textbf{31:48}
  Jos 22,27; Jos 24,27} \bibverse{49} Y Mizpa, por cuanto dijo: Atalaye
Jehová entre mí y entre ti, cuando nos apartáremos el uno del otro.
\bibverse{50} Si afligieres mis hijas, ó si tomares otras mujeres además
de mis hijas, nadie está con nosotros; mira, Dios es testigo entre mí y
entre ti. \bibverse{51} Dijo más Labán á Jacob: He aquí este majano, y
he aquí este título, que he erigido entre mí y ti. \bibverse{52} Testigo
sea este majano, y testigo sea este título, que ni yo pasaré contra ti
este majano, ni tú pasarás contra mí este majano ni este título, para
mal. \bibverse{53} El Dios de Abraham, y el Dios de Nachôr juzgue entre
nosotros, el Dios de sus padres. Y Jacob juró por el temor de Isaac su
padre. \bibverse{54} Entonces Jacob inmoló víctimas en el monte, y llamó
á sus hermanos á comer pan: y comieron pan, y durmieron aquella noche en
el monte. \footnote{\textbf{31:54} Gén 31,42} \bibverse{55} Y levantóse
Labán de mañana, y besó sus hijos y sus hijas, y los bendijo; y
retrocedió y volvióse á su lugar.

\hypertarget{jacob-envuxeda-mensajeros-a-esauxfa}{%
\subsection{Jacob envía mensajeros a
Esaú}\label{jacob-envuxeda-mensajeros-a-esauxfa}}

\hypertarget{section-31}{%
\section{32}\label{section-31}}

\bibverse{1} Y JACOB se fué su camino, y saliéronle al encuentro ángeles
de Dios. \bibverse{2} Y dijo Jacob cuando los vió: El campo de Dios es
este: y llamó el nombre de aquel lugar Mahanaim.

\bibverse{3} Y envió Jacob mensajeros delante de sí á Esaú su hermano, á
la tierra de Seir, campo de Edom. \bibverse{4} Y mandóles diciendo: Así
diréis á mi señor Esaú: Así dice tu siervo Jacob: Con Labán he morado, y
detenídome hasta ahora; \footnote{\textbf{32:4} Gén 36,8} \bibverse{5} Y
tengo vacas, y asnos, y ovejas, y siervos y siervas; y envío á decirlo á
mi señor, por hallar gracia en tus ojos. \bibverse{6} Y los mensajeros
volvieron á Jacob, diciendo: Vinimos á tu hermano Esaú, y él también
viene á recibirte, y cuatrocientos hombres con él. \bibverse{7} Entonces
Jacob tuvo gran temor, y angustióse; y partió el pueblo que tenía
consigo, y las ovejas y las vacas y los camellos, en dos cuadrillas;
\bibverse{8} Y dijo: Si viniere Esaú á la una cuadrilla y la hiriere, la
otra cuadrilla escapará.

\hypertarget{la-oracion-de-jacob-por-la-ayuda-de-dios}{%
\subsection{La oracion de Jacob por la ayuda de
Dios}\label{la-oracion-de-jacob-por-la-ayuda-de-dios}}

\bibverse{9} Y dijo Jacob: Dios de mi padre Abraham, y Dios de mi padre
Isaac, Jehová, que me dijiste: Vuélvete á tu tierra y á tu parentela, y
yo te haré bien; \bibverse{10} Menor soy que todas las misericordias, y
que toda la verdad que has usado para con tu siervo; que con mi bordón
pasé este Jordán, y ahora estoy sobre dos cuadrillas. \bibverse{11}
Líbrame ahora de la mano de mi hermano, de la mano de Esaú, porque le
temo; no venga quizá, y me hiera la madre con los hijos. \footnote{\textbf{32:11}
  2Sam 7,18} \bibverse{12} Y tú has dicho: Yo te haré bien, y pondré tu
simiente como la arena del mar, que no se puede contar por la multitud.

\hypertarget{jacob-enviuxe1-regalos-a-esau}{%
\subsection{Jacob enviá regalos a
Esau}\label{jacob-enviuxe1-regalos-a-esau}}

\bibverse{13} Y durmió allí aquella noche, y tomó de lo que le vino á la
mano un presente para su hermano Esaú: \bibverse{14} Doscientas cabras y
veinte machos de cabrío, doscientas ovejas y veinte carneros,
\bibverse{15} Treinta camellas paridas, con sus hijos, cuarenta vacas y
diez novillos, veinte asnas y diez borricos. \bibverse{16} Y entrególo
en mano de sus siervos, cada manada de por sí; y dijo á sus siervos:
Pasad delante de mí, y poned espacio entre manada y manada.
\bibverse{17} Y mandó al primero, diciendo: Si Esaú mi hermano te
encontrare, y te preguntare, diciendo: ¿De quién eres? ¿y adónde vas? ¿y
para quién es esto que llevas delante de ti? \bibverse{18} Entonces
dirás: Presente es de tu siervo Jacob, que envía á mi señor Esaú; y he
aquí también él viene tras nosotros. \bibverse{19} Y mandó también al
segundo, y al tercero, y á todos los que iban tras aquellas manadas,
diciendo: Conforme á esto hablaréis á Esaú, cuando le hallareis.
\bibverse{20} Y diréis también: He aquí tu siervo Jacob viene tras
nosotros. Porque dijo: Apaciguaré su ira con el presente que va delante
de mí, y después veré su rostro: quizá le seré acepto.

\bibverse{21} Y pasó el presente delante de él; y él durmió aquella
noche en el campamento.

\hypertarget{jacob-luchando-con-dios-por-la-noche}{%
\subsection{Jacob luchando con Dios por la
noche}\label{jacob-luchando-con-dios-por-la-noche}}

\bibverse{22} Y levantóse aquella noche, y tomó sus dos mujeres, y sus
dos siervas, y sus once hijos, y pasó el vado de Jaboc. \bibverse{23}
Tomólos pues, y pasólos el arroyo, é hizo pasar lo que tenía.
\bibverse{24} Y quedóse Jacob solo, y luchó con él un varón hasta que
rayaba el alba. \bibverse{25} Y como vió que no podía con él, tocó en el
sitio del encaje de su muslo, y descoyuntóse el muslo de Jacob mientras
con él luchaba. \footnote{\textbf{32:25} Os 12,4-5} \bibverse{26} Y
dijo: Déjame, que raya el alba. Y él dijo: No te dejaré, si no me
bendices.

\bibverse{27} Y él le dijo: ¿Cuál es tu nombre? Y él respondió: Jacob.

\bibverse{28} Y él dijo: No se dirá más tu nombre Jacob, sino Israel:
porque has peleado con Dios y con los hombres, y has vencido.

\bibverse{29} Entonces Jacob le preguntó, y dijo: Declárame ahora tu
nombre. Y él respondió: ¿Por qué preguntas por mi nombre? Y bendíjolo
allí. \footnote{\textbf{32:29} Gén 35,10}

\bibverse{30} Y llamó Jacob el nombre de aquel lugar, Peniel: porque vi
á Dios cara á cara, y fué librada mi alma. \footnote{\textbf{32:30} Jue
  13,17-18} \bibverse{31} Y salióle el sol pasado que hubo á Peniel; y
cojeaba de su anca. \footnote{\textbf{32:31} Éxod 33,20} \bibverse{32}
Por esto no comen los hijos de Israel, hasta hoy día, del tendón que se
contrajo, el cual está en el encaje del muslo: porque tocó á Jacob este
sitio de su muslo en el tendón que se contrajo.

\hypertarget{la-reconciliaciuxf3n-de-jacob-con-esauxfa}{%
\subsection{La reconciliación de Jacob con
Esaú}\label{la-reconciliaciuxf3n-de-jacob-con-esauxfa}}

\hypertarget{section-32}{%
\section{33}\label{section-32}}

\bibverse{1} Y ALZANDO Jacob sus ojos miró, y he aquí venía Esaú, y los
cuatrocientos hombres con él: entonces repartió él los niños entre Lea y
Rachêl y las dos siervas. \bibverse{2} Y puso las siervas y sus niños
delante; luego á Lea y á sus niños; y á Rachêl y á José los postreros.
\bibverse{3} Y él pasó delante de ellos, é inclinóse á tierra siete
veces, hasta que llegó á su hermano.

\bibverse{4} Y Esaú corrió á su encuentro, y abrazóle, y echóse sobre su
cuello, y le besó; y lloraron. \bibverse{5} Y alzó sus ojos, y vió las
mujeres y los niños, y dijo: ¿Qué te tocan éstos? Y él respondió: Son
los niños que Dios ha dado á tu siervo. \footnote{\textbf{33:5} Sal
  127,3}

\bibverse{6} Y se llegaron las siervas, ellas y sus niños, é
inclináronse. \bibverse{7} Y llegóse Lea con sus niños, é inclináronse:
y después llegó José y Rachêl, y también se inclinaron.

\bibverse{8} Y él dijo: ¿Qué te propones con todas estas cuadrillas que
he encontrado? Y él respondió: El hallar gracia en los ojos de mi señor.

\bibverse{9} Y dijo Esaú: Harto tengo yo, hermano mío: sea para ti lo
que es tuyo.

\bibverse{10} Y dijo Jacob: No, yo te ruego, si he hallado ahora gracia
en tus ojos, toma mi presente de mi mano, pues que así he visto tu
rostro, como si hubiera visto el rostro de Dios; y hazme placer.
\footnote{\textbf{33:10} 2Sam 14,17} \bibverse{11} Toma, te ruego, mi
dádiva que te es traída; porque Dios me ha hecho merced, y todo lo que
hay aquí es mío. Y porfió con él, y tomóla. \footnote{\textbf{33:11}
  1Sam 25,27; 1Sam 30,26}

\hypertarget{jacob-se-niega-a-escoltar-a-esauxfa-esto-vuelve-a-seir}{%
\subsection{Jacob se niega a escoltar a Esaú; esto vuelve a
Seir}\label{jacob-se-niega-a-escoltar-a-esauxfa-esto-vuelve-a-seir}}

\bibverse{12} Y dijo: Anda, y vamos; y yo iré delante de ti.

\bibverse{13} Y él le dijo: Mi señor sabe que los niños son tiernos, y
que tengo ovejas y vacas paridas; y si las fatigan, en un día morirán
todas las ovejas. \bibverse{14} Pase ahora mi señor delante de su
siervo, y yo me iré poco á poco al paso de la hacienda que va delante de
mí, y al paso de los niños, hasta que llegue á mi señor á Seir.

\bibverse{15} Y Esaú dijo: Dejaré ahora contigo de la gente que viene
conmigo. Y él dijo: ¿Para qué esto? halle yo gracia en los ojos de mi
señor.

\bibverse{16} Así se volvió Esaú aquel día por su camino á Seir.

\hypertarget{jacob-se-traslada-a-succoth-y-se-instala-con-sichuxeam}{%
\subsection{Jacob se traslada a Succoth y se instala con
Sichêm}\label{jacob-se-traslada-a-succoth-y-se-instala-con-sichuxeam}}

\bibverse{17} Y Jacob se partió á Succoth, y edificó allí casa para sí,
é hizo cabañas para su ganado: por tanto llamó el nombre de aquel lugar
Succoth.

\bibverse{18} Y vino Jacob sano á la ciudad de Sichêm, que está en la
tierra de Canaán, cuando venía de Padan-aram; y acampó delante de la
ciudad. \bibverse{19} Y compró una parte del campo, donde tendió su
tienda, de mano de los hijos de Hamor, padre de Sichêm, por cien piezas
de moneda. \footnote{\textbf{33:19} Jos 24,32} \bibverse{20} Y erigió
allí un altar, y llamóle: El Dios de Israel. \footnote{\textbf{33:20}
  Gén 12,7-8}

\hypertarget{la-ofensa-de-sichuxeam-contra-dina}{%
\subsection{La ofensa de Sichêm contra
Dina}\label{la-ofensa-de-sichuxeam-contra-dina}}

\hypertarget{section-33}{%
\section{34}\label{section-33}}

\bibverse{1} Y SALIÓ Dina la hija de Lea, la cual había ésta parido á
Jacob, á ver las hijas del país. \footnote{\textbf{34:1} Gén 30,21}
\bibverse{2} Y vióla Sichêm, hijo de Hamor Heveo, príncipe de aquella
tierra, y tomóla, y echóse con ella, y la deshonró. \bibverse{3} Mas su
alma se apegó á Dina la hija de Lea, y enamoróse de la moza, y habló al
corazón de la joven. \bibverse{4} Y habló Sichêm á Hamor su padre,
diciendo: Tómame por mujer esta moza.

\bibverse{5} Y oyó Jacob que había Sichêm amancillado á Dina su hija: y
estando sus hijos con su ganado en el campo, calló Jacob hasta que ellos
viniesen.

\hypertarget{hemor-corteja-a-dina-de-los-hijos-de-jacob}{%
\subsection{Hemor corteja a Dina de los hijos de
Jacob}\label{hemor-corteja-a-dina-de-los-hijos-de-jacob}}

\bibverse{6} Y dirigióse Hamor padre de Sichêm á Jacob, para hablar con
él. \bibverse{7} Y los hijos de Jacob vinieron del campo cuando lo
supieron; y se entristecieron los varones, y se ensañaron mucho, porque
hizo vileza en Israel echándose con la hija de Jacob, lo que no se debía
haber hecho. \bibverse{8} Y Hamor habló con ellos, diciendo: El alma de
mi hijo Sichêm se ha apegado á vuestra hija; ruégoos que se la deis por
mujer. \bibverse{9} Y emparentad con nosotros; dadnos vuestras hijas, y
tomad vosotros las nuestras. \bibverse{10} Y habitad con nosotros;
porque la tierra estará delante de vosotros; morad y negociad en ella, y
tomad en ella posesión.

\bibverse{11} Sichêm también dijo á su padre y á sus hermanos: Halle yo
gracia en vuestros ojos, y daré lo que me dijereis. \bibverse{12}
Aumentad á cargo mío mucho dote y dones, que yo daré cuanto me dijereis,
y dadme la moza por mujer. \footnote{\textbf{34:12} Éxod 22,15}

\hypertarget{la-demanda-de-los-hijos-de-jacob-es-aceptada-por-los-sichuxeamitas.}{%
\subsection{La demanda de los hijos de Jacob es aceptada por los
sichêmitas.}\label{la-demanda-de-los-hijos-de-jacob-es-aceptada-por-los-sichuxeamitas.}}

\bibverse{13} Y respondieron los hijos de Jacob á Sichêm y á Hamor su
padre con engaño; y parlaron, por cuanto había amancillado á Dina su
hermana. \bibverse{14} Y dijéronles: No podemos hacer esto de dar
nuestra hermana á hombre que tiene prepucio; porque entre nosotros es
abominación. \bibverse{15} Mas con esta condición os haremos placer: si
habéis de ser como nosotros, que se circuncide entre vosotros todo
varón; \bibverse{16} Entonces os daremos nuestras hijas, y tomaremos
nosotros las vuestras; y habitaremos con vosotros, y seremos un pueblo.
\bibverse{17} Mas si no nos prestareis oído para circuncidaros,
tomaremos nuestra hija, y nos iremos.

\bibverse{18} Y parecieron bien sus palabras á Hamor y á Sichêm, hijo de
Hamor. \bibverse{19} Y no dilató el mozo hacer aquello, porque la hija
de Jacob le había agradado: y él era el más honrado de toda la casa de
su padre. \bibverse{20} Entonces Hamor y Sichêm su hijo vinieron á la
puerta de su ciudad, y hablaron á los varones de su ciudad, diciendo:
\bibverse{21} Estos varones son pacíficos con nosotros, y habitarán en
el país, y traficarán en él: pues he aquí la tierra es bastante ancha
para ellos: nosotros tomaremos sus hijas por mujeres, y les daremos las
nuestras. \bibverse{22} Mas con esta condición nos harán estos hombres
el placer de habitar con nosotros, para que seamos un pueblo: si se
circuncidare en nosotros todo varón, así como ellos son circuncidados.
\bibverse{23} Sus ganados, y su hacienda y todas sus bestias, serán
nuestras: solamente convengamos con ellos, y habitarán con nosotros.

\bibverse{24} Y obedecieron á Hamor y á Sichêm su hijo todos los que
salían por la puerta de la ciudad, y circuncidaron á todo varón, á
cuantos salían por la puerta de su ciudad.

\hypertarget{la-venganza-engauxf1osa-de-los-hijos-de-jacob}{%
\subsection{La venganza engañosa de los hijos de
Jacob}\label{la-venganza-engauxf1osa-de-los-hijos-de-jacob}}

\bibverse{25} Y sucedió que al tercer día, cuando sentían ellos el mayor
dolor, los dos hijos de Jacob, Simeón y Leví, hermanos de Dina, tomaron
cada uno su espada, y vinieron contra la ciudad animosamente, y mataron
á todo varón. \bibverse{26} Y á Hamor y á Sichêm su hijo los mataron á
filo de espada: y tomaron á Dina de casa de Sichêm, y saliéronse.
\bibverse{27} Y los hijos de Jacob vinieron á los muertos, y saquearon
la ciudad; por cuanto habían amancillado á su hermana. \bibverse{28}
Tomaron sus ovejas y vacas y sus asnos, y lo que había en la ciudad y en
el campo, \bibverse{29} Y toda su hacienda; se llevaron cautivos á todos
sus niños y sus mujeres, y robaron todo lo que había en casa.

\hypertarget{el-disgusto-de-jacob-por-el-acto-reprensible-de-sus-hijos}{%
\subsection{El disgusto de Jacob por el acto reprensible de sus
hijos}\label{el-disgusto-de-jacob-por-el-acto-reprensible-de-sus-hijos}}

\bibverse{30} Entonces dijo Jacob á Simeón y á Leví: Habéisme turbado
con hacerme abominable á los moradores de aquesta tierra, el Cananeo y
el Pherezeo; y teniendo yo pocos hombres, juntarse han contra mí, y me
herirán, y seré destruído yo y mi casa. \footnote{\textbf{34:30} Éxod
  5,21}

\bibverse{31} Y ellos respondieron: ¿Había él de tratar á nuestra
hermana como á una ramera?

\hypertarget{por-amonestaciuxf3n-de-dios-jacob-parte-de-siquem}{%
\subsection{Por amonestación de Dios, Jacob parte de
Siquem}\label{por-amonestaciuxf3n-de-dios-jacob-parte-de-siquem}}

\hypertarget{section-34}{%
\section{35}\label{section-34}}

\bibverse{1} Y DIJO Dios á Jacob: Levántate, sube á Beth-el, y estáte
allí; y haz allí un altar al Dios que te apareció cuando huías de tu
hermano Esaú.

\bibverse{2} Entonces Jacob dijo á su familia y á todos los que con él
estaban: Quitad los dioses ajenos que hay entre vosotros, y limpiaos, y
mudad vuestros vestidos. \footnote{\textbf{35:2} Gén 31,19; Jos 24,23;
  1Sam 7,3} \bibverse{3} Y levantémonos, y subamos á Beth-el; y haré
allí altar al Dios que me respondió en el día de mi angustia, y ha sido
conmigo en el camino que he andado. \footnote{\textbf{35:3} Gén 28,15;
  Gén 28,20-22}

\bibverse{4} Así dieron á Jacob todos los dioses ajenos que había en
poder de ellos, y los zarzillos que estaban en sus orejas; y Jacob los
escondió debajo de una encina, que estaba junto á Sichêm. \footnote{\textbf{35:4}
  Jos 24,26; Jue 9,6} \bibverse{5} Y partiéronse, y el terror de Dios
fué sobre las ciudades que había en sus alrededores, y no siguieron tras
los hijos de Jacob.

\hypertarget{llegada-de-jacob-y-construcciuxf3n-del-altar-en-betel}{%
\subsection{Llegada de Jacob y construcción del altar en
Betel}\label{llegada-de-jacob-y-construcciuxf3n-del-altar-en-betel}}

\bibverse{6} Y llegó Jacob á Luz, que está en tierra de Canaán, (esta es
Beth-el) él y todo el pueblo que con él estaba; \bibverse{7} Y edificó
allí un altar, y llamó al lugar El-Beth-el, porque allí le había
aparecido Dios, cuando huía de su hermano. \bibverse{8} Entonces murió
Débora, ama de Rebeca, y fué sepultada á las raíces de Beth-el, debajo
de una encina: y llamóse su nombre Allon-Bacuth.

\hypertarget{jacob-bendecido-por-dios}{%
\subsection{Jacob bendecido por Dios}\label{jacob-bendecido-por-dios}}

\bibverse{9} Y aparecióse otra vez Dios á Jacob, cuando se había vuelto
de Padan-aram, y bendíjole. \bibverse{10} Y díjole Dios: Tu nombre es
Jacob; no se llamará más tu nombre Jacob, sino Israel será tu nombre: y
llamó su nombre Israel. \footnote{\textbf{35:10} Gén 32,29}
\bibverse{11} Y díjole Dios: Yo soy el Dios Omnipotente: crece y
multiplícate; una nación y conjunto de naciones procederá de ti, y reyes
saldrán de tus lomos: \footnote{\textbf{35:11} Gén 17,1; Gén 28,3-4; Gén
  17,6} \bibverse{12} Y la tierra que yo he dado á Abraham y á Isaac, la
daré á ti: y á tu simiente después de ti daré la tierra.

\bibverse{13} Y fuése de él Dios, del lugar donde con él había hablado.
\footnote{\textbf{35:13} Gén 17,22} \bibverse{14} Y Jacob erigió un
título en el lugar donde había hablado con él, un título de piedra, y
derramó sobre él libación, y echó sobre él aceite. \footnote{\textbf{35:14}
  Gén 28,18-19} \bibverse{15} Y llamó Jacob el nombre de aquel lugar
donde Dios había hablado con él, Beth-el.

\hypertarget{salida-de-betel-rahel-muere-al-dar-a-luz-a-benjamuxedn}{%
\subsection{Salida de Betel; Rahel muere al dar a luz a
Benjamín}\label{salida-de-betel-rahel-muere-al-dar-a-luz-a-benjamuxedn}}

\bibverse{16} Y partieron de Beth-el, y había aún como media legua de
tierra para llegar á Ephrata, cuando parió Rachêl, y hubo trabajo en su
parto. \bibverse{17} Y aconteció, que como había trabajo en su parir,
díjole la partera: No temas, que también tendrás este hijo.

\bibverse{18} Y acaeció que al salírsele el alma, (pues murió) llamó su
nombre Benoni; mas su padre lo llamó Benjamín. \bibverse{19} Así murió
Rachêl, y fué sepultada en el camino de Ephrata, la cual es Beth-lehem.
\footnote{\textbf{35:19} Miq 5,1} \bibverse{20} Y puso Jacob un título
sobre su sepultura: este es el título de la sepultura de Rachêl hasta
hoy.

\hypertarget{la-indignaciuxf3n-de-rubens-los-doce-hijos-de-jacob-su-regreso-a-hebruxf3n-muerte-y-entierro-de-isaac}{%
\subsection{La indignación de Rubens; Los doce hijos de Jacob; su
regreso a Hebrón; Muerte y entierro de
Isaac}\label{la-indignaciuxf3n-de-rubens-los-doce-hijos-de-jacob-su-regreso-a-hebruxf3n-muerte-y-entierro-de-isaac}}

\bibverse{21} Y partió Israel, y tendió su tienda de la otra parte de
Migdaleder. \bibverse{22} Y acaeció, morando Israel en aquella tierra,
que fué Rubén y durmió con Bilha la concubina de su padre; lo cual llegó
á entender Israel. Ahora bien, los hijos de Israel fueron doce:
\footnote{\textbf{35:22} Gén 49,4}

\bibverse{23} Los hijos de Lea: Rubén el primogénito de Jacob, y Simeón,
y Leví, y Judá, é Issachâr, y Zabulón. \bibverse{24} Los hijos de
Rachêl: José, y Benjamín. \bibverse{25} Y los hijos de Bilha, sierva de
Rachêl: Dan, y Nephtalí. \bibverse{26} Y los hijos de Zilpa, sierva de
Lea: Gad, y Aser. Estos fueron los hijos de Jacob, que le nacieron en
Padan-aram. \bibverse{27} Y vino Jacob á Isaac su padre á Mamre, á la
ciudad de Arba, que es Hebrón, donde habitaron Abraham é Isaac.

\bibverse{28} Y fueron los días de Isaac ciento ochenta años.
\bibverse{29} Y exhaló Isaac el espíritu, y murió, y fué recogido á sus
pueblos, viejo y harto de días; y sepultáronlo Esaú y Jacob sus hijos.

\hypertarget{la-familia-y-la-residencia-de-esauxfa}{%
\subsection{La familia y la residencia de
Esaú}\label{la-familia-y-la-residencia-de-esauxfa}}

\hypertarget{section-35}{%
\section{36}\label{section-35}}

\bibverse{1} Y ESTAS son las generaciones de Esaú, el cual es Edom.
\footnote{\textbf{36:1} Gén 25,30} \bibverse{2} Esaú tomó sus mujeres de
las hijas de Canaán: á Ada, hija de Elón Hetheo, y á Aholibama, hija de
Ana, hija de Zibeón el Heveo; \footnote{\textbf{36:2} Gén 26,34}
\bibverse{3} Y á Basemath, hija de Ismael, hermana de Navaioth.
\footnote{\textbf{36:3} Gén 28,9} \bibverse{4} Y Ada parió á Esaú á
Eliphaz; y Basemath parió á Reuel. \bibverse{5} Y Aholibama parió á
Jeús, y á Jaalam, y á Cora: estos son los hijos de Esaú, que le nacieron
en la tierra de Canaán. \bibverse{6} Y Esaú tomó sus mujeres, y sus
hijos, y sus hijas, y todas las personas de su casa, y sus ganados, y
todas sus bestias, y toda su hacienda que había adquirido en la tierra
de Canaán, y fuése á otra tierra de delante de Jacob su hermano:
\bibverse{7} Porque la hacienda de ellos era grande, y no podían habitar
juntos, ni la tierra de su peregrinación los podía sostener á causa de
sus ganados. \bibverse{8} Y Esaú habitó en el monte de Seir: Esaú es
Edom.

\hypertarget{los-hijos-y-nietos-de-esauxfa-como-progenitores}{%
\subsection{Los hijos y nietos de Esaú como
progenitores}\label{los-hijos-y-nietos-de-esauxfa-como-progenitores}}

\bibverse{9} Estos son los linajes de Esaú, padre de Edom, en el monte
de Seir. \bibverse{10} Estos son los nombres de los hijos de Esaú:
Eliphaz, hijo de Ada, mujer de Esaú; Reuel, hijo de Basemath, mujer de
Esaú. \bibverse{11} Y los hijos de Eliphaz fueron Temán, Omar, Zepho,
Gatam, y Cenaz. \bibverse{12} Y Timna fué concubina de Eliphaz, hijo de
Esaú, la cual le parió á Amalec: estos son los hijos de Ada, mujer de
Esaú. \bibverse{13} Y los hijos de Reuel fueron Nahath, Zera, Samma, y
Mizza: estos son los hijos de Basemath, mujer de Esaú. \bibverse{14}
Estos fueron los hijos de Aholibama, mujer de Esaú, hija de Ana, que fué
hija de Zibeón: ella parió á Esaú á Jeús, Jaalam, y Cora.

\hypertarget{los-duques-descendieron-de-esauxfa}{%
\subsection{Los duques descendieron de
Esaú}\label{los-duques-descendieron-de-esauxfa}}

\bibverse{15} Estos son los duques de los hijos de Esaú. Hijos de
Eliphaz, primogénito de Esaú: el duque Temán, el duque Omar, el duque
Zepho, el duque Cenaz, \bibverse{16} El duque Cora, el duque Gatam, y el
duque Amalec: estos son los duques de Eliphaz en la tierra de Edom;
estos fueron los hijos de Ada. \bibverse{17} Y estos son los hijos de
Reuel, hijo de Esaú: el duque Nahath, el duque Zera, el duque Samma, y
el duque Mizza: estos son los duques de la línea de Reuel en la tierra
de Edom; estos hijos vienen de Basemath, mujer de Esaú. \bibverse{18} Y
estos son los hijos de Aholibama, mujer de Esaú: el duque Jeús, el duque
Jaalam, y el duque Cora: estos fueron los duques que salieron de
Aholibama, mujer de Esaú, hija de Ana. \bibverse{19} Estos, pues, son
los hijos de Esaú, y sus duques: él es Edom.

\hypertarget{los-horeos-que-eran-independientes-de-esauxfa}{%
\subsection{Los horeos que eran independientes de
Esaú}\label{los-horeos-que-eran-independientes-de-esauxfa}}

\bibverse{20} Y estos son los hijos de Seir Horeo, moradores de aquella
tierra: Lotán, Sobal, Zibeón, Ana, \footnote{\textbf{36:20} Gén 14,6;
  Deut 2,12} \bibverse{21} Disón, Ezer, y Disán: estos son los duques de
los Horeos, hijos de Seir en la tierra de Edom. \bibverse{22} Los hijos
de Lotán fueron Hori y Hemán; y Timna fué hermana de Lotán.
\bibverse{23} Y los hijos de Sobal fueron Alván, Manahath, Ebal, Sepho,
y Onán. \bibverse{24} Y los hijos de Zibeón fueron Aja, y Ana. Este Ana
es el que descubrió los mulos en el desierto, cuando apacentaba los
asnos de Zibeón su padre. \bibverse{25} Los hijos de Ana fueron Disón, y
Aholibama, hija de Ana. \bibverse{26} Y estos fueron los hijos de Disón:
Hemdán, Eshbán, Ithram, y Cherán. \bibverse{27} Y estos fueron los hijos
de Ezer: Bilhán, Zaaván, y Acán. \bibverse{28} Estos fueron los hijos de
Disán: Huz, y Arán. \bibverse{29} Y estos fueron los duques de los
Horeos: el duque Lotán, el duque Sobal, el duque Zibeón, el duque Ana,
\bibverse{30} El duque Disón, el duque Ezer, el duque Disán: estos
fueron los duques de los Horeos: por sus ducados en la tierra de Seir.

\hypertarget{los-reyes-de-la-tierra-de-edom-hasta-david}{%
\subsection{Los reyes de la tierra de Edom hasta
David}\label{los-reyes-de-la-tierra-de-edom-hasta-david}}

\bibverse{31} Y los reyes que reinaron en la tierra de Edom, antes que
reinase rey sobre los hijos de Israel, fueron estos: \bibverse{32} Bela,
hijo de Beor, reinó en Edom: y el nombre de su ciudad fué Dinaba.
\bibverse{33} Y murió Bela, y reinó en su lugar Jobab, hijo de Zera, de
Bosra. \bibverse{34} Y murió Jobab, y en su lugar reinó Husam, de tierra
de Temán. \bibverse{35} Y murió Husam, y reinó en su lugar Adad, hijo de
Badad, el que hirió á Midián en el campo de Moab: y el nombre de su
ciudad fué Avith. \bibverse{36} Y murió Adad, y en su lugar reinó Samla,
de Masreca. \bibverse{37} Y murió Samla, y reinó en su lugar Saúl, de
Rehoboth del Río. \bibverse{38} Y murió Saúl, y en lugar suyo reinó
Baalanán, hijo de Achbor. \bibverse{39} Y murió Baalanán, hijo de
Achbor, y reinó Adar en lugar suyo: y el nombre de su ciudad fué Pau; y
el nombre de su mujer Meetabel, hija de Matred, hija de Mezaab.

\hypertarget{los-duques-de-edom-por-sus-lugares}{%
\subsection{Los duques de Edom por sus
lugares}\label{los-duques-de-edom-por-sus-lugares}}

\bibverse{40} Estos, pues, son los nombres de los duques de Esaú por sus
linajes, por sus lugares, y sus nombres: el duque Timna, el duque Alva,
el duque Jetheth, \bibverse{41} El duque Aholibama, el duque Ela, el
duque Pinón, \bibverse{42} El duque Cenaz, el duque Temán, el duque
Mibzar, \bibverse{43} El duque Magdiel, y el duque Hiram. Estos fueron
los duques de Edom por sus habitaciones en la tierra de su posesión.
Edom es el mismo Esaú, padre de los Idumeos.

\hypertarget{los-inicios-de-la-enemistad-de-los-hermanos-contra-josuxe9}{%
\subsection{Los inicios de la enemistad de los hermanos contra
José}\label{los-inicios-de-la-enemistad-de-los-hermanos-contra-josuxe9}}

\hypertarget{section-36}{%
\section{37}\label{section-36}}

\bibverse{1} Y HABITÓ Jacob en la tierra donde peregrinó su padre, en la
tierra de Canaán. \bibverse{2} Estas fueron las generaciones de Jacob.
José, siendo de edad de diez y siete años apacentaba las ovejas con sus
hermanos; y el joven estaba con los hijos de Bilha, y con los hijos de
Zilpa, mujeres de su padre: y noticiaba José á su padre la mala fama de
ellos. \bibverse{3} Y amaba Israel á José más que á todos sus hijos,
porque le había tenido en su vejez: y le hizo una ropa de diversos
colores. \bibverse{4} Y viendo sus hermanos que su padre lo amaba más
que á todos sus hermanos, aborrecíanle, y no le podían hablar
pacíficamente.

\hypertarget{los-sueuxf1os-de-josuxe9}{%
\subsection{Los sueños de José}\label{los-sueuxf1os-de-josuxe9}}

\bibverse{5} Y soñó José un sueño, y contólo á sus hermanos; y ellos
vinieron á aborrecerle más todavía. \bibverse{6} Y él les dijo: Oíd
ahora este sueño que he soñado: \bibverse{7} He aquí que atábamos
manojos en medio del campo, y he aquí que mi manojo se levantaba, y
estaba derecho, y que vuestros manojos estaban alrededor, y se
inclinaban al mío.

\bibverse{8} Y respondiéronle sus hermanos: ¿Has de reinar tú sobre
nosotros, ó te has de enseñorear sobre nosotros? Y le aborrecieron aún
más á causa de sus sueños y de sus palabras. \bibverse{9} Y soñó aún
otro sueño, y contólo á sus hermanos, diciendo: He aquí que he soñado
otro sueño, y he aquí que el sol y la luna y once estrellas se
inclinaban á mí. \bibverse{10} Y contólo á su padre y á sus hermanos: y
su padre le reprendió, y díjole: ¿Qué sueño es este que soñaste? ¿Hemos
de venir yo y tu madre, y tus hermanos, á inclinarnos á ti á tierra?
\bibverse{11} Y sus hermanos le tenían envidia, mas su padre paraba la
consideración en ello.

\hypertarget{la-oportunidad-de-deshacerse-de-joseph}{%
\subsection{La oportunidad de deshacerse de
Joseph}\label{la-oportunidad-de-deshacerse-de-joseph}}

\bibverse{12} Y fueron sus hermanos á apacentar las ovejas de su padre
en Sichêm. \bibverse{13} Y dijo Israel á José: Tus hermanos apacientan
las ovejas en Sichêm: ven, y te enviaré á ellos. Y él respondió: Heme
aquí.

\bibverse{14} Y él le dijo: Ve ahora, mira cómo están tus hermanos y
cómo están las ovejas, y tráeme la respuesta. Y enviólo del valle de
Hebrón, y llegó á Sichêm. \footnote{\textbf{37:14} Gén 35,27}
\bibverse{15} Y hallólo un hombre, andando él perdido por el campo, y
preguntóle aquel hombre, diciendo: ¿Qué buscas?

\bibverse{16} Y él respondió: Busco á mis hermanos: ruégote que me
muestres dónde pastan.

\bibverse{17} Y aquel hombre respondió: Ya se han ido de aquí; y yo les
oí decir: Vamos á Dothán. Entonces José fué tras de sus hermanos, y
hallólos en Dothán.

\bibverse{18} Y como ellos lo vieron de lejos, antes que cerca de ellos
llegara, proyectaron contra él para matarle. \bibverse{19} Y dijeron el
uno al otro: He aquí viene el soñador; \bibverse{20} Ahora pues, venid,
y matémoslo y echémosle en una cisterna, y diremos: Alguna mala bestia
le devoró: y veremos qué serán sus sueños.

\hypertarget{rubuxe9n-y-juduxe1-intentan-salvar-a-josuxe9}{%
\subsection{Rubén y Judá intentan salvar a
José}\label{rubuxe9n-y-juduxe1-intentan-salvar-a-josuxe9}}

\bibverse{21} Y como Rubén oyó esto, librólo de sus manos, y dijo: No lo
matemos. \bibverse{22} Y díjoles Rubén: No derraméis sangre; echadlo en
esta cisterna que está en el desierto, y no pongáis mano en él; por
librarlo así de sus manos, para hacerlo volver á su padre. \bibverse{23}
Y sucedió que, cuando llegó José á sus hermanos, ellos hicieron desnudar
á José su ropa, la ropa de colores que tenía sobre sí; \footnote{\textbf{37:23}
  Gén 37,3} \bibverse{24} Y tomáronlo, y echáronle en la cisterna; mas
la cisterna estaba vacía, no había en ella agua. \footnote{\textbf{37:24}
  Jer 38,6}

\bibverse{25} Y sentáronse á comer pan: y alzando los ojos miraron, y he
aquí una compañía de Ismaelitas que venía de Galaad, y sus camellos
traían aromas y bálsamo y mirra, é iban á llevarlo á Egipto.
\bibverse{26} Entonces Judá dijo á sus hermanos: ¿Qué provecho el que
matemos á nuestro hermano y encubramos su muerte? \bibverse{27} Venid, y
vendámosle á los Ismaelitas, y no sea nuestra mano sobre él; que nuestro
hermano es nuestra carne. Y sus hermanos acordaron con él.

\hypertarget{josuxe9-es-vendido-a-egipto}{%
\subsection{José es vendido a
Egipto}\label{josuxe9-es-vendido-a-egipto}}

\bibverse{28} Y como pasaban los Midianitas mercaderes, sacaron ellos á
José de la cisterna, y trajéronle arriba, y le vendieron á los
Ismaelitas por veinte piezas de plata. Y llevaron á José á Egipto.
\footnote{\textbf{37:28} Gén 25,2}

\bibverse{29} Y Rubén volvió á la cisterna, y no halló á José dentro, y
rasgó sus vestidos. \footnote{\textbf{37:29} Gén 44,13; 2Sam 1,11}
\bibverse{30} Y tornó á sus hermanos, y dijo: El mozo no parece; y yo,
¿adónde iré yo? \bibverse{31} Entonces tomaron ellos la ropa de José, y
degollaron un cabrito de las cabras, y tiñeron la ropa con la sangre;
\bibverse{32} Y enviaron la ropa de colores y trajéronla á su padre, y
dijeron: Esta hemos hallado, reconoce ahora si es ó no la ropa de tu
hijo.

\hypertarget{el-dolor-de-jacob-josuxe9-vendido-a-potifar-en-egipto}{%
\subsection{El dolor de Jacob; José vendido a Potifar en
Egipto}\label{el-dolor-de-jacob-josuxe9-vendido-a-potifar-en-egipto}}

\bibverse{33} Y él la conoció, y dijo: La ropa de mi hijo es; alguna
mala bestia le devoró; José ha sido despedazado. \footnote{\textbf{37:33}
  Gén 37,20} \bibverse{34} Entonces Jacob rasgó sus vestidos, y puso
saco sobre sus lomos, y enlutóse por su hijo muchos días. \footnote{\textbf{37:34}
  Gén 37,29} \bibverse{35} Y levantáronse todos sus hijos y todas sus
hijas para consolarlo; mas él no quiso tomar consolación, y dijo: Porque
yo tengo de descender á mi hijo enlutado hasta la sepultura. Y llorólo
su padre. \bibverse{36} Y los Midianitas lo vendieron en Egipto á
Potiphar, eunuco de Faraón, capitán de los de la guardia.

\hypertarget{los-hijos-de-juda-y-thamar}{%
\subsection{Los hijos de Juda y
Thamar}\label{los-hijos-de-juda-y-thamar}}

\hypertarget{section-37}{%
\section{38}\label{section-37}}

\bibverse{1} Y ACONTECIÓ en aquel tiempo, que Judá descendió de con sus
hermanos, y fuése á un varón Adullamita, que se llamaba Hira.
\bibverse{2} Y vió allí Judá la hija de un hombre Cananeo, el cual se
llamaba Súa; y tomóla, y entró á ella: \bibverse{3} La cual concibió, y
parió un hijo; y llamó su nombre Er. \bibverse{4} Y concibió otra vez, y
parió un hijo, y llamó su nombre Onán. \bibverse{5} Y volvió á concebir,
y parió un hijo, y llamó su nombre Sela. Y estaba en Chezib cuando lo
parió. \bibverse{6} Y Judá tomó mujer para su primogénito Er, la cual se
llamaba Thamar. \bibverse{7} Y Er, el primogénito de Judá, fué malo á
los ojos de Jehová, y quitóle Jehová la vida. \bibverse{8} Entonces Judá
dijo á Onán: Entra á la mujer de tu hermano, y despósate con ella, y
suscita simiente á tu hermano. \footnote{\textbf{38:8} Deut 25,5}
\bibverse{9} Y sabiendo Onán que la simiente no había de ser suya,
sucedía que cuando entraba á la mujer de su hermano vertía en tierra,
por no dar simiente á su hermano. \bibverse{10} Y desagradó en ojos de
Jehová lo que hacía, y también quitó á él la vida. \bibverse{11} Y Judá
dijo á Thamar su nuera: Estáte viuda en casa de tu padre, hasta que
crezca Sela mi hijo; porque dijo: Que quizá no muera él también como sus
hermanos. Y fuése Thamar, y estúvose en casa de su padre.

\hypertarget{thamar-usa-astucia-para-obtener-descendencia-de-su-suegro-judah}{%
\subsection{Thamar usa astucia para obtener descendencia de su suegro
Judah}\label{thamar-usa-astucia-para-obtener-descendencia-de-su-suegro-judah}}

\bibverse{12} Y pasaron muchos días, y murió la hija de Súa, mujer de
Judá; y Judá se consoló, y subía á los trasquiladores de sus ovejas á
Timnath, él y su amigo Hira el Adullamita. \bibverse{13} Y fué dado
aviso á Thamar, diciendo: He aquí tu suegro sube á Timnath á trasquilar
sus ovejas. \bibverse{14} Entonces quitó ella de sobre sí los vestidos
de su viudez, y cubrióse con un velo, y arrebozóse, y se puso á la
puerta de las aguas que están junto al camino de Timnath; porque veía
que había crecido Sela, y ella no era dada á él por mujer. \bibverse{15}
Y vióla Judá, y túvola por ramera, porque había ella cubierto su rostro.
\bibverse{16} Y apartóse del camino hacia ella, y díjole: Ea, pues,
ahora entraré á ti; porque no sabía que era su nuera; y ella dijo: ¿Qué
me has de dar, si entrares á mí?

\bibverse{17} El respondió: Yo te enviaré del ganado un cabrito de las
cabras. Y ella dijo: Hasme de dar prenda hasta que lo envíes.

\bibverse{18} Entonces él dijo: ¿Qué prenda te daré? Ella respondió: Tu
anillo, y tu manto, y tu bordón que tienes en tu mano. Y él se los dió,
y entró á ella, la cual concibió de él.

\bibverse{19} Y levantóse, y fuése: y quitóse el velo de sobre sí, y
vistióse las ropas de su viudez. \bibverse{20} Y Judá envió el cabrito
de las cabras por mano de su amigo el Adullamita, para que tomase la
prenda de mano de la mujer; mas no la halló. \bibverse{21} Y preguntó á
los hombres de aquel lugar, diciendo: ¿Dónde está la ramera de las aguas
junto al camino? Y ellos le dijeron: No ha estado aquí ramera.

\bibverse{22} Entonces él se volvió á Judá, y dijo: No la he hallado; y
también los hombres del lugar dijeron: Aquí no ha estado ramera.
\bibverse{23} Y Judá dijo: Tómeselo para sí, porque no seamos
menospreciados: he aquí yo he enviado este cabrito, y tú no la hallaste.

\hypertarget{judas-juicio-justo-sobre-suxed-mismo-y-thamar}{%
\subsection{Judas juicio justo sobre sí mismo y
Thamar}\label{judas-juicio-justo-sobre-suxed-mismo-y-thamar}}

\bibverse{24} Y acaeció que al cabo de unos tres meses fué dado aviso á
Judá, diciendo: Thamar tu nuera ha fornicado, y aun cierto está preñada
de las fornicaciones. Y Judá dijo: Sacadla, y sea quemada.

\bibverse{25} Y ella, cuando la sacaban, envió á decir á su suegro: Del
varón cuyas son estas cosas, estoy preñada: y dijo más: Mira ahora cúyas
son estas cosas, el anillo, y el manto, y el bordón.

\bibverse{26} Entonces Judá los reconoció, y dijo: Más justa es que yo,
por cuanto no la he dado á Sela mi hijo. Y nunca más la conoció.

\hypertarget{thamar-da-a-luz-a-los-gemelos-puxe9rez-y-serah}{%
\subsection{Thamar da a luz a los gemelos Pérez y
Serah}\label{thamar-da-a-luz-a-los-gemelos-puxe9rez-y-serah}}

\bibverse{27} Y aconteció que al tiempo del parir, he aquí había dos en
su vientre. \bibverse{28} Y sucedió, cuando paría, que sacó la mano el
uno, y la partera tomó y ató á su mano un hilo de grana, diciendo: Este
salió primero. \bibverse{29} Empero fué que tornando él á meter la mano,
he aquí su hermano salió; y ella dijo: ¿Por qué has hecho sobre ti
rotura? Y llamó su nombre Phares. \footnote{\textbf{38:29} Mat 1,3}
\bibverse{30} Y después salió su hermano, el que tenía en su mano el
hilo de grana, y llamó su nombre Zara.

\hypertarget{josuxe9-en-la-casa-de-potifar}{%
\subsection{José en la casa de
Potifar}\label{josuxe9-en-la-casa-de-potifar}}

\hypertarget{section-38}{%
\section{39}\label{section-38}}

\bibverse{1} Y LLEVADO José á Egipto, comprólo Potiphar, eunuco de
Faraón, capitán de los de la guardia, varón Egipcio, de mano de los
Ismaelitas que lo habían llevado allá. \bibverse{2} Mas Jehová fué con
José, y fué varón prosperado: y estaba en la casa de su señor el
Egipcio. \bibverse{3} Y vió su señor que Jehová era con él, y que todo
lo que él hacía, Jehová lo hacía prosperar en su mano. \bibverse{4} Así
halló José gracia en sus ojos, y servíale; y él le hizo mayordomo de su
casa, y entregó en su poder todo lo que tenía. \bibverse{5} Y aconteció
que, desde cuando le dió el encargo de su casa, y de todo lo que tenía,
Jehová bendijo la casa del Egipcio á causa de José; y la bendición de
Jehová fué sobre todo lo que tenía, así en casa como en el campo.
\footnote{\textbf{39:5} Gén 30,27} \bibverse{6} Y dejó todo lo que tenía
en mano de José; ni con él sabía de nada más que del pan que comía. Y
era José de hermoso semblante y bella presencia.

\hypertarget{la-seducciuxf3n-de-la-esposa-de-potifar}{%
\subsection{La seducción de la esposa de
Potifar}\label{la-seducciuxf3n-de-la-esposa-de-potifar}}

\bibverse{7} Y aconteció después de esto, que la mujer de su señor puso
sus ojos en José, y dijo: Duerme conmigo.

\bibverse{8} Y él no quiso, y dijo á la mujer de su señor: He aquí que
mi señor no sabe conmigo lo que hay en casa, y ha puesto en mi mano todo
lo que tiene: \bibverse{9} No hay otro mayor que yo en esta casa, y
ninguna cosa me ha reservado sino á ti, por cuanto tú eres su mujer;
¿cómo, pues, haría yo este grande mal, y pecaría contra Dios?
\footnote{\textbf{39:9} Éxod 20,14}

\bibverse{10} Y fué que hablando ella á José cada día, y no escuchándola
él para acostarse al lado de ella, para estar con ella, \bibverse{11}
Aconteció que entró él un día en casa para hacer su oficio, y no había
nadie de los de casa allí en casa: \bibverse{12} Y asiólo ella por su
ropa, diciendo: Duerme conmigo. Entonces dejóla él su ropa en las manos,
y huyó, y salióse fuera.

\bibverse{13} Y acaeció que cuando vió ella que le había dejado su ropa
en sus manos, y había huído fuera, \bibverse{14} Llamó á los de casa, y
hablóles diciendo: Mirad, nos ha traído un Hebreo, para que hiciese
burla de nosotros: vino él á mí para dormir conmigo, y yo dí grandes
voces; \bibverse{15} Y viendo que yo alzaba la voz y gritaba, dejó junto
á mí su ropa, y huyó, y salióse fuera. \bibverse{16} Y ella puso junto á
sí la ropa de él, hasta que vino su señor á su casa. \bibverse{17}
Entonces le habló ella semejantes palabras, diciendo: El siervo Hebreo
que nos trajiste, vino á mí para deshonrarme; \bibverse{18} Y como yo
alcé mi voz y grité, él dejó su ropa junto á mí, y huyó fuera.

\hypertarget{josuxe9-en-el-carcel}{%
\subsection{José en el Carcel}\label{josuxe9-en-el-carcel}}

\bibverse{19} Y sucedió que como oyó su señor las palabras que su mujer
le hablara, diciendo: Así me ha tratado tu siervo; encendióse su furor.
\bibverse{20} Y tomó su señor á José, y púsole en la casa de la cárcel,
donde estaban los presos del rey, y estuvo allí en la casa de la cárcel.
\bibverse{21} Mas Jehová fué con José, y extendió á él su misericordia,
y dióle gracia en ojos del principal de la casa de la cárcel.
\bibverse{22} Y el principal de la casa de la cárcel entregó en mano de
José todos los presos que había en aquella prisión; todo lo que hacían
allí, él lo hacía. \bibverse{23} No veía el principal de la cárcel cosa
alguna que en su mano estaba; porque Jehová era con él, y lo que él
hacía, Jehová lo prosperaba.

\hypertarget{encarcelamiento-del-copero-y-panadero-del-farauxf3n}{%
\subsection{Encarcelamiento del copero y panadero del
faraón}\label{encarcelamiento-del-copero-y-panadero-del-farauxf3n}}

\hypertarget{section-39}{%
\section{40}\label{section-39}}

\bibverse{1} Y ACONTECIÓ después de estas cosas, que el copero del rey
de Egipto y el panadero delinquieron contra su señor el rey de Egipto.
\bibverse{2} Y enojóse Faraón contra sus dos eunucos, contra el
principal de los coperos, y contra el principal de los panaderos:
\bibverse{3} Y púsolos en prisión en la casa del capitán de los de la
guardia, en la casa de la cárcel donde José estaba preso. \bibverse{4} Y
el capitán de los de la guardia dió cargo de ellos á José, y él les
servía: y estuvieron días en la prisión.

\hypertarget{josuxe9-consuela-a-los-dos-oficiales-de-la-corte}{%
\subsection{José consuela a los dos oficiales de la
corte}\label{josuxe9-consuela-a-los-dos-oficiales-de-la-corte}}

\bibverse{5} Y ambos á dos, el copero y el panadero del rey de Egipto,
que estaban arrestados en la prisión, vieron un sueño, cada uno su sueño
en una misma noche, cada uno conforme á la declaración de su sueño.
\bibverse{6} Y vino á ellos José por la mañana, y mirólos, y he aquí que
estaban tristes. \bibverse{7} Y él preguntó á aquellos eunucos de
Faraón, que estaban con él en la prisión de la casa de su señor,
diciendo: ¿Por qué parecen hoy mal vuestros semblantes?

\bibverse{8} Y ellos le dijeron: Hemos tenido un sueño, y no hay quien
lo declare. Entonces les dijo José: ¿No son de Dios las declaraciones?
Contádmelo ahora. \footnote{\textbf{40:8} Gén 41,16; Dan 2,27-28}

\hypertarget{el-sueuxf1o-del-copero-y-su-interpretaciuxf3n}{%
\subsection{El sueño del copero y su
interpretación}\label{el-sueuxf1o-del-copero-y-su-interpretaciuxf3n}}

\bibverse{9} Entonces el principal de los coperos contó su sueño á José,
y díjole: Yo soñaba que veía una vid delante de mí, \bibverse{10} Y en
la vid tres sarmientos; y ella como que brotaba, y arrojaba su flor,
viniendo á madurar sus racimos de uvas: \bibverse{11} Y que la copa de
Faraón estaba en mi mano, y tomaba yo las uvas, y las exprimía en la
copa de Faraón, y daba yo la copa en mano de Faraón.

\bibverse{12} Y díjole José: Esta es su declaración: Los tres sarmientos
son tres días: \bibverse{13} Al cabo de tres días Faraón te hará
levantar cabeza, y te restituirá á tu puesto: y darás la copa á Faraón
en su mano, como solías cuando eras su copero. \bibverse{14} Acuérdate,
pues, de mí para contigo cuando tuvieres ese bien, y ruégote que uses
conmigo de misericordia, y hagas mención de mí á Faraón, y me saques de
esta casa: \bibverse{15} Porque hurtado he sido de la tierra de los
Hebreos; y tampoco he hecho aquí porqué me hubiesen de poner en la
cárcel.

\hypertarget{el-sueuxf1o-del-panadero-y-su-interpretaciuxf3n}{%
\subsection{El sueño del panadero y su
interpretación}\label{el-sueuxf1o-del-panadero-y-su-interpretaciuxf3n}}

\bibverse{16} Y viendo el principal de los panaderos que había declarado
para bien, dijo á José: También yo soñaba que veía tres canastillos
blancos sobre mi cabeza; \bibverse{17} Y en el canastillo más alto había
de todas las viandas de Faraón, obra de panadero; y que las aves las
comían del canastillo de sobre mi cabeza.

\bibverse{18} Entonces respondió José, y dijo: Esta es su declaración:
Los tres canastillos tres días son; \bibverse{19} Al cabo de tres días
quitará Faraón tu cabeza de sobre ti, y te hará colgar en la horca, y
las aves comerán tu carne de sobre ti.

\hypertarget{el-cumplimiento-de-ambos-sueuxf1os}{%
\subsection{El cumplimiento de ambos
sueños}\label{el-cumplimiento-de-ambos-sueuxf1os}}

\bibverse{20} Y fué el tercero día el día del nacimiento de Faraón, é
hizo banquete á todos sus sirvientes: y alzó la cabeza del principal de
los coperos, y la cabeza del principal de los panaderos, entre sus
servidores. \bibverse{21} E hizo volver á su oficio al principal de los
coperos; y dió él la copa en mano de Faraón. \bibverse{22} Mas hizo
ahorcar al principal de los panaderos, como le había declarado José.
\bibverse{23} Y el principal de los coperos no se acordó de José, sino
que le olvidó. \footnote{\textbf{40:23} Gén 40,14}

\hypertarget{los-dos-sueuxf1os-del-farauxf3n-son-insolubles-para-los-intuxe9rpretes-egipcios}{%
\subsection{Los dos sueños del faraón son insolubles para los
intérpretes
egipcios}\label{los-dos-sueuxf1os-del-farauxf3n-son-insolubles-para-los-intuxe9rpretes-egipcios}}

\hypertarget{section-40}{%
\section{41}\label{section-40}}

\bibverse{1} Y ACONTECIÓ que pasados dos años tuvo Faraón un sueño:
Parecíale que estaba junto al río; \bibverse{2} Y que del río subían
siete vacas, hermosas á la vista, y muy gordas, y pacían en el prado:
\bibverse{3} Y que otras siete vacas subían tras ellas del río, de fea
vista, y enjutas de carne, y se pararon cerca de las vacas hermosas á la
orilla del río: \bibverse{4} Y que las vacas de fea vista y enjutas de
carne devoraban á las siete vacas hermosas y muy gordas. Y despertó
Faraón. \bibverse{5} Durmióse de nuevo, y soñó la segunda vez: Que siete
espigas llenas y hermosas subían de una sola caña: \bibverse{6} Y que
otras siete espigas menudas y abatidas del Solano, salían después de
ellas: \bibverse{7} Y las siete espigas menudas devoraban á las siete
espigas gruesas y llenas. Y despertó Faraón, y he aquí que era sueño.
\bibverse{8} Y acaeció que á la mañana estaba agitado su espíritu; y
envió é hizo llamar á todos los magos de Egipto, y á todos sus sabios: y
contóles Faraón sus sueños, mas no había quien á Faraón los declarase.

\hypertarget{el-copero-hace-los-arreglos-para-que-vayan-a-buscar-a-josuxe9}{%
\subsection{El copero hace los arreglos para que vayan a buscar a
José}\label{el-copero-hace-los-arreglos-para-que-vayan-a-buscar-a-josuxe9}}

\bibverse{9} Entonces el principal de los coperos habló á Faraón,
diciendo: Acuérdome hoy de mis faltas: \bibverse{10} Faraón se enojó
contra sus siervos, y á mí me echó á la prisión de la casa del capitán
de los de la guardia, á mí y al principal de los panaderos:
\bibverse{11} Y yo y él vimos un sueño una misma noche: cada uno soñó
conforme á la declaración de su sueño. \bibverse{12} Y estaba allí con
nosotros un mozo Hebreo, sirviente del capitán de los de la guardia; y
se lo contamos, y él nos declaró nuestros sueños, y declaró á cada uno
conforme á su sueño. \bibverse{13} Y aconteció que como él nos declaró,
así fué: á mí me hizo volver á mi puesto, é hizo colgar al otro.

\bibverse{14} Entonces Faraón envió y llamó á José; é hiciéronle salir
corriendo de la cárcel, y le cortaron el pelo, y mudaron sus vestidos, y
vino á Faraón.

\hypertarget{josuxe9-interpreta-los-sueuxf1os-del-farauxf3n}{%
\subsection{José interpreta los sueños del
faraón}\label{josuxe9-interpreta-los-sueuxf1os-del-farauxf3n}}

\bibverse{15} Y dijo Faraón á José: Yo he tenido un sueño, y no hay
quien lo declare; mas he oído decir de ti, que oyes sueños para
declararlos.

\bibverse{16} Y respondió José á Faraón, diciendo: No está en mí; Dios
será el que responda paz á Faraón. \footnote{\textbf{41:16} Gén 40,8}

\bibverse{17} Entonces Faraón dijo á José: En mi sueño parecíame que
estaba á la orilla del río: \bibverse{18} Y que del río subían siete
vacas de gruesas carnes y hermosa apariencia, que pacían en el prado:
\bibverse{19} Y que otras siete vacas subían después de ellas, flacas y
de muy fea traza; tan extenuadas, que no he visto otras semejantes en
toda la tierra de Egipto en fealdad: \bibverse{20} Y las vacas flacas y
feas devoraban á las siete primeras vacas gruesas: \bibverse{21} Y
entraban en sus entrañas, mas no se conocía que hubiesen entrado en
ellas, porque su parecer era aún malo, como de primero. Y yo desperté.
\bibverse{22} Vi también soñando, que siete espigas subían en una misma
caña llenas y hermosas; \bibverse{23} Y que otras siete espigas menudas,
marchitas, abatidas del Solano, subían después de ellas: \bibverse{24} Y
las espigas menudas devoraban á las siete espigas hermosas: y helo dicho
á los magos, mas no hay quien me lo declare.

\bibverse{25} Entonces respondió José á Faraón: El sueño de Faraón es
uno mismo: Dios ha mostrado á Faraón lo que va á hacer. \bibverse{26}
Las siete vacas hermosas siete años son; y las espigas hermosas son
siete años: el sueño es uno mismo. \bibverse{27} También las siete vacas
flacas y feas que subían tras ellas, son siete años; y las siete espigas
menudas y marchitas del Solano, siete años serán de hambre.
\bibverse{28} Esto es lo que respondo á Faraón. Lo que Dios va á hacer,
halo mostrado á Faraón. \bibverse{29} He aquí vienen siete años de
grande hartura en toda la tierra de Egipto: \bibverse{30} Y levantarse
han tras ellos siete años de hambre; y toda la hartura será olvidada en
la tierra de Egipto; y el hambre consumirá la tierra; \bibverse{31} Y
aquella abundancia no se echará de ver á causa del hambre siguiente, la
cual será gravísima. \bibverse{32} Y el suceder el sueño á Faraón dos
veces, significa que la cosa es firme de parte de Dios, y que Dios se
apresura á hacerla.

\hypertarget{el-consejo-de-josuxe9-sobre-cuxf3mo-hacer-uso-de-la-interpretaciuxf3n-de-sus-sueuxf1os}{%
\subsection{El consejo de José sobre cómo hacer uso de la interpretación
de sus
sueños}\label{el-consejo-de-josuxe9-sobre-cuxf3mo-hacer-uso-de-la-interpretaciuxf3n-de-sus-sueuxf1os}}

\bibverse{33} Por tanto, provéase ahora Faraón de un varón prudente y
sabio, y póngalo sobre la tierra de Egipto. \bibverse{34} Haga esto
Faraón, y ponga gobernadores sobre el país, y quinte la tierra de Egipto
en los siete años de la hartura; \bibverse{35} Y junten toda la
provisión de estos buenos años que vienen, y alleguen el trigo bajo la
mano de Faraón para mantenimiento de las ciudades; y guárdenlo.
\bibverse{36} Y esté aquella provisión en depósito para el país, para
los siete años del hambre que serán en la tierra de Egipto; y el país no
perecerá de hambre.

\hypertarget{joseph-fue-ascendido-a-funcionario-muxe1s-alto-del-estado}{%
\subsection{Joseph fue ascendido a funcionario más alto del
estado}\label{joseph-fue-ascendido-a-funcionario-muxe1s-alto-del-estado}}

\bibverse{37} Y el negocio pareció bien á Faraón, y á sus siervos.
\bibverse{38} Y dijo Faraón á sus siervos: ¿Hemos de hallar otro hombre
como éste, en quien haya espíritu de Dios? \bibverse{39} Y dijo Faraón á
José: Pues que Dios te ha hecho saber todo esto, no hay entendido ni
sabio como tú: \bibverse{40} Tú serás sobre mi casa, y por tu dicho se
gobernará todo mi pueblo: solamente en el trono seré yo mayor que tú.
\footnote{\textbf{41:40} Ecl 4,14; Sal 113,7; Sal 37,37} \bibverse{41}
Dijo más Faraón á José: He aquí yo te he puesto sobre toda la tierra de
Egipto. \footnote{\textbf{41:41} Hech 7,10} \bibverse{42} Entonces
Faraón quitó su anillo de su mano, y púsolo en la mano de José, é hízole
vestir de ropas de lino finísimo, y puso un collar de oro en su cuello;
\footnote{\textbf{41:42} Est 3,10; Est 8,2; Dan 5,29} \bibverse{43} E
hízolo subir en su segundo carro, y pregonaron delante de él: Doblad la
rodilla: y púsole sobre toda la tierra de Egipto. \bibverse{44} Y dijo
Faraón á José: Yo Faraón; y sin ti ninguno alzará su mano ni su pie en
toda la tierra de Egipto. \bibverse{45} Y llamó Faraón el nombre de
José, Zaphnath-paaneah; y dióle por mujer á Asenath, hija de Potipherah,
sacerdote de On. Y salió José por toda la tierra de Egipto.

\hypertarget{medidas-de-josuxe9-durante-los-siete-auxf1os-fuxe9rtiles-el-nacimiento-de-sus-dos-hijos}{%
\subsection{Medidas de José durante los siete años fértiles; el
nacimiento de sus dos
hijos}\label{medidas-de-josuxe9-durante-los-siete-auxf1os-fuxe9rtiles-el-nacimiento-de-sus-dos-hijos}}

\bibverse{46} Y era José de edad de treinta años cuando fué presentado
delante de Faraón, rey de Egipto: y salió José de delante de Faraón, y
transitó por toda la tierra de Egipto. \bibverse{47} E hizo la tierra en
aquellos siete años de hartura á montones. \bibverse{48} Y él juntó todo
el mantenimiento de los siete años que fueron en la tierra de Egipto, y
guardó mantenimiento en las ciudades, poniendo en cada ciudad el
mantenimiento del campo de sus alrededores. \bibverse{49} Y acopió José
trigo como arena de la mar, mucho en extremo, hasta no poderse contar,
porque no tenía número. \bibverse{50} Y nacieron á José dos hijos antes
que viniese el primer año del hambre, los cuales le parió Asenath, hija
de Potipherah, sacerdote de On. \bibverse{51} Y llamó José el nombre del
primogénito Manasés; porque Dios (dijo) me hizo olvidar todo mi trabajo,
y toda la casa de mi padre. \bibverse{52} Y el nombre del segundo
llamólo Ephraim; porque Dios (dijo) me hizo fértil en la tierra de mi
aflicción.

\hypertarget{los-siete-auxf1os-estuxe9riles-y-las-ventas-de-cereales-de-josuxe9-durante-la-hambruna}{%
\subsection{Los siete años estériles y las ventas de cereales de José
durante la
hambruna}\label{los-siete-auxf1os-estuxe9riles-y-las-ventas-de-cereales-de-josuxe9-durante-la-hambruna}}

\bibverse{53} Y cumpliéronse los siete años de la hartura, que hubo en
la tierra de Egipto. \bibverse{54} Y comenzaron á venir los siete años
del hambre, como José había dicho: y hubo hambre en todos los países,
mas en toda la tierra de Egipto había pan. \bibverse{55} Y cuando se
sintió el hambre en toda la tierra de Egipto, el pueblo clamó á Faraón
por pan. Y dijo Faraón á todos los Egipcios: Id á José, y haced lo que
él os dijere. \bibverse{56} Y el hambre estaba por toda la extensión del
país. Entonces abrió José todo granero donde había, y vendía á los
Egipcios; porque había crecido el hambre en la tierra de Egipto.
\bibverse{57} Y toda la tierra venía á Egipto para comprar de José,
porque por toda la tierra había crecido el hambre.

\hypertarget{los-diez-hijos-mayores-de-jacob-se-mudan-a-egipto-para-comprar-grano}{%
\subsection{Los diez hijos mayores de Jacob se mudan a Egipto para
comprar
grano}\label{los-diez-hijos-mayores-de-jacob-se-mudan-a-egipto-para-comprar-grano}}

\hypertarget{section-41}{%
\section{42}\label{section-41}}

\bibverse{1} Y VIENDO Jacob que en Egipto había alimentos, dijo á sus
hijos: ¿Por qué os estáis mirando? \bibverse{2} Y dijo: He aquí, yo he
oído que hay víveres en Egipto; descended allá, y comprad de allí para
nosotros, para que podamos vivir, y no nos muramos. \bibverse{3} Y
descendieron los diez hermanos de José á comprar trigo á Egipto.
\bibverse{4} Mas Jacob no envió á Benjamín hermano de José con sus
hermanos; porque dijo: No sea acaso que le acontezca algún desastre.
\bibverse{5} Y vinieron los hijos de Israel á comprar entre los que
venían: porque había hambre en la tierra de Canaán.

\hypertarget{la-primera-conversaciuxf3n-dura-de-josuxe9-con-sus-hermanos}{%
\subsection{La primera conversación dura de José con sus
hermanos}\label{la-primera-conversaciuxf3n-dura-de-josuxe9-con-sus-hermanos}}

\bibverse{6} Y José era el señor de la tierra, que vendía á todo el
pueblo de la tierra: y llegaron los hermanos de José, é inclináronse á
él rostro por tierra. \bibverse{7} Y José como vió á sus hermanos,
conociólos; mas hizo que no los conocía, y hablóles ásperamente, y les
dijo: ¿De dónde habéis venido? Ellos respondieron: De la tierra de
Canaán á comprar alimentos.

\bibverse{8} José, pues, conoció á sus hermanos; pero ellos no le
conocieron. \bibverse{9} Entonces se acordó José de los sueños que había
tenido de ellos, y díjoles: Espías sois; por ver lo descubierto del país
habéis venido. \footnote{\textbf{42:9} Gén 37,5-9}

\bibverse{10} Y ellos le respondieron: No, señor mío: mas tus siervos
han venido á comprar alimentos. \bibverse{11} Todos nosotros somos hijos
de un varón: somos hombres de verdad: tus siervos nunca fueron espías.

\bibverse{12} Y él les dijo: No; á ver lo descubierto del país habéis
venido.

\bibverse{13} Y ellos respondieron: Tus siervos somos doce hermanos,
hijos de un varón en la tierra de Canaán; y he aquí el menor está hoy
con nuestro padre, y otro no parece.

\bibverse{14} Y José les dijo: Eso es lo que os he dicho, afirmando que
sois espías: \bibverse{15} En esto seréis probados: Vive Faraón que no
saldréis de aquí, sino cuando vuestro hermano menor aquí viniere.
\bibverse{16} Enviad uno de vosotros, y traiga á vuestro hermano; y
vosotros quedad presos, y vuestras palabras serán probadas, si hay
verdad con vosotros: y si no, vive Faraón, que sois espías.
\bibverse{17} Y juntólos en la cárcel por tres días.

\hypertarget{la-segunda-conversaciuxf3n-simeuxf3n-como-rehuxe9n}{%
\subsection{La segunda conversación: Simeón como
rehén}\label{la-segunda-conversaciuxf3n-simeuxf3n-como-rehuxe9n}}

\bibverse{18} Y al tercer día díjoles José: Haced esto, y vivid: Yo temo
á Dios: \bibverse{19} Si sois hombres de verdad, quede preso en la casa
de vuestra cárcel uno de vuestros hermanos; y vosotros id, llevad el
alimento para el hambre de vuestra casa: \bibverse{20} Pero habéis de
traerme á vuestro hermano menor, y serán verificadas vuestras palabras,
y no moriréis. Y ellos lo hicieron así.

\bibverse{21} Y decían el uno al otro: Verdaderamente hemos pecado
contra nuestro hermano, que vimos la angustia de su alma cuando nos
rogaba, y no le oímos: por eso ha venido sobre nosotros esta angustia.
\bibverse{22} Entonces Rubén les respondió, diciendo: ¿No os hablé yo y
dije: No pequéis contra el mozo; y no escuchasteis? He aquí también su
sangre es requerida. \footnote{\textbf{42:22} Gén 37,21-22}
\bibverse{23} Y ellos no sabían que los entendía José, porque había
intérprete entre ellos. \bibverse{24} Y apartóse él de ellos, y lloró:
después volvió á ellos, y les habló, y tomó de entre ellos á Simeón, y
aprisionóle á vista de ellos.

\hypertarget{regreso-de-los-hermanos-a-canauxe1n}{%
\subsection{Regreso de los hermanos a
Canaán}\label{regreso-de-los-hermanos-a-canauxe1n}}

\bibverse{25} Y mandó José que llenaran sus sacos de trigo, y
devolviesen el dinero de cada uno de ellos, poniéndolo en su saco, y les
diesen comida para el camino: é hízose así con ellos.

\bibverse{26} Y ellos pusieron su trigo sobre sus asnos, y fuéronse de
allí. \bibverse{27} Y abriendo uno de ellos su saco para dar de comer á
su asno en el mesón, vió su dinero que estaba en la boca de su costal.
\bibverse{28} Y dijo á sus hermanos: Mi dinero se me ha devuelto, y aun
helo aquí en mi saco. Sobresaltóseles entonces el corazón, y espantados
dijeron el uno al otro: ¿Qué es esto que nos ha hecho Dios?
\bibverse{29} Y venidos á Jacob su padre en tierra de Canaán, contáronle
todo lo que les había acaecido, diciendo: \bibverse{30} Aquel varón,
señor de la tierra, nos habló ásperamente, y nos trató como á espías de
la tierra: \bibverse{31} Y nosotros le dijimos: Somos hombres de verdad,
nunca fuimos espías: \bibverse{32} Somos doce hermanos, hijos de nuestro
padre; uno no parece, y el menor está hoy con nuestro padre en la tierra
de Canaán. \bibverse{33} Y aquel varón, señor de la tierra, nos dijo: En
esto conoceré que sois hombres de verdad; dejad conmigo uno de vuestros
hermanos, y tomad para el hambre de vuestras casas, y andad,
\bibverse{34} Y traedme á vuestro hermano el menor, para que yo sepa que
no sois espías, sino hombres de verdad: así os daré á vuestro hermano, y
negociaréis en la tierra.

\bibverse{35} Y aconteció que vaciando ellos sus sacos, he aquí que en
el saco de cada uno estaba el atado de su dinero: y viendo ellos y su
padre los atados de su dinero, tuvieron temor. \bibverse{36} Entonces su
padre Jacob les dijo: Habéisme privado de mis hijos; José no parece, ni
Simeón tampoco, y á Benjamín le llevaréis: contra mí son todas estas
cosas.

\bibverse{37} Y Rubén habló á su padre, diciendo: Harás morir á mis dos
hijos, si no te lo volviere; entrégalo en mi mano, que yo lo volveré á
ti.

\bibverse{38} Y él dijo: No descenderá mi hijo con vosotros; que su
hermano es muerto, y él solo ha quedado: y si le aconteciere algún
desastre en el camino por donde vais, haréis descender mis canas con
dolor á la sepultura.

\hypertarget{segundo-viaje-de-los-hermanos-de-josuxe9-a-egipto-con-benjamuxedn}{%
\subsection{Segundo viaje de los hermanos de José a Egipto con
Benjamín}\label{segundo-viaje-de-los-hermanos-de-josuxe9-a-egipto-con-benjamuxedn}}

\hypertarget{section-42}{%
\section{43}\label{section-42}}

\bibverse{1} Y EL hambre era grande en la tierra. \bibverse{2} Y
aconteció que como acabaron de comer el trigo que trajeron de Egipto,
díjoles su padre: Volved, y comprad para nosotros un poco de alimento.

\bibverse{3} Y respondió Judá, diciendo: Aquel varón nos protestó con
ánimo resuelto, diciendo: No veréis mi rostro sin vuestro hermano con
vosotros. \bibverse{4} Si enviares á nuestro hermano con nosotros,
descenderemos y te compraremos alimento: \bibverse{5} Pero si no le
enviares, no descenderemos: porque aquel varón nos dijo: No veréis mi
rostro sin vuestro hermano con vosotros.

\bibverse{6} Y dijo Israel: ¿Por qué me hicisteis tanto mal, declarando
al varón que teníais más hermano?

\bibverse{7} Y ellos respondieron: Aquel varón nos preguntó expresamente
por nosotros, y por nuestra parentela, diciendo: ¿Vive aún vuestro
padre? ¿tenéis otro hermano? y declarámosle conforme á estas palabras.
¿Podíamos nosotros saber que había de decir: Haced venir á vuestro
hermano? \footnote{\textbf{43:7} Gén 42,7-13}

\bibverse{8} Entonces Judá dijo á Israel su padre: Envía al mozo
conmigo, y nos levantaremos é iremos, á fin que vivamos y no muramos
nosotros, y tú, y nuestros niños. \bibverse{9} Yo lo fío; á mí me
pedirás cuenta de él: si yo no te lo volviere y lo pusiere delante de
ti, seré para ti el culpante todos los días: \bibverse{10} Que si no nos
hubiéramos detenido, cierto ahora hubiéramos ya vuelto dos veces.

\bibverse{11} Entonces Israel su padre les respondió: Pues que así es,
hacedlo; tomad de lo mejor de la tierra en vuestros vasos, y llevad á
aquel varón un presente, un poco de bálsamo, y un poco de miel, aromas y
mirra, nueces y almendras. \bibverse{12} Y tomad en vuestras manos
doblado dinero, y llevad en vuestra mano el dinero vuelto en las bocas
de vuestros costales; quizá fué yerro. \footnote{\textbf{43:12} Gén
  42,27; Gén 42,35} \bibverse{13} Tomad también á vuestro hermano, y
levantaos, y volved á aquel varón. \bibverse{14} Y el Dios Omnipotente
os dé misericordias delante de aquel varón, y os suelte al otro vuestro
hermano, y á este Benjamín. Y si he de ser privado de mis hijos, séalo.

\bibverse{15} Entonces tomaron aquellos varones el presente, y tomaron
en su mano doblado dinero, y á Benjamín; y se levantaron, y descendieron
á Egipto, y presentáronse delante de José.

\hypertarget{acogida-amistosa-por-parte-de-josuxe9-de-sus-hermanos}{%
\subsection{Acogida amistosa por parte de José de sus
hermanos}\label{acogida-amistosa-por-parte-de-josuxe9-de-sus-hermanos}}

\bibverse{16} Y vió José á Benjamín con ellos, y dijo al mayordomo de su
casa: Mete en casa á esos hombres, y degüella víctima, y aderézala;
porque estos hombres comerán conmigo al medio día.

\bibverse{17} E hizo el hombre como José dijo; y metió aquel hombre á
los hombres en casa de José. \bibverse{18} Y aquellos hombres tuvieron
temor, cuando fueron metidos en casa de José, y decían: Por el dinero
que fué vuelto en nuestros costales la primera vez nos han metido aquí,
para revolver contra nosotros, y dar sobre nosotros, y tomarnos por
siervos á nosotros, y á nuestros asnos. \footnote{\textbf{43:18} Gén
  42,28} \bibverse{19} Y llegáronse al mayordomo de la casa de José, y
le hablaron á la entrada de la casa. \bibverse{20} Y dijeron: Ay, señor
mío, nosotros en realidad de verdad descendimos al principio á comprar
alimentos: \bibverse{21} Y aconteció que como vinimos al mesón y abrimos
nuestros costales, he aquí el dinero de cada uno estaba en la boca de su
costal, nuestro dinero en su justo peso; y hémoslo vuelto en nuestras
manos. \bibverse{22} Hemos también traído en nuestras manos otro dinero
para comprar alimentos: nosotros no sabemos quién haya puesto nuestro
dinero en nuestros costales.

\bibverse{23} Y él respondió: Paz á vosotros, no temáis; vuestro Dios y
el Dios de vuestro padre os dió el tesoro en vuestros costales: vuestro
dinero vino á mí. Y sacó á Simeón á ellos. \bibverse{24} Y metió aquel
varón á aquellos hombres en casa de José: y dióles agua, y lavaron sus
pies: y dió de comer á sus asnos. \footnote{\textbf{43:24} Gén 18,4}
\bibverse{25} Y ellos prepararon el presente entretanto que venía José
al medio día, porque habían oído que allí habían de comer pan.

\hypertarget{josuxe9-recibe-y-entretiene-a-sus-hermanos-de-la-manera-muxe1s-amistosa}{%
\subsection{José recibe y entretiene a sus hermanos de la manera más
amistosa}\label{josuxe9-recibe-y-entretiene-a-sus-hermanos-de-la-manera-muxe1s-amistosa}}

\bibverse{26} Y vino José á casa, y ellos le trajeron el presente que
tenían en su mano dentro de casa, é inclináronse á él hasta tierra.
\bibverse{27} Entonces les preguntó él cómo estaban, y dijo: ¿Vuestro
padre, el anciano que dijisteis, lo pasa bien? ¿vive todavía?

\bibverse{28} Y ellos respondieron: Bien va á tu siervo nuestro padre;
aun vive. Y se inclinaron, é hicieron reverencia. \footnote{\textbf{43:28}
  Gén 37,7; Gén 37,9} \bibverse{29} Y alzando él sus ojos vió á Benjamín
su hermano, hijo de su madre, y dijo: ¿Es éste vuestro hermano menor, de
quien me hablasteis? Y dijo: Dios tenga misericordia de ti, hijo mío.
\bibverse{30} Entonces José se apresuró, porque se conmovieron sus
entrañas á causa de su hermano, y procuró donde llorar: y entróse en su
cámara, y lloró allí. \bibverse{31} Y lavó su rostro, y salió fuera, y
reprimióse, y dijo: Poned pan.

\bibverse{32} Y pusieron para él aparte, y separadamente para ellos, y
aparte para los Egipcios que con él comían: porque los Egipcios no
pueden comer pan con los Hebreos, lo cual es abominación á los Egipcios.
\bibverse{33} Y sentáronse delante de él, el mayor conforme á su
mayoría, y el menor conforme á su menoría; y estaban aquellos hombres
atónitos mirándose el uno al otro. \bibverse{34} Y él tomó viandas de
delante de sí para ellos; mas la porción de Benjamín era cinco veces
como cualquiera de las de ellos. Y bebieron, y alegráronse con él.

\hypertarget{josuxe9-estuxe1-probando-a-sus-hermanos-por-uxfaltima-vez}{%
\subsection{José está probando a sus hermanos por última
vez}\label{josuxe9-estuxe1-probando-a-sus-hermanos-por-uxfaltima-vez}}

\hypertarget{section-43}{%
\section{44}\label{section-43}}

\bibverse{1} Y MANDÓ José al mayordomo de su casa, diciendo: Hinche los
costales de aquestos varones de alimentos, cuanto pudieren llevar, y pon
el dinero de cada uno en la boca de su costal: \bibverse{2} Y pondrás mi
copa, la copa de plata, en la boca del costal del menor, con el dinero
de su trigo. Y él hizo como dijo José. \bibverse{3} Venida la mañana,
los hombres fueron despedidos con sus asnos. \bibverse{4} Habiendo ellos
salido de la ciudad, de la que aun no se habían alejado, dijo José á su
mayordomo: Levántate, y sigue á esos hombres; y cuando los alcanzares,
diles: ¿Por qué habéis vuelto mal por bien? \bibverse{5} ¿No es ésta en
la que bebe mi señor, y por la que suele adivinar? habéis hecho mal en
lo que hicisteis. \bibverse{6} Y como él los alcanzó, díjoles estas
palabras.

\bibverse{7} Y ellos le respondieron: ¿Por qué dice mi señor tales
cosas? Nunca tal hagan tus siervos. \bibverse{8} He aquí, el dinero que
hallamos en la boca de nuestros costales, te lo volvimos á traer desde
la tierra de Canaán; ¿cómo, pues, habíamos de hurtar de casa de tu señor
plata ni oro? \footnote{\textbf{44:8} Gén 43,22} \bibverse{9} Aquel de
tus siervos en quien fuere hallada la copa, que muera, y aun nosotros
seremos siervos de mi señor.

\bibverse{10} Y él dijo: También ahora sea conforme á vuestras palabras;
aquél en quien se hallare, será mi siervo, y vosotros seréis sin culpa.

\bibverse{11} Ellos entonces se dieron prisa, y derribando cada uno su
costal en tierra, abrió cada cual el costal suyo. \bibverse{12} Y buscó;
desde el mayor comenzó, y acabó en el menor; y la copa fué hallada en el
costal de Benjamín. \bibverse{13} Entonces ellos rasgaron sus vestidos,
y cargó cada uno su asno, y volvieron á la ciudad.

\hypertarget{los-hermanos-regresan-a-la-ciudad-y-se-humillan-ante-josuxe9}{%
\subsection{Los hermanos regresan a la ciudad y se humillan ante
José}\label{los-hermanos-regresan-a-la-ciudad-y-se-humillan-ante-josuxe9}}

\bibverse{14} Y llegó Judá con sus hermanos á casa de José, que aun
estaba allí, y postráronse delante de él en tierra. \bibverse{15} Y
díjoles José: ¿Qué obra es esta que habéis hecho? ¿no sabéis que un
hombre como yo sabe adivinar?

\bibverse{16} Entonces dijo Judá: ¿Qué diremos á mi señor? ¿qué
hablaremos? ¿ó con qué nos justificaremos? Dios ha hallado la maldad de
tus siervos: he aquí, nosotros somos siervos de mi señor, nosotros, y
también aquél en cuyo poder fué hallada la copa. \footnote{\textbf{44:16}
  Gén 42,21-22; Lam 1,14}

\bibverse{17} Y él respondió: Nunca yo tal haga: el varón en cuyo poder
fué hallada la copa, él será mi siervo; vosotros id en paz á vuestro
padre.

\bibverse{18} Entonces Judá se llegó á él, y dijo: Ay señor mío, ruégote
que hable tu siervo una palabra en oídos de mi señor, y no se encienda
tu enojo contra tu siervo, pues que tú eres como Faraón. \bibverse{19}
Mi señor preguntó á sus siervos, diciendo: ¿Tenéis padre ó hermano?
\bibverse{20} Y nosotros respondimos á mi señor: Tenemos un padre
anciano, y un mozo que le nació en su vejez, pequeño aún; y un hermano
suyo murió, y él quedó solo de su madre, y su padre lo ama.
\bibverse{21} Y tú dijiste á tus siervos: Traédmelo, y pondré mis ojos
sobre él. \bibverse{22} Y nosotros dijimos á mi señor: El mozo no puede
dejar á su padre, porque si le dejare, su padre morirá. \bibverse{23} Y
dijiste á tus siervos: Si vuestro hermano menor no descendiere con
vosotros, no veáis más mi rostro. \footnote{\textbf{44:23} Gén 42,15;
  Gén 43,3-5} \bibverse{24} Aconteció pues, que como llegamos á mi padre
tu siervo, contámosle las palabras de mi señor. \bibverse{25} Y dijo
nuestro padre: Volved á comprarnos un poco de alimento. \bibverse{26} Y
nosotros respondimos: No podemos ir: si nuestro hermano fuere con
nosotros, iremos; porque no podemos ver el rostro del varón, no estando
con nosotros nuestro hermano el menor. \bibverse{27} Entonces tu siervo
mi padre nos dijo: Vosotros sabéis que dos me parió mi mujer;
\bibverse{28} Y el uno salió de conmigo, y pienso de cierto que fué
despedazado, y hasta ahora no le he visto; \bibverse{29} Y si tomareis
también éste de delante de mí, y le aconteciere algún desastre, haréis
descender mis canas con dolor á la sepultura. \footnote{\textbf{44:29}
  Gén 42,38} \bibverse{30} Ahora, pues, cuando llegare yo á tu siervo mi
padre, y el mozo no fuere conmigo, como su alma está ligada al alma de
él, \bibverse{31} Sucederá que cuando no vea al mozo, morirá: y tus
siervos harán descender las canas de tu siervo nuestro padre con dolor á
la sepultura. \bibverse{32} Como tu siervo salió por fiador del mozo con
mi padre, diciendo: Si no te lo volviere, entonces yo seré culpable para
mi padre todos los días; \bibverse{33} Ruégote por tanto que quede ahora
tu siervo por el mozo por siervo de mi señor, y que el mozo vaya con sus
hermanos. \bibverse{34} Porque ¿cómo iré yo á mi padre sin el mozo? No
podré, por no ver el mal que sobrevendrá á mi padre.

\hypertarget{josuxe9-se-revela-a-sus-hermanos}{%
\subsection{José se revela a sus
hermanos}\label{josuxe9-se-revela-a-sus-hermanos}}

\hypertarget{section-44}{%
\section{45}\label{section-44}}

\bibverse{1} NO podía ya José contenerse delante de todos los que
estaban al lado suyo, y clamó: Haced salir de conmigo á todos. Y no
quedó nadie con él, al darse á conocer José á sus hermanos. \bibverse{2}
Entonces se dió á llorar á voz en grito; y oyeron los Egipcios, y oyó
también la casa de Faraón. \bibverse{3} Y dijo José á sus hermanos: Yo
soy José: ¿vive aún mi padre? Y sus hermanos no pudieron responderle,
porque estaban turbados delante de él.

\bibverse{4} Entonces dijo José á sus hermanos: Llegaos ahora á mí. Y
ellos se llegaron. Y él dijo: Yo soy José vuestro hermano el que
vendisteis para Egipto. \footnote{\textbf{45:4} Gén 37,28}

\bibverse{5} Ahora pues, no os entristezcáis, ni os pese de haberme
vendido acá; que para preservación de vida me envió Dios delante de
vosotros: \footnote{\textbf{45:5} Gén 50,20}

\hypertarget{josuxe9-fue-enviado-por-dios-para-la-salvaciuxf3n-de-israel}{%
\subsection{José fue enviado por Dios para la salvación de
Israel}\label{josuxe9-fue-enviado-por-dios-para-la-salvaciuxf3n-de-israel}}

\bibverse{6} Que ya ha habido dos años de hambre en medio de la tierra,
y aun quedan cinco años en que ni habrá arada ni siega. \bibverse{7} Y
Dios me envió delante de vosotros, para que vosotros quedaseis en la
tierra, y para daros vida por medio de grande salvamento. \bibverse{8}
Así pues, no me enviasteis vosotros acá, sino Dios, que me ha puesto por
padre de Faraón, y por señor de toda su casa, y por gobernador en toda
la tierra de Egipto. \footnote{\textbf{45:8} Gén 41,40-43} \bibverse{9}
Daos priesa, id á mi padre y decidle: Así dice tu hijo José: Dios me ha
puesto por señor de todo Egipto; ven á mí, no te detengas: \bibverse{10}
Y habitarás en la tierra de Gosén, y estarás cerca de mí, tú y tus
hijos, y los hijos de tus hijos, tus ganados y tus vacas, y todo lo que
tienes. \bibverse{11} Y allí te alimentaré, pues aun quedan cinco años
de hambre, porque no perezcas de pobreza tú y tu casa, y todo lo que
tienes: \bibverse{12} Y he aquí, vuestros ojos ven, y los ojos de mi
hermano Benjamín, que mi boca os habla. \bibverse{13} Haréis pues saber
á mi padre toda mi gloria en Egipto, y todo lo que habéis visto: y daos
priesa, y traed á mi padre acá. \bibverse{14} Y echóse sobre el cuello
de Benjamín su hermano, y lloró; y también Benjamín lloró sobre su
cuello. \bibverse{15} Y besó á todos sus hermanos, y lloró sobre ellos:
y después sus hermanos hablaron con él.

\hypertarget{la-amable-invitaciuxf3n-del-farauxf3n-a-jacob-para-que-se-mudara-a-egipto}{%
\subsection{La amable invitación del faraón a Jacob para que se mudara a
Egipto}\label{la-amable-invitaciuxf3n-del-farauxf3n-a-jacob-para-que-se-mudara-a-egipto}}

\bibverse{16} Y oyóse la noticia en la casa de Faraón, diciendo: Los
hermanos de José han venido. Y plugo en los ojos de Faraón y de sus
siervos. \bibverse{17} Y dijo Faraón á José: Di á tus hermanos: Haced
esto: cargad vuestras bestias, é id, volved á la tierra de Canaán;
\bibverse{18} Y tomad á vuestro padre y vuestras familias, y venid á mí,
que yo os daré lo bueno de la tierra de Egipto y comeréis la grosura de
la tierra. \bibverse{19} Y tú manda: Haced esto: tomaos de la tierra de
Egipto carros para vuestros niños y vuestras mujeres; y tomad á vuestro
padre, y venid. \bibverse{20} Y no se os dé nada de vuestras alhajas,
porque el bien de la tierra de Egipto será vuestro.

\hypertarget{josuxe9-da-obsequios-generosos-a-sus-hermanos-que-regresan-a-casa-y-los-amonesta-con-amor}{%
\subsection{José da obsequios generosos a sus hermanos que regresan a
casa y los amonesta con
amor}\label{josuxe9-da-obsequios-generosos-a-sus-hermanos-que-regresan-a-casa-y-los-amonesta-con-amor}}

\bibverse{21} E hiciéronlo así los hijos de Israel: y dióles José carros
conforme á la orden de Faraón, y suministróles víveres para el camino.
\bibverse{22} A cada uno de todos ellos dió mudas de vestidos, y á
Benjamín dió trescientas piezas de plata, y cinco mudas de vestidos.
\bibverse{23} Y á su padre envió esto: diez asnos cargados de lo mejor
de Egipto, y diez asnas cargadas de trigo, y pan y comida, para su padre
en el camino. \bibverse{24} Y despidió á sus hermanos, y fuéronse. Y él
les dijo: No riñáis por el camino.

\hypertarget{jacob-se-muda-a-su-hijo-en-egipto}{%
\subsection{Jacob se muda a su hijo en
Egipto}\label{jacob-se-muda-a-su-hijo-en-egipto}}

\bibverse{25} Y subieron de Egipto, y llegaron á la tierra de Canaán á
Jacob su padre. \bibverse{26} Y diéronle las nuevas, diciendo: José vive
aún; y él es señor en toda la tierra de Egipto. Y su corazón se desmayó;
pues no los creía. \bibverse{27} Y ellos le contaron todas las palabras
de José, que él les había hablado; y viendo él los carros que José
enviaba para llevarlo, el espíritu de Jacob su padre revivió.
\bibverse{28} Entonces dijo Israel: Basta; José mi hijo vive todavía:
iré, y le veré antes que yo muera. \footnote{\textbf{45:28} Gén 46,30}

\hypertarget{dios-aprueba-el-traslado-de-jacob-a-beerseba-en-una-revelaciuxf3n}{%
\subsection{Dios aprueba el traslado de Jacob a Beerseba en una
revelación}\label{dios-aprueba-el-traslado-de-jacob-a-beerseba-en-una-revelaciuxf3n}}

\hypertarget{section-45}{%
\section{46}\label{section-45}}

\bibverse{1} Y PARTIÓSE Israel con todo lo que tenía, y vino á
Beer-seba, y ofreció sacrificios al Dios de su padre Isaac. \footnote{\textbf{46:1}
  Gén 26,23-25} \bibverse{2} Y habló Dios á Israel en visiones de noche,
y dijo: Jacob, Jacob. Y él respondió: Heme aquí.

\bibverse{3} Y dijo: Yo soy Dios, el Dios de tu padre; no temas de
descender á Egipto, porque yo te pondré allí en gran gente. \bibverse{4}
Yo descenderé contigo á Egipto, y yo también te haré volver: y José
pondrá su mano sobre tus ojos.

\bibverse{5} Y levantóse Jacob de Beer-seba; y tomaron los hijos de
Israel á su padre Jacob, y á sus niños, y á sus mujeres, en los carros
que Faraón había enviado para llevarlo. \bibverse{6} Y tomaron sus
ganados, y su hacienda que había adquirido en la tierra de Canaán, y
viniéronse á Egipto, Jacob, y toda su simiente consigo; \bibverse{7} Sus
hijos, y los hijos de sus hijos consigo; sus hijas, y las hijas de sus
hijos, y á toda su simiente trajo consigo á Egipto.

\hypertarget{el-linaje-de-toda-la-familia-de-jacob}{%
\subsection{El linaje de toda la familia de
Jacob}\label{el-linaje-de-toda-la-familia-de-jacob}}

\bibverse{8} Y estos son los nombres de los hijos de Israel, que
entraron en Egipto, Jacob y sus hijos: Rubén, el primogénito de Jacob.
\footnote{\textbf{46:8} Éxod 6,14-16} \bibverse{9} Y los hijos de Rubén:
Hanoch, y Phallu, y Hezrón, y Carmi. \bibverse{10} Y los hijos de
Simeón: Jemuel, y Jamín, y Ohad, y Jachîn, y Zohar, y Saúl, hijo de la
Cananea. \bibverse{11} Y los hijos de Leví: Gersón, y Coath, y Merari.
\bibverse{12} Y los hijos de Judá: Er, y Onán, y Sela, y Phares, y Zara:
mas Er y Onán, murieron en la tierra de Canaán. Y los hijos de Phares
fueron Hezrón y Hamul. \bibverse{13} Y los hijos de Issachâr: Thola, y
Phua, y Job, y Simrón. \bibverse{14} Y los hijos de Zabulón: Sered, y
Elón, y Jahleel. \bibverse{15} Estos fueron los hijos de Lea, los que
parió á Jacob en Padan-aram, y además su hija Dina: treinta y tres las
almas todas de sus hijos é hijas. \bibverse{16} Y los hijos de Gad:
Ziphión, y Aggi, y Ezbón, y Suni, y Heri, y Arodi, y Areli.
\bibverse{17} Y los hijos de Aser: Jimna, é Ishua, é Isui, y Beria, y
Sera, hermana de ellos. Los hijos de Beria: Heber, y Malchîel.
\bibverse{18} Estos fueron los hijos de Zilpa, la que Labán dió á su
hija Lea, y parió estos á Jacob; todas diez y seis almas. \bibverse{19}
Y los hijos de Rachêl, mujer de Jacob: José y Benjamín. \bibverse{20} Y
nacieron á José en la tierra de Egipto Manasés y Ephraim, los que le
parió Asenath, hija de Potipherah, sacerdote de On. \footnote{\textbf{46:20}
  Gén 41,50-52} \bibverse{21} Y los hijos de Benjamín fueron Bela, y
Bechêr y Asbel, y Gera, y Naamán, y Ehi, y Ros y Muppim, y Huppim, y
Ard. \bibverse{22} Estos fueron los hijos de Rachêl, que nacieron á
Jacob: en todas, catorce almas. \bibverse{23} Y los hijos de Dan: Husim.
\bibverse{24} Y los hijos de Nephtalí: Jahzeel, y Guni, y Jezer, y
Shillem. \bibverse{25} Estos fueron los hijos de Bilha, la que dió Labán
á Rachêl su hija, y parió estos á Jacob; todas siete almas.
\bibverse{26} Todas las personas que vinieron con Jacob á Egipto,
procedentes de sus lomos, sin las mujeres de los hijos de Jacob, todas
las personas fueron sesenta y seis. \bibverse{27} Y los hijos de José,
que le nacieron en Egipto, dos personas. Todas las almas de la casa de
Jacob, que entraron en Egipto, fueron setenta.

\hypertarget{josuxe9-saluda-a-su-padre-en-gosen}{%
\subsection{José saluda a su padre en
Gosen}\label{josuxe9-saluda-a-su-padre-en-gosen}}

\bibverse{28} Y envió á Judá delante de sí á José, para que le viniese á
ver á Gosén; y llegaron á la tierra de Gosén. \footnote{\textbf{46:28}
  Gén 45,10} \bibverse{29} Y José unció su carro y vino á recibir á
Israel su padre á Gosén; y se manifestó á él, y echóse sobre su cuello,
y lloró sobre su cuello bastante. \bibverse{30} Entonces Israel dijo á
José: Muera yo ahora, ya que he visto tu rostro, pues aun vives.

\bibverse{31} Y José dijo á sus hermanos, y á la casa de su padre:
Subiré y haré saber á Faraón, y diréle: Mis hermanos y la casa de mi
padre, que estaban en la tierra de Canaán, han venido á mí;
\bibverse{32} Y los hombres son pastores de ovejas, porque son hombres
ganaderos: y han traído sus ovejas y sus vacas, y todo lo que tenían.
\bibverse{33} Y cuando Faraón os llamare y dijere: ¿cuál es vuestro
oficio? \bibverse{34} Entonces diréis: Hombres de ganadería han sido tus
siervos desde nuestra mocedad hasta ahora, nosotros y nuestros padres; á
fin que moréis en la tierra de Gosén, porque los Egipcios abominan todo
pastor de ovejas. \footnote{\textbf{46:34} Gén 43,32}

\hypertarget{el-farauxf3n-promete-a-los-hijos-de-jacob-establecerse-en-gosen}{%
\subsection{El faraón promete a los hijos de Jacob establecerse en
Gosen}\label{el-farauxf3n-promete-a-los-hijos-de-jacob-establecerse-en-gosen}}

\hypertarget{section-46}{%
\section{47}\label{section-46}}

\bibverse{1} Y JOSÉ vino, é hizo saber á Faraón, y dijo: Mi padre y mis
hermanos, y sus ovejas y sus vacas, con todo lo que tienen, han venido
de la tierra de Canaán, y he aquí, están en la tierra de Gosén.
\bibverse{2} Y de los postreros de sus hermanos tomó cinco varones, y
presentólos delante de Faraón. \bibverse{3} Y Faraón dijo á sus
hermanos: ¿Cuál es vuestro oficio? Y ellos respondieron á Faraón:
Pastores de ovejas son tus siervos, así nosotros como nuestros padres.

\bibverse{4} Dijeron además á Faraón: Por morar en esta tierra hemos
venido; porque no hay pasto para las ovejas de tus siervos, pues el
hambre es grave en la tierra de Canaán: por tanto, te rogamos ahora que
habiten tus siervos en la tierra de Gosén.

\bibverse{5} Entonces Faraón habló á José, diciendo: Tu padre y tus
hermanos han venido á ti; \bibverse{6} La tierra de Egipto delante de ti
está; en lo mejor de la tierra haz habitar á tu padre y á tus hermanos;
habiten en la tierra de Gosén; y si entiendes que hay entre ellos
hombres eficaces, ponlos por mayorales del ganado mío.

\hypertarget{jacob-se-presentuxf3-al-farauxf3n-y-luego-se-instaluxf3-en-gosen}{%
\subsection{Jacob se presentó al faraón y luego se instaló en
Gosen}\label{jacob-se-presentuxf3-al-farauxf3n-y-luego-se-instaluxf3-en-gosen}}

\bibverse{7} Y José introdujo á su padre, y presentólo delante de
Faraón; y Jacob bendijo á Faraón. \bibverse{8} Y dijo Faraón á Jacob:
¿Cuántos son los días de los años de tu vida?

\bibverse{9} Y Jacob respondió á Faraón: Los días de los años de mi
peregrinación son ciento treinta años; pocos y malos han sido los días
de los años de mi vida, y no han llegado á los días de los años de la
vida de mis padres en los días de su peregrinación. \footnote{\textbf{47:9}
  Sal 90,10; Sal 39,13} \bibverse{10} Y Jacob bendijo á Faraón, y
salióse de delante de Faraón.

\bibverse{11} Así José hizo habitar á su padre y á sus hermanos, y
dióles posesión en la tierra de Egipto, en lo mejor de la tierra, en la
tierra de Rameses como mandó Faraón. \bibverse{12} Y alimentaba José á
su padre y á sus hermanos, y á toda la casa de su padre, de pan, hasta
la boca del niño.

\hypertarget{josuxe9-compra-la-tierra-para-el-farauxf3n}{%
\subsection{José compra la tierra para el
faraón}\label{josuxe9-compra-la-tierra-para-el-farauxf3n}}

\bibverse{13} Y no había pan en toda la tierra, y el hambre era muy
grave; por lo que desfalleció de hambre la tierra de Egipto y la tierra
de Canaán. \bibverse{14} Y recogió José todo el dinero que se halló en
la tierra de Egipto y en la tierra de Canaán, por los alimentos que de
él compraban; y metió José el dinero en casa de Faraón. \bibverse{15} Y
acabado el dinero de la tierra de Egipto y de la tierra de Canaán, vino
todo Egipto á José diciendo: Danos pan: ¿por qué moriremos delante de
ti, por haberse acabado el dinero?

\bibverse{16} Y José dijo: Dad vuestros ganados, y yo os daré por
vuestros ganados, si se ha acabado el dinero.

\bibverse{17} Y ellos trajeron sus ganados á José; y José les dió
alimentos por caballos, y por el ganado de las ovejas, y por el ganado
de las vacas, y por asnos: y sustentólos de pan por todos sus ganados
aquel año. \bibverse{18} Y acabado aquel año, vinieron á él el segundo
año, y le dijeron: No encubriremos á nuestro señor que el dinero
ciertamente se ha acabado; también el ganado es ya de nuestro señor;
nada ha quedado delante de nuestro señor sino nuestros cuerpos y nuestra
tierra. \bibverse{19} ¿Por qué moriremos delante de tus ojos, así
nosotros como nuestra tierra? Cómpranos á nosotros y á nuestra tierra
por pan, y seremos nosotros y nuestra tierra siervos de Faraón: y danos
simiente para que vivamos y no muramos, y no sea asolada la tierra.

\bibverse{20} Entonces compró José toda la tierra de Egipto para Faraón;
pues los Egipcios vendieron cada uno sus tierras, porque se agravó el
hambre sobre ellos: y la tierra vino á ser de Faraón. \bibverse{21} Y al
pueblo hízolo pasar á las ciudades desde el un cabo del término de
Egipto hasta el otro cabo. \bibverse{22} Solamente la tierra de los
sacerdotes no compró, por cuanto los sacerdotes tenían ración de Faraón,
y ellos comían su ración que Faraón les daba: por eso no vendieron su
tierra. \bibverse{23} Y José dijo al pueblo: He aquí os he hoy comprado
y á vuestra tierra para Faraón: ved aquí simiente, y sembraréis la
tierra. \bibverse{24} Y será que de los frutos daréis el quinto á
Faraón, y las cuatro partes serán vuestras para sembrar las tierras, y
para vuestro mantenimiento, y de los que están en vuestras casas, y para
que coman vuestros niños.

\bibverse{25} Y ellos respondieron: La vida nos has dado: hallemos
gracia en ojos de mi señor, y seamos siervos de Faraón.

\bibverse{26} Entonces José lo puso por fuero hasta hoy sobre la tierra
de Egipto, señalando para Faraón el quinto; excepto sólo la tierra de
los sacerdotes, que no fué de Faraón.

\hypertarget{feliz-situaciuxf3n-para-los-israelitas-en-egipto-el-uxfaltimo-deseo-de-jacob-con-respecto-a-su-funeral}{%
\subsection{Feliz situación para los israelitas en Egipto; El último
deseo de Jacob con respecto a su
funeral}\label{feliz-situaciuxf3n-para-los-israelitas-en-egipto-el-uxfaltimo-deseo-de-jacob-con-respecto-a-su-funeral}}

\bibverse{27} Así habitó Israel en la tierra de Egipto, en la tierra de
Gosén; y aposesionáronse en ella, y se aumentaron, y multiplicaron en
gran manera. \footnote{\textbf{47:27} Gén 46,3; Éxod 1,7; Éxod 1,12}
\bibverse{28} Y vivió Jacob en la tierra de Egipto diecisiete años: y
fueron los días de Jacob, los años de su vida, ciento cuarenta y siete
años. \bibverse{29} Y llegáronse los días de Israel para morir, y llamó
á José su hijo, y le dijo: Si he hallado ahora gracia en tus ojos,
ruégote que pongas tu mano debajo de mi muslo, y harás conmigo
misericordia y verdad; ruégote que no me entierres en Egipto;
\bibverse{30} Mas cuando durmiere con mis padres, llevarme has de
Egipto, y me sepultarás en el sepulcro de ellos. Y él respondió: Yo haré
como tú dices. \footnote{\textbf{47:30} Gén 25,9-10; Gén 49,29-32}

\bibverse{31} Y él dijo: Júramelo. Y él le juró. Entonces Israel se
inclinó sobre la cabecera de la cama.

\hypertarget{jacob-toma-a-los-dos-hijos-de-josuxe9-en-lugar-de-niuxf1os}{%
\subsection{Jacob toma a los dos hijos de José en lugar de
niños}\label{jacob-toma-a-los-dos-hijos-de-josuxe9-en-lugar-de-niuxf1os}}

\hypertarget{section-47}{%
\section{48}\label{section-47}}

\bibverse{1} Y SUCEDIÓ después de estas cosas el haberse dicho á José:
He aquí tu padre está enfermo. Y él tomó consigo sus dos hijos Manasés y
Ephraim. \bibverse{2} Y se hizo saber á Jacob, diciendo: He aquí tu hijo
José viene á ti. Entonces se esforzó Israel, y sentóse sobre la cama;
\bibverse{3} Y dijo á José: El Dios Omnipotente me apareció en Luz en la
tierra de Canaán, y me bendijo, \bibverse{4} Y díjome: He aquí, yo te
haré crecer, y te multiplicaré, y te pondré por estirpe de pueblos: y
daré esta tierra á tu simiente después de ti por heredad perpetua.
\footnote{\textbf{48:4} Gén 35,11-12} \bibverse{5} Y ahora tus dos hijos
Ephraim y Manasés, que te nacieron en la tierra de Egipto, antes que
viniese á ti á la tierra de Egipto, míos son; como Rubén y Simeón, serán
míos: \footnote{\textbf{48:5} Gén 41,50-52} \bibverse{6} Y los que
después de ellos has engendrado, serán tuyos; por el nombre de sus
hermanos serán llamados en sus heredades. \bibverse{7} Porque cuando yo
venía de Padan-aram, se me murió Rachêl en la tierra de Canaán, en el
camino, como media legua de tierra viniendo á Ephrata; y sepultéla allí
en el camino de Ephrata, que es Bethlehem. \footnote{\textbf{48:7} Gén
  35,19}

\hypertarget{jacob-bendice-a-los-dos-hijos-de-josuxe9}{%
\subsection{Jacob bendice a los dos hijos de
José}\label{jacob-bendice-a-los-dos-hijos-de-josuxe9}}

\bibverse{8} Y vió Israel los hijos de José, y dijo: ¿Quiénes son éstos?

\bibverse{9} Y respondió José á su padre: Son mis hijos, que Dios me ha
dado aquí. Y él dijo: Allégalos ahora á mí, y los bendeciré.

\bibverse{10} Y los ojos de Israel estaban tan agravados de la vejez,
que no podía ver. Hízoles, pues, llegar á él, y él los besó y abrazó.
\bibverse{11} Y dijo Israel á José: No pensaba yo ver tu rostro, y he
aquí Dios me ha hecho ver también tu simiente. \footnote{\textbf{48:11}
  Gén 37,33; Gén 37,35; Gén 45,26; Sal 128,6} \bibverse{12} Entonces
José los sacó de entre sus rodillas, é inclinóse á tierra. \bibverse{13}
Y tomólos José á ambos, Ephraim á su diestra, á la siniestra de Israel;
y á Manasés á su izquierda, á la derecha de Israel; é hízoles llegar á
él. \bibverse{14} Entonces Israel extendió su diestra, y púsola sobre la
cabeza de Ephraim, que era el menor, y su siniestra sobre la cabeza de
Manasés, colocando así sus manos adrede, aunque Manasés era el
primogénito. \bibverse{15} Y bendijo á José, y dijo: El Dios en cuya
presencia anduvieron mis padres Abraham é Isaac, el Dios que me mantiene
desde que yo soy hasta este día, \bibverse{16} El Angel que me liberta
de todo mal, bendiga á estos mozos: y mi nombre sea llamado en ellos, y
el nombre de mis padres Abraham é Isaac: y multipliquen en gran manera
en medio de la tierra. \footnote{\textbf{48:16} Gén 31,11-13}

\bibverse{17} Entonces viendo José que su padre ponía la mano derecha
sobre la cabeza de Ephraim, causóle esto disgusto; y asió la mano de su
padre, para mudarla de sobre la cabeza de Ephraim á la cabeza de
Manasés. \bibverse{18} Y dijo José á su padre: No así, padre mío, porque
éste es el primogénito; pon tu diestra sobre su cabeza.

\bibverse{19} Mas su padre no quiso, y dijo: Lo sé, hijo mío, lo sé:
también él vendrá á ser un pueblo, y será también acrecentado; pero su
hermano menor será más grande que él, y su simiente será plenitud de
gentes. \bibverse{20} Y bendíjolos aquel día, diciendo: En ti bendecirá
Israel, diciendo: Póngate Dios como á Ephraim y como á Manasés. Y puso á
Ephraim delante de Manasés. \footnote{\textbf{48:20} Heb 11,21}
\bibverse{21} Y dijo Israel á José: He aquí, yo muero, mas Dios será con
vosotros, y os hará volver á la tierra de vuestros padres. \bibverse{22}
Y yo te he dado á ti una parte sobre tus hermanos, la cual tomé yo de
mano del Amorrheo con mi espada y con mi arco.

\hypertarget{las-profecuxedas-de-jacob-sobre-sus-hijos}{%
\subsection{Las profecías de Jacob sobre sus
hijos}\label{las-profecuxedas-de-jacob-sobre-sus-hijos}}

\hypertarget{section-48}{%
\section{49}\label{section-48}}

\bibverse{1} Y LLAMÓ Jacob á sus hijos, y dijo: Juntaos, y os declararé
lo que os ha de acontecer en los postreros días. \bibverse{2} Juntaos y
oid, hijos de Jacob; y escuchad á vuestro padre Israel.

\bibverse{3} Rubén, tú eres mi primogénito, mi fortaleza, y el principio
de mi vigor; principal en dignidad, principal en poder. \footnote{\textbf{49:3}
  Gén 29,32; Deut 21,17} \bibverse{4} Corriente como las aguas, no seas
el principal; por cuanto subiste al lecho de tu padre: entonces te
envileciste, subiendo á mi estrado. \footnote{\textbf{49:4} Gén 35,22}

\bibverse{5} Simeón y Leví, hermanos: armas de iniquidad sus armas.
\bibverse{6} En su secreto no entre mi alma, ni mi honra se junte en su
compañía; que en su furor mataron varón, y en su voluntad arrancaron
muro. \footnote{\textbf{49:6} Sal 16,9; Sal 30,13; Gén 34,25}
\bibverse{7} Maldito su furor, que fué fiero; y su ira, que fué dura: yo
los apartaré en Jacob, y los esparciré en Israel. \footnote{\textbf{49:7}
  Jos 19,1-9; Jos 21,1-42}

\bibverse{8} Judá, alabarte han tus hermanos: tu mano en la cerviz de
tus enemigos: los hijos de tu padre se inclinarán á ti. \footnote{\textbf{49:8}
  Núm 10,14; Jue 1,1-2} \bibverse{9} Cachorro de león Judá: de la presa
subiste, hijo mío: encorvóse, echóse como león, así como león viejo;
¿quién lo despertará? \footnote{\textbf{49:9} Núm 23,24; Apoc 5,5}
\bibverse{10} No será quitado el cetro de Judá, y el legislador de entre
sus piés, hasta que venga Shiloh; y á él se congregarán los pueblos.
\footnote{\textbf{49:10} Núm 24,17; 1Cró 5,2; Heb 7,14} \bibverse{11}
Atando á la vid su pollino, y á la cepa el hijo de su asna, lavó en el
vino su vestido, y en la sangre de uvas su manto: \footnote{\textbf{49:11}
  Jl 4,18} \bibverse{12} Sus ojos bermejos del vino, y los dientes
blancos de la leche.

\bibverse{13} Zabulón en puertos de mar habitará, y será para puerto de
navíos; y su término hasta Sidón. \footnote{\textbf{49:13} Jos 19,10-16}

\bibverse{14} Issachâr, asno huesudo echado entre dos tercios:
\bibverse{15} Y vió que el descanso era bueno, y que la tierra era
deleitosa; y bajó su hombro para llevar, y sirvió en tributo.

\bibverse{16} Dan juzgará á su pueblo, como una de las tribus de Israel.
\bibverse{17} Será Dan serpiente junto al camino, cerasta junto á la
senda, que muerde los talones de los caballos, y hace caer por detrás al
cabalgador de ellos. \bibverse{18} Tu salud esperé, oh Jehová.
\footnote{\textbf{49:18} Sal 119,166; Hab 2,3}

\bibverse{19} Gad, ejército lo acometerá; mas él acometerá al fin.

\bibverse{20} El pan de Aser será grueso, y él dará deleites al rey.

\bibverse{21} Nephtalí, sierva dejada, que dará dichos hermosos.
\footnote{\textbf{49:21} Jue 4,6-10}

\bibverse{22} Ramo fructífero José, ramo fructífero junto á fuente,
cuyos vástagos se extienden sobre el muro. \footnote{\textbf{49:22} Os
  13,15} \bibverse{23} Y causáronle amargura, y asaeteáronle, y
aborreciéronle los archeros: \bibverse{24} Mas su arco quedó en
fortaleza, y los brazos de sus manos se corroboraron por las manos del
Fuerte de Jacob, (de allí el pastor, y la piedra de Israel,)
\bibverse{25} Del Dios de tu padre, el cual te ayudará, y del
Omnipotente, el cual te bendecirá con bendiciones de los cielos de
arriba, con bendiciones del abismo que está abajo, con bendiciones del
seno y de la matriz. \bibverse{26} Las bendiciones de tu padre fueron
mayores que las bendiciones de mis progenitores: hasta el término de los
collados eternos serán sobre la cabeza de José, y sobre la mollera del
Nazareo de sus hermanos. \footnote{\textbf{49:26} Gén 45,8}

\bibverse{27} Benjamín, lobo arrebatador: á la mañana comerá la presa, y
á la tarde repartirá los despojos. \footnote{\textbf{49:27} Jue 20,25;
  1Sam 9,1-2}

\hypertarget{la-solicitud-de-jacob-para-su-entierro-en-hebruxf3n}{%
\subsection{La solicitud de Jacob para su entierro en
Hebrón}\label{la-solicitud-de-jacob-para-su-entierro-en-hebruxf3n}}

\bibverse{28} Todos estos fueron las doce tribus de Israel: y esto fué
lo que su padre les dijo, y bendíjolos; á cada uno por su bendición los
bendijo. \bibverse{29} Mandóles luego, y díjoles: Yo voy á ser reunido
con mi pueblo: sepultadme con mis padres en la cueva que está en el
campo de Ephrón el Hetheo; \footnote{\textbf{49:29} Gén 23,16-20; Gén
  47,30} \bibverse{30} En la cueva que está en el campo de Macpela, que
está delante de Mamre en la tierra de Canaán, la cual compró Abraham con
el mismo campo de Ephrón el Hetheo, para heredad de sepultura.
\bibverse{31} Allí sepultaron á Abraham y á Sara su mujer; allí
sepultaron á Isaac y á Rebeca su mujer; allí también sepulté yo á Lea.
\bibverse{32} La compra del campo y de la cueva que está en él, fué de
los hijos de Heth. \bibverse{33} Y como acabó Jacob de dar órdenes á sus
hijos, encogió sus pies en la cama, y espiró: y fué reunido con sus
padres.

\hypertarget{embalsamamiento-y-traslado-solemne-de-jacob-despuuxe9s-del-entierro-hereditario-en-hebruxf3n}{%
\subsection{Embalsamamiento y traslado solemne de Jacob después del
entierro hereditario en
Hebrón}\label{embalsamamiento-y-traslado-solemne-de-jacob-despuuxe9s-del-entierro-hereditario-en-hebruxf3n}}

\hypertarget{section-49}{%
\section{50}\label{section-49}}

\bibverse{1} ENTONCES se echó José sobre el rostro de su padre, y lloró
sobre él, y besólo. \footnote{\textbf{50:1} Gén 46,4} \bibverse{2} Y
mandó José á sus médicos familiares que embalsamasen á su padre: y los
médicos embalsamaron á Israel. \bibverse{3} Y cumpliéronle cuarenta
días, porque así cumplían los días de los embalsamados, y lloráronlo los
Egipcios setenta días.

\bibverse{4} Y pasados los días de su luto, habló José á los de la casa
de Faraón, diciendo: Si he hallado ahora gracia en vuestros ojos, os
ruego que habléis en oídos de Faraón, diciendo: \bibverse{5} Mi padre me
conjuró diciendo: He aquí yo muero; en mi sepulcro que yo cavé para mí
en la tierra de Canaán, allí me sepultarás; ruego pues que vaya yo
ahora, y sepultaré á mi padre, y volveré.

\bibverse{6} Y Faraón dijo: Ve, y sepulta á tu padre, como él te
conjuró.

\bibverse{7} Entonces José subió á sepultar á su padre; y subieron con
él todos los siervos de Faraón, los ancianos de su casa, y todos los
ancianos de la tierra de Egipto, \bibverse{8} Y toda la casa de José, y
sus hermanos, y la casa de su padre: solamente dejaron en la tierra de
Gosén sus niños, y sus ovejas y sus vacas. \bibverse{9} Y subieron
también con él carros y gente de á caballo, é hízose un escuadrón muy
grande. \bibverse{10} Y llegaron hasta la era de Atad, que está á la
otra parte del Jordán, y endecharon allí con grande y muy grave
lamentación: y José hizo á su padre duelo por siete días. \bibverse{11}
Y viendo los moradores de la tierra, los Cananeos, el llanto en la era
de Atad, dijeron: Llanto grande es este de los Egipcios: por eso fué
llamado su nombre Abelmizraim, que está á la otra parte del Jordán.
\bibverse{12} Hicieron, pues, sus hijos con él, según les había mandado:
\footnote{\textbf{50:12} Gén 49,29} \bibverse{13} Pues lleváronlo sus
hijos á la tierra de Canaán, y le sepultaron en la cueva del campo de
Macpela, la que había comprado Abraham con el mismo campo, para heredad
de sepultura, de Ephrón el Hetheo, delante de Mamre. \footnote{\textbf{50:13}
  Gén 23,16} \bibverse{14} Y tornóse José á Egipto, él y sus hermanos, y
todos los que subieron con él á sepultar á su padre, después que le hubo
sepultado.

\hypertarget{la-generosidad-de-josuxe9-hacia-sus-hermanos}{%
\subsection{La generosidad de José hacia sus
hermanos}\label{la-generosidad-de-josuxe9-hacia-sus-hermanos}}

\bibverse{15} Y viendo los hermanos de José que su padre era muerto,
dijeron: Quizá nos aborrecerá José, y nos dará el pago de todo el mal
que le hicimos. \bibverse{16} Y enviaron á decir á José: Tu padre mandó
antes de su muerte, diciendo: \bibverse{17} Así diréis á José: Ruégote
que perdones ahora la maldad de tus hermanos y su pecado, porque mal te
trataron: por tanto ahora te rogamos que perdones la maldad de los
siervos del Dios de tu padre. Y José lloró mientras hablaban.
\bibverse{18} Y vinieron también sus hermanos, y postráronse delante de
él, y dijeron: Henos aquí por tus siervos. \bibverse{19} Y respondióles
José: No temáis: ¿estoy yo en lugar de Dios? \bibverse{20} Vosotros
pensasteis mal sobre mí, mas Dios lo encaminó á bien, para hacer lo que
vemos hoy, para mantener en vida á mucho pueblo. \footnote{\textbf{50:20}
  Gén 45,5; Is 28,29} \bibverse{21} Ahora, pues, no tengáis miedo; yo os
sustentaré á vosotros y á vuestros hijos. Así los consoló, y les habló
al corazón.

\hypertarget{la-vejez-y-la-muerte-de-josuxe9-su-ultimo-deseo}{%
\subsection{La vejez y la muerte de José; su ultimo
deseo}\label{la-vejez-y-la-muerte-de-josuxe9-su-ultimo-deseo}}

\bibverse{22} Y estuvo José en Egipto, él y la casa de su padre: y vivió
José ciento diez años. \bibverse{23} Y vió José los hijos de Ephraim
hasta la tercera generación: también los hijos de Machîr, hijo de
Manasés, fueron criados sobre las rodillas de José. \bibverse{24} Y José
dijo á sus hermanos: Yo me muero; mas Dios ciertamente os visitará, y os
hará subir de aquesta tierra á la tierra que juró á Abraham, á Isaac, y
á Jacob. \^{}\^{} \bibverse{25} Y conjuró José á los hijos de Israel,
diciendo: Dios ciertamente os visitará, y haréis llevar de aquí mis
huesos. \bibverse{26} Y murió José de edad de ciento diez años; y
embalsamáronlo, y fué puesto en un ataúd en Egipto.
