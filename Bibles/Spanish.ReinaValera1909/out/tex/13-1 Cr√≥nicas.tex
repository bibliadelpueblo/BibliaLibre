\hypertarget{los-antepasados-hasta-el-diluvio}{%
\subsection{Los antepasados \hspace{0pt}\hspace{0pt}hasta el
diluvio}\label{los-antepasados-hasta-el-diluvio}}

\hypertarget{section}{%
\section{1}\label{section}}

\bibverse{1} Adam, Seth, Enos, \bibverse{2} Cainán, Mahalaleel, Jared,
\bibverse{3} Enoch, Mathusalem, Lamech, \bibverse{4} Noé, Sem, Châm, y
Japhet.

\hypertarget{los-descendientes-de-nouxe9-excepto-abraham-los-jafetitas}{%
\subsection{Los descendientes de Noé excepto Abraham; Los
jafetitas}\label{los-descendientes-de-nouxe9-excepto-abraham-los-jafetitas}}

\bibverse{5} Los hijos de Japhet: Gomer, Magog, Dadai, Javán, Tubal,
Mesec, y Thiras. \footnote{\textbf{1:5} Gén 10,2-5} \bibverse{6} Los
hijos de Gomer: Askenaz, Riphath, y Thogorma. \bibverse{7} Los hijos de
Javán: Elisa, Tharsis, Chîthim, y Dodanim.

\hypertarget{los-camitas}{%
\subsection{Los camitas}\label{los-camitas}}

\bibverse{8} Los hijos de Châm: Chûs, Misraim, Phuth, y Canaán.
\bibverse{9} Los hijos de Chûs: Seba, Havila, Sabtha, Raema, y Sabtechâ.
Y los hijos de Raema: Seba y Dedán. \bibverse{10} Chûs engendró á
Nimrod: éste comenzó á ser poderoso en la tierra. \bibverse{11} Misram
engendró á Ludim, Ananim, Laabim, Nephtuim, \bibverse{12} Phetrusim y
Casluim: de éstos salieron los Filisteos, y los Caphtoreos.
\bibverse{13} Canaán engendró á Sidón, su primogénito; \bibverse{14} Y
al Hetheo, y al Jebuseo, y al Amorrheo, y al Gergeseo; \bibverse{15} Y
al Heveo, y al Araceo, y al Sineo; \bibverse{16} Al Aradeo, y al
Samareo, y al Hamatheo.

\hypertarget{los-semitas}{%
\subsection{Los semitas}\label{los-semitas}}

\bibverse{17} Los hijos de Sem: Elam, Assur, Arphaxad, Lud, Aram, Hus,
Hul, Gether, y Mesec. \footnote{\textbf{1:17} Gén 10,21-31}
\bibverse{18} Arphaxad engendró á Sela, y Sela engendró á Heber.
\bibverse{19} Y á Heber nacieron dos hijos: el nombre del uno fué Peleg,
por cuanto en sus días fué dividida la tierra; y el nombre de su hermano
fué Joctán. \bibverse{20} Y Joctán engendró á Elmodad, Seleph,
Asarmaveth, y Jera, \bibverse{21} A Adoram también, á Uzal, Dicla,
\bibverse{22} Hebal, Abimael, Seba, \bibverse{23} Ophir, Havila, y
Jobab: todos hijos de Joctán.

\hypertarget{la-luxednea-recta-de-sem-a-abraham}{%
\subsection{La línea recta de Sem a
Abraham}\label{la-luxednea-recta-de-sem-a-abraham}}

\bibverse{24} Sem, Arphaxad, Sela, \bibverse{25} Heber, Peleg, Reu,
\bibverse{26} Serug, Nachôr, Thare, \bibverse{27} Y Abram, el cual es
Abraham.

\hypertarget{los-ismaelitas}{%
\subsection{Los ismaelitas}\label{los-ismaelitas}}

\bibverse{28} Los hijos de Abraham: Isaac é Ismael. \footnote{\textbf{1:28}
  Gén 21,3; Gén 16,16} \bibverse{29} Y estas son sus descendencias: el
primogénito de Ismael, Nabajoth; después Cedar, Adbeel, Misam,
\footnote{\textbf{1:29} Gén 25,13-16} \bibverse{30} Misma, Duma, Maasa,
Hadad, Thema, Jetur, Naphis, y Cedma. Estos son los hijos de Ismael.
\bibverse{31} Y Cethura, concubina de Abraham, parió á Zimram, Jocsán,
Medán, Madián, Isbac, y á Súa.

\hypertarget{los-descendientes-de-ketura}{%
\subsection{Los descendientes de
Ketura}\label{los-descendientes-de-ketura}}

\bibverse{32} Los hijos de Jobsán: Seba y Dedán. \footnote{\textbf{1:32}
  Gén 25,1-3} \bibverse{33} Los hijos de Madián: Epha, Epher, Henoch,
Abida, y Eldaa; todos estos fueron hijos de Cethura.

\hypertarget{los-descendientes-de-esauxfa}{%
\subsection{Los descendientes de
Esaú}\label{los-descendientes-de-esauxfa}}

\bibverse{34} Y Abraham engendró á Isaac: y los hijos de Isaac fueron
Esaú é Israel. \bibverse{35} Los hijos de Esaú: Eliphas, Rehuel, Jeus,
Jalam, y Cora. \footnote{\textbf{1:35} Gén 36,10-19} \bibverse{36} Los
hijos de Eliphas: Themán, Omar, Sephi, Hatham, Chênas, Timna, y Amalec.
\bibverse{37} Los hijos de Rehuel: Nahath, Zera, Samma, y Mizza.

\bibverse{38} Los hijos de Seir: Lotán, Sobal, Sibeón, Ana, Disón, Eser,
y Disán. \bibverse{39} Los hijos de Lotán: Hori, y Homam: y Timna fué
hermana de Lotán. \bibverse{40} Los hijos de Sobal: Alian, Manahach,
Ebal, Sephi y Oman. Los hijos de Sibehom: Aia y Ana. \bibverse{41} Disón
fué hijo de Ana: y los hijos de Disón; Hamrán, Hesbán, Ithran y Chêrán.
\bibverse{42} Los hijos de Eser: Bilham, Zaaván, y Jaacán. Los hijos de
Disán: Hus y Arán.

\hypertarget{los-reyes-y-jefes-edomitas}{%
\subsection{Los reyes y jefes
edomitas}\label{los-reyes-y-jefes-edomitas}}

\bibverse{43} Y estos son los reyes que reinaron en la tierra de Edom,
antes que reinase rey sobre los hijos de Israel: Belah, hijo de Beor; y
el nombre de su ciudad fué Dinaba. \footnote{\textbf{1:43} Gén 36,31-43}

\bibverse{44} Y muerto Belah, reinó en su lugar Jobab hijo de Zera, de
Bosra. \bibverse{45} Y muerto Jobab, reinó en su lugar Husam, de la
tierra de los Themanos. \bibverse{46} Muerto Husam, reinó en su lugar
Adad hijo de Bedad, el cual hirió á Madián en la campaña de Moab: y el
nombre de su ciudad fué Avith. \bibverse{47} Muerto Adad, reinó en su
lugar Samla, de Masreca. \bibverse{48} Muerto también Samla, reinó en su
lugar Saúl de Rehoboth, que está junto al río. \bibverse{49} Y muerto
Saúl, reinó en su lugar Baal-hanán hijo de Achbor. \bibverse{50} Y
muerto Baal-hanán, reinó en su lugar Adad, el nombre de cuya ciudad fué
Pai; y el nombre de su mujer Meetabel, hija de Matred, y ésta de Mezaab.
\bibverse{51} Muerto Adad, sucedieron los duques en Edom: el duque
Timna, el duque Alia, el duque Jetheth, \bibverse{52} El duque
Oholibama, el duque Ela, el duque Phinón, \bibverse{53} El duque Chênaz,
el duque Themán, el duque Mibzar, \bibverse{54} El duque Magdiel, el
duque Iram. Estos fueron los duques de Edom.

\hypertarget{los-hijos-de-jacob-israel-y-las-familias-de-la-tribu-de-juduxe1}{%
\subsection{Los hijos de Jacob Israel y las familias de la tribu de
Judá}\label{los-hijos-de-jacob-israel-y-las-familias-de-la-tribu-de-juduxe1}}

\hypertarget{section-1}{%
\section{2}\label{section-1}}

\bibverse{1} Estos son los hijos de Israel: Rubén, Simeón, Leví, Judá,
Issachâr, Zabulón, \bibverse{2} Dan, José, Benjamín, Nephtalí, Gad, y
Aser.

\hypertarget{de-juduxe1-a-hezruxf3n}{%
\subsection{De Judá a Hezrón}\label{de-juduxe1-a-hezruxf3n}}

\bibverse{3} Los hijos de Judá: Er, Onán, y Sela. Estos tres le nacieron
de la hija de Sua, Cananea. Y Er, primogénito de Judá, fué malo delante
de Jehová; y matólo. \footnote{\textbf{2:3} Gén 38,1-7} \bibverse{4} Y
Thamar su nuera le parió á Phares y á Zara. Todos los hijos de Judá
fueron cinco. \footnote{\textbf{2:4} Gén 38,29-30}

\bibverse{5} Los hijos de Phares: Hesrón y Hamul. \footnote{\textbf{2:5}
  Gén 46,12} \bibverse{6} Y los hijos de Zara: Zimri, Ethán, Hemán, y
Calcol, y Darda; en todos cinco. \bibverse{7} Hijo de Chârmi fué Achâr,
el que alborotó á Israel, porque prevaricó en el anatema. \bibverse{8}
Azaría fué hijo de Ethán.

\hypertarget{de-hezron-a-david-la-luxednea-ram}{%
\subsection{De Hezron a David (la línea
Ram)}\label{de-hezron-a-david-la-luxednea-ram}}

\bibverse{9} Los hijos que nacieron á Hesrón: Jerameel, Ram, y Chêlubai.
\footnote{\textbf{2:9} Rut 4,19-22; Mat 1,3; 1Cró 2,18; 1Cró 2,42}
\bibverse{10} Y Ram engendró á Aminadab; y Aminadab engendró á Nahasón,
príncipe de los hijos de Judá; \bibverse{11} Y Nahasón engendró á Salma,
y Salma engendró á Booz; \bibverse{12} Y Booz engendró á Obed, y Obed
engendró á Isaí; \bibverse{13} E Isaí engendró á Eliab, su primogénito,
y el segundo Abinadab, y Sima el tercero; \bibverse{14} El cuarto
Nathanael, el quinto Radai; \bibverse{15} El sexto Osem, el séptimo
David: \footnote{\textbf{2:15} 1Sam 17,12} \bibverse{16} De los cuales
Sarvia y Abigail fueron hermanas. Los hijos de Sarvia fueron tres:
Abisai, Joab, y Asael. \footnote{\textbf{2:16} 2Sam 2,18} \bibverse{17}
Abigail engendró á Amasa, cuyo padre fué Jether Ismaelita. \footnote{\textbf{2:17}
  2Sam 17,25}

\hypertarget{la-luxednea-caleb}{%
\subsection{La línea Caleb}\label{la-luxednea-caleb}}

\bibverse{18} Caleb hijo de Hesrón engendró á Jerioth de su mujer Azuba.
Y los hijos de ella fueron Jeser, Sobad, y Ardón. \footnote{\textbf{2:18}
  1Cró 2,9; 1Cró 2,42} \bibverse{19} Y muerta Azuba, tomó Caleb por
mujer á Ephrata, la cual le parió á Hur. \footnote{\textbf{2:19} 1Cró
  2,50} \bibverse{20} Y Hur engendró á Uri, y Uri engendró á Bezaleel.
\footnote{\textbf{2:20} Éxod 31,2}

\bibverse{21} Después entró Hesrón á la hija de Machîr padre de Galaad,
la cual tomó siendo él de sesenta años, y ella le parió á Segub.
\bibverse{22} Y Segub engendró á Jair, el cual tuvo veintitrés ciudades
en la tierra de Galaad. \footnote{\textbf{2:22} Jue 10,3} \bibverse{23}
Y Gesur y Aram tomaron las ciudades de Jair de ellos, y á Cenath con sus
aldeas, sesenta lugares. Todos estos fueron de los hijos de Machîr padre
de Galaad. \footnote{\textbf{2:23} 1Re 4,13} \bibverse{24} Y muerto
Hesrón en Caleb de Ephrata, Abia mujer de Hesrón le parió á Ashur padre
de Tecoa. \footnote{\textbf{2:24} 1Cró 4,5}

\hypertarget{la-luxednea-jerameel}{%
\subsection{La línea Jerameel}\label{la-luxednea-jerameel}}

\bibverse{25} Y los hijos de Jerameel primogénito de Hesrón fueron Ram
su primogénito, Buna, Orem, Osem, y Achîa. \footnote{\textbf{2:25} 1Cró
  2,9} \bibverse{26} Y tuvo Jerameel otra mujer llamada Atara, que fué
madre de Onam. \bibverse{27} Y los hijos de Ram primogénito de Jerameel
fueron Maas, Jamín, y Acar. \bibverse{28} Y los hijos de Onam fueron
Sammai, y Jada. Los hijos de Sammai: Nadab, y Abisur. \bibverse{29} Y el
nombre de la mujer de Abisur fué Abihail, la cual le parió á Abán, y á
Molib. \bibverse{30} Y los hijos de Nadab: Seled y Aphaim. Y Seled murió
sin hijos. \bibverse{31} E Isi fué hijo de Aphaim; y Sesam, hijo de Isi;
é hijo de Sesam, Alai. \bibverse{32} Los hijos de Jada hermano de
Simmai: Jether y Jonathán. Y murió Jether sin hijos. \bibverse{33} Y los
hijos de Jonathán: Peleth, y Zaza. Estos fueron los hijos de Jerameel.
\bibverse{34} Y Sesán no tuvo hijos, sino hijas. \bibverse{35} Y tuvo
Sesán un siervo Egipcio, llamado Jarha, al cual dió Sesán por mujer á su
hija; y ella le parió á Athai. \bibverse{36} Y Athai engendró á Nathán,
y Nathán engendró á Zabad: \bibverse{37} Y Zabad engendró á Ephlal, y
Ephlal engendró á Obed; \bibverse{38} Y Obed engendró á Jehú, y Jehú
engendró á Azarías; \bibverse{39} Y Azarías engendró á Heles, y Heles
engedró á Elasa; \bibverse{40} Elasa engendró á Sismai, y Sismai
engendró á Sallum; \bibverse{41} Y Sallum engendró á Jecamía, y Jecamía
engendró á Elisama.

\hypertarget{la-luxednea-caleb-1}{%
\subsection{La línea Caleb}\label{la-luxednea-caleb-1}}

\bibverse{42} Los hijos de Caleb hermano de Jerameel fueron Mesa su
primogénito, que fué el padre de Ziph; y los hijos de Maresa padre de
Hebrón. \footnote{\textbf{2:42} 1Cró 2,18} \bibverse{43} Y los hijos de
Hebrón: Core, y Thaphua, y Recem, y Sema. \bibverse{44} Y Sema engendró
á Raham, padre de Jorcaam; y Recem engendró á Sammai. \bibverse{45} Maón
fué hijo de Sammai, y Maón padre de Beth-zur. \bibverse{46} Y Epha,
concubina de Caleb, le parió á Harán, y á Mosa, y á Gazez. Y Harán
engendró á Gazez. \bibverse{47} Y los hijos de Joddai: Regem, Jotham,
Gesán, Pelet, Epho, y Saaph. \bibverse{48} Maachâ, concubina de Caleb,
le parió á Sebet, y á Thirana. \bibverse{49} Y también le parió á Saaph
padre de Madmannah, y á Seva padre de Macbena, y padre de Ghiba. Y Achsa
fué hija de Caleb.

\bibverse{50} Estos fueron los hijos de Caleb, hijo de Hur, primogénito
de Ephrata: Sobal, padre de Chîriath-jearim; \footnote{\textbf{2:50}
  1Cró 2,19} \bibverse{51} Salma, padre de Beth-lehem; Hareph, padre de
Beth-gader. \bibverse{52} Y los hijos de Sobal padre de Chîriath-jearim
fueron Haroeh, la mitad de los Manahethitas. \bibverse{53} Y las
familias de Chîriath-jearim fueron los Ithreos, y los Phuteos, y los
Samatheos, y los Misraiteos; de los cuales salieron los Soratitas, y los
Estaolitas. \bibverse{54} Los hijos de Salma: Beth-lehem, y los
Nethophatitas, los cuales son las coronas de la casa de Joab, y de la
mitad de los Manahethitas, los Soraitas. \footnote{\textbf{2:54} 1Cró
  9,16} \bibverse{55} Y las familias de los escribas, que moraban en
Jabes, fueron los Thiratheos, Simatheos, Sucatheos; los cuales son los
Cineos que vinieron de Hamath, padre de la casa de Rechâb. \footnote{\textbf{2:55}
  Jue 1,16; Jer 35,-1}

\hypertarget{los-hijos-de-david}{%
\subsection{Los hijos de David}\label{los-hijos-de-david}}

\hypertarget{section-2}{%
\section{3}\label{section-2}}

\bibverse{1} Estos son los hijos de David, que le nacieron en Hebrón:
Amnón el primogénito, de Achînoam Jezreelita; el segundo, Daniel, de
Abigail de Carmelo; \footnote{\textbf{3:1} 2Sam 3,2-5} \bibverse{2} El
tercero, Absalom, hijo de Maachâ hija de Talmai rey de Gesur; el cuarto,
Adonías hijo de Aggith; \bibverse{3} El quinto, Sephatías, de Abithal;
el sexto, Itream, de Egla su mujer. \bibverse{4} Estos seis le nacieron
en Hebrón, donde reinó siete años y seis meses: y en Jerusalem reinó
treinta y tres años. \bibverse{5} Estos cuatro le nacieron en Jerusalem:
Simma, Sobab, Nathán, y Salomón, de Beth-sua hija de Ammiel.
\bibverse{6} Y otros nueve: Ibaar, Elisama, y Eliphelet, \bibverse{7}
Noga, Nepheg, y Japhia, \bibverse{8} Elisama, Eliada, y Eliphelet.
\bibverse{9} Todos estos fueron los hijos de David, sin los hijos de las
concubinas. Y Thamar fué hermana de ellos. \footnote{\textbf{3:9} 2Sam
  13,1}

\hypertarget{los-reyes-davuxeddicos-desde-salomuxf3n-hasta-la-destrucciuxf3n-de-jerusaluxe9n}{%
\subsection{Los reyes davídicos desde Salomón hasta la destrucción de
Jerusalén}\label{los-reyes-davuxeddicos-desde-salomuxf3n-hasta-la-destrucciuxf3n-de-jerusaluxe9n}}

\bibverse{10} Hijo de Salomón fué Roboam, cuyo hijo fué Abía, del cual
fué hijo Asa, cuyo hijo fué Josaphat; \footnote{\textbf{3:10} Mat 1,7-12}
\bibverse{11} De quien fué hijo Joram, cuyo hijo fué Ochôzías, hijo del
cual fué Joas; \bibverse{12} Del cual fué hijo Amasías, cuyo hijo fué
Azarías, é hijo de éste Jotham; \bibverse{13} E hijo del cual fué Achâz,
del que fué hijo Ezechîas, cuyo hijo fué Manasés; \bibverse{14} Del cual
fué hijo Amón, cuyo hijo fué Josías. \bibverse{15} Y los hijos de
Josías: Johanán su primogénito, el segundo Joacim, el tercero Sedecías,
el cuarto Sallum. \bibverse{16} Los hijos de Joacim: Jechônías su hijo,
hijo del cual fué Sedecías.

\hypertarget{los-otros-descendientes-de-david-desde-jechonja-en-adelante}{%
\subsection{Los otros descendientes de David (desde Jechonja en
adelante)}\label{los-otros-descendientes-de-david-desde-jechonja-en-adelante}}

\bibverse{17} Y los hijos de Jechônías: Asir, Salathiel, \footnote{\textbf{3:17}
  2Cró 36,10}

\bibverse{18} Mechiram, Pedaía, Seneaser, y Jecamía, Hosama, y Nedabía.
\bibverse{19} Y los hijos de Pedaía: Zorobabel, y Simi. Y los hijos de
Zorobabel: Mesullam, Hananías, y Selomith su hermana. \bibverse{20} Y de
Mesullam: Hasuba, Ohel, y Berechîas, Hasadía, y Jusabhesed; cinco en
todos. \bibverse{21} Los hijos de Hananías: Pelatías, y Jesaías, hijo de
Rephaías, hijo de Arnán, hijo de Obdías, hijo de Sechânías.
\bibverse{22} Hijo de Sechânías fué Semaías; y los hijos de Semaías:
Hattus, Igheal, Barias, Nearías, y Saphat; seis. \bibverse{23} Los hijos
de Nearías fueron estos tres: Elioenai, Ezechîas, y Azricam.
\bibverse{24} Los hijos de Elioenai fueron estos siete: Odavias,
Eliasib, Palaías, Accub, Johanán, Dalaías, y Anani.

\hypertarget{muxe1s-informaciuxf3n-sobre-las-familias-de-la-tribu-de-juduxe1}{%
\subsection{Más información sobre las familias de la tribu de
Judá}\label{muxe1s-informaciuxf3n-sobre-las-familias-de-la-tribu-de-juduxe1}}

\hypertarget{section-3}{%
\section{4}\label{section-3}}

\bibverse{1} Los hijos de Judá: Phares, Hesrón, Carmi, Hur, y Sobal.
\footnote{\textbf{4:1} 1Cró 2,4-5; 1Cró 2,7; 1Cró 2,19; 1Cró 2,50}
\bibverse{2} Y Reaias hijo de Sobal, engendró á Jahath; y Jahath
engendró á Ahumai y á Laad. Estas son las familias de los Sorathitas.
\footnote{\textbf{4:2} 1Cró 2,53} \bibverse{3} Y estas son las del padre
de Etham: Jezreel, Isma, é Ibdas. Y el nombre de su hermana fué
Haslelponi. \bibverse{4} Y Penuel fué padre de Gedor, y Ezer padre de
Husa. Estos fueron los hijos de Hur, primogénito de Ephrata, padre de
Beth-lehem. \footnote{\textbf{4:4} 1Cró 2,19; 1Cró 2,50} \bibverse{5} Y
Asur padre de Tecoa tuvo dos mujeres, á saber, Helea, y Naara.
\bibverse{6} Y Naara le parió á Auzam, y á Hepher, á Themeni, y á
Ahastari. Estos fueron los hijos de Naara. \bibverse{7} Y los hijos de
Helea: Sereth, Jesohar, Ethnán. \bibverse{8} Y Cos engendró á Anob, y á
Sobeba, y la familia de Aharhel hijo de Arum.

\bibverse{9} Y Jabes fué más ilustre que sus hermanos, al cual su madre
llamó Jabes, diciendo: Por cuanto le parí en dolor.

\bibverse{10} E invocó Jabes al Dios de Israel, diciendo: ¡Oh si me
dieras bendición, y ensancharas mi término, y si tu mano fuera conmigo,
y me libraras de mal, que no me dañe! E hizo Dios que le viniese lo que
pidió.

\bibverse{11} Y Caleb hermano de Sua engendró á Machîr, el cual fué
padre de Esthón. \bibverse{12} Y Esthón engendró á Beth-rapha, á Phasea,
y á Tehinna, padre de la ciudad de Naas: estos son los varones de Rechâ.
\bibverse{13} Los hijos de Cenes: Othniel, y Seraiah. Los hijos de
Othniel: Hathath, \footnote{\textbf{4:13} Jos 15,17; Jue 1,13}
\bibverse{14} Y Maonathi, el cual engendró á Ophra: y Seraiah engendró á
Joab, padre de los habitantes en el valle llamado de Carisim, porque
fueron artífices. \bibverse{15} Los hijos de Caleb hijo de Jephone: Iru,
Ela, y Naham; é hijo de Ela, fué Cenez. \bibverse{16} Los hijos de
Jaleleel: Zip, Ziphas, Tirias, y Asareel. \bibverse{17} Y los hijos de
Ezra: Jeter, Mered, Epher, y Jalón: también engendró á Mariam, y Sammai,
y á Isba, padre de Esthemoa. \bibverse{18} Y su mujer Judaía le parió á
Jered padre de Gedor, y á Heber padre de Sochô, y á Icuthiel padre de
Zanoa. Estos fueron los hijos de Bethia hija de Faraón, con la cual casó
Mered. \bibverse{19} Y los hijos de la mujer de Odías, hermana de Naham,
fueron el padre de Keila de Garmi, y Esthemoa de Maachâti. \bibverse{20}
Y los hijos de Simón: Amnón, y Rinna, hijo de Hanán, y Tilón. Y los
hijos de Isi: Zoheth, y Benzoheth. \bibverse{21} Los hijos de Sela, hijo
de Judá: Er, padre de Lechâ, y Laada, padre de Maresa, y de la familia
de la casa del oficio del lino en la casa de Asbea; \footnote{\textbf{4:21}
  1Cró 2,3} \bibverse{22} Y Joacim, y los varones de Chôzeba, y Joas, y
Saraph, los cuales moraron en Moab, y Jasubi-lehem, que son palabras
antiguas. \bibverse{23} Estos fueron alfareros y se hallaban en medio de
plantíos y cercados, los cuales moraron allá con el rey en su obra.

\hypertarget{informaciuxf3n-sobre-los-descendientes-de-simeuxf3n}{%
\subsection{Información sobre los descendientes de
Simeón}\label{informaciuxf3n-sobre-los-descendientes-de-simeuxf3n}}

\bibverse{24} Los hijos de Simeón: Nemuel, Jamín, Jarib, Zera, Saúl;
\bibverse{25} También Sallum su hijo, Mibsam su hijo, y Misma su hijo.
\bibverse{26} Los hijos de Misma: Hamuel su hijo, Zachûr su hijo, y Simi
su hijo. \bibverse{27} Los hijos de Simi fueron diez y seis, y seis
hijas: mas sus hermanos no tuvieron muchos hijos, ni multiplicaron toda
su familia como los hijos de Judá.

\hypertarget{las-residencias-muxe1s-antiguas-de-la-tribu}{%
\subsection{Las residencias más antiguas de la
tribu}\label{las-residencias-muxe1s-antiguas-de-la-tribu}}

\bibverse{28} Y habitaron en Beer-seba, y en Molada, y en Hasar-sual,
\footnote{\textbf{4:28} Jos 19,2-8} \bibverse{29} Y en Bala, y en Esem,
y en Tholad, \bibverse{30} Y en Bethuel, y en Horma, y en Siclag,
\bibverse{31} Y en Beth-marchâboth, y en Hasasusim, y en Beth-birai, y
en Saaraim. Estas fueron sus ciudades hasta el reino de David.
\bibverse{32} Y sus aldeas fueron Etam, Ain, Rimmón, y Tochên, y Asán,
cinco pueblos; \bibverse{33} Y todos su villajes que estaban en contorno
de estas ciudades hasta Baal. Esta fué su habitación, y esta su
descendencia.

\hypertarget{indicaciuxf3n-de-otros-jefes-de-familia-simeonitas-las-dos-conquistas-de-los-simeonitas}{%
\subsection{Indicación de otros jefes de familia simeonitas; las dos
conquistas de los
simeonitas}\label{indicaciuxf3n-de-otros-jefes-de-familia-simeonitas-las-dos-conquistas-de-los-simeonitas}}

\bibverse{34} Y Mesobab, y Jamlech, y Josías hijo de Amasías;
\bibverse{35} Joel, y Jehú hijo de Josibias, hijo de Seraíah, hijo de
Aziel; \bibverse{36} Y Elioenai, Jacoba, Jesohaía, Asaías, Adiel,
Jesimiel, Benaías; \bibverse{37} Y Ziza hijo de Siphi, hijo de Allón,
hijo de Jedaía, hijo de Simri, hijo de Semaías. \bibverse{38} Estos por
sus nombres son los principales que vinieron en sus familias, y que
fueron multiplicados muy mucho en las casas de sus padres.

\bibverse{39} Y llegaron hasta la entrada de Gador hasta el oriente del
valle, buscando pastos para sus ganados. \bibverse{40} Y hallaron
gruesos y buenos pastos, y tierra ancha y espaciosa, y quieta y
reposada, porque los de Châm la habitaban de antes. \bibverse{41} Y
estos que han sido escritos por sus nombres, vinieron en días de
Ezechîas rey de Judá, y desbarataron sus tiendas y estancias que allí
hallaron, y destruyéronlos, hasta hoy, y habitaron allí en lugar de
ellos; por cuanto había allí pastos para sus ganados. \footnote{\textbf{4:41}
  2Re 18,1}

\bibverse{42} Y asimismo quinientos hombres de ellos, de los hijos de
Simeón, se fueron al monte de Seir, llevando por capitanes á Pelatía, y
á Nearías, y á Rephaías, y á Uzziel, hijos de Isi; \bibverse{43} E
hirieron á las reliquias que habían quedado de Amalec, y habitaron allí
hasta hoy.

\hypertarget{informaciuxf3n-sobre-rubuxe9n-y-sus-descendientes}{%
\subsection{Información sobre Rubén y sus
descendientes}\label{informaciuxf3n-sobre-rubuxe9n-y-sus-descendientes}}

\hypertarget{section-4}{%
\section{5}\label{section-4}}

\bibverse{1} Y los hijos de Rubén, primogénito de Israel, (porque él era
el primogénito, mas como violó el lecho de su padre, sus derechos de
primogenitura fueron dados á los hijos de José, hijo de Israel; y no fué
contado por primogénito. \footnote{\textbf{5:1} Gén 35,22; Gén 49,4}
\bibverse{2} Porque Judá fué el mayorazgo sobre sus hermanos, y el
príncipe de ellos: mas el derecho de primogenitura fué de José.)
\footnote{\textbf{5:2} Gén 49,8; Gén 49,10; Gén 49,22; Deut 33,7; Deut
  33,13-17} \bibverse{3} Fueron pues los hijos de Rubén, primogénito de
Israel: Enoch, Phallu, Esrón y Charmi. \footnote{\textbf{5:3} Éxod 6,14}
\bibverse{4} Los hijos de Joel: Semaías su hijo, Gog su hijo, Simi su
hijo; \bibverse{5} Michâ su hijo, Recaía su hijo, Baal su hijo;
\bibverse{6} Beera su hijo, el cual fué trasportado por
Thiglath-pilneser rey de los Asirios. Este era principal de los
Rubenitas. \bibverse{7} Y sus hermanos por sus familias, cuando eran
contados en sus descendencias, tenían por príncipes á Jeiel y á
Zachârías. \bibverse{8} Y Bela hijo de Azaz, hijo de Sema, hijo de Joel,
habitó en Aroer hasta Nebo y Baal-meón.

\hypertarget{informaciuxf3n-histuxf3rica-sobre-bela}{%
\subsection{Información histórica sobre
Bela}\label{informaciuxf3n-histuxf3rica-sobre-bela}}

\bibverse{9} Habitó también desde el oriente hasta la entrada del
desierto desde el río Eufrates: porque tenía muchos ganados en la tierra
de Galaad.

\bibverse{10} Y en los días de Saúl trajeron guerra contra los Agarenos,
los cuales cayeron en su mano; y ellos habitaron en sus tiendas sobre
toda la haz oriental de Galaad.

\hypertarget{informaciuxf3n-sobre-la-estirpe-y-lugares-de-residencia-asuxed-como-sobre-la-valoraciuxf3n-de-los-gaditas.}{%
\subsection{Información sobre la estirpe y lugares de residencia, así
como sobre la valoración de los
gaditas.}\label{informaciuxf3n-sobre-la-estirpe-y-lugares-de-residencia-asuxed-como-sobre-la-valoraciuxf3n-de-los-gaditas.}}

\bibverse{11} Y los hijos de Gad habitaron enfrente de ellos en la
tierra de Basán hasta Salca. \bibverse{12} Y Joel fué el principal en
Basán, el segundo Sephán, luego Janai, después Saphat. \bibverse{13} Y
sus hermanos, según las familias de sus padres, fueron Michâel,
Mesullam, Seba, Jorai, Jachân, Zia, y Heber; en todos siete.
\bibverse{14} Estos fueron los hijos de Abihail hijo de Huri, hijo de
Jaroa, hijo de Galaad, hijo de Michâel, hijo de Jesiaí, hijo de Jaddo,
hijo de Buz. \bibverse{15} También Ahí, hijo de Abdiel, hijo de Guni,
fué principal en la casa de sus padres. \bibverse{16} Los cuales
habitaron en Galaad, en Basán, y en sus aldeas, y en todos los ejidos de
Sarón hasta salir de ellos. \bibverse{17} Todos estos fueron contados
por sus generaciones en días de Jothán rey de Judá, y en días de
Jeroboam rey de Israel. \footnote{\textbf{5:17} 2Re 15,32; 2Re 14,23}

\hypertarget{la-lucha-de-las-tres-tribus-de-transjordania-con-los-agaritascon-los-agaritas}{%
\subsection{La lucha de las tres tribus de Transjordania con los
agaritascon los
agaritas}\label{la-lucha-de-las-tres-tribus-de-transjordania-con-los-agaritascon-los-agaritas}}

\bibverse{18} Los hijos de Rubén, y de Gad, y la media tribu de Manasés,
hombres valientes, hombres que traían escudo y espada, que entesaban
arco, y diestros en guerra, eran cuarenta y cuatro mil setecientos y
sesenta que salían á batalla. \bibverse{19} Y tuvieron guerra con los
Agarenos, y Jethur, y Naphis, y Nodab. \bibverse{20} Y fueron ayudados
contra ellos, y los Agarenos se dieron en sus manos, y todos los que con
ellos estaban; porque clamaron á Dios en la guerra, y fuéles favorable,
porque esperaron en él. \bibverse{21} Y tomaron sus ganados, cincuenta
mil camellos, y doscientas cincuenta mil ovejas, dos mil asnos, y cien
mil personas. \bibverse{22} Y cayeron muchos heridos, porque la guerra
era de Dios; y habitaron en sus lugares hasta la transmigración.

\hypertarget{las-residencias-y-la-divisiuxf3n-de-guxe9nero-de-los-manasitas}{%
\subsection{Las residencias y la división de género de los
manasitas}\label{las-residencias-y-la-divisiuxf3n-de-guxe9nero-de-los-manasitas}}

\bibverse{23} Y los hijos de la media tribu de Manasés habitaron en la
tierra, desde Basán hasta Baal-Hermón, y Senir y el monte de Hermón,
multiplicados en gran manera. \bibverse{24} Y estas fueron las cabezas
de las casas de sus padres: Epher, Isi, y Eliel, Azriel, y Jeremías, y
Odavia, y Jadiel, hombres valientes y de esfuerzo, varones de nombre y
cabeceras de las casas de sus padres.

\hypertarget{castigo-por-la-apostasuxeda-de-las-tres-tribus-de-jordania-oriental-por-los-reyes-asirios}{%
\subsection{Castigo por la apostasía de las tres tribus de Jordania
Oriental por los reyes
asirios}\label{castigo-por-la-apostasuxeda-de-las-tres-tribus-de-jordania-oriental-por-los-reyes-asirios}}

\bibverse{25} Mas se rebelaron contra el Dios de sus padres, y
fornicaron siguiendo los dioses de los pueblos de la tierra, á los
cuales había Jehová quitado de delante de ellos. \bibverse{26} Por lo
cual el Dios de Israel excitó el espíritu de Phul rey de los Asirios, y
el espíritu de Thiglath-pilneser rey de los Asirios, el cual trasportó á
los Rubenitas y Gaditas y á la media tribu de Manasés, y llevólos á
Halad, y á Habor y á Ara, y al río de Gozán, hasta hoy. \footnote{\textbf{5:26}
  2Re 15,19; 2Re 15,29}

\hypertarget{de-levi-a-los-hijos-de-aaruxf3n}{%
\subsection{De Levi a los hijos de
Aarón}\label{de-levi-a-los-hijos-de-aaruxf3n}}

\hypertarget{section-5}{%
\section{6}\label{section-5}}

\bibverse{1} Los hijos de Leví: Gersón, Coath, y Merari. \footnote{\textbf{6:1}
  1Cró 5,27; Éxod 6,16-19} \bibverse{2} Los hijos de Coath: Amram,
Ishar, Hebrón y Uzziel. \bibverse{3} Los hijos de Amram: Aarón, Moisés,
y Mariam. Los hijos de Aarón: Nadab, Abiú, Eleazar, é Ithamar.

\hypertarget{la-luxednea-de-sumo-sacerdote-desde-eleazar-hasta-el-cautiverio-babiluxf3nico}{%
\subsection{La línea de sumo sacerdote desde Eleazar hasta el cautiverio
babilónico}\label{la-luxednea-de-sumo-sacerdote-desde-eleazar-hasta-el-cautiverio-babiluxf3nico}}

\bibverse{4} Eleazar engendró á Phinees, y Phinees engendró á Abisua:
\bibverse{5} Y Abisua engendró á Bucci, y Bucci engendró á Uzzi;
\bibverse{6} Y Uzzi engendró á Zeraías, y Zeraías engendró á Meraioth;
\bibverse{7} Y Meraioth engendró á Amarías, y Amarías engendró á
Achîtob; \footnote{\textbf{6:7} Éxod 6,24} \bibverse{8} Y Achîtob
engendró á Sadoc, y Sadoc engendró á Achîmaas; \bibverse{9} Y Achîmaas
engendró á Azarías, y Azarías engendró á Johanan; \bibverse{10} Y
Johanan engendró á Azarías, el que tuvo el sacerdocio en la casa que
Salomón edificó en Jerusalem; \bibverse{11} Y Azarías engendró á
Amarías, y Amarías engendró á Achîtob; \bibverse{12} Y Achîtob engendró
á Sadoc, y Sadoc engendró á Sallum; \bibverse{13} Y Sallum engendró á
Hilcías, é Hilcías engendró á Azarías; \footnote{\textbf{6:13} 1Sam 8,2}
\bibverse{14} Y Azarías engendró á Seraíah, y Seraíah engendró á
Josadec. \bibverse{15} Y Josadec fué cautivo cuando Jehová trasportó á
Judá y á Jerusalem por mano de Nabucodonosor.

\hypertarget{los-descendientes-de-levi}{%
\subsection{Los descendientes de Levi}\label{los-descendientes-de-levi}}

\bibverse{16} Los hijos de Leví: Gersón, Coath, y Merari. \bibverse{17}
Y estos son los nombres de los hijos de Gersón: Libni, y Simi.
\bibverse{18} Los hijos de Coath: Amram, Ishar, Hebrón, y Uzziel.
\bibverse{19} Los hijos de Merari: Mahali, y Musi. Estas son las
familias de Leví, según sus descendencias. \bibverse{20} Gersón: Libni
su hijo, Joath su hijo, Zimma su hijo, \bibverse{21} Joab su hijo, Iddo
su hijo, Zera su hijo, Jeothrai su hijo. \bibverse{22} Los hijos de
Coath: Aminadab su hijo, Coré su hijo, Asir su hijo, \bibverse{23}
Elcana su hijo, Abiasaph su hijo, Asir su hijo, \bibverse{24} Thahath su
hijo, Uriel su hijo, Uzzia su hijo, y Saúl su hijo. \footnote{\textbf{6:24}
  1Cró 15,17} \bibverse{25} Los hijos de Elcana: Amasai, Achîmoth, y
Elcana. \bibverse{26} Los hijos de Elcana: Sophai su hijo, Nahath su
hijo, \bibverse{27} Eliab su hijo, Jeroham su hijo, Elcana su hijo.
\bibverse{28} Los hijos de Samuel: el primogénito Vasni, y Abías.
\bibverse{29} Los hijos de Merari: Mahali, Libni su hijo, Simi su hijo,
Uzza su hijo, \bibverse{30} Sima su hijo, Haggía su hijo, Assía su hijo.

\hypertarget{las-tres-familias-de-cantantes-levuxedticos-hemuxe1n-asaf-y-etuxe1n}{%
\subsection{Las tres familias de cantantes Levíticos, Hemán, Asaf y
Etán}\label{las-tres-familias-de-cantantes-levuxedticos-hemuxe1n-asaf-y-etuxe1n}}

\bibverse{31} Y estos son á los que David dió cargo de las cosas de la
música de la casa de Jehová, después que el arca tuvo reposo:
\bibverse{32} Los cuales servían delante de la tienda del tabernáculo
del testimonio en cantares, hasta que Salomón edificó la casa de Jehová
en Jerusalem: después estuvieron en su ministerio según su costumbre.
\bibverse{33} Estos pues con sus hijos asistían: de los hijos de Coath,
Hemán cantor, hijo de Joel, hijo de Samuel; \bibverse{34} Hijo de
Elcana, hijo de Jeroham, hijo de Eliel, hijo de Thoa; \footnote{\textbf{6:34}
  Éxod 28,1; Lev 16,-1} \bibverse{35} Hijo de Suph, hijo de Elcana, hijo
Mahath, hijo de Amasai; \footnote{\textbf{6:35} 1Cró 5,29-34}
\bibverse{36} Hijo de Elcana, hijo de Joel, hijo de Azarías, hijo de
Sophonías; \bibverse{37} Hijo de Thahat, hijo de Asir, hijo de Abiasaph,
hijo de Core; \bibverse{38} Hijo de Ishar, hijo de Coath, hijo de Leví,
hijo de Israel.

\bibverse{39} Y su hermano Asaph, el cual estaba á su mano derecha:
Asaph, hijo de Berachîas, hijo de Sima; \bibverse{40} Hijo de Michâel,
hijo de Baasías, hijo de Malchîas; \bibverse{41} Hijo de Athanai, hijo
de Zera, hijo de Adaia; \bibverse{42} Hijo de Ethán, hijo de Zimma, hijo
de Simi; \bibverse{43} Hijo de Jahat, hijo de Gersón, hijo de Leví.

\bibverse{44} Mas los hijos de Merari sus hermanos estaban á la mano
siniestra, es á saber, Ethán hijo de Chîsi, hijo de Abdi, hijo de
Maluch; \bibverse{45} Hijo de Hasabías, hijo de Amasías, hijo de
Hilcías; \bibverse{46} Hijo de Amasai, hijo de Bani, hijo de Semer;
\footnote{\textbf{6:46} 1Cró 6,51-55} \bibverse{47} Hijo de Mahali, hijo
de Musi, hijo de Merari, hijo de Leví. \footnote{\textbf{6:47} 1Cró
  6,56-61}

\hypertarget{los-levitas-y-los-aaronitas-en-el-servicio-del-templo}{%
\subsection{Los levitas y los aaronitas en el servicio del
templo}\label{los-levitas-y-los-aaronitas-en-el-servicio-del-templo}}

\bibverse{48} Y sus hermanos los Levitas fueron puestos sobre todo el
ministerio del tabernáculo de la casa de Dios. \footnote{\textbf{6:48}
  1Cró 6,62-66} \bibverse{49} Mas Aarón y sus hijos ofrecían perfume
sobre el altar del holocausto, y sobre el altar del perfume, en toda la
obra del lugar santísimo, y para hacer las expiaciones sobre Israel,
conforme á todo lo que Moisés siervo de Dios había mandado.

\hypertarget{segunda-luxednea-de-sumos-sacerdotes-desde-aaruxf3n-hasta-ahimaas}{%
\subsection{Segunda línea de sumos sacerdotes desde Aarón hasta
Ahimaas}\label{segunda-luxednea-de-sumos-sacerdotes-desde-aaruxf3n-hasta-ahimaas}}

\bibverse{50} Y los hijos de Aarón son estos: Eleazar su hijo, Phinees
su hijo, Abisua su hijo; \bibverse{51} Bucci su hijo, Uzzi su hijo,
Zeraías su hijo; \bibverse{52} Meraioth su hijo, Amarías su hijo,
Achîtob su hijo; \bibverse{53} Sadoc su hijo, Achîmaas su hijo.

\hypertarget{las-ciudades-levitas}{%
\subsection{Las ciudades levitas}\label{las-ciudades-levitas}}

\bibverse{54} Y estas son sus habitaciones, conforme á sus domicilios y
sus términos, las de los hijos de Aarón por las familias de los
Coathitas, porque de ellos fué la suerte: \bibverse{55} Les dieron pues
á Hebrón en tierra de Judá, y sus ejidos alrededor de ella.
\bibverse{56} Mas el territorio de la ciudad y sus aldeas se dieron á
Caleb, hijo de Jephone. \bibverse{57} Y á los hijos de Aarón dieron las
ciudades de Judá de acogimiento, es á saber, á Hebrón, y á Libna con sus
ejidos; \bibverse{58} A Jathir, y Esthemoa con sus ejidos, y á Hilem con
sus ejidos, y á Debir con sus ejidos; \bibverse{59} A Asán con sus
ejidos, y á Beth-semes con sus ejidos: \bibverse{60} Y de la tribu de
Benjamín, á Geba con sus ejidos, y á Alemeth con sus ejidos, y á
Anathoth con sus ejidos. Todas sus ciudades fueron trece ciudades,
repartidas por sus linajes.

\bibverse{61} A los hijos de Coath, que quedaron de su parentela, dieron
diez ciudades de la media tribu de Manasés por suerte. \bibverse{62} Y á
los hijos de Gersón, por sus linajes, dieron de la tribu de Issachâr, y
de la tribu de Aser, y de la tribu de Nephtalí, y de la tribu de Manasés
en Basán, trece ciudades. \bibverse{63} Y á los hijos de Merari, por sus
linajes, de la tribu de Rubén, y de la tribu de Gad, y de la tribu de
Zabulón, se dieron por suerte doce ciudades. \bibverse{64} Y dieron los
hijos de Israel á los Levitas ciudades con sus ejidos. \bibverse{65} Y
dieron por suerte de la tribu de los hijos de Judá, y de la tribu de los
hijos de Simeón, y de la tribu de los hijos de Benjamín, las ciudades
que nombraron por sus nombres.

\bibverse{66} Y á los linajes de los hijos de Coath dieron ciudades con
sus términos de la tribu de Ephraim. \bibverse{67} Y diéronles las
ciudades de acogimiento, á Sichêm con sus ejidos en el monte de Ephraim,
y á Gezer con sus ejidos, \bibverse{68} Y á Jocmeam con sus ejidos, y á
Beth-oron con sus ejidos, \bibverse{69} Y á Ajalón con sus ejidos, y á
Gath-rimmón con sus ejidos. \bibverse{70} De la media tribu de Manasés,
á Aner con sus ejidos, y á Bilam con sus ejidos, para los del linaje de
los hijos de Coath que habían quedado.

\bibverse{71} Y á los hijos de Gersón dieron de la familia de la media
tribu de Manasés, á Golan en Basán con sus ejidos y á Astharoth con sus
ejidos; \bibverse{72} Y de la tribu de Issachâr, á Cedes con sus ejidos,
á Dobrath con sus ejidos, \bibverse{73} Y á Ramoth con sus ejidos, y á
Anem con sus ejidos; \bibverse{74} Y de la tribu de Aser, á Masal con
sus ejidos, y á Abdón con sus ejidos, \bibverse{75} Y á Ucoc con sus
ejidos, y á Rehob con sus ejidos; \bibverse{76} Y de la tribu de
Nephtalí, á Cedes en Galilea con sus ejidos, y á Ammón con sus ejidos, á
Chîriath-jearim con sus ejidos.

\bibverse{77} Y á los hijos de Merari que habían quedado, dieron de la
tribu de Zabulón á Rimmono con sus ejidos, y á Thabor con sus ejidos;
\bibverse{78} Y de la otra parte del Jordán de Jericó, al oriente del
Jordán, dieron, de la tribu de Rubén, á Beser en el desierto con sus
ejidos; y á Jasa con sus ejidos, \bibverse{79} Y á Chêdemoth con sus
ejidos, y á Mephaath con sus ejidos; \bibverse{80} Y de la tribu de Gad,
á Ramot en Galaad con sus ejidos, y á Mahanaim con sus ejidos,
\bibverse{81} Y á Hesbón con sus ejidos, y á Jacer con sus ejidos.

\hypertarget{la-tribu-de-isacar}{%
\subsection{La tribu de Isacar}\label{la-tribu-de-isacar}}

\hypertarget{section-6}{%
\section{7}\label{section-6}}

\bibverse{1} Los hijos de Issachâr, cuatro: Thola, Phúa, Jabsub, y
Simrón. \footnote{\textbf{7:1} Gén 46,13; Núm 26,23-24} \bibverse{2} Los
hijos de Thola: Uzzi, Rephaías, Jeriel, Jamai, Jibsam y Samuel, cabezas
en las familias de sus padres. De Thola fueron contados por sus linajes
en el tiempo de David, veintidós mil seiscientos hombres muy valerosos.
\bibverse{3} Hijo de Uzzi fué Izrahías; y los hijos de Izrahías:
Michâel, Obadías, Joel, é Isías: todos, cinco príncipes. \bibverse{4} Y
había con ellos en sus linajes, por las familias de sus padres, treinta
y seis mil hombres de guerra: porque tuvieron muchas mujeres é hijos.
\bibverse{5} Y sus hermanos por todas las familias de Issachâr, contados
todos por sus genealogías, eran ochenta y siete mil hombres valientes en
extremo.

\hypertarget{la-tribu-de-benjamuxedn}{%
\subsection{La tribu de Benjamín}\label{la-tribu-de-benjamuxedn}}

\bibverse{6} Los hijos de Benjamín fueron tres: Bela, Bechêr, y Jediael.
\bibverse{7} Los hijos de Bela: Esbon, Uzzi, Uzziel, Jerimoth, é Iri;
cinco cabezas de casas de linajes, hombres de gran valor, y de cuya
descendencia fueron contados veintidós mil treinta y cuatro.
\bibverse{8} Los hijos de Bechêr: Zemira, Joas, Eliezer, Elioenai, Omri,
Jerimoth, Abías, Anathoth y Alemeth; todos estos fueron hijos de Bechêr.
\bibverse{9} Y contados por sus descendencias, por sus linajes, los que
eran cabezas de sus familias, resultaron veinte mil y doscientos hombres
de grande esfuerzo. \bibverse{10} Hijo de Jediael fué Bilhán; y los
hijos de Bilhán: Jebús, Benjamín, Aod, Chênaana, Zethán, Tharsis, y
Ahisahar. \bibverse{11} Todos estos fueron hijos de Jediael, cabezas de
familias, hombres muy valerosos, diecisiete mil y doscientos que salían
á combatir en la guerra. \bibverse{12} Y Suppim y Huppim fueron hijos de
Hir: y Husim, hijo de Aher.

\hypertarget{la-tribu-de-neftaluxed}{%
\subsection{La tribu de Neftalí}\label{la-tribu-de-neftaluxed}}

\bibverse{13} Los hijos de Nephtalí: Jaoel, Guni, Jezer, y Sallum, hijos
de Bilha. \footnote{\textbf{7:13} Gén 46,24}

\hypertarget{la-tribu-de-manasuxe9s}{%
\subsection{La tribu de Manasés}\label{la-tribu-de-manasuxe9s}}

\bibverse{14} Los hijos de Manasés: Asriel, el cual le parió su
concubina la Sira: (la cual también le parió á Machîr, padre de Galaad:
\footnote{\textbf{7:14} Núm 26,29-33} \bibverse{15} Y Machîr tomó por
mujer la hermana de Huppim y Suppim, cuya hermana tuvo por nombre
Maachâ:) y el nombre del segundo fué Salphaad. Y Salphaad tuvo hijas.
\footnote{\textbf{7:15} Núm 27,1} \bibverse{16} Y Maachâ mujer de Machîr
le parió un hijo, y llamóle Peres; y el nombre de su hermano fué Seres,
cuyos hijos fueron Ulam y Recem. \bibverse{17} Hijo de Ulam fué Bedán.
Estos fueron los hijos de Galaad, hijo de Machîr, hijo de Manasés.
\bibverse{18} Y su hermana Molechêt parió á Ischôd, y á Abiezer, y
Mahala. \bibverse{19} Y los hijos de Semida fueron Ahián, Sechêm, Licci,
y Aniam.

\hypertarget{la-tribu-de-ephraim}{%
\subsection{La tribu de Ephraim}\label{la-tribu-de-ephraim}}

\bibverse{20} Los hijos de Ephraim: Suthela, Bered su hijo, su hijo
Thahath, Elada su hijo, Thahath su hijo, \bibverse{21} Zabad su hijo, y
Suthela su hijo, Ezer, y Elad. Mas los hijos de Gath, naturales de
aquella tierra, los mataron, porque vinieron á tomarles sus ganados.
\bibverse{22} Y Ephraim su padre hizo duelo por muchos días, y vinieron
sus hermanos á consolarlo. \bibverse{23} Entrando él después á su mujer
ella concibió, y parió un hijo, al cual puso por nombre Bería; por
cuanto había estado en aflicción en su casa. \bibverse{24} Y su hija fué
Seera, la cual edificó á Beth-oron la baja y la alta, y á Uzzen-seera.
\bibverse{25} Hijo de este Bería fué Repha, y Reseph, y Thela su hijo, y
Taán su hijo, \bibverse{26} Laadán su hijo, Ammiud su hijo, Elisama su
hijo, \footnote{\textbf{7:26} Núm 1,10} \bibverse{27} Nun su hijo, Josué
su hijo. \footnote{\textbf{7:27} Núm 13,8}

\hypertarget{residencias-de-la-tribu}{%
\subsection{Residencias de la tribu}\label{residencias-de-la-tribu}}

\bibverse{28} Y la heredad y habitación de ellos fué Beth-el con sus
aldeas: y hacia el oriente Naarán, y á la parte del occidente Gezer y
sus aldeas: asimismo Sichêm con sus aldeas, hasta Asa y sus aldeas;
\footnote{\textbf{7:28} Jos 16,1; Jos 16,10} \bibverse{29} Y á la parte
de los hijos de Manasés, Beth-seán con sus aldeas, Thanach con sus
aldeas, Megiddo con sus aldeas, Dor con sus aldeas. En estos lugares
habitaron los hijos de José, hijo de Israel. \footnote{\textbf{7:29} Jos
  17,11}

\hypertarget{la-tribu-de-asser}{%
\subsection{La tribu de Asser}\label{la-tribu-de-asser}}

\bibverse{30} Los hijos de Aser: Imna, Isua, Isui, Bería, y su hermana
Sera. \footnote{\textbf{7:30} Gén 46,17}

\bibverse{31} Los hijos de Bería: Heber, y Machîel, el cual fué padre de
Birzabith. \bibverse{32} Y Heber engendró á Japhlet, Semer, Hotham, y
Sua hermana de ellos. \bibverse{33} Los hijos de Japhlet: Pasac, Bimhal,
y Asvath. Aquestos los hijos de Japhlet. \bibverse{34} Y los hijos de
Semer: Ahi, Roega, Jehubba, y Aram. \bibverse{35} Los hijos de Helem su
hermano: Sopha, Imna, Selles, y Amal. \bibverse{36} Los hijos de Sopha:
Sua, Harnapher, Sual, Beri, Imra, \bibverse{37} Beser, Hod, Samma,
Silsa, Ithrán y Beera. \bibverse{38} Los hijos de Jether: Jephone,
Pispa, y Ara. \bibverse{39} Y los hijos de Ulla; Ara, y Haniel, y Resia.
\bibverse{40} Y todos estos fueron hijos de Aser, cabezas de familias
paternas, escogidos, esforzados, cabezas de príncipes: y contados que
fueron por sus linajes entre los de armas tomar, el número de ellos fué
veintiséis mil hombres.

\hypertarget{hijos-y-descendientes-de-benjamuxedn-a-travuxe9s-de-bela}{%
\subsection{Hijos y descendientes de Benjamín a través de
Bela}\label{hijos-y-descendientes-de-benjamuxedn-a-travuxe9s-de-bela}}

\hypertarget{section-7}{%
\section{8}\label{section-7}}

\bibverse{1} Benjamín engendró á Bela su primogénito, Asbel el segundo,
Ara el tercero, \bibverse{2} Noha el cuarto, y Rapha el quinto.
\bibverse{3} Y los hijos de Bela fueron Addar, Gera, Abiud, \bibverse{4}
Abisua, Naamán, Ahoa, \bibverse{5} Y Gera, Sephuphim, y Huram.

\hypertarget{los-hijos-de-ehud}{%
\subsection{Los hijos de Ehud}\label{los-hijos-de-ehud}}

\bibverse{6} Y estos son los hijos de Ehud, estos las cabezas de padres
que habitaron en Gabaa, y fueron trasportados á Manahath: \bibverse{7}
Es á saber: Naamán, Achîas, y Gera: éste los trasportó, y engendró á
Uzza, y á Ahihud.

\hypertarget{la-familia-de-saharaim}{%
\subsection{La familia de Saharaim}\label{la-familia-de-saharaim}}

\bibverse{8} Y Saharaim engendró hijos en la provincia de Moab, después
que dejó á Husim y á Baara que eran sus mujeres. \bibverse{9} Engendró
pues de Chôdes su mujer, á Jobab, Sibias, Mesa, Malchâm, \bibverse{10}
Jeus, Sochîas, y Mirma. Estos son sus hijos, cabezas de familias.
\bibverse{11} Mas de Husim engendró á Abitob, y á Elphaal. \bibverse{12}
Y los hijos de Elphaal: Heber, Misam, y Semeb, (el cual edificó á Ono, y
á Loth con sus aldeas,)

\hypertarget{cinco-familias-benjaminitas-en-ajaluxf3n-y-jerusaluxe9n}{%
\subsection{Cinco familias benjaminitas en Ajalón y
Jerusalén}\label{cinco-familias-benjaminitas-en-ajaluxf3n-y-jerusaluxe9n}}

\bibverse{13} Berías también, y Sema, que fueron las cabezas de las
familias de los moradores de Ajalón, los cuales echaron á los moradores
de Gath; \bibverse{14} Y Ahío, Sasac, Jeremoth; \bibverse{15} Zebadías,
Arad, Heder; \bibverse{16} Michâel, Ispha, y Joa, hijos de Berías;
\bibverse{17} Y Zebadías, Mesullam, Hizchî, Heber; \bibverse{18} Ismari,
Izlia, y Jobab, hijos de Elphaal. \bibverse{19} Y Jacim, Zichri, Zabdi;
\bibverse{20} Elioenai, Silithai, Eliel; \bibverse{21} Adaías, Baraías,
y Simrath, hijos de Simi; \bibverse{22} E Isphán, Heber, Eliel;
\bibverse{23} Adón, Zichri, Hanán; \bibverse{24} Hananía, Belam,
Anathothías; \bibverse{25} Iphdaías, y Peniel, hijos de Sasac;
\bibverse{26} Y Samseri, Seharías, Atalía; \bibverse{27} Jaarsías,
Elías, Zichri, hijos de Jeroham. \bibverse{28} Estos fueron jefes
principales de familias por sus linajes, y habitaron en Jerusalem.

\hypertarget{la-familia-del-rey-sauxfal}{%
\subsection{La familia del rey Saúl}\label{la-familia-del-rey-sauxfal}}

\bibverse{29} Y en Gabaón habitaron Abiga-baón, la mujer del cual se
llamó Maachâ: \footnote{\textbf{8:29} 1Cró 9,35-44} \bibverse{30} Y su
hijo primogénito, Abdón, luego Sur, Chîs, Baal, Nadab, \bibverse{31}
Gedor, Ahíe, y Zechêr. \bibverse{32} Y Micloth engendró á Simea. Estos
también habitaron con sus hermanos en Jerusalem, enfrente de ellos.
\bibverse{33} Y Ner engendró á Cis, y Cis engendró á Saúl, y Saúl
engendró á Jonathán, Malchî-súa, Abinadab, y Esbaal. \bibverse{34} Hijo
de Jonathán fué Merib-baal, y Merib-baal engendró á Michâ. \bibverse{35}
Los hijos de Michâ: Phitón, Melech, Thaarea y Ahaz. \bibverse{36} Y Ahaz
engendró á Joadda; y Joadda engendró á Elemeth, y á Azmaveth, y á Zimri;
y Zimri engendró á Mosa; \bibverse{37} Y Mosa engendró á Bina, hijo del
cual fué Rapha, hijo del cual fué Elasa, cuyo hijo fué Asel.
\bibverse{38} Y los hijos de Asel fueron seis, cuyos nombres son
Azricam, Bochru, Ismael, Searías, Obadías, y Hanán: todos estos fueron
hijos de Asel. \bibverse{39} Y los hijos de Esec su hermano: Ulam su
primogénito, Jehus el segundo, Elipheleth el tercero. \bibverse{40} Y
fueron los hijos de Ulam hombres valientes y vigorosos, flecheros
diestros, los cuales tuvieron muchos hijos y nietos, ciento y cincuenta.
Todos estos fueron de los hijos de Benjamín. \footnote{\textbf{8:40}
  1Cró 12,2}

\hypertarget{directorio-de-residentes-destacados-de-jerusaluxe9n-en-el-peruxedodo-posterior-al-cautiverio}{%
\subsection{Directorio de residentes destacados de Jerusalén (en el
período posterior al
cautiverio)}\label{directorio-de-residentes-destacados-de-jerusaluxe9n-en-el-peruxedodo-posterior-al-cautiverio}}

\hypertarget{section-8}{%
\section{9}\label{section-8}}

\bibverse{1} Y contado todo Israel por el orden de los linajes, fueron
escritos en el libro de los reyes de Israel y de Judá, que fueron
trasportados á Babilonia por su rebelión. \footnote{\textbf{9:1} 2Re
  24,15-16} \bibverse{2} Los primeros moradores que entraron en sus
posesiones en sus ciudades, fueron así de Israel, como de los
sacerdotes, Levitas, y Nethineos. \footnote{\textbf{9:2} Jos 9,23; Esd
  8,20}

\hypertarget{el-pueblo-de-jerusaluxe9n}{%
\subsection{El pueblo de Jerusalén}\label{el-pueblo-de-jerusaluxe9n}}

\bibverse{3} Y habitaron en Jerusalem de los hijos de Judá, de los hijos
de Benjamín, de los hijos de Ephraim y Manasés: \footnote{\textbf{9:3}
  Neh 11,3-19}

\bibverse{4} Urai hijo de Amiud, hijo de Omri, hijo de Imrai, hijo de
Bani, de los hijos de Phares hijo de Judá. \bibverse{5} Y de Siloni,
Asaías el primogénito, y sus hijos. \bibverse{6} Y de los hijos de Zara,
Jehuel y sus hermanos, seiscientos noventa.

\bibverse{7} Y de los hijos de Benjamín: Sallu hijo de Mesullam, hijo de
Odavía, hijo de Asenua; \bibverse{8} E Ibnías hijo de Jeroham, y Ela
hijo de Uzzi, hijo de Michri; y Mesullam hijo de Sephatías, hijo de
Rehuel, hijo de Ibnías. \bibverse{9} Y sus hermanos por sus linajes
fueron nuevecientos cincuenta y seis. Todos estos hombres fueron cabezas
de familia en las casas de sus padres.

\bibverse{10} Y de los sacerdotes: Jedaía, Joiarib, Joachîm;
\bibverse{11} Y Azarías hijo de Hilcías, hijo de Mesullam, hijo de
Sadoc, hijo de Meraioth, hijo de Achîtob, príncipe de la casa de Dios;
\footnote{\textbf{9:11} 1Cró 5,39} \bibverse{12} Y Adaías hijo de
Jeroham, hijo de Phasur, hijo de Machîas; y Masai hijo de Adiel, hijo de
Jazera, hijo de Mesullam, hijo de Mesillemith, hijo de Immer;
\bibverse{13} Y sus hermanos, cabezas de las casas de sus padres, en
número de mil setecientos sesenta, hombres de grande eficacia en la obra
del ministerio en la casa de Dios.

\bibverse{14} Y de los Levitas: Semeías, hijo de Hassub, hijo de
Azricam, hijo de Hasabías, de los hijos de Merari; \bibverse{15} Y
Bacbaccar, Heres, y Galal, y Mattanía hijo de Michâs, hijo de Zichri,
hijo de Asaph; \bibverse{16} Y Obadías hijo de Semeías, hijo de Galal,
hijo de Iduthum: y Berachîas hijo de Asa, hijo de Elcana, el cual habitó
en las aldeas de Nethophati.

\hypertarget{los-porteros-y-sus-servicios}{%
\subsection{Los porteros y sus
servicios}\label{los-porteros-y-sus-servicios}}

\bibverse{17} Y los porteros: Sallum, Accub, Talmon, Ahiman, y sus
hermanos. Sallum era el jefe. \bibverse{18} Y hasta ahora entre las
cuadrillas de los hijos de Leví han sido estos los porteros en la puerta
del rey que está al oriente. \bibverse{19} Y Sallum hijo de Core, hijo
de Abiasath, hijo de Corah, y sus hermanos los Coraitas por la casa de
su padre, tuvieron cargo de la obra del ministerio, guardando las
puertas del tabernáculo; y sus padres fueron sobre la cuadrilla de
Jehová guardas de la entrada. \footnote{\textbf{9:19} Núm 4,18-20}
\bibverse{20} Y Phinees hijo de Eleazar fué antes capitán sobre ellos,
siendo Jehová con él. \footnote{\textbf{9:20} Núm 25,7-13} \bibverse{21}
Y Zacarías hijo de Meselemia era portero de la puerta del tabernáculo
del testimonio. \bibverse{22} Todos estos, escogidos para guardas en las
puertas, eran doscientos doce cuando fueron contados por el orden de sus
linajes en sus villas, á los cuales constituyó en su oficio David y
Samuel el vidente. \footnote{\textbf{9:22} 1Sam 9,9; 1Sam 9,11}
\bibverse{23} Así ellos y sus hijos eran porteros por sus turnos á las
puertas de la casa de Jehová, y de la casa del tabernáculo.
\bibverse{24} Y estaban los porteros á los cuatro vientos, al oriente,
al occidente, al septentrión, y al mediodía. \bibverse{25} Y sus
hermanos que estaban en sus aldeas, venían cada siete días por sus
tiempos con ellos.

\hypertarget{informaciuxf3n-sobre-los-deberes-oficiales-de-los-levitas}{%
\subsection{Información sobre los deberes oficiales de los
levitas}\label{informaciuxf3n-sobre-los-deberes-oficiales-de-los-levitas}}

\bibverse{26} Porque cuatro principales de los porteros Levitas estaban
en el oficio, y tenían cargo de las cámaras, y de los tesoros de la casa
de Dios. \bibverse{27} Estos moraban alrededor de la casa de Dios,
porque tenían cargo de la guardia, y el de abrir aquélla todas las
mañanas.

\bibverse{28} Algunos de estos tenían cargo de los vasos del ministerio,
los cuales se metían por cuenta, y por cuenta se sacaban. \bibverse{29}
Y otros de ellos tenían cargo de la vajilla, y de todos los vasos del
santuario, y de la harina, y del vino, y del aceite, y del incienso, y
de los aromas.

\bibverse{30} Y algunos de los hijos de los sacerdotes hacían los
ungüentos aromáticos. \bibverse{31} Y Mathathías, uno de los Levitas,
primogénito de Sallum Coraita, tenía cargo de las cosas que se hacían en
sartén. \bibverse{32} Y algunos de los hijos de Coath, y de sus
hermanos, tenían el cargo de los panes de la proposición, los cuales
ponían por orden cada sábado. \footnote{\textbf{9:32} Lev 24,5; Lev 24,8}

\hypertarget{informaciuxf3n-sobre-los-cantantes-del-templo-palabra-final}{%
\subsection{Información sobre los cantantes del templo; Palabra
final}\label{informaciuxf3n-sobre-los-cantantes-del-templo-palabra-final}}

\bibverse{33} Y de estos había cantores, principales de familias de los
Levitas, los cuales estaban en sus cámaras exentos; porque de día y de
noche estaban en aquella obra. \footnote{\textbf{9:33} 1Cró 9,14-16}
\bibverse{34} Estos eran jefes de familias de los Levitas por sus
linajes, jefes que habitaban en Jerusalem.

\hypertarget{apuxe9ndice-los-habitantes-de-gabauxf3n-y-una-segunda-genealoguxeda-de-la-casa-de-sauxfal}{%
\subsection{Apéndice: Los habitantes de Gabaón y una segunda genealogía
de la casa de
Saúl}\label{apuxe9ndice-los-habitantes-de-gabauxf3n-y-una-segunda-genealoguxeda-de-la-casa-de-sauxfal}}

\bibverse{35} Y en Gabaón habitaban Jehiel padre de Gabaón, el nombre de
cuya mujer era Maachâ; \footnote{\textbf{9:35} 1Cró 8,29-38}

\bibverse{36} Y su hijo primogénito Abdón, luego Sur, Chîs, Baal, Ner,
Nadab; \bibverse{37} Gedor, Ahio, Zachârías, y Micloth. \bibverse{38} Y
Micloth engendró á Samaán. Y estos habitaban también en Jerusalem con
sus hermanos enfrente de ellos. \bibverse{39} Y Ner engendró á Cis, y
Cis engendró á Saúl, y Saúl engendró á Jonathán, Malchîsua, Abinadab, y
Esbaal. \bibverse{40} E hijo de Jonathán fué Merib-baal, y Merib-baal
engendró á Michâ. \bibverse{41} Y los hijos de Michâ: Phitón, Melech,
Tharea, y Ahaz. \bibverse{42} Ahaz engendró á Jara, y Jara engendró á
Alemeth, Azmaveth, y Zimri: y Zimri engendró á Mosa; \bibverse{43} Y
Mosa engendró á Bina, cuyo hijo fué Rephaía, del que fué hijo Elasa,
cuyo hijo fué Asel. \bibverse{44} Y Asel tuvo seis hijos, los nombres de
los cuales son: Azricam, Bochru, Ismael, Seraía, Obadías, y Hanán: estos
fueron los hijos de Asel.

\hypertarget{israel-derrotado-por-los-filisteos-en-el-monte-gilboa-muerte-de-sauxfal-y-sus-tres-hijos}{%
\subsection{Israel derrotado por los filisteos en el monte Gilboa;
Muerte de Saúl y sus tres
hijos}\label{israel-derrotado-por-los-filisteos-en-el-monte-gilboa-muerte-de-sauxfal-y-sus-tres-hijos}}

\hypertarget{section-9}{%
\section{10}\label{section-9}}

\bibverse{1} Los Filisteos pelearon con Israel; y huyeron delante de
ellos los Israelitas, y cayeron heridos en el monte de Gilboa.
\bibverse{2} Y los Filisteos siguieron á Saúl y á sus hijos; y mataron
los Filisteos á Jonathán, y á Abinadab, y á Malchîsua, hijos de Saúl.
\bibverse{3} Y agravóse la batalla sobre Saúl, y le alcanzaron los
flecheros, y fué de los flecheros herido. \bibverse{4} Entonces dijo
Saúl á su escudero: Saca tu espada, y pásame con ella, porque no vengan
estos incircuncisos, y hagan escarnio de mí; mas su escudero no quiso,
porque tenía gran miedo. Entonces Saúl tomó la espada, y echóse sobre
ella.

\bibverse{5} Y como su escudero vió á Saúl muerto, él también se echó
sobre su espada, y matóse. \bibverse{6} Así murió Saúl, y sus tres
hijos; y toda su casa murió juntamente con él. \bibverse{7} Y viendo
todos los de Israel que habitaban en el valle, que habían huído, y que
Saúl y sus hijos eran muertos, dejaron sus ciudades, y huyeron: y
vinieron los Filisteos, y habitaron en ellas.

\hypertarget{el-destino-de-los-caduxe1veres-de-sauxfal-y-sus-hijos}{%
\subsection{El destino de los cadáveres de Saúl y sus
hijos}\label{el-destino-de-los-caduxe1veres-de-sauxfal-y-sus-hijos}}

\bibverse{8} Y fué que viniendo el día siguiente los Filisteos á
despojar los muertos, hallaron á Saúl y á sus hijos tendidos en el monte
de Gilboa. \bibverse{9} Y luego que le hubieron desnudado, tomaron su
cabeza y sus armas, y enviáronlo todo á la tierra de los Filisteos por
todas partes, para que fuese denunciado á sus ídolos y al pueblo.
\bibverse{10} Y pusieron sus armas en el templo de su dios, y colgaron
la cabeza en el templo de Dagón. \bibverse{11} Y oyendo todos los de
Jabes de Galaad lo que los Filisteos habían hecho de Saúl, \bibverse{12}
Levantáronse todos los hombres valientes, y tomaron el cuerpo de Saúl, y
los cuerpos de sus hijos, y trajéronlos á Jabes; y enterraron sus huesos
debajo del alcornoque en Jabes, y ayunaron siete días.

\hypertarget{revisiuxf3n-de-la-deuda-de-sauxfal-con-dios}{%
\subsection{Revisión de la deuda de Saúl con
Dios}\label{revisiuxf3n-de-la-deuda-de-sauxfal-con-dios}}

\bibverse{13} Así murió Saúl por su rebelión con que prevaricó contra
Jehová, contra la palabra de Jehová, la cual no guardó; y porque
consultó al pythón, preguntándole, \footnote{\textbf{10:13} 1Sam 15,11;
  1Sam 28,8}

\bibverse{14} Y no consultó á Jehová: por esta causa lo mató, y traspasó
el reino á David, hijo de Isaí.

\hypertarget{la-unciuxf3n-de-david-en-hebruxf3n-y-la-conquista-de-jerusaluxe9n}{%
\subsection{La unción de David en Hebrón y la conquista de
Jerusalén}\label{la-unciuxf3n-de-david-en-hebruxf3n-y-la-conquista-de-jerusaluxe9n}}

\hypertarget{section-10}{%
\section{11}\label{section-10}}

\bibverse{1} Entonces todo Israel se juntó á David en Hebrón, diciendo:
He aquí nosotros somos tu hueso y tu carne. \bibverse{2} Y además antes
de ahora, aun mientras Saúl reinaba, tú sacabas y metías á Israel.
También Jehová tu Dios te ha dicho: Tú apacentarás mi pueblo Israel, y
tú serás príncipe sobre Israel mi pueblo.

\bibverse{3} Y vinieron todos los ancianos de Israel al rey en Hebrón, y
David hizo con ellos alianza delante de Jehová; y ungieron á David por
rey sobre Israel, conforme á la palabra de Jehová por mano de Samuel.
\footnote{\textbf{11:3} 1Sam 16,1; 1Sam 16,3; 1Sam 16,12}

\bibverse{4} Entonces se fué David con todo Israel á Jerusalem, la cual
es Jebus; y allí era el Jebuseo habitador de aquella tierra.
\bibverse{5} Y los moradores de Jebus dijeron á David: No entrarás acá.
Mas David tomó la fortaleza de Sión, que es la ciudad de David.
\bibverse{6} Y David había dicho: El que primero hiriere al Jebuseo,
será cabeza y jefe. Entonces Joab hijo de Sarvia subió el primero, y fué
hecho jefe. \bibverse{7} Y David habitó en la fortaleza, y por esto le
llamaron la ciudad de David. \bibverse{8} Y edificó la ciudad alrededor,
desde Millo hasta la cerca: y Joab reparó el resto de la ciudad.
\bibverse{9} Y David iba adelantando y creciendo, y Jehová de los
ejércitos era con él.

\hypertarget{directorio-y-hazauxf1as-de-los-guerreros-de-david}{%
\subsection{Directorio y hazañas de los guerreros de
David}\label{directorio-y-hazauxf1as-de-los-guerreros-de-david}}

\bibverse{10} Estos son los principales de los valientes que David tuvo,
y los que le ayudaron en su reino, con todo Israel, para hacerle rey
sobre Israel, conforme á la palabra de Jehová.

\bibverse{11} Y este es el número de los valientes que David tuvo:
Jasobam hijo de Hachmoni, caudillo de los treinta, el cual blandió su
lanza una vez contra trescientos, á los cuales mató. \footnote{\textbf{11:11}
  1Cró 27,2} \bibverse{12} Tras de éste fué Eleazar hijo de Dodo,
Ahohita, el cual era de los tres valientes. \footnote{\textbf{11:12}
  1Cró 27,4} \bibverse{13} Este estuvo con David en Pasdammin, estando
allí juntos en batalla los Filisteos: y había allí una suerte de tierra
llena de cebada, y huyendo el pueblo delante de los Filisteos,
\bibverse{14} Pusiéronse ellos en medio de la haza, y la defendieron, y
vencieron á los Filisteos; y favoreciólos Jehová con grande salvamento.

\hypertarget{wagnis-dreier-helden}{%
\subsection{Wagnis dreier Helden}\label{wagnis-dreier-helden}}

\bibverse{15} Y tres de los treinta principales descendieron á la peña á
David, á la cueva de Adullam, estando el campo de los Filisteos en el
valle de Raphaim. \footnote{\textbf{11:15} 1Sam 22,1} \bibverse{16} Y
David estaba entonces en la fortaleza, y había á la sazón guarnición de
Filisteos en Beth-lehem. \bibverse{17} David deseó entonces, y dijo:
¡Quién me diera á beber de las aguas del pozo de Beth-lehem, que está á
la puerta!

\bibverse{18} Y aquellos tres rompieron por el campo de los Filisteos, y
sacaron agua del pozo de Beth-lehem, que está á la puerta, y tomaron y
trajéronla á David: mas él no la quiso beber, sino que la derramó á
Jehová, y dijo: \bibverse{19} Guárdeme mi Dios de hacer esto: ¿había yo
de beber la sangre de estos varones con sus vidas, que con peligro de
sus vidas la han traído? Y no la quiso beber. Esto hicieron aquellos
tres valientes.

\hypertarget{abisai-y-benauxedas}{%
\subsection{Abisai y Benaías}\label{abisai-y-benauxedas}}

\bibverse{20} Y Abisai, hermano de Joab, era cabeza de los tres, el cual
blandió su lanza sobre trescientos, á los cuales hirió; y fué entre los
tres nombrado. \bibverse{21} De los tres fué más ilustre que los otros
dos, y fué el principal de ellos: mas no llegó á los tres primeros.

\bibverse{22} Benaías hijo de Joiada, hijo de varón de esfuerzo, de
grandes hechos, de Cabseel: él venció los dos leones de Moab: también
descendió, é hirió un león en mitad de un foso en tiempo de nieve.
\bibverse{23} El mismo venció á un Egipcio, hombre de cinco codos de
estatura: y el Egipcio traía una lanza como un enjullo de tejedor; mas
él descendió á él con un bastón, y arrebató al Egipcio la lanza de la
mano, y matólo con su misma lanza. \bibverse{24} Esto hizo Benaías hijo
de Joiada, y fué nombrado entre los tres valientes. \footnote{\textbf{11:24}
  1Cró 27,5-6} \bibverse{25} Y fué el más honrado de los treinta, mas no
llegó á los tres primeros. A éste puso David en su consejo.

\hypertarget{una-lista-de-otros-huxe9roes-de-david}{%
\subsection{Una lista de otros héroes de
David}\label{una-lista-de-otros-huxe9roes-de-david}}

\bibverse{26} Y los valientes de los ejércitos: Asael hermano de Joab, y
Elchânan hijo de Dodo de Beth-lehem; \bibverse{27} Samoth de Arori,
Helles Pelonita; \bibverse{28} Ira hijo de Acces Tecoita, Abiezer
Anathothita; \bibverse{29} Sibbecai Husatita, Ilai Ahohita;
\bibverse{30} Maharai Nethophathita, Heled hijo de Baana Nethophathita;
\bibverse{31} Ithai hijo de Ribai de Gabaath de los hijos de Benjamín,
Benaías Phirathita; \bibverse{32} Hurai del río Gaas, Abiel Arbathonita;
\bibverse{33} Azmaveth Baharumita, Eliaba Saalbonita; \bibverse{34} Los
hijos de Asem Gizonita, Jonathán hijo de Sajé Hararita; \bibverse{35}
Ahiam hijo de Sachâr Ararita, Eliphal hijo de Ur; \bibverse{36} Hepher
Mechêrathita, Ahía Phelonita; \bibverse{37} Hesro Carmelita, Nahari hijo
de Ezbai; \bibverse{38} Joel hermano de Nathán, Mibhar hijo de Agrai;
\bibverse{39} Selec Ammonita, Naarai Berothita, escudero de Joab hijo de
Sarvia; \bibverse{40} Ira Ithreo, Yared Ithreo; \bibverse{41} Uría
Hetheo, Zabad hijo de Ahli; \footnote{\textbf{11:41} 2Sam 11,3}

\bibverse{42} Adina hijo de Siza Rubenita, príncipe de los Rubenitas, y
con él treinta; \bibverse{43} Hanán hijo de Maachâ, y Josaphat Mithnita;
\bibverse{44} Uzzías Astarothita, Samma y Jehiel hijos de Hotham
Arorita; \bibverse{45} Jediael hijo de Simri, y Joha su hermano,
Thisaita; \bibverse{46} Eliel de Mahaví, Jeribai y Josabia hijos de
Elnaam, é Ithma Moabita; \bibverse{47} Eliel, y Obed, y Jaasiel de
Mesobia.

\hypertarget{los-seguidores-de-david-en-siclag-y-adullam-mientras-sauxfal-todavuxeda-estaba-vivo}{%
\subsection{Los seguidores de David en Siclag y Adullam mientras Saúl
todavía estaba
vivo}\label{los-seguidores-de-david-en-siclag-y-adullam-mientras-sauxfal-todavuxeda-estaba-vivo}}

\hypertarget{section-11}{%
\section{12}\label{section-11}}

\bibverse{1} Estos son los que vinieron á David á Siclag, estando él aún
encerrado por causa de Saúl hijo de Cis, y eran de los valientes
ayudadores de la guerra. \bibverse{2} Estaban armados de arcos, y usaban
de ambas manos en tirar piedras con honda, y saetas con arco. De los
hermanos de Saúl de Benjamín: \footnote{\textbf{12:2} 1Cró 8,40}
\bibverse{3} El principal Ahiezer, después Joas, hijos de Semaa
Gabaathita; y Jeziel, y Pheleth, hijos de Azmaveth, y Beracah, y Jehú
Anathothita; \bibverse{4} E Ismaías Gabaonita, valiente entre los
treinta, y más que los treinta; y Jeremías, Jahaziel, Joanán, Jozabad
Gederathita, \bibverse{5} Eluzai, y Jeremoth, Bealías, Semarías, y
Sephatías Haruphita; \bibverse{6} Elcana, é Isías, y Azareel, y Joezer,
y Jasobam, de Coré; \bibverse{7} Y Joela, y Zebadías, hijos de Jeroham
de Gedor.

\bibverse{8} También de los de Gad se huyeron á David, estando en la
fortaleza en el desierto, muy valientes hombres de guerra para pelear,
dispuestos á hacerlo con escudo y pavés: sus rostros como rostros de
leones, y ligeros como las cabras monteses. \footnote{\textbf{12:8} 2Sam
  2,18} \bibverse{9} Eser el primero, Obadías el segundo, Eliab el
tercero, \bibverse{10} Mismana el cuarto, Jeremías el quinto,
\bibverse{11} Attai el sexto, Eliel el séptimo, \bibverse{12} Johanán el
octavo, Elzabad el nono, \bibverse{13} Jeremías el décimo, Machbani el
undécimo. \bibverse{14} Estos fueron capitanes del ejército de los hijos
de Gad. El menor tenía cargo de cien hombres, y el mayor de mil.
\bibverse{15} Estos pasaron el Jordán en el mes primero, cuando había
salido sobre todas sus riberas; é hicieron huir á todos los de los
valles al oriente y al poniente.

\bibverse{16} Asimismo algunos de los hijos de Benjamín y de Judá
vinieron á David á la fortaleza. \bibverse{17} Y David salió á ellos, y
hablóles diciendo: Si habéis venido á mí para paz y para ayudarme, mi
corazón será unido con vosotros; mas si para engañarme en pro de mis
enemigos, siendo mis manos sin iniquidad, véalo el Dios de nuestros
padres, y demándelo. \bibverse{18} Entonces se envistió el espíritu en
Amasai, príncipe de treinta, y dijo: Por ti, oh David, y contigo, oh
hijo de Isaí. Paz, paz contigo, y paz con tus ayudadores; pues que
también tu Dios te ayuda. Y David los recibió, y púsolos entre los
capitanes de la cuadrilla.

\bibverse{19} También se pasaron á David algunos de Manasés, cuando vino
con los Filisteos á la batalla contra Saúl, aunque no les ayudaron;
porque los sátrapas de los Filisteos, habido consejo, lo despidieron,
diciendo: Con nuestras cabezas se pasará á su señor Saúl.

\bibverse{20} Así que viniendo él á Siclag, se pasaron á él de los de
Manasés, Adnas, Jozabad, Jediaiel, Michâel, Jozabad, Eliú, y Sillethai,
príncipes de millares de los de Manasés. \bibverse{21} Estos ayudaron á
David contra aquella compañía; porque todos ellos eran hombres
valientes, y fueron capitanes en el ejército. \bibverse{22} Porque
entonces todos los días venía ayuda á David, hasta hacerse un grande
ejército, como ejército de Dios.

\bibverse{23} Y este es el número de los principales que estaban á punto
de guerra, y vinieron á David en Hebrón, para traspasarle el reino de
Saúl, conforme á la palabra de Jehová:

\hypertarget{nuxfamero-de-guerreros-en-la-elecciuxf3n-de-david-como-rey-en-hebruxf3n}{%
\subsection{Número de guerreros en la elección de David como rey en
Hebrón}\label{nuxfamero-de-guerreros-en-la-elecciuxf3n-de-david-como-rey-en-hebruxf3n}}

\bibverse{24} De los hijos de Judá que traían escudo y lanza, seis mil y
ochocientos, á punto de guerra. \bibverse{25} De los hijos de Simeón,
valientes y esforzados hombres para la guerra, siete mil y ciento.
\bibverse{26} De los hijos de Leví, cuatro mil y seiscientos;
\bibverse{27} Asimismo Joiada, príncipe de los del linaje de Aarón, y
con él tres mil y setecientos; \bibverse{28} Y Sadoc, mancebo valiente y
esforzado, con veinte y dos de los principales de la casa de su padre.
\footnote{\textbf{12:28} 2Sam 15,24; 1Cró 5,34} \bibverse{29} De los
hijos de Benjamín hermanos de Saúl, tres mil; porque aun en aquel tiempo
muchos de ellos tenían la parte de la casa de Saúl. \bibverse{30} Y de
los hijos de Ephraim, veinte mil y ochocientos, muy valientes, varones
ilustres en las casas de sus padres. \bibverse{31} De la media tribu de
Manasés, diez y ocho mil, los cuales fueron tomados por lista para venir
á poner á David por rey. \bibverse{32} Y de los hijos de Issachâr,
doscientos principales, entendidos en los tiempos, y que sabían lo que
Israel debía hacer, cuyo dicho seguían todos sus hermanos. \bibverse{33}
Y de Zabulón cincuenta mil, que salían á campaña á punto de guerra, con
todas armas de guerra, dispuestos á pelear sin doblez de corazón.
\bibverse{34} Y de Nephtalí mil capitanes, y con ellos treinta y siete
mil con escudo y lanza. \bibverse{35} De los de Dan, dispuestos á
pelear, veinte y ocho mil y seiscientos. \bibverse{36} Y de Aser, á
punto de guerra y aparejados á pelear, cuarenta mil. \bibverse{37} Y de
la otra parte del Jordán, de los Rubenitas y de los de Gad y de la media
tribu de Manasés, ciento y veinte mil con toda suerte de armas de
guerra.

\bibverse{38} Todos estos hombres de guerra, dispuestos para guerrear,
vinieron con corazón perfecto á Hebrón, para poner á David por rey sobre
todo Israel; asimismo todos los demás de Israel estaban de un mismo
ánimo para poner á David por rey. \bibverse{39} Y estuvieron allí con
David tres días comiendo y bebiendo, porque sus hermanos habían
prevenido para ellos. \bibverse{40} Y también los que les eran vecinos,
hasta Issachâr y Zabulón y Nephtalí, trajeron pan en asnos, y camellos,
y mulos, y bueyes; y provisión de harina, masas de higos, y pasas, vino
y aceite, bueyes y ovejas en abundancia, porque en Israel había alegría.

\hypertarget{movilizaciuxf3n-de-todo-el-pueblo-con-fines-de-recuperaciuxf3n.}{%
\subsection{Movilización de todo el pueblo con fines de
recuperación.}\label{movilizaciuxf3n-de-todo-el-pueblo-con-fines-de-recuperaciuxf3n.}}

\hypertarget{section-12}{%
\section{13}\label{section-12}}

\bibverse{1} Entonces David tomó consejo con los capitanes de millares y
de cientos, y con todos los jefes. \bibverse{2} Y dijo David á todo el
congreso de Israel: Si os parece bien y de Jehová nuestro Dios,
enviaremos á todas partes á llamar nuestros hermanos que han quedado en
todas las tierras de Israel, y á los sacerdotes y Levitas que están con
ellos en sus ciudades y ejidos que se junten con nosotros; \bibverse{3}
Y traigamos el arca de nuestro Dios á nosotros, porque desde el tiempo
de Saúl no hemos hecho caso de ella.

\bibverse{4} Y dijo todo el congreso que se hiciese así, porque la cosa
parecía bien á todo el pueblo. \bibverse{5} Entonces juntó David á todo
Israel, desde Sihor de Egipto hasta entrar en Hamath, para que trajesen
el arca de Dios de Chîriath-jearim.

\hypertarget{fracaso-del-plan}{%
\subsection{Fracaso del plan}\label{fracaso-del-plan}}

\bibverse{6} Y subió David con todo Israel á Baala de Chîriath-jearim,
que es en Judá, para pasar de allí el arca de Jehová Dios que habita
entre los querubines, sobre la cual su nombre es invocado. \bibverse{7}
Y lleváronse el arca de Dios de la casa de Abinadab en un carro nuevo; y
Uzza y su hermano guiaban el carro. \bibverse{8} Y David y todo Israel
hacían alegrías delante de Dios con todas sus fuerzas, con canciones,
arpas, salterios, tamboriles, címbalos y trompetas.

\bibverse{9} Y como llegaron á la era de Chidón, Uzza extendió su mano
al arca para tenerla, porque los bueyes se desmandaban. \bibverse{10} Y
el furor de Jehová se encendió contra Uzza, é hiriólo, porque había
extendido su mano al arca: y murió allí delante de Dios. \bibverse{11} Y
David tuvo pesar, porque Jehová había hecho rotura en Uzza; por lo que
llamó aquel lugar Pérez-uzza, hasta hoy.

\hypertarget{el-cajuxf3n-se-encuentra-en-la-casa-de-obed-edom}{%
\subsection{El cajón se encuentra en la casa de
Obed-Edom}\label{el-cajuxf3n-se-encuentra-en-la-casa-de-obed-edom}}

\bibverse{12} Y David temió á Dios aquel día, y dijo: ¿Cómo he de traer
á mi casa el arca de Dios? \bibverse{13} Y no trajo David el arca á su
casa en la ciudad de David, sino llevóla á casa de Obed-edom Getheo.
\bibverse{14} Y el arca de Dios estuvo en casa de Obed-edom, en su casa,
tres meses: y bendijo Jehová la casa de Obed-edom, y todas las cosas que
tenía.

\hypertarget{el-edificio-del-palacio-de-david-y-los-nuevos-matrimonios-sus-guerras-victoriosas-con-los-filisteos}{%
\subsection{El edificio del palacio de David y los nuevos matrimonios;
sus guerras victoriosas con los
filisteos}\label{el-edificio-del-palacio-de-david-y-los-nuevos-matrimonios-sus-guerras-victoriosas-con-los-filisteos}}

\hypertarget{section-13}{%
\section{14}\label{section-13}}

\bibverse{1} E hiram rey de Tiro envió embajadores á David, y madera de
cedro, y albañiles y carpinteros, que le edificasen una casa.
\bibverse{2} Y entendió David que Jehová lo había confirmado por rey
sobre Israel, y que había ensalzado su reino sobre su pueblo Israel.

\hypertarget{los-hijos-de-david-nacidos-en-jerusaluxe9n}{%
\subsection{Los hijos de David nacidos en
Jerusalén}\label{los-hijos-de-david-nacidos-en-jerusaluxe9n}}

\bibverse{3} Entonces David tomó también mujeres en Jerusalem y aun
engendró David hijos é hijas. \bibverse{4} Y estos son los nombres de
los que le nacieron en Jerusalem: Samua, Sobab, Nathán, Salomón,
\bibverse{5} Ibhar, Elisua, Eliphelet, \footnote{\textbf{14:5} 2Sam
  5,17-25} \bibverse{6} Noga, Nepheg, Japhías, \bibverse{7} Elisama,
Beel-iada y Eliphelet.

\hypertarget{dos-batallas-victoriosas-entre-david-y-los-filisteos}{%
\subsection{Dos batallas victoriosas entre David y los
filisteos}\label{dos-batallas-victoriosas-entre-david-y-los-filisteos}}

\bibverse{8} Y oyendo los Filisteos que David había sido ungido por rey
sobre todo Israel, subieron todos los Filisteos en busca de David. Y
como David lo oyó, salió contra ellos. \bibverse{9} Y vinieron los
Filisteos, y extendiéronse por el valle de Raphaim. \bibverse{10}
Entonces David consultó á Dios, diciendo: ¿Subiré contra los Filisteos?
¿los entregarás en mi mano? Y Jehová le dijo: Sube, que yo los entregaré
en tus manos.

\bibverse{11} Subieron pues á Baal-perasim, y allí los hirió David. Dijo
luego David: Dios rompió mis enemigos por mi mano, como se rompen las
aguas. Por esto llamaron el nombre de aquel lugar Baal-perasim.
\bibverse{12} Y dejaron allí sus dioses, y David dijo que los quemasen
al fuego.

\bibverse{13} Y volviendo los Filisteos á extenderse por el valle,
\bibverse{14} David volvió á consultar á Dios, y Dios le dijo: No subas
tras ellos, sino rodéalos, para venir á ellos por delante de los
morales; \bibverse{15} Y así que oyeres venir un estruendo por las copas
de los morales, sal luego á la batalla: porque Dios saldrá delante de
ti, y herirá el campo de los Filisteos.

\bibverse{16} Hizo pues David como Dios le mandó, é hirieron el campo de
los Filisteos desde Gabaón hasta Gezer. \bibverse{17} Y la fama de David
fué divulgada por todas aquellas tierras: y puso Jehová temor de David
sobre todas las gentes.

\hypertarget{preparativos-para-el-traslado-del-arca-sagrada-designaciuxf3n-e-instrucciuxf3n-de-los-levitas-a-cargo}{%
\subsection{Preparativos para el traslado del arca sagrada; Designación
e instrucción de los levitas a
cargo}\label{preparativos-para-el-traslado-del-arca-sagrada-designaciuxf3n-e-instrucciuxf3n-de-los-levitas-a-cargo}}

\hypertarget{section-14}{%
\section{15}\label{section-14}}

\bibverse{1} Hizo también casas para sí en la ciudad de David, y labró
un lugar para el arca de Dios, y tendióle una tienda. \bibverse{2}
Entonces dijo David: El arca de Dios no debe ser traída sino por los
Levitas; porque á ellos ha elegido Jehová para que lleven el arca de
Jehová, y le sirvan perpetuamente.

\bibverse{3} Y juntó David á todo Israel en Jerusalem, para que pasasen
el arca de Jehová á su lugar, el cual le había él preparado.
\bibverse{4} Juntó también David á los hijos de Aarón y á los Levitas:
\bibverse{5} De los hijos de Coath, Uriel el principal, y sus hermanos,
ciento y veinte; \bibverse{6} De los hijos de Merari, Asaías el
principal, y sus hermanos, doscientos y veinte; \bibverse{7} De los
hijos de Gersón, Joel el principal, y sus hermanos, ciento y treinta;
\bibverse{8} De los hijos de Elisaphán, Semeías el principal, y sus
hermanos, doscientos; \bibverse{9} De los hijos de Hebrón, Eliel el
principal, y sus hermanos, ochenta; \bibverse{10} De los hijos de
Uzziel, Aminadab el principal, y sus hermanos, ciento y doce.

\bibverse{11} Y llamó David á Sadoc y á Abiathar, sacerdotes, y á los
Levitas, Uriel, Asaías, Joel, Semeías, Eliel, y Aminadab; \footnote{\textbf{15:11}
  2Sam 15,29} \bibverse{12} Y díjoles: Vosotros que sois los principales
de padres entre los Levitas, santificaos, vosotros y vuestros hermanos,
y pasad el arca de Jehová Dios de Israel al lugar que le he preparado;
\bibverse{13} Pues por no haberlo hecho así vosotros la primera vez,
Jehová nuestro Dios hizo en nosotros rotura, por cuanto no le buscamos
según la ordenanza.

\bibverse{14} Así los sacerdotes y los Levitas se santificaron para
traer el arca de Jehová Dios de Israel. \bibverse{15} Y los hijos de los
Levitas trajeron el arca de Dios puesta sobre sus hombros en las barras,
como lo había mandado Moisés conforme á la palabra de Jehová.
\footnote{\textbf{15:15} Éxod 25,14; Núm 4,15}

\hypertarget{orden-de-los-cantantes-muxfasicos-y-porteadores-levuxedticos}{%
\subsection{Orden de los cantantes, músicos y porteadores
levíticos}\label{orden-de-los-cantantes-muxfasicos-y-porteadores-levuxedticos}}

\bibverse{16} Asimismo dijo David á los principales de los Levitas, que
constituyesen de sus hermanos cantores, con instrumentos de música, con
salterios, y arpas, y címbalos, que resonasen, y alzasen la voz con
alegría. \bibverse{17} Y los Levitas constituyeron á Hemán hijo de Joel;
y de sus hermanos, á Asaph hijo de Berechîas; y de los hijos de Merari y
de sus hermanos, á Ethán hijo de Cusaías; \bibverse{18} Y con ellos á
sus hermanos del segundo orden, á Zachârías, Ben y Jaaziel, Semiramoth,
Jehiel, Unni, Eliab, Benaías, Maasías, y Mathithías, Eliphelehu,
Micnías, Obed-edom, y Jehiel, los porteros. \bibverse{19} Así Hemán,
Asaph, y Ethán, que eran cantores, sonaban con címbalos de metal.
\bibverse{20} Y Zachârías, Jaaziel, Semiramoth, Jehiel, Unni, Eliab,
Maasías, y Benaías, con salterios sobre Alamoth. \bibverse{21} Y
Mathithías, Eliphelehu, Micnías, Obed-edom, Jehiel, y Azazías, cantaban
con arpas en la octava sobresaliendo. \bibverse{22} Y Chênanías,
principal de los Levitas, estaba para la entonación; pues él presidía en
el canto, porque era entendido. \bibverse{23} Y Berechîas y Elcana eran
porteros del arca. \bibverse{24} Y Sebanías, Josaphat, Nathanael,
Amasai, Zachârías, Benaías, y Eliezer, sacerdotes, tocaban las trompetas
delante del arca de Dios: Obed-edom y Jehías eran también porteros del
arca.

\hypertarget{la-participaciuxf3n-personal-de-david-en-la-transferencia-la-fiesta-del-sacrificio-y-la-acciuxf3n-de-gracias}{%
\subsection{La participación personal de David en la transferencia; la
fiesta del sacrificio y la acción de
gracias}\label{la-participaciuxf3n-personal-de-david-en-la-transferencia-la-fiesta-del-sacrificio-y-la-acciuxf3n-de-gracias}}

\bibverse{25} David pues y los ancianos de Israel, y los capitanes de
millares, fueron á traer el arca del pacto de Jehová, de casa de
Obed-edom, con alegría. \footnote{\textbf{15:25} 2Sam 6,12-16}

\bibverse{26} Y ayudando Dios á los Levitas que llevaban el arca del
pacto de Jehová, sacrificaban siete novillos y siete carneros.
\bibverse{27} Y David iba vestido de lino fino y también todos los
Levitas que llevaban el arca, y asimismo los cantores; y Chênanías era
maestro de canto entre los cantores. Llevaba también David sobre sí un
ephod de lino. \bibverse{28} De esta manera llevaba todo Israel el arca
del pacto de Jehová, con júbilo y sonido de bocinas, y trompetas, y
címbalos, y al son de salterios y arpas. \bibverse{29} Y como el arca
del pacto de Jehová llegó á la ciudad de David, Michâl, hija de Saúl,
mirando por una ventana, vió al rey David que saltaba y bailaba; y
menospreciólo en su corazón.

\hypertarget{section-15}{%
\section{16}\label{section-15}}

\bibverse{1} Así trajeron el arca de Dios, y asentáronla en medio de la
tienda que David había tendido para ella: y ofrecieron holocaustos y
pacíficos delante de Dios. \bibverse{2} Y como David hubo acabado de
ofrecer el holocausto y los pacíficos, bendijo al pueblo en el nombre de
Jehová. \bibverse{3} Y repartió á todo Israel, así á hombres como á
mujeres, á cada uno una torta de pan, y una pieza de carne, y un frasco
de vino.

\hypertarget{orden-del-servicio-de-canto-y-muxfasica-en-el-arca}{%
\subsection{Orden del servicio de canto y música en el
Arca}\label{orden-del-servicio-de-canto-y-muxfasica-en-el-arca}}

\bibverse{4} Y puso delante del arca de Jehová ministros de los Levitas,
para que recordasen, y confesasen, y loasen á Jehová Dios de Israel:
\bibverse{5} Asaph el primero, el segundo después de él Zachârías,
Jeiel, Semiramoth, Jehiel, Mathithías, Eliab, Benaías, Obed-edom, y
Jehiel, con sus instrumentos de salterios y arpas; mas Asaph hacía
sonido con címbalos: \bibverse{6} Benaías también y Jahaziel,
sacerdotes, continuamente con trompetas delante del arca del pacto de
Dios.

\hypertarget{canto-de-agradecimiento-y-alabanza-de-david}{%
\subsection{Canto de agradecimiento y alabanza de
David}\label{canto-de-agradecimiento-y-alabanza-de-david}}

\bibverse{7} Entonces, en aquel día, dió David principio á celebrar á
Jehová por mano de Asaph y de sus hermanos: \bibverse{8} Confesad á
Jehová, invocad su nombre, haced notorias en los pueblos sus obras.
\footnote{\textbf{16:8} Sal 105,1-15} \bibverse{9} Cantad á él, cantadle
salmos; hablad de todas sus maravillas. \bibverse{10} Gloriaos en su
santo nombre; alégrese el corazón de los que buscan á Jehová.
\bibverse{11} Buscad á Jehová y su fortaleza; buscad su rostro
continuamente. \bibverse{12} Haced memoria de sus maravillas que ha
obrado, de sus prodigios, y de los juicios de su boca, \bibverse{13} Oh
vosotros, simiente de Israel su siervo, hijos de Jacob, sus escogidos.
\bibverse{14} Jehová, él es nuestro Dios; sus juicios en toda la tierra.
\bibverse{15} Haced memoria de su alianza perpetuamente, y de la palabra
que él mandó en mil generaciones; \bibverse{16} Del pacto que concertó
con Abraham, y de su juramento á Isaac; \bibverse{17} El cual confirmó á
Jacob por estatuto, y á Israel por pacto sempiterno, \bibverse{18}
Diciendo: A ti daré la tierra de Canaán, suerte de vuestra herencia;
\bibverse{19} Cuando erais pocos en número, pocos y peregrinos en ella;
\bibverse{20} Y andaban de nación en nación, y de un reino á otro
pueblo. \bibverse{21} No permitió que nadie los oprimiese: antes por
amor de ellos castigó á los reyes. \bibverse{22} No toquéis, dijo, á mis
ungidos, ni hagáis mal á mis profetas. \bibverse{23} Cantad á Jehová,
toda la tierra, anunciad de día en día su salud. \footnote{\textbf{16:23}
  Sal 96,-1} \bibverse{24} Cantad entre las gentes su gloria, y en todos
los pueblos sus maravillas. \bibverse{25} Porque grande es Jehová, y
digno de ser grandemente loado, y de ser temido sobre todos los dioses.
\bibverse{26} Porque todos los dioses de los pueblos son nada: mas
Jehová hizo los cielos. \bibverse{27} Poderío y hermosura delante de él;
fortaleza y alegría en su morada. \bibverse{28} Atribuid á Jehová, oh
familias de los pueblos, atribuid á Jehová gloria y potencia.
\bibverse{29} Tributad á Jehová la gloria debida á su nombre: traed
ofrenda, y venid delante de él; postraos delante de Jehová en la
hermosura de su santidad. \bibverse{30} Temed en su presencia, toda la
tierra: el mundo será aún establecido, para que no se conmueva.
\bibverse{31} Alégrense los cielos, y gócese la tierra, y digan en las
naciones: Reina Jehová. \bibverse{32} Resuene la mar, y la plenitud de
ella: alégrese el campo, y todo lo que contiene. \bibverse{33} Entonces
cantarán los árboles de los bosques delante de Jehová, porque viene á
juzgar la tierra. \bibverse{34} Celebrad á Jehová, porque es bueno;
porque su misericordia es eterna. \footnote{\textbf{16:34} Sal 106,47-48}
\bibverse{35} Y decid: Sálvanos, oh Dios, salud nuestra: júntanos, y
líbranos de las gentes, para que confesemos tu santo nombre, y nos
gloriemos en tus alabanzas. \bibverse{36} Bendito sea Jehová Dios de
Israel, de eternidad á eternidad. Y dijo todo el pueblo, Amén: y alabó á
Jehová.

\hypertarget{establecimiento-del-servicio-de-portero-sacerdote-y-cantante-en-el-arca-fin-del-festival}{%
\subsection{Establecimiento del servicio de portero, sacerdote y
cantante en el arca; Fin del
festival}\label{establecimiento-del-servicio-de-portero-sacerdote-y-cantante-en-el-arca-fin-del-festival}}

\bibverse{37} Y dejó allí, delante del arca del pacto de Jehová, á Asaph
y á sus hermanos, para que ministrasen de continuo delante del arca,
cada cosa en su día: \bibverse{38} Y á Obed-edom y á sus hermanos,
sesenta y ocho; y á Obed-edom hijo de Jeduthún, y á Asa, por porteros:
\bibverse{39} Asimismo á Sadoc el sacerdote, y á sus hermanos los
sacerdotes, delante del tabernáculo de Jehová en el alto que estaba en
Gabaón, \footnote{\textbf{16:39} 1Cró 21,29} \bibverse{40} Para que
sacrificasen continuamente, á mañana y tarde, holocaustos á Jehová en el
altar del holocausto, conforme á todo lo que está escrito en la ley de
Jehová, que él prescribió á Israel; \footnote{\textbf{16:40} Éxod
  29,38-39}

\bibverse{41} Y con ellos á Hemán y á Jeduthún, y los otros escogidos
declarados por sus nombres, para glorificar á Jehová, porque es eterna
su misericordia; \bibverse{42} Con ellos á Hemán y á Jeduthún con
trompetas y címbalos para tañer, y con otros instrumentos de música de
Dios; y á los hijos de Jeduthún, por porteros. \bibverse{43} Y todo el
pueblo se fué cada uno á su casa; y David se volvió para bendecir su
casa.

\hypertarget{natuxe1n-aprueba-el-plan-de-david-para-construir-el-templo}{%
\subsection{Natán aprueba el plan de David para construir el
templo}\label{natuxe1n-aprueba-el-plan-de-david-para-construir-el-templo}}

\hypertarget{section-16}{%
\section{17}\label{section-16}}

\bibverse{1} Y aconteció que morando David en su casa, dijo David al
profeta Nathán: He aquí yo habito en casa de cedro, y el arca del pacto
de Jehová debajo de cortinas.

\bibverse{2} Y Nathán dijo á David: Haz todo lo que está en tu corazón,
porque Dios es contigo.

\hypertarget{dios-rechaza-el-plan-el-discurso-profuxe9tico-de-nathan-el-templo-seruxe1-construido-por-el-hijo-de-david}{%
\subsection{Dios rechaza el plan; El discurso profético de Nathan; el
templo será construido por el hijo de
David}\label{dios-rechaza-el-plan-el-discurso-profuxe9tico-de-nathan-el-templo-seruxe1-construido-por-el-hijo-de-david}}

\bibverse{3} En aquella misma noche fué palabra de Dios á Nathán,
diciendo: \bibverse{4} Ve y di á David mi siervo: Así ha dicho Jehová:
Tú no me edificarás casa en que habite: \bibverse{5} Porque no he
habitado en casa alguna desde el día que saqué á los hijos de Israel
hasta hoy; antes estuve de tienda en tienda, y de tabernáculo en
tabernáculo. \bibverse{6} En todo cuanto anduve con todo Israel ¿hablé
una palabra á alguno de los jueces de Israel, á los cuales mandé que
apacentasen mi pueblo, para decirles: Por qué no me edificáis una casa
de cedro?

\bibverse{7} Por tanto, ahora dirás á mi siervo David: Así dijo Jehová
de los ejércitos: Yo te tomé de la majada, de detrás del ganado, para
que fueses príncipe sobre mi pueblo Israel; \bibverse{8} Y he sido
contigo en todo cuanto has andado, y he talado á todos tus enemigos de
delante de ti, y hete hecho grande nombre, como el nombre de los grandes
que son en la tierra. \bibverse{9} Asimismo he dispuesto lugar á mi
pueblo Israel, y lo he plantado para que habite por sí, y que no sea más
conmovido: ni los hijos de iniquidad lo consumirán más, como antes,
\bibverse{10} Y desde el tiempo que puse los jueces sobre mi pueblo
Israel; mas humillaré á todos tus enemigos. Hágote además saber que
Jehová te ha de edificar casa. \bibverse{11} Y será que, cuando tus días
fueren cumplidos para irte con tus padres, levantaré tu simiente después
de ti, la cual será de tus hijos, y afirmaré su reino. \bibverse{12} El
me edificará casa, y yo confirmaré su trono eternalmente. \footnote{\textbf{17:12}
  1Cró 22,10; 1Cró 28,6}

\hypertarget{la-gran-proclamaciuxf3n-de-salvaciuxf3n-de-dios-a-david-con-respecto-a-la-duraciuxf3n-eterna-de-su-casa}{%
\subsection{La gran proclamación de salvación de Dios a David con
respecto a la duración eterna de su
casa}\label{la-gran-proclamaciuxf3n-de-salvaciuxf3n-de-dios-a-david-con-respecto-a-la-duraciuxf3n-eterna-de-su-casa}}

\bibverse{13} Yo le seré por padre, y él me será por hijo: y no quitaré
de él mi misericordia, como la quité de aquel que fué antes de ti;
\bibverse{14} Mas yo lo confirmaré en mi casa y en mi reino
eternalmente; y su trono será firme para siempre.

\hypertarget{acciuxf3n-de-gracias-y-suxfaplica-de-david}{%
\subsection{Acción de gracias y súplica de
David}\label{acciuxf3n-de-gracias-y-suxfaplica-de-david}}

\bibverse{15} Conforme á todas estas palabras, y conforme á toda esta
visión, así habló Nathán á David.

\bibverse{16} Y entró el rey David, y estuvo delante de Jehová, y dijo:
Jehová Dios, ¿quién soy yo, y cuál es mi casa, que me has traído hasta
este lugar? \bibverse{17} Y aun esto, oh Dios, te ha parecido poco, pues
que has hablado de la casa de tu siervo para más lejos, y me has mirado
como á un hombre excelente, oh Jehová Dios. \bibverse{18} ¿Qué más puede
añadir David pidiendo de ti para glorificar á tu siervo? mas tú conoces
á tu siervo. \bibverse{19} Oh Jehová, por amor de tu siervo y según tu
corazón, has hecho toda esta grandeza, para hacer notorias todas tus
grandezas. \bibverse{20} Jehová, no hay semejante á ti, ni hay Dios sino
tú, según todas las cosas que hemos oído con nuestros oídos.
\bibverse{21} ¿Y qué gente hay en la tierra como tu pueblo Israel, cuyo
Dios fuese y se redimiera un pueblo, para hacerte nombre con grandezas y
maravillas, echando las gentes de delante de tu pueblo, que tú
rescataste de Egipto? \bibverse{22} Tú has constituído á tu pueblo
Israel por pueblo tuyo para siempre; y tú, Jehová, has venido á ser su
Dios. \bibverse{23} Ahora pues, Jehová, la palabra que has hablado
acerca de tu siervo y de su casa, sea firme para siempre, y haz como has
dicho. \bibverse{24} Permanezca pues, y sea engrandecido tu nombre para
siempre, á fin de que se diga: Jehová de los ejércitos, Dios de Israel,
es Dios para Israel. Y sea la casa de tu siervo David firme delante de
ti. \bibverse{25} Porque tú, Dios mío, revelaste al oído á tu siervo que
le has de edificar casa; por eso ha hallado tu siervo motivo de orar
delante de ti. \bibverse{26} Ahora pues, Jehová, tú eres el Dios que has
hablado de tu siervo este bien; \bibverse{27} Y ahora has querido
bendecir la casa de tu siervo, para que permanezca perpetuamente delante
de ti: porque tú, Jehová, la has bendecido, y será bendita para siempre.

\hypertarget{las-victorias-de-david-sobre-los-filisteos-moabitas-sirios-y-edomitas}{%
\subsection{Las victorias de David sobre los filisteos, moabitas, sirios
y
edomitas}\label{las-victorias-de-david-sobre-los-filisteos-moabitas-sirios-y-edomitas}}

\hypertarget{section-17}{%
\section{18}\label{section-17}}

\bibverse{1} Después de estas cosas aconteció que David hirió á los
Filisteos, y los humilló; y tomó á Gath y sus villas de mano de los
Filisteos. \bibverse{2} También hirió á Moab; y los Moabitas fueron
siervos de David trayéndole presentes.

\hypertarget{las-victorias-de-david-sobre-los-sirios-el-uso-del-botuxedn-felicitaciones-del-rey-tou}{%
\subsection{Las victorias de David sobre los sirios; el uso del botín;
Felicitaciones del rey
Tou}\label{las-victorias-de-david-sobre-los-sirios-el-uso-del-botuxedn-felicitaciones-del-rey-tou}}

\bibverse{3} Asimismo hirió David á Adarezer rey de Soba, en Hamath,
yendo él á asegurar su dominio al río de Eufrates. \bibverse{4} Y
tomóles David mil carros, y siete mil de á caballo, y veinte mil hombres
de á pie: y desjarretó David los caballos de todos los carros, excepto
los de cien carros que dejó. \bibverse{5} Y viniendo los Siros de
Damasco en ayuda de Adarezer rey de Soba, David hirió de los Siros
veintidós mil hombres. \bibverse{6} Y puso David guarnición en Siria la
de Damasco, y los Siros fueron hechos siervos de David, trayéndole
presentes: porque Jehová salvaba á David donde quiera que iba.
\bibverse{7} Tomó también David los escudos de oro que llevaban los
siervos de Adarezer, y trájolos á Jerusalem. \bibverse{8} Asimismo de
Thibath y de Chûn ciudades de Adarezer, tomó David muy mucho metal, de
que Salomón hizo el mar de bronce, las columnas, y vasos de bronce.
\footnote{\textbf{18:8} 1Re 7,23; 1Re 7,15}

\bibverse{9} Y oyendo Tou rey de Hamath, que David había deshecho todo
el ejército de Adarezer, rey de Soba, \bibverse{10} Envió á Adoram su
hijo al rey David, á saludarle y á bendecirle por haber peleado con
Adarezer, y haberle vencido; porque Tou tenía guerra con Adarezer.
Envióle también toda suerte de vasos de oro, de plata y de metal;
\bibverse{11} Los cuales el rey David dedicó á Jehová, con la plata y
oro que había tomado de todas las naciones, de Edom, de Moab, de los
hijos de Ammón, de los Filisteos, y de Amalec.

\hypertarget{derrota-y-subyugaciuxf3n-de-los-edomitas}{%
\subsection{Derrota y subyugación de los
edomitas}\label{derrota-y-subyugaciuxf3n-de-los-edomitas}}

\bibverse{12} A más de esto Abisai hijo de Sarvia hirió en el valle de
la Sal dieciocho mil Idumeos. \bibverse{13} Y puso guarnición en Edom, y
todos los Idumeos fueron siervos de David: porque Jehová guardaba á
David donde quiera que iba.

\hypertarget{los-altos-funcionarios-de-david}{%
\subsection{Los altos funcionarios de
David}\label{los-altos-funcionarios-de-david}}

\bibverse{14} Y reinó David sobre todo Israel, y hacía juicio y justicia
á todo su pueblo. \bibverse{15} Y Joab hijo de Sarvia era general del
ejército; y Josaphat hijo de Ahilud, canciller; \bibverse{16} Y Sadoc
hijo de Achîtob, y Abimelec hijo de Abiathar, eran sacerdotes; y Sausa,
secretario; \bibverse{17} Y Benaías hijo de Joiada era sobre los
Ceretheos y Peletheos; y los hijos de David eran los príncipes cerca del
rey.

\hypertarget{el-vergonzoso-crimen-de-los-amonitas-contra-el-mensajero-de-david}{%
\subsection{El vergonzoso crimen de los amonitas contra el mensajero de
David}\label{el-vergonzoso-crimen-de-los-amonitas-contra-el-mensajero-de-david}}

\hypertarget{section-18}{%
\section{19}\label{section-18}}

\bibverse{1} Después de estas cosas aconteció que murió Naas rey de los
hijos de Ammón, y reinó en su lugar su hijo. \bibverse{2} Y dijo David:
Haré misericordia con Hanán hijo de Naas, porque también su padre hizo
conmigo misericordia. Así David envió embajadores que lo consolasen de
la muerte de su padre. Mas venidos los siervos de David en la tierra de
los hijos de Ammón á Hanán, para consolarle,

\bibverse{3} Los príncipes de los hijos de Ammón dijeron á Hanán: ¿A tu
parecer honra David á tu padre, que te ha enviado consoladores? ¿no
vienen antes sus siervos á ti para escudriñar, é inquirir, y reconocer
la tierra? \footnote{\textbf{19:3} 1Sam 3,18} \bibverse{4} Entonces
Hanán tomó los siervos de David, y rapólos, y cortóles los vestidos por
medio, hasta las nalgas, y despachólos. \bibverse{5} Fuéronse pues, y
dada que fué la nueva á David de aquellos varones, él envió á
recibirlos, porque estaban muy afrentados. E hízoles decir el rey:
Estaos en Jericó hasta que os crezca la barba, y entonces volveréis.

\hypertarget{comienzo-de-la-guerra-primeros-trabajos-ganados}{%
\subsection{Comienzo de la guerra; primeros trabajos
ganados}\label{comienzo-de-la-guerra-primeros-trabajos-ganados}}

\bibverse{6} Y viendo los hijos de Ammón que se habían hecho odiosos á
David, Hanán y los hijos de Ammón enviaron mil talentos de plata, para
tomar á sueldo carros y gente de á caballo de Siria de los ríos, y de la
Siria de Maachâ, y de Soba. \bibverse{7} Y tomaron á sueldo treinta y
dos mil carros, y al rey de Maachâ y á su pueblo, los cuales vinieron y
asentaron su campo delante de Medeba. Y juntáronse también los hijos de
Ammón de sus ciudades, y vinieron á la guerra. \bibverse{8} Oyéndolo
David, envió á Joab con todo el ejército de los hombres valientes.
\bibverse{9} Y los hijos de Ammón salieron, y ordenaron su tropa á la
entrada de la ciudad; y los reyes que habían venido, estaban por sí en
el campo.

\bibverse{10} Y viendo Joab que la haz de la batalla estaba contra él
delante y á las espaldas, escogió de todos los más aventajados que había
en Israel, y ordenó su escuadrón contra los Siros. \bibverse{11} Puso
luego el resto de la gente en mano de Abisai su hermano, ordenándolos en
batalla contra los Ammonitas. \bibverse{12} Y dijo: Si los Siros fueren
más fuertes que yo, tú me salvarás; y si los Ammonitas fueren más
fuertes que tú, yo te salvaré. \bibverse{13} Esfuérzate, y esforcémonos
por nuestro pueblo, y por las ciudades de nuestro Dios; y haga Jehová lo
que bien le pareciere.

\bibverse{14} Acercóse luego Joab y el pueblo que tenía consigo, para
pelear contra los Siros; mas ellos huyeron delante de él. \bibverse{15}
Y los hijos de Ammón, viendo que los Siros habían huído, huyeron también
ellos delante de Abisai su hermano, y entráronse en la ciudad. Entonces
Joab se volvió á Jerusalem.

\hypertarget{david-personalmente-en-el-campo-su-victoria-sobre-los-sirios-aliados-con-los-amonitas}{%
\subsection{David personalmente en el campo; su victoria sobre los
sirios aliados con los
amonitas}\label{david-personalmente-en-el-campo-su-victoria-sobre-los-sirios-aliados-con-los-amonitas}}

\bibverse{16} Y viendo los Siros que habían caído delante de Israel,
enviaron embajadores, y trajeron á los Siros que estaban de la otra
parte del río, cuyo capitán era Sophach, general del ejército de
Adarezer. \bibverse{17} Luego que fué dado aviso á David, juntó á todo
Israel, y pasando el Jordán vino á ellos, y ordenó contra ellos su
ejército. Y como David hubo ordenado su tropa contra ellos, pelearon con
él los Siros. \bibverse{18} Mas el Siro huyó delante de Israel; y mató
David de los Siros siete mil hombres de los carros, y cuarenta mil
hombres de á pie: asimismo mató á Sophach, general del ejército.
\bibverse{19} Y viendo los Siros de Adarezer que habían caído delante de
Israel, concertaron paz con David, y fueron sus siervos; y nunca más
quiso el Siro ayudar á los hijos de Ammón.

\hypertarget{joab-conquista-rabuxe1-el-triunfo-de-david-y-el-castigo-de-los-amonitas}{%
\subsection{Joab conquista Rabá; El triunfo de David y el castigo de los
amonitas}\label{joab-conquista-rabuxe1-el-triunfo-de-david-y-el-castigo-de-los-amonitas}}

\hypertarget{section-19}{%
\section{20}\label{section-19}}

\bibverse{1} Y aconteció á la vuelta del año, en el tiempo que suelen
los reyes salir á la guerra, que Joab sacó las fuerzas del ejército, y
destruyó la tierra de los hijos de Ammón, y vino y cercó á Rabba. Mas
David estaba en Jerusalem: y Joab batió á Rabba, y destruyóla.
\bibverse{2} Y tomó David la corona de su rey de encima de su cabeza, y
hallóla de peso de un talento de oro, y había en ella piedras preciosas;
y fué puesta sobre la cabeza de David. Y además de esto sacó de la
ciudad un muy gran despojo. \bibverse{3} Sacó también al pueblo que
estaba en ella, y cortólos con sierras, y con trillos de hierro, y
segures. Lo mismo hizo David á todas las ciudades de los hijos de Ammón.
Y volvióse David con todo el pueblo á Jerusalem.

\hypertarget{algunas-hazauxf1as-de-los-guerreros-de-david-en-las-guerras-filisteas}{%
\subsection{Algunas hazañas de los guerreros de David en las guerras
filisteas}\label{algunas-hazauxf1as-de-los-guerreros-de-david-en-las-guerras-filisteas}}

\bibverse{4} Después de esto aconteció que se levantó guerra en Gezer
con los Filisteos; é hirió Sibbecai Husathita á Sippai, del linaje de
los gigantes; y fueron humillados.

\bibverse{5} Y volvióse á levantar guerra con los Filisteos; é hirió
Elhanán hijo de Jair á Lahmi, hermano de Goliath Getheo, el asta de cuya
lanza era como un enjullo de tejedores. \bibverse{6} Y volvió á haber
guerra en Gath, donde hubo un hombre de grande estatura, el cual tenía
seis dedos en pies y manos, en todos veinticuatro: y también era hijo de
Rapha. \bibverse{7} Denostó él á Israel, mas hiriólo Jonathán, hijo de
Sima hermano de David. \footnote{\textbf{20:7} 1Sam 17,10}

\bibverse{8} Estos fueron hijos de Rapha en Gath, los cuales cayeron por
mano de David y de sus siervos.

\hypertarget{david-a-instigaciuxf3n-de-satanuxe1s-completa-el-censo-a-pesar-de-la-advertencia-de-joab-resultado-del-recuento}{%
\subsection{David, a instigación de Satanás, completa el censo a pesar
de la advertencia de Joab; Resultado del
recuento}\label{david-a-instigaciuxf3n-de-satanuxe1s-completa-el-censo-a-pesar-de-la-advertencia-de-joab-resultado-del-recuento}}

\hypertarget{section-20}{%
\section{21}\label{section-20}}

\bibverse{1} Mas Satanás se levantó contra Israel, é incitó á David á
que contase á Israel. \bibverse{2} Y dijo David á Joab y á los príncipes
del pueblo: Id, contad á Israel desde Beer-seba hasta Dan, y traedme el
número de ellos para que yo lo sepa.

\bibverse{3} Y dijo Joab: Añada Jehová á su pueblo cien veces otros
tantos. Rey señor mío, ¿no son todos estos siervos de mi señor? ¿para
qué procura mi señor esto, que será pernicioso á Israel?

\bibverse{4} Mas el mandamiento del rey pudo más que Joab. Salió por
tanto Joab, y fué por todo Israel; y volvió á Jerusalem, y dió la cuenta
del número del pueblo á David. \bibverse{5} Y hallóse en todo Israel que
sacaban espada, once veces cien mil; y de Judá cuatrocientos y setenta
mil hombres que sacaban espada. \bibverse{6} Entre estos no fueron
contados los Levitas, ni los hijos de Benjamín, porque Joab abominaba el
mandamiento del rey.

\hypertarget{el-arrepentimiento-de-david-intervenciuxf3n-del-profeta-gad-david-elige-una-muerte-popular-para-expiar-su-culpa}{%
\subsection{El arrepentimiento de David; Intervención del profeta Gad;
David elige una muerte popular para expiar su
culpa}\label{el-arrepentimiento-de-david-intervenciuxf3n-del-profeta-gad-david-elige-una-muerte-popular-para-expiar-su-culpa}}

\bibverse{7} Asimismo desagradó este negocio á los ojos de Dios, é hirió
á Israel. \footnote{\textbf{21:7} 1Cró 27,24} \bibverse{8} Y dijo David
á Dios: He pecado gravemente en hacer esto: ruégote que hagas pasar la
iniquidad de tu siervo, porque yo he hecho muy locamente.

\bibverse{9} Y habló Jehová á Gad, vidente de David, diciendo:
\bibverse{10} Ve, y habla á David, y dile: Así ha dicho Jehová: Tres
cosas te propongo; escoge de ellas una que yo haga contigo.

\bibverse{11} Y viniendo Gad á David, díjole: Así ha dicho Jehová:
\bibverse{12} Escógete, ó tres años de hambre; ó ser por tres meses
deshecho delante de tus enemigos, y que la espada de tus adversarios te
alcance; ó por tres días la espada de Jehová y pestilencia en la tierra,
y que el ángel de Jehová destruya en todo el término de Israel: mira
pues qué he de responder al que me ha enviado.

\bibverse{13} Entonces David dijo á Gad: Estoy en grande angustia: ruego
que yo caiga en la mano de Jehová; porque sus misericordias son muchas
en extremo, y que no caiga yo en manos de hombres.

\hypertarget{el-juicio-divino-la-penitencia-y-la-suxfaplica-de-david}{%
\subsection{El juicio divino; La penitencia y la súplica de
David}\label{el-juicio-divino-la-penitencia-y-la-suxfaplica-de-david}}

\bibverse{14} Así Jehová dió pestilencia en Israel, y cayeron de Israel
setenta mil hombres. \bibverse{15} Y envió Jehová el ángel á Jerusalem
para destruirla: pero estando él destruyendo, miró Jehová, y
arrepintióse de aquel mal, \bibverse{16} Y dijo al ángel que destruía:
Basta ya; detén tu mano. Y el ángel de Jehová estaba junto á la era de
Ornán Jebuseo.

\bibverse{17} Y alzando David sus ojos, vió al ángel de Jehová, que
estaba entre el cielo y la tierra, teniendo un espada desnuda en su
mano, extendida contra Jerusalem. Entonces David y los ancianos se
postraron sobre sus rostros, cubiertos de sacos.

\hypertarget{david-adquiere-la-era-de-ornuxe1n-y-la-dedica-a-un-lugar-de-sacrificio-y-templo-fin-de-la-plaga}{%
\subsection{David adquiere la era de Ornán y la dedica a un lugar de
sacrificio y templo; Fin de la
plaga}\label{david-adquiere-la-era-de-ornuxe1n-y-la-dedica-a-un-lugar-de-sacrificio-y-templo-fin-de-la-plaga}}

\bibverse{18} Y dijo David á Dios: ¿No soy yo el que hizo contar el
pueblo? Yo mismo soy el que pequé, y ciertamente he hecho mal; mas estas
ovejas, ¿qué han hecho? Jehová Dios mío, sea ahora tu mano contra mí, y
contra la casa de mi padre, y no haya plaga en tu pueblo. \bibverse{19}
Y el ángel de Jehová ordenó á Gad que dijese á David, que subiese y
construyese un altar á Jehová en la era de Ornán Jebuseo.

\bibverse{20} Entonces David subió, conforme á la palabra de Gad que le
había dicho en nombre de Jehová. \bibverse{21} Y volviéndose Ornán vió
al ángel; por lo que se escondieron cuatro hijos suyos que con él
estaban. Y Ornán trillaba el trigo.

\bibverse{22} Y viniendo David á Ornán, miró éste, y vió á David: y
saliendo de la era, postróse en tierra á David.

\bibverse{23} Entonces dijo David á Ornán: Dame este lugar de la era, en
que edifique un altar á Jehová, y dámelo por su cabal precio, para que
cese la plaga del pueblo.

\bibverse{24} Y Ornán respondió á David: Tómalo para ti, y haga mi señor
el rey lo que bien le pareciere: y aun los bueyes daré para el
holocausto, y los trillos para leña, y trigo para el presente: yo lo doy
todo.

\bibverse{25} Entonces el rey David dijo á Ornán: No, sino que
efectivamente la compraré por su justo precio: porque no tomaré para
Jehová lo que es tuyo, ni sacrificaré holocausto que nada me cueste.
\bibverse{26} Y dió David á Ornán por el lugar seiscientos siclos de oro
por peso. \footnote{\textbf{21:26} 1Re 18,24}

\bibverse{27} Y edificó allí David un altar á Jehová, en el que ofreció
holocaustos y sacrificios pacíficos, é invocó á Jehová, el cual le
respondió por fuego de los cielos en el altar del holocausto.

\bibverse{28} Y como Jehová habló al ángel, él volvió su espada á la
vaina. \bibverse{29} Entonces viendo David que Jehová le había oído en
la era de Ornán Jebuseo, sacrificó allí. \bibverse{30} Y el tabernáculo
de Jehová que Moisés había hecho en el desierto, y el altar del
holocausto, estaban entonces en el alto de Gabaón: Mas David no pudo ir
allá á consultar á Dios, porque estaba espantado á causa de la espada
del ángel de Jehová. \footnote{\textbf{21:30} 1Cró 21,16}

\hypertarget{section-21}{%
\section{22}\label{section-21}}

\bibverse{1} Y dijo David: Esta es la casa de Jehová Dios, y este es el
altar del holocausto para Israel. \footnote{\textbf{22:1} 2Cró 3,1}

\hypertarget{los-preparativos-de-david-para-la-construcciuxf3n-del-templo-colecciuxf3n-de-materiales-de-construcciuxf3n}{%
\subsection{Los preparativos de David para la construcción del templo;
Colección de materiales de
construcción}\label{los-preparativos-de-david-para-la-construcciuxf3n-del-templo-colecciuxf3n-de-materiales-de-construcciuxf3n}}

\bibverse{2} Después mandó David que se juntasen los extranjeros que
estaban en la tierra de Israel, y señaló de ellos canteros que labrasen
piedras para edificar la casa de Dios. \footnote{\textbf{22:2} 2Cró 2,16}
\bibverse{3} Asimismo aparejó David mucho hierro para la clavazón de las
puertas, y para las junturas; y mucho metal sin peso, y madera de cedro
sin cuenta. \bibverse{4} Porque los Sidonios y Tirios habían traído á
David madera de cedro innumerable. \bibverse{5} Y dijo David: Salomón mi
hijo es muchacho y tierno, y la casa que se ha de edificar á Jehová ha
de ser magnífica por excelencia, para nombre y honra en todas las
tierras; ahora pues yo le aparejaré lo necesario. Y preparó David antes
de su muerte en grande abundancia.

\hypertarget{instrucciones-de-david-a-su-hijo-salomuxf3n}{%
\subsection{Instrucciones de David a su hijo
Salomón}\label{instrucciones-de-david-a-su-hijo-salomuxf3n}}

\bibverse{6} Llamó entonces David á Salomón su hijo, y mandóle que
edificase casa á Jehová Dios de Israel. \bibverse{7} Y dijo David á
Salomón: Hijo mío, en mi corazón tuve el edificar templo al nombre de
Jehová mi Dios. \footnote{\textbf{22:7} 1Cró 17,1-14; 1Cró 28,2-7}
\bibverse{8} Mas vino á mí palabra de Jehová, diciendo: Tú has derramado
mucha sangre, y has traído grandes guerras: no edificarás casa á mi
nombre, porque has derramado mucha sangre en la tierra delante de mí:
\bibverse{9} He aquí, un hijo te nacerá, el cual será varón de reposo,
porque yo le daré quietud de todos sus enemigos en derredor; por tanto
su nombre será Salomón; y yo daré paz y reposo sobre Israel en sus días:
\bibverse{10} El edificará casa á mi nombre, y él me será á mí por hijo,
y yo le seré por padre; y afirmaré el trono de su reino sobre Israel
para siempre. \bibverse{11} Ahora pues, hijo mío, sea contigo Jehová, y
seas prosperado, y edifiques casa á Jehová tu Dios, como él ha dicho de
ti. \bibverse{12} Y Jehová te dé entendimiento y prudencia, y él te dé
mandamientos para Israel; y que tú guardes la ley de Jehová tu Dios.
\bibverse{13} Entonces serás prosperado, si cuidares de poner por obra
los estatutos y derechos que Jehová mandó á Moisés para Israel.
Esfuérzate pues, y cobra ánimo; no temas, ni desmayes. \bibverse{14} He
aquí, yo en mi estrechez he prevenido para la casa de Jehová cien mil
talentos de oro, y un millar de millares de talentos de plata: no tiene
peso el metal ni el hierro, porque es mucho. Asimismo he aprestado
madera y piedra, á lo cual tú añadirás. \footnote{\textbf{22:14} 1Cró
  29,2} \bibverse{15} Tú tienes contigo muchos oficiales, canteros,
albañiles, y carpinteros, y todo hombre experto en toda obra.
\bibverse{16} Del oro, de la plata, del metal, y del hierro, no hay
número. Levántate pues, y á la obra; que Jehová será contigo.

\hypertarget{la-amonestaciuxf3n-de-david-a-los-pruxedncipes-de-israel}{%
\subsection{La amonestación de David a los príncipes de
Israel}\label{la-amonestaciuxf3n-de-david-a-los-pruxedncipes-de-israel}}

\bibverse{17} Asimismo mandó David á todos los principales de Israel que
diesen ayuda á Salomón su hijo, diciendo: \bibverse{18} ¿No es con
vosotros Jehová vuestro Dios, el cual os ha dado quietud de todas
partes? porque él ha entregado en mi mano los moradores de la tierra, y
la tierra ha sido sujetada delante de Jehová, y delante de su pueblo.
\bibverse{19} Poned, pues, ahora vuestros corazones y vuestros ánimos en
buscar á Jehová vuestro Dios; y levantaos, y edificad el santuario del
Dios Jehová, para traer el arca del pacto de Jehová, y lo santos vasos
de Dios, á la casa edificada al nombre de Jehová.

\hypertarget{contando-y-ejecutando-los-levitas}{%
\subsection{Contando y ejecutando los
levitas}\label{contando-y-ejecutando-los-levitas}}

\hypertarget{section-22}{%
\section{23}\label{section-22}}

\bibverse{1} Siendo pues David ya viejo y harto de días, hizo á Salomón
su hijo rey sobre Israel. \footnote{\textbf{23:1} 1Re 1,28-40}
\bibverse{2} Y juntando á todos los principales de Israel, y á los
sacerdotes y Levitas, \bibverse{3} Fueron contados los Levitas de
treinta años arriba; y fué el número de ellos por sus cabezas, contados
uno á uno, treinta y ocho mil. \bibverse{4} De éstos, veinticuatro mil
para dar prisa á la obra de la casa de Jehová; y gobernadores y jueces,
seis mil; \bibverse{5} Además cuatro mil porteros; y cuatro mil para
alabar á Jehová, dijo David, con los instrumentos que he hecho para
rendir alabanzas.

\hypertarget{clasificaciuxf3n-de-los-levitas-seguxfan-gerson-kehath-y-merari}{%
\subsection{Clasificación de los levitas según Gerson, Kehath y
Merari}\label{clasificaciuxf3n-de-los-levitas-seguxfan-gerson-kehath-y-merari}}

\bibverse{6} Y repartiólos David en órdenes conforme á los hijos de
Leví, Gersón y Coath y Merari.

\bibverse{7} Los hijos de Gersón: Ladán, y Simi. \bibverse{8} Los hijos
de Ladán, tres: Jehiel el primero, después Zetham y Joel. \footnote{\textbf{23:8}
  1Cró 26,21} \bibverse{9} Los hijos de Simi, tres: Selomith, Haziel, y
Arán. Estos fueron los príncipes de las familias de Ladán. \bibverse{10}
Y los hijos de Simi: Jahath, Zinat, Jeus, y Berías. Estos cuatro fueron
los hijos de Simi. \bibverse{11} Jahath era el primero, Zinat el
segundo; mas Jeus y Berías no multiplicaron en hijos, por lo cual fueron
contados por una familia.

\bibverse{12} Los hijos de Coath: Amram, Ishar, Hebrón, y Uzziel, ellos
cuatro. \bibverse{13} Los hijos de Amram: Aarón y Moisés. Y Aarón fué
apartado para ser dedicado á las más santas cosas, él y sus hijos para
siempre, para que quemasen perfumes delante de Jehová, y le ministrasen,
y bendijesen en su nombre, para siempre. \footnote{\textbf{23:13} 1Cró
  6,34; Heb 5,4; Deut 10,8} \bibverse{14} Y los hijos de Moisés, varón
de Dios, fueron contados en la tribu de Leví. \footnote{\textbf{23:14}
  Deut 33,1} \bibverse{15} Los hijos de Moisés fueron Gersón y Eliezer.
\footnote{\textbf{23:15} Éxod 18,3-4} \bibverse{16} Hijo de Gersón fué
Sebuel el primero. \footnote{\textbf{23:16} 1Cró 26,24} \bibverse{17} E
hijo de Eliezer fué Rehabía el primero. Y Eliezer no tuvo otros hijos;
mas los hijos de Rehabía fueron muchos. \footnote{\textbf{23:17} 1Cró
  24,21-30} \bibverse{18} Hijo de Ishar fué Selomith el primero.
\bibverse{19} Los hijos de Hebrón: Jería el primero, Amarías el segundo,
Jahaziel el tercero, y Jecamán el cuarto. \bibverse{20} Los hijos de
Uzziel: Michâ el primero, é Isía el segundo.

\bibverse{21} Los hijos de Merari: Mahali y Musi. Los hijos de Mahali:
Eleazar y Cis. \bibverse{22} Y murió Eleazar sin hijos, mas tuvo hijas;
y los hijos de Cis, sus hermanos, las tomaron por mujeres. \bibverse{23}
Los hijos de Musi: Mahali, Eder y Jerimoth, ellos tres.

\hypertarget{instrucciones-oficiales-para-los-levitas}{%
\subsection{Instrucciones oficiales para los
levitas}\label{instrucciones-oficiales-para-los-levitas}}

\bibverse{24} Estos son los hijos de Leví en las familias de sus padres,
cabeceras de familias en sus delineaciones, contados por sus nombres,
por sus cabezas, los cuales hacían obra en el ministerio de la casa de
Jehová, de veinte años arriba. \bibverse{25} Porque David dijo: Jehová
Dios de Israel ha dado reposo á su pueblo Israel, y el habitar en
Jerusalem para siempre. \footnote{\textbf{23:25} Jl 4,21}

\bibverse{26} Y también los Levitas no llevarán más el tabernáculo, y
todos sus vasos para su ministerio. \bibverse{27} Así que, conforme á
las postreras palabras de David, fué la cuenta de los hijos de Leví de
veinte años arriba. \bibverse{28} Y estaban bajo la mano de los hijos de
Aarón, para ministrar en la casa de Jehová, en los atrios y en las
cámaras, y en la purificación de toda cosa santificada, y en la demás
obra del ministerio de la casa de Dios; \bibverse{29} Asimismo para los
panes de la proposición, y para la flor de la harina para el sacrificio,
y para las hojuelas sin levadura, y para la fruta de sartén, y para lo
tostado, y para toda medida y cuenta; \bibverse{30} Y para que
asistiesen cada mañana todos los días á confesar y alabar á Jehová, y
asimismo á la tarde; \bibverse{31} Y para ofrecer todos los holocaustos
á Jehová los sábados, nuevas lunas, y solemnidades, por la cuenta y
forma que tenían, continuamente delante de Jehová. \bibverse{32} Y para
que tuviesen la guarda del tabernáculo del testimonio, y la guarda del
santuario, y las órdenes de los hijos de Aarón sus hermanos, en el
ministerio de la casa de Jehová.

\hypertarget{el-dibujo-de-las-24-clases-sacerdotales}{%
\subsection{El dibujo de las 24 clases
sacerdotales}\label{el-dibujo-de-las-24-clases-sacerdotales}}

\hypertarget{section-23}{%
\section{24}\label{section-23}}

\bibverse{1} También los hijos de Aarón tuvieron sus repartimientos. Los
hijos de Aarón: Nadab, Abiú, Eleazar é Ithamar. \footnote{\textbf{24:1}
  1Cró 23,6; 1Cró 5,29} \bibverse{2} Mas Nadab y Abiú murieron antes que
su padre, y no tuvieron hijos: Eleazar é Ithamar tuvieron el sacerdocio.
\footnote{\textbf{24:2} Lev 10,1-2; Lev 10,12} \bibverse{3} Y David los
repartió, siendo Sadoc de los hijos de Eleazar, y Ahimelech de los hijos
de Ithamar, por sus turnos en su ministerio. \footnote{\textbf{24:3}
  2Cró 8,14} \bibverse{4} Y los hijos de Eleazar fueron hallados, cuanto
á sus principales varones, muchos más que los hijos de Ithamar; y
repartiéronlos así: De los hijos de Eleazar había dieciséis cabezas de
familias paternas; y de los hijos de Ithamar por las familias de sus
padres, ocho. \bibverse{5} Repartiéronlos pues por suerte los unos con
los otros: porque de los hijos de Eleazar y de los hijos de Ithamar hubo
príncipes del santuario, y príncipes de la casa de Dios. \bibverse{6} Y
Semeías escriba, hijo de Nathanael, de los Levitas, escribiólos delante
del rey y de los príncipes, y delante de Sadoc el sacerdote, y de
Ahimelech hijo de Abiathar, y de los príncipes de las familias de los
sacerdotes y Levitas: y adscribían una familia á Eleazar, y á Ithamar
otra.

\bibverse{7} Y la primera suerte salió por Joiarib, la segunda por
Jedaía; \bibverse{8} La tercera por Harim, la cuarta por Seorim;
\bibverse{9} La quinta por Malchîas, la sexta por Miamim; \bibverse{10}
La séptima por Cos, la octava por Abías; \footnote{\textbf{24:10} Luc
  1,5} \bibverse{11} La nona por Jesua, la décima por Sechânía;
\bibverse{12} La undécima por Eliasib, la duodécima por Jacim;
\bibverse{13} La décimatercia por Uppa, la décimacuarta por Isebeab;
\bibverse{14} La décimaquinta por Bilga, la décimasexta por Immer;
\bibverse{15} La décimaséptima por Hezir, la décimaoctava por Aphses;
\bibverse{16} La décimanona por Pethaía, la vigésima por Hezeciel;
\bibverse{17} La vigésimaprima por Jachim, la vigésimasegunda por Hamul;
\bibverse{18} La vigésimatercia por Delaía, la vigésimacuarta por
Maazía. \bibverse{19} Estos fueron contados en su ministerio, para que
entrasen en la casa de Jehová, conforme á su ordenanza, bajo el mando de
Aarón su padre, de la manera que le había mandado Jehová el Dios de
Israel.

\hypertarget{las-clases-levitas-y-sus-luxedderes}{%
\subsection{Las clases levitas y sus
líderes}\label{las-clases-levitas-y-sus-luxedderes}}

\bibverse{20} Y de los hijos de Leví que quedaron: Subael, de los hijos
de Amram; y de los hijos de Subael, Jehedías. \bibverse{21} Y de los
hijos de Rehabía, Isias el principal. \bibverse{22} De los Ishareos,
Selemoth; é hijo de Selemoth, Jath. \bibverse{23} Y de los hijos de
Hebrón: Jeria el primero, el segundo Amarías, el tercero Jahaziel, el
cuarto Jecamán. \bibverse{24} Hijo de Uzziel, Michâ; é hijo de Michâ,
Samir. \bibverse{25} Hermano de Michâ, Isía; é hijo de Isía, Zachârías.
\bibverse{26} Los hijos de Merari: Mahali y Musi; hijo de Jaazia, Benno.
\bibverse{27} Los hijos de Merari por Jaazia: Benno, y Soam, Zachûr é
Ibri. \bibverse{28} Y de Mahali, Eleazar, el cual no tuvo hijos.
\bibverse{29} Hijo de Cis, Jerameel. \bibverse{30} Los hijos de Musi:
Maheli, Eder y Jerimoth. Estos fueron los hijos de los Levitas conforme
á las casas de sus familias. \bibverse{31} Estos también echaron
suertes, como sus hermanos los hijos de Aarón, delante del rey David, y
de Sadoc y de Ahimelech, y de los príncipes de las familias de los
sacerdotes y Levitas: el principal de los padres igualmente que el menor
de sus hermanos. \footnote{\textbf{24:31} 1Cró 25,8}

\hypertarget{el-sorteo-de-las-24-divisiones-de-los-cantantes-y-muxfasicos-sagrados}{%
\subsection{El sorteo de las 24 divisiones de los cantantes y músicos
sagrados}\label{el-sorteo-de-las-24-divisiones-de-los-cantantes-y-muxfasicos-sagrados}}

\hypertarget{section-24}{%
\section{25}\label{section-24}}

\bibverse{1} Asimismo David y los príncipes del ejército apartaron para
el ministerio á los hijos de Asaph, y de Hemán, y de Jeduthún, los
cuales profetizasen con arpas, salterios, y címbalos: y el número de
ellos fué, de hombres idóneos para la obra de su ministerio respectivo:
\footnote{\textbf{25:1} 1Cró 15,19} \bibverse{2} De los hijos de Asaph:
Zachûr, José, Methanías, y Asareela, hijos de Asaph, bajo la dirección
de Asaph, el cual profetizaba á la orden del rey. \bibverse{3} De
Jeduthún: los hijos de Jeduthún, Gedalías, Sesi, Jesaías, Hasabías, y
Mathithías, y Simi: seis, bajo la mano de su padre Jeduthún, el cual
profetizaba con arpa, para celebrar y alabar á Jehová. \bibverse{4} De
Hemán: los hijos de Hemán, Buccia, Mathanía, Uzziel, Sebuel, Jerimoth,
Hananías, Hanani, Eliatha, Gidalthi, Romamti-ezer, Josbecasa, Mallothi,
Othir, y Mahazioth. \bibverse{5} Todos estos fueron hijos de Hemán,
vidente del rey en palabras de Dios, para ensalzar el poder suyo: y dió
Dios á Hemán catorce hijos y tres hijas. \footnote{\textbf{25:5} 1Cró
  21,9; 2Cró 35,15} \bibverse{6} Y todos estos estaban bajo la dirección
de su padre en la música, en la casa de Jehová, con címbalos, salterios
y arpas, para el ministerio del templo de Dios, por disposición del rey
acerca de Asaph, de Jeduthún, y de Hemán. \bibverse{7} Y el número de
ellos con sus hermanos instruídos en música de Jehová, todos los aptos,
fué doscientos ochenta y ocho. \bibverse{8} Y echaron suertes para los
turnos del servicio, entrando el pequeño con el grande, lo mismo el
maestro que el discípulo.

\bibverse{9} Y la primera suerte salió por Asaph, á José: la segunda á
Gedalías, quien con sus hermanos é hijos fueron doce; \bibverse{10} La
tercera á Zachûr, con sus hijos y sus hermanos, doce; \bibverse{11} La
cuarta á Isri, con sus hijos y sus hermanos, doce; \bibverse{12} La
quinta á Nethanías, con sus hijos y sus hermanos, doce; \bibverse{13} La
sexta á Buccia, con sus hijos y sus hermanos, doce; \bibverse{14} La
séptima á Jesarela, con sus hijos y sus hermanos, doce; \bibverse{15} La
octava á Jesahías, con sus hijos y sus hermanos, doce; \bibverse{16} La
nona á Mathanías, con sus hijos y sus hermanos, doce; \bibverse{17} La
décima á Simi, con sus hijos y sus hermanos, doce; \bibverse{18} La
undécima á Azareel, con sus hijos y sus hermanos, doce; \bibverse{19} La
duodécima á Hasabías, con sus hijos y sus hermanos, doce; \bibverse{20}
La décimatercia á Subael, con sus hijos y sus hermanos, doce;
\bibverse{21} La décimacuarta á Mathithías, con sus hijos y sus
hermanos, doce; \bibverse{22} La décimaquinta á Jerimoth, con sus hijos
y sus hermanos, doce; \bibverse{23} La décimasexta á Hananías, con sus
hijos y sus hermanos, doce; \bibverse{24} La décimaséptima á Josbecasa,
con sus hijos y sus hermanos, doce; \bibverse{25} La décimaoctava á
Hanani, con sus hijos y sus hermanos, doce; \bibverse{26} La décimanona
á Mallothi, con sus hijos y sus hermanos, doce; \bibverse{27} La
vigésima á Eliatha, con sus hijos y sus hermanos, doce; \bibverse{28} La
vigésimaprima á Othir, con sus hijos y sus hermanos, doce; \bibverse{29}
La vigésimasegunda á Giddalthi, con sus hijos y sus hermanos, doce;
\bibverse{30} La vigésimatercia á Mahazioth, con sus hijos y sus
hermanos, doce; \bibverse{31} La vigésimacuarta á Romamti-ezer, con sus
hijos y sus hermanos, doce.

\hypertarget{divisiones-de-los-porteros-levuxedticos}{%
\subsection{Divisiones de los porteros
levíticos}\label{divisiones-de-los-porteros-levuxedticos}}

\hypertarget{section-25}{%
\section{26}\label{section-25}}

\bibverse{1} Cuanto á los repartimientos de los porteros: De los
Coraitas: Meselemia hijo de Coré, de los hijos de Asaph. \footnote{\textbf{26:1}
  2Cró 8,14; 2Cró 35,15} \bibverse{2} Los hijos de Meselemia: Zachârías
el primogénito, Jediael el segundo, Zebadías el tercero, Jatnael el
cuarto; \bibverse{3} Elam el quinto, Johanam el sexto, Elioenai el
séptimo. \bibverse{4} Los hijos de Obed-edom: Semeías el primogénito,
Jozabad el segundo, Joab el tercero, el cuarto Sachâr, el quinto
Nathanael; \bibverse{5} El sexto Anmiel, el séptimo Issachâr, el octavo
Peullethai: porque Dios había bendecido á Obed-edom. \bibverse{6}
También de Semeías su hijo nacieron hijos que fueron señores sobre la
casa de sus padres; porque eran varones muy valerosos. \bibverse{7} Los
hijos de Semeías: Othni, Raphael, Obed, Elzabad, y sus hermanos, hombres
esforzados; asimismo Eliú, y Samachîas. \bibverse{8} Todos estos de los
hijos de Obed-edom: ellos con sus hijos y sus hermanos, hombres robustos
y fuertes para el ministerio; sesenta y dos, de Obed-edom. \bibverse{9}
Y los hijos de Meselemia y sus hermanos, dieciocho hombres valientes.
\bibverse{10} De Hosa, de los hijos de Merari: Simri el principal,
(aunque no era el primogénito, mas su padre lo puso para que fuese
cabeza;) \bibverse{11} El segundo Hilcías, el tercero Tebelías, el
cuarto Zachârías: todos los hijos de Hosa y sus hermanos fueron trece.

\hypertarget{la-distribuciuxf3n-de-los-porteros-a-las-diferentes-localizaciones}{%
\subsection{La distribución de los porteros a las diferentes
localizaciones}\label{la-distribuciuxf3n-de-los-porteros-a-las-diferentes-localizaciones}}

\bibverse{12} Entre estos se hizo la distribución de los porteros,
alternando los principales de los varones en la guardia con sus
hermanos, para servir en la casa de Jehová. \bibverse{13} Y echaron
suertes, el pequeño con el grande, por las casas de sus padres, para
cada puerta. \bibverse{14} Y cayó la suerte al oriente á Selemía. Y á
Zachârías su hijo, consejero entendido, metieron en las suertes: y salió
la suerte suya al norte. \bibverse{15} Y por Obed-edom, al mediodía; y
por sus hijos, la casa de la consulta. \bibverse{16} Por Suppim y Hosa
al occidente, con la puerta de Sallechêt al camino de la subida, guardia
contra guardia. \bibverse{17} Al oriente seis Levitas, al norte cuatro
de día; al mediodía cuatro de día; y á la casa de la consulta, de dos en
dos. \bibverse{18} En la cámara de los vasos al occidente, cuatro al
camino, y dos en la cámara. \bibverse{19} Estos son los repartimientos
de los porteros, hijos de los Coraitas, y de los hijos de Merari.

\hypertarget{los-tesoreros-levuxedticos-y-los-funcionarios-de-la-administraciuxf3n}{%
\subsection{Los tesoreros levíticos y los funcionarios de la
administración}\label{los-tesoreros-levuxedticos-y-los-funcionarios-de-la-administraciuxf3n}}

\bibverse{20} Y de los Levitas, Achîas tenía cargo de los tesoros de la
casa de Dios, y de los tesoros de las cosas santificadas. \bibverse{21}
Cuanto á los hijos de Ladán, hijos de Gersón: de Ladán, los príncipes de
las familias de Ladán fueron Gersón, y Jehieli. \footnote{\textbf{26:21}
  1Cró 23,8} \bibverse{22} Los hijos de Jehieli, Zethán y Joel su
hermano, tuvieron cargo de los tesoros de la casa de Jehová.
\bibverse{23} Acerca de los Amramitas, de los Isharitas, de los
Hebronitas, y de los Uzzielitas, \bibverse{24} Sebuel hijo de Gersón,
hijo de Moisés, era principal sobre los tesoros. \bibverse{25} En orden
á su hermano Eliezer, hijo de éste era Rehabía, hijo de éste Isaías,
hijo de éste Joram, hijo de éste Zichri, del que fué hijo Selomith.
\footnote{\textbf{26:25} 1Cró 23,17} \bibverse{26} Este Selomith y sus
hermanos tenían cargo de todos los tesoros de todas las cosas
santificadas, que había consagrado el rey David, y los príncipes de las
familias, y los capitanes de millares y de cientos, y los jefes del
ejército; \bibverse{27} De lo que habían consagrado de las guerras y de
los despojos, para reparar la casa de Jehová. \bibverse{28} Asimismo
todas las cosas que había consagrado Samuel vidente, y Saúl hijo de Cis,
y Abner hijo de Ner, y Joab hijo de Sarvia: y todo lo que cualquiera
consagraba, estaba bajo la mano de Selomith y de sus hermanos.

\bibverse{29} De los Isharitas, Chenanía y sus hijos eran gobernadores y
jueces sobre Israel en las obras de fuera. \bibverse{30} De los
Hebronitas, Hasabías y sus hermanos, hombres de vigor, mil y
setecientos, gobernaban á Israel de la otra parte del Jordán, al
occidente, en toda la obra de Jehová, y en el servicio del rey.
\bibverse{31} De los Hebronitas, Jerías era el principal entre los
Hebronitas repartidos en sus linajes por sus familias. En el año
cuarenta del reinado de David se registraron, y halláronse entre ellos
fuertes y vigorosos en Jazer de Galaad. \bibverse{32} Y sus hermanos,
hombres valientes, eran dos mil y setecientos, cabezas de familias, los
cuales el rey David constituyó sobre los Rubenitas, Gaditas, y sobre la
media tribu de Manasés, para todas las cosas de Dios, y los negocios del
rey.

\hypertarget{los-doce-jefes-militares-los-caudillos-y-los-demuxe1s-altos-funcionarios-de-david-la-divisiuxf3n-del-ejuxe9rcito-en-doce}{%
\subsection{Los doce jefes militares, los caudillos y los demás altos
funcionarios de David; La división del ejército en
doce}\label{los-doce-jefes-militares-los-caudillos-y-los-demuxe1s-altos-funcionarios-de-david-la-divisiuxf3n-del-ejuxe9rcito-en-doce}}

\hypertarget{section-26}{%
\section{27}\label{section-26}}

\bibverse{1} Y los hijos de Israel según su número, á saber, príncipes
de familias, tribunos, centuriones y oficiales de los que servían al rey
en todos los negocios de las divisiones que entraban y salían cada mes
en todos los meses del año, eran en cada división veinte y cuatro mil.

\bibverse{2} Sobre la primera división del primer mes estaba Jasobam
hijo de Zabdiel; y había en su división veinte y cuatro mil.
\bibverse{3} De los hijos de Phares fué él jefe de todos los capitanes
de las compañías del primer mes. \bibverse{4} Sobre la división del
segundo mes estaba Dodai Ahohita: y Micloth era mayor general en su
división, en la que también había veinte y cuatro mil. \bibverse{5} El
jefe de la tercera división para el tercer mes era Benaías, hijo de
Joiada sumo sacerdote; y en su división había veinte y cuatro mil.
\bibverse{6} Este Benaías era valiente entre los treinta y sobre los
treinta; y en su división estaba Amisabad su hijo. \bibverse{7} El
cuarto jefe para el cuarto mes era Asael hermano de Joab, y después de
él Zebadías su hijo; y en su división había veinte y cuatro mil.
\bibverse{8} El quinto jefe para el quinto mes era Sambuth Izrita: y en
su división había veinte y cuatro mil. \bibverse{9} El sexto para el
sexto mes era Hira hijo de Icces, de Tecoa; y en su división veinte y
cuatro mil. \bibverse{10} El séptimo para el séptimo mes era Helles
Pelonita, de los hijos de Ephraim; y en su división veinte y cuatro mil.
\bibverse{11} El octavo para el octavo mes era Sibbecai Husatita, de
Zarahi; y en su división veinte y cuatro mil. \bibverse{12} El noveno
para el noveno mes era Abiezer Anathothita, de los Benjamitas; y en su
división veinte y cuatro mil. \bibverse{13} El décimo para el décimo mes
era Maharai Nethophathita, de Zarahi; y en su división veinte y cuatro
mil. \bibverse{14} El undécimo para el undécimo mes era Benaías
Piratonita, de los hijos de Ephraim; y en su división veinte y cuatro
mil. \bibverse{15} El duodécimo para el duodécimo mes era Heldai
Nethophathita, de Othniel; y en su división veinte y cuatro mil.

\hypertarget{los-doce-pruxedncipes-tribales-de-israel}{%
\subsection{Los doce príncipes tribales de
Israel}\label{los-doce-pruxedncipes-tribales-de-israel}}

\bibverse{16} Asimismo sobre las tribus de Israel: el jefe de los
Rubenitas era Eliezer hijo de Zichri; de los Simeonitas, Sephatías, hijo
de Maachâ: \bibverse{17} De los Levitas, Hasabías hijo de Camuel; de los
Aaronitas, Sadoc; \bibverse{18} De Judá, Eliú, uno de los hermanos de
David; de los de Issachâr, Omri hijo de Michâel; \bibverse{19} De los de
Zabulón, Ismaías hijo de Abdías; de los de Nephtalí, Jerimoth hijo de
Azriel; \bibverse{20} De los hijos de Ephraim, Oseas hijo de Azazía; de
la media tribu de Manasés, Joel hijo de Pedaía; \bibverse{21} De la otra
media tribu de Manasés en Galaad, Iddo hijo de Zachârías; de los de
Benjamín, Jaaciel hijo de Abner; \bibverse{22} Y de Dan, Azarael hijo de
Jeroam. Estos fueron los jefes de las tribus de Israel.

\hypertarget{comentar-el-censo-incompleto}{%
\subsection{Comentar el censo
incompleto}\label{comentar-el-censo-incompleto}}

\bibverse{23} Y no tomó David el número de los que eran de veinte años
abajo, por cuanto Jehová había dicho que él había de multiplicar á
Israel como las estrellas del cielo. \footnote{\textbf{27:23} Gén 22,17}
\bibverse{24} Joab hijo de Sarvia había comenzado á contar, mas no
acabó, pues por esto vino la ira sobre Israel: y así el número no fué
puesto en el registro de las crónicas del rey David. \footnote{\textbf{27:24}
  1Cró 21,14}

\hypertarget{los-administradores-de-la-propiedad-real-tesorero-y-maestro-de-alquileres}{%
\subsection{Los administradores de la propiedad real (tesorero y maestro
de
alquileres)}\label{los-administradores-de-la-propiedad-real-tesorero-y-maestro-de-alquileres}}

\bibverse{25} Y Azmaveth hijo de Adiel tenía cargo de los tesoros del
rey; y de los tesoros de los campos, y de las ciudades, y de las aldeas
y castillos, Jonathán hijo de Uzzías; \bibverse{26} Y de los que
trabajaban en la labranza de las tierras, Ezri hijo de Chêlud;
\bibverse{27} Y de las viñas Simi Ramathita; y del fruto de las viñas
para las bodegas, Zabdías Siphmita; \bibverse{28} Y de los olivares é
higuerales que había en las campiñas, Baal-hanan Gederita; y de los
almacenes del aceite, Joas; \bibverse{29} De las vacas que pastaban en
Sarón, Sitrai Saronita; y de las vacas que estaban en los valles, Saphat
hijo de Adlai; \bibverse{30} Y de los camellos, Obil Ismaelita; y de las
asnas, Jedías Meronothita; \bibverse{31} Y de las ovejas, Jaziz Agareno.
Todos estos eran superintendentes de la hacienda del rey David.

\hypertarget{los-muxe1s-altos-funcionarios-imperiales-consejeros-del-rey}{%
\subsection{Los más altos funcionarios imperiales (consejeros del
rey)}\label{los-muxe1s-altos-funcionarios-imperiales-consejeros-del-rey}}

\bibverse{32} Y Jonathán, tío de David, era consejero, varón prudente y
escriba; y Jehiel hijo de Hacmoni estaba con los hijos del rey.
\bibverse{33} Y también Achitophel era consejero del rey; y Husai
Arachîta amigo del rey. \footnote{\textbf{27:33} 2Sam 15,12; 2Sam 15,37}
\bibverse{34} Después de Achitophel era Joiada hijo de Benaías, y
Abiathar. Y Joab era el general del ejército del rey. \footnote{\textbf{27:34}
  2Sam 8,16}

\hypertarget{el-discurso-de-david-a-los-jefes-de-israel}{%
\subsection{El discurso de David a los jefes de
Israel}\label{el-discurso-de-david-a-los-jefes-de-israel}}

\hypertarget{section-27}{%
\section{28}\label{section-27}}

\bibverse{1} Y juntó David en Jerusalem á todos los principales de
Israel, los príncipes de las tribus, y los jefes de las divisiones que
servían al rey, los tribunos y centuriones, con los superintendentes de
toda la hacienda y posesión del rey, y sus hijos, con los eunucos, los
poderosos, y todos sus hombres valientes.

\hypertarget{david-presenta-al-superior-del-pueblo-a-salomuxf3n-como-su-sucesor}{%
\subsection{David presenta al superior del pueblo a Salomón como su
sucesor}\label{david-presenta-al-superior-del-pueblo-a-salomuxf3n-como-su-sucesor}}

\bibverse{2} Y levantándose el rey David, puesto en pie dijo: Oidme,
hermanos míos, y pueblo mío. Yo tenía en propósito edificar una casa,
para que en ella reposara el arca del pacto de Jehová, y para el estrado
de los pies de nuestro Dios; y había ya aprestado todo para edificar.
\footnote{\textbf{28:2} 1Cró 22,7-10} \bibverse{3} Mas Dios me dijo: Tú
no edificarás casa á mi nombre: porque eres hombre de guerra, y has
derramado mucha sangre. \footnote{\textbf{28:3} 2Sam 7,5} \bibverse{4}
Empero Jehová el Dios de Israel me eligió de toda la casa de mi padre,
para que perpetuamente fuese rey sobre Israel: porque á Judá escogió por
caudillo, y de la casa de Judá la familia de mi padre; y de entre los
hijos de mi padre agradóse de mí para ponerme por rey sobre todo Israel;
\footnote{\textbf{28:4} Gén 49,10; 1Sam 16,1; 1Sam 16,12} \bibverse{5} Y
de todos mis hijos (porque Jehová me ha dado muchos hijos,) eligió á mi
hijo Salomón para que se siente en el trono del reino de Jehová sobre
Israel. \bibverse{6} Y me ha dicho: Salomón tu hijo, él edificará mi
casa y mis atrios: porque á éste me he escogido por hijo, y yo le seré á
él por padre. \bibverse{7} Asimismo yo confirmaré su reino para siempre,
si él se esforzare á poner por obra mis mandamientos y mis juicios, como
aqueste día.

\bibverse{8} Ahora pues, delante de lo ojos de todo Israel, congregación
de Jehová, y en oídos de nuestro Dios, guardad é inquirid todos los
preceptos de Jehová vuestro Dios, para que poseáis la buena tierra, y la
dejéis por heredad á vuestros hijos después de vosotros perpetuamente.

\hypertarget{las-instrucciones-de-david-y-su-contribuciuxf3n-a-salomuxf3n}{%
\subsection{Las instrucciones de David y su contribución a
Salomón}\label{las-instrucciones-de-david-y-su-contribuciuxf3n-a-salomuxf3n}}

\bibverse{9} Y tú, Salomón, hijo mío, conoce al Dios de tu padre, y
sírvele con corazón perfecto, y con ánimo voluntario; porque Jehová
escudriña los corazones de todos, y entiende toda imaginación de los
pensamientos. Si tú le buscares, lo hallarás; mas si lo dejares, él te
desechará para siempre. \footnote{\textbf{28:9} Sal 7,10} \bibverse{10}
Mira, pues, ahora que Jehová te ha elegido para que edifiques casa para
santuario: esfuérzate, y hazla.

\hypertarget{david-le-da-a-salomuxf3n-el-modelo-de-la-casa-del-templo-y-los-tesoros-recolectados-para-su-construcciuxf3n}{%
\subsection{David le da a Salomón el modelo de la casa del templo y los
tesoros recolectados para su
construcción}\label{david-le-da-a-salomuxf3n-el-modelo-de-la-casa-del-templo-y-los-tesoros-recolectados-para-su-construcciuxf3n}}

\bibverse{11} Y David dió á Salomón su hijo la traza del pórtico, y de
sus casas, y de sus oficinas, y de sus salas, y de sus recámaras, y de
la casa del propiciatorio. \bibverse{12} Asimismo la traza de todas las
cosas que tenía en su voluntad, para los atrios de la casa de Jehová, y
para todas las cámaras en derredor, para los tesoros de la casa de Dios,
y para los tesoros de las cosas santificadas: \bibverse{13} También para
los órdenes de los sacerdotes y de los Levitas, y para toda la obra del
ministerio de la casa de Jehová, y para todos los vasos del ministerio
de la casa de Jehová. \bibverse{14} Y dió oro por peso para lo de oro,
para todos los vasos de cada servicio: y plata por peso para todos los
vasos, para todos los vasos de cada servicio. \bibverse{15} Oro por peso
para los candeleros de oro, y para sus candilejas; por peso el oro para
cada candelero y sus candilejas: y para los candeleros de plata, plata
por peso para el candelero y sus candilejas, conforme al servicio de
cada candelero. \bibverse{16} Asimismo dió oro por peso para las mesas
de la proposición, para cada mesa: del mismo modo plata para las mesas
de plata: \bibverse{17} También oro puro para los garfios y para las
palanganas, y para los incensarios, y para los tazones de oro, para cada
tazón por peso; y para los tazones de plata, por peso para cada tazón:
\bibverse{18} Además, oro puro por peso para el altar del perfume, y
para el á manera de carro de los querubines de oro, que con las alas
extendidas cubrían el arca del pacto de Jehová. \bibverse{19} Todas
estas cosas, dijo David, se me han representado por la mano de Jehová
que me hizo entender todas las obras del diseño.

\bibverse{20} Dijo más David á Salomón su hijo: Anímate y esfuérzate, y
ponlo por obra; no temas, ni desmayes, porque el Dios Jehová, mi Dios,
será contigo: él no te dejará, ni te desamparará, hasta que acabes toda
la obra para el servicio de la casa de Jehová. \footnote{\textbf{28:20}
  1Cró 22,13; Deut 31,6}

\bibverse{21} He aquí los órdenes de los sacerdotes y de los Levitas,
para todo el ministerio de la casa de Dios, serán contigo en toda la
obra: asimismo todos los voluntarios é inteligentes para cualquiera
especie de industria; y los príncipes, y todo el pueblo para ejecutar
todas tus órdenes.

\hypertarget{la-contribuciuxf3n-de-los-pruxedncipes-a-la-construcciuxf3n-del-templo-siguiendo-la-amonestaciuxf3n-de-david}{%
\subsection{La contribución de los príncipes a la construcción del
templo siguiendo la amonestación de
David}\label{la-contribuciuxf3n-de-los-pruxedncipes-a-la-construcciuxf3n-del-templo-siguiendo-la-amonestaciuxf3n-de-david}}

\hypertarget{section-28}{%
\section{29}\label{section-28}}

\bibverse{1} Después dijo el rey David á toda la asamblea: A solo
Salomón mi hijo ha elegido Dios; él es joven y tierno, y la obra grande;
porque la casa no es para hombre, sino para Jehová Dios. \bibverse{2} Yo
empero con todas mis fuerzas he preparado para la casa de mi Dios, oro
para las cosas de oro, y plata para las cosas de plata, y metal para las
de metal, y hierro para las de hierro, y madera para las de madera, y
piedras oniquinas, y piedras preciosas, y piedras negras, y piedras de
diversos colores, y toda suerte de piedras preciosas, y piedras de
mármol en abundancia. \bibverse{3} A más de esto, por cuanto tengo mi
gusto en la casa de mi Dios, yo guardo en mi tesoro particular oro y
plata que, además de todas las cosas que he aprestado para la casa del
santuario, he dado para la casa de mi Dios; \bibverse{4} A saber, tres
mil talentos de oro, de oro de Ophir, y siete mil talentos de plata
afinada para cubrir las paredes de las casas: \bibverse{5} Oro pues para
las cosas de oro, y plata para las cosas de plata, y para toda la obra
de manos de los oficiales. ¿Y quién quiere hacer hoy ofrenda á Jehová?
\footnote{\textbf{29:5} Éxod 35,5}

\bibverse{6} Entonces los príncipes de las familias, y los príncipes de
las tribus de Israel, tribunos y centuriones, con los superintendentes
de la hacienda del rey, ofrecieron de su voluntad; \bibverse{7} Y dieron
para el servicio de la casa de Dios cinco mil talentos de oro y diez mil
sueldos, y diez mil talentos de plata, y dieciocho mil talentos de
metal, y cinco mil talentos de hierro. \bibverse{8} Y todo el que se
halló con piedras preciosas, diólas para el tesoro de la casa de Jehová,
en mano de Jehiel Gersonita. \bibverse{9} Y holgóse el pueblo de haber
contribuído de su voluntad; porque con entero corazón ofrecieron á
Jehová voluntariamente.

\hypertarget{oraciuxf3n-final-de-david}{%
\subsection{Oración final de David}\label{oraciuxf3n-final-de-david}}

\bibverse{10} Asimismo holgóse mucho el rey David, y bendijo á Jehová
delante de toda la congregación; y dijo David: Bendito seas tú, oh
Jehová, Dios de Israel nuestro padre, de uno á otro siglo. \bibverse{11}
Tuya es, oh Jehová, la magnificencia, y el poder, y la gloria, la
victoria, y el honor; porque todas las cosas que están en los cielos y
en la tierra son tuyas. Tuyo, oh Jehová, es el reino, y la altura sobre
todos los que están por cabeza. \footnote{\textbf{29:11} Apoc 4,11; Apoc
  5,13} \bibverse{12} Las riquezas y la gloria están delante de ti, y tú
señoreas á todos: y en tu mano está la potencia y la fortaleza, y en tu
mano la grandeza y fuerza de todas las cosas. \footnote{\textbf{29:12}
  2Cró 20,6} \bibverse{13} Ahora pues, Dios nuestro, nosotros te
confesamos, y loamos tu glorioso nombre. \bibverse{14} Porque ¿quién soy
yo, y quién es mi pueblo, para que pudiésemos ofrecer de nuestra
voluntad cosas semejantes? porque todo es tuyo, y lo recibido de tu mano
te damos. \bibverse{15} Porque nosotros, extranjeros y advenedizos somos
delante de ti, como todos nuestros padres; y nuestros días cual sombra
sobre la tierra, y no dan espera. \footnote{\textbf{29:15} Sal 39,13;
  Heb 11,13; Job 14,2} \bibverse{16} Oh Jehová Dios nuestro, toda esta
abundancia que hemos aprestado para edificar casa á tu santo nombre, de
tu mano es, y todo es tuyo. \bibverse{17} Yo sé, Dios mío, que tú
escudriñas los corazones, y que la rectitud te agrada: por eso yo con
rectitud de mi corazón voluntariamente te he ofrecido todo esto, y ahora
he visto con alegría que tu pueblo, que aquí se ha hallado ahora, ha
dado para ti espontáneamente. \bibverse{18} Jehová, Dios de Abraham, de
Isaac, y de Israel, nuestros padres, conserva perpetuamente esta
voluntad del corazón de tu pueblo, y encamina su corazón á ti.
\bibverse{19} Asimismo da á mi hijo Salomón corazón perfecto, para que
guarde tus mandamientos, tus testimonios y tus estatutos, y para que
haga todas las cosas, y te edifique la casa para la cual yo he hecho el
apresto.

\hypertarget{final-solemne-de-la-reuniuxf3n-la-unciuxf3n-de-salomuxf3n-como-rey-fin-del-reinado-de-david}{%
\subsection{Final solemne de la reunión; La unción de Salomón como rey;
Fin del reinado de
David}\label{final-solemne-de-la-reuniuxf3n-la-unciuxf3n-de-salomuxf3n-como-rey-fin-del-reinado-de-david}}

\bibverse{20} Después dijo David á toda la congregación: Bendecid ahora
á Jehová vuestro Dios. Entonces toda la congregación bendijo á Jehová
Dios de sus padres, é inclinándose adoraron delante de Jehová, y del
rey.

\bibverse{21} Y sacrificaron víctimas á Jehová, y ofrecieron á Jehová
holocaustos el día siguiente, mil becerros, mil carneros, mil corderos
con sus libaciones, y muchos sacrificios por todo Israel. \bibverse{22}
Y comieron y bebieron delante de Jehová aquel día con gran gozo; y
dieron la segunda vez la investidura del reino á Salomón hijo de David,
y ungiéronlo á Jehová por príncipe, y á Sadoc por sacerdote. \footnote{\textbf{29:22}
  1Cró 23,1}

\bibverse{23} Y sentóse Salomón por rey en el trono de Jehová en lugar
de David su padre, y fué prosperado; y obedecióle todo Israel.
\footnote{\textbf{29:23} 1Cró 28,5; 1Re 1,35; 1Re 1,39} \bibverse{24} Y
todos los príncipes y poderosos, y todos los hijos del rey David,
prestaron homenaje al rey Salomón. \bibverse{25} Y Jehová engrandeció en
extremo á Salomón á los ojos de todo Israel, y dióle gloria del reino,
cual ningún rey la tuvo antes de él en Israel. \footnote{\textbf{29:25}
  2Cró 1,1}

\hypertarget{el-final-de-david-y-las-fuentes-de-su-historia}{%
\subsection{El final de David y las fuentes de su
historia}\label{el-final-de-david-y-las-fuentes-de-su-historia}}

\bibverse{26} Así reinó David hijo de Isaí sobre todo Israel.
\bibverse{27} Y el tiempo que reinó sobre Israel fué cuarenta años.
Siete años reinó en Hebrón, y treinta y tres reinó en Jerusalem.
\bibverse{28} Y murió en buena vejez, lleno de días, de riquezas, y de
gloria: y reinó en su lugar Salomón su hijo. \bibverse{29} Y los hechos
del rey David, primeros y postreros, están escritos en el libro de las
crónicas de Samuel vidente, y en las crónicas del profeta Nathán, y en
las crónicas de Gad vidente, \bibverse{30} Con todo lo relativo á su
reinado, y su poder, y los tiempos que pasaron sobre él, y sobre Israel,
y sobre todos los reinos de aquellas tierras.
