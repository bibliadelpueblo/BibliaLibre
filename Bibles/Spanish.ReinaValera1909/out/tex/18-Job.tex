\hypertarget{section}{%
\section{1}\label{section}}

\bibverse{1} Hubo un varón en tierra de Hus, llamado Job; y era este
hombre perfecto y recto, y temeroso de Dios, y apartado del mal.
\bibverse{2} Y naciéronle siete hijos y tres hijas. \bibverse{3} Y su
hacienda era siete mil ovejas, y tres mil camellos, y quinientas yuntas
de bueyes, y quinientas asnas, y muchísimos criados: y era aquel varón
grande más que todos los Orientales. \bibverse{4} E iban sus hijos y
hacían banquetes en sus casas, cada uno en su día; y enviaban á llamar
sus tres hermanas, para que comiesen y bebiesen con ellos. \bibverse{5}
Y acontecía que, habiendo pasado en turno los días del convite, Job
enviaba y santificábalos, y levantábase de mañana y ofrecía holocaustos
conforme al número de todos ellos. Porque decía Job: Quizá habrán pecado
mis hijos, y habrán blasfemado á Dios en sus corazones. De esta manera
hacía todos los días.

\bibverse{6} Y un día vinieron los hijos de Dios á presentarse delante
de Jehová, entre los cuales vino también Satán. \bibverse{7} Y dijo
Jehová á Satán: ¿De dónde vienes? Y respondiendo Satán á Jehová, dijo:
De rodear la tierra, y de andar por ella.

\bibverse{8} Y Jehová dijo á Satán: ¿No has considerado á mi siervo Job,
que no hay otro como él en la tierra, varón perfecto y recto, temeroso
de Dios, y apartado de mal?

\bibverse{9} Y respondiendo Satán á Jehová, dijo: ¿Teme Job á Dios de
balde? \bibverse{10} ¿No le has tú cercado á él, y á su casa, y á todo
lo que tiene en derredor? Al trabajo de sus manos has dado bendición;
por tanto su hacienda ha crecido sobre la tierra. \bibverse{11} Mas
extiende ahora tu mano, y toca á todo lo que tiene, y verás si no te
blasfema en tu rostro.

\bibverse{12} Y dijo Jehová á Satán: He aquí, todo lo que tiene está en
tu mano: solamente no pongas tu mano sobre él. Y salióse Satán de
delante de Jehová.

\bibverse{13} Y un día aconteció que sus hijos é hijas comían y bebían
vino en casa de su hermano el primogénito, \bibverse{14} Y vino un
mensajero á Job, que le dijo: Estando arando los bueyes, y las asnas
paciendo cerca de ellos, \bibverse{15} Acometieron los Sabeos, y
tomáronlos, é hirieron á los mozos á filo de espada: solamente escapé yo
para traerte las nuevas.

\bibverse{16} Aun estaba éste hablando, y vino otro que dijo: Fuego de
Dios cayó del cielo, que quemó las ovejas y los mozos, y los consumió:
solamente escapé yo solo para traerte las nuevas.

\bibverse{17} Todavía estaba éste hablando, y vino otro que dijo: Los
Caldeos hicieron tres escuadrones, y dieron sobre los camellos, y
tomáronlos, é hirieron á los mozos á filo de espada; y solamente escapé
yo solo para traerte las nuevas.

\bibverse{18} Entre tanto que éste hablaba, vino otro que dijo: Tus
hijos y tus hijas estaban comiendo y bebiendo vino en casa de su hermano
el primogénito; \bibverse{19} Y he aquí un gran viento que vino del lado
del desierto, é hirió las cuatro esquinas de la casa, y cayó sobre los
mozos, y murieron; y solamente escapé yo solo para traerte las nuevas.

\bibverse{20} Entonces Job se levantó, y rasgó su manto, y trasquiló su
cabeza, y cayendo en tierra adoró; \bibverse{21} Y dijo: Desnudo salí
del vientre de mi madre, y desnudo tornaré allá. Jehová dió, y Jehová
quitó: sea el nombre de Jehová bendito. \bibverse{22} En todo esto no
pecó Job, ni atribuyó á Dios despropósito alguno.

\hypertarget{section-1}{%
\section{2}\label{section-1}}

\bibverse{1} Y otro día aconteció que vinieron los hijos de Dios para
presentarse delante de Jehová, y Satán vino también entre ellos
pareciendo delante de Jehová. \bibverse{2} Y dijo Jehová á Satán: ¿De
dónde vienes? Respondió Satán á Jehová, y dijo: De rodear la tierra, y
de andar por ella.

\bibverse{3} Y Jehová dijo á Satán: ¿No has considerado á mi siervo Job,
que no hay otro como él en la tierra, varón perfecto y recto, temeroso
de Dios y apartado de mal, y que aun retiene su perfección, habiéndome
tú incitado contra él, para que lo arruinara sin causa?

\bibverse{4} Y respondiendo Satán dijo á Jehová: Piel por piel, todo lo
que el hombre tiene dará por su vida. \bibverse{5} Mas extiende ahora tu
mano, y toca á su hueso y á su carne, y verás si no te blasfema en tu
rostro.

\bibverse{6} Y Jehová dijo á Satán: He aquí, él está en tu mano; mas
guarda su vida.

\bibverse{7} Y salió Satán de delante de Jehová, é hirió á Job de una
maligna sarna desde la planta de su pie hasta la mollera de su cabeza.
\bibverse{8} Y tomaba una teja para rascarse con ella, y estaba sentado
en medio de ceniza. \bibverse{9} Díjole entonces su mujer: ¿Aun retienes
tú tu simplicidad? Bendice á Dios, y muérete.

\bibverse{10} Y él le dijo: Como suele hablar cualquiera de las mujeres
fatuas, has hablado. También recibimos el bien de Dios, ¿y el mal no
recibiremos? En todo esto no pecó Job con sus labios.

\bibverse{11} Y tres amigos de Job, Eliphaz Temanita, y Bildad Suhita, y
Sophar Naamathita, luego que oyeron todo este mal que le había
sobrevenido, vinieron cada uno de su lugar; porque habían concertado de
venir juntos á condolecerse de él, y á consolarle. \bibverse{12} Los
cuales alzando los ojos desde lejos, no lo conocieron, y lloraron á voz
en grito; y cada uno de ellos rasgó su manto, y esparcieron polvo sobre
sus cabezas hacia el cielo. \bibverse{13} Así se sentaron con él en
tierra por siete días y siete noches, y ninguno le hablaba palabra,
porque veían que el dolor era muy grande.

\hypertarget{section-2}{%
\section{3}\label{section-2}}

\bibverse{1} Después de esto abrió Job su boca, y maldijo su día.
\bibverse{2} Y exclamó Job, y dijo: \bibverse{3} Perezca el día en que
yo nací, y la noche que se dijo: Varón es concebido. \bibverse{4} Sea
aquel día sombrío, y Dios no cuide de él desde arriba, ni claridad sobre
él resplandezca. \bibverse{5} Aféenlo tinieblas y sombra de muerte;
repose sobre él nublado, que lo haga horrible como caliginoso día.
\bibverse{6} Ocupe la oscuridad aquella noche; no sea contada entre los
días del año, ni venga en el número de los meses. \bibverse{7} ¡Oh si
fuere aquella noche solitaria, que no viniera canción alguna en ella!
\bibverse{8} Maldíganla los que maldicen al día, los que se aprestan
para levantar su llanto. \bibverse{9} Oscurézcanse las estrellas de su
alba; espere la luz, y no venga, ni vea los párpados de la mañana:
\bibverse{10} Por cuanto no cerró las puertas del vientre donde yo
estaba, ni escondió de mis ojos la miseria. \bibverse{11} ¿Por qué no
morí yo desde la matriz, o fuí traspasado en saliendo del vientre?
\bibverse{12} ¿Por qué me previnieron las rodillas? ¿y para qué las
tetas que mamase? \bibverse{13} Pues que ahora yaciera yo, y reposara;
durmiera, y entonces tuviera reposo, \bibverse{14} Con los reyes y con
los consejeros de la tierra, que edifican para sí los desiertos;
\bibverse{15} O con los príncipes que poseían el oro, que henchían sus
casas de plata. \bibverse{16} O ¿por qué no fuí escondido como aborto,
como los pequeñitos que nunca vieron luz? \bibverse{17} Allí los impíos
dejan el perturbar, y allí descansan los de cansadas fuerzas.
\bibverse{18} Allí asimismo reposan los cautivos; no oyen la voz del
exactor. \bibverse{19} Allí están el chico y el grande; y el siervo
libre de su señor. \bibverse{20} ¿Por qué se da luz al trabajado, y vida
á los de ánimo en amargura, \bibverse{21} Que esperan la muerte, y ella
no llega, aunque la buscan más que tesoros; \bibverse{22} Que se alegran
sobremanera, y se gozan, cuando hallan el sepulcro? \bibverse{23} ¿Por
qué al hombre que no sabe por donde vaya, y al cual Dios ha encerrado?
\bibverse{24} Pues antes que mi pan viene mi suspiro; y mis gemidos
corren como aguas. \bibverse{25} Porque el temor que me espantaba me ha
venido, y hame acontecido lo que temía. \bibverse{26} No he tenido paz,
no me aseguré, ni me estuve reposado; vínome no obstante turbación.

\hypertarget{section-3}{%
\section{4}\label{section-3}}

\bibverse{1} Y respondió Eliphaz el Temanita, y dijo: \bibverse{2} Si
probáremos á hablarte, serte ha molesto; mas ¿quién podrá detener las
palabras? \bibverse{3} He aquí, tú enseñabas á muchos, y las manos
flacas corroborabas; \bibverse{4} Al que vacilaba, enderezaban tus
palabras, y esforzabas las rodillas que decaían. \bibverse{5} Mas ahora
que el mal sobre ti ha venido, te es duro; y cuando ha llegado hasta ti,
te turbas. \bibverse{6} ¿Es este tu temor, tu confianza, tu esperanza, y
la perfección de tus caminos? \bibverse{7} Recapacita ahora, ¿quién que
fuera inocente se perdiera? y ¿en dónde los rectos fueron cortados?
\bibverse{8} Como yo he visto, los que aran iniquidad y siembran
injuria, la siegan. \bibverse{9} Perecen por el aliento de Dios, y por
el espíritu de su furor son consumidos. \bibverse{10} El bramido del
león, y la voz del león, y los dientes de los leoncillos son
quebrantados. \bibverse{11} El león viejo perece por falta de presa, y
los hijos del león son esparcidos. \bibverse{12} El negocio también me
era á mí oculto; mas mi oído ha percibido algo de ello. \bibverse{13} En
imaginaciones de visiones nocturnas, cuando el sueño cae sobre los
hombres, \bibverse{14} Sobrevínome un espanto y un temblor, que
estremeció todos mis huesos: \bibverse{15} Y un espíritu pasó por
delante de mí, que hizo se erizara el pelo de mi carne. \bibverse{16}
Paróse un fantasma delante de mis ojos, cuyo rostro yo no conocí, y
quedo, oí que decía: \bibverse{17} ¿Si será el hombre más justo que
Dios? ¿si será el varón más limpio que el que lo hizo? \bibverse{18} He
aquí que en sus siervos no confía, y notó necedad en sus ángeles;
\bibverse{19} ¡Cuánto más en los que habitan en casas de lodo, cuyo
fundamento está en el polvo, y que serán quebrantados de la polilla!
\bibverse{20} De la mañana á la tarde son quebrantados, y se pierden
para siempre, sin haber quien lo considere. \bibverse{21} ¿Su hermosura,
no se pierde con ellos mismos? Mueren, y sin sabiduría.

\hypertarget{section-4}{%
\section{5}\label{section-4}}

\bibverse{1} Ahora pues da voces, si habrá quien te responda; ¿y á cuál
de los santos te volverás? \bibverse{2} Es cierto que al necio la ira lo
mata, y al codicioso consume la envidia. \bibverse{3} Yo he visto al
necio que echaba raíces, y en la misma hora maldije su habitación.
\bibverse{4} Sus hijos estarán lejos de la salud, y en la puerta serán
quebrantados, y no habrá quien los libre. \bibverse{5} Su mies comerán
los hambrientos, y sacaránla de entre las espinas, y los sedientos
beberán su hacienda. \bibverse{6} Porque la iniquidad no sale del polvo,
ni la molestia brota de la tierra. \bibverse{7} Empero como las
centellas se levantan para volar por el aire, así el hombre nace para la
aflicción. \bibverse{8} Ciertamente yo buscaría á Dios, y depositaría en
él mis negocios: \bibverse{9} El cual hace cosas grandes é
inescrutables, y maravillas que no tienen cuento: \bibverse{10} Que da
la lluvia sobre la haz de la tierra, y envía las aguas por los campos:
\bibverse{11} Que pone los humildes en altura, y los enlutados son
levantados á salud: \bibverse{12} Que frustra los pensamientos de los
astutos, para que sus manos no hagan nada: \bibverse{13} Que prende á
los sabios en la astucia de ellos, y el consejo de los perversos es
entontecido; \bibverse{14} De día se topan con tinieblas, y en mitad del
día andan á tientas como de noche: \bibverse{15} Y libra de la espada al
pobre, de la boca de los impíos, y de la mano violenta; \bibverse{16}
Pues es esperanza al menesteroso, y la iniquidad cerrará su boca.
\bibverse{17} He aquí, bienaventurado es el hombre á quien Dios castiga:
por tanto no menosprecies la corrección del Todopoderoso. \bibverse{18}
Porque él es el que hace la llaga, y él la vendará: él hiere, y sus
manos curan. \bibverse{19} En seis tribulaciones te librará, y en la
séptima no te tocará el mal. \bibverse{20} En el hambre te redimirá de
la muerte, y en la guerra de las manos de la espada. \bibverse{21} Del
azote de la lengua serás encubierto; ni temerás de la destrucción cuando
viniere. \bibverse{22} De la destrucción y del hambre te reirás, y no
temerás de las bestias del campo: \bibverse{23} Pues aun con las piedras
del campo tendrás tu concierto, y las bestias del campo te serán
pacíficas. \bibverse{24} Y sabrás que hay paz en tu tienda; y visitarás
tu morada, y no pecarás. \bibverse{25} Asimismo echarás de ver que tu
simiente es mucha, y tu prole como la hierba de la tierra. \bibverse{26}
Y vendrás en la vejez á la sepultura, como el montón de trigo que se
coge á su tiempo. \bibverse{27} He aquí lo que hemos inquirido, lo cual
es así: óyelo, y juzga tú para contigo.

\hypertarget{section-5}{%
\section{6}\label{section-5}}

\bibverse{1} Y respondió Job y dijo: \bibverse{2} ¡Oh si pesasen al
justo mi queja y mi tormento, y se alzasen igualmente en balanza!
\bibverse{3} Porque pesaría aquél más que la arena del mar: y por tanto
mis palabras son cortadas. \bibverse{4} Porque las saetas del
Todopoderoso están en mí, cuyo veneno bebe mi espíritu; y terrores de
Dios me combaten. \bibverse{5} ¿Acaso gime el asno montés junto á la
hierba? ¿muge el buey junto á su pasto? \bibverse{6} ¿Comeráse lo
desabrido sin sal? ¿ó habrá gusto en la clara del huevo? \bibverse{7}
Las cosas que mi alma no quería tocar, por los dolores son mi comida.
\bibverse{8} ¡Quién me diera que viniese mi petición, y que Dios me
otorgase lo que espero; \bibverse{9} Y que pluguiera á Dios
quebrantarme; que soltara su mano, y me deshiciera! \bibverse{10} Y
sería aún mi consuelo, si me asaltase con dolor sin dar más tregua, que
yo no he escondido las palabras del Santo. \bibverse{11} ¿Cuál es mi
fortaleza para esperar aún? ¿y cuál mi fin para dilatar mi vida?
\bibverse{12} ¿Es mi fortaleza la de las piedras? ¿ó mi carne, es de
acero? \bibverse{13} ¿No me ayudo cuanto puedo, y el poder me falta del
todo? \bibverse{14} El atribulado es consolado de su compañero: mas hase
abandonado el temor del Omnipotente. \bibverse{15} Mis hermanos han
mentido cual arroyo: pasáronse como corrientes impetuosas, \bibverse{16}
Que están escondidas por la helada, y encubiertas con nieve;
\bibverse{17} Que al tiempo del calor son deshechas, y en calentándose,
desaparecen de su lugar; \bibverse{18} Apártanse de la senda de su
rumbo, van menguando y piérdense. \bibverse{19} Miraron los caminantes
de Temán, los caminantes de Saba esperaron en ellas: \bibverse{20} Mas
fueron avergonzados por su esperanza; porque vinieron hasta ellas, y
halláronse confusos. \bibverse{21} Ahora ciertamente como ellas sois
vosotros: que habéis visto el tormento, y teméis. \bibverse{22} ¿Os he
dicho yo: Traedme, y pagad por mí de vuestra hacienda; \bibverse{23} Y
libradme de la mano del opresor, y redimidme del poder de los violentos?
\bibverse{24} Enseñadme, y yo callaré: y hacedme entender en qué he
errado. \bibverse{25} ¡Cuán fuertes son las palabras de rectitud! Mas
¿qué reprende el que reprende de vosotros? \bibverse{26} ¿Pensáis
censurar palabras, y los discursos de un desesperado, que son como el
viento? \bibverse{27} También os arrojáis sobre el huérfano, y hacéis
hoyo delante de vuestro amigo. \bibverse{28} Ahora pues, si queréis,
mirad en mí, y ved si miento delante de vosotros. \bibverse{29} Tornad
ahora, y no haya iniquidad; volved aún á considerar mi justicia en esto.
\bibverse{30} ¿Hay iniquidad en mi lengua? ¿no puede mi paladar
discernir las cosas depravadas?

\hypertarget{section-6}{%
\section{7}\label{section-6}}

\bibverse{1} Ciertamente tiempo limitado tiene el hombre sobre la
tierra, y sus días son como los días del jornalero. \bibverse{2} Como el
siervo anhela la sombra, y como el jornalero espera el reposo de su
trabajo: \bibverse{3} Así poseo yo meses de vanidad, y noches de trabajo
me dieron por cuenta. \bibverse{4} Cuando estoy acostado, digo: ¿Cuándo
me levantaré? Y mide mi corazón la noche, y estoy harto de devaneos
hasta el alba. \bibverse{5} Mi carne está vestida de gusanos, y de
costras de polvo; mi piel hendida y abominable. \bibverse{6} Y mis días
fueron más ligeros que la lanzadera del tejedor, y fenecieron sin
esperanza. \bibverse{7} Acuérdate que mi vida es viento, y que mis ojos
no volverán á ver el bien. \bibverse{8} Los ojos de los que me ven, no
me verán más: tus ojos sobre mí, y dejaré de ser. \bibverse{9} La nube
se consume, y se va: así el que desciende al sepulcro no subirá;
\bibverse{10} No tornará más á su casa, ni su lugar le conocerá más.
\bibverse{11} Por tanto yo no reprimiré mi boca; hablaré en la angustia
de mi espíritu, y quejaréme con la amargura de mi alma. \bibverse{12}
¿Soy yo la mar, ó ballena, que me pongas guarda? \bibverse{13} Cuando
digo: Mi cama me consolará, mi cama atenuará mis quejas; \bibverse{14}
Entonces me quebrantarás con sueños, y me turbarás con visiones.
\bibverse{15} Y así mi alma tuvo por mejor el ahogamiento, y quiso la
muerte más que mis huesos. \bibverse{16} Aburríme: no he de vivir yo
para siempre; déjame, pues que mis días son vanidad. \bibverse{17} ¿Qué
es el hombre, para que lo engrandezcas, y que pongas sobre él tu
corazón, \bibverse{18} Y lo visites todas las mañanas, y todos los
momentos lo pruebes? \bibverse{19} ¿Hasta cuándo no me dejarás, ni me
soltarás hasta que trague mi saliva? \bibverse{20} Pequé, ¿qué te haré,
oh Guarda de los hombres? ¿por qué me has puesto contrario á ti, y que á
mí mismo sea pesado? \bibverse{21} ¿Y por qué no quitas mi rebelión, y
perdonas mi iniquidad? porque ahora dormiré en el polvo, y si me
buscares de mañana, ya no seré.

\hypertarget{section-7}{%
\section{8}\label{section-7}}

\bibverse{1} Y respondió Bildad Suhita, y dijo: \bibverse{2} ¿Hasta
cuándo hablarás tales cosas, y las palabras de tu boca serán como un
viento fuerte? \bibverse{3} ¿Acaso pervertirá Dios el derecho, ó el
Todopoderoso pervertirá la justicia? \bibverse{4} Si tus hijos pecaron
contra él, él los echó en el lugar de su pecado. \bibverse{5} Si tú de
mañana buscares á Dios, y rogares al Todopoderoso; \bibverse{6} Si
fueres limpio y derecho, cierto luego se despertará sobre ti, y hará
próspera la morada de tu justicia. \bibverse{7} Y tu principio habrá
sido pequeño, y tu postrimería acrecerá en gran manera. \bibverse{8}
Porque pregunta ahora á la edad pasada, y disponte para inquirir de sus
padres de ellos; \bibverse{9} Pues nosotros somos de ayer, y no sabemos,
siendo nuestros días sobre la tierra como sombra. \bibverse{10} ¿No te
enseñarán ellos, te dirán, y de su corazón sacarán palabras?
\bibverse{11} ¿Crece el junco sin lodo? ¿crece el prado sin agua?
\bibverse{12} Aun él en su verdor no será cortado, y antes de toda
hierba se secará. \bibverse{13} Tales son los caminos de todos los que
olvidan á Dios: y la esperanza del impío perecerá: \bibverse{14} Porque
su esperanza será cortada, y su confianza es casa de araña.
\bibverse{15} Apoyaráse él sobre su casa, mas no permanecerá en pie;
atendráse á ella, mas no se afirmará. \bibverse{16} A manera de un
árbol, está verde delante del sol, y sus renuevos salen sobre su huerto;
\bibverse{17} Vanse entretejiendo sus raíces junto á una fuente, y
enlazándose hasta un lugar pedregoso. \bibverse{18} Si le arrancaren de
su lugar, éste negarále entonces, diciendo: Nunca te vi. \bibverse{19}
Ciertamente éste será el gozo de su camino; y de la tierra de donde se
traspusiere, nacerán otros. \bibverse{20} He aquí, Dios no aborrece al
perfecto, ni toma la mano de los malignos. \bibverse{21} Aun henchirá tu
boca de risa, y tus labios de júbilo. \bibverse{22} Los que te
aborrecen, serán vestidos de confusión; y la habitación de los impíos
perecerá.

\hypertarget{section-8}{%
\section{9}\label{section-8}}

\bibverse{1} Y respondió Job, y dijo: \bibverse{2} Ciertamente yo
conozco que es así: ¿y cómo se justificará el hombre con Dios?
\bibverse{3} Si quisiere contender con él, no le podrá responder á una
cosa de mil. \bibverse{4} El es sabio de corazón, y poderoso en
fortaleza: ¿quién se endureció contra él, y quedó en paz? \bibverse{5}
Que arranca los montes con su furor, y no conocen quién los trastornó:
\bibverse{6} Que remueve la tierra de su lugar, y hace temblar sus
columnas: \bibverse{7} Que manda al sol, y no sale; y sella las
estrellas: \bibverse{8} El que extiende solo los cielos, y anda sobre
las alturas de la mar: \bibverse{9} El que hizo el Arcturo, y el Orión,
y las Pléyadas, y los lugares secretos del mediodía: \bibverse{10} El
que hace cosas grandes é incomprensibles, y maravillosas, sin número.
\bibverse{11} He aquí que él pasará delante de mí, y yo no lo veré; y
pasará, y no lo entenderé. \bibverse{12} He aquí, arrebatará; ¿quién le
hará restituir? ¿Quién le dirá, Qué haces? \bibverse{13} Dios no tornará
atrás su ira, y debajo de él se encorvan los que ayudan á los soberbios.
\bibverse{14} ¿Cuánto menos le responderé yo, y hablaré con él palabras
estudiadas? \bibverse{15} Que aunque fuese yo justo, no responderé;
antes habré de rogar á mi juez. \bibverse{16} Que si yo le invocase, y
él me respondiese, aun no creeré que haya escuchado mi voz.
\bibverse{17} Porque me ha quebrado con tempestad, y ha aumentado mis
heridas sin causa. \bibverse{18} No me ha concedido que tome mi aliento;
mas hame hartado de amarguras. \bibverse{19} Si habláremos de su
potencia, fuerte por cierto es; si de juicio, ¿quién me emplazará?
\bibverse{20} Si yo me justificare, me condenará mi boca; si me dijere
perfecto, esto me hará inicuo. \bibverse{21} Bien que yo fuese íntegro,
no conozco mi alma: reprocharé mi vida. \bibverse{22} Una cosa resta que
yo diga: Al perfecto y al impío él los consume. \bibverse{23} Si azote
mata de presto, ríese de la prueba de los inocentes. \bibverse{24} La
tierra es entregada en manos de los impíos, y él cubre el rostro de sus
jueces. Si no es él, ¿quién es? ¿dónde está? \bibverse{25} Mis días han
sido más ligeros que un correo; huyeron, y no vieron el bien.
\bibverse{26} Pasaron cual navíos veloces: como el águila que se arroja
á la comida. \bibverse{27} Si digo: Olvidaré mi queja, dejaré mi
aburrimiento, y esforzaréme: \bibverse{28} Contúrbanme todos mis
trabajos; sé que no me darás por libre. \bibverse{29} Yo soy impío,
¿para qué trabajaré en vano? \bibverse{30} Aunque me lave con aguas de
nieve, y limpie mis manos con la misma limpieza, \bibverse{31} Aun me
hundirás en el hoyo, y mis propios vestidos me abominarán. \bibverse{32}
Porque no es hombre como yo, para que yo le responda, y vengamos
juntamente á juicio. \bibverse{33} No hay entre nosotros árbitro que
ponga su mano sobre nosotros ambos. \bibverse{34} Quite de sobre mí su
vara, y su terror no me espante. \bibverse{35} Entonces hablaré, y no le
temeré: porque así no estoy en mí mismo.

\hypertarget{section-9}{%
\section{10}\label{section-9}}

\bibverse{1} Está mi alma aburrida de mi vida: daré yo suelta á mi queja
sobre mí, hablaré con amargura de mi alma. \bibverse{2} Diré á Dios: no
me condenes; hazme entender por qué pleiteas conmigo. \bibverse{3}
¿Parécete bien que oprimas, que deseches la obra de tus manos, y que
resplandezcas sobre el consejo de los impíos? \bibverse{4} ¿Tienes tú
ojos de carne? ¿ves tú como ve el hombre? \bibverse{5} ¿Son tus días
como los días del hombre, ó tus años como los tiempos humanos,
\bibverse{6} Para que inquieras mi iniquidad, y busques mi pecado,
\bibverse{7} Sobre saber tú que no soy impío, y que no hay quien de tu
mano libre? \bibverse{8} Tus manos me formaron y me compusieron todo en
contorno: ¿y así me deshaces? \bibverse{9} Acuérdate ahora que como á
lodo me diste forma: ¿y en polvo me has de tornar? \bibverse{10} ¿No me
fundiste como leche, y como un queso me cuajaste? \bibverse{11}
Vestísteme de piel y carne, y cubrísteme de huesos y nervios.
\bibverse{12} Vida y misericordia me concediste, y tu visitación guardó
mi espíritu. \bibverse{13} Y estas cosas tienes guardadas en tu corazón;
yo sé que esto está cerca de ti. \bibverse{14} Si pequé, tú me has
observado, y no me limpias de mi iniquidad. \bibverse{15} Si fuere malo,
¡ay de mí! y si fuere justo, no levantaré mi cabeza, estando harto de
deshonra, y de verme afligido. \bibverse{16} Y subirá de punto, pues me
cazas como á león, y tornas á hacer en mí maravillas. \bibverse{17}
Renuevas contra mí tus plagas, y aumentas conmigo tu furor, remudándose
sobre mí ejércitos. \bibverse{18} ¿Por qué me sacaste de la matriz?
Habría yo espirado, y no me vieran ojos. \bibverse{19} Fuera, como si
nunca hubiera sido, llevado desde el vientre á la sepultura.
\bibverse{20} ¿No son mis días poca cosa? Cesa pues, y déjame, para que
me conforte un poco. \bibverse{21} Antes que vaya para no volver, á la
tierra de tinieblas y de sombra de muerte; \bibverse{22} Tierra de
oscuridad, lóbrega como sombra de muerte, sin orden, y que aparece como
la oscuridad misma.

\hypertarget{section-10}{%
\section{11}\label{section-10}}

\bibverse{1} Y respondió Sophar Naamathita, y dijo: \bibverse{2} ¿Las
muchas palabras no han de tener respuesta? ¿y el hombre parlero será
justificado? \bibverse{3} ¿Harán tus falacias callar á los hombres? ¿y
harás escarnio, y no habrá quien te avergüence? \bibverse{4} Tú dices:
Mi conversar es puro, y yo soy limpio delante de tus ojos. \bibverse{5}
Mas ¡oh quién diera que Dios hablara, y abriera sus labios contigo,
\bibverse{6} Y que te declarara los arcanos de la sabiduría, que son de
doble valor que la hacienda! Conocerías entonces que Dios te ha
castigado menos que tu iniquidad merece. \bibverse{7} ¿Alcanzarás tú el
rastro de Dios? ¿llegarás tú á la perfección del Todopoderoso?
\bibverse{8} Es más alto que los cielos: ¿qué harás? Es más profundo que
el infierno: ¿cómo lo conocerás? \bibverse{9} Su dimensión es más larga
que la tierra, y más ancha que la mar. \bibverse{10} Si cortare, ó
encerrare, ó juntare, ¿quién podrá contrarrestarle? \bibverse{11} Porque
él conoce á los hombres vanos: ve asimismo la iniquidad, ¿y no hará
caso? \bibverse{12} El hombre vano se hará entendido, aunque nazca como
el pollino del asno montés. \bibverse{13} Si tú apercibieres tu corazón,
y extendieres á él tus manos; \bibverse{14} Si alguna iniquidad hubiere
en tu mano, y la echares de ti, y no consintieres que more maldad en tus
habitaciones; \bibverse{15} Entonces levantarás tu rostro limpio de
mancha, y serás fuerte y no temerás: \bibverse{16} Y olvidarás tu
trabajo, ó te acordarás de él como de aguas que pasaron: \bibverse{17} Y
en mitad de la siesta se levantará bonanza; resplandecerás, y serás como
la mañana: \bibverse{18} Y confiarás, que habrá esperanza; y cavarás, y
dormirás seguro: \bibverse{19} Y te acostarás, y no habrá quien te
espante: y muchos te rogarán. \bibverse{20} Mas los ojos de los malos se
consumirán, y no tendrán refugio; y su esperanza será agonía del alma.

\hypertarget{section-11}{%
\section{12}\label{section-11}}

\bibverse{1} Y respondió Job, y dijo: \bibverse{2} Ciertamente que
vosotros sois el pueblo; y con vosotros morirá la sabiduría.
\bibverse{3} También tengo yo seso como vosotros; no soy yo menos que
vosotros: ¿y quién habrá que no pueda decir otro tanto? \bibverse{4} Yo
soy uno de quien su amigo se mofa, que invoca á Dios, y él le responde:
con todo, el justo y perfecto es escarnecido. \bibverse{5} Aquel cuyos
pies van á resbalar, es como una lámpara despreciada de aquel que está á
sus anchuras. \bibverse{6} Prosperan las tiendas de los ladrones, y los
que provocan á Dios viven seguros; en cuyas manos él ha puesto cuanto
tienen. \bibverse{7} Y en efecto, pregunta ahora á las bestias, que
ellas te enseñarán; y á las aves de los cielos, que ellas te lo
mostrarán: \bibverse{8} O habla á la tierra, que ella te enseñará; los
peces de la mar te lo declararán también. \bibverse{9} ¿Qué cosa de
todas estas no entiende que la mano de Jehová la hizo? \bibverse{10} En
su mano está el alma de todo viviente, y el espíritu de toda carne
humana. \bibverse{11} Ciertamente el oído distingue las palabras, y el
paladar gusta las viandas. \bibverse{12} En los viejos está la ciencia,
y en la larga edad la inteligencia. \bibverse{13} Con Dios está la
sabiduría y la fortaleza; suyo es el consejo y la inteligencia.
\bibverse{14} He aquí, él derribará, y no será edificado: encerrará al
hombre, y no habrá quien le abra. \bibverse{15} He aquí, él detendrá las
aguas, y se secarán; él las enviará, y destruirán la tierra.
\bibverse{16} Con él está la fortaleza y la existencia; suyo es el que
yerra, y el que hace errar. \bibverse{17} El hace andar á los consejeros
desnudos de consejo, y hace enloquecer á los jueces. \bibverse{18} El
suelta la atadura de los tiranos, y ata el cinto á sus lomos.
\bibverse{19} El lleva despojados á los príncipes, y trastorna á los
poderosos. \bibverse{20} El impide el labio á los que dicen verdad, y
quita á los ancianos el consejo. \bibverse{21} El derrama menosprecio
sobre los príncipes, y enflaquece la fuerza de los esforzados.
\bibverse{22} El descubre las profundidades de las tinieblas, y saca á
luz la sombra de muerte. \bibverse{23} El multiplica las gentes, y él
las destruye: él esparce las gentes, y las torna á recoger.
\bibverse{24} El quita el seso de las cabezas del pueblo de la tierra, y
háceles que se pierdan vagueando sin camino: \bibverse{25} Van á tientas
como en tinieblas y sin luz, y los hace errar como borrachos.

\hypertarget{section-12}{%
\section{13}\label{section-12}}

\bibverse{1} He AQUÍ que todas estas cosas han visto mis ojos, y oído y
entendido de por sí mis oídos. \bibverse{2} Como vosotros lo sabéis, lo
sé yo; no soy menos que vosotros. \bibverse{3} Mas yo hablaría con el
Todopoderoso, y querría razonar con Dios. \bibverse{4} Que ciertamente
vosotros sois fraguadores de mentira; sois todos vosotros médicos nulos.
\bibverse{5} Ojalá callarais del todo, porque os fuera sabiduría.
\bibverse{6} Oid ahora mi razonamiento, y estad atentos á los argumentos
de mis labios. \bibverse{7} ¿Habéis de hablar iniquidad por Dios?
¿habéis de hablar por él engaño? \bibverse{8} ¿Habéis de hacer acepción
de su persona? ¿habéis de pleitear vosotros por Dios? \bibverse{9}
¿Sería bueno que él os escudriñase? ¿os burlaréis de él como quien se
burla de algún hombre? \bibverse{10} El os reprochará de seguro, si
solapadamente hacéis acepción de personas. \bibverse{11} De cierto su
alteza os había de espantar, y su pavor había de caer sobre vosotros.
\bibverse{12} Vuestras memorias serán comparadas á la ceniza, y vuestros
cuerpos como cuerpos de lodo. \bibverse{13} Escuchadme, y hablaré yo, y
véngame después lo que viniere. \bibverse{14} ¿Por qué quitaré yo mi
carne con mis dientes, y pondré mi alma en mi mano? \bibverse{15} He
aquí, aunque me matare, en él esperaré; empero defenderé delante de él
mis caminos. \bibverse{16} Y él mismo me será salud, porque no entrará
en su presencia el hipócrita. \bibverse{17} Oid con atención mi
razonamiento, y mi denunciación con vuestros oídos. \bibverse{18} He
aquí ahora, si yo me apercibiere á juicio, sé que seré justificado.
\bibverse{19} ¿Quién es el que pleiteará conmigo? porque si ahora yo
callara, fenecería. \bibverse{20} A lo menos dos cosas no hagas conmigo;
entonces no me esconderé de tu rostro: \bibverse{21} Aparta de mí tu
mano, y no me asombre tu terror. \bibverse{22} Llama luego, y yo
responderé; ó yo hablaré, y respóndeme tú. \bibverse{23} ¿Cuántas
iniquidades y pecados tengo yo? hazme entender mi prevaricación y mi
pecado. \bibverse{24} ¿Por qué escondes tu rostro, y me cuentas por tu
enemigo? \bibverse{25} ¿A la hoja arrebatada has de quebrantar? ¿y á una
arista seca has de perseguir? \bibverse{26} ¿Por qué escribes contra mí
amarguras, y me haces cargo de los pecados de mi mocedad? \bibverse{27}
Pones además mis pies en el cepo, y guardas todos mis caminos,
imprimiéndolo á las raíces de mis pies. \bibverse{28} Y el cuerpo mío se
va gastando como de carcoma, como vestido que se come de polilla.

\hypertarget{section-13}{%
\section{14}\label{section-13}}

\bibverse{1} El HOMBRE nacido de mujer, corto de días, y harto de
sinsabores: \bibverse{2} Que sale como una flor y es cortado; y huye
como la sombra, y no permanece. \bibverse{3} ¿Y sobre éste abres tus
ojos, y me traes á juicio contigo? \bibverse{4} ¿Quién hará limpio de
inmundo? Nadie. \bibverse{5} Ciertamente sus días están determinados, y
el número de sus meses está cerca de ti: tú le pusiste términos, de los
cuales no pasará. \bibverse{6} Si tú lo dejares, él dejará de ser: entre
tanto deseará, como el jornalero, su día. \bibverse{7} Porque si el
árbol fuere cortado, aun queda de él esperanza; retoñecerá aún, y sus
renuevos no faltarán. \bibverse{8} Si se envejeciere en la tierra su
raíz, y su tronco fuere muerto en el polvo, \bibverse{9} Al percibir el
agua reverdecerá, y hará copa como planta. \bibverse{10} Mas el hombre
morirá, y será cortado; y perecerá el hombre, ¿y dónde estará él?
\bibverse{11} Las aguas de la mar se fueron, y agotóse el río, secóse.
\bibverse{12} Así el hombre yace, y no se tornará á levantar: hasta que
no haya cielo no despertarán, ni se levantarán de su sueño.
\bibverse{13} ¡Oh quién me diera que me escondieses en el sepulcro, que
me encubrieras hasta apaciguarse tu ira, que me pusieses plazo, y de mí
te acordaras! \bibverse{14} Si el hombre muriere, ¿volverá á vivir?
Todos los días de mi edad esperaré, hasta que venga mi mutación.
\bibverse{15} Aficionado á la obra de tus manos, llamarás, y yo te
responderé. \bibverse{16} Pues ahora me cuentas los pasos, y no das
tregua á mi pecado. \bibverse{17} Tienes sellada en saco mi
prevaricación, y coacervas mi iniquidad. \bibverse{18} Y ciertamente el
monte que cae se deshace, y las peñas son traspasadas de su lugar;
\bibverse{19} Las piedras son desgastadas con el agua impetuosa, que se
lleva el polvo de la tierra: de tal manera haces tú perecer la esperanza
del hombre. \bibverse{20} Para siempre serás más fuerte que él, y él se
va; demudarás su rostro, y enviaráslo. \bibverse{21} Sus hijos serán
honrados, y él no lo sabrá; ó serán humillados, y no entenderá de ellos.
\bibverse{22} Mas su carne sobre él se dolerá, y entristecerse ha en él
su alma.

\hypertarget{section-14}{%
\section{15}\label{section-14}}

\bibverse{1} Y respondió Eliphaz Temanita, y dijo: \bibverse{2} ¿Si
proferirá el sabio vana sabiduría, y henchirá su vientre de viento
solano? \bibverse{3} ¿Disputará con palabras inútiles, y con razones sin
provecho? \bibverse{4} Tú también disipas el temor, y menoscabas la
oración delante de Dios. \bibverse{5} Porque tu boca declaró tu
iniquidad, pues has escogido el hablar de los astutos. \bibverse{6} Tu
boca te condenará, y no yo; y tus labios testificarán contra ti.
\bibverse{7} ¿Naciste tú primero que Adam? ¿ó fuiste formado antes que
los collados? \bibverse{8} ¿Oíste tú el secreto de Dios, que detienes en
ti solo la sabiduría? \bibverse{9} ¿Qué sabes tú que no sepamos? ¿qué
entiendes que no se halle en nosotros? \bibverse{10} Entre nosotros
también hay cano, también hay viejo mucho mayor en días que tu padre.
\bibverse{11} ¿En tan poco tienes las consolaciones de Dios? ¿tienes
acaso alguna cosa oculta cerca de ti? \bibverse{12} ¿Por qué te enajena
tu corazón, y por qué guiñan tus ojos, \bibverse{13} Pues haces frente á
Dios con tu espíritu, y sacas tales palabras de tu boca? \bibverse{14}
¿Qué cosa es el hombre para que sea limpio, y que se justifique el
nacido de mujer? \bibverse{15} He aquí que en sus santos no confía, y ni
los cielos son limpios delante de sus ojos: \bibverse{16} ¿Cuánto menos
el hombre abominable y vil, que bebe la iniquidad como agua?
\bibverse{17} Escúchame; yo te mostraré y te contaré lo que he visto:
\bibverse{18} (Lo que los sabios nos contaron de sus padres, y no lo
encubrieron; \bibverse{19} A los cuales solos fué dada la tierra, y no
pasó extraño por medio de ellos:) \bibverse{20} Todos los días del
impío, él es atormentado de dolor, y el número de años es escondido al
violento. \bibverse{21} Estruendos espantosos hay en sus oídos; en la
paz le vendrá quien lo asuele. \bibverse{22} El no creerá que ha de
volver de las tinieblas, y está mirando al cuchillo. \bibverse{23}
Desasosegado á comer siempre, sabe que le está aparejado día de
tinieblas. \bibverse{24} Tribulación y angustia le asombrarán, y
esforzaránse contra él como un rey apercibido para la batalla.
\bibverse{25} Por cuanto él extendió su mano contra Dios, y se esforzó
contra el Todopoderoso, \bibverse{26} El le acometerá en la cerviz, en
lo grueso de las hombreras de sus escudos: \bibverse{27} Porque cubrió
su rostro con su gordura, é hizo pliegues sobre los ijares;
\bibverse{28} Y habitó las ciudades asoladas, las casas inhabitadas, que
estaban puestas en montones. \bibverse{29} No enriquecerá, ni será firme
su potencia, ni extenderá por la tierra su hermosura. \bibverse{30} No
se escapará de las tinieblas: la llama secará sus ramos, y con el
aliento de su boca perecerá. \bibverse{31} No confíe el iluso en la
vanidad; porque ella será su recompensa. \bibverse{32} El será cortado
antes de su tiempo, y sus renuevos no reverdecerán. \bibverse{33} El
perderá su agraz como la vid, y derramará su flor como la oliva.
\bibverse{34} Porque la sociedad de los hipócritas será asolada, y fuego
consumirá las tiendas de soborno. \bibverse{35} Concibieron dolor, y
parieron iniquidad; y las entrañas de ellos meditan engaño.

\hypertarget{section-15}{%
\section{16}\label{section-15}}

\bibverse{1} Y respondió Job, y dijo: \bibverse{2} Muchas veces he oído
cosas como estas: consoladores molestos sois todos vosotros.
\bibverse{3} ¿Tendrán fin las palabras ventosas? ó ¿qué te animará á
responder? \bibverse{4} También yo hablaría como vosotros. Ojalá vuestra
alma estuviera en lugar de la mía, que yo os tendría compañía en las
palabras, y sobre vosotros movería mi cabeza. \bibverse{5} Mas yo os
alentaría con mis palabras, y la consolación de mis labios apaciguaría
el dolor vuestro. \bibverse{6} Si hablo, mi dolor no cesa; y si dejo de
hablar, no se aparta de mí. \bibverse{7} Empero ahora me ha fatigado:
has tú asolado toda mi compañía. \bibverse{8} Tú me has arrugado;
testigo es mi flacura, que se levanta contra mí para testificar en mi
rostro. \bibverse{9} Su furor me destrizó, y me ha sido contrario:
crujió sus dientes contra mí; contra mí aguzó sus ojos mi enemigo.
\bibverse{10} Abrieron contra mí su boca; hirieron mis mejillas con
afrenta; contra mí se juntaron todos. \bibverse{11} Hame entregado Dios
al mentiroso, y en las manos de los impíos me hizo estremecer
\bibverse{12} Próspero estaba, y desmenuzóme: y arrebatóme por la
cerviz, y despedazóme, y púsome por blanco suyo. \bibverse{13}
Cercáronme sus flecheros, partió mis riñones, y no perdonó: mi hiel
derramó por tierra. \bibverse{14} Quebrantóme de quebranto sobre
quebranto; corrió contra mí como un gigante. \bibverse{15} Yo cosí saco
sobre mi piel, y cargué mi cabeza de polvo. \bibverse{16} Mi rostro está
enlodado con lloro, y mis párpados entenebrecidos: \bibverse{17} A pesar
de no haber iniquidad en mis manos, y de haber sido mi oración pura.
\bibverse{18} ¡Oh tierra! no cubras mi sangre, y no haya lugar á mi
clamor. \bibverse{19} Mas he aquí que en los cielos está mi testigo, y
mi testimonio en las alturas. \bibverse{20} Disputadores son mis amigos:
mas á Dios destilarán mis ojos. \bibverse{21} ¡Ojalá pudiese disputar el
hombre con Dios, como con su prójimo! \bibverse{22} Mas los años
contados vendrán, y yo iré el camino por donde no volveré.

\hypertarget{section-16}{%
\section{17}\label{section-16}}

\bibverse{1} Mi ALIENTO está corrompido, acórtanse mis días, y me está
aparejado el sepulcro. \bibverse{2} No hay conmigo sino escarnecedores,
en cuya acrimonia se detienen mis ojos. \bibverse{3} Pon ahora, dame
fianza para litigar contigo: ¿quién tocará ahora mi mano? \bibverse{4}
Porque á éstos has tú escondido su corazón de inteligencia: por tanto,
no los ensalzarás. \bibverse{5} El que denuncia lisonjas á sus prójimos,
los ojos de sus hijos desfallezcan. \bibverse{6} El me ha puesto por
parábola de pueblos, y delante de ellos he sido como tamboril.
\bibverse{7} Y mis ojos se oscurecieron de desabrimiento, y mis
pensamientos todos son como sombra. \bibverse{8} Los rectos se
maravillarán de esto, y el inocente se levantará contra el hipócrita.
\bibverse{9} No obstante, proseguirá el justo su camino, y el limpio de
manos aumentará la fuerza. \bibverse{10} Mas volved todos vosotros, y
venid ahora, que no hallaré entre vosotros sabio. \bibverse{11}
Pasáronse mis días, fueron arrancados mis pensamientos, los designios de
mi corazón. \bibverse{12} Pusieron la noche por día, y la luz se acorta
delante de las tinieblas. \bibverse{13} Si yo espero, el sepulcro es mi
casa: haré mi cama en las tinieblas. \bibverse{14} A la huesa tengo
dicho: Mi padre eres tú; á los gusanos: Mi madre y mi hermana.
\bibverse{15} ¿Dónde pues estará ahora mi esperanza? y mi esperanza
¿quién la verá? \bibverse{16} A los rincones de la huesa descenderán, y
juntamente descansarán en el polvo.

\hypertarget{section-17}{%
\section{18}\label{section-17}}

\bibverse{1} Y respondió Bildad Suhita, y dijo: \bibverse{2} ¿Cuándo
pondréis fin á las palabras? Entended, y después hablemos. \bibverse{3}
¿Por qué somos tenidos por bestias, y en vuestros ojos somos viles?
\bibverse{4} Oh tú, que despedazas tu alma con tu furor, ¿será dejada la
tierra por tu causa, y serán traspasadas de su lugar las peñas?
\bibverse{5} Ciertamente la luz de los impíos será apagada, y no
resplandecerá la centella de su fuego. \bibverse{6} La luz se oscurecerá
en su tienda, y apagaráse sobre él su lámpara. \bibverse{7} Los pasos de
su pujanza serán acortados, y precipitarálo su mismo consejo.
\bibverse{8} Porque red será echada en sus pies, y sobre red andará.
\bibverse{9} Lazo prenderá su calcañar: afirmaráse la trampa contra él.
\bibverse{10} Su cuerda está escondida en la tierra, y su torzuelo sobre
la senda. \bibverse{11} De todas partes lo asombrarán temores, y haránle
huir desconcertado. \bibverse{12} Su fuerza será hambrienta, y á su lado
estará aparejado quebrantamiento. \bibverse{13} El primogénito de la
muerte comerá los ramos de su piel, y devorará sus miembros.
\bibverse{14} Su confianza será arrancada de su tienda, y harále esto
llevar al rey de los espantos. \bibverse{15} En su tienda morará como si
no fuese suya: piedra azufre será esparcida sobre su morada.
\bibverse{16} Abajo se secarán sus raíces, y arriba serán cortadas sus
ramas. \bibverse{17} Su memoria perecerá de la tierra, y no tendrá
nombre por las calles. \bibverse{18} De la luz será lanzado á las
tinieblas, y echado fuera del mundo. \bibverse{19} No tendrá hijo ni
nieto en su pueblo, ni quien le suceda en sus moradas. \bibverse{20}
Sobre su día se espantarán los por venir, como ocupó el pavor á los que
fueron antes. \bibverse{21} Ciertamente tales son las moradas del impío,
y este será el lugar del que no conoció á Dios.

\hypertarget{section-18}{%
\section{19}\label{section-18}}

\bibverse{1} Y respondió Job, y dijo: \bibverse{2} ¿Hasta cuándo
angustiaréis mi alma, y me moleréis con palabras? \bibverse{3} Ya me
habéis vituperado diez veces: ¿no os avergonzáis de descomediros delante
de mí? \bibverse{4} Sea así que realmente haya yo errado, conmigo se
quedará mi yerro. \bibverse{5} Mas si vosotros os engrandeciereis contra
mí, y adujereis contra mí mi oprobio, \bibverse{6} Sabed ahora que Dios
me ha trastornado, y traído en derredor su red sobre mí. \bibverse{7} He
aquí yo clamaré agravio, y no seré oído: daré voces, y no habrá juicio.
\bibverse{8} Cercó de vallado mi camino, y no pasaré; y sobre mis
veredas puso tinieblas. \bibverse{9} Hame despojado de mi gloria, y
quitado la corona de mi cabeza. \bibverse{10} Arruinóme por todos lados,
y perezco; y ha hecho pasar mi esperanza como árbol arrancado.
\bibverse{11} E hizo inflamar contra mí su furor, y contóme para sí
entre sus enemigos. \bibverse{12} Vinieron sus ejércitos á una, y
trillaron sobre mí su camino, y asentaron campo en derredor de mi
tienda. \bibverse{13} Hizo alejar de mí mis hermanos, y positivamente se
extrañaron de mí mis conocidos. \bibverse{14} Mis parientes se
detuvieron, y mis conocidos se olvidaron de mí. \bibverse{15} Los
moradores de mi casa y mis criadas me tuvieron por extraño: forastero
fuí yo en sus ojos. \bibverse{16} Llamé á mi siervo, y no respondió; de
mi propia boca le suplicaba. \bibverse{17} Mi aliento vino á ser extraño
á mi mujer, aunque por los hijos de mis entrañas le rogaba.
\bibverse{18} Aun los muchachos me menospreciaron: en levantándome,
hablaban contra mí. \bibverse{19} Todos mis confidentes me aborrecieron;
y los que yo amaba, se tornaron contra mí. \bibverse{20} Mi cuero y mi
carne se pegaron á mis huesos; y he escapado con la piel de mis dientes.
\bibverse{21} Oh vosotros mis amigos, tened compasión de mí, tened
compasión de mí; porque la mano de Dios me ha tocado. \bibverse{22} ¿Por
qué me perseguís como Dios, y no os hartáis de mis carnes? \bibverse{23}
¡Quién diese ahora que mis palabras fuesen escritas! ¡quién diese que se
escribieran en un libro! \bibverse{24} ¡Que con cincel de hierro y con
plomo fuesen en piedra esculpidas para siempre! \bibverse{25} Yo sé que
mi Redentor vive, y al fin se levantará sobre el polvo: \bibverse{26} Y
después de deshecha esta mi piel, aun he de ver en mi carne á Dios;
\bibverse{27} Al cual yo tengo de ver por mí, y mis ojos lo verán, y no
otro, aunque mis riñones se consuman dentro de mí. \bibverse{28} Mas
debierais decir: ¿Por qué lo perseguimos? ya que la raíz del negocio en
mí se halla. \bibverse{29} Temed vosotros delante de la espada; porque
sobreviene el furor de la espada á causa de las injusticias, para que
sepáis que hay un juicio.

\hypertarget{section-19}{%
\section{20}\label{section-19}}

\bibverse{1} Y respondió Sophar Naamathita, y dijo: \bibverse{2} Por
cierto mis pensamientos me hacen responder, y por tanto me apresuro.
\bibverse{3} La reprensión de mi censura he oído, y háceme responder el
espíritu de mi inteligencia. \bibverse{4} ¿No sabes esto que fué
siempre, desde el tiempo que fué puesto el hombre sobre la tierra,
\bibverse{5} Que la alegría de los impíos es breve, y el gozo del
hipócrita por un momento? \bibverse{6} Si subiere su altivez hasta el
cielo, y su cabeza tocare en las nubes, \bibverse{7} Con su estiércol
perecerá para siempre: los que le hubieren visto, dirán: ¿Qué es de él?
\bibverse{8} Como sueño volará, y no será hallado: y disiparáse como
visión nocturna. \bibverse{9} El ojo que le habrá visto, nunca más le
verá; ni su lugar le echará más de ver. \bibverse{10} Sus hijos pobres
andarán rogando; y sus manos tornarán lo que él robó. \bibverse{11} Sus
huesos están llenos de sus mocedades, y con él serán sepultados en el
polvo. \bibverse{12} Si el mal se endulzó en su boca, si lo ocultaba
debajo de su lengua; \bibverse{13} Si le parecía bien, y no lo dejaba,
mas antes lo detenía entre su paladar; \bibverse{14} Su comida se mudará
en sus entrañas, hiel de áspides será dentro de él. \bibverse{15} Devoró
riquezas, mas vomitarálas; de su vientre las sacará Dios. \bibverse{16}
Veneno de áspides chupará; matarálo lengua de víbora. \bibverse{17} No
verá los arroyos, los ríos, los torrentes de miel y de manteca.
\bibverse{18} Restituirá el trabajo conforme á la hacienda que tomó; y
no tragará, ni gozará. \bibverse{19} Por cuanto quebrantó y desamparó á
los pobres, robó casas, y no las edificó; \bibverse{20} Por tanto, no
sentirá él sosiego en su vientre, ni salvará nada de lo que codiciaba.
\bibverse{21} No quedó nada que no comiese: por tanto su bien no será
durable. \bibverse{22} Cuando fuere lleno su bastimento, tendrá
angustia: las manos todas de los malvados vendrán sobre él.
\bibverse{23} Cuando se pusiere á henchir su vientre, Dios enviará sobre
él el furor de su ira, y harála llover sobre él y sobre su comida.
\bibverse{24} Huirá de las armas de hierro, y el arco de acero le
atravesará. \bibverse{25} Desenvainará y sacará saeta de su aljaba, y
relumbrante pasará por su hiel: sobre él vendrán terrores. \bibverse{26}
Todas tinieblas están guardadas para sus secretos: fuego no soplado lo
devorará; su sucesor será quebrantado en su tienda. \bibverse{27} Los
cielos descubrirán su iniquidad, y la tierra se levantará contra él.
\bibverse{28} Los renuevos de su casa serán trasportados; serán
derramados en el día de su furor. \bibverse{29} Esta es la parte que
Dios apareja al hombre impío, y la heredad que Dios le señala por su
palabra.

\hypertarget{section-20}{%
\section{21}\label{section-20}}

\bibverse{1} Y respondió Job, y dijo: \bibverse{2} Oid atentamente mi
palabra, y sea esto vuestros consuelos. \bibverse{3} Soportadme, y yo
hablaré; y después que hubiere hablado, escarneced. \bibverse{4} ¿Hablo
yo á algún hombre? y ¿por qué no se ha de angustiar mi espíritu?
\bibverse{5} Miradme, y espantaos, y poned la mano sobre la boca.
\bibverse{6} Aun yo mismo, cuando me acuerdo, me asombro, y toma temblor
mi carne. \bibverse{7} ¿Por qué viven los impíos, y se envejecen, y aun
crecen en riquezas? \bibverse{8} Su simiente con ellos, compuesta
delante de ellos; y sus renuevos delante de sus ojos. \bibverse{9} Sus
casas seguras de temor, ni hay azote de Dios sobre ellos. \bibverse{10}
Sus vacas conciben, no abortan; paren sus vacas, y no malogran su cría.
\bibverse{11} Salen sus chiquitos como manada, y sus hijos andan
saltando. \bibverse{12} Al son de tamboril y de cítara saltan, y se
huelgan al son del órgano. \bibverse{13} Gastan sus días en bien, y en
un momento descienden á la sepultura. \bibverse{14} Dicen pues á Dios:
Apártate de nosotros, que no queremos el conocimiento de tus caminos.
\bibverse{15} ¿Quién es el Todopoderoso, para que le sirvamos? ¿y de qué
nos aprovechará que oremos á él? \bibverse{16} He aquí que su bien no
está en manos de ellos: el consejo de los impíos lejos esté de mí.
\bibverse{17} ¡Oh cuántas veces la lámpara de los impíos es apagada, y
viene sobre ellos su quebranto, y Dios en su ira les reparte dolores!
\bibverse{18} Serán como la paja delante del viento, y como el tamo que
arrebata el torbellino. \bibverse{19} Dios guardará para sus hijos su
violencia; y le dará su pago, para que conozca. \bibverse{20} Verán sus
ojos su quebranto, y beberá de la ira del Todopoderoso. \bibverse{21}
Porque ¿qué deleite tendrá él de su casa después de sí, siendo cortado
el número de sus meses? \bibverse{22} ¿Enseñará alguien á Dios
sabiduría, juzgando él á los que están elevados? \bibverse{23} Este
morirá en el vigor de su hermosura, todo quieto y pacífico.
\bibverse{24} Sus colodras están llenas de leche, y sus huesos serán
regados de tuétano. \bibverse{25} Y estotro morirá en amargura de ánimo,
y no habiendo comido jamás con gusto. \bibverse{26} Igualmente yacerán
ellos en el polvo, y gusanos los cubrirán. \bibverse{27} He aquí, yo
conozco vuestros pensamientos, y las imaginaciones que contra mí
forjáis. \bibverse{28} Porque decís: ¿Qué es de la casa del príncipe, y
qué de la tienda de las moradas de los impíos? \bibverse{29} ¿No habéis
preguntado á los que pasan por los caminos, por cuyas señas no negaréis,
\bibverse{30} Que el malo es reservado para el día de la destrucción?
Presentados serán en el día de las iras. \bibverse{31} ¿Quién le
denunciará en su cara su camino? Y de lo que él hizo, ¿quién le dará el
pago? \bibverse{32} Porque llevado será él á los sepulcros, y en el
montón permanecerá. \bibverse{33} Los terrones del valle le serán
dulces; y tras de él será llevado todo hombre, y antes de él han ido
innumerables. \bibverse{34} ¿Cómo pues me consoláis en vano, viniendo á
parar vuestras respuestas en falacia?

\hypertarget{section-21}{%
\section{22}\label{section-21}}

\bibverse{1} Y respondió Eliphaz Temanita, y dijo: \bibverse{2} ¿Traerá
el hombre provecho á Dios, porque el sabio sea provechoso á sí mismo?
\bibverse{3} ¿Tiene su contentamiento el Omnipotente en que tú seas
justificado, ó provecho de que tú hagas perfectos tus caminos?
\bibverse{4} ¿Castigaráte acaso, ó vendrá contigo á juicio porque te
teme? \bibverse{5} Por cierto tu malicia es grande, y tus maldades no
tienen fin. \bibverse{6} Porque sacaste prenda á tus hermanos sin causa,
é hiciste desnudar las ropas de los desnudos. \bibverse{7} No diste de
beber agua al cansado, y detuviste el pan al hambriento. \bibverse{8}
Empero el hombre pudiente tuvo la tierra; y habitó en ella el
distinguido. \bibverse{9} Las viudas enviaste vacías, y los brazos de
los huérfanos fueron quebrados. \bibverse{10} Por tanto hay lazos
alrededor de ti, y te turba espanto repentino; \bibverse{11} O
tinieblas, porque no veas; y abundancia de agua te cubre. \bibverse{12}
¿No está Dios en la altura de los cielos? Mira lo encumbrado de las
estrellas, cuán elevadas están. \bibverse{13} ¿Y dirás tú: Qué sabe
Dios? ¿cómo juzgará por medio de la oscuridad? \bibverse{14} Las nubes
son su escondedero, y no ve; y por el circuito del cielo se pasea.
\bibverse{15} ¿Quieres tú guardar la senda antigua, que pisaron los
hombres perversos? \bibverse{16} Los cuales fueron cortados antes de
tiempo, cuyo fundamento fué como un río derramado: \bibverse{17} Que
decían á Dios: Apártate de nosotros. ¿Y qué les había hecho el
Omnipotente? \bibverse{18} Habíales él henchido sus casas de bienes. Sea
empero el consejo de ellos lejos de mí. \bibverse{19} Verán los justos y
se gozarán; y el inocente los escarnecerá, diciendo: \bibverse{20} Fué
cortada nuestra sustancia, habiendo consumido el fuego el resto de
ellos. \bibverse{21} Amístate ahora con él, y tendrás paz; y por ello te
vendrá bien. \bibverse{22} Toma ahora la ley de su boca, y pon sus
palabras en tu corazón. \bibverse{23} Si te tornares al Omnipotente,
serás edificado; alejarás de tu tienda la aflicción; \bibverse{24} Y
tendrás más oro que tierra, y como piedras de arroyos oro de Ophir;
\bibverse{25} Y el Todopoderoso será tu defensa, y tendrás plata á
montones. \bibverse{26} Porque entonces te deleitarás en el Omnipotente,
y alzarás á Dios tu rostro. \bibverse{27} Orarás á él, y él te oirá; y
tú pagarás tus votos. \bibverse{28} Determinarás asimismo una cosa, y
serte ha firme; y sobre tus caminos resplandecerá luz. \bibverse{29}
Cuando fueren abatidos, dirás tú: Ensalzamiento habrá: y Dios salvará al
humilde de ojos. \bibverse{30} El libertará la isla del inocente; y por
la limpieza de tus manos será librada.

\hypertarget{section-22}{%
\section{23}\label{section-22}}

\bibverse{1} Y respondió Job, y dijo: \bibverse{2} Hoy también hablaré
con amargura; que es más grave mi llaga que mi gemido. \bibverse{3}
¡Quién me diera el saber dónde hallar á Dios! yo iría hasta su silla.
\bibverse{4} Ordenaría juicio delante de él, y henchiría mi boca de
argumentos. \bibverse{5} Yo sabría lo que él me respondería, y
entendería lo que me dijese. \bibverse{6} ¿Pleitearía conmigo con
grandeza de fuerza? No: antes él la pondría en mí. \bibverse{7} Allí el
justo razonaría con él: y escaparía para siempre de mi juez.
\bibverse{8} He aquí yo iré al oriente, y no lo hallaré; y al occidente,
y no lo percibiré: \bibverse{9} Si al norte él obrare, yo no lo veré; al
mediodía se esconderá, y no lo veré. \bibverse{10} Mas él conoció mi
camino: probaráme, y saldré como oro. \bibverse{11} Mis pies tomaron su
rastro; guardé su camino, y no me aparté. \bibverse{12} Del mandamiento
de sus labios nunca me separé; guardé las palabras de su boca más que mi
comida. \bibverse{13} Empero si él se determina en una cosa, ¿quién lo
apartará? Su alma deseó, é hizo. \bibverse{14} El pues acabará lo que ha
determinado de mí: y muchas cosas como estas hay en él. \bibverse{15}
Por lo cual yo me espanto en su presencia: consideraré, y temerélo.
\bibverse{16} Dios ha enervado mi corazón, y hame turbado el
Omnipotente. \bibverse{17} ¿Por qué no fuí yo cortado delante de las
tinieblas, y cubrió con oscuridad mi rostro?

\hypertarget{section-23}{%
\section{24}\label{section-23}}

\bibverse{1} Puesto que no son ocultos los tiempos al Todopoderoso, ¿por
qué los que le conocen no ven sus días? \bibverse{2} Traspasan los
términos, roban los ganados, y apaciéntanlos. \bibverse{3} Llévanse el
asno de los huérfanos; prenden el buey de la viuda. \bibverse{4} Hacen
apartar del camino á los menesterosos: y todos los pobres de la tierra
se esconden. \bibverse{5} He aquí, como asnos monteses en el desierto,
salen á su obra madrugando para robar; el desierto es mantenimiento de
sus hijos. \bibverse{6} En el campo siegan su pasto, y los impíos
vendimian la viña ajena. \bibverse{7} Al desnudo hacen dormir sin ropa,
y que en el frío no tenga cobertura. \bibverse{8} Con las avenidas de
los montes se mojan, y abrazan las peñas sin tener abrigo. \bibverse{9}
Quitan el pecho á los huérfanos, y de sobre el pobre toman la prenda.
\bibverse{10} Al desnudo hacen andar sin vestido, y á los hambrientos
quitan los hacecillos. \bibverse{11} De dentro de sus paredes exprimen
el aceite, pisan los lagares, y mueren de sed. \bibverse{12} De la
ciudad gimen los hombres, y claman las almas de los heridos de muerte:
mas Dios no puso estorbo. \bibverse{13} Ellos son los que, rebeldes á la
luz, nunca conocieron sus caminos, ni estuvieron en sus veredas.
\bibverse{14} A la luz se levanta el matador, mata al pobre y al
necesitado, y de noche es como ladrón. \bibverse{15} El ojo del adúltero
está aguardando la noche, diciendo: No me verá nadie: y esconde su
rostro. \bibverse{16} En las tinieblas minan las casas, que de día para
sí señalaron; no conocen la luz. \bibverse{17} Porque la mañana es á
todos ellos como sombra de muerte; si son conocidos, terrores de sombra
de muerte los toman. \bibverse{18} Son instables más que la superficie
de las aguas; su porción es maldita en la tierra; no andarán por el
camino de las viñas. \bibverse{19} La sequía y el calor arrebatan las
aguas de la nieve; y el sepulcro á los pecadores. \bibverse{20}
Olvidaráse de ellos el seno materno; de ellos sentirán los gusanos
dulzura; nunca más habrá de ellos memoria, y como un árbol serán los
impíos quebrantados. \bibverse{21} A la mujer estéril que no paría,
afligió; y á la viuda nunca hizo bien. \bibverse{22} Mas á los fuertes
adelantó con su poder: levantóse, y no se da por segura la vida.
\bibverse{23} Le dieron á crédito, y se afirmó: sus ojos están sobre los
caminos de ellos. \bibverse{24} Fueron ensalzados por un poco, mas
desaparecen, y son abatidos como cada cual: serán encerrados, y cortados
como cabezas de espigas. \bibverse{25} Y si no, ¿quién me desmentirá
ahora, ó reducirá á nada mis palabras?

\hypertarget{section-24}{%
\section{25}\label{section-24}}

\bibverse{1} Y respondió Bildad Suhita, y dijo: \bibverse{2} El señorío
y el temor están con él: él hace paz en sus alturas. \bibverse{3}
¿Tienen sus ejércitos número? ¿y sobre quién no está su luz?
\bibverse{4} ¿Cómo pues se justificará el hombre con Dios? ¿y cómo será
limpio el que nace de mujer? \bibverse{5} He aquí que ni aun la misma
luna será resplandeciente, ni las estrellas son limpias delante de sus
ojos: \bibverse{6} ¿Cuánto menos el hombre que es un gusano, y el hijo
de hombre, también gusano?

\hypertarget{section-25}{%
\section{26}\label{section-25}}

\bibverse{1} Y respondió Job, y dijo: \bibverse{2} ¿En qué ayudaste al
que no tiene fuerza? ¿has amparado al brazo sin fortaleza? \bibverse{3}
¿En qué aconsejaste al que no tiene ciencia, y mostraste bien sabiduría?
\bibverse{4} ¿A quién has anunciado palabras, y cúyo es el espíritu que
de ti sale? \bibverse{5} Cosas inanimadas son formadas debajo de las
aguas, y los habitantes de ellas. \bibverse{6} El sepulcro es
descubierto delante de él, y el infierno no tiene cobertura.
\bibverse{7} Extiende el aquilón sobre vacío, cuelga la tierra sobre
nada. \bibverse{8} Ata las aguas en sus nubes, y las nubes no se rompen
debajo de ellas. \bibverse{9} El restriñe la faz de su trono, y sobre él
extiende su nube. \bibverse{10} El cercó con término la superficie de
las aguas, hasta el fin de la luz y las tinieblas. \bibverse{11} Las
columnas del cielo tiemblan, y se espantan de su reprensión.
\bibverse{12} El rompe la mar con su poder, y con su entendimiento hiere
la hinchazón suya. \bibverse{13} Su espíritu adornó los cielos; su mano
crió la serpiente tortuosa. \bibverse{14} He aquí, estas son partes de
sus caminos: ¡mas cuán poco hemos oído de él! Porque el estruendo de sus
fortalezas, ¿quién lo detendrá?

\hypertarget{section-26}{%
\section{27}\label{section-26}}

\bibverse{1} Y reasumió Job su discurso, y dijo: \bibverse{2} Vive Dios,
el cual ha apartado mi causa, y el Omnipotente, que amargó el alma mía,
\bibverse{3} Que todo el tiempo que mi alma estuviere en mí, y hubiere
hálito de Dios en mis narices, \bibverse{4} Mis labios no hablarán
iniquidad, ni mi lengua pronunciará engaño. \bibverse{5} Nunca tal
acontezca que yo os justifique: hasta morir no quitaré de mí mi
integridad. \bibverse{6} Mi justicia tengo asida, y no la cederé: no me
reprochará mi corazón en el tiempo de mi vida. \bibverse{7} Sea como el
impío mi enemigo, y como el inicuo mi adversario. \bibverse{8} Porque
¿cuál es la esperanza del hipócrita, por mucho que hubiere robado,
cuando Dios arrebatare su alma? \bibverse{9} ¿Oirá Dios su clamor cuando
la tribulación sobre él viniere? \bibverse{10} ¿Deleitaráse en el
Omnipotente? ¿invocará á Dios en todo tiempo? \bibverse{11} Yo os
enseñaré en orden á la mano de Dios: no esconderé lo que hay para con el
Omnipotente. \bibverse{12} He aquí que todos vosotros lo habéis visto:
¿por qué pues os desvanecéis con fantasía? \bibverse{13} Esta es para
con Dios la suerte del hombre impío, y la herencia que los violentos han
de recibir del Omnipotente. \bibverse{14} Si sus hijos fueren
multiplicados, serán para el cuchillo; y sus pequeños no se hartarán de
pan; \bibverse{15} Los que le quedaren, en muerte serán sepultados; y no
llorarán sus viudas. \bibverse{16} Si amontonare plata como polvo, y si
preparare ropa como lodo; \bibverse{17} Habrála él preparado, mas el
justo se vestirá, y el inocente repartirá la plata. \bibverse{18}
Edificó su casa como la polilla, y cual cabaña que el guarda hizo.
\bibverse{19} El rico dormirá, mas no será recogido: abrirá sus ojos,
mas él no será. \bibverse{20} Asirán de él terrores como aguas:
torbellino lo arrebatará de noche. \bibverse{21} Lo antecogerá el
solano, y partirá; y tempestad lo arrebatará del lugar suyo.
\bibverse{22} Dios pues descargará sobre él, y no perdonará: hará él por
huir de su mano. \bibverse{23} Batirán sus manos sobre él, y desde su
lugar le silbarán.

\hypertarget{section-27}{%
\section{28}\label{section-27}}

\bibverse{1} Ciertamente la plata tiene sus veneros, y el oro lugar
donde se forma. \bibverse{2} El hierro se saca del polvo, y de la piedra
es fundido el metal. \bibverse{3} A las tinieblas puso término, y
examina todo á la perfección, las piedras que hay en la oscuridad y en
la sombra de muerte. \bibverse{4} Brota el torrente de junto al morador,
aguas que el pie había olvidado: sécanse luego, vanse del hombre.
\bibverse{5} De la tierra nace el pan, y debajo de ella estará como
convertida en fuego. \bibverse{6} Lugar hay cuyas piedras son zafiro, y
sus polvos de oro. \bibverse{7} Senda que nunca la conoció ave, ni ojo
de buitre la vió: \bibverse{8} Nunca la pisaron animales fieros, ni león
pasó por ella. \bibverse{9} En el pedernal puso su mano, y trastornó los
montes de raíz. \bibverse{10} De los peñascos cortó ríos, y sus ojos
vieron todo lo preciado. \bibverse{11} Detuvo los ríos en su nacimiento,
é hizo salir á luz lo escondido. \bibverse{12} Empero ¿dónde se hallará
la sabiduría? ¿y dónde está el lugar de la prudencia? \bibverse{13} No
conoce su valor el hombre, ni se halla en la tierra de los vivientes.
\bibverse{14} El abismo dice: No está en mí: y la mar dijo: Ni conmigo.
\bibverse{15} No se dará por oro, ni su precio será á peso de plata.
\bibverse{16} No puede ser apreciada con oro de Ophir, ni con onique
precioso, ni con zafiro. \bibverse{17} El oro no se le igualará, ni el
diamante; ni se trocará por vaso de oro fino. \bibverse{18} De coral ni
de perlas no se hará mención: la sabiduría es mejor que piedras
preciosas. \bibverse{19} No se igualará con ella esmeralda de Ethiopía;
no se podrá apreciar con oro fino. \bibverse{20} ¿De dónde pues vendrá
la sabiduría? ¿y dónde está el lugar de la inteligencia? \bibverse{21}
Porque encubierta está á los ojos de todo viviente, y á toda ave del
cielo es oculta. \bibverse{22} El infierno y la muerte dijeron: Su fama
hemos oído con nuestros oídos. \bibverse{23} Dios entiende el camino de
ella, y él conoce su lugar. \bibverse{24} Porque él mira hasta los fines
de la tierra, y ve debajo de todo el cielo. \bibverse{25} Al dar peso al
viento, y poner las aguas por medida; \bibverse{26} Cuando él hizo ley á
la lluvia, y camino al relámpago de los truenos; \bibverse{27} Entonces
la veía él, y la manifestaba; preparóla y descubrióla también.
\bibverse{28} Y dijo al hombre: He aquí que el temor del Señor es la
sabiduría, y el apartarse del mal la inteligencia.

\hypertarget{section-28}{%
\section{29}\label{section-28}}

\bibverse{1} Y volvió Job á tomar su propósito, y dijo: \bibverse{2}
¡Quién me tornase como en los meses pasados, como en los días que Dios
me guardaba, \bibverse{3} Cuando hacía resplandecer su candela sobre mi
cabeza, á la luz de la cual yo caminaba en la oscuridad; \bibverse{4}
Como fué en los días de mi mocedad, cuando el secreto de Dios estaba en
mi tienda; \bibverse{5} Cuando aun el Omnipotente estaba conmigo, y mis
hijos alrededor de mí; \bibverse{6} Cuando lavaba yo mis caminos con
manteca, y la piedra me derramaba ríos de aceite! \bibverse{7} Cuando
salía á la puerta á juicio, y en la plaza hacía preparar mi asiento,
\bibverse{8} Los mozos me veían, y se escondían; y los viejos se
levantaban, y estaban en pie; \bibverse{9} Los príncipes detenían sus
palabras, ponían la mano sobre su boca; \bibverse{10} La voz de los
principales se ocultaba, y su lengua se pegaba á su paladar:
\bibverse{11} Cuando los oídos que me oían, me llamaban bienaventurado,
y los ojos que me veían, me daban testimonio: \bibverse{12} Porque
libraba al pobre que gritaba, y al huérfano que carecía de ayudador.
\bibverse{13} La bendición del que se iba á perder venía sobre mí; y al
corazón de la viuda daba alegría. \bibverse{14} Vestíame de justicia, y
ella me vestía como un manto; y mi toca era juicio. \bibverse{15} Yo era
ojos al ciego, y pies al cojo. \bibverse{16} A los menesterosos era
padre; y de la causa que no entendía, me informaba con diligencia:
\bibverse{17} Y quebraba los colmillos del inicuo, y de sus dientes
hacía soltar la presa. \bibverse{18} Y decía yo: En mi nido moriré, y
como arena multiplicaré días. \bibverse{19} Mi raíz estaba abierta junto
á las aguas, y en mis ramas permanecía el rocío. \bibverse{20} Mi honra
se renovaba en mí, y mi arco se corroboraba en mi mano. \bibverse{21}
Oíanme, y esperaban; y callaban á mi consejo. \bibverse{22} Tras mi
palabra no replicaban, y mi razón destilaba sobre ellos. \bibverse{23} Y
esperábanme como á la lluvia, y abrían su boca como á la lluvia tardía.
\bibverse{24} Si me reía con ellos, no lo creían: y no abatían la luz de
mi rostro. \bibverse{25} Calificaba yo el camino de ellos, y sentábame
en cabecera; y moraba como rey en el ejército, como el que consuela
llorosos.

\hypertarget{section-29}{%
\section{30}\label{section-29}}

\bibverse{1} Mas ahora los más mozos de días que yo, se ríen de mí;
cuyos padres yo desdeñara ponerlos con los perros de mi ganado.
\bibverse{2} Porque ¿para qué yo habría menester la fuerza de sus manos,
en los cuales había perecido con el tiempo? \bibverse{3} Por causa de la
pobreza y del hambre andaban solos; huían á la soledad, á lugar
tenebroso, asolado y desierto. \bibverse{4} Que cogían malvas entre los
arbustos, y raíces de enebro para calentarse. \bibverse{5} Eran echados
de entre las gentes, y todos les daban grita como al ladrón.
\bibverse{6} Habitaban en las barrancas de los arroyos, en las cavernas
de la tierra, y en las rocas. \bibverse{7} Bramaban entre las matas, y
se reunían debajo de las espinas. \bibverse{8} Hijos de viles, y hombres
sin nombre, más bajos que la misma tierra. \bibverse{9} Y ahora yo soy
su canción, y he sido hecho su refrán. \bibverse{10} Abomínanme,
aléjanse de mí, y aun de mi rostro no detuvieron su saliva.
\bibverse{11} Porque Dios desató mi cuerda, y me afligió, por eso se
desenfrenaron delante de mi rostro. \bibverse{12} A la mano derecha se
levantaron los jóvenes; empujaron mis pies, y sentaron contra mí las
vías de su ruina. \bibverse{13} Mi senda desbarataron, aprovecháronse de
mi quebrantamiento, contra los cuales no hubo ayudador. \bibverse{14}
Vinieron como por portillo ancho, revolviéronse á mi calamidad.
\bibverse{15} Hanse revuelto turbaciones sobre mí; combatieron como
viento mi alma, y mi salud pasó como nube. \bibverse{16} Y ahora mi alma
está derramada en mí; días de aflicción me han aprehendido.
\bibverse{17} De noche taladra sobre mí mis huesos, y mis pulsos no
reposan. \bibverse{18} Con la grande copia de materia mi vestidura está
demudada; cíñeme como el cuello de mi túnica. \bibverse{19} Derribóme en
el lodo, y soy semejante al polvo y á la ceniza. \bibverse{20} Clamo á
ti, y no me oyes; preséntome, y no me atiendes. \bibverse{21} Haste
tornado cruel para mí: con la fortaleza de tu mano me amenazas.
\bibverse{22} Levantásteme, é hicísteme cabalgar sobre el viento, y
disolviste mi sustancia. \bibverse{23} Porque yo conozco que me reduces
á la muerte; y á la casa determinada á todo viviente. \bibverse{24} Mas
él no extenderá la mano contra el sepulcro; ¿clamarán los sepultados
cuando él los quebrantare? \bibverse{25} ¿No lloré yo al afligido? Y mi
alma ¿no se entristeció sobre el menesteroso? \bibverse{26} Cuando
esperaba yo el bien, entonces vino el mal; y cuando esperaba luz, la
oscuridad vino. \bibverse{27} Mis entrañas hierven, y no reposan; días
de aflicción me han sobrecogido. \bibverse{28} Denegrido ando, y no por
el sol: levantádome he en la congregación, y clamado. \bibverse{29} He
venido á ser hermano de los dragones, y compañero de los buhos.
\bibverse{30} Mi piel está denegrida sobre mí, y mis huesos se secaron
con ardentía. \bibverse{31} Y hase tornado mi arpa en luto, y mi órgano
en voz de lamentadores.

\hypertarget{section-30}{%
\section{31}\label{section-30}}

\bibverse{1} Hice pacto con mis ojos: ¿cómo pues había yo de pensar en
virgen? \bibverse{2} Porque ¿qué galardón me daría de arriba Dios, y qué
heredad el Omnipotente de las alturas? \bibverse{3} ¿No hay
quebrantamiento para el impío, y extrañamiento para los que obran
iniquidad? \bibverse{4} ¿No ve él mis caminos, y cuenta todos mis pasos?
\bibverse{5} Si anduve con mentira, y si mi pie se apresuró á engaño,
\bibverse{6} Péseme Dios en balanzas de justicia, y conocerá mi
integridad. \bibverse{7} Si mis pasos se apartaron del camino, y si mi
corazón se fué tras mis ojos, y si algo se apegó á mis manos,
\bibverse{8} Siembre yo, y otro coma, y mis verduras sean arrancadas.
\bibverse{9} Si fué mi corazón engañado acerca de mujer, y si estuve
acechando á la puerta de mi prójimo: \bibverse{10} Muela para otro mi
mujer, y sobre ella otros se encorven. \bibverse{11} Porque es maldad é
iniquidad, que han de castigar los jueces. \bibverse{12} Porque es fuego
que devoraría hasta el sepulcro, y desarraigaría toda mi hacienda.
\bibverse{13} Si hubiera tenido en poco el derecho de mi siervo y de mi
sierva, cuando ellos pleitearan conmigo, \bibverse{14} ¿Qué haría yo
cuando Dios se levantase? y cuando él visitara, ¿qué le respondería yo?
\bibverse{15} El que en el vientre me hizo á mí, ¿no lo hizo á él? ¿y no
nos dispuso uno mismo en la matriz? \bibverse{16} Si estorbé el contento
de los pobres, é hice desfallecer los ojos de la viuda; \bibverse{17} Y
si comí mi bocado solo, y no comió de él el huérfano; \bibverse{18}
(Porque desde mi mocedad creció conmigo como con padre, y desde el
vientre de mi madre fuí guía de la viuda;) \bibverse{19} Si he visto que
pereciera alguno sin vestido, y al menesteroso sin cobertura;
\bibverse{20} Si no me bendijeron sus lomos, y del vellón de mis ovejas
se calentaron; \bibverse{21} Si alcé contra el huérfano mi mano, aunque
viese que me ayudarían en la puerta; \bibverse{22} Mi espalda se caiga
de mi hombro, y mi brazo sea quebrado de mi canilla. \bibverse{23}
Porque temí el castigo de Dios, contra cuya alteza yo no tendría poder.
\bibverse{24} Si puse en oro mi esperanza, y dije al oro: Mi confianza
eres tú; \bibverse{25} Si me alegré de que mi hacienda se multiplicase,
y de que mi mano hallase mucho; \bibverse{26} Si he mirado al sol cuando
resplandecía, y á la luna cuando iba hermosa, \bibverse{27} Y mi corazón
se engañó en secreto, y mi boca besó mi mano: \bibverse{28} Esto también
fuera maldad juzgada; porque habría negado al Dios soberano.
\bibverse{29} Si me alegré en el quebrantamiento del que me aborrecía, y
me regocijé cuando le halló el mal; \bibverse{30} (Que ni aun entregué
al pecado mi paladar, pidiendo maldición para su alma;) \bibverse{31}
Cuando mis domésticos decían: ¡Quién nos diese de su carne! nunca nos
hartaríamos. \bibverse{32} El extranjero no tenía fuera la noche; mis
puertas abría al caminante. \bibverse{33} Si encubrí, como los hombres
mis prevaricaciones, escondiendo en mi seno mi iniquidad; \bibverse{34}
Porque quebrantaba á la gran multitud, y el menosprecio de las familias
me atemorizó, y callé, y no salí de mi puerta: \bibverse{35} ¡Quién me
diera quien me oyese! He aquí mi impresión es que el Omnipotente
testificaría por mí, aunque mi adversario me hiciera el proceso.
\bibverse{36} Ciertamente yo lo llevaría sobre mi hombro, y me lo ataría
en lugar de corona. \bibverse{37} Yo le contaría el número de mis pasos,
y como príncipe me llegaría á él. \bibverse{38} Si mi tierra clama
contra mí, y lloran todos sus surcos; \bibverse{39} Si comí su sustancia
sin dinero, ó afligí el alma de sus dueños; \bibverse{40} En lugar de
trigo me nazcan abrojos, y espinas en lugar de cebada. Acábanse las
palabras de Job.

\hypertarget{section-31}{%
\section{32}\label{section-31}}

\bibverse{1} Y cesaron estos tres varones de responder á Job, por cuanto
él era justo en sus ojos. \bibverse{2} Entonces Eliú hijo de Barachêl,
Bucita,, de la familia de Ram, se enojó con furor contra Job: enojóse
con furor, por cuanto justificaba su vida más que á Dios. \bibverse{3}
Enojóse asimismo con furor contra sus tres amigos, porque no hallaban
qué responder, aunque habían condenado á Job. \bibverse{4} Y Eliú había
esperado á Job en la disputa, porque eran más viejos de días que él.
\bibverse{5} Empero viendo Eliú que no había respuesta en la boca de
aquellos tres varones, su furor se encendió. \bibverse{6} Y respondió
Eliú hijo de Barachêl, Bucita, y dijo: Yo soy menor de días, y vosotros
viejos; he tenido por tanto miedo, y temido declararos mi opinión.
\bibverse{7} Yo decía: Los días hablarán, y la muchedumbre de años
declarará sabiduría. \bibverse{8} Ciertamente espíritu hay en el hombre,
é inspiración del Omnipotente los hace que entiendan. \bibverse{9} No
los grandes son los sabios, ni los viejos entienden el derecho.
\bibverse{10} Por tanto yo dije: Escuchadme; declararé yo también mi
sabiduría. \bibverse{11} He aquí yo he esperado á vuestras razones, he
escuchado vuestros argumentos, en tanto que buscabais palabras.
\bibverse{12} Os he pues prestado atención, y he aquí que no hay de
vosotros quien redarguya á Job, y responda á sus razones. \bibverse{13}
Porque no digáis: Nosotros hemos hallado sabiduría: lanzólo Dios, no el
hombre. \bibverse{14} Ahora bien, Job no enderezó á mí sus palabras, ni
yo le responderé con vuestras razones. \bibverse{15} Espantáronse, no
respondieron más: fuéronseles los razonamientos. \bibverse{16} Yo pues
he esperado, porque no hablaban, antes pararon, y no respondieron más.
\bibverse{17} Por eso yo también responderé mi parte, también yo
declararé mi juicio. \bibverse{18} Porque lleno estoy de palabras, y el
espíritu de mi vientre me constriñe. \bibverse{19} De cierto mi vientre
está como el vino que no tiene respiradero, y se rompe como odres
nuevos. \bibverse{20} Hablaré pues y respiraré; abriré mis labios, y
responderé. \bibverse{21} No haré ahora acepción de personas, ni usaré
con hombre de lisonjeros títulos. \bibverse{22} Porque no sé hablar
lisonjas: de otra manera en breve mi Hacedor me consuma.

\hypertarget{section-32}{%
\section{33}\label{section-32}}

\bibverse{1} Por tanto, Job, oye ahora mis razones, y escucha todas mis
palabras. \bibverse{2} He aquí yo abriré ahora mi boca, y mi lengua
hablará en mi garganta. \bibverse{3} Mis razones declararán la rectitud
de mi corazón, y mis labios proferirán pura sabiduría. \bibverse{4} El
espíritu de Dios me hizo, y la inspiración del Omnipotente me dió vida.
\bibverse{5} Si pudieres, respóndeme; dispón tus palabras, está delante
de mí. \bibverse{6} Heme aquí á mí en lugar de Dios, conforme á tu
dicho: de lodo soy yo también formado. \bibverse{7} He aquí que mi
terror no te espantará, ni mi mano se agravará sobre ti. \bibverse{8} De
cierto tú dijiste á oídos míos, y yo oí la voz de tus palabras que
decían: \bibverse{9} Yo soy limpio y sin defecto; y soy inocente, y no
hay maldad en mí. \bibverse{10} He aquí que él buscó achaques contra mí,
y me tiene por su enemigo; \bibverse{11} Puso mis pies en el cepo, y
guardó todas mis sendas. \bibverse{12} He aquí en esto no has hablado
justamente: yo te responderé que mayor es Dios que el hombre.
\bibverse{13} ¿Por qué tomaste pleito contra él? Porque él no da cuenta
de ninguna de sus razones. \bibverse{14} Sin embargo, en una ó en dos
maneras habla Dios; mas el hombre no entiende. \bibverse{15} Por sueño
de visión nocturna, cuando el sueño cae sobre los hombres, cuando se
adormecen sobre el lecho; \bibverse{16} Entonces revela al oído de los
hombres, y les señala su consejo; \bibverse{17} Para quitar al hombre de
su obra, y apartar del varón la soberbia. \bibverse{18} Detendrá su alma
de corrupción, y su vida de que pase á cuchillo. \bibverse{19} También
sobre su cama es castigado con dolor fuerte en todos sus huesos,
\bibverse{20} Que le hace que su vida aborrezca el pan, y su alma la
comida suave. \bibverse{21} Su carne desfallece sin verse, y sus huesos,
que antes no se veían, aparecen. \bibverse{22} Y su alma se acerca al
sepulcro, y su vida á los que causan la muerte. \bibverse{23} Si tuviera
cerca de él algún elocuente anunciador muy escogido, que anuncie al
hombre su deber; \bibverse{24} Que le diga que Dios tuvo de él
misericordia, que lo libró de descender al sepulcro, que halló
redención: \bibverse{25} Enterneceráse su carne más que de niño, volverá
á los días de su mocedad. \bibverse{26} Orará á Dios, y le amará, y verá
su faz con júbilo: y él restituirá al hombre su justicia. \bibverse{27}
El mira sobre los hombres; y el que dijere: Pequé, y pervertí lo recto,
y no me ha aprovechado; \bibverse{28} Dios redimirá su alma, que no pase
al sepulcro, y su vida se verá en luz. \bibverse{29} He aquí, todas
estas cosas hace Dios dos y tres veces con el hombre; \bibverse{30} Para
apartar su alma del sepulcro, y para iluminarlo con la luz de los
vivientes. \bibverse{31} Escucha, Job, y óyeme; calla, y yo hablaré.
\bibverse{32} Que si tuvieres razones, respóndeme: habla, porque yo te
quiero justificar. \bibverse{33} Y si no, óyeme tú á mí; calla, y
enseñarte he sabiduría.

\hypertarget{section-33}{%
\section{34}\label{section-33}}

\bibverse{1} Además respondió Eliú, y dijo: \bibverse{2} Oid, sabios,
mis palabras; y vosotros, doctos, estadme atentos. \bibverse{3} Porque
el oído prueba las palabras, como el paladar gusta para comer.
\bibverse{4} Escojamos para nosotros el juicio, conozcamos entre
nosotros cuál sea lo bueno: \bibverse{5} Porque Job ha dicho: Yo soy
justo, y Dios me ha quitado mi derecho. \bibverse{6} ¿He de mentir yo
contra mi razón? Mi saeta es gravosa sin haber yo prevaricado.
\bibverse{7} ¿Qué hombre hay como Job, que bebe el escarnio como agua?
\bibverse{8} Y va en compañía con los que obran iniquidad, y anda con
los hombres maliciosos. \bibverse{9} Porque ha dicho: De nada servirá al
hombre el conformar su voluntad con Dios. \bibverse{10} Por tanto,
varones de seso, oidme: Lejos esté de Dios la impiedad, y del
Omnipotente la iniquidad. \bibverse{11} Porque él pagará al hombre según
su obra, y él le hará hallar conforme á su camino. \bibverse{12} Sí, por
cierto, Dios no hará injusticia, y el Omnipotente no pervertirá el
derecho. \bibverse{13} ¿Quién visitó por él la tierra? ¿y quién puso en
orden todo el mundo? \bibverse{14} Si él pusiese sobre el hombre su
corazón, y recogiese así su espíritu y su aliento, \bibverse{15} Toda
carne perecería juntamente, y el hombre se tornaría en polvo.
\bibverse{16} Si pues hay en ti entendimiento, oye esto: escucha la voz
de mis palabras. \bibverse{17} ¿Enseñorearáse el que aborrece juicio? ¿y
condenarás tú al que es tan justo? \bibverse{18} ¿Hase de decir al rey:
Perverso; y á los príncipes: Impíos? \bibverse{19} ¿Cuánto menos á aquel
que no hace acepción de personas de príncipes, ni el rico es de él más
respetado que el pobre? porque todos son obras de sus manos.
\bibverse{20} En un momento morirán, y á media noche se alborotarán los
pueblos, y pasarán, y sin mano será quitado el poderoso. \bibverse{21}
Porque sus ojos están sobre los caminos del hombre, y ve todos sus
pasos. \bibverse{22} No hay tinieblas ni sombra de muerte donde se
encubran los que obran maldad. \bibverse{23} No carga pues él al hombre
más de lo justo, para que vaya con Dios á juicio. \bibverse{24} El
quebrantará á los fuertes sin pesquisa, y hará estar otros en su lugar.
\bibverse{25} Por tanto él hará notorias las obras de ellos, cuando los
trastornará en la noche, y serán quebrantados. \bibverse{26} Como á
malos los herirá en lugar donde sean vistos: \bibverse{27} Por cuanto
así se apartaron de él, y no consideraron todos sus caminos;
\bibverse{28} Haciendo venir delante de él el clamor del pobre, y que
oiga el clamor de los necesitados. \bibverse{29} Y si él diere reposo,
¿quién inquietará? si escondiere el rostro, ¿quién lo mirará? Esto sobre
una nación, y lo mismo sobre un hombre; \bibverse{30} Haciendo que no
reine el hombre hipócrita para vejaciones del pueblo. \bibverse{31} De
seguro conviene se diga á Dios: Llevado he ya castigo, no más ofenderé:
\bibverse{32} Enséñame tú lo que yo no veo: que si hice mal, no lo haré
más. \bibverse{33} ¿Ha de ser eso según tu mente? El te retribuirá, ora
rehuses, ora aceptes, y no yo: di si no, lo que tú sabes. \bibverse{34}
Los hombres de seso dirán conmigo, y el hombre sabio me oirá:
\bibverse{35} Que Job no habla con sabiduría, y que sus palabras no son
con entendimiento. \bibverse{36} Deseo yo que Job sea probado
ampliamente, á causa de sus respuestas por los hombres inicuos.
\bibverse{37} Porque á su pecado añadió impiedad: bate las manos entre
nosotros, y contra Dios multiplica sus palabras.

\hypertarget{section-34}{%
\section{35}\label{section-34}}

\bibverse{1} Y procediendo Eliú en su razonamiento, dijo: \bibverse{2}
¿Piensas ser conforme á derecho esto que dijiste: Más justo soy yo que
Dios? \bibverse{3} Porque dijiste: ¿Qué ventaja sacarás tú de ello? ¿ó
qué provecho tendré de mi pecado? \bibverse{4} Yo te responderé razones,
y á tus compañeros contigo. \bibverse{5} Mira á los cielos, y ve, y
considera que las nubes son más altas que tú. \bibverse{6} Si pecares,
¿qué habrás hecho contra él? y si tus rebeliones se multiplicaren, ¿qué
le harás tú? \bibverse{7} Si fueres justo, ¿qué le darás á él? ¿ó qué
recibirá de tu mano? \bibverse{8} Al hombre como tú dañará tu impiedad,
y al hijo del hombre aprovechará tu justicia. \bibverse{9} A causa de la
multitud de las violencias clamarán, y se lamentarán por el poderío de
los grandes. \bibverse{10} Y ninguno dice: ¿Dónde está Dios mi Hacedor,
que da canciones en la noche, \bibverse{11} Que nos enseña más que á las
bestias de la tierra, y nos hace sabios más que las aves del cielo?
\bibverse{12} Allí clamarán, y él no oirá, por la soberbia de los malos.
\bibverse{13} Ciertamente Dios no oirá la vanidad, ni la mirará el
Omnipotente. \bibverse{14} Aunque más digas, No lo mirará; haz juicio
delante de él, y en él espera. \bibverse{15} Mas ahora, porque en su ira
no visita, ni conoce con rigor, por eso Job abrió su boca vanamente, y
multiplica palabras sin sabiduría. \bibverse{16}

\hypertarget{section-35}{%
\section{36}\label{section-35}}

\bibverse{1} Y añadió Eliú, y dijo: \bibverse{2} Espérame un poco, y
enseñarte he; porque todavía tengo razones en orden á Dios. \bibverse{3}
Tomaré mi noticia de lejos, y atribuiré justicia á mi Hacedor.
\bibverse{4} Porque de cierto no son mentira mis palabras; contigo está
el que es íntegro en sus conceptos. \bibverse{5} He aquí que Dios es
grande, mas no desestima á nadie: es poderoso en fuerza de sabiduría.
\bibverse{6} No otorgará vida al impío, y á los afligidos dará su
derecho. \bibverse{7} No quitará sus ojos del justo; antes bien con los
reyes los pondrá en solio para siempre, y serán ensalzados. \bibverse{8}
Y si estuvieren prendidos en grillos, y aprisionados en las cuerdas de
aflicción, \bibverse{9} El les dará á conocer la obra de ellos, y que
prevalecieron sus rebeliones. \bibverse{10} Despierta además el oído de
ellos para la corrección, y díceles que se conviertan de la iniquidad.
\bibverse{11} Si oyeren, y le sirvieren, acabarán sus días en bien, y
sus años en deleites. \bibverse{12} Mas si no oyeren, serán pasados á
cuchillo, y perecerán sin sabiduría. \bibverse{13} Empero los hipócritas
de corazón lo irritarán más, y no clamarán cuando él los atare.
\bibverse{14} Fallecerá el alma de ellos en su mocedad, y su vida entre
los sodomitas. \bibverse{15} Al pobre librará de su pobreza, y en la
aflicción despertará su oído. \bibverse{16} Asimismo te apartaría de la
boca de la angustia á lugar espacioso, libre de todo apuro; y te
asentará mesa llena de grosura. \bibverse{17} Mas tú has llenado el
juicio del impío, en vez de sustentar el juicio y la justicia.
\bibverse{18} Por lo cual teme que en su ira no te quite con golpe, el
cual no puedas apartar de ti con gran rescate. \bibverse{19} ¿Hará él
estima de tus riquezas, ni del oro, ni de todas las fuerzas del poder?
\bibverse{20} No anheles la noche, en que desaparecen los pueblos de su
lugar. \bibverse{21} Guárdate, no tornes á la iniquidad; pues ésta
escogiste más bien que la aflicción. \bibverse{22} He aquí que Dios es
excelso con su potencia: ¿qué enseñador semejante á él? \bibverse{23}
¿Quién le ha prescrito su camino? ¿y quién le dirá: Iniquidad has hecho?
\bibverse{24} Acuérdate de engrandecer su obra, la cual contemplan los
hombres. \bibverse{25} Los hombres todos la ven; mírala el hombre de
lejos. \bibverse{26} He aquí, Dios es grande, y nosotros no le
conocemos; ni se puede rastrear el número de sus años. \bibverse{27} El
reduce las gotas de las aguas, al derramarse la lluvia según el vapor;
\bibverse{28} Las cuales destilan las nubes, goteando en abundancia
sobre los hombres. \bibverse{29} ¿Quién podrá tampoco comprender la
extensión de las nubes, y el sonido estrepitoso de su pabellón?
\bibverse{30} He aquí que sobre él extiende su luz, y cobija con ella
las raíces de la mar. \bibverse{31} Bien que por esos medios castiga á
los pueblos, á la multitud da comida. \bibverse{32} Con las nubes
encubre la luz, y mándale no brillar, interponiendo aquéllas.
\bibverse{33} Tocante á ella anunciará el trueno, su compañero, que hay
acumulación de ira sobre el que se eleva.

\hypertarget{section-36}{%
\section{37}\label{section-36}}

\bibverse{1} A esto también se espanta mi corazón, y salta de su lugar.
\bibverse{2} Oid atentamente su voz terrible, y el sonido que sale de su
boca. \bibverse{3} Debajo de todos los cielos lo dirige, y su luz hasta
los fines de la tierra. \bibverse{4} Después de ella bramará el sonido,
tronará él con la voz de su magnificencia; y aunque sea oída su voz, no
los detiene. \bibverse{5} Tronará Dios maravillosamente con su voz; él
hace grandes cosas, que nosotros no entendemos. \bibverse{6} Porque á la
nieve dice: Desciende á la tierra; también á la llovizna, y á los
aguaceros de su fortaleza. \bibverse{7} Así hace retirarse á todo
hombre, para que los hombres todos reconozcan su obra. \bibverse{8} La
bestia se entrará en su escondrijo, y estaráse en sus moradas.
\bibverse{9} Del mediodía viene el torbellino, y el frío de los vientos
del norte. \bibverse{10} Por el soplo de Dios se da el hielo, y las
anchas aguas son constreñidas. \bibverse{11} Regando también llega á
disipar la densa nube, y con su luz esparce la niebla. \bibverse{12}
Asimismo por sus designios se revuelven las nubes en derredor, para
hacer sobre la haz del mundo, en la tierra, lo que él les mandara.
\bibverse{13} Unas veces por azote, otras por causa de su tierra, otras
por misericordia las hará parecer. \bibverse{14} Escucha esto, Job;
repósate, y considera las maravillas de Dios. \bibverse{15} ¿Supiste tú
cuándo Dios las ponía en concierto, y hacía levantar la luz de su nube?
\bibverse{16} ¿Has tú conocido las diferencias de las nubes, las
maravillas del Perfecto en sabiduría? \bibverse{17} ¿Por qué están
calientes tus vestidos cuando se fija el viento del mediodía sobre la
tierra? \bibverse{18} ¿Extendiste tú con él los cielos, firmes como un
espejo sólido? \bibverse{19} Muéstranos qué le hemos de decir; porque
nosotros no podemos componer las ideas á causa de las tinieblas.
\bibverse{20} ¿Será preciso contarle cuando yo hablaré? Por más que el
hombre razone, quedará como abismado. \bibverse{21} He aquí aún: no se
puede mirar la luz esplendente en los cielos, luego que pasa el viento y
los limpia, \bibverse{22} Viniendo de la parte del norte la dorada
claridad. En Dios hay una majestad terrible. \bibverse{23} El es
Todopoderoso, al cual no alcanzamos, grande en potencia; y en juicio y
en multitud de justicia no afligirá. \bibverse{24} Temerlo han por tanto
los hombres: él no mira á los sabios de corazón.

\hypertarget{section-37}{%
\section{38}\label{section-37}}

\bibverse{1} Y respondió Jehová á Job desde un torbellino, y dijo:
\bibverse{2} ¿Quién es ése que oscurece el consejo con palabras sin
sabiduría? \bibverse{3} Ahora ciñe como varón tus lomos; yo te
preguntaré, y hazme saber tú. \bibverse{4} ¿Dónde estabas cuando yo
fundaba la tierra? házmelo saber, si tienes inteligencia. \bibverse{5}
¿Quién ordenó sus medidas, si lo sabes? ¿ó quién extendió sobre ella
cordel? \bibverse{6} ¿Sobre qué están fundadas sus basas? ¿ó quién puso
su piedra angular, \bibverse{7} Cuando las estrellas todas del alba
alababan, y se regocijaban todos los hijos de Dios? \bibverse{8} ¿Quién
encerró con puertas la mar, cuando se derramaba por fuera como saliendo
de madre; \bibverse{9} Cuando puse yo nubes por vestidura suya, y por su
faja oscuridad. \bibverse{10} Y establecí sobre ella mi decreto, y le
puse puertas y cerrojo, \bibverse{11} Y dije: Hasta aquí vendrás, y no
pasarás adelante, y ahí parará la hinchazón de tus ondas? \bibverse{12}
¿Has tú mandado á la mañana en tus días? ¿has mostrado al alba su lugar,
\bibverse{13} Para que ocupe los fines de la tierra, y que sean
sacudidos de ella los impíos? \bibverse{14} Trasmúdase como lodo bajo de
sello, y viene á estar como con vestidura: \bibverse{15} Mas la luz de
los impíos es quitada de ellos, y el brazo enaltecido es quebrantado.
\bibverse{16} ¿Has entrado tú hasta los profundos de la mar, y has
andado escudriñando el abismo? \bibverse{17} ¿Hante sido descubiertas
las puertas de la muerte, y has visto las puertas de la sombra de
muerte? \bibverse{18} ¿Has tú considerado hasta las anchuras de la
tierra? Declara si sabes todo esto. \bibverse{19} ¿Por dónde va el
camino á la habitación de la luz, y dónde está el lugar de las
tinieblas? \bibverse{20} ¿Si llevarás tú ambas cosas á sus términos, y
entenderás las sendas de su casa? \bibverse{21} ¿Sabíaslo tú porque
hubieses ya nacido, ó porque es grande el número de tus días?
\bibverse{22} ¿Has tú entrado en los tesoros de la nieve, ó has visto
los tesoros del granizo, \bibverse{23} Lo cual tengo yo reservado para
el tiempo de angustia, para el día de la guerra y de la batalla?
\bibverse{24} ¿Por qué camino se reparte la luz, y se esparce el viento
solano sobre la tierra? \bibverse{25} ¿Quién repartió conducto al
turbión, y camino á los relámpagos y truenos, \bibverse{26} Haciendo
llover sobre la tierra deshabitada, sobre el desierto, donde no hay
hombre, \bibverse{27} Para hartar la tierra desierta é inculta, y para
hacer brotar la tierna hierba? \bibverse{28} ¿Tiene la lluvia padre? ¿ó
quién engendró las gotas del rocío? \bibverse{29} ¿De qué vientre salió
el hielo? y la escarcha del cielo, ¿quién la engendró? \bibverse{30} Las
aguas se endurecen á manera de piedra, y congélase la haz del abismo.
\bibverse{31} ¿Podrás tú impedir las delicias de las Pléyades, ó
desatarás las ligaduras del Orión? \bibverse{32} ¿Sacarás tú á su tiempo
los signos de los cielos, ó guiarás el Arcturo con sus hijos?
\bibverse{33} ¿Supiste tú las ordenanzas de los cielos? ¿dispondrás tú
de su potestad en la tierra? \bibverse{34} ¿Alzarás tú á las nubes tu
voz, para que te cubra muchedumbre de aguas? \bibverse{35} ¿Enviarás tú
los relámpagos, para que ellos vayan? ¿y diránte ellos: Henos aquí?
\bibverse{36} ¿Quién puso la sabiduría en el interior? ¿ó quién dió al
entendimiento la inteligencia? \bibverse{37} ¿Quién puso por cuenta los
cielos con sabiduría? y los odres de los cielos, ¿quién los hace parar,
\bibverse{38} Cuando el polvo se ha convertido en dureza, y los terrones
se han pegado unos con otros? \bibverse{39} \bibverse{40} \bibverse{41}

\hypertarget{section-38}{%
\section{39}\label{section-38}}

\bibverse{1} ¿CAZARÁS tú la presa para el león? ¿y saciarás el hambre de
los leoncillos, \bibverse{2} Cuando están echados en las cuevas, ó se
están en sus guaridas para acechar? \bibverse{3} ¿Quién preparó al
cuervo su alimento, cuando sus pollos claman á Dios, bullendo de un lado
á otro por carecer de comida? \bibverse{4} ¿Sabes tú el tiempo en que
paren las cabras monteses? ¿ó miraste tú las ciervas cuando están
pariendo? \bibverse{5} ¿Contaste tú los meses de su preñez, y sabes el
tiempo cuando han de parir? \bibverse{6} Encórvanse, hacen salir sus
hijos, pasan sus dolores. \bibverse{7} Sus hijos están sanos, crecen con
el pasto: salen y no vuelven á ellas. \bibverse{8} ¿Quién echó libre al
asno montés, y quién soltó sus ataduras? \bibverse{9} Al cual yo puse
casa en la soledad, y sus moradas en lugares estériles. \bibverse{10}
Búrlase de la multitud de la ciudad: no oye las voces del arriero.
\bibverse{11} Lo oculto de los montes es su pasto, y anda buscando todo
lo que está verde. \bibverse{12} ¿Querrá el unicornio servirte á ti, ni
quedar á tu pesebre? \bibverse{13} ¿Atarás tú al unicornio con su
coyunda para el surco? ¿labrará los valles en pos de ti? \bibverse{14}
¿Confiarás tú en él, por ser grande su fortaleza, y le fiarás tu labor?
\bibverse{15} ¿Fiarás de él que te tornará tu simiente, y que la
allegará en tu era? \bibverse{16} ¿Diste tú hermosas alas al pavo real,
ó alas y plumas al avestruz? \bibverse{17} El cual desampara en la
tierra sus huevos, y sobre el polvo los calienta, \bibverse{18} Y
olvídase de que los pisará el pie, y que los quebrará bestia del campo.
\bibverse{19} Endurécese para con sus hijos, como si no fuesen suyos, no
temiendo que su trabajo haya sido en vano: \bibverse{20} Porque le privó
Dios de sabiduría, y no le dió inteligencia. \bibverse{21} Luego que se
levanta en alto, búrlase del caballo y de su jinete. \bibverse{22}
¿Diste tú al caballo la fortaleza? ¿vestiste tú su cerviz de relincho?
\bibverse{23} ¿Le intimidarás tú como á alguna langosta? El resoplido de
su nariz es formidable: \bibverse{24} Escarba la tierra, alégrase en su
fuerza, sale al encuentro de las armas: \bibverse{25} Hace burla del
espanto, y no teme, ni vuelve el rostro delante de la espada.
\bibverse{26} Contra él suena la aljaba, el hierro de la lanza y de la
pica: \bibverse{27} Y él con ímpetu y furor escarba la tierra, sin
importarle el sonido de la bocina; \bibverse{28} Antes como que dice
entre los clarines: ¡Ea!, y desde lejos huele la batalla, el grito de
los capitanes, y la vocería. \bibverse{29} ¿Vuela el gavilán por tu
industria, y extiende hacia el mediodía sus alas? \bibverse{30} ¿Se
remonta el águila por tu mandamiento, y pone en alto su nido? Ella
habita y está en la piedra, en la cumbre del peñasco y de la roca. Desde
allí acecha la comida: sus ojos observan de muy lejos. Sus pollos chupan
la sangre: y donde hubiere cadáveres, allí está. A más de eso respondió
Jehová á Job, y dijo: ¿Es sabiduría contender con el Omnipotente? El que
disputa con Dios, responda á esto. Y respondió Job á Jehová, y dijo: He
aquí que yo soy vil, ¿qué te responderé? Mi mano pongo sobre mi boca.
Una vez hablé, y no responderé: aun dos veces, mas no tornaré á hablar.

\hypertarget{section-39}{%
\section{40}\label{section-39}}

\bibverse{1} Entonces respondió Jehová á Job desde la oscuridad, y dijo:
\bibverse{2} Cíñete ahora como varón tus lomos; yo te preguntaré, y
explícame. \bibverse{3} ¿Invalidarás tú también mi juicio? ¿me
condenarás á mí, para justificarte á ti? \bibverse{4} ¿Tienes tú brazo
como Dios? ¿y tronarás tú con voz como él? \bibverse{5} Atavíate ahora
de majestad y de alteza: y vístete de honra y de hermosura. \bibverse{6}
Esparce furores de tu ira: y mira á todo soberbio, y abátelo.
\bibverse{7} Mira á todo soberbio, y humíllalo, y quebranta á los impíos
en su asiento. \bibverse{8} Encúbrelos á todos en el polvo, venda sus
rostros en la oscuridad; \bibverse{9} Y yo también te confesaré que
podrá salvarte tu diestra. \bibverse{10} He aquí ahora behemoth, al cual
yo hice contigo; hierba come como buey. \bibverse{11} He aquí ahora que
su fuerza está en sus lomos, y su fortaleza en el ombligo de su vientre.
\bibverse{12} Su cola mueve como un cedro, y los nervios de sus
genitales son entretejidos. \bibverse{13} Sus huesos son fuertes como
bronce, y sus miembros como barras de hierro. \bibverse{14} El es la
cabeza de los caminos de Dios: el que lo hizo, puede hacer que su
cuchillo á él se acerque. \bibverse{15} Ciertamente los montes producen
hierba para él: y toda bestia del campo retoza allá. \bibverse{16}
Echaráse debajo de las sombras, en lo oculto de las cañas, y de los
lugares húmedos. \bibverse{17} Los árboles sombríos lo cubren con su
sombra; los sauces del arroyo lo cercan. \bibverse{18} He aquí que él
tomará el río sin inmutarse: y confíase que el Jordán pasará por su
boca. \bibverse{19} ¿Tomarálo alguno por sus ojos en armadijos, y
horadará su nariz? \bibverse{20} \bibverse{21} \bibverse{22}
\bibverse{23} \bibverse{24}

\hypertarget{section-40}{%
\section{41}\label{section-40}}

\bibverse{1} ¿SACARÁS tú al leviathán con el anzuelo, ó con la cuerda
que le echares en su lengua? \bibverse{2} ¿Pondrás tú garfio en sus
narices, y horadarás con espinas su quijada? \bibverse{3} ¿Multiplicará
él ruegos para contigo? ¿hablaráte él lisonjas? \bibverse{4} ¿Hará
concierto contigo para que lo tomes por siervo perpetuo? \bibverse{5}
¿Jugarás tú con él como con pájaro, ó lo atarás para tus niñas?
\bibverse{6} ¿Harán de él banquete los compañeros? ¿partiránlo entre los
mercaderes? \bibverse{7} ¿Cortarás tú con cuchillo su cuero, ó con asta
de pescadores su cabeza? \bibverse{8} Pon tu mano sobre él; te acordarás
de la batalla, y nunca más tornarás. \bibverse{9} He aquí que la
esperanza acerca de él será burlada: porque aun á su sola vista se
desmayarán. \bibverse{10} Nadie hay tan osado que lo despierte: ¿quién
pues podrá estar delante de mí? \bibverse{11} ¿Quién me ha anticipado,
para que yo restituya? Todo lo que hay debajo del cielo es mío.
\bibverse{12} Yo no callaré sus miembros, ni lo de sus fuerzas y la
gracia de su disposición. \bibverse{13} ¿Quién descubrirá la delantera
de su vestidura? ¿quién se llegará á él con freno doble? \bibverse{14}
¿Quién abrirá las puertas de su rostro? Los órdenes de sus dientes
espantan. \bibverse{15} La gloria de su vestido son escudos fuertes,
cerrados entre sí estrechamente. \bibverse{16} El uno se junta con el
otro, que viento no entra entre ellos. \bibverse{17} Pegado está el uno
con el otro, están trabados entre sí, que no se pueden apartar.
\bibverse{18} Con sus estornudos encienden lumbre, y sus ojos son como
los párpados del alba. \bibverse{19} De su boca salen hachas de fuego;
centellas de fuego proceden. \bibverse{20} De sus narices sale humo,
como de una olla ó caldero que hierve. \bibverse{21} Su aliento enciende
los carbones, y de su boca sale llama. \bibverse{22} En su cerviz mora
la fortaleza, y espárcese el desaliento delante de él. \bibverse{23} Las
partes momias de su carne están apretadas: están en él firmes, y no se
mueven. \bibverse{24} Su corazón es firme como una piedra, y fuerte como
la muela de abajo. \bibverse{25} De su grandeza tienen temor los
fuertes, y á causa de su desfallecimiento hacen por purificarse.
\bibverse{26} Cuando alguno lo alcanzare, ni espada, ni lanza, ni dardo,
ni coselete durará. \bibverse{27} El hierro estima por pajas, y el acero
por leño podrido. \bibverse{28} Saeta no le hace huir; las piedras de
honda se le tornan aristas. \bibverse{29} Tiene toda arma por
hojarascas, y del blandir de la pica se burla. \bibverse{30} Por debajo
tiene agudas conchas; Imprime su agudez en el suelo. \bibverse{31} Hace
hervir como una olla la profunda mar, y tórnala como una olla de
ungüento. \bibverse{32} En pos de sí hace resplandecer la senda, que
parece que la mar es cana. \bibverse{33} No hay sobre la tierra su
semejante, hecho para nada temer. \bibverse{34} Menosprecia toda cosa
alta: es rey sobre todos los soberbios.

\hypertarget{section-41}{%
\section{42}\label{section-41}}

\bibverse{1} Y respondió Job á Jehová, y dijo: \bibverse{2} Yo conozco
que todo lo puedes, y que no hay pensamiento que se esconda de ti.
\bibverse{3} ¿Quién es el que oscurece el consejo sin ciencia? por tanto
yo denunciaba lo que no entendía; cosas que me eran ocultas, y que no
las sabía. \bibverse{4} Oye, te ruego, y hablaré: te preguntaré, y tú me
enseñarás. \bibverse{5} De oídas te había oído; mas ahora mis ojos te
ven. \bibverse{6} Por tanto me aborrezco, y me arrepiento en el polvo y
en la ceniza. \bibverse{7} Y aconteció que después que habló Jehová
estas palabras á Job, Jehová dijo á Eliphaz Temanita: Mi ira se encendió
contra ti y tus dos compañeros: porque no habéis hablado por mí lo
recto, como mi siervo Job. \bibverse{8} Ahora pues, tomaos siete
becerros y siete carneros, y andad á mi siervo Job, y ofreced holocausto
por vosotros, y mi siervo Job orará por vosotros; porque de cierto á él
atenderé para no trataros afrentosamente, por cuanto no habéis hablado
por mí con rectitud, como mi siervo Job.

\bibverse{9} Fueron pues Eliphaz Temanita, y Bildad Suhita, y Sophar
Naamatita, é hicieron como Jehová les dijo: y Jehová atendió á Job.

\bibverse{10} Y mudó Jehová la aflicción de Job, orando él por sus
amigos: y aumentó al doble todas las cosas que habían sido de Job.
\bibverse{11} Y vinieron á él todos sus hermanos, y todas sus hermanas,
y todos los que antes le habían conocido, y comieron con él pan en su
casa, y condoliéronse de él, y consoláronle de todo aquel mal que sobre
él había Jehová traído; y cada uno de ellos le dió una pieza de moneda,
y un zarcillo de oro.

\bibverse{12} Y bendijo Jehová la postrimería de Job más que su
principio; porque tuvo catorce mil ovejas, y seis mil camellos, y mil
yuntas de bueyes, y mil asnas. \bibverse{13} Y tuvo siete hijos y tres
hijas. \bibverse{14} Y llamó el nombre de la una, Jemimah, y el nombre
de la segunda, Cesiah, y el nombre de la tercera, Keren-happuch.
\bibverse{15} Y no se hallaron mujeres tan hermosas como las hijas de
Job en toda la tierra: y dióles su padre herencia entre sus hermanos.
\bibverse{16} Y después de esto vivió Job ciento y cuarenta años, y vió
á sus hijos, y á los hijos de sus hijos, hasta la cuarta generación.
\bibverse{17} Murió pues Job viejo, y lleno de días.
