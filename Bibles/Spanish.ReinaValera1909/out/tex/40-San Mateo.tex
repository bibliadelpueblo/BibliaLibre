\hypertarget{uxe1rbol-genealuxf3gico-de-jesuxfas-como-descendiente-de-abraham-y-david}{%
\subsection{Árbol genealógico de Jesús como descendiente de Abraham y
David}\label{uxe1rbol-genealuxf3gico-de-jesuxfas-como-descendiente-de-abraham-y-david}}

\hypertarget{section}{%
\section{1}\label{section}}

\bibverse{1} Libro de la generación de Jesucristo, hijo de David, hijo
de Abraham.

\bibverse{2} Abraham engendró á Isaac: é Isaac engendró á Jacob: y Jacob
engendró á Judas y á sus hermanos: \footnote{\textbf{1:2} Gén 21,3; Gén
  21,12; Gén 25,26; Gén 29,35; Gén 49,10} \bibverse{3} Y Judas engendró
de Thamar á Phares y á Zara: y Phares engendró á Esrom: y Esrom engendró
á Aram: \footnote{\textbf{1:3} Gén 38,29-30; Rut 4,18-22} \bibverse{4} Y
Aram engendró á Aminadab: y Aminadab engendró á Naassón: y Naassón
engendró á Salmón: \bibverse{5} Y Salmón engendró de Rachâb á Booz, y
Booz engendró de Ruth á Obed: y Obed engendró á Jessé: \footnote{\textbf{1:5}
  Jos 2,1; Rut 4,13-17} \bibverse{6} Y Jessé engendró al rey David: y el
rey David engendró á Salomón de la que fué mujer de Urías: \footnote{\textbf{1:6}
  2Sam 12,24} \bibverse{7} Y Salomón engendró á Roboam: y Roboam
engendró á Abía: y Abía engendró á Asa: \footnote{\textbf{1:7} 1Cró
  3,10-16} \bibverse{8} Y Asa engendró á Josaphat: y Josaphat engendró á
Joram: y Joram engendró á Ozías: \bibverse{9} Y Ozías engendró á Joatam:
y Joatam engendró á Achâz: y Achâz engendró á Ezechîas: \bibverse{10} Y
Ezechîas engendró á Manasés: y Manasés engendró á Amón: y Amón engendró
á Josías: \bibverse{11} Y Josías engendró á Jechônías y á sus hermanos,
en la transmigración de Babilonia.

\bibverse{12} Y después de la transmigración de Babilonia, Jechônías
engendró á Salathiel: y Salathiel engendró á Zorobabel: \footnote{\textbf{1:12}
  1Cró 3,17; Esd 3,2} \bibverse{13} Y Zorobabel engendró á Abiud: y
Abiud engendró á Eliachîm: y Eliachîm engendró á Azor: \bibverse{14} Y
Azor engendró á Sadoc: y Sadoc engendró á Achîm: y Achîm engendró á
Eliud: \bibverse{15} Y Eliud engendró á Eleazar: y Eleazar engendró á
Mathán: y Mathán engendró á Jacob: \bibverse{16} Y Jacob engendró á
José, marido de María, de la cual nació Jesús, el cual es llamado el
Cristo.

\bibverse{17} De manera que todas las generaciones desde Abraham hasta
David son catorce generaciones: y desde David hasta la transmigración de
Babilonia, catorce generaciones: y desde la transmigración de Babilonia
hasta Cristo, catorce generaciones.

\hypertarget{nacimiento-y-nombre-de-jesuxfas}{%
\subsection{Nacimiento y nombre de
Jesús}\label{nacimiento-y-nombre-de-jesuxfas}}

\bibverse{18} Y el nacimiento de Jesucristo fué así: Que siendo María su
madre desposada con José, antes que se juntasen, se halló haber
concebido del Espíritu Santo. \footnote{\textbf{1:18} Luc 1,35}
\bibverse{19} Y José su marido, como era justo, y no quisiese infamarla,
quiso dejarla secretamente. \bibverse{20} Y pensando él en esto, he aquí
el ángel del Señor le aparece en sueños, diciendo: José, hijo de David,
no temas de recibir á María tu mujer, porque lo que en ella es
engendrado, del Espíritu Santo es. \bibverse{21} Y parirá un hijo, y
llamarás su nombre JESUS, porque él salvará á su pueblo de sus pecados.

\bibverse{22} Todo esto aconteció para que se cumpliese lo que fué dicho
por el Señor, por el profeta que dijo: \bibverse{23} He aquí la virgen
concebirá y parirá un hijo, y llamarás su nombre Emmanuel, que
declarado, es: Con nosotros Dios.

\bibverse{24} Y despertando José del sueño, hizo como el ángel del Señor
le había mandado, y recibió á su mujer. \bibverse{25} Y no la conoció
hasta que parió á su hijo primogénito: y llamó su nombre JESUS.
\footnote{\textbf{1:25} Luc 2,7}

\hypertarget{los-magos-del-oriente-vienen-al-niuxf1o-jesuxfas-y-le-rinden-homenaje.}{%
\subsection{Los magos del oriente vienen al niño Jesús y le rinden
homenaje.}\label{los-magos-del-oriente-vienen-al-niuxf1o-jesuxfas-y-le-rinden-homenaje.}}

\hypertarget{section-1}{%
\section{2}\label{section-1}}

\bibverse{1} Y como fué nacido Jesús en Bethlehem de Judea en días del
rey Herodes, he aquí unos magos vinieron del oriente á Jerusalem,
\footnote{\textbf{2:1} Luc 2,1-7} \bibverse{2} Diciendo: ¿Dónde está el
Rey de los Judíos, que ha nacido? porque su estrella hemos visto en el
oriente, y venimos á adorarle. \footnote{\textbf{2:2} Núm 24,17}
\bibverse{3} Y oyendo esto el rey Herodes, se turbó, y toda Jerusalem
con él. \bibverse{4} Y convocados todos los príncipes de los sacerdotes,
y los escribas del pueblo, les preguntó dónde había de nacer el Cristo.
\bibverse{5} Y ellos le dijeron: En Bethlehem de Judea; porque así está
escrito por el profeta: \bibverse{6} Y tú, Bethlehem, de tierra de Judá,
no eres muy pequeña entre los príncipes de Judá; porque de ti saldrá un
guiador, que apacentará á mi pueblo Israel.

\bibverse{7} Entonces Herodes, llamando en secreto á los magos, entendió
de ellos diligentemente el tiempo del aparecimiento de la estrella;
\bibverse{8} Y enviándolos á Bethlehem, dijo: Andad allá, y preguntad
con diligencia por el niño; y después que le hallareis, hacédmelo saber,
para que yo también vaya y le adore.

\bibverse{9} Y ellos, habiendo oído al rey, se fueron: y he aquí la
estrella que habían visto en el oriente, iba delante de ellos, hasta que
llegando, se puso sobre donde estaba el niño. \bibverse{10} Y vista la
estrella, se regocijaron con muy grande gozo. \bibverse{11} Y entrando
en la casa, vieron al niño con su madre María, y postrándose, le
adoraron; y abriendo sus tesoros, le ofrecieron dones, oro é incienso y
mirra. \footnote{\textbf{2:11} Sal 72,10; Sal 72,15; Is 60,6}
\bibverse{12} Y siendo avisados por revelación en sueños que no
volviesen á Herodes, se volvieron á su tierra por otro camino.

\hypertarget{la-huida-de-josuxe9-a-egipto}{%
\subsection{La huida de José a
Egipto}\label{la-huida-de-josuxe9-a-egipto}}

\bibverse{13} Y partidos ellos, he aquí el ángel del Señor aparece en
sueños á José, diciendo: Levántate, y toma al niño y á su madre, y huye
á Egipto, y estáte allá hasta que yo te lo diga; porque ha de acontecer,
que Herodes buscará al niño para matarlo.

\bibverse{14} Y él despertando, tomó al niño y á su madre de noche, y se
fué á Egipto; \bibverse{15} Y estuvo allá hasta la muerte de Herodes:
para que se cumpliese lo que fué dicho por el Señor, por el profeta que
dijo: De Egipto llamé á mi Hijo.

\hypertarget{el-asesinato-del-niuxf1o-de-herodes-en-beluxe9n}{%
\subsection{El asesinato del niño de Herodes en
Belén}\label{el-asesinato-del-niuxf1o-de-herodes-en-beluxe9n}}

\bibverse{16} Herodes entonces, como se vió burlado de los magos, se
enojó mucho, y envió, y mató á todos los niños que había en Bethlehem y
en todos sus términos, de edad de dos años abajo, conforme al tiempo que
había entendido de los magos. \bibverse{17} Entonces fué cumplido lo que
se había dicho por el profeta Jeremías, que dijo: \bibverse{18} Voz fué
oída en Ramá, grande lamentación, lloro y gemido: Rachêl que llora sus
hijos; y no quiso ser consolada, porque perecieron.

\hypertarget{el-regreso-de-josuxe9-de-egipto-y-su-asentamiento-en-nazaret}{%
\subsection{El regreso de José de Egipto y su asentamiento en
Nazaret}\label{el-regreso-de-josuxe9-de-egipto-y-su-asentamiento-en-nazaret}}

\bibverse{19} Mas muerto Herodes, he aquí el ángel del Señor aparece en
sueños á José en Egipto, \bibverse{20} Diciendo: Levántate, y toma al
niño y á su madre, y vete á tierra de Israel; que muertos son los que
procuraban la muerte del niño. \footnote{\textbf{2:20} Éxod 4,19}

\bibverse{21} Entonces él se levantó, y tomó al niño y á su madre, y se
vino á tierra de Israel. \bibverse{22} Y oyendo que Archelao reinaba en
Judea en lugar de Herodes su padre, temió ir allá: mas amonestado por
revelación en sueños, se fué á las partes de Galilea. \bibverse{23} Y
vino, y habitó en la ciudad que se llama Nazaret: para que se cumpliese
lo que fué dicho por los profetas, que había de ser llamado Nazareno.

\hypertarget{apariciuxf3n-y-sermuxf3n-penitencial-de-juan-el-bautista}{%
\subsection{Aparición y sermón penitencial de Juan el
Bautista}\label{apariciuxf3n-y-sermuxf3n-penitencial-de-juan-el-bautista}}

\hypertarget{section-2}{%
\section{3}\label{section-2}}

\bibverse{1} Y en aquellos días vino Juan el Bautista predicando en el
desierto de Judea, \footnote{\textbf{3:1} Luc 1,13} \bibverse{2} Y
diciendo: Arrepentíos, que el reino de los cielos se ha acercado.
\footnote{\textbf{3:2} Mat 4,17; Rom 12,2} \bibverse{3} Porque éste es
aquel del cual fué dicho por el profeta Isaías, que dijo: Voz de uno que
clama en el desierto: Aparejad el camino del Señor, enderezad sus
veredas. \footnote{\textbf{3:3} Juan 1,23}

\bibverse{4} Y tenía Juan su vestido de pelos de camellos, y una cinta
de cuero alrededor de sus lomos; y su comida era langostas y miel
silvestre. \footnote{\textbf{3:4} 2Re 1,8} \bibverse{5} Entonces salía á
él Jerusalem, y toda Judea, y toda la provincia de alrededor del Jordán;
\bibverse{6} Y eran bautizados de él en el Jordán, confesando sus
pecados.

\bibverse{7} Y viendo él muchos de los Fariseos y de los Saduceos, que
venían á su bautismo, decíales: Generación de víboras, ¿quién os ha
enseñado á huir de la ira que vendrá? \footnote{\textbf{3:7} Mat 23,33}
\bibverse{8} Haced pues frutos dignos de arrepentimiento, \bibverse{9} Y
no penséis decir dentro de vosotros: A Abraham tenemos por padre: porque
yo os digo, que puede Dios despertar hijos á Abraham aun de estas
piedras. \bibverse{10} Ahora, ya también la segur está puesta á la raíz
de los árboles; y todo árbol que no hace buen fruto, es cortado y echado
en el fuego. \footnote{\textbf{3:10} Luc 13,6-9}

\bibverse{11} Yo á la verdad os bautizo en agua para arrepentimiento;
mas el que viene tras mí, más poderoso es que yo; los zapatos del cual
yo no soy digno de llevar; él os bautizará en Espíritu Santo y en fuego.
\footnote{\textbf{3:11} Juan 1,26-27; Juan 1,33; Hech 1,5; Hech 2,3-4}
\bibverse{12} Su aventador en su mano está, y aventará su era: y
allegará su trigo en el alfolí, y quemará la paja en fuego que nunca se
apagará. \footnote{\textbf{3:12} Mat 13,30}

\hypertarget{el-bautismo-y-la-consagraciuxf3n-del-mesuxedas-de-jesuxfas}{%
\subsection{El bautismo y la consagración del Mesías de
Jesús}\label{el-bautismo-y-la-consagraciuxf3n-del-mesuxedas-de-jesuxfas}}

\bibverse{13} Entonces Jesús vino de Galilea á Juan al Jordán, para ser
bautizado de él. \bibverse{14} Mas Juan lo resistía mucho, diciendo: Yo
he menester ser bautizado de ti, ¿y tú vienes á mí?

\bibverse{15} Empero respondiendo Jesús le dijo: Deja ahora; porque así
nos conviene cumplir toda justicia. Entonces le dejó.

\bibverse{16} Y Jesús, después que fué bautizado, subió luego del agua;
y he aquí los cielos le fueron abiertos, y vió al Espíritu de Dios que
descendía como paloma, y venía sobre él. \footnote{\textbf{3:16} Is 11,2}
\bibverse{17} Y he aquí una voz de los cielos que decía: Este es mi Hijo
amado, en el cual tengo contentamiento. \footnote{\textbf{3:17} Mat
  17,5; Is 42,1}

\hypertarget{la-tentaciuxf3n-de-jesuxfas-como-prueba-de-mesuxedas}{%
\subsection{La tentación de Jesús como prueba de
Mesías}\label{la-tentaciuxf3n-de-jesuxfas-como-prueba-de-mesuxedas}}

\hypertarget{section-3}{%
\section{4}\label{section-3}}

\bibverse{1} Entonces Jesús fué llevado del Espíritu al desierto, para
ser tentado del diablo. \footnote{\textbf{4:1} Heb 4,15} \bibverse{2} Y
habiendo ayunado cuarenta días y cuarenta noches, después tuvo hambre.
\footnote{\textbf{4:2} Éxod 34,28; 1Re 19,8} \bibverse{3} Y llegándose á
él el tentador, dijo: Si eres Hijo de Dios, di que estas piedras se
hagan pan. \footnote{\textbf{4:3} Gén 3,1-7}

\bibverse{4} Mas él respondiendo, dijo: Escrito está: No con solo el pan
vivirá el hombre, mas con toda palabra que sale de la boca de Dios.

\bibverse{5} Entonces el diablo le pasa á la santa ciudad, y le pone
sobre las almenas del templo, \bibverse{6} Y le dice: Si eres Hijo de
Dios, échate abajo; que escrito está: A sus ángeles mandará por ti, y te
alzarán en las manos, para que nunca tropieces con tu pie en piedra.

\bibverse{7} Jesús le dijo: Escrito está además: No tentarás al Señor tu
Dios.

\bibverse{8} Otra vez le pasa el diablo á un monte muy alto, y le
muestra todos los reinos del mundo, y su gloria, \bibverse{9} Y dícele:
Todo esto te daré, si postrado me adorares.

\bibverse{10} Entonces Jesús le dice: Vete, Satanás, que escrito está:
Al Señor tu Dios adorarás y á él solo servirás.

\bibverse{11} El diablo entonces le dejó: y he aquí los ángeles llegaron
y le servían. \footnote{\textbf{4:11} Juan 1,51; Heb 1,6; Heb 1,14}

\hypertarget{jesuxfas-asume-enseuxf1ar-en-capernaum}{%
\subsection{Jesús asume enseñar en
Capernaum}\label{jesuxfas-asume-enseuxf1ar-en-capernaum}}

\bibverse{12} Mas oyendo Jesús que Juan era preso, se volvió á Galilea;
\footnote{\textbf{4:12} Mat 14,3} \bibverse{13} Y dejando á Nazaret,
vino y habitó en Capernaum, ciudad marítima, en los confines de Zabulón
y de Nephtalim: \bibverse{14} Para que se cumpliese lo que fué dicho por
el profeta Isaías, que dijo: \bibverse{15} La tierra de Zabulón, y la
tierra de Nephtalim, camino de la mar, de la otra parte del Jordán,
Galilea de los Gentiles; \bibverse{16} El pueblo asentado en tinieblas,
vió gran luz; y á los sentados en región y sombra de muerte, luz les
esclareció. \footnote{\textbf{4:16} Juan 8,12}

\bibverse{17} Desde entonces comenzó Jesús á predicar, y á decir:
Arrepentíos, que el reino de los cielos se ha acercado. \footnote{\textbf{4:17}
  Mat 3,2}

\hypertarget{jesuxfas-llama-a-los-dos-primeros-pares-de-discuxedpulos}{%
\subsection{Jesús llama a los dos primeros pares de
discípulos}\label{jesuxfas-llama-a-los-dos-primeros-pares-de-discuxedpulos}}

\bibverse{18} Y andando Jesús junto á la mar de Galilea, vió á dos
hermanos, Simón, que es llamado Pedro, y Andrés su hermano, que echaban
la red en la mar; porque eran pescadores. \bibverse{19} Y díceles: Venid
en pos de mí, y os haré pescadores de hombres. \footnote{\textbf{4:19}
  Mat 28,19-20}

\bibverse{20} Ellos entonces, dejando luego las redes, le siguieron.
\footnote{\textbf{4:20} Mat 19,27} \bibverse{21} Y pasando de allí vió
otros dos hermanos, Jacobo, hijo de Zebedeo, y Juan su hermano, en el
barco con Zebedeo, su padre, que remendaban sus redes; y los llamó.
\bibverse{22} Y ellos, dejando luego el barco y á su padre, le
siguieron.

\hypertarget{descripciuxf3n-de-los-efectos-de-enseuxf1anza-y-curaciuxf3n-de-jesuxfas-y-su-uxe9xito}{%
\subsection{Descripción de los efectos de enseñanza y curación de Jesús
y su
éxito}\label{descripciuxf3n-de-los-efectos-de-enseuxf1anza-y-curaciuxf3n-de-jesuxfas-y-su-uxe9xito}}

\bibverse{23} Y rodeó Jesús toda Galilea, enseñando en las sinagogas de
ellos, y predicando el evangelio del reino, y sanando toda enfermedad y
toda dolencia en el pueblo. \footnote{\textbf{4:23} Mar 1,39; Luc 4,44}
\bibverse{24} Y corría su fama por toda la Siria; y le trajeron todos
los que tenían mal: los tomados de diversas enfermedades y tormentos, y
los endemoniados, y lunáticos, y paralíticos, y los sanó. \footnote{\textbf{4:24}
  Mar 6,55} \bibverse{25} Y le siguieron muchas gentes de Galilea y de
Decápolis y de Jerusalem y de Judea y de la otra parte del Jordán.
\footnote{\textbf{4:25} Mar 3,7-8; Luc 6,17-19}

\hypertarget{el-sermuxf3n-del-monte}{%
\subsection{El sermón del monte}\label{el-sermuxf3n-del-monte}}

\hypertarget{section-4}{%
\section{5}\label{section-4}}

\bibverse{1} Y viendo las gentes, subió al monte; y sentándose, se
llegaron á él sus discípulos. \bibverse{2} Y abriendo su boca, les
enseñaba, diciendo:

\hypertarget{las-bienaventuranzas}{%
\subsection{Las Bienaventuranzas}\label{las-bienaventuranzas}}

\bibverse{3} Bienaventurados los pobres en espíritu: porque de ellos es
el reino de los cielos. \bibverse{4} Bienaventurados los que lloran:
porque ellos recibirán consolación. \footnote{\textbf{5:4} Sal 126,5;
  Apoc 7,17} \bibverse{5} Bienaventurados los mansos: porque ellos
recibirán la tierra por heredad. \footnote{\textbf{5:5} Mat 11,29; Sal
  37,11} \bibverse{6} Bienaventurados los que tienen hambre y sed de
justicia: porque ellos serán hartos. \footnote{\textbf{5:6} Luc 18,9-14;
  Juan 6,35} \bibverse{7} Bienaventurados los misericordiosos: porque
ellos alcanzarán misericordia. \footnote{\textbf{5:7} Mat 25,35-46; Sant
  2,13} \bibverse{8} Bienaventurados los de limpio corazón: porque ellos
verán á Dios. \footnote{\textbf{5:8} Sal 24,3-5; Sal 51,12; 1Jn 3,2; 1Jn
  1,3} \bibverse{9} Bienaventurados los pacificadores: porque ellos
serán llamados hijos de Dios. \footnote{\textbf{5:9} Heb 12,14}
\bibverse{10} Bienaventurados los que padecen persecución por causa de
la justicia: porque de ellos es el reino de los cielos. \footnote{\textbf{5:10}
  1Pe 3,14}

\bibverse{11} Bienaventurados sois cuando os vituperaren y os
persiguieren, y dijeren de vosotros todo mal por mi causa, mintiendo.
\footnote{\textbf{5:11} Mat 10,22; Hech 5,41; 1Pe 4,14} \bibverse{12}
Gozaos y alegraos; porque vuestra merced es grande en los cielos: que
así persiguieron á los profetas que fueron antes de vosotros.
\footnote{\textbf{5:12} Sant 5,10; Heb 11,33-38}

\hypertarget{sal-de-la-tierra-luz-del-mundo}{%
\subsection{Sal de la tierra, luz del
mundo}\label{sal-de-la-tierra-luz-del-mundo}}

\bibverse{13} Vosotros sois la sal de la tierra: y si la sal se
desvaneciere ¿con qué será salada? no vale más para nada, sino para ser
echada fuera y hollada de los hombres. \footnote{\textbf{5:13} Mar 9,50;
  Luc 14,34-35}

\bibverse{14} Vosotros sois la luz del mundo: una ciudad asentada sobre
un monte no se puede esconder. \footnote{\textbf{5:14} Juan 8,12}
\bibverse{15} Ni se enciende una lámpara y se pone debajo de un almud,
mas sobre el candelero, y alumbra á todos los que están en casa.
\footnote{\textbf{5:15} Mar 4,21; Luc 8,16} \bibverse{16} Así alumbre
vuestra luz delante de los hombres, para que vean vuestras obras buenas,
y glorifiquen á vuestro Padre que está en los cielos. \footnote{\textbf{5:16}
  Juan 15,8; Efes 5,8-9; Fil 2,14-15}

\hypertarget{perfecciuxf3n-comparada-con-las-exigencias-del-antiguo-pacto}{%
\subsection{Perfección comparada con las exigencias del antiguo
pacto}\label{perfecciuxf3n-comparada-con-las-exigencias-del-antiguo-pacto}}

\bibverse{17} No penséis que he venido para abrogar la ley ó los
profetas: no he venido para abrogar, sino á cumplir. \footnote{\textbf{5:17}
  Mat 3,15; Rom 3,31; 1Jn 2,7} \bibverse{18} Porque de cierto os digo,
que hasta que perezca el cielo y la tierra, ni una jota ni un tilde
perecerá de la ley, hasta que todas las cosas sean hechas. \footnote{\textbf{5:18}
  Luc 16,17} \bibverse{19} De manera que cualquiera que infringiere uno
de estos mandamientos muy pequeños, y así enseñare á los hombres, muy
pequeño será llamado en el reino de los cielos: mas cualquiera que
hiciere y enseñare, éste será llamado grande en el reino de los cielos.
\footnote{\textbf{5:19} Sant 2,10} \bibverse{20} Porque os digo, que si
vuestra justicia no fuere mayor que la de los escribas y de los
Fariseos, no entraréis en el reino de los cielos. \footnote{\textbf{5:20}
  Mat 23,2-33}

\hypertarget{acerca-de-matar-y-juzgar}{%
\subsection{Acerca de matar y juzgar}\label{acerca-de-matar-y-juzgar}}

\bibverse{21} Oísteis que fué dicho á los antiguos: No matarás; mas
cualquiera que matare, será culpado del juicio. \bibverse{22} Mas yo os
digo, que cualquiera que se enojare locamente con su hermano, será
culpado del juicio; y cualquiera que dijere á su hermano, Raca, será
culpado del concejo; y cualquiera que dijere, Fatuo, será culpado del
infierno del fuego.

\bibverse{23} Por tanto, si trajeres tu presente al altar, y allí te
acordares de que tu hermano tiene algo contra ti, \bibverse{24} Deja
allí tu presente delante del altar, y vete, vuelve primero en amistad
con tu hermano, y entonces ven y ofrece tu presente. \footnote{\textbf{5:24}
  Mar 11,25} \bibverse{25} Concíliate con tu adversario presto, entre
tanto que estás con él en el camino; porque no acontezca que el
adversario te entregue al juez, y el juez te entregue al alguacil, y
seas echado en prisión. \footnote{\textbf{5:25} Mat 18,23-35; Luc
  12,58-59} \bibverse{26} De cierto te digo, que no saldrás de allí,
hasta que pagues el último cuadrante.

\hypertarget{sobre-adulteria}{%
\subsection{Sobre adulteria}\label{sobre-adulteria}}

\bibverse{27} Oísteis que fué dicho: No adulterarás: \bibverse{28} Mas
yo os digo, que cualquiera que mira á una mujer para codiciarla, ya
adulteró con ella en su corazón. \footnote{\textbf{5:28} 2Sam 11,2; Job
  31,1; 2Pe 2,14} \bibverse{29} Por tanto, si tu ojo derecho te fuere
ocasión de caer, sácalo, y échalo de ti: que mejor te es que se pierda
uno de tus miembros, que no que todo tu cuerpo sea echado al infierno.
\footnote{\textbf{5:29} Mat 18,8-9; Mar 9,43; Mar 9,47; Col 3,5}
\bibverse{30} Y si tu mano derecha te fuere ocasión de caer, córtala, y
échala de ti: que mejor te es que se pierda uno de tus miembros, que no
que todo tu cuerpo sea echado al infierno.

\bibverse{31} También fué dicho: Cualquiera que repudiare á su mujer,
déle carta de divorcio: \footnote{\textbf{5:31} Mat 19,3-9; Mar 10,4-12}
\bibverse{32} Mas yo os digo, que el que repudiare á su mujer, fuera de
causa de fornicación, hace que ella adultere; y el que se casare con la
repudiada, comete adulterio. \footnote{\textbf{5:32} Luc 16,18; 1Cor
  7,10-11}

\hypertarget{sobre-jurar}{%
\subsection{Sobre jurar}\label{sobre-jurar}}

\bibverse{33} Además habéis oído que fué dicho á los antiguos: No te
perjurarás; mas pagarás al Señor tus juramentos. \bibverse{34} Mas yo os
digo: No juréis en ninguna manera: ni por el cielo, porque es el trono
de Dios; \footnote{\textbf{5:34} Mat 2,16-22; Is 66,1} \bibverse{35} Ni
por la tierra, porque es el estrado de sus pies; ni por Jerusalem,
porque es la ciudad del gran Rey. \footnote{\textbf{5:35} Sal 48,3}
\bibverse{36} Ni por tu cabeza jurarás, porque no puedes hacer un
cabello blanco ó negro. \bibverse{37} Mas sea vuestro hablar: Sí, sí;
No, no; porque lo que es más de esto, de mal procede. \footnote{\textbf{5:37}
  Sant 5,12}

\hypertarget{sobre-lamor-al-projimo-y-al-enemigo}{%
\subsection{Sobre l'amor al projimo y al
enemigo}\label{sobre-lamor-al-projimo-y-al-enemigo}}

\bibverse{38} Oísteis que fué dicho á los antiguos: Ojo por ojo, y
diente por diente. \bibverse{39} Mas yo os digo: No resistáis al mal;
antes á cualquiera que te hiriere en tu mejilla diestra, vuélvele
también la otra; \bibverse{40} Y al que quisiere ponerte á pleito y
tomarte tu ropa, déjale también la capa; \footnote{\textbf{5:40} 1Cor
  6,7; Heb 10,34} \bibverse{41} Y á cualquiera que te cargare por una
milla, ve con él dos. \bibverse{42} Al que te pidiere, dale; y al que
quisiere tomar de ti prestado, no se lo rehuses.

\bibverse{43} Oísteis que fué dicho: Amarás á tu prójimo, y aborrecerás
á tu enemigo. \bibverse{44} Mas yo os digo: Amad á vuestros enemigos,
bendecid á los que os maldicen, haced bien á los que os aborrecen, y
orad por los que os ultrajan y os persiguen; \bibverse{45} Para que
seáis hijos de vuestro Padre que está en los cielos: que hace que su sol
salga sobre malos y buenos, y llueve sobre justos é injustos.
\footnote{\textbf{5:45} Efes 5,1}

\bibverse{46} Porque si amareis á los que os aman, ¿qué recompensa
tendréis? ¿no hacen también lo mismo los publicanos? \bibverse{47} Y si
abrazareis á vuestros hermanos solamente, ¿qué hacéis de más? ¿no hacen
también así los Gentiles? \bibverse{48} Sed, pues, vosotros perfectos,
como vuestro Padre que está en los cielos es perfecto.

\hypertarget{ten-cuidado-al-dar-limosna}{%
\subsection{Ten cuidado al dar
limosna}\label{ten-cuidado-al-dar-limosna}}

\hypertarget{section-5}{%
\section{6}\label{section-5}}

\bibverse{1} Mirad que no hagáis vuestra justicia delante de los
hombres, para ser vistos de ellos: de otra manera no tendréis merced de
vuestro Padre que está en los cielos. \bibverse{2} Cuando pues haces
limosna, no hagas tocar trompeta delante de ti, como hacen los
hipócritas en las sinagogas y en las plazas, para ser estimados de los
hombres: de cierto os digo, que ya tienen su recompensa. \footnote{\textbf{6:2}
  1Cor 13,3} \bibverse{3} Mas cuando tú haces limosna, no sepa tu
izquierda lo que hace tu derecha; \footnote{\textbf{6:3} Mat 25,37-40;
  Rom 12,8} \bibverse{4} Para que sea tu limosna en secreto: y tu Padre
que ve en secreto, él te recompensará en público.

\hypertarget{ten-cuidado-cuando-oras}{%
\subsection{Ten cuidado cuando oras}\label{ten-cuidado-cuando-oras}}

\bibverse{5} Y cuando oras, no seas como los hipócritas; porque ellos
aman el orar en las sinagogas, y en los cantones de las calles en pie,
para ser vistos de los hombres: de cierto os digo, que ya tienen su
pago. \bibverse{6} Mas tú, cuando oras, éntrate en tu cámara, y cerrada
tu puerta, ora á tu Padre que está en secreto; y tu Padre que ve en
secreto, te recompensará en público. \bibverse{7} Y orando, no seáis
prolijos, como los Gentiles; que piensan que por su parlería serán
oídos. \footnote{\textbf{6:7} Is 1,15} \bibverse{8} No os hagáis, pues,
semejantes á ellos; porque vuestro Padre sabe de qué cosas tenéis
necesidad, antes que vosotros le pidáis. \bibverse{9} Vosotros pues,
oraréis así: Padre nuestro que estás en los cielos, santificado sea tu
nombre. \bibverse{10} Venga tu reino. Sea hecha tu voluntad, como en el
cielo, así también en la tierra. \footnote{\textbf{6:10} Luc 22,42}
\bibverse{11} Danos hoy nuestro pan cotidiano. \bibverse{12} Y
perdónanos nuestras deudas, como también nosotros perdonamos á nuestros
deudores. \bibverse{13} Y no nos metas en tentación, mas líbranos del
mal: porque tuyo es el reino, y el poder, y la gloria, por todos los
siglos. Amén. \footnote{\textbf{6:13} 1Cró 29,11-13; Juan 17,15}

\bibverse{14} Porque si perdonareis á los hombres sus ofensas, os
perdonará también á vosotros vuestro Padre celestial. \bibverse{15} Mas
si no perdonareis á los hombres sus ofensas, tampoco vuestro Padre os
perdonará vuestras ofensas.

\hypertarget{ten-cuidado-cuando-ayunas}{%
\subsection{Ten cuidado cuando ayunas}\label{ten-cuidado-cuando-ayunas}}

\bibverse{16} Y cuando ayunáis, no seáis como los hipócritas, austeros;
porque ellos demudan sus rostros para parecer á los hombres que ayunan:
de cierto os digo, que ya tienen su pago. \footnote{\textbf{6:16} Is
  58,5-9} \bibverse{17} Mas tú, cuando ayunas, unge tu cabeza y lava tu
rostro; \bibverse{18} Para no parecer á los hombres que ayunas, sino á
tu Padre que está en secreto: y tu Padre que ve en secreto, te
recompensará en público.

\hypertarget{recoge-tesoros-en-el-cielo}{%
\subsection{Recoge tesoros en el
cielo}\label{recoge-tesoros-en-el-cielo}}

\bibverse{19} No os hagáis tesoros en la tierra, donde la polilla y el
orín corrompe, y donde ladrones minan y hurtan; \bibverse{20} Mas haceos
tesoros en el cielo, donde ni polilla ni orín corrompe, y donde ladrones
no minan ni hurtan: \bibverse{21} Porque donde estuviere vuestro tesoro,
allí estará vuestro corazón.

\bibverse{22} La lámpara del cuerpo es el ojo: así que, si tu ojo fuere
sincero, todo tu cuerpo será luminoso: \footnote{\textbf{6:22} Luc
  11,34-36} \bibverse{23} Mas si tu ojo fuere malo, todo tu cuerpo será
tenebroso. Así que, si la lumbre que en ti hay son tinieblas, ¿cuántas
serán las mismas tinieblas? \footnote{\textbf{6:23} Juan 11,10}

\bibverse{24} Ninguno puede servir á dos señores; porque ó aborrecerá al
uno y amará al otro, ó se llegará al uno y menospreciará al otro: no
podéis servir á Dios y á Mammón. \footnote{\textbf{6:24} Luc 16,9; Luc
  16,13; Sant 4,4}

\hypertarget{busque-el-reino-de-dios-primero}{%
\subsection{Busque el reino de Dios
primero}\label{busque-el-reino-de-dios-primero}}

\bibverse{25} Por tanto os digo: No os congojéis por vuestra vida, qué
habéis de comer, ó qué habéis de beber; ni por vuestro cuerpo, qué
habéis de vestir: ¿no es la vida más que el alimento, y el cuerpo que el
vestido? \footnote{\textbf{6:25} Fil 4,6; 1Pe 5,7; Luc 12,22-31}
\bibverse{26} Mirad las aves del cielo, que no siembran, ni siegan, ni
allegan en alfolíes; y vuestro Padre celestial las alimenta. ¿No sois
vosotros mucho mejores que ellas? \footnote{\textbf{6:26} Mat 10,29-31;
  Luc 12,6-7}

\bibverse{27} Mas ¿quién de vosotros podrá, congojándose, añadir á su
estatura un codo? \bibverse{28} Y por el vestido ¿por qué os congojáis?
Reparad los lirios del campo, cómo crecen; no trabajan ni hilan;
\bibverse{29} Mas os digo, que ni aun Salomón con toda su gloria fué
vestido así como uno de ellos. \bibverse{30} Y si la hierba del campo
que hoy es, y mañana es echada en el horno, Dios la viste así, ¿no hará
mucho más á vosotros, hombres de poca fe?

\bibverse{31} No os congojéis pues, diciendo: ¿Qué comeremos, ó qué
beberemos, ó con qué nos cubriremos? \bibverse{32} Porque los Gentiles
buscan todas estas cosas: que vuestro Padre celestial sabe que de todas
estas cosas habéis menester. \bibverse{33} Mas buscad primeramente el
reino de Dios y su justicia, y todas estas cosas os serán añadidas.
\footnote{\textbf{6:33} Rom 14,17; 1Re 3,13-14; Sal 37,4; Sal 37,25}
\bibverse{34} Así que, no os congojéis por el día de mañana; que el día
de mañana traerá su fatiga: basta al día su afán. \footnote{\textbf{6:34}
  Éxod 16,19}

\hypertarget{no-juzguuxe9is}{%
\subsection{No juzguéis}\label{no-juzguuxe9is}}

\hypertarget{section-6}{%
\section{7}\label{section-6}}

\bibverse{1} No juzguéis, para que no seáis juzgados. \footnote{\textbf{7:1}
  Rom 2,1; 1Cor 4,5} \bibverse{2} Porque con el juicio con que juzgáis,
seréis juzgados; y con la medida con que medís, os volverán á medir.
\footnote{\textbf{7:2} Mar 4,24} \bibverse{3} Y ¿por qué miras la mota
que está en el ojo de tu hermano, y no echas de ver la viga que está en
tu ojo? \bibverse{4} O ¿cómo dirás á tu hermano: Espera, echaré de tu
ojo la mota, y he aquí la viga en tu ojo? \bibverse{5} ¡Hipócrita! echa
primero la viga de tu ojo, y entonces mirarás en echar la mota del ojo
de tu hermano.

\bibverse{6} No deis lo santo á los perros, ni echéis vuestras perlas
delante de los puercos; porque no las rehuellen con sus pies, y vuelvan
y os despedacen. \footnote{\textbf{7:6} Mat 10,11; Luc 23,9}

\hypertarget{pedid-buscad-llaman}{%
\subsection{Pedid, buscad, llaman}\label{pedid-buscad-llaman}}

\bibverse{7} Pedid, y se os dará; buscad, y hallaréis; llamad, y se os
abrirá. \footnote{\textbf{7:7} Jer 29,13-14; Mar 11,24; Luc 11,5-13;
  Juan 14,13} \bibverse{8} Porque cualquiera que pide, recibe; y el que
busca, halla; y al que llama, se abrirá. \bibverse{9} ¿Qué hombre hay de
vosotros, á quien si su hijo pidiere pan, le dará una piedra?
\bibverse{10} ¿Y si le pidiere un pez, le dará una serpiente?
\bibverse{11} Pues si vosotros, siendo malos, sabéis dar buenas dádivas
á vuestros hijos, ¿cuánto más vuestro Padre que está en los cielos, dará
buenas cosas á los que le piden? \footnote{\textbf{7:11} Sant 1,17}

\hypertarget{la-regla-de-oro-de-la-caridad}{%
\subsection{La regla de oro de la
caridad}\label{la-regla-de-oro-de-la-caridad}}

\bibverse{12} Así que, todas las cosas que quisierais que los hombres
hiciesen con vosotros, así también haced vosotros con ellos; porque esta
es la ley y los profetas. \footnote{\textbf{7:12} Mat 22,36-40; Rom
  13,8-10; Gal 5,14}

\bibverse{13} Entrad por la puerta estrecha: porque ancha es la puerta,
y espacioso el camino que lleva á perdición, y muchos son los que entran
por ella. \footnote{\textbf{7:13} Luc 13,24} \bibverse{14} Porque
estrecha es la puerta, y angosto el camino que lleva á la vida, y pocos
son los que la hallan. \footnote{\textbf{7:14} Mat 19,24; Hech 14,22}

\hypertarget{guardaos-de-los-falsos-profetas}{%
\subsection{Guardaos de los falsos
profetas}\label{guardaos-de-los-falsos-profetas}}

\bibverse{15} Y guardaos de los falsos profetas, que vienen á vosotros
con vestidos de ovejas, mas de dentro son lobos rapaces. \footnote{\textbf{7:15}
  Mat 24,4-5; Mat 24,24; 2Cor 11,13-15} \bibverse{16} Por sus frutos los
conoceréis. ¿Cógense uvas de los espinos, ó higos de los abrojos?
\footnote{\textbf{7:16} Gal 5,19-22; Sant 3,12} \bibverse{17} Así, todo
buen árbol lleva buenos frutos; mas el árbol maleado lleva malos frutos.
\footnote{\textbf{7:17} Mat 12,33} \bibverse{18} No puede el buen árbol
llevar malos frutos, ni el árbol maleado llevar frutos buenos.
\bibverse{19} Todo árbol que no lleva buen fruto, córtase y échase en el
fuego. \bibverse{20} Así que, por sus frutos los conoceréis.

\hypertarget{sea-el-hacedor-de-la-palabra-no-solo-un-oyente}{%
\subsection{Sea el hacedor de la palabra, no solo un
oyente}\label{sea-el-hacedor-de-la-palabra-no-solo-un-oyente}}

\bibverse{21} No todo el que me dice: Señor, Señor, entrará en el reino
de los cielos: mas el que hiciere la voluntad de mi Padre que está en
los cielos. \footnote{\textbf{7:21} Rom 2,13; Sant 1,22} \bibverse{22}
Muchos me dirán en aquel día: Señor, Señor, ¿no profetizamos en tu
nombre, y en tu nombre lanzamos demonios, y en tu nombre hicimos muchos
milagros? \footnote{\textbf{7:22} Jer 27,13; Luc 13,25-27} \bibverse{23}
Y entonces les protestaré: Nunca os conocí; apartaos de mí, obradores de
maldad. \footnote{\textbf{7:23} Mat 25,12; 2Tim 2,19}

\bibverse{24} Cualquiera, pues, que me oye estas palabras, y las hace,
le compararé á un hombre prudente, que edificó su casa sobre la peña;
\bibverse{25} Y descendió lluvia, y vinieron ríos, y soplaron vientos, y
combatieron aquella casa; y no cayó: porque estaba fundada sobre la
peña. \bibverse{26} Y cualquiera que me oye estas palabras, y no las
hace, le compararé á un hombre insensato, que edificó su casa sobre la
arena; \bibverse{27} Y descendió lluvia, y vinieron ríos, y soplaron
vientos, é hicieron ímpetu en aquella casa; y cayó, y fué grande su
ruina.

\bibverse{28} Y fué que, como Jesús acabó estas palabras, las gentes se
admiraban de su doctrina; \footnote{\textbf{7:28} Hech 2,12}
\bibverse{29} Porque les enseñaba como quien tiene autoridad, y no como
los escribas. \footnote{\textbf{7:29} Juan 7,16; Juan 7,46}

\hypertarget{sanando-a-un-leproso}{%
\subsection{Sanando a un leproso}\label{sanando-a-un-leproso}}

\hypertarget{section-7}{%
\section{8}\label{section-7}}

\bibverse{1} Y como descendió del monte, le seguían muchas gentes.
\bibverse{2} Y he aquí un leproso vino, y le adoraba, diciendo: Señor,
si quisieres, puedes limpiarme.

\bibverse{3} Y extendiendo Jesús su mano, le tocó, diciendo: Quiero; sé
limpio. Y luego su lepra fué limpiada. \bibverse{4} Entonces Jesús le
dijo: Mira, no lo digas á nadie; mas ve, muéstrate al sacerdote, y
ofrece el presente que mandó Moisés, para testimonio á ellos.
\footnote{\textbf{8:4} Mar 8,30; Lev 14,2-32}

\hypertarget{sanaciuxf3n-del-siervo-del-centuriuxf3n-de-capernaum}{%
\subsection{Sanación del siervo del centurión de
Capernaum}\label{sanaciuxf3n-del-siervo-del-centuriuxf3n-de-capernaum}}

\bibverse{5} Y entrando Jesús en Capernaum, vino á él un centurión,
rogándole, \bibverse{6} Y diciendo: Señor, mi mozo yace en casa
paralítico, gravemente atormentado.

\bibverse{7} Y Jesús le dijo: Yo iré y le sanaré.

\bibverse{8} Y respondió el centurión, y dijo: Señor, no soy digno de
que entres debajo de mi techado; mas solamente di la palabra, y mi mozo
sanará. \bibverse{9} Porque también yo soy hombre bajo de potestad, y
tengo bajo de mí soldados: y digo á éste: Ve, y va; y al otro: Ven, y
viene; y á mi siervo: Haz esto, y lo hace.

\bibverse{10} Y oyendo Jesús, se maravilló, y dijo á los que le seguían:
De cierto os digo, que ni aun en Israel he hallado fe tanta.
\bibverse{11} Y os digo que vendrán muchos del oriente y del occidente,
y se sentarán con Abraham, é Isaac, y Jacob, en el reino de los cielos:
\footnote{\textbf{8:11} Luc 13,28-29} \bibverse{12} Mas los hijos del
reino serán echados á las tinieblas de afuera: allí será el lloro y el
crujir de dientes. \bibverse{13} Entonces Jesús dijo al centurión: Ve, y
como creiste te sea hecho. Y su mozo fué sano en el mismo momento.

\hypertarget{sanaciuxf3n-de-la-suegra-de-pedro-y-de-muchos-otros-enfermos-en-cafarnauxfam}{%
\subsection{Sanación de la suegra de Pedro y de muchos otros enfermos en
Cafarnaúm}\label{sanaciuxf3n-de-la-suegra-de-pedro-y-de-muchos-otros-enfermos-en-cafarnauxfam}}

\bibverse{14} Y vino Jesús á casa de Pedro, y vió á su suegra echada en
cama, y con fiebre. \footnote{\textbf{8:14} 1Cor 9,5} \bibverse{15} Y
tocó su mano, y la fiebre la dejó: y ella se levantó, y les servía.
\bibverse{16} Y como fué ya tarde, trajeron á él muchos endemoniados; y
echó los demonios con la palabra, y sanó á todos los enfermos;
\bibverse{17} Para que se cumpliese lo que fué dicho por el profeta
Isaías, que dijo: El mismo tomó nuestras enfermedades, y llevó nuestras
dolencias.

\hypertarget{jesuxfas-escapa-a-la-otra-orilla-del-lago-proverbios-sobre-seguir-a-jesuxfas}{%
\subsection{Jesús escapa a la otra orilla del lago; Proverbios sobre
seguir a
Jesús}\label{jesuxfas-escapa-a-la-otra-orilla-del-lago-proverbios-sobre-seguir-a-jesuxfas}}

\bibverse{18} Y viendo Jesús muchas gentes alrededor de sí, mandó pasar
á la otra parte del lago.

\bibverse{19} Y llegándose un escriba, le dijo: Maestro, te seguiré á
donde quiera que fueres.

\bibverse{20} Y Jesús le dijo: Las zorras tienen cavernas, y las aves
del cielo nidos; mas el Hijo del hombre no tiene donde recueste su
cabeza.

\bibverse{21} Y otro de sus discípulos le dijo: Señor, dame licencia
para que vaya primero, y entierre á mi padre. \footnote{\textbf{8:21}
  Mat 10,37}

\bibverse{22} Y Jesús le dijo: Sígueme; deja que los muertos entierren á
sus muertos.

\hypertarget{jesuxfas-apacigua-la-tormenta-del-mar}{%
\subsection{Jesús apacigua la tormenta del
mar}\label{jesuxfas-apacigua-la-tormenta-del-mar}}

\bibverse{23} Y entrando él en el barco, sus discípulos le siguieron.
\bibverse{24} Y he aquí, fué hecho en la mar un gran movimiento, que el
barco se cubría de las ondas; mas él dormía. \bibverse{25} Y llegándose
sus discípulos, le despertaron, diciendo: Señor, sálvanos, que
perecemos.

\bibverse{26} Y él les dice: ¿Por qué teméis, hombres de poca fe?
Entonces, levantándose, reprendió á los vientos y á la mar; y fué grande
bonanza.

\bibverse{27} Y los hombres se maravillaron, diciendo: ¿Qué hombre es
éste, que aun los vientos y la mar le obedecen?

\hypertarget{curaciuxf3n-de-dos-poseuxeddos-en-la-tierra-de-los-gadarenos}{%
\subsection{Curación de dos poseídos en la tierra de los
gadarenos}\label{curaciuxf3n-de-dos-poseuxeddos-en-la-tierra-de-los-gadarenos}}

\bibverse{28} Y como él hubo llegado en la otra ribera al país de los
Gergesenos, le vinieron al encuentro dos endemoniados que salían de los
sepulcros, fieros en gran manera, que nadie podía pasar por aquel
camino. \footnote{\textbf{8:28} Luc 4,41; 2Pe 2,4; Sant 2,19}
\bibverse{29} Y he aquí clamaron, diciendo: ¿Qué tenemos contigo, Jesús,
Hijo de Dios? ¿has venido acá á molestarnos antes de tiempo?
\bibverse{30} Y estaba lejos de ellos un hato de muchos puercos
paciendo. \bibverse{31} Y los demonios le rogaron, diciendo: Si nos
echas, permítenos ir á aquel hato de puercos.

\bibverse{32} Y les dijo: Id. Y ellos salieron, y se fueron á aquel hato
de puercos: y he aquí, todo el hato de los puercos se precipitó de un
despeñadero en la mar, y murieron en las aguas.

\bibverse{33} Y los porqueros huyeron, y viniendo á la ciudad, contaron
todas las cosas, y lo que había pasado con los endemoniados.
\bibverse{34} Y he aquí, toda la ciudad salió á encontrar á Jesús: y
cuando le vieron, le rogaban que saliese de sus términos.

\hypertarget{curaciuxf3n-de-un-paraluxedtico-en-capernaum-jesuxfas-perdona-los-pecados}{%
\subsection{Curación de un paralítico en Capernaum; Jesús perdona los
pecados}\label{curaciuxf3n-de-un-paraluxedtico-en-capernaum-jesuxfas-perdona-los-pecados}}

\hypertarget{section-8}{%
\section{9}\label{section-8}}

\bibverse{1} Entonces entrando en el barco, pasó á la otra parte, y vino
á su ciudad. \bibverse{2} Y he aquí le trajeron un paralítico, echado en
una cama: y viendo Jesús la fe de ellos, dijo al paralítico: Confía,
hijo; tus pecados te son perdonados. \footnote{\textbf{9:2} Éxod 34,6-7;
  Sal 103,3}

\bibverse{3} Y he aquí, algunos de los escribas decían dentro de sí:
Este blasfema. \footnote{\textbf{9:3} Mat 26,65}

\bibverse{4} Y viendo Jesús sus pensamientos, dijo: ¿Por qué pensáis mal
en vuestros corazones? \footnote{\textbf{9:4} Juan 2,25} \bibverse{5}
Porque, ¿qué es más fácil, decir: Los pecados te son perdonados; ó
decir: Levántate, y anda? \bibverse{6} Pues para que sepáis que el Hijo
del hombre tiene potestad en la tierra de perdonar pecados, (dice
entonces al paralítico): Levántate, toma tu cama, y vete á tu casa.

\bibverse{7} Entonces él se levantó y se fué á su casa. \bibverse{8} Y
las gentes, viéndolo, se maravillaron, y glorificaron á Dios, que había
dado tal potestad á los hombres.

\hypertarget{llamada-del-recaudador-de-impuestos-mateo-jesuxfas-como-compauxf1ero-de-mesa-para-recaudadores-de-impuestos-y-pecadores}{%
\subsection{Llamada del recaudador de impuestos Mateo; Jesús como
compañero de mesa para recaudadores de impuestos y
pecadores}\label{llamada-del-recaudador-de-impuestos-mateo-jesuxfas-como-compauxf1ero-de-mesa-para-recaudadores-de-impuestos-y-pecadores}}

\bibverse{9} Y pasando Jesús de allí, vió á un hombre que estaba sentado
al banco de los públicos tributos, el cual se llamaba Mateo; y dícele:
Sígueme. Y se levantó, y le siguió. \footnote{\textbf{9:9} Mat 10,3}
\bibverse{10} Y aconteció que estando él sentado á la mesa en casa, he
aquí que muchos publicanos y pecadores, que habían venido, se sentaron
juntamente á la mesa con Jesús y sus discípulos. \bibverse{11} Y viendo
esto los Fariseos, dijeron á sus discípulos: ¿Por qué come vuestro
Maestro con los publicanos y pecadores?

\bibverse{12} Y oyéndolo Jesús, les dijo: Los que están sanos no tienen
necesidad de médico, sino los enfermos. \bibverse{13} Andad pues, y
aprended qué cosa es: Misericordia quiero, y no sacrificio: porque no he
venido á llamar justos, sino pecadores á arrepentimiento. \footnote{\textbf{9:13}
  1Sam 15,22; Mat 18,11}

\hypertarget{la-pregunta-del-ayuno-de-los-discuxedpulos-de-juan}{%
\subsection{La pregunta del ayuno de los discípulos de
Juan}\label{la-pregunta-del-ayuno-de-los-discuxedpulos-de-juan}}

\bibverse{14} Entonces los discípulos de Juan vienen á él, diciendo:
¿Por qué nosotros y los Fariseos ayunamos muchas veces, y tus discípulos
no ayunan? \footnote{\textbf{9:14} Luc 18,12}

\bibverse{15} Y Jesús les dijo: ¿Pueden los que son de bodas tener luto
entre tanto que el esposo está con ellos? mas vendrán días cuando el
esposo será quitado de ellos, y entonces ayunarán. \footnote{\textbf{9:15}
  Juan 3,29} \bibverse{16} Y nadie echa remiendo de paño recio en
vestido viejo; porque el tal remiendo tira del vestido, y se hace peor
la rotura. \footnote{\textbf{9:16} Rom 7,6} \bibverse{17} Ni echan vino
nuevo en cueros viejos: de otra manera los cueros se rompen, y el vino
se derrama, y se pierden los cueros; mas echan el vino nuevo en cueros
nuevos, y lo uno y lo otro se conserva juntamente.

\hypertarget{resucitar-a-la-hija-de-jairo-y-curar-a-la-mujer-asolada-por-la-sangre}{%
\subsection{Resucitar a la hija de Jairo y curar a la mujer asolada por
la
sangre}\label{resucitar-a-la-hija-de-jairo-y-curar-a-la-mujer-asolada-por-la-sangre}}

\bibverse{18} Hablando él estas cosas á ellos, he aquí vino un
principal, y le adoraba, diciendo: Mi hija es muerta poco ha: mas ven y
pon tu mano sobre ella, y vivirá.

\bibverse{19} Y se levantó Jesús, y le siguió, y sus discípulos.
\bibverse{20} Y he aquí una mujer enferma de flujo de sangre doce años
había, llegándose por detrás, tocó la franja de su vestido:
\bibverse{21} Porque decía entre sí: Si tocare solamente su vestido,
seré salva. \footnote{\textbf{9:21} Mat 14,36}

\bibverse{22} Mas Jesús volviéndose, y mirándola, dijo: Confía, hija, tu
fe te ha salvado. Y la mujer fué salva desde aquella hora.

\bibverse{23} Y llegado Jesús á casa del principal, viendo los tañedores
de flautas, y la gente que hacía bullicio, \bibverse{24} Díceles:
Apartaos, que la muchacha no es muerta, mas duerme. Y se burlaban de él.

\bibverse{25} Y como la gente fué echada fuera, entró, y tomóla de la
mano, y se levantó la muchacha. \bibverse{26} Y salió esta fama por toda
aquella tierra.

\hypertarget{curaciuxf3n-de-dos-ciegos-y-un-mudo-endemoniado-graduaciuxf3n}{%
\subsection{Curación de dos ciegos y un mudo endemoniado;
Graduación}\label{curaciuxf3n-de-dos-ciegos-y-un-mudo-endemoniado-graduaciuxf3n}}

\bibverse{27} Y pasando Jesús de allí, le siguieron dos ciegos, dando
voces y diciendo: Ten misericordia de nosotros, Hijo de David.
\footnote{\textbf{9:27} Mat 20,29-34} \bibverse{28} Y llegado á la casa,
vinieron á él los ciegos; y Jesús les dice: ¿Creéis que puedo hacer
esto? Ellos dicen: Sí, Señor. \footnote{\textbf{9:28} Hech 14,9}

\bibverse{29} Entonces tocó los ojos de ellos, diciendo: Conforme á
vuestra fe os sea hecho. \footnote{\textbf{9:29} Mat 8,13} \bibverse{30}
Y los ojos de ellos fueron abiertos. Y Jesús les encargó rigurosamente,
diciendo: Mirad que nadie lo sepa. \footnote{\textbf{9:30} Mat 8,4}
\bibverse{31} Mas ellos salidos, divulgaron su fama por toda aquella
tierra.

\bibverse{32} Y saliendo ellos, he aquí, le trajeron un hombre mudo,
endemoniado. \bibverse{33} Y echado fuera el demonio, el mudo habló; y
las gentes se maravillaron, diciendo: Nunca ha sido vista cosa semejante
en Israel.

\bibverse{34} Mas los Fariseos decían: Por el príncipe de los demonios
echa fuera los demonios. \footnote{\textbf{9:34} Mat 12,24-32}

\bibverse{35} Y rodeaba Jesús por todas las ciudades y aldeas, enseñando
en las sinagogas de ellos, y predicando el evangelio del reino, y
sanando toda enfermedad y todo achaque en el pueblo.

\hypertarget{la-compasiuxf3n-de-jesuxfas-a-la-vista-de-la-gente-la-palabra-de-la-cosecha}{%
\subsection{La compasión de Jesús a la vista de la gente; la palabra de
la
cosecha}\label{la-compasiuxf3n-de-jesuxfas-a-la-vista-de-la-gente-la-palabra-de-la-cosecha}}

\bibverse{36} Y viendo las gentes, tuvo compasión de ellas; porque
estaban derramadas y esparcidas como ovejas que no tienen pastor.
\bibverse{37} Entonces dice á sus discípulos: A la verdad la mies es
mucha, mas los obreros pocos. \footnote{\textbf{9:37} Luc 10,2}

\bibverse{38} Rogad, pues, al Señor de la mies, que envíe obreros á su
mies.

\hypertarget{llamadas-y-nombres-de-los-doce-discuxedpulos}{%
\subsection{Llamadas y nombres de los doce
discípulos}\label{llamadas-y-nombres-de-los-doce-discuxedpulos}}

\hypertarget{section-9}{%
\section{10}\label{section-9}}

\bibverse{1} Entonces llamando á sus doce discípulos, les dió potestad
contra los espíritus inmundos, para que los echasen fuera, y sanasen
toda enfermedad y toda dolencia. \bibverse{2} Y los nombres de los doce
apóstoles son estos: el primero, Simón, que es dicho Pedro, y Andrés su
hermano; Jacobo, hijo de Zebedeo, y Juan su hermano; \bibverse{3}
Felipe, y Bartolomé; Tomás, y Mateo el publicano; Jacobo hijo de Alfeo,
y Lebeo, por sobrenombre Tadeo; \bibverse{4} Simón el Cananita y Judas
Iscariote, que también le entregó.

\hypertarget{el-mensaje-enviado-a-los-doce-discuxedpulos}{%
\subsection{El mensaje enviado a los doce
discípulos}\label{el-mensaje-enviado-a-los-doce-discuxedpulos}}

\bibverse{5} A estos doce envió Jesús, á los cuales dió mandamiento,
diciendo: Por el camino de los Gentiles no iréis, y en ciudad de
Samaritanos no entréis; \bibverse{6} Mas id antes á las ovejas perdidas
de la casa de Israel. \footnote{\textbf{10:6} Mat 15,24; Hech 13,46}
\bibverse{7} Y yendo, predicad, diciendo: El reino de los cielos se ha
acercado. \footnote{\textbf{10:7} Mat 4,17; Luc 10,9} \bibverse{8} Sanad
enfermos, limpiad leprosos, resucitad muertos, echad fuera demonios: de
gracia recibisteis, dad de gracia. \footnote{\textbf{10:8} Mar 16,17;
  Hech 20,33} \bibverse{9} No aprestéis oro, ni plata, ni cobre en
vuestras bolsas; \bibverse{10} Ni alforja para el camino, ni dos ropas
de vestir, ni zapatos, ni bordón; porque el obrero digno es de su
alimento. \bibverse{11} Mas en cualquier ciudad, ó aldea donde
entrareis, investigad quién sea en ella digno, y reposad allí hasta que
salgáis. \bibverse{12} Y entrando en la casa, saludadla. \footnote{\textbf{10:12}
  Luc 10,5-6} \bibverse{13} Y si la casa fuere digna, vuestra paz vendrá
sobre ella; mas si no fuere digna, vuestra paz se volverá á vosotros.
\bibverse{14} Y cualquiera que no os recibiere, ni oyere vuestras
palabras, salid de aquella casa ó ciudad, y sacudid el polvo de vuestros
pies. \bibverse{15} De cierto os digo, que el castigo será más tolerable
á la tierra de los de Sodoma y de los de Gomorra en el día del juicio,
que á aquella ciudad. \footnote{\textbf{10:15} Gén 19,1-29}

\bibverse{16} He aquí, yo os envío como á ovejas en medio de lobos: sed
pues prudentes como serpientes, y sencillos como palomas. \footnote{\textbf{10:16}
  Rom 16,19; Efes 5,15}

\hypertarget{anuncio-de-los-sufrimientos-que-vendruxe1n-a-los-discuxedpulos}{%
\subsection{Anuncio de los sufrimientos que vendrán a los
discípulos}\label{anuncio-de-los-sufrimientos-que-vendruxe1n-a-los-discuxedpulos}}

\bibverse{17} Y guardaos de los hombres: porque os entregarán en
concilios, y en sus sinagogas os azotarán; \footnote{\textbf{10:17} Hech
  5,40; 2Cor 11,24} \bibverse{18} Y aun á príncipes y á reyes seréis
llevados por causa de mí, por testimonio á ellos y á los Gentiles.
\footnote{\textbf{10:18} Hech 25,23; Hech 27,24} \bibverse{19} Mas
cuando os entregaren, no os apuréis por cómo ó qué hablaréis; porque en
aquella hora os será dado qué habéis de hablar. \footnote{\textbf{10:19}
  Luc 12,11-12; Hech 4,8} \bibverse{20} Porque no sois vosotros los que
habláis, sino el Espíritu de vuestro Padre que habla en vosotros.
\footnote{\textbf{10:20} 1Cor 2,4}

\bibverse{21} Y el hermano entregará al hermano á la muerte, y el padre
al hijo; y los hijos se levantarán contra los padres, y los harán morir.
\footnote{\textbf{10:21} Miq 7,6} \bibverse{22} Y seréis aborrecidos de
todos por mi nombre; mas el que soportare hasta el fin, éste será salvo.
\footnote{\textbf{10:22} Mat 24,9-13; 2Tim 2,12} \bibverse{23} Mas
cuando os persiguieren en esta ciudad, huid á la otra: porque de cierto
os digo, que no acabaréis de andar todas las ciudades de Israel, que no
venga el Hijo del hombre. \footnote{\textbf{10:23} Mat 16,28; Hech 8,1}

\bibverse{24} El discípulo no es más que su maestro, ni el siervo más
que su señor. \footnote{\textbf{10:24} Luc 6,40; Juan 13,16; Juan 15,20}
\bibverse{25} Bástale al discípulo ser como su maestro, y al siervo como
su señor. Si al padre de la familia llamaron Beelzebub, ¿cuánto más á
los de su casa? \footnote{\textbf{10:25} Mat 12,24}

\hypertarget{aliento-para-perseverar-fielmente-y-consuelo-en-tiempos-de-tribulaciuxf3n}{%
\subsection{Aliento para perseverar fielmente y consuelo en tiempos de
tribulación}\label{aliento-para-perseverar-fielmente-y-consuelo-en-tiempos-de-tribulaciuxf3n}}

\bibverse{26} Así que, no los temáis; porque nada hay encubierto, que no
haya de ser manifestado; ni oculto, que no haya de saberse. \footnote{\textbf{10:26}
  Mar 4,22; Luc 8,17} \bibverse{27} Lo que os digo en tinieblas, decidlo
en la luz; y lo que oís al oído predicadlo desde los terrados.
\bibverse{28} Y no temáis á los que matan el cuerpo, mas al alma no
pueden matar: temed antes á aquel que puede destruir el alma y el cuerpo
en el infierno. \footnote{\textbf{10:28} Heb 10,31; Sant 4,12}

\bibverse{29} ¿No se venden dos pajarillos por un cuarto? Con todo, ni
uno de ellos cae á tierra sin vuestro Padre. \bibverse{30} Pues aun
vuestros cabellos están todos contados. \bibverse{31} Así que, no
temáis: más valéis vosotros que muchos pajarillos. \footnote{\textbf{10:31}
  Mat 6,26} \bibverse{32} Cualquiera pues, que me confesare delante de
los hombres, le confesaré yo también delante de mi Padre que está en los
cielos. \footnote{\textbf{10:32} Apoc 3,5} \bibverse{33} Y cualquiera
que me negare delante de los hombres, le negaré yo también delante de mi
Padre que está en los cielos. \footnote{\textbf{10:33} Mar 8,38; Luc
  9,26; 2Tim 2,12}

\hypertarget{paz-y-espada-puxe9rdida-y-ganancia}{%
\subsection{Paz y espada, pérdida y
ganancia}\label{paz-y-espada-puxe9rdida-y-ganancia}}

\bibverse{34} No penséis que he venido para meter paz en la tierra: no
he venido para meter paz, sino espada. \footnote{\textbf{10:34} Luc
  12,51-53} \bibverse{35} Porque he venido para hacer disensión del
hombre contra su padre, y de la hija contra su madre, y de la nuera
contra su suegra. \bibverse{36} Y los enemigos del hombre serán los de
su casa. \footnote{\textbf{10:36} Miq 7,6} \bibverse{37} El que ama
padre ó madre más que á mí, no es digno de mí; y el que ama hijo ó hija
más que á mí, no es digno de mí. \footnote{\textbf{10:37} Deut 13,7-12;
  Deut 33,9; Luc 14,26-27} \bibverse{38} Y el que no toma su cruz, y
sigue en pos de mí, no es digno de mí. \footnote{\textbf{10:38} Mat
  16,24-25} \bibverse{39} El que hallare su vida, la perderá; y el que
perdiere su vida por causa de mí, la hallará. \footnote{\textbf{10:39}
  Luc 9,24; Juan 12,25}

\hypertarget{promesas-de-ayuda-fraternal}{%
\subsection{Promesas de ayuda
fraternal}\label{promesas-de-ayuda-fraternal}}

\bibverse{40} El que os recibe á vosotros, á mí recibe; y el que á mí
recibe, recibe al que me envió. \footnote{\textbf{10:40} Luc 9,48; Juan
  13,20; Gal 4,14} \bibverse{41} El que recibe profeta en nombre de
profeta, merced de profeta recibirá; y el que recibe justo en nombre de
justo, merced de justo recibirá. \footnote{\textbf{10:41} 1Re 17,9-24}
\bibverse{42} Y cualquiera que diere á uno de estos pequeñitos un vaso
de agua fría solamente, en nombre de discípulo, de cierto os digo, que
no perderá su recompensa. \footnote{\textbf{10:42} Mat 25,40; Mar 9,41}

\hypertarget{section-10}{%
\section{11}\label{section-10}}

\bibverse{1} Y fué, que acabando Jesús de dar mandamientos á sus doce
discípulos, se fué de allí á enseñar y á predicar en las ciudades de
ellos.

\hypertarget{embajada-de-juan-el-bautista-desde-la-prisiuxf3n-la-respuesta-y-el-testimonio-de-jesuxfas-sobre-juan}{%
\subsection{Embajada de Juan el Bautista desde la prisión; La respuesta
y el testimonio de Jesús sobre
Juan}\label{embajada-de-juan-el-bautista-desde-la-prisiuxf3n-la-respuesta-y-el-testimonio-de-jesuxfas-sobre-juan}}

\bibverse{2} Y oyendo Juan en la prisión los hechos de Cristo, le envió
dos de sus discípulos, \bibverse{3} Diciendo: ¿Eres tú aquél que había
de venir, ó esperaremos á otro? \footnote{\textbf{11:3} Mat 3,11; Mal
  3,1}

\bibverse{4} Y respondiendo Jesús, les dijo: Id, y haced saber á Juan
las cosas que oís y veis: \bibverse{5} Los ciegos ven, y los cojos
andan; los leprosos son limpiados, y los sordos oyen; los muertos son
resucitados, y á los pobres es anunciado el evangelio. \bibverse{6} Y
bienaventurado es el que no fuere escandalizado en mí. \footnote{\textbf{11:6}
  Mat 13,57; Mat 26,31}

\bibverse{7} E idos ellos, comenzó Jesús á decir de Juan á las gentes:
¿Qué salisteis á ver al desierto? ¿una caña que es meneada del viento?
\footnote{\textbf{11:7} Mat 3,1; Mat 3,5} \bibverse{8} Mas ¿qué
salisteis á ver? ¿un hombre cubierto de delicados vestidos? He aquí, los
que traen vestidos delicados, en las casas de los reyes están.
\bibverse{9} Mas ¿qué salisteis á ver? ¿un profeta? También os digo, y
más que profeta. \footnote{\textbf{11:9} Luc 1,76; Luc 20,6}
\bibverse{10} Porque éste es de quien está escrito: He aquí, yo envío mi
mensajero delante de tu faz, que aparejará tu camino delante de ti.
\bibverse{11} De cierto os digo, que no se levantó entre los que nacen
de mujeres otro mayor que Juan el Bautista; mas el que es muy más
pequeño en el reino de los cielos, mayor es que él. \bibverse{12} Desde
los días de Juan el Bautista hasta ahora, al reino de los cielos se hace
fuerza, y los valientes lo arrebatan. \footnote{\textbf{11:12} Luc 16,16}
\bibverse{13} Porque todos los profetas y la ley hasta Juan
profetizaron. \bibverse{14} Y si queréis recibir, él es aquel Elías que
había de venir. \bibverse{15} El que tiene oídos para oir, oiga.

\bibverse{16} Mas ¿á quién compararé esta generación? Es semejante á los
muchachos que se sientan en las plazas, y dan voces á sus compañeros,
\bibverse{17} Y dicen: Os tañimos flauta, y no bailasteis; os
endechamos, y no lamentasteis. \footnote{\textbf{11:17} Prov 29,9; Juan
  5,35} \bibverse{18} Porque vino Juan, que ni comía ni bebía, y dicen:
Demonio tiene. \footnote{\textbf{11:18} Mat 3,4; Juan 10,20}
\bibverse{19} Vino el Hijo del hombre, que come y bebe, y dicen: He aquí
un hombre comilón, y bebedor de vino, amigo de publicanos y de
pecadores. Mas la sabiduría es justificada por sus hijos. \footnote{\textbf{11:19}
  Mat 9,10-15; Juan 2,2; 1Cor 1,24-30}

\hypertarget{lamento-de-jesuxfas-por-las-ciudades-galileas-impenitentes}{%
\subsection{Lamento de Jesús por las ciudades galileas
impenitentes}\label{lamento-de-jesuxfas-por-las-ciudades-galileas-impenitentes}}

\bibverse{20} Entonces comenzó á reconvenir á las ciudades en las cuales
habían sido hechas muy muchas de sus maravillas, porque no se habían
arrepentido, diciendo: \bibverse{21} ¡Ay de ti, Corazín! ¡Ay de ti,
Bethsaida! porque si en Tiro y en Sidón fueran hechas las maravillas que
han sido hechas en vosotras, en otro tiempo se hubieran arrepentido en
saco y en ceniza. \bibverse{22} Por tanto os digo, que á Tiro y á Sidón
será más tolerable el castigo en el día del juicio, que á vosotras.
\bibverse{23} Y tú, Capernaum, que eres levantada hasta el cielo, hasta
los infiernos serás abajada; porque si en los de Sodoma fueran hechas
las maravillas que han sido hechas en ti, hubieran quedado hasta el día
de hoy. \footnote{\textbf{11:23} Mat 4,13; Mat 8,5; Mat 9,1; Is 14,13;
  Is 14,15} \bibverse{24} Por tanto os digo, que á la tierra de los de
Sodoma será más tolerable el castigo en el día del juicio, que á ti.
\footnote{\textbf{11:24} Mat 10,15}

\hypertarget{el-juxfabilo-y-la-alabanza-de-jesuxfas-por-el-padre}{%
\subsection{El júbilo y la alabanza de Jesús por el
Padre}\label{el-juxfabilo-y-la-alabanza-de-jesuxfas-por-el-padre}}

\bibverse{25} En aquel tiempo, respondiendo Jesús, dijo: Te alabo,
Padre, Señor del cielo y de la tierra, que hayas escondido estas cosas
de los sabios y de los entendidos, y las hayas revelado á los niños.
\footnote{\textbf{11:25} 1Cor 1,19-29; Is 29,14; Luc 10,21-22; Juan
  17,25} \bibverse{26} Así, Padre, pues que así agradó en tus ojos.
\bibverse{27} Todas las cosas me son entregadas de mi Padre: y nadie
conoció al Hijo, sino el Padre; ni al Padre conoció alguno, sino el
Hijo, y aquel á quien el Hijo lo quisiere revelar.

\hypertarget{el-llamado-del-salvador-a-los-cansados-y-agobiados}{%
\subsection{El llamado del Salvador a los cansados
\hspace{0pt}\hspace{0pt}y
agobiados}\label{el-llamado-del-salvador-a-los-cansados-y-agobiados}}

\bibverse{28} Venid á mí todos los que estáis trabajados y cargados, que
yo os haré descansar. \footnote{\textbf{11:28} Mat 12,20; Mat 23,4; Jer
  31,25} \bibverse{29} Llevad mi yugo sobre vosotros, y aprended de mí,
que soy manso y humilde de corazón; y hallaréis descanso para vuestras
almas. \footnote{\textbf{11:29} Is 28,12; Jer 6,16; Zac 9,9}
\bibverse{30} Porque mi yugo es fácil, y ligera mi carga. \footnote{\textbf{11:30}
  Luc 11,46; 1Jn 5,3}

\hypertarget{los-ouxeddos-de-los-discuxedpulos-en-suxe1bado-la-primera-disputa-de-jesuxfas-con-los-fariseos-sobre-la-santificaciuxf3n-del-duxeda-de-reposo}{%
\subsection{Los oídos de los discípulos en sábado; La primera disputa de
Jesús con los fariseos sobre la santificación del día de
reposo}\label{los-ouxeddos-de-los-discuxedpulos-en-suxe1bado-la-primera-disputa-de-jesuxfas-con-los-fariseos-sobre-la-santificaciuxf3n-del-duxeda-de-reposo}}

\hypertarget{section-11}{%
\section{12}\label{section-11}}

\bibverse{1} En aquel tiempo iba Jesús por los sembrados en sábado; y
sus discípulos tenían hambre, y comenzaron á coger espigas, y á comer.
\footnote{\textbf{12:1} Deut 23,26} \bibverse{2} Y viéndolo los
Fariseos, le dijeron: He aquí tus discípulos hacen lo que no es lícito
hacer en sábado. \footnote{\textbf{12:2} Éxod 20,10}

\bibverse{3} Y él les dijo: ¿No habéis leído qué hizo David, teniendo él
hambre y los que con él estaban: \footnote{\textbf{12:3} 1Sam 21,7}
\bibverse{4} Cómo entró en la casa de Dios, y comió los panes de la
proposición, que no le era lícito comer, ni á los que estaban con él,
sino á solos los sacerdotes? \footnote{\textbf{12:4} Lev 24,9}
\bibverse{5} O ¿no habéis leído en la ley, que los sábados en el templo
los sacerdotes profanan el sábado, y son sin culpa? \footnote{\textbf{12:5}
  Núm 28,9} \bibverse{6} Pues os digo que uno mayor que el templo está
aquí. \bibverse{7} Mas si supieseis qué es: Misericordia quiero y no
sacrificio, no condenaríais á los inocentes: \footnote{\textbf{12:7} Mat
  9,13} \bibverse{8} Porque Señor es del sábado el Hijo del hombre.

\hypertarget{sanaciuxf3n-del-hombre-con-el-brazo-paralizado-en-suxe1bado-el-segundo-argumento-sobre-la-observancia-del-suxe1bado}{%
\subsection{Sanación del hombre con el brazo paralizado en sábado; el
segundo argumento sobre la observancia del
sábado}\label{sanaciuxf3n-del-hombre-con-el-brazo-paralizado-en-suxe1bado-el-segundo-argumento-sobre-la-observancia-del-suxe1bado}}

\bibverse{9} Y partiéndose de allí, vino á la sinagoga de ellos.
\bibverse{10} Y he aquí había allí uno que tenía una mano seca: y le
preguntaron, diciendo: ¿Es lícito curar en sábado? por acusarle.

\bibverse{11} Y él les dijo: ¿Qué hombre habrá de vosotros, que tenga
una oveja, y si cayere ésta en una fosa en sábado, no le eche mano, y la
levante? \bibverse{12} Pues ¿cuánto más vale un hombre que una oveja?
Así que, lícito es en los sábados hacer bien. \bibverse{13} Entonces
dijo á aquel hombre: Extiende tu mano. Y él la extendió, y fué
restituída sana como la otra. \bibverse{14} Y salidos los Fariseos,
consultaron contra él para destruirle. \footnote{\textbf{12:14} Juan
  5,16}

\hypertarget{jesuxfas-evade-la-persecuciuxf3n-su-actividad-sanadora-piadosa}{%
\subsection{Jesús evade la persecución; su actividad sanadora
piadosa}\label{jesuxfas-evade-la-persecuciuxf3n-su-actividad-sanadora-piadosa}}

\bibverse{15} Mas sabiéndolo Jesús, se apartó de allí: y le siguieron
muchas gentes, y sanaba á todos. \bibverse{16} Y él les encargaba
eficazmente que no le descubriesen: \bibverse{17} Para que se cumpliese
lo que estaba dicho por el profeta Isaías, que dijo: \bibverse{18} He
aquí mi siervo, al cual he escogido; mi Amado, en el cual se agrada mi
alma: pondré mi Espíritu sobre él, y á los Gentiles anunciará juicio.
\footnote{\textbf{12:18} Mat 3,17; Hech 3,13; Hech 3,26} \bibverse{19}
No contenderá, ni voceará: ni nadie oirá en las calles su voz.
\bibverse{20} La caña cascada no quebrará, y el pábilo que humea no
apagará, hasta que saque á victoria el juicio. \bibverse{21} Y en su
nombre esperarán los Gentiles.

\hypertarget{jesuxfas-se-defiende-de-la-blasfemia-de-los-fariseos-contra-beelzebul}{%
\subsection{Jesús se defiende de la blasfemia de los fariseos contra
Beelzebul}\label{jesuxfas-se-defiende-de-la-blasfemia-de-los-fariseos-contra-beelzebul}}

\bibverse{22} Entonces fué traído á él un endemoniado, ciego y mudo, y
le sanó; de tal manera, que el ciego y mudo hablaba y veía.
\bibverse{23} Y todas las gentes estaban atónitas, y decían: ¿Será éste
aquel Hijo de David? \bibverse{24} Mas los Fariseos, oyéndolo, decían:
Este no echa fuera los demonios, sino por Beelzebub, príncipe de los
demonios.

\bibverse{25} Y Jesús, como sabía los pensamientos de ellos, les dijo:
Todo reino dividido contra sí mismo, es desolado; y toda ciudad ó casa
dividida contra sí misma, no permanecerá. \bibverse{26} Y si Satanás
echa fuera á Satanás, contra sí mismo está dividido; ¿cómo, pues,
permanecerá su reino? \bibverse{27} Y si yo por Beelzebub echo fuera los
demonios, ¿vuestros hijos por quién los echan? Por tanto, ellos serán
vuestros jueces. \bibverse{28} Y si por espíritu de Dios yo echo fuera
los demonios, ciertamente ha llegado á vosotros el reino de Dios.
\footnote{\textbf{12:28} 1Jn 3,8} \bibverse{29} Porque, ¿cómo puede
alguno entrar en la casa del valiente, y saquear sus alhajas, si primero
no prendiere al valiente? y entonces saqueará su casa. \footnote{\textbf{12:29}
  Mat 4,1-11; Is 49,24}

\bibverse{30} El que no es conmigo, contra mí es; y el que conmigo no
recoge, derrama. \footnote{\textbf{12:30} Mar 9,40}

\hypertarget{advertencia-de-la-blasfemia-del-espuxedritu-del-uxe1rbol-y-los-frutos}{%
\subsection{Advertencia de la blasfemia del espíritu; del árbol y los
frutos}\label{advertencia-de-la-blasfemia-del-espuxedritu-del-uxe1rbol-y-los-frutos}}

\bibverse{31} Por tanto os digo: Todo pecado y blasfemia será perdonado
á los hombres: mas la blasfemia contra el Espíritu no será perdonada á
los hombres. \footnote{\textbf{12:31} Heb 6,4-6; Heb 10,26; 1Jn 5,16}
\bibverse{32} Y cualquiera que hablare contra el Hijo del hombre, le
será perdonado: mas cualquiera que hablare contra el Espíritu Santo, no
le será perdonado, ni en este siglo, ni en el venidero. \footnote{\textbf{12:32}
  1Tim 1,13}

\bibverse{33} O haced el árbol bueno, y su fruto bueno, ó haced el árbol
corrompido, y su fruto dañado; porque por el fruto es conocido el árbol.
\footnote{\textbf{12:33} Mat 7,17} \bibverse{34} Generación de víboras,
¿cómo podéis hablar bien, siendo malos? porque de la abundancia del
corazón habla la boca. \footnote{\textbf{12:34} Mat 3,7} \bibverse{35}
El hombre bueno del buen tesoro del corazón saca buenas cosas: y el
hombre malo del mal tesoro saca malas cosas. \bibverse{36} Mas yo os
digo, que toda palabra ociosa que hablaren los hombres, de ella darán
cuenta en el día del juicio. \bibverse{37} Porque por tus palabras serás
justificado, y por tus palabras serás condenado.

\hypertarget{el-rechazo-de-jesuxfas-a-la-demanda-de-seuxf1ales-la-seuxf1al-de-jonuxe1s-la-paruxe1bola-de-la-recauxedda}{%
\subsection{El rechazo de Jesús a la demanda de señales; la señal de
Jonás; la parábola de la
recaída}\label{el-rechazo-de-jesuxfas-a-la-demanda-de-seuxf1ales-la-seuxf1al-de-jonuxe1s-la-paruxe1bola-de-la-recauxedda}}

\bibverse{38} Entonces respondieron algunos de los escribas y de los
Fariseos, diciendo: Maestro, deseamos ver de ti señal.

\bibverse{39} Y él respondió, y les dijo: La generación mala y
adulterina demanda señal; mas señal no le será dada, sino la señal de
Jonás profeta. \bibverse{40} Porque como estuvo Jonás en el vientre de
la ballena tres días y tres noches, así estará el Hijo del hombre en el
corazón de la tierra tres días y tres noches. \footnote{\textbf{12:40}
  Jon 2,1-2; Efes 4,9; 1Pe 3,19} \bibverse{41} Los hombres de Nínive se
levantarán en el juicio con esta generación, y la condenarán; porque
ellos se arrepintieron á la predicación de Jonás; y he aquí más que
Jonás en este lugar. \footnote{\textbf{12:41} Jon 3,5} \bibverse{42} La
reina del Austro se levantará en el juicio con esta generación, y la
condenará; porque vino de los fines de la tierra para oir la sabiduría
de Salomón: y he aquí más que Salomón en este lugar. \footnote{\textbf{12:42}
  1Re 10,1-10}

\bibverse{43} Cuando el espíritu inmundo ha salido del hombre, anda por
lugares secos, buscando reposo, y no lo halla. \bibverse{44} Entonces
dice: Me volveré á mi casa de donde salí: y cuando viene, la halla
desocupada, barrida y adornada. \bibverse{45} Entonces va, y toma
consigo otros siete espíritus peores que él, y entrados, moran allí; y
son peores las cosas últimas del tal hombre que las primeras: así
también acontecerá á esta generación mala.

\hypertarget{los-verdaderos-parientes-de-jesuxfas}{%
\subsection{Los verdaderos parientes de
Jesús}\label{los-verdaderos-parientes-de-jesuxfas}}

\bibverse{46} Y estando él aún hablando á las gentes, he aquí su madre y
sus hermanos estaban fuera, que le querían hablar. \footnote{\textbf{12:46}
  Mat 13,55} \bibverse{47} Y le dijo uno: He aquí tu madre y tus
hermanos están fuera, que te quieren hablar.

\bibverse{48} Y respondiendo él al que le decía esto, dijo: ¿Quién es mi
madre y quiénes son mis hermanos? \bibverse{49} Y extendiendo su mano
hacia sus discípulos, dijo: He aquí mi madre y mis hermanos. \footnote{\textbf{12:49}
  Heb 2,11} \bibverse{50} Porque todo aquel que hiciere la voluntad de
mi Padre que está en los cielos, ese es mi hermano, y hermana, y madre.
\footnote{\textbf{12:50} Rom 8,29}

\hypertarget{la-paruxe1bola-del-sembrador-y-el-campo-cuuxe1druple}{%
\subsection{la parábola del sembrador y el campo
cuádruple}\label{la-paruxe1bola-del-sembrador-y-el-campo-cuuxe1druple}}

\hypertarget{section-12}{%
\section{13}\label{section-12}}

\bibverse{1} Y aquel día, saliendo Jesús de casa, se sentó junto á la
mar. \bibverse{2} Y se allegaron á él muchas gentes; y entrándose él en
el barco, se sentó, y toda la gente estaba á la ribera. \bibverse{3} Y
les habló muchas cosas por parábolas, diciendo: He aquí el que sembraba
salió á sembrar. \bibverse{4} Y sembrando, parte de la simiente cayó
junto al camino; y vinieron las aves, y la comieron. \bibverse{5} Y
parte cayó en pedregales, donde no tenía mucha tierra; y nació luego,
porque no tenía profundidad de tierra: \bibverse{6} Mas en saliendo el
sol, se quemó; y secóse, porque no tenía raíz. \bibverse{7} Y parte cayó
en espinas; y las espinas crecieron, y la ahogaron. \bibverse{8} Y parte
cayó en buena tierra, y dió fruto, cuál á ciento, cuál á sesenta, y cuál
á treinta. \bibverse{9} Quien tiene oídos para oir, oiga.

\hypertarget{explicaciuxf3n-de-jesuxfas-de-la-razuxf3n-y-el-propuxf3sito-de-sus-paruxe1bolas}{%
\subsection{Explicación de Jesús de la razón y el propósito de sus
parábolas}\label{explicaciuxf3n-de-jesuxfas-de-la-razuxf3n-y-el-propuxf3sito-de-sus-paruxe1bolas}}

\bibverse{10} Entonces, llegándose los discípulos, le dijeron: ¿Por qué
les hablas por parábolas?

\bibverse{11} Y él respondiendo, les dijo: Por que á vosotros es
concedido saber los misterios del reino de los cielos; mas á ellos no es
concedido. \footnote{\textbf{13:11} 1Cor 2,10} \bibverse{12} Porque á
cualquiera que tiene, se le dará, y tendrá más; pero al que no tiene,
aun lo que tiene le será quitado. \footnote{\textbf{13:12} Mat 25,28-29;
  Mar 4,25; Luc 8,18} \bibverse{13} Por eso les hablo por parábolas;
porque viendo no ven, y oyendo no oyen, ni entienden. \footnote{\textbf{13:13}
  Deut 29,3; Juan 16,25} \bibverse{14} De manera que se cumple en ellos
la profecía de Isaías, que dice: De oído oiréis, y no entenderéis; y
viendo veréis, y no miraréis. \bibverse{15} Porque el corazón de este
pueblo está engrosado, y de los oídos oyen pesadamente, y de sus ojos
guiñan: para que no vean de los ojos, y oigan de los oídos, y del
corazón entiendan, y se conviertan, y yo los sane.

\bibverse{16} Mas bienaventurados vuestros ojos, porque ven; y vuestros
oídos, porque oyen. \footnote{\textbf{13:16} Luc 10,23-24} \bibverse{17}
Porque de cierto os digo, que muchos profetas y justos desearon ver lo
que veis, y no lo vieron: y oir lo que oís, y no lo oyeron. \footnote{\textbf{13:17}
  1Pe 1,10}

\hypertarget{interpretaciuxf3n-de-la-paruxe1bola-del-sembrador}{%
\subsection{Interpretación de la parábola del
sembrador}\label{interpretaciuxf3n-de-la-paruxe1bola-del-sembrador}}

\bibverse{18} Oid, pues, vosotros la parábola del que siembra:
\bibverse{19} Oyendo cualquiera la palabra del reino, y no
entendiéndola, viene el malo, y arrebata lo que fué sembrado en su
corazón: éste es el que fué sembrado junto al camino. \bibverse{20} Y el
que fué sembrado en pedregales, éste es el que oye la palabra, y luego
la recibe con gozo. \bibverse{21} Mas no tiene raíz en sí, antes es
temporal: que venida la aflicción ó la persecución por la palabra, luego
se ofende. \bibverse{22} Y el que fué sembrado en espinas, éste es el
que oye la palabra; pero el afán de este siglo y el engaño de las
riquezas, ahogan la palabra, y hácese infructuosa. \footnote{\textbf{13:22}
  Mat 6,19-34; 1Tim 6,9} \bibverse{23} Mas el que fué sembrado en buena
tierra, éste es el que oye y entiende la palabra, y el que lleva fruto:
y lleva uno á ciento, y otro á sesenta, y otro á treinta.

\hypertarget{la-paruxe1bola-de-la-cizauxf1a-entre-el-trigo}{%
\subsection{La parábola de la cizaña entre el
trigo}\label{la-paruxe1bola-de-la-cizauxf1a-entre-el-trigo}}

\bibverse{24} Otra parábola les propuso, diciendo: El reino de los
cielos es semejante al hombre que siembra buena simiente en su campo:
\bibverse{25} Mas durmiendo los hombres, vino su enemigo, y sembró
cizaña entre el trigo, y se fué. \bibverse{26} Y como la hierba salió é
hizo fruto, entonces apareció también la cizaña. \bibverse{27} Y
llegándose los siervos del padre de la familia, le dijeron: Señor, ¿no
sembraste buena simiente en tu campo? ¿de dónde, pues, tiene cizaña?

\bibverse{28} Y él les dijo: Un hombre enemigo ha hecho esto. Y los
siervos le dijeron: ¿Quieres, pues, que vayamos y la cojamos?

\bibverse{29} Y él dijo: No; porque cogiendo la cizaña, no arranquéis
también con ella el trigo. \bibverse{30} Dejad crecer juntamente lo uno
y lo otro hasta la siega; y al tiempo de la siega yo diré á los
segadores: Coged primero la cizaña, y atadla en manojos para quemarla;
mas recoged el trigo en mi alfolí.

\hypertarget{las-dos-paruxe1bolas-de-la-semilla-de-mostaza-y-la-levadura}{%
\subsection{Las dos parábolas de la semilla de mostaza y la
levadura}\label{las-dos-paruxe1bolas-de-la-semilla-de-mostaza-y-la-levadura}}

\bibverse{31} Otra parábola les propuso, diciendo: El reino de los
cielos es semejante al grano de mostaza, que tomándolo alguno lo sembró
en su campo: \bibverse{32} El cual á la verdad es la más pequeña de
todas las simientes; mas cuando ha crecido, es la mayor de las
hortalizas, y se hace árbol, que vienen las aves del cielo y hacen nidos
en sus ramas. \footnote{\textbf{13:32} Ezeq 17,23}

\bibverse{33} Otra parábola les dijo: El reino de los cielos es
semejante á la levadura que tomó una mujer, y escondió en tres medidas
de harina, hasta que todo quedó leudo. \footnote{\textbf{13:33} Luc
  13,20-21}

\hypertarget{interpretaciuxf3n-de-la-paruxe1bola-de-la-cizauxf1a-del-trigo}{%
\subsection{Interpretación de la parábola de la cizaña del
trigo}\label{interpretaciuxf3n-de-la-paruxe1bola-de-la-cizauxf1a-del-trigo}}

\bibverse{34} Todo esto habló Jesús por parábolas á las gentes, y sin
parábolas no les hablaba: \footnote{\textbf{13:34} Mar 4,33-34}
\bibverse{35} Para que se cumpliese lo que fué dicho por el profeta, que
dijo: Abriré en parábolas mi boca; rebosaré cosas escondidas desde la
fundación del mundo.

\bibverse{36} Entonces, despedidas las gentes, Jesús se vino á casa; y
llegándose á él sus discípulos, le dijeron: Decláranos la parábola de la
cizaña del campo.

\bibverse{37} Y respondiendo él, les dijo: El que siembra la buena
simiente es el Hijo del hombre; \bibverse{38} Y el campo es el mundo; y
la buena simiente son los hijos del reino, y la cizaña son los hijos del
malo; \bibverse{39} Y el enemigo que la sembró, es el diablo; y la siega
es el fin del mundo, y los segadores son los ángeles. \bibverse{40} De
manera que como es cogida la cizaña, y quemada al fuego, así será en el
fin de este siglo. \bibverse{41} Enviará el Hijo del hombre sus ángeles,
y cogerán de su reino todos los escándalos, y los que hacen iniquidad,
\footnote{\textbf{13:41} Mat 25,31-46} \bibverse{42} Y los echarán en el
horno de fuego: allí será el lloro y el crujir de dientes. \bibverse{43}
Entonces los justos resplandecerán como el sol en el reino de su Padre:
el que tiene oídos para oir, oiga.

\hypertarget{las-uxfaltimas-tres-paruxe1bolas-tesoro-en-el-campo-perla-preciosa-red-de-pesca-conclusiuxf3n-de-la-paruxe1bola}{%
\subsection{Las últimas tres parábolas (tesoro en el campo; perla
preciosa; red de pesca); Conclusión de la
parábola}\label{las-uxfaltimas-tres-paruxe1bolas-tesoro-en-el-campo-perla-preciosa-red-de-pesca-conclusiuxf3n-de-la-paruxe1bola}}

\bibverse{44} Además, el reino de los cielos es semejante al tesoro
escondido en el campo; el cual hallado, el hombre lo encubre, y de gozo
de ello va, y vende todo lo que tiene, y compra aquel campo. \footnote{\textbf{13:44}
  Mat 19,29; Luc 14,33; Fil 3,7}

\bibverse{45} También el reino de los cielos es semejante al hombre
tratante, que busca buenas perlas; \bibverse{46} Que hallando una
preciosa perla, fué y vendió todo lo que tenía, y la compró.

\bibverse{47} Asimismo el reino de los cielos es semejante á la red, que
echada en la mar, coge de todas suertes de peces: \bibverse{48} La cual
estando llena, la sacaron á la orilla; y sentados, cogieron lo bueno en
vasos, y lo malo echaron fuera. \bibverse{49} Así será al fin del siglo:
saldrán los ángeles, y apartarán á los malos de entre los justos,
\footnote{\textbf{13:49} Mat 25,32} \bibverse{50} Y los echarán en el
horno del fuego: allí será el lloro y el crujir de dientes.
\bibverse{51} Díceles Jesús: ¿Habéis entendido todas estas cosas? Ellos
responden: Sí, Señor.

\bibverse{52} Y él les dijo: Por eso todo escriba docto en el reino de
los cielos, es semejante á un padre de familia, que saca de su tesoro
cosas nuevas y cosas viejas.

\hypertarget{rechazo-y-fracaso-de-jesuxfas-en-su-natal-nazaret}{%
\subsection{Rechazo y fracaso de Jesús en su natal
Nazaret}\label{rechazo-y-fracaso-de-jesuxfas-en-su-natal-nazaret}}

\bibverse{53} Y aconteció que acabando Jesús estas parábolas, pasó de
allí. \bibverse{54} Y venido á su tierra, les enseñaba en la sinagoga de
ellos, de tal manera que ellos estaban atónitos, y decían: ¿De dónde
tiene éste esta sabiduría, y estas maravillas? \bibverse{55} ¿No es éste
el hijo del carpintero? ¿no se llama su madre María, y sus hermanos
Jacobo y José, y Simón, y Judas? \bibverse{56} ¿Y no están todas sus
hermanas con nosotros? ¿De dónde, pues, tiene éste todas estas cosas?
\bibverse{57} Y se escandalizaban en él. Mas Jesús les dijo: No hay
profeta sin honra sino en su tierra y en su casa. \footnote{\textbf{13:57}
  Juan 4,44}

\bibverse{58} Y no hizo allí muchas maravillas, á causa de la
incredulidad de ellos.

\hypertarget{jesuxfas-y-herodes-el-fin-de-juan-el-bautista}{%
\subsection{Jesús y Herodes; el fin de Juan el
Bautista}\label{jesuxfas-y-herodes-el-fin-de-juan-el-bautista}}

\hypertarget{section-13}{%
\section{14}\label{section-13}}

\bibverse{1} En aquel tiempo Herodes el tetrarca oyó la fama de Jesús,
\bibverse{2} Y dijo á sus criados: Este es Juan el Bautista: él ha
resucitado de los muertos, y por eso virtudes obran en él. \bibverse{3}
Porque Herodes había prendido á Juan, y le había aprisionado y puesto en
la cárcel, por causa de Herodías, mujer de Felipe su hermano;
\bibverse{4} Porque Juan le decía: No te es lícito tenerla. \footnote{\textbf{14:4}
  Mat 19,9; Lev 18,16} \bibverse{5} Y quería matarle, mas temía al
pueblo; porque le tenían como á profeta. \footnote{\textbf{14:5} Mat
  21,26} \bibverse{6} Mas celebrándose el día del nacimiento de Herodes,
la hija de Herodías danzó en medio, y agradó á Herodes. \bibverse{7} Y
prometió él con juramento de darle todo lo que pidiese. \bibverse{8} Y
ella, instruída primero de su madre, dijo: Dame aquí en un plato la
cabeza de Juan el Bautista.

\bibverse{9} Entonces el rey se entristeció; mas por el juramento, y por
los que estaban juntamente á la mesa, mandó que se le diese.
\bibverse{10} Y enviando, degolló á Juan en la cárcel. \bibverse{11} Y
fué traída su cabeza en un plato, y dada á la muchacha; y ella la
presentó á su madre. \bibverse{12} Entonces llegaron sus discípulos, y
tomaron el cuerpo, y lo enterraron; y fueron, y dieron las nuevas á
Jesús.

\hypertarget{alimentando-a-los-cinco-mil}{%
\subsection{Alimentando a los cinco
mil}\label{alimentando-a-los-cinco-mil}}

\bibverse{13} Y oyéndolo Jesús, se apartó de allí en un barco á un lugar
descierto, apartado: y cuando las gentes lo oyeron, le siguieron á pie
de las ciudades.

\bibverse{14} Y saliendo Jesús, vió un gran gentío, y tuvo compasión de
ellos, y sanó á los que de ellos había enfermos. \bibverse{15} Y cuando
fué la tarde del día, se llegaron á él sus discípulos, diciendo: El
lugar es desierto, y el tiempo es ya pasado: despide las gentes, para
que se vayan por las aldeas, y compren para sí de comer.

\bibverse{16} Y Jesús les dijo: No tienen necesidad de irse: dadles
vosotros de comer.

\bibverse{17} Y ellos dijeron: No tenemos aquí sino cinco panes y dos
peces.

\bibverse{18} Y él les dijo: Traédmelos acá. \bibverse{19} Y mandando á
las gentes recostarse sobre la hierba, tomando los cinco panes y los dos
peces, alzando los ojos al cielo, bendijo, y partió y dió los panes á
los discípulos, y los discípulos á las gentes. \bibverse{20} Y comieron
todos, y se hartaron; y alzaron lo que sobró de los pedazos, doce cestas
llenas. \footnote{\textbf{14:20} 2Re 4,44} \bibverse{21} Y los que
comieron fueron como cinco mil hombres, sin las mujeres y los niños.

\hypertarget{regreso-de-los-discuxedpulos-al-otro-lado-del-lago-por-la-noche-el-caminar-de-jesuxfas-sobre-el-lago-el-desembarco-en-gennesaret}{%
\subsection{Regreso de los discípulos al otro lado del lago por la
noche; el caminar de Jesús sobre el lago; el desembarco en
Gennesaret}\label{regreso-de-los-discuxedpulos-al-otro-lado-del-lago-por-la-noche-el-caminar-de-jesuxfas-sobre-el-lago-el-desembarco-en-gennesaret}}

\bibverse{22} Y luego Jesús hizo á sus discípulos entrar en el barco, é
ir delante de él á la otra parte del lago, entre tanto que él despedía á
las gentes. \bibverse{23} Y despedidas las gentes, subió al monte,
apartado, á orar: y como fué la tarde del día, estaba allí solo.
\bibverse{24} Y ya el barco estaba en medio de la mar, atormentado de
las ondas; porque el viento era contrario. \bibverse{25} Mas á la cuarta
vela de la noche, Jesús fué á ellos andando sobre la mar. \bibverse{26}
Y los discípulos, viéndole andar sobre la mar, se turbaron, diciendo:
Fantasma es. Y dieron voces de miedo. \footnote{\textbf{14:26} Luc 24,37}
\bibverse{27} Mas luego Jesús les habló, diciendo: Confiad, yo soy; no
tengáis miedo.

\bibverse{28} Entonces le respondió Pedro, y dijo: Señor, si tú eres,
manda que yo vaya á ti sobre las aguas.

\bibverse{29} Y él dijo: Ven. Y descendiendo Pedro del barco, andaba
sobre las aguas para ir á Jesús.

\bibverse{30} Mas viendo el viento fuerte, tuvo miedo; y comenzándose á
hundir, dió voces, diciendo: Señor, sálvame.

\bibverse{31} Y luego Jesús, extendiendo la mano, trabó de él, y le
dice: Oh hombre de poca fe, ¿por qué dudaste? \bibverse{32} Y como ellos
entraron en el barco, sosegóse el viento. \bibverse{33} Entonces los que
estaban en el barco, vinieron y le adoraron, diciendo: Verdaderamente
eres Hijo de Dios.

\hypertarget{la-reuniuxf3n-de-personas-y-la-curaciuxf3n-de-los-enfermos-en-gennesaret}{%
\subsection{La reunión de personas y la curación de los enfermos en
Gennesaret}\label{la-reuniuxf3n-de-personas-y-la-curaciuxf3n-de-los-enfermos-en-gennesaret}}

\bibverse{34} Y llegando á la otra parte, vinieron á la tierra de
Genezaret. \bibverse{35} Y como le conocieron los hombres de aquel
lugar, enviaron por toda aquella tierra alrededor, y trajeron á él todos
los enfermos; \bibverse{36} Y le rogaban que solamente tocasen el borde
de su manto; y todos los que tocaron, quedaron sanos. \footnote{\textbf{14:36}
  Mat 9,21; Luc 6,19}

\hypertarget{la-disputa-de-jesuxfas-con-sus-oponentes-por-lavarse-las-manos-su-advertencia-de-los-estatutos-humanos-y-la-marca-de-la-verdadera-impureza}{%
\subsection{La disputa de Jesús con sus oponentes por lavarse las manos;
su advertencia de los estatutos humanos y la marca de la verdadera
impureza}\label{la-disputa-de-jesuxfas-con-sus-oponentes-por-lavarse-las-manos-su-advertencia-de-los-estatutos-humanos-y-la-marca-de-la-verdadera-impureza}}

\hypertarget{section-14}{%
\section{15}\label{section-14}}

\bibverse{1} Entonces llegaron á Jesús ciertos escribas y Fariseos de
Jerusalem, diciendo: \bibverse{2} ¿Por qué tus discípulos traspasan la
tradición de los ancianos? porque no se lavan las manos cuando comen
pan.

\bibverse{3} Y él respondiendo, les dijo: ¿Por qué también vosotros
traspasáis el mandamiento de Dios por vuestra tradición? \bibverse{4}
Porque Dios mandó, diciendo: Honra al padre y á la madre, y, El que
maldijere al padre ó á la madre, muera de muerte. \bibverse{5} Mas
vosotros decís: Cualquiera que dijere al padre ó á la madre: Es ya
ofrenda mía á Dios todo aquello con que pudiera valerte; \footnote{\textbf{15:5}
  Prov 28,24} \bibverse{6} No deberá honrar á su padre ó á su madre con
socorro. Así habéis invalidado el mandamiento de Dios por vuestra
tradición. \footnote{\textbf{15:6} 1Tim 5,8} \bibverse{7} Hipócritas,
bien profetizó de vosotros Isaías, diciendo: \bibverse{8} Este pueblo de
labios me honra; mas su corazón lejos está de mí. \bibverse{9} Mas en
vano me honran, enseñando doctrinas y mandamientos de hombres.

\bibverse{10} Y llamando á sí las gentes, les dijo: Oid, y entended:
\bibverse{11} No lo que entra en la boca contamina al hombre; mas lo que
sale de la boca, esto contamina al hombre. \footnote{\textbf{15:11} Hech
  10,15; 1Tim 4,4; Tit 1,15}

\bibverse{12} Entonces llegándose sus discípulos, le dijeron: ¿Sabes que
los Fariseos oyendo esta palabra se ofendieron?

\bibverse{13} Mas respondiendo él, dijo: Toda planta que no plantó mi
Padre celestial, será desarraigada. \bibverse{14} Dejadlos: son ciegos
guías de ciegos; y si el ciego guiare al ciego, ambos caerán en el hoyo.
\footnote{\textbf{15:14} Mat 23,24; Luc 6,39; Rom 2,19}

\bibverse{15} Y respondiendo Pedro, le dijo: Decláranos esta parábola.

\bibverse{16} Y Jesús dijo: ¿Aun también vosotros sois sin
entendimiento? \bibverse{17} ¿No entendéis aún, que todo lo que entra en
la boca, va al vientre, y es echado en la letrina? \bibverse{18} Mas lo
que sale de la boca, del corazón sale; y esto contamina al hombre.
\bibverse{19} Porque del corazón salen los malos pensamientos, muertes,
adulterios, fornicaciones, hurtos, falsos testimonios, blasfemias.
\bibverse{20} Estas cosas son las que contaminan al hombre: que comer
con las manos por lavar no contamina al hombre.

\hypertarget{jesuxfas-y-la-mujer-cananea-en-el-uxe1rea-de-tiro-y-siduxf3n}{%
\subsection{Jesús y la mujer cananea en el área de Tiro y
Sidón}\label{jesuxfas-y-la-mujer-cananea-en-el-uxe1rea-de-tiro-y-siduxf3n}}

\bibverse{21} Y saliendo Jesús de allí, se fué á las partes de Tiro y de
Sidón. \bibverse{22} Y he aquí una mujer Cananea, que había salido de
aquellos términos, clamaba, diciéndole: Señor, Hijo de David, ten
misericordia de mí; mi hija es malamente atormentada del demonio.

\bibverse{23} Mas él no le respondió palabra. Entonces llegándose sus
discípulos, le rogaron, diciendo: Despáchala, pues da voces tras
nosotros.

\bibverse{24} Y él respondiendo, dijo: No soy enviado sino á las ovejas
perdidas de la casa de Israel. \footnote{\textbf{15:24} Mat 10,5-6; Rom
  15,8}

\bibverse{25} Entonces ella vino, y le adoró, diciendo: Señor,
socórreme.

\bibverse{26} Y respondiendo él, dijo: No es bien tomar el pan de los
hijos, y echarlo á los perrillos.

\bibverse{27} Y ella dijo: Sí, Señor; mas los perrillos comen de las
migajas que caen de la mesa de sus señores.

\bibverse{28} Entonces respondiendo Jesús, dijo: Oh mujer, grande es tu
fe; sea hecho contigo como quieres. Y fué sana su hija desde aquella
hora.

\hypertarget{actividad-curativa-de-jesuxfas-en-galilea-en-la-orilla-oriental-del-lago-alimentando-a-los-cuatro-mil}{%
\subsection{Actividad curativa de Jesús en Galilea en la orilla oriental
del lago; Alimentando a los cuatro
mil}\label{actividad-curativa-de-jesuxfas-en-galilea-en-la-orilla-oriental-del-lago-alimentando-a-los-cuatro-mil}}

\bibverse{29} Y partido Jesús de allí, vino junto al mar de Galilea: y
subiendo al monte, se sentó allí. \bibverse{30} Y llegaron á él muchas
gentes, que tenían consigo cojos, ciegos, mudos, mancos, y otros muchos
enfermos: y los echaron á los pies de Jesús, y los sanó: \bibverse{31}
De manera que se maravillaban las gentes, viendo hablar los mudos, los
mancos sanos, andar los cojos, y ver los ciegos: y glorificaron al Dios
de Israel. \footnote{\textbf{15:31} Mar 7,37}

\bibverse{32} Y Jesús llamando á sus discípulos, dijo: Tengo lástima de
la gente, que ya hace tres días que perseveran conmigo, y no tienen qué
comer; y enviarlos ayunos no quiero, porque no desmayen en el camino.
\footnote{\textbf{15:32} Mat 14,13-21}

\bibverse{33} Entonces sus discípulos le dicen: ¿Dónde tenemos nosotros
tantos panes en el desierto, que hartemos á tan gran compañía?

\bibverse{34} Y Jesús les dice: ¿Cuántos panes tenéis? Y ellos dijeron:
Siete, y unos pocos pececillos.

\bibverse{35} Y mandó á las gentes que se recostasen sobre la tierra.
\bibverse{36} Y tomando los siete panes y los peces, haciendo gracias,
partió y dió á sus discípulos; y los discípulos á la gente.
\bibverse{37} Y comieron todos, y se hartaron: y alzaron lo que sobró de
los pedazos, siete espuertas llenas. \bibverse{38} Y eran los que habían
comido, cuatro mil hombres, sin las mujeres y los niños. \bibverse{39}
Entonces, despedidas las gentes, subió en el barco: y vino á los
términos de Magdalá.

\hypertarget{rechazo-de-la-demanda-de-los-oponentes-de-seuxf1ales-y-advertencia-contra-la-levadura-de-los-fariseos}{%
\subsection{Rechazo de la demanda de los oponentes de señales y
advertencia contra la levadura de los
fariseos}\label{rechazo-de-la-demanda-de-los-oponentes-de-seuxf1ales-y-advertencia-contra-la-levadura-de-los-fariseos}}

\hypertarget{section-15}{%
\section{16}\label{section-15}}

\bibverse{1} Y llegándose los Fariseos y los Saduceos para tentarle, le
pedían que les mostrase señal del cielo. \footnote{\textbf{16:1} Mat
  12,38} \bibverse{2} Mas él respondiendo, les dijo: Cuando es la tarde
del día, decís: Sereno; porque el cielo tiene arreboles. \bibverse{3} Y
á la mañana: Hoy tempestad; porque tiene arreboles el cielo triste.
Hipócritas, que sabéis hacer diferencia en la faz del cielo; ¿y en las
señales de los tiempos no podéis? \bibverse{4} La generación mala y
adulterina demanda señal; mas señal no le será dada, sino la señal de
Jonás profeta. Y dejándolos, se fué. \footnote{\textbf{16:4} Mat
  12,39-40}

\bibverse{5} Y viniendo sus discípulos de la otra parte del lago, se
habían olvidado de tomar pan. \bibverse{6} Y Jesús les dijo: Mirad, y
guardaos de la levadura de los Fariseos y de los Saduceos.

\bibverse{7} Y ellos pensaban dentro de sí, diciendo: Esto dice porque
no tomamos pan.

\bibverse{8} Y entendiéndolo Jesús, les dijo: ¿Por qué pensáis dentro de
vosotros, hombres de poca fe, que no tomasteis pan? \bibverse{9} ¿No
entendéis aún, ni os acordáis de los cinco panes entre cinco mil
hombres, y cuántos cestos alzasteis? \footnote{\textbf{16:9} Mat
  14,17-21} \bibverse{10} ¿Ni de los siete panes entre cuatro mil, y
cuántas espuertas tomasteis? \footnote{\textbf{16:10} Mat 15,34-38}
\bibverse{11} ¿Cómo es que no entendéis que no por el pan os dije, que
os guardaseis de la levadura de los Fariseos y de los Saduceos?

\bibverse{12} Entonces entendieron que no les había dicho que se
guardasen de la levadura de pan, sino de la doctrina de los Fariseos y
de los Saduceos.

\hypertarget{la-confesiuxf3n-del-mesuxedas-de-pedro-en-cesarea-de-filipo-llamando-a-pedro-a-ser-el-fundador-y-luxedder-de-la-iglesia}{%
\subsection{La Confesión del Mesías de Pedro en Cesarea de Filipo;
Llamando a Pedro a ser el fundador y líder de la
iglesia}\label{la-confesiuxf3n-del-mesuxedas-de-pedro-en-cesarea-de-filipo-llamando-a-pedro-a-ser-el-fundador-y-luxedder-de-la-iglesia}}

\bibverse{13} Y viniendo Jesús á las partes de Cesarea de Filipo,
preguntó á sus discípulos, diciendo: ¿Quién dicen los hombres que es el
Hijo del hombre?

\bibverse{14} Y ellos dijeron: Unos, Juan el Bautista; y otros, Elías; y
otros, Jeremías, ó alguno de los profetas. \footnote{\textbf{16:14} Mat
  14,2; Mat 17,10; Luc 7,16}

\bibverse{15} El les dice: Y vosotros, ¿quién decís que soy?

\bibverse{16} Y respondiendo Simón Pedro, dijo: Tú eres el Cristo, el
Hijo del Dios viviente.

\bibverse{17} Entonces, respondiendo Jesús, le dijo: Bienaventurado
eres, Simón, hijo de Jonás; porque no te lo reveló carne ni sangre, mas
mi Padre que está en los cielos. \footnote{\textbf{16:17} Mat 11,27; Gal
  1,15-16} \bibverse{18} Mas yo también te digo, que tú eres Pedro, y
sobre esta piedra edificaré mi iglesia; y las puertas del infierno no
prevalecerán contra ella. \footnote{\textbf{16:18} Juan 1,42; Efes 2,20}
\bibverse{19} Y á ti daré las llaves del reino de los cielos; y todo lo
que ligares en la tierra será ligado en los cielos; y todo lo que
desatares en la tierra será desatado en los cielos. \footnote{\textbf{16:19}
  Mat 18,18} \bibverse{20} Entonces mandó á sus discípulos que á nadie
dijesen que él era Jesús el Cristo. \footnote{\textbf{16:20} Mat 17,9}

\hypertarget{primer-anuncio-de-sufrimiento}{%
\subsection{Primer anuncio de
sufrimiento}\label{primer-anuncio-de-sufrimiento}}

\bibverse{21} Desde aquel tiempo comenzó Jesús á declarar á sus
discípulos que le convenía ir á Jerusalem, y padecer mucho de los
ancianos, y de los príncipes de los sacerdotes, y de los escribas; y ser
muerto, y resucitar al tercer día. \footnote{\textbf{16:21} Mat 12,40;
  Juan 2,19}

\bibverse{22} Y Pedro, tomándolo aparte, comenzó á reprenderle,
diciendo: Señor, ten compasión de ti: en ninguna manera esto te
acontezca.

\bibverse{23} Entonces él, volviéndose, dijo á Pedro: Quítate de delante
de mí, Satanás; me eres escándalo; porque no entiendes lo que es de Dios
sino lo que es de los hombres.

\hypertarget{proverbios-sobre-el-seguimiento-de-los-discuxedpulos-en-el-sufrimiento}{%
\subsection{Proverbios sobre el seguimiento de los discípulos en el
sufrimiento}\label{proverbios-sobre-el-seguimiento-de-los-discuxedpulos-en-el-sufrimiento}}

\bibverse{24} Entonces Jesús dijo á sus discípulos: Si alguno quiere
venir en pos de mí, niéguese á sí mismo, y tome su cruz, y sígame.
\bibverse{25} Porque cualquiera que quisiere salvar su vida, la perderá,
y cualquiera que perdiere su vida por causa de mí, la hallará.
\footnote{\textbf{16:25} Apoc 12,11} \bibverse{26} Porque ¿de qué
aprovecha al hombre, si granjeare todo el mundo, y perdiere su alma? O
¿qué recompensa dará el hombre por su alma? \footnote{\textbf{16:26} Luc
  12,20} \bibverse{27} Porque el Hijo del hombre vendrá en la gloria de
su Padre con sus ángeles, y entonces pagará á cada uno conforme á sus
obras. \footnote{\textbf{16:27} Rom 2,6} \bibverse{28} De cierto os
digo: hay algunos de los que están aquí, que no gustarán la muerte,
hasta que hayan visto al Hijo del hombre viniendo en su reino.
\footnote{\textbf{16:28} Mat 10,23}

\hypertarget{la-transfiguraciuxf3n-de-jesuxfas-en-la-montauxf1a}{%
\subsection{La transfiguración de Jesús en la
montaña}\label{la-transfiguraciuxf3n-de-jesuxfas-en-la-montauxf1a}}

\hypertarget{section-16}{%
\section{17}\label{section-16}}

\bibverse{1} Y después de seis días, Jesús toma á Pedro, y á Jacobo, y á
Juan su hermano, y los lleva aparte á un monte alto: \footnote{\textbf{17:1}
  Mat 26,37; Mar 5,37; Mar 13,3; Mar 14,33; Luc 8,51} \bibverse{2} Y se
transfiguró delante de ellos; y resplandeció su rostro como el sol, y
sus vestidos fueron blancos como la luz. \footnote{\textbf{17:2} 2Pe
  1,16-18; Apoc 1,16} \bibverse{3} Y he aquí les aparecieron Moisés y
Elías, hablando con él.

\bibverse{4} Y respondiendo Pedro, dijo á Jesús: Señor, bien es que nos
quedemos aquí: si quieres, hagamos aquí tres pabellones: para ti uno, y
para Moisés otro, y otro para Elías.

\bibverse{5} Y estando aún él hablando, he aquí una nube de luz que los
cubrió; y he aquí una voz de la nube, que dijo: Este es mi Hijo amado,
en el cual tomo contentamiento: á él oid. \footnote{\textbf{17:5} Mat
  3,17}

\bibverse{6} Y oyendo esto los discípulos, cayeron sobre sus rostros, y
temieron en gran manera. \bibverse{7} Entonces Jesús llegando, los tocó,
y dijo: Levantaos, y no temáis. \bibverse{8} Y alzando ellos sus ojos, á
nadie vieron, sino á solo Jesús.

\bibverse{9} Y como descendieron del monte, les mandó Jesús, diciendo:
No digáis á nadie la visión, hasta que el Hijo del hombre resucite de
los muertos.

\bibverse{10} Entonces sus discípulos le preguntaron, diciendo: ¿Por qué
dicen pues los escribas que es menester que Elías venga primero?
\footnote{\textbf{17:10} Mat 11,14}

\bibverse{11} Y respondiendo Jesús, les dijo: A la verdad, Elías vendrá
primero, y restituirá todas las cosas. \bibverse{12} Mas os digo, que ya
vino Elías, y no le conocieron; antes hicieron en él todo lo que
quisieron: así también el Hijo del hombre padecerá de ellos.
\bibverse{13} Los discípulos entonces entendieron, que les habló de Juan
el Bautista. \footnote{\textbf{17:13} Luc 1,17}

\hypertarget{curaciuxf3n-de-un-niuxf1o-epiluxe9ptico-enseuxf1ando-sobre-el-fracaso-de-los-discuxedpulos}{%
\subsection{Curación de un niño epiléptico; Enseñando sobre el fracaso
de los
discípulos}\label{curaciuxf3n-de-un-niuxf1o-epiluxe9ptico-enseuxf1ando-sobre-el-fracaso-de-los-discuxedpulos}}

\bibverse{14} Y como ellos llegaron al gentío, vino á él un hombre
hincándosele de rodillas, \bibverse{15} Y diciendo: Señor, ten
misericordia de mi hijo, que es lunático, y padece malamente; porque
muchas veces cae en el fuego, y muchas en el agua. \bibverse{16} Y le he
presentado á tus discípulos, y no le han podido sanar.

\bibverse{17} Y respondiendo Jesús, dijo: ¡Oh generación infiel y
torcida! ¿hasta cuándo tengo de estar con vosotros? ¿hasta cuándo os
tengo de sufrir? traédmele acá. \bibverse{18} Y Jesús le reprendió, y
salió el demonio de él; y el mozo fué sano desde aquella hora.

\bibverse{19} Entonces, llegándose los discípulos á Jesús, aparte,
dijeron: ¿Por qué nosotros no lo pudimos echar fuera?

\bibverse{20} Y Jesús les dijo: Por vuestra incredulidad; porque de
cierto os digo, que si tuviereis fe como un grano de mostaza, diréis á
este monte: Pásate de aquí allá: y se pasará: y nada os será imposible.
\footnote{\textbf{17:20} Mat 21,21; Luc 17,6; 1Cor 13,2} \bibverse{21}
Mas este linaje no sale sino por oración y ayuno. \footnote{\textbf{17:21}
  Mar 9,29}

\hypertarget{segundo-anuncio-del-sufrimiento-en-galilea}{%
\subsection{Segundo anuncio del sufrimiento en
Galilea}\label{segundo-anuncio-del-sufrimiento-en-galilea}}

\bibverse{22} Y estando ellos en Galilea, Jesús les dijo: El Hijo del
hombre será entregado en manos de hombres, \footnote{\textbf{17:22} Mat
  16,21; Mat 20,18-19} \bibverse{23} Y le matarán; mas al tercer día
resucitará. Y ellos se entristecieron en gran manera.

\hypertarget{el-impuesto-del-templo-y-su-maravilloso-pago-en-capernaum}{%
\subsection{El impuesto del templo y su maravilloso pago en
Capernaum}\label{el-impuesto-del-templo-y-su-maravilloso-pago-en-capernaum}}

\bibverse{24} Y como llegaron á Capernaum, vinieron á Pedro los que
cobraban las dos dracmas, y dijeron: ¿Vuestro Maestro no paga las dos
dracmas? \bibverse{25} El dice: Sí. Y entrando él en casa, Jesús le
habló antes, diciendo: ¿Qué te parece, Simón? Los reyes de la tierra,
¿de quién cobran los tributos ó el censo? ¿de sus hijos ó de los
extraños?

\bibverse{26} Pedro le dice: De los extraños. Jesús le dijo: Luego los
hijos son francos.

\bibverse{27} Mas porque no los escandalicemos, ve á la mar, y echa el
anzuelo, y el primer pez que viniere, tómalo, y abierta su boca,
hallarás un estatero: tómalo, y dáselo por mí y por ti.

\hypertarget{controversia-entre-discuxedpulos-la-exhortaciuxf3n-de-jesuxfas-a-la-humildad}{%
\subsection{Controversia entre discípulos; La exhortación de Jesús a la
humildad}\label{controversia-entre-discuxedpulos-la-exhortaciuxf3n-de-jesuxfas-a-la-humildad}}

\hypertarget{section-17}{%
\section{18}\label{section-17}}

\bibverse{1} En aquel tiempo se llegaron los discípulos á Jesús,
diciendo: ¿Quién es el mayor en el reino de los cielos?

\bibverse{2} Y llamando Jesús á un niño, le puso en medio de ellos,
\bibverse{3} Y dijo: De cierto os digo, que si no os volviereis, y
fuereis como niños, no entraréis en el reino de los cielos. \footnote{\textbf{18:3}
  Mat 19,14} \bibverse{4} Así que, cualquiera que se humillare como este
niño, éste es el mayor en el reino de los cielos. \bibverse{5} Y
cualquiera que recibiere á un tal niño en mi nombre, á mí recibe.

\hypertarget{la-preocupaciuxf3n-de-jesuxfas-por-los-pequeuxf1os-y-los-duxe9biles-advertencia-contra-los-seductores-del-mal}{%
\subsection{La preocupación de Jesús por los pequeños y los débiles;
Advertencia contra los seductores del
mal}\label{la-preocupaciuxf3n-de-jesuxfas-por-los-pequeuxf1os-y-los-duxe9biles-advertencia-contra-los-seductores-del-mal}}

\bibverse{6} Y cualquiera que escandalizare á alguno de estos pequeños
que creen en mí, mejor le fuera que se le colgase al cuello una piedra
de molino de asno, y que se le anegase en el profundo de la mar.
\footnote{\textbf{18:6} Luc 17,1-2}

\bibverse{7} ¡Ay del mundo por los escándalos! porque necesario es que
vengan escándalos; mas ¡ay de aquel hombre por el cual viene el
escándalo! \bibverse{8} Por tanto, si tu mano ó tu pie te fuere ocasión
de caer, córtalo y échalo de ti: mejor te es entrar cojo ó manco en la
vida, que teniendo dos manos ó dos pies ser echado en el fuego eterno.
\bibverse{9} Y si tu ojo te fuere ocasión de caer, sácalo y échalo de
ti: mejor te es entrar con un solo ojo en la vida, que teniendo dos ojos
ser echado en el infierno del fuego. \bibverse{10} Mirad no tengáis en
poco á alguno de estos pequeños; porque os digo que sus ángeles en los
cielos ven siempre la faz de mi Padre que está en los cielos.
\footnote{\textbf{18:10} Heb 1,14} \bibverse{11} Porque el Hijo del
hombre ha venido para salvar lo que se había perdido. \footnote{\textbf{18:11}
  Mat 9,13; Luc 19,10}

\hypertarget{la-paruxe1bola-de-la-oveja-perdida}{%
\subsection{La parábola de la oveja
perdida}\label{la-paruxe1bola-de-la-oveja-perdida}}

\bibverse{12} ¿Qué os parece? Si tuviese algún hombre cien ovejas, y se
descarriase una de ellas, ¿no iría por los montes, dejadas las noventa y
nueve, á buscar la que se había descarriado? \bibverse{13} Y si
aconteciese hallarla, de cierto os digo, que más se goza de aquélla, que
de las noventa y nueve que no se descarriaron. \bibverse{14} Así, no es
la voluntad de vuestro Padre que está en los cielos, que se pierda uno
de estos pequeños.

\hypertarget{de-comportamiento-hacia-el-hermano-pecador-sobre-el-efecto-del-juicio-y-la-oraciuxf3n-de-la-iglesia}{%
\subsection{De comportamiento hacia el hermano pecador; sobre el efecto
del juicio y la oración de la
iglesia}\label{de-comportamiento-hacia-el-hermano-pecador-sobre-el-efecto-del-juicio-y-la-oraciuxf3n-de-la-iglesia}}

\bibverse{15} Por tanto, si tu hermano pecare contra ti, ve, y
redargúyele entre ti y él solo: si te oyere, has ganado á tu hermano.
\footnote{\textbf{18:15} Lev 19,17; Luc 17,3; Gal 6,1} \bibverse{16} Mas
si no te oyere, toma aún contigo uno ó dos, para que en boca de dos ó de
tres testigos conste toda palabra. \footnote{\textbf{18:16} Deut 19,15}
\bibverse{17} Y si no oyere á ellos, dilo á la iglesia: y si no oyere á
la iglesia, tenle por étnico y publicano. \footnote{\textbf{18:17} 1Cor
  5,13; 2Tes 3,6; Tit 3,10} \bibverse{18} De cierto os digo que todo lo
que ligareis en la tierra, será ligado en el cielo; y todo lo que
desatareis en la tierra, será desatado en el cielo. \footnote{\textbf{18:18}
  Mat 16,19; Juan 20,23} \bibverse{19} Otra vez os digo, que si dos de
vosotros se convinieren en la tierra, de toda cosa que pidieren, les
será hecho por mi Padre que está en los cielos. \footnote{\textbf{18:19}
  Mar 11,24} \bibverse{20} Porque donde están dos ó tres congregados en
mi nombre, allí estoy en medio de ellos. \footnote{\textbf{18:20} Mat
  28,20}

\hypertarget{del-perduxf3n-la-paruxe1bola-del-sinverguxfcenza}{%
\subsection{Del perdón; la parábola del
sinvergüenza}\label{del-perduxf3n-la-paruxe1bola-del-sinverguxfcenza}}

\bibverse{21} Entonces Pedro, llegándose á él, dijo: Señor, ¿cuántas
veces perdonaré á mi hermano que pecare contra mí? ¿hasta siete?

\bibverse{22} Jesús le dice: No te digo hasta siete, mas aun hasta
setenta veces siete. \footnote{\textbf{18:22} Gén 4,24; Luc 17,4; Efes
  4,32} \bibverse{23} Por lo cual, el reino de los cielos es semejante á
un hombre rey, que quiso hacer cuentas con sus siervos. \bibverse{24} Y
comenzando á hacer cuentas, le fué presentado uno que le debía diez mil
talentos. \bibverse{25} Mas á éste, no pudiendo pagar, mandó su señor
venderle, y á su mujer é hijos, con todo lo que tenía, y que se le
pagase. \bibverse{26} Entonces aquel siervo, postrado, le adoraba,
diciendo: Señor, ten paciencia conmigo, y yo te lo pagaré todo.
\bibverse{27} El señor, movido á misericordia de aquel siervo, le soltó
y le perdonó la deuda.

\bibverse{28} Y saliendo aquel siervo, halló á uno de sus consiervos,
que le debía cien denarios; y trabando de él, le ahogaba, diciendo:
Págame lo que debes.

\bibverse{29} Entonces su consiervo, postrándose á sus pies, le rogaba,
diciendo: Ten paciencia conmigo, y yo te lo pagaré todo. \bibverse{30}
Mas él no quiso; sino fué, y le echó en la cárcel hasta que pagase la
deuda. \bibverse{31} Y viendo sus consiervos lo que pasaba, se
entristecieron mucho, y viniendo, declararon á su señor todo lo que
había pasado. \bibverse{32} Entonces llamándole su señor, le dice:
Siervo malvado, toda aquella deuda te perdoné, porque me rogaste:
\bibverse{33} ¿No te convenía también á ti tener misericordia de tu
consiervo, como también yo tuve misericordia de ti? \footnote{\textbf{18:33}
  1Jn 4,11} \bibverse{34} Entonces su señor, enojado, le entregó á los
verdugos, hasta que pagase todo lo que le debía. \footnote{\textbf{18:34}
  Mat 5,26} \bibverse{35} Así también hará con vosotros mi Padre
celestial, si no perdonareis de vuestros corazones cada uno á su hermano
sus ofensas. \footnote{\textbf{18:35} Mat 6,14-15; Sant 2,13}

\hypertarget{salida-hacia-jerusaluxe9n-y-caminata-por-la-ribera-oriental-conversaciones-sobre-el-matrimonio-sobre-el-divorcio-y-la-renuncia-al-matrimonio}{%
\subsection{Salida hacia Jerusalén y caminata por la Ribera Oriental;
Conversaciones sobre el matrimonio, sobre el divorcio y la renuncia al
matrimonio}\label{salida-hacia-jerusaluxe9n-y-caminata-por-la-ribera-oriental-conversaciones-sobre-el-matrimonio-sobre-el-divorcio-y-la-renuncia-al-matrimonio}}

\hypertarget{section-18}{%
\section{19}\label{section-18}}

\bibverse{1} Y aconteció que acabando Jesús estas palabras, se pasó de
Galilea, y vino á los términos de Judea, pasado el Jordán. \bibverse{2}
Y le siguieron muchas gentes, y los sanó allí.

\bibverse{3} Entonces se llegaron á él los Fariseos, tentándole, y
diciéndole: ¿Es lícito al hombre repudiar á su mujer por cualquiera
causa?

\bibverse{4} Y él respondiendo, les dijo: ¿No habéis leído que el que
los hizo al principio, macho y hembra los hizo, \footnote{\textbf{19:4}
  Gén 1,27} \bibverse{5} Y dijo: Por tanto, el hombre dejará padre y
madre, y se unirá á su mujer, y serán dos en una carne? \bibverse{6} Así
que, no son ya más dos, sino una carne: por tanto, lo que Dios juntó, no
lo aparte el hombre.

\bibverse{7} Dícenle: ¿Por qué, pues, Moisés mandó dar carta de
divorcio, y repudiarla? \footnote{\textbf{19:7} Deut 24,1}

\bibverse{8} Díceles: Por la dureza de vuestro corazón Moisés os
permitió repudiar á vuestras mujeres: mas al principio no fué así.
\bibverse{9} Y yo os digo que cualquiera que repudiare á su mujer, si no
fuere por causa de fornicación, y se casare con otra, adultera: y el que
se casare con la repudiada, adultera.

\bibverse{10} Dícenle sus discípulos: Si así es la condición del hombre
con su mujer, no conviene casarse.

\bibverse{11} Entonces él les dijo: No todos reciben esta palabra, sino
aquellos á quienes es dado. \footnote{\textbf{19:11} 1Cor 7,7; 1Cor 7,17}
\bibverse{12} Porque hay eunucos que nacieron así del vientre de su
madre; y hay eunucos, que son hechos eunucos por los hombres; y hay
eunucos que se hicieron á sí mismos eunucos por causa del reino de los
cielos; el que pueda ser capaz de eso, séalo.

\hypertarget{jesuxfas-bendice-a-los-niuxf1os}{%
\subsection{Jesús bendice a los
niños}\label{jesuxfas-bendice-a-los-niuxf1os}}

\bibverse{13} Entonces le fueron presentados unos niños, para que
pusiese las manos sobre ellos, y orase; y los discípulos les riñeron.
\bibverse{14} Y Jesús dijo: Dejad á los niños, y no les impidáis de
venir á mí; porque de los tales es el reino de los cielos. \bibverse{15}
Y habiendo puesto sobre ellos las manos, se partió de allí.

\hypertarget{la-conversaciuxf3n-de-jesuxfas-con-el-joven-rico-el-peligro-de-la-riqueza}{%
\subsection{La conversación de Jesús con el joven rico; el peligro de la
riqueza}\label{la-conversaciuxf3n-de-jesuxfas-con-el-joven-rico-el-peligro-de-la-riqueza}}

\bibverse{16} Y he aquí, uno llegándose le dijo: Maestro bueno, ¿qué
bien haré para tener la vida eterna?

\bibverse{17} Y él le dijo: ¿Por qué me llamas bueno? Ninguno es bueno
sino uno, es á saber, Dios: y si quieres entrar en la vida, guarda los
mandamientos.

\bibverse{18} Dícele: ¿Cuáles? Y Jesús dijo: No matarás: No adulterarás:
No hurtarás: No dirás falso testimonio: \footnote{\textbf{19:18} Éxod
  20,12-16}

\bibverse{19} Honra á tu padre y á tu madre: y, Amarás á tu prójimo como
á ti mismo. \footnote{\textbf{19:19} Lev 19,18}

\bibverse{20} Dícele el mancebo: Todo esto guardé desde mi juventud:
¿qué más me falta?

\bibverse{21} Dícele Jesús: Si quieres ser perfecto, anda, vende lo que
tienes, y dalo á los pobres, y tendrás tesoro en el cielo; y ven,
sígueme. \footnote{\textbf{19:21} Mat 6,20; Luc 12,33} \bibverse{22} Y
oyendo el mancebo esta palabra, se fué triste, porque tenía muchas
posesiones. \footnote{\textbf{19:22} Sal 62,11}

\bibverse{23} Entonces Jesús dijo á sus discípulos: De cierto os digo,
que un rico difícilmente entrará en el reino de los cielos.
\bibverse{24} Mas os digo, que más liviano trabajo es pasar un camello
por el ojo de una aguja, que entrar un rico en el reino de Dios.
\footnote{\textbf{19:24} Mat 7,14}

\bibverse{25} Mas sus discípulos, oyendo estas cosas, se espantaron en
gran manera, diciendo: ¿Quién pues podrá ser salvo?

\bibverse{26} Y mirándolos Jesús, les dijo: Para con los hombres
imposible es esto; mas para con Dios todo es posible.

\hypertarget{la-recompensa-de-seguir-a-jesuxfas-y-la-renuncia}{%
\subsection{La recompensa de seguir a Jesús y la
renuncia}\label{la-recompensa-de-seguir-a-jesuxfas-y-la-renuncia}}

\bibverse{27} Entonces respondiendo Pedro, le dijo: He aquí, nosotros
hemos dejado todo, y te hemos seguido: ¿qué pues tendremos? \footnote{\textbf{19:27}
  Mat 4,20; Mat 4,22}

\bibverse{28} Y Jesús les dijo: De cierto os digo, que vosotros que me
habéis seguido, en la regeneración, cuando se sentará el Hijo del hombre
en el trono de su gloria, vosotros también os sentaréis sobre doce
tronos, para juzgar á las doce tribus de Israel. \footnote{\textbf{19:28}
  Luc 22,30; 1Cor 6,2; Apoc 3,21}

\bibverse{29} Y cualquiera que dejare casas, ó hermanos, ó hermanas, ó
padre, ó madre, ó mujer, ó hijos, ó tierras, por mi nombre, recibirá
cien veces tanto, y heredará la vida eterna. \bibverse{30} Mas muchos
primeros serán postreros, y postreros primeros.

\hypertarget{paruxe1bola-de-los-trabajadores-de-la-viuxf1a}{%
\subsection{Parábola de los trabajadores de la
viña}\label{paruxe1bola-de-los-trabajadores-de-la-viuxf1a}}

\hypertarget{section-19}{%
\section{20}\label{section-19}}

\bibverse{1} Porque el reino de los cielos es semejante á un hombre,
padre de familia, que salió por la mañana á ajustar obreros para su
viña. \bibverse{2} Y habiéndose concertado con los obreros en un denario
al día, los envió á su viña. \bibverse{3} Y saliendo cerca de la hora de
las tres, vió otros que estaban en la plaza ociosos; \bibverse{4} Y les
dijo: Id también vosotros á mi viña, y os daré lo que fuere justo. Y
ellos fueron. \bibverse{5} Salió otra vez cerca de las horas sexta y
nona, é hizo lo mismo. \bibverse{6} Y saliendo cerca de la hora
undécima, halló otros que estaban ociosos; y díceles: ¿Por qué estáis
aquí todo el día ociosos?

\bibverse{7} Dícenle: Porque nadie nos ha ajustado. Díceles: Id también
vosotros á la viña, y recibiréis lo que fuere justo.

\bibverse{8} Y cuando fué la tarde del día, el señor de la viña dijo á
su mayordomo: Llama á los obreros y págales el jornal, comenzando desde
los postreros hasta los primeros. \bibverse{9} Y viniendo los que habían
ido cerca de la hora undécima, recibieron cada uno un denario.
\bibverse{10} Y viniendo también los primeros, pensaron que habían de
recibir más; pero también ellos recibieron cada uno un denario.
\bibverse{11} Y tomándolo, murmuraban contra el padre de la familia,
\bibverse{12} Diciendo: Estos postreros sólo han trabajado una hora, y
los has hecho iguales á nosotros, que hemos llevado la carga y el calor
del día.

\bibverse{13} Y él respondiendo, dijo á uno de ellos: Amigo, no te hago
agravio; ¿no te concertaste conmigo por un denario? \bibverse{14} Toma
lo que es tuyo, y vete; mas quiero dar á este postrero, como á ti.
\bibverse{15} ¿No me es lícito á mí hacer lo que quiero con lo mío? ó
¿es malo tu ojo, porque yo soy bueno? \footnote{\textbf{20:15} Rom 9,16;
  Rom 9,21} \bibverse{16} Así los primeros serán postreros, y los
postreros primeros: porque muchos son llamados, mas pocos escogidos.

\hypertarget{salida-hacia-jerusaluxe9n-tercer-anuncio-del-sufrimiento-de-jesuxfas}{%
\subsection{Salida hacia Jerusalén; tercer anuncio del sufrimiento de
Jesús}\label{salida-hacia-jerusaluxe9n-tercer-anuncio-del-sufrimiento-de-jesuxfas}}

\bibverse{17} Y subiendo Jesús á Jerusalem, tomó sus doce discípulos
aparte en el camino, y les dijo: \bibverse{18} He aquí subimos á
Jerusalem, y el Hijo del hombre será entregado á los príncipes de los
sacerdotes y á los escribas, y le condenarán á muerte; \bibverse{19} Y
le entregarán á los Gentiles para que le escarnezcan, y azoten, y
crucifiquen; mas al tercer día resucitará.

\hypertarget{solicitud-ambiciosa-de-salomuxe9-para-sus-hijos-santiago-y-juan}{%
\subsection{Solicitud ambiciosa de Salomé para sus hijos Santiago y
Juan}\label{solicitud-ambiciosa-de-salomuxe9-para-sus-hijos-santiago-y-juan}}

\bibverse{20} Entonces se llegó á él la madre de los hijos de Zebedeo
con sus hijos, adorándole, y pidiéndole algo. \footnote{\textbf{20:20}
  Mat 10,2} \bibverse{21} Y él le dijo: ¿Qué quieres? Ella le dijo: Di
que se sienten estos dos hijos míos, el uno á tu mano derecha, y el otro
á tu izquierda, en tu reino. \footnote{\textbf{20:21} Mat 19,28}

\bibverse{22} Entonces Jesús respondiendo, dijo: No sabéis lo que pedís:
¿podéis beber el vaso que yo he de beber, y ser bautizados del bautismo
de que yo soy bautizado? Y ellos le dicen: Podemos. \footnote{\textbf{20:22}
  Mat 26,39; Luc 12,50}

\bibverse{23} Y él les dice: A la verdad mi vaso beberéis, y del
bautismo de que yo soy bautizado, seréis bautizados; mas el sentaros á
mi mano derecha y á mi izquierda, no es mío darlo, sino á aquellos para
quienes está aparejado de mi Padre. \footnote{\textbf{20:23} Hech 12,2;
  Apoc 1,9}

\hypertarget{del-deber-de-servir-por-el-reino-de-los-cielos}{%
\subsection{Del deber de servir por el reino de los
cielos}\label{del-deber-de-servir-por-el-reino-de-los-cielos}}

\bibverse{24} Y como los diez oyeron esto, se enojaron de los dos
hermanos. \footnote{\textbf{20:24} Luc 22,24-26}

\bibverse{25} Entonces Jesús llamándolos, dijo: Sabéis que los príncipes
de los Gentiles se enseñorean sobre ellos, y los que son grandes ejercen
sobre ellos potestad. \bibverse{26} Mas entre vosotros no será así; sino
el que quisiere entre vosotros hacerse grande, será vuestro servidor;
\bibverse{27} Y el que quisiere entre vosotros ser el primero, será
vuestro siervo: \footnote{\textbf{20:27} Mar 9,35} \bibverse{28} Como el
Hijo del hombre no vino para ser servido, sino para servir, y para dar
su vida en rescate por muchos. \footnote{\textbf{20:28} Luc 22,27; Fil
  2,7; 1Pe 1,18-19}

\hypertarget{curaciuxf3n-de-dos-ciegos-cerca-de-jericuxf3}{%
\subsection{Curación de dos ciegos cerca de
Jericó}\label{curaciuxf3n-de-dos-ciegos-cerca-de-jericuxf3}}

\bibverse{29} Entonces saliendo ellos de Jericó, le seguía gran
compañía. \bibverse{30} Y he aquí dos ciegos sentados junto al camino,
como oyeron que Jesús pasaba, clamaron, diciendo: Señor, Hijo de David,
ten misericordia de nosotros. \bibverse{31} Y la gente les reñía para
que callasen; mas ellos clamaban más, diciendo: Señor, Hijo de David,
ten misericordia de nosotros.

\bibverse{32} Y parándose Jesús, los llamó, y dijo: ¿Qué queréis que
haga por vosotros?

\bibverse{33} Ellos le dicen: Señor, que sean abiertos nuestros ojos.

\bibverse{34} Entonces Jesús, teniendo misericordia de ellos, les tocó
los ojos, y luego sus ojos recibieron la vista; y le siguieron.

\hypertarget{entrada-de-jesuxfas-a-jerusaluxe9n}{%
\subsection{Entrada de Jesús a
Jerusalén}\label{entrada-de-jesuxfas-a-jerusaluxe9n}}

\hypertarget{section-20}{%
\section{21}\label{section-20}}

\bibverse{1} Y como se acercaron á Jerusalem, y vinieron á Bethfagé, al
monte de las Olivas, entonces Jesús envió dos discípulos, \bibverse{2}
Diciéndoles: Id á la aldea que está delante de vosotros, y luego
hallaréis una asna atada, y un pollino con ella: desatadla, y
traédmelos. \bibverse{3} Y si alguno os dijere algo, decid: El Señor los
ha menester. Y luego los dejará. \footnote{\textbf{21:3} Mat 26,18}

\bibverse{4} Y todo esto fué hecho, para que se cumpliese lo que fué
dicho por el profeta, que dijo: \bibverse{5} Decid á la hija de Sión: He
aquí, tu Rey viene á ti, manso, y sentado sobre una asna, y sobre un
pollino, hijo de animal de yugo.

\bibverse{6} Y los discípulos fueron, é hicieron como Jesús les mandó;
\bibverse{7} Y trajeron el asna y el pollino, y pusieron sobre ellos sus
mantos; y se sentó sobre ellos. \bibverse{8} Y la compañía, que era muy
numerosa, tendía sus mantos en el camino: y otros cortaban ramos de los
árboles, y los tendían por el camino. \bibverse{9} Y las gentes que iban
delante, y las que iban detrás, aclamaban diciendo: ¡Hosanna al Hijo de
David! ¡Bendito el que viene en el nombre del Señor! ¡Hosanna en las
alturas! \footnote{\textbf{21:9} Sal 118,25-26}

\bibverse{10} Y entrando él en Jerusalem, toda la ciudad se alborotó,
diciendo: ¿Quién es éste?

\bibverse{11} Y las gentes decían: Este es Jesús, el profeta, de Nazaret
de Galilea.

\hypertarget{la-limpieza-del-templo}{%
\subsection{La limpieza del templo}\label{la-limpieza-del-templo}}

\bibverse{12} Y entró Jesús en el templo de Dios, y echó fuera todos los
que vendían y compraban en el templo, y trastornó las mesas de los
cambiadores, y las sillas de los que vendían palomas; \bibverse{13} Y
les dice: Escrito está: Mi casa, casa de oración será llamada; mas
vosotros cueva de ladrones la habéis hecho.

\hypertarget{sanaciones-en-el-templo-y-homenaje-a-los-niuxf1os}{%
\subsection{Sanaciones en el templo y homenaje a los
niños}\label{sanaciones-en-el-templo-y-homenaje-a-los-niuxf1os}}

\bibverse{14} Entonces vinieron á él ciegos y cojos en el templo, y los
sanó. \bibverse{15} Mas los príncipes de los sacerdotes y los escribas,
viendo las maravillas que hacía, y á los muchachos aclamando en el
templo y diciendo: ¡Hosanna al Hijo de David! se indignaron,
\bibverse{16} Y le dijeron: ¿Oyes lo que éstos dicen? Y Jesús les dice:
Sí: ¿nunca leísteis: De la boca de los niños y de los que maman
perfeccionaste la alabanza?

\bibverse{17} Y dejándolos, se salió fuera de la ciudad, á Bethania; y
posó allí.

\hypertarget{la-maldiciuxf3n-de-la-higuera-estuxe9ril}{%
\subsection{La maldición de la higuera
estéril}\label{la-maldiciuxf3n-de-la-higuera-estuxe9ril}}

\bibverse{18} Y por la mañana volviendo á la ciudad, tuvo hambre.
\bibverse{19} Y viendo una higuera cerca del camino, vino á ella, y no
halló nada en ella, sino hojas solamente; y le dijo: Nunca más para
siempre nazca de ti fruto. Y luego se secó la higuera. \footnote{\textbf{21:19}
  Luc 13,6}

\bibverse{20} Y viendo esto los discípulos, maravillados decían: ¿Cómo
se secó luego la higuera?

\bibverse{21} Y respondiendo Jesús les dijo: De cierto os digo, que si
tuviereis fe, y no dudareis, no sólo haréis esto de la higuera: mas si á
este monte dijereis: Quítate y échate en la mar, será hecho.
\bibverse{22} Y todo lo que pidiereis en oración, creyendo, lo
recibiréis.

\hypertarget{la-pregunta-del-sumo-consejo-sobre-la-autoridad-de-jesuxfas}{%
\subsection{La pregunta del sumo consejo sobre la autoridad de
Jesús}\label{la-pregunta-del-sumo-consejo-sobre-la-autoridad-de-jesuxfas}}

\bibverse{23} Y como vino al templo, llegáronse á él cuando estaba
enseñando, los príncipes de los sacerdotes y los ancianos del pueblo,
diciendo: ¿Con qué autoridad haces esto? ¿y quién te dió esta autoridad?
\footnote{\textbf{21:23} Juan 2,18; Hech 4,7}

\bibverse{24} Y respondiendo Jesús, les dijo: Yo también os preguntaré
una palabra, la cual si me dijereis, también yo os diré con qué
autoridad hago esto. \bibverse{25} El bautismo de Juan, ¿de dónde era?
¿del cielo, ó de los hombres? Ellos entonces pensaron entre sí,
diciendo: Si dijéremos, del cielo, nos dirá: ¿Por qué pues no le
creísteis?

\bibverse{26} Y si dijéremos, de los hombres, tememos al pueblo; porque
todos tienen á Juan por profeta. \bibverse{27} Y respondiendo á Jesús,
dijeron: No sabemos. Y él también les dijo: Ni yo os digo con qué
autoridad hago esto.

\hypertarget{la-paruxe1bola-de-los-dos-hijos-desiguales}{%
\subsection{La parábola de los dos hijos
desiguales}\label{la-paruxe1bola-de-los-dos-hijos-desiguales}}

\bibverse{28} Mas, ¿qué os parece? Un hombre tenía dos hijos, y llegando
al primero, le dijo: Hijo, ve hoy á trabajar en mi viña. \bibverse{29} Y
respondiendo él, dijo: No quiero; mas después, arrepentido, fué.
\bibverse{30} Y llegando al otro, le dijo de la misma manera; y
respondiendo él, dijo: Yo, señor, voy. Y no fué. \footnote{\textbf{21:30}
  Mat 7,21} \bibverse{31} ¿Cuál de los dos hizo la voluntad de su padre?
Dicen ellos: El primero. Díceles Jesús: De cierto os digo, que los
publicanos y las rameras os van delante al reino de Dios. \footnote{\textbf{21:31}
  Luc 18,14}

\bibverse{32} Porque vino á vosotros Juan en camino de justicia, y no le
creísteis; y los publicanos y las rameras le creyeron; y vosotros,
viendo esto, no os arrepentisteis después para creerle. \footnote{\textbf{21:32}
  Luc 7,29}

\hypertarget{la-paruxe1bola-de-los-viticultores-infieles}{%
\subsection{La parábola de los viticultores
infieles}\label{la-paruxe1bola-de-los-viticultores-infieles}}

\bibverse{33} Oid otra parábola: Fué un hombre, padre de familia, el
cual plantó una viña; y la cercó de vallado, y cavó en ella un lagar, y
edificó una torre, y la dió á renta á labradores, y se partió lejos.
\footnote{\textbf{21:33} Is 5,1-2} \bibverse{34} Y cuando se acercó el
tiempo de los frutos, envió sus siervos á los labradores, para que
recibiesen sus frutos. \bibverse{35} Mas los labradores, tomando á los
siervos, al uno hirieron, y al otro mataron, y al otro apedrearon.
\bibverse{36} Envió de nuevo otros siervos, más que los primeros; é
hicieron con ellos de la misma manera. \bibverse{37} Y á la postre les
envió su hijo, diciendo: Tendrán respeto á mi hijo. \bibverse{38} Mas
los labradores, viendo al hijo, dijeron entre sí: Este es el heredero;
venid, matémosle, y tomemos su heredad. \footnote{\textbf{21:38} Mat
  26,3-5; Juan 1,11} \bibverse{39} Y tomado, le echaron fuera de la
viña, y le mataron. \bibverse{40} Pues cuando viniere el señor de la
viña, ¿qué hará á aquellos labradores?

\bibverse{41} Dícenle: á los malos destruirá miserablemente, y su viña
dará á renta á otros labradores, que le paguen el fruto á sus tiempos.

\bibverse{42} Díceles Jesús: ¿Nunca leísteis en las Escrituras: La
piedra que desecharon los que edificaban, ésta fué hecha por cabeza de
esquina: por el Señor es hecho esto, y es cosa maravillosa en nuestros
ojos?

\bibverse{43} Por tanto os digo, que el reino de Dios será quitado de
vosotros, y será dado á gente que haga los frutos de él. \bibverse{44} Y
el que cayere sobre esta piedra, será quebrantado; y sobre quien ella
cayere, le desmenuzará. \footnote{\textbf{21:44} Dan 2,34-35; Dan
  2,44-45}

\bibverse{45} Y oyendo los príncipes de los sacerdotes y los Fariseos
sus parábolas, entendieron que hablaba de ellos. \bibverse{46} Y
buscando cómo echarle mano, temieron al pueblo; porque le tenían por
profeta.

\hypertarget{la-paruxe1bola-de-la-cena-de-bodas-real}{%
\subsection{La parábola de la cena de bodas
real}\label{la-paruxe1bola-de-la-cena-de-bodas-real}}

\hypertarget{section-21}{%
\section{22}\label{section-21}}

\bibverse{1} Y respondiendo Jesús, les volvió á hablar en parábolas,
diciendo: \bibverse{2} El reino de los cielos es semejante á un hombre
rey, que hizo bodas á su hijo; \bibverse{3} Y envió sus siervos para que
llamasen los llamados á las bodas; mas no quisieron venir. \bibverse{4}
Volvió á enviar otros siervos, diciendo: Decid á los llamados: He aquí,
mi comida he aparejado; mis toros y animales engordados son muertos, y
todo está prevenido: venid á las bodas. \bibverse{5} Mas ellos no se
cuidaron, y se fueron, uno á su labranza, y otro á sus negocios;
\bibverse{6} Y otros, tomando á sus siervos, los afrentaron y los
mataron. \footnote{\textbf{22:6} Mat 21,35} \bibverse{7} Y el rey,
oyendo esto, se enojó; y enviando sus ejércitos, destruyó á aquellos
homicidas, y puso fuego á su ciudad. \footnote{\textbf{22:7} Mat 24,2}

\bibverse{8} Entonces dice á sus siervos: Las bodas á la verdad están
aparejadas; mas los que eran llamados no eran dignos. \bibverse{9} Id
pues á las salidas de los caminos, y llamad á las bodas á cuantos
hallareis. \footnote{\textbf{22:9} Mat 13,47} \bibverse{10} Y saliendo
los siervos por los caminos, juntaron á todos los que hallaron,
juntamente malos y buenos: y las bodas fueron llenas de convidados.

\bibverse{11} Y entró el rey para ver los convidados, y vió allí un
hombre no vestido de boda. \bibverse{12} Y le dijo: Amigo, ¿cómo
entraste aquí no teniendo vestido de boda? Mas él cerró la boca.
\bibverse{13} Entonces el rey dijo á los que servían: Atado de pies y de
manos tomadle, y echadle en las tinieblas de afuera: allí será el lloro
y el crujir de dientes. \bibverse{14} Porque muchos son llamados, y
pocos escogidos.

\hypertarget{la-cuestiuxf3n-fiscal-de-los-fariseos}{%
\subsection{La cuestión fiscal de los
fariseos}\label{la-cuestiuxf3n-fiscal-de-los-fariseos}}

\bibverse{15} Entonces, idos los Fariseos, consultaron cómo le tomarían
en alguna palabra. \bibverse{16} Y envían á él los discípulos de ellos,
con los Herodianos, diciendo: Maestro, sabemos que eres amador de la
verdad, y que enseñas con verdad el camino de Dios, y que no te curas de
nadie, porque no tienes acepción de persona de hombres. \footnote{\textbf{22:16}
  Juan 3,2} \bibverse{17} Dinos pues, ¿qué te parece? ¿es lícito dar
tributo á César, ó no?

\bibverse{18} Mas Jesús, entendida la malicia de ellos, les dice: ¿Por
qué me tentáis, hipócritas? \bibverse{19} Mostradme la moneda del
tributo. Y ellos le presentaron un denario.

\bibverse{20} Entonces les dice: ¿Cúya es esta figura, y lo que está
encima escrito?

\bibverse{21} Dícenle: De César. Y díceles: Pagad pues á César lo que es
de César, y á Dios lo que es de Dios.

\bibverse{22} Y oyendo esto, se maravillaron; y dejándole se fueron.

\hypertarget{sobre-la-resurrecciuxf3n-de-los-muertos}{%
\subsection{Sobre la resurrección de los
muertos}\label{sobre-la-resurrecciuxf3n-de-los-muertos}}

\bibverse{23} Aquel día llegaron á él los Saduceos, que dicen no haber
resurrección, y le preguntaron, \footnote{\textbf{22:23} Hech 4,2; Hech
  23,6; Hech 23,8} \bibverse{24} Diciendo: Maestro, Moisés dijo: Si
alguno muriere sin hijos, su hermano se casará con su mujer, y
despertará simiente á su hermano. \bibverse{25} Fueron pues, entre
nosotros siete hermanos: y el primero tomó mujer, y murió; y no teniendo
generación, dejó su mujer á su hermano. \bibverse{26} De la misma manera
también el segundo, y el tercero, hasta los siete. \bibverse{27} Y
después de todos murió también la mujer. \bibverse{28} En la
resurrección pues, ¿de cuál de los siete será ella mujer? porque todos
la tuvieron.

\bibverse{29} Entonces respondiendo Jesús, les dijo: Erráis ignorando
las Escrituras, y el poder de Dios. \bibverse{30} Porque en la
resurrección, ni los hombres tomarán mujeres, ni las mujeres maridos;
mas son como los ángeles de Dios en el cielo. \bibverse{31} Y de la
resurrección de los muertos, ¿no habéis leído lo que os es dicho por
Dios, que dice: \bibverse{32} Yo soy el Dios de Abraham, y el Dios de
Isaac, y el Dios de Jacob? Dios no es Dios de muertos, sino de vivos.

\bibverse{33} Y oyendo esto las gentes, estaban atónitas de su doctrina.

\hypertarget{la-pregunta-de-un-intuxe9rprete-de-la-ley-sobre-el-mandamiento-muxe1s-noble}{%
\subsection{La pregunta de un intérprete de la ley sobre el mandamiento
más
noble}\label{la-pregunta-de-un-intuxe9rprete-de-la-ley-sobre-el-mandamiento-muxe1s-noble}}

\bibverse{34} Entonces los Fariseos, oyendo que había cerrado la boca á
los Saduceos, se juntaron á una. \bibverse{35} Y preguntó uno de ellos,
intérprete de la ley, tentándole y diciendo: \bibverse{36} Maestro,
¿cuál es el mandamiento grande en la ley?

\bibverse{37} Y Jesús le dijo: Amarás al Señor tu Dios de todo tu
corazón, y de toda tu alma, y de toda tu mente. \bibverse{38} Este es el
primero y el grande mandamiento. \bibverse{39} Y el segundo es semejante
á éste: Amarás á tu prójimo como á ti mismo. \bibverse{40} De estos dos
mandamientos depende toda la ley y los profetas.

\hypertarget{la-contrapregunta-de-jesuxfas-sobre-el-mesuxedas-como-hijo-de-david}{%
\subsection{La contrapregunta de Jesús sobre el Mesías como hijo de
David}\label{la-contrapregunta-de-jesuxfas-sobre-el-mesuxedas-como-hijo-de-david}}

\bibverse{41} Y estando juntos los Fariseos, Jesús les preguntó,
\bibverse{42} Diciendo: ¿Qué os parece del Cristo? ¿de quién es Hijo?
Dícenle: De David. \footnote{\textbf{22:42} Is 11,1; Juan 7,42}

\bibverse{43} El les dice: ¿Pues cómo David en Espíritu le llama Señor,
diciendo: \bibverse{44} Dijo el Señor á mi Señor: Siéntate á mi diestra,
entre tanto que pongo tus enemigos por estrado de tus pies?

\bibverse{45} Pues si David le llama Señor, ¿cómo es su Hijo?

\bibverse{46} Y nadie le podía responder palabra; ni osó alguno desde
aquel día preguntarle más.

\hypertarget{el-gran-discurso-de-castigo-de-jesuxfas-contra-los-escribas-y-fariseos}{%
\subsection{El gran discurso de castigo de Jesús contra los escribas y
fariseos}\label{el-gran-discurso-de-castigo-de-jesuxfas-contra-los-escribas-y-fariseos}}

\hypertarget{section-22}{%
\section{23}\label{section-22}}

\bibverse{1} Entonces habló Jesús á las gentes y á sus discípulos,

\hypertarget{reprimenda-por-el-comportamiento-reprobable-de-los-luxedderes-espirituales-del-pueblo-en-su-alto-cargo}{%
\subsection{Reprimenda por el comportamiento reprobable de los líderes
espirituales del pueblo en su alto
cargo}\label{reprimenda-por-el-comportamiento-reprobable-de-los-luxedderes-espirituales-del-pueblo-en-su-alto-cargo}}

\bibverse{2} Diciendo: Sobre la cátedra de Moisés se sentaron los
escribas y los Fariseos: \bibverse{3} Así que, todo lo que os dijeren
que guardéis, guardadlo y hacedlo; mas no hagáis conforme á sus obras:
porque dicen, y no hacen. \footnote{\textbf{23:3} Mal 2,7-8; Rom 2,21-23}
\bibverse{4} Porque atan cargas pesadas y difíciles de llevar, y las
ponen sobre los hombros de los hombres; mas ni aun con su dedo las
quieren mover. \footnote{\textbf{23:4} Mat 11,28-30; Hech 15,10; Hech
  15,28} \bibverse{5} Antes, todas sus obras hacen para ser mirados de
los hombres; porque ensanchan sus filacterias, y extienden los flecos de
sus mantos; \footnote{\textbf{23:5} Mat 6,1; Éxod 13,9; Núm 15,38-39}
\bibverse{6} Y aman los primeros asientos en las cenas, y las primeras
sillas en las sinagogas; \footnote{\textbf{23:6} Luc 14,7} \bibverse{7}
Y las salutaciones en las plazas, y ser llamados de los hombres Rabbí,
Rabbí. \bibverse{8} Mas vosotros, no queráis ser llamados Rabbí; porque
uno es vuestro Maestro, el Cristo; y todos vosotros sois hermanos.
\bibverse{9} Y vuestro padre no llaméis á nadie en la tierra; porque uno
es vuestro Padre, el cual está en los cielos. \bibverse{10} Ni seáis
llamados maestros; porque uno es vuestro Maestro, el Cristo.
\bibverse{11} El que es el mayor de vosotros, sea vuestro siervo.
\footnote{\textbf{23:11} Mat 20,26-27} \bibverse{12} Porque el que se
ensalzare, será humillado; y el que se humillare, será ensalzado.
\footnote{\textbf{23:12} Prov 29,23; Job 22,29; Ezeq 21,31; Luc 18,14;
  1Pe 5,5}

\hypertarget{los-siete-ayes-de-los-escribas-y-fariseos}{%
\subsection{Los siete ayes de los escribas y
fariseos}\label{los-siete-ayes-de-los-escribas-y-fariseos}}

\bibverse{13} Mas ¡ay de vosotros, escribas y Fariseos, hipócritas!
porque cerráis el reino de los cielos delante de los hombres; que ni
vosotros entráis, ni á los que están entrando dejáis entrar.

\bibverse{14} ¡Ay de vosotros, escribas y Fariseos, hipócritas! porque
coméis las casas de las viudas, y por pretexto hacéis larga oración: por
esto llevaréis más grave juicio. \footnote{\textbf{23:14} Ezeq 22,25}
\bibverse{15} ¡Ay de vosotros, escribas y Fariseos, hipócritas! porque
rodeáis la mar y la tierra por hacer un prosélito; y cuando fuere hecho,
le hacéis hijo del infierno doble más que vosotros.

\bibverse{16} ¡Ay de vosotros, guías ciegos! que decís: Cualquiera que
jurare por el templo es nada; mas cualquiera que jurare por el oro del
templo, deudor es. \bibverse{17} ¡Insensatos y ciegos! porque ¿cuál es
mayor, el oro, ó el templo que santifica al oro? \bibverse{18} Y:
Cualquiera que jurare por el altar, es nada; mas cualquiera que jurare
por el presente que está sobre él, deudor es. \bibverse{19} ¡Necios y
ciegos! porque, ¿cuál es mayor, el presente, ó el altar que santifica al
presente? \footnote{\textbf{23:19} Éxod 29,37} \bibverse{20} Pues el que
jurare por el altar, jura por él, y por todo lo que está sobre él;
\bibverse{21} Y el que jurare por el templo, jura por él, y por Aquél
que habita en él; \bibverse{22} Y el que jura por el cielo, jura por el
trono de Dios, y por Aquél que está sentado sobre él.

\bibverse{23} ¡Ay de vosotros, escribas y Fariseos, hipócritas! porque
diezmáis la menta y el eneldo y el comino, y dejasteis lo que es lo más
grave de la ley, es á saber, el juicio y la misericordia y la fe: esto
era menester hacer, y no dejar lo otro. \bibverse{24} ¡Guías ciegos, que
coláis el mosquito, mas tragáis el camello!

\bibverse{25} ¡Ay de vosotros, escribas y Fariseos, hipócritas! porque
limpiáis lo que está de fuera del vaso y del plato; mas de dentro están
llenos de robo y de injusticia. \footnote{\textbf{23:25} Mar 7,4; Mar
  7,8} \bibverse{26} ¡Fariseo ciego, limpia primero lo de dentro del
vaso y del plato, para que también lo de fuera se haga limpio!
\footnote{\textbf{23:26} Juan 9,40; Tit 1,15}

\bibverse{27} ¡Ay de vosotros, escribas y Fariseos, hipócritas! porque
sois semejantes á sepulcros blanqueados, que de fuera, á la verdad, se
muestran hermosos, mas de dentro están llenos de huesos de muertos y de
toda suciedad. \bibverse{28} Así también vosotros de fuera, á la verdad,
os mostráis justos á los hombres; mas de dentro, llenos estáis de
hipocresía é iniquidad.

\bibverse{29} ¡Ay de vosotros, escribas y Fariseos, hipócritas! porque
edificáis los sepulcros de los profetas, y adornáis los monumentos de
los justos, \bibverse{30} Y decís: Si fuéramos en los días de nuestros
padres, no hubiéramos sido sus compañeros en la sangre de los profetas.
\bibverse{31} Así que, testimonio dais á vosotros mismos, que sois hijos
de aquellos que mataron á los profetas. \footnote{\textbf{23:31} Mat
  5,12; Hech 7,52} \bibverse{32} ¡Vosotros también henchid la medida de
vuestros padres! \bibverse{33} ¡Serpientes, generación de víboras! ¿cómo
evitaréis el juicio del infierno?

\hypertarget{amenaza-contra-las-personas-manchadas-de-sangre-que-se-resisten-a-su-salvaciuxf3n}{%
\subsection{Amenaza contra las personas manchadas de sangre que se
resisten a su
salvación}\label{amenaza-contra-las-personas-manchadas-de-sangre-que-se-resisten-a-su-salvaciuxf3n}}

\bibverse{34} Por tanto, he aquí, yo envío á vosotros profetas, y
sabios, y escribas: y de ellos, á unos mataréis y crucificaréis, y á
otros de ellos azotaréis en vuestras sinagogas, y perseguiréis de ciudad
en ciudad: \bibverse{35} Para que venga sobre vosotros toda la sangre
justa que se ha derramado sobre la tierra, desde la sangre de Abel el
justo, hasta la sangre de Zacarías, hijo de Barachîas, al cual matasteis
entre el templo y el altar. \bibverse{36} De cierto os digo que todo
esto vendrá sobre esta generación.

\hypertarget{salida-de-jesuxfas-de-la-ciudad-de-jerusaluxe9n-y-anuncio-de-su-regreso}{%
\subsection{Salida de Jesús de la ciudad de Jerusalén y anuncio de su
regreso}\label{salida-de-jesuxfas-de-la-ciudad-de-jerusaluxe9n-y-anuncio-de-su-regreso}}

\bibverse{37} ¡Jerusalem, Jerusalem, que matas á los profetas, y
apedreas á los que son enviados á ti! ¡cuántas veces quise juntar tus
hijos, como la gallina junta sus pollos debajo de las alas, y no
quisiste! \bibverse{38} He aquí vuestra casa os es dejada desierta.
\footnote{\textbf{23:38} 1Re 9,7-8} \bibverse{39} Porque os digo que
desde ahora no me veréis, hasta que digáis: Bendito el que viene en el
nombre del Señor. \footnote{\textbf{23:39} Mat 21,9; Mat 26,64}

\hypertarget{el-monte-de-los-olivos-de-jesuxfas-a-sus-discuxedpulos-desde-la-destrucciuxf3n-del-templo-desde-el-fin-de-este-mundo-desde-su-segunda-venida-y-desde-el-juicio-sobre-los-pueblos}{%
\subsection{El monte de los Olivos de Jesús a sus discípulos desde la
destrucción del templo, desde el fin de este mundo, desde su segunda
venida y desde el juicio sobre los
pueblos}\label{el-monte-de-los-olivos-de-jesuxfas-a-sus-discuxedpulos-desde-la-destrucciuxf3n-del-templo-desde-el-fin-de-este-mundo-desde-su-segunda-venida-y-desde-el-juicio-sobre-los-pueblos}}

\hypertarget{section-23}{%
\section{24}\label{section-23}}

\bibverse{1} Y salido Jesús, íbase del templo; y se llegaron sus
discípulos, para mostrarle los edificios del templo. \bibverse{2} Y
respondiendo él, les dijo: ¿Veis todo esto? de cierto os digo, que no
será dejada aquí piedra sobre piedra, que no sea destruída. \footnote{\textbf{24:2}
  Luc 19,44}

\bibverse{3} Y sentándose él en el monte de las Olivas, se llegaron á él
los discípulos aparte, diciendo: Dinos, ¿cuándo serán estas cosas, y qué
señal habrá de tu venida, y del fin del mundo? \footnote{\textbf{24:3}
  Hech 1,6-8}

\hypertarget{el-fin-de-este-tiempo-mundial-las-primeras-seuxf1ales}{%
\subsection{El fin de este tiempo mundial; Las primeras
señales}\label{el-fin-de-este-tiempo-mundial-las-primeras-seuxf1ales}}

\bibverse{4} Y respondiendo Jesús, les dijo: Mirad que nadie os engañe.
\bibverse{5} Porque vendrán muchos en mi nombre, diciendo: Yo soy el
Cristo; y á muchos engañarán. \footnote{\textbf{24:5} Juan 5,43; 1Jn
  2,18} \bibverse{6} Y oiréis guerras, y rumores de guerras: mirad que
no os turbéis; porque es menester que todo esto acontezca; mas aún no es
el fin. \bibverse{7} Porque se levantará nación contra nación, y reino
contra reino; y habrá pestilencias, y hambres, y terremotos por los
lugares. \bibverse{8} Y todas estas cosas, principio de dolores.

\hypertarget{las-persecuciones-de-los-discuxedpulos}{%
\subsection{Las persecuciones de los
discípulos}\label{las-persecuciones-de-los-discuxedpulos}}

\bibverse{9} Entonces os entregarán para ser afligidos, y os matarán; y
seréis aborrecidos de todas las gentes por causa de mi nombre.
\bibverse{10} Y muchos entonces serán escandalizados; y se entregarán
unos á otros, y unos á otros se aborrecerán. \bibverse{11} Y muchos
falsos profetas se levantarán y engañarán á muchos. \footnote{\textbf{24:11}
  2Pe 2,1; 1Jn 4,1} \bibverse{12} Y por haberse multiplicado la maldad,
la caridad de muchos se resfriará. \footnote{\textbf{24:12} 2Tim 3,1-5}
\bibverse{13} Mas el que perseverare hasta el fin, éste será salvo.
\footnote{\textbf{24:13} Apoc 13,10} \bibverse{14} Y será predicado este
evangelio del reino en todo el mundo, por testimonio á todos los
Gentiles; y entonces vendrá el fin. \footnote{\textbf{24:14} Mat 28,19}

\hypertarget{el-cluxedmax-de-la-tribulaciuxf3n-en-judea}{%
\subsection{El clímax de la tribulación en
Judea}\label{el-cluxedmax-de-la-tribulaciuxf3n-en-judea}}

\bibverse{15} Por tanto, cuando viereis la abominación del asolamiento,
que fué dicha por Daniel profeta, que estará en el lugar santo, (el que
lee, entienda), \bibverse{16} Entonces los que están en Judea, huyan á
los montes; \bibverse{17} Y el que sobre el terrado, no descienda á
tomar algo de su casa; \footnote{\textbf{24:17} Luc 17,31} \bibverse{18}
Y el que en el campo, no vuelva atrás á tomar sus vestidos.
\bibverse{19} Mas ¡ay de las preñadas, y de las que crían en aquellos
días! \bibverse{20} Orad, pues, que vuestra huída no sea en invierno ni
en sábado; \bibverse{21} Porque habrá entonces grande aflicción, cual no
fué desde el principio del mundo hasta ahora, ni será. \footnote{\textbf{24:21}
  Dan 12,1} \bibverse{22} Y si aquellos días no fuesen acortados,
ninguna carne sería salva; mas por causa de los escogidos, aquellos días
serán acortados.

\hypertarget{profecuxeda-de-los-falsos-profetas}{%
\subsection{Profecía de los falsos
profetas}\label{profecuxeda-de-los-falsos-profetas}}

\bibverse{23} Entonces, si alguno os dijere: He aquí está el Cristo, ó
allí, no creáis. \bibverse{24} Porque se levantarán falsos Cristos, y
falsos profetas, y darán señales grandes y prodigios; de tal manera que
engañarán, si es posible, aun á los escogidos.

\bibverse{25} He aquí os lo he dicho antes.

\bibverse{26} Así que, si os dijeren: He aquí en el desierto está; no
salgáis: He aquí en las cámaras; no creáis. \bibverse{27} Porque como el
relámpago que sale del oriente y se muestra hasta el occidente, así será
también la venida del Hijo del hombre. \footnote{\textbf{24:27} Luc
  17,23-24} \bibverse{28} Porque donde quiera que estuviere el cuerpo
muerto, allí se juntarán las águilas. \footnote{\textbf{24:28} Job
  39,30; Luc 17,37; Apoc 19,17-18}

\hypertarget{los-uxfaltimos-augurios-y-la-segunda-venida-del-hijo-del-hombre-con-los-fenuxf3menos-que-los-acompauxf1an}{%
\subsection{Los últimos augurios y la segunda venida del Hijo del Hombre
con los fenómenos que los
acompañan}\label{los-uxfaltimos-augurios-y-la-segunda-venida-del-hijo-del-hombre-con-los-fenuxf3menos-que-los-acompauxf1an}}

\bibverse{29} Y luego después de la aflicción de aquellos días, el sol
se obscurecerá, y la luna no dará su lumbre, y las estrellas caerán del
cielo, y las virtudes de los cielos serán conmovidas. \footnote{\textbf{24:29}
  Is 13,10; 2Pe 3,10; Apoc 6,12-13} \bibverse{30} Y entonces se mostrará
la señal del Hijo del hombre en el cielo; y entonces lamentarán todas
las tribus de la tierra, y verán al Hijo del hombre que vendrá sobre las
nubes del cielo, con grande poder y gloria. \footnote{\textbf{24:30} Mat
  26,64; Dan 7,13-14; Apoc 1,7; Apoc 19,11-13} \bibverse{31} Y enviará
sus ángeles con gran voz de trompeta, y juntarán sus escogidos de los
cuatro vientos, de un cabo del cielo hasta el otro. \footnote{\textbf{24:31}
  1Cor 15,52; Apoc 8,1-2}

\bibverse{32} De la higuera aprended la parábola: Cuando ya su rama se
enternece, y las hojas brotan, sabéis que el verano está cerca.
\bibverse{33} Así también vosotros, cuando viereis todas estas cosas,
sabed que está cercano, á las puertas. \bibverse{34} De cierto os digo,
que no pasará esta generación, que todas estas cosas no acontezcan.
\bibverse{35} El cielo y la tierra pasarán, mas mis palabras no pasarán.

\bibverse{36} Empero del día y hora nadie sabe, ni aun los ángeles de
los cielos, sino mi Padre solo. \footnote{\textbf{24:36} Hech 1,7}
\bibverse{37} Mas como los días de Noé, así será la venida del Hijo del
hombre. \footnote{\textbf{24:37} Gén 6,11-13; Luc 17,26-27}
\bibverse{38} Porque como en los días antes del diluvio estaban comiendo
y bebiendo, casándose y dando en casamiento, hasta el día que Noé entró
en el arca, \bibverse{39} Y no conocieron hasta que vino el diluvio y
llevó á todos, así será también la venida del Hijo del hombre.
\bibverse{40} Entonces estarán dos en el campo; el uno será tomado, y el
otro será dejado: \footnote{\textbf{24:40} Luc 17,35-36} \bibverse{41}
Dos mujeres moliendo á un molinillo; la una será tomada, y la otra será
dejada.

\hypertarget{advertencia-de-estar-alerta-en-general}{%
\subsection{Advertencia de estar alerta en
general}\label{advertencia-de-estar-alerta-en-general}}

\bibverse{42} Velad pues, porque no sabéis á qué hora ha de venir
vuestro Señor. \bibverse{43} Esto empero sabed, que si el padre de la
familia supiese á cuál vela el ladrón había de venir, velaría, y no
dejaría minar su casa. \bibverse{44} Por tanto, también vosotros estad
apercibidos; porque el Hijo del hombre ha de venir á la hora que no
pensáis. \footnote{\textbf{24:44} 1Tes 5,2}

\hypertarget{paruxe1bola-del-siervo-fiel-y-del-infiel}{%
\subsection{Parábola del siervo fiel y del
infiel}\label{paruxe1bola-del-siervo-fiel-y-del-infiel}}

\bibverse{45} ¿Quién pues es el siervo fiel y prudente, al cual puso su
señor sobre su familia para que les dé alimento á tiempo? \bibverse{46}
Bienaventurado aquel siervo, al cual, cuando su señor viniere, le
hallare haciendo así. \bibverse{47} De cierto os digo, que sobre todos
sus bienes le pondrá. \bibverse{48} Y si aquel siervo malo dijere en su
corazón: Mi señor se tarda en venir: \footnote{\textbf{24:48} 2Pe 3,4}

\bibverse{49} Y comenzare á herir á sus consiervos, y aun á comer y á
beber con los borrachos; \bibverse{50} Vendrá el señor de aquel siervo
en el día que no espera, y á la hora que no sabe, \bibverse{51} Y le
cortará por medio, y pondrá su parte con los hipócritas: allí será el
lloro y el crujir de dientes.

\hypertarget{la-paruxe1bola-de-las-diez-vuxedrgenes-prudentes-y-necias}{%
\subsection{La parábola de las diez vírgenes prudentes y
necias}\label{la-paruxe1bola-de-las-diez-vuxedrgenes-prudentes-y-necias}}

\hypertarget{section-24}{%
\section{25}\label{section-24}}

\bibverse{1} Entonces el reino de los cielos será semejante á diez
vírgenes, que tomando sus lámparas, salieron á recibir al esposo.
\bibverse{2} Y las cinco de ellas eran prudentes, y las cinco fatuas.
\bibverse{3} Las que eran fatuas, tomando sus lámparas, no tomaron
consigo aceite; \bibverse{4} Mas las prudentes tomaron aceite en sus
vasos, juntamente con sus lámparas. \bibverse{5} Y tardándose el esposo,
cabecearon todas, y se durmieron. \bibverse{6} Y á la media noche fué
oído un clamor: He aquí, el esposo viene; salid á recibirle.
\bibverse{7} Entonces todas aquellas vírgenes se levantaron, y
aderezaron sus lámparas. \bibverse{8} Y las fatuas dijeron á las
prudentes: Dadnos de vuestro aceite; porque nuestras lámparas se apagan.
\bibverse{9} Mas las prudentes respondieron, diciendo: Porque no nos
falte á nosotras y á vosotras, id antes á los que venden, y comprad para
vosotras. \bibverse{10} Y mientras que ellas iban á comprar, vino el
esposo; y las que estaban apercibidas, entraron con él á las bodas; y se
cerró la puerta. \bibverse{11} Y después vinieron también las otras
vírgenes, diciendo: Señor, Señor, ábrenos. \footnote{\textbf{25:11} Luc
  13,25; Luc 13,27} \bibverse{12} Mas respondiendo él, dijo: De cierto
os digo, que no os conozco. \footnote{\textbf{25:12} Mat 7,23}
\bibverse{13} Velad, pues, porque no sabéis el día ni la hora en que el
Hijo del hombre ha de venir. \footnote{\textbf{25:13} Mat 24,42}

\hypertarget{paruxe1bola-de-los-talentos-confiados}{%
\subsection{Parábola de los talentos
confiados}\label{paruxe1bola-de-los-talentos-confiados}}

\bibverse{14} Porque el reino de los cielos es como un hombre que
partiéndose lejos llamó á sus siervos, y les entregó sus bienes.
\bibverse{15} Y á éste dió cinco talentos, y al otro dos, y al otro uno:
á cada uno conforme á su facultad; y luego se partió lejos.
\bibverse{16} Y el que había recibido cinco talentos se fué, y granjeó
con ellos, é hizo otros cinco talentos. \bibverse{17} Asimismo el que
había recibido dos, ganó también él otros dos. \bibverse{18} Mas el que
había recibido uno, fué y cavó en la tierra, y escondió el dinero de su
señor.

\bibverse{19} Y después de mucho tiempo, vino el señor de aquellos
siervos, é hizo cuentas con ellos. \bibverse{20} Y llegando el que había
recibido cinco talentos, trajo otros cinco talentos, diciendo: Señor,
cinco talentos me entregaste; he aquí otros cinco talentos he ganado
sobre ellos.

\bibverse{21} Y su señor le dijo: Bien, buen siervo y fiel; sobre poco
has sido fiel, sobre mucho te pondré: entra en el gozo de tu señor.
\footnote{\textbf{25:21} Mat 24,45-47}

\bibverse{22} Y llegando también el que había recibido dos talentos,
dijo: Señor, dos talentos me entregaste; he aquí otros dos talentos he
ganado sobre ellos.

\bibverse{23} Su señor le dijo: Bien, buen siervo y fiel; sobre poco has
sido fiel, sobre mucho te pondré: entra en el gozo de tu señor.

\bibverse{24} Y llegando también el que había recibido un talento, dijo:
Señor, te conocía que eres hombre duro, que siegas donde no sembraste, y
recoges donde no esparciste; \bibverse{25} Y tuve miedo, y fuí, y
escondí tu talento en la tierra: he aquí tienes lo que es tuyo.

\bibverse{26} Y respondiendo su señor, le dijo: Malo y negligente
siervo, sabías que siego donde no sembré y que recojo donde no esparcí;
\bibverse{27} Por tanto te convenía dar mi dinero á los banqueros, y
viniendo yo, hubiera recibido lo que es mío con usura. \bibverse{28}
Quitadle pues el talento, y dadlo al que tiene diez talentos.
\bibverse{29} Porque á cualquiera que tuviere, le será dado, y tendrá
más; y al que no tuviere, aun lo que tiene le será quitado.
\bibverse{30} Y al siervo inútil echadle en las tinieblas de afuera:
allí será el lloro y el crujir de dientes.

\hypertarget{el-juicio-de-jesuxfas-sobre-los-pueblos-y-las-personas-la-separaciuxf3n-de-las-ovejas-de-las-cabras}{%
\subsection{El juicio de Jesús sobre los pueblos y las personas; la
separación de las ovejas de las
cabras}\label{el-juicio-de-jesuxfas-sobre-los-pueblos-y-las-personas-la-separaciuxf3n-de-las-ovejas-de-las-cabras}}

\bibverse{31} Y cuando el Hijo del hombre venga en su gloria, y todos
los santos ángeles con él, entonces se sentará sobre el trono de su
gloria. \footnote{\textbf{25:31} Mat 16,27; Apoc 20,11-13} \bibverse{32}
Y serán reunidas delante de él todas las gentes: y los apartará los unos
de los otros, como aparta el pastor las ovejas de los cabritos.
\footnote{\textbf{25:32} Mat 13,49; Rom 14,10} \bibverse{33} Y pondrá
las ovejas á su derecha, y los cabritos á la izquierda. \footnote{\textbf{25:33}
  Ezeq 34,17} \bibverse{34} Entonces el Rey dirá á los que estarán á su
derecha: Venid, benditos de mi Padre, heredad el reino preparado para
vosotros desde la fundación del mundo: \bibverse{35} Porque tuve hambre,
y me disteis de comer; tuve sed, y me disteis de beber; fuí huésped, y
me recogisteis; \bibverse{36} Desnudo, y me cubristeis; enfermo, y me
visitasteis; estuve en la cárcel, y vinisteis á mí.

\bibverse{37} Entonces los justos le responderán, diciendo: Señor,
¿cuándo te vimos hambriento, y te sustentamos? ¿ó sediento, y te dimos
de beber? \footnote{\textbf{25:37} Mat 6,3} \bibverse{38} ¿Y cuándo te
vimos huésped, y te recogimos? ¿ó desnudo, y te cubrimos? \bibverse{39}
¿O cuándo te vimos enfermo, ó en la cárcel, y vinimos á ti?

\bibverse{40} Y respondiendo el Rey, les dirá: De cierto os digo que en
cuanto lo hicisteis á uno de estos mis hermanos pequeñitos, á mí lo
hicisteis. \bibverse{41} Entonces dirá también á los que estarán á la
izquierda: Apartaos de mí, malditos, al fuego eterno preparado para el
diablo y para sus ángeles: \footnote{\textbf{25:41} Apoc 20,10; Apoc
  20,15} \bibverse{42} Porque tuve hambre, y no me disteis de comer;
tuve sed, y no me disteis de beber; \bibverse{43} Fuí huésped, y no me
recogisteis; desnudo, y no me cubristeis; enfermo, y en la cárcel, y no
me visitasteis.

\bibverse{44} Entonces también ellos le responderán, diciendo: Señor,
¿cuándo te vimos hambriento, ó sediento, ó huésped, ó desnudo, ó
enfermo, ó en la cárcel, y no te servimos?

\bibverse{45} Entonces les responderá, diciendo: De cierto os digo que
en cuanto no lo hicisteis á uno de estos pequeñitos, ni á mí lo
hicisteis. \bibverse{46} E irán éstos al tormento eterno, y los justos á
la vida eterna.

\hypertarget{uxfaltimo-anuncio-del-sufrimiento-de-jesuxfas-intento-de-asesinato-por-parte-de-los-luxedderes-del-pueblo}{%
\subsection{Último anuncio del sufrimiento de Jesús; Intento de
asesinato por parte de los líderes del
pueblo}\label{uxfaltimo-anuncio-del-sufrimiento-de-jesuxfas-intento-de-asesinato-por-parte-de-los-luxedderes-del-pueblo}}

\hypertarget{section-25}{%
\section{26}\label{section-25}}

\bibverse{1} Y aconteció que, como hubo acabado Jesús todas estas
palabras, dijo á sus discípulos: \bibverse{2} Sabéis que dentro de dos
días se hace la pascua, y el Hijo del hombre es entregado para ser
crucificado. \footnote{\textbf{26:2} Mat 20,18; Éxod 12,1-20}

\bibverse{3} Entonces los príncipes de los sacerdotes, y los escribas, y
los ancianos del pueblo se juntaron al patio del pontífice, el cual se
llamaba Caifás; \footnote{\textbf{26:3} Luc 3,1-2} \bibverse{4} Y
tuvieron consejo para prender por engaño á Jesús, y matarle.
\bibverse{5} Y decían: No en el día de la fiesta, porque no se haga
alboroto en el pueblo.

\hypertarget{unciuxf3n-de-jesuxfas-en-betania}{%
\subsection{Unción de Jesús en
Betania}\label{unciuxf3n-de-jesuxfas-en-betania}}

\bibverse{6} Y estando Jesús en Bethania, en casa de Simón el leproso,
\bibverse{7} Vino á él una mujer, teniendo un vaso de alabastro de
ungüento de gran precio, y lo derramó sobre la cabeza de él, estando
sentado á la mesa. \bibverse{8} Lo cual viendo sus discípulos, se
enojaron, diciendo: ¿Por qué se pierde esto? \bibverse{9} Porque esto se
podía vender por gran precio, y darse á los pobres.

\bibverse{10} Y entendiéndolo Jesús, les dijo: ¿Por qué dais pena á esta
mujer? pues ha hecho conmigo buena obra. \bibverse{11} Porque siempre
tendréis pobres con vosotros, mas á mí no siempre me tendréis.
\footnote{\textbf{26:11} Deut 15,11} \bibverse{12} Porque echando este
ungüento sobre mi cuerpo, para sepultarme lo ha hecho. \bibverse{13} De
cierto os digo, que donde quiera que este evangelio fuere predicado en
todo el mundo, también será dicho para memoria de ella, lo que ésta ha
hecho.

\hypertarget{traiciuxf3n-de-judas}{%
\subsection{Traición de Judas}\label{traiciuxf3n-de-judas}}

\bibverse{14} Entonces uno de los doce, que se llamaba Judas Iscariote,
fué á los príncipes de los sacerdotes, \bibverse{15} Y les dijo: ¿Qué me
queréis dar, y yo os lo entregaré? Y ellos le señalaron treinta piezas
de plata. \bibverse{16} Y desde entonces buscaba oportunidad para
entregarle.

\hypertarget{preparaciuxf3n-de-la-comida-pascual}{%
\subsection{Preparación de la comida
pascual}\label{preparaciuxf3n-de-la-comida-pascual}}

\bibverse{17} Y el primer día de la fiesta de los panes sin levadura,
vinieron los discípulos á Jesús, diciéndole: ¿Dónde quieres que
aderecemos para ti para comer la pascua? \footnote{\textbf{26:17} Éxod
  12,18-20}

\bibverse{18} Y él dijo: Id á la ciudad á cierto hombre, y decidle: El
Maestro dice: Mi tiempo está cerca; en tu casa haré la pascua con mis
discípulos. \footnote{\textbf{26:18} Mat 21,3}

\bibverse{19} Y los discípulos hicieron como Jesús les mandó, y
aderezaron la pascua.

\hypertarget{la-uxfaltima-cena-de-jesuxfas-en-el-cuxedrculo-de-los-discuxedpulos-exposiciuxf3n-de-la-traiciuxf3n-de-judas-instituciuxf3n-de-la-santa-comuniuxf3n}{%
\subsection{La última cena de Jesús en el círculo de los discípulos;
Exposición de la traición de Judas; Institución de la santa
comunión}\label{la-uxfaltima-cena-de-jesuxfas-en-el-cuxedrculo-de-los-discuxedpulos-exposiciuxf3n-de-la-traiciuxf3n-de-judas-instituciuxf3n-de-la-santa-comuniuxf3n}}

\bibverse{20} Y como fué la tarde del día, se sentó á la mesa con los
doce. \bibverse{21} Y comiendo ellos, dijo: De cierto os digo, que uno
de vosotros me ha de entregar.

\bibverse{22} Y entristecidos ellos en gran manera, comenzó cada uno de
ellos á decirle: ¿Soy yo, Señor?

\bibverse{23} Entonces él respondiendo, dijo: El que mete la mano
conmigo en el plato, ése me ha de entregar. \bibverse{24} A la verdad el
Hijo del hombre va, como está escrito de él; mas ¡ay de aquel hombre por
quien el Hijo del hombre es entregado! bueno le fuera al tal hombre no
haber nacido. \footnote{\textbf{26:24} Luc 17,1}

\bibverse{25} Entonces respondiendo Judas, que le entregaba, dijo: ¿Soy
yo, Maestro? Dícele: Tú lo has dicho.

\bibverse{26} Y comiendo ellos, tomó Jesús el pan, y bendijo, y lo
partió, y dió á sus discípulos, y dijo: Tomad, comed: esto es mi cuerpo.
\bibverse{27} Y tomando el vaso, y hechas gracias, les dió, diciendo:
Bebed de él todos; \bibverse{28} Porque esto es mi sangre del nuevo
pacto, la cual es derramada por muchos para remisión de los pecados.
\footnote{\textbf{26:28} Éxod 24,8; Jer 31,31; Heb 9,15-16}
\bibverse{29} Y os digo, que desde ahora no beberé más de este fruto de
la vid, hasta aquel día, cuando lo tengo de beber nuevo con vosotros en
el reino de mi Padre.

\hypertarget{camina-a-getsemanuxed}{%
\subsection{Camina a Getsemaní}\label{camina-a-getsemanuxed}}

\bibverse{30} Y habiendo cantado el himno, salieron al monte de las
Olivas.

\bibverse{31} Entonces Jesús les dice: Todos vosotros seréis
escandalizados en mí esta noche; porque escrito está: Heriré al Pastor,
y las ovejas de la manada serán dispersas. \footnote{\textbf{26:31} Juan
  16,32} \bibverse{32} Mas después que haya resucitado, iré delante de
vosotros á Galilea. \footnote{\textbf{26:32} Mat 28,7}

\bibverse{33} Y respondiendo Pedro, le dijo: Aunque todos sean
escandalizados en ti, yo nunca seré escandalizado.

\bibverse{34} Jesús le dice: De cierto te digo que esta noche, antes que
el gallo cante, me negarás tres veces. \footnote{\textbf{26:34} Juan
  13,18}

\bibverse{35} Dícele Pedro: Aunque me sea menester morir contigo, no te
negaré. Y todos los discípulos dijeron lo mismo.

\hypertarget{el-conflicto-y-la-oraciuxf3n-de-jesuxfas-en-getsemanuxed-debilidad-de-los-discuxedpulos}{%
\subsection{El conflicto y la oración de Jesús en Getsemaní; Debilidad
de los
discípulos}\label{el-conflicto-y-la-oraciuxf3n-de-jesuxfas-en-getsemanuxed-debilidad-de-los-discuxedpulos}}

\bibverse{36} Entonces llegó Jesús con ellos á la aldea que se llama
Gethsemaní, y dice á sus discípulos: Sentaos aquí, hasta que vaya allí y
ore. \bibverse{37} Y tomando á Pedro, y á los dos hijos de Zebedeo,
comenzó á entristecerse y á angustiarse en gran manera. \bibverse{38}
Entonces Jesús les dice: Mi alma está muy triste hasta la muerte;
quedaos aquí, y velad conmigo. \footnote{\textbf{26:38} Juan 12,27}

\bibverse{39} Y yéndose un poco más adelante, se postró sobre su rostro,
orando, y diciendo: Padre mío, si es posible, pase de mí este vaso;
empero no como yo quiero, sino como tú. \footnote{\textbf{26:39} Juan
  6,38; Juan 18,11; Heb 5,8}

\bibverse{40} Y vino á sus discípulos, y los halló durmiendo, y dijo á
Pedro: ¿Así no habéis podido velar conmigo una hora? \bibverse{41} Velad
y orad, para que no entréis en tentación: el espíritu á la verdad está
presto, mas la carne enferma. \footnote{\textbf{26:41} Efes 6,18; Heb
  2,18}

\bibverse{42} Otra vez fué, segunda vez, y oró diciendo: Padre mío, si
no puede este vaso pasar de mí sin que yo lo beba, hágase tu voluntad.

\bibverse{43} Y vino, y los halló otra vez durmiendo; porque los ojos de
ellos estaban agravados. \bibverse{44} Y dejándolos fuése de nuevo, y
oró tercera vez, diciendo las mismas palabras. \bibverse{45} Entonces
vino á sus discípulos y díceles: Dormid ya, y descansad: he aquí ha
llegado la hora, y el Hijo del hombre es entregado en manos de
pecadores. \bibverse{46} Levantaos, vamos: he aquí ha llegado el que me
ha entregado.

\hypertarget{encarcelamiento-de-jesuxfas-escape-de-los-discuxedpulos}{%
\subsection{Encarcelamiento de Jesús; Escape de los
discípulos}\label{encarcelamiento-de-jesuxfas-escape-de-los-discuxedpulos}}

\bibverse{47} Y hablando aún él, he aquí Judas, uno de los doce, vino, y
con él mucha gente con espadas y con palos, de parte de los príncipes de
los sacerdotes, y de los ancianos del pueblo. \bibverse{48} Y el que le
entregaba les había dado señal, diciendo: Al que yo besare, aquél es:
prendedle. \bibverse{49} Y luego que llegó á Jesús, dijo: Salve,
Maestro. Y le besó.

\bibverse{50} Y Jesús le dijo: Amigo, ¿á qué vienes? Entonces llegaron,
y echaron mano á Jesús, y le prendieron.

\bibverse{51} Y he aquí, uno de los que estaban con Jesús, extendiendo
la mano, sacó su espada, é hiriendo á un siervo del pontífice, le quitó
la oreja.

\bibverse{52} Entonces Jesús le dice: Vuelve tu espada á su lugar;
porque todos los que tomaren espada, á espada perecerán. \footnote{\textbf{26:52}
  Gén 9,6} \bibverse{53} ¿Acaso piensas que no puedo ahora orar á mi
Padre, y él me daría más de doce legiones de ángeles? \footnote{\textbf{26:53}
  Mat 4,11} \bibverse{54} ¿Cómo, pues, se cumplirían las Escrituras, que
así conviene que sea hecho?

\bibverse{55} En aquella hora dijo Jesús á las gentes: ¿Como á ladrón
habéis salido con espadas y con palos á prenderme? Cada día me sentaba
con vosotros enseñando en el templo, y no me prendisteis. \bibverse{56}
Mas todo esto se hace, para que se cumplan las Escrituras de los
profetas. Entonces todos los discípulos huyeron, dejándole.

\hypertarget{el-interrogatorio-y-la-condena-de-jesuxfas-ante-el-sumo-sacerdote-y-el-concilio}{%
\subsection{El interrogatorio y la condena de Jesús ante el sumo
sacerdote y el
concilio}\label{el-interrogatorio-y-la-condena-de-jesuxfas-ante-el-sumo-sacerdote-y-el-concilio}}

\bibverse{57} Y ellos, prendido Jesús, le llevaron á Caifás pontífice,
donde los escribas y los ancianos estaban juntos. \bibverse{58} Mas
Pedro le seguía de lejos hasta el patio del pontífice; y entrando
dentro, estábase sentado con los criados, para ver el fin.

\bibverse{59} Y los príncipes de los sacerdotes, y los ancianos, y todo
el consejo, buscaban falso testimonio contra Jesús, para entregarle á la
muerte; \bibverse{60} Y no lo hallaron, aunque muchos testigos falsos se
llegaban; mas á la postre vinieron dos testigos falsos, \bibverse{61}
Que dijeron: Este dijo: Puedo derribar el templo de Dios, y en tres días
reedificarlo. \footnote{\textbf{26:61} Juan 2,19-21; Hech 6,14}

\bibverse{62} Y levantándose el pontífice, le dijo: ¿No respondes nada?
¿qué testifican éstos contra ti? \bibverse{63} Mas Jesús callaba.
Respondiendo el pontífice, le dijo: Te conjuro por el Dios viviente, que
nos digas si eres tú el Cristo, Hijo de Dios.

\bibverse{64} Jesús le dijo: Tú lo has dicho: y aun os digo, que desde
ahora habéis de ver al Hijo del hombre sentado á la diestra de la
potencia de Dios, y que viene en las nubes del cielo. \footnote{\textbf{26:64}
  Sal 110,1; Mat 16,27; Mat 24,30; 2Cor 13,4}

\bibverse{65} Entonces el pontífice rasgó sus vestidos, diciendo:
Blasfemado ha: ¿qué más necesidad tenemos de testigos? He aquí, ahora
habéis oído su blasfemia. \footnote{\textbf{26:65} Juan 10,33}
\bibverse{66} ¿Qué os parece? Y respondiendo ellos, dijeron: Culpado es
de muerte. \footnote{\textbf{26:66} Juan 19,7; Lev 24,16}

\bibverse{67} Entonces le escupieron en el rostro, y le dieron de
bofetadas; y otros le herían con mojicones, \footnote{\textbf{26:67} Is
  50,6} \bibverse{68} Diciendo: Profetízanos tú, Cristo, quién es el que
te ha herido.

\hypertarget{negaciuxf3n-y-arrepentimiento-de-pedro}{%
\subsection{Negación y arrepentimiento de
Pedro}\label{negaciuxf3n-y-arrepentimiento-de-pedro}}

\bibverse{69} Y Pedro estaba sentado fuera en el patio: y se llegó á él
una criada, diciendo: Y tú con Jesús el Galileo estabas.

\bibverse{70} Mas él negó delante de todos, diciendo: No sé lo que
dices.

\bibverse{71} Y saliendo él á la puerta, le vió otra, y dijo á los que
estaban allí: También éste estaba con Jesús Nazareno.

\bibverse{72} Y negó otra vez con juramento: No conozco al hombre.

\bibverse{73} Y un poco después llegaron los que estaban por allí, y
dijeron á Pedro: Verdaderamente también tú eres de ellos, porque aun tu
habla te hace manifiesto.

\bibverse{74} Entonces comenzó á hacer imprecaciones, y á jurar,
diciendo: No conozco al hombre. Y el gallo cantó luego.

\bibverse{75} Y se acordó Pedro de las palabras de Jesús, que le dijo:
Antes que cante el gallo, me negarás tres veces. Y saliéndose fuera,
lloró amargamente.

\hypertarget{uxfaltima-deliberaciuxf3n-del-sumo-consejo-extradiciuxf3n-de-los-condenados-al-gobernador-romano-pilato}{%
\subsection{Última deliberación del sumo consejo; Extradición de los
condenados al gobernador romano
Pilato}\label{uxfaltima-deliberaciuxf3n-del-sumo-consejo-extradiciuxf3n-de-los-condenados-al-gobernador-romano-pilato}}

\hypertarget{section-26}{%
\section{27}\label{section-26}}

\bibverse{1} Y venida la mañana, entraron en consejo todos los príncipes
de los sacerdotes, y los ancianos del pueblo, contra Jesús, para
entregarle á muerte. \bibverse{2} Y le llevaron atado, y le entregaron á
Poncio Pilato presidente.

\bibverse{3} Entonces Judas, el que le había entregado, viendo que era
condenado, volvió arrepentido las treinta piezas de plata á los
príncipes de los sacerdotes y á los ancianos, \footnote{\textbf{27:3}
  Mat 26,15} \bibverse{4} Diciendo: Yo he pecado entregando la sangre
inocente. Mas ellos dijeron: ¿Qué se nos da á nosotros? Viéraslo tú.

\bibverse{5} Y arrojando las piezas de plata en el templo, partióse; y
fué, y se ahorcó.

\bibverse{6} Y los príncipes de los sacerdotes, tomando las piezas de
plata, dijeron: No es lícito echarlas en el tesoro de los dones, porque
es precio de sangre. \footnote{\textbf{27:6} Deut 23,19} \bibverse{7}
Mas habido consejo, compraron con ellas el campo del alfarero, por
sepultura para los extranjeros. \bibverse{8} Por lo cual fué llamado
aquel campo, Campo de sangre, hasta el día de hoy. \bibverse{9} Entonces
se cumplió lo que fué dicho por el profeta Jeremías, que dijo: Y tomaron
las treinta piezas de plata, precio del apreciado, que fué apreciado por
los hijos de Israel; \bibverse{10} Y las dieron para el campo del
alfarero, como me ordenó el Señor.

\hypertarget{interrogatorio-de-jesuxfas-ante-pilato-jesuxfas-rechazado-por-la-gente-su-condenaciuxf3n-y-flagelaciuxf3n}{%
\subsection{Interrogatorio de Jesús ante Pilato; Jesús rechazado por la
gente; su condenación y
flagelación}\label{interrogatorio-de-jesuxfas-ante-pilato-jesuxfas-rechazado-por-la-gente-su-condenaciuxf3n-y-flagelaciuxf3n}}

\bibverse{11} Y Jesús estuvo delante del presidente; y el presidente le
preguntó, diciendo: ¿Eres tú el Rey de los judíos? Y Jesús le dijo: Tú
lo dices.

\bibverse{12} Y siendo acusado por los príncipes de los sacerdotes, y
por los ancianos, nada respondió. \bibverse{13} Pilato entonces le dice:
¿No oyes cuántas cosas testifican contra ti?

\bibverse{14} Y no le respondió ni una palabra; de tal manera que el
presidente se maravillaba mucho. \footnote{\textbf{27:14} Juan 19,9}

\hypertarget{jesuxfas-y-barrabuxe1s}{%
\subsection{Jesús y Barrabás}\label{jesuxfas-y-barrabuxe1s}}

\bibverse{15} Y en el día de la fiesta acostumbraba el presidente soltar
al pueblo un preso, cual quisiesen. \bibverse{16} Y tenían entonces un
preso famoso que se llamaba Barrabás. \bibverse{17} Y juntos ellos, les
dijo Pilato: ¿Cuál queréis que os suelte? ¿á Barrabás, ó á Jesús que se
dice el Cristo? \bibverse{18} Porque sabía que por envidia le habían
entregado.

\bibverse{19} Y estando él sentado en el tribunal, su mujer envió á él,
diciendo: No tengas que ver con aquel justo; porque hoy he padecido
muchas cosas en sueños por causa de él.

\bibverse{20} Mas los príncipes de los sacerdotes y los ancianos,
persuadieron al pueblo que pidiese á Barrabás, y á Jesús matase.
\bibverse{21} Y respondiendo el presidente les dijo: ¿Cuál de los dos
queréis que os suelte? Y ellos dijeron: A Barrabás.

\bibverse{22} Pilato les dijo: ¿Qué pues haré de Jesús que se dice el
Cristo? Dícenle todos: Sea crucificado.

\bibverse{23} Y el presidente les dijo: Pues ¿qué mal ha hecho? Mas
ellos gritaban más, diciendo: Sea crucificado.

\bibverse{24} Y viendo Pilato que nada adelantaba, antes se hacía más
alboroto, tomando agua se lavó las manos delante del pueblo, diciendo:
Inocente soy yo de la sangre de este justo: veréislo vosotros.
\footnote{\textbf{27:24} Deut 21,6}

\bibverse{25} Y respondiendo todo el pueblo, dijo: Su sangre sea sobre
nosotros, y sobre nuestros hijos. \footnote{\textbf{27:25} Hech 5,28}

\bibverse{26} Entonces les soltó á Barrabás: y habiendo azotado á Jesús,
le entregó para ser crucificado.

\hypertarget{la-burla-y-el-maltrato-de-jesuxfas-por-parte-de-los-soldados-romanos}{%
\subsection{La burla y el maltrato de Jesús por parte de los soldados
romanos}\label{la-burla-y-el-maltrato-de-jesuxfas-por-parte-de-los-soldados-romanos}}

\bibverse{27} Entonces los soldados del presidente llevaron á Jesús al
pretorio, y juntaron á él toda la cuadrilla; \bibverse{28} Y
desnudándole, le echaron encima un manto de grana; \bibverse{29} Y
pusieron sobre su cabeza una corona tejida de espinas, y una caña en su
mano derecha; é hincando la rodilla delante de él, le burlaban,
diciendo: ¡Salve, Rey de los Judíos! \bibverse{30} Y escupiendo en él,
tomaron la caña, y le herían en la cabeza. \footnote{\textbf{27:30} Is
  50,6} \bibverse{31} Y después que le hubieron escarnecido, le
desnudaron el manto, y le vistieron de sus vestidos, y le llevaron para
crucificarle.

\hypertarget{el-curso-de-la-muerte-de-jesuxfas-despuuxe9s-del-guxf3lgota-su-crucifixiuxf3n-y-su-muerte}{%
\subsection{El curso de la muerte de Jesús después del Gólgota, su
crucifixión y su
muerte}\label{el-curso-de-la-muerte-de-jesuxfas-despuuxe9s-del-guxf3lgota-su-crucifixiuxf3n-y-su-muerte}}

\bibverse{32} Y saliendo, hallaron á un Cireneo, que se llamaba Simón: á
éste cargaron para que llevase su cruz. \bibverse{33} Y como llegaron al
lugar que se llama Gólgotha, que es dicho, El lugar de la calavera,
\bibverse{34} Le dieron á beber vinagre mezclado con hiel; y gustando,
no quiso beberlo. \bibverse{35} Y después que le hubieron crucificado,
repartieron sus vestidos, echando suertes: para que se cumpliese lo que
fué dicho por el profeta: Se repartieron mis vestidos, y sobre mi ropa
echaron suertes. \footnote{\textbf{27:35} Juan 19,24} \bibverse{36} Y
sentados le guardaban allí. \bibverse{37} Y pusieron sobre su cabeza su
causa escrita: ESTE ES JESUS EL REY DE LOS JUDIOS.

\bibverse{38} Entonces crucificaron con él dos ladrones, uno á la
derecha, y otro á la izquierda.

\bibverse{39} Y los que pasaban, le decían injurias, meneando sus
cabezas, \footnote{\textbf{27:39} Sal 22,8} \bibverse{40} Y diciendo:
Tú, el que derribas el templo, y en tres días lo reedificas, sálvate á
ti mismo: si eres Hijo de Dios, desciende de la cruz. \footnote{\textbf{27:40}
  Mat 26,61; Juan 2,19}

\bibverse{41} De esta manera también los príncipes de los sacerdotes,
escarneciendo con los escribas y los Fariseos y los ancianos, decían:
\bibverse{42} A otros salvó, á sí mismo no puede salvar: si es el Rey de
Israel, descienda ahora de la cruz, y creeremos en él. \bibverse{43}
Confió en Dios: líbrele ahora si le quiere: porque ha dicho: Soy Hijo de
Dios. \footnote{\textbf{27:43} Sal 22,9} \bibverse{44} Lo mismo también
le zaherían los ladrones que estaban crucificados con él.

\hypertarget{la-muerte-de-jesuxfas-las-seuxf1ales-milagrosas-de-su-muerte}{%
\subsection{La muerte de Jesús; las señales milagrosas de su
muerte}\label{la-muerte-de-jesuxfas-las-seuxf1ales-milagrosas-de-su-muerte}}

\bibverse{45} Y desde la hora de sexta fueron tinieblas sobre toda la
tierra hasta la hora de nona. \bibverse{46} Y cerca de la hora de nona,
Jesús exclamó con grande voz, diciendo: Eli, Eli, ¿lama sabachtani? Esto
es: Dios mío, Dios mío, ¿por qué me has desamparado?

\bibverse{47} Y algunos de los que estaban allí, oyéndolo, decían: A
Elías llama éste.

\bibverse{48} Y luego, corriendo uno de ellos, tomó una esponja, y la
hinchió de vinagre, y poniéndola en una caña, dábale de beber.
\footnote{\textbf{27:48} Sal 69,22} \bibverse{49} Y los otros decían:
Deja, veamos si viene Elías á librarle.

\bibverse{50} Mas Jesús, habiendo otra vez exclamado con grande voz, dió
el espíritu.

\bibverse{51} Y he aquí, el velo del templo se rompió en dos, de alto á
bajo: y la tierra tembló, y las piedras se hendieron; \bibverse{52} Y
abriéronse los sepulcros, y muchos cuerpos de santos que habían dormido,
se levantaron; \bibverse{53} Y salidos de los sepulcros, después de su
resurrección, vinieron á la santa ciudad, y aparecieron á muchos.

\bibverse{54} Y el centurión, y los que estaban con él guardando á
Jesús, visto el terremoto, y las cosas que habían sido hechas, temieron
en gran manera, diciendo: Verdaderamente Hijo de Dios era éste.

\bibverse{55} Y estaban allí muchas mujeres mirando de lejos, las cuales
habían seguido de Galilea á Jesús, sirviéndole: \footnote{\textbf{27:55}
  Luc 8,2-3} \bibverse{56} Entre las cuales estaban María Magdalena, y
María la madre de Jacobo y de José, y la madre de los hijos de Zebedeo.

\hypertarget{entierro-de-jesuxfas-orden-de-los-guardias-de-la-tumba}{%
\subsection{Entierro de Jesús; Orden de los guardias de la
tumba}\label{entierro-de-jesuxfas-orden-de-los-guardias-de-la-tumba}}

\bibverse{57} Y como fué la tarde del día, vino un hombre rico de
Arimatea, llamado José, el cual también había sido discípulo de Jesús.
\bibverse{58} Este llegó á Pilato, y pidió el cuerpo de Jesús: entonces
Pilato mandó que se le diese el cuerpo. \bibverse{59} Y tomando José el
cuerpo, lo envolvió en una sábana limpia, \bibverse{60} Y lo puso en su
sepulcro nuevo, que había labrado en la peña: y revuelta una grande
piedra á la puerta del sepulcro, se fué. \footnote{\textbf{27:60} Is
  53,9} \bibverse{61} Y estaban allí María Magdalena, y la otra María,
sentadas delante del sepulcro.

\bibverse{62} Y el siguiente día, que es después de la preparación, se
juntaron los príncipes de los sacerdotes y los Fariseos á Pilato,
\bibverse{63} Diciendo: Señor, nos acordamos que aquel engañador dijo,
viviendo aún: Después de tres días resucitaré. \bibverse{64} Manda,
pues, que se asegure el sepulcro hasta el día tercero; porque no vengan
sus discípulos de noche, y le hurten, y digan al pueblo: Resucitó de los
muertos. Y será el postrer error peor que el primero.

\bibverse{65} Y Pilato les dijo: Tenéis una guardia: id, aseguradlo como
sabéis. \bibverse{66} Y yendo ellos, aseguraron el sepulcro, sellando la
piedra, con la guardia.

\hypertarget{las-dos-mujeres-junto-a-la-tumba-vacuxeda-en-la-mauxf1ana-de-pascua-la-primera-apariciuxf3n-de-jesuxfas-engauxf1ar-al-luxedder-del-pueblo}{%
\subsection{Las dos mujeres junto a la tumba vacía en la mañana de
Pascua; La primera aparición de Jesús; Engañar al líder del
pueblo}\label{las-dos-mujeres-junto-a-la-tumba-vacuxeda-en-la-mauxf1ana-de-pascua-la-primera-apariciuxf3n-de-jesuxfas-engauxf1ar-al-luxedder-del-pueblo}}

\hypertarget{section-27}{%
\section{28}\label{section-27}}

\bibverse{1} Y la víspera de sábado, que amanece para el primer día de
la semana, vino María Magdalena, y la otra María, á ver el sepulcro.
\footnote{\textbf{28:1} Hech 20,7; 1Cor 16,2; Apoc 1,10} \bibverse{2} Y
he aquí, fué hecho un gran terremoto: porque el ángel del Señor,
descendiendo del cielo y llegando, había revuelto la piedra, y estaba
sentado sobre ella. \bibverse{3} Y su aspecto era como un relámpago, y
su vestido blanco como la nieve. \bibverse{4} Y de miedo de él los
guardas se asombraron, y fueron vueltos como muertos. \bibverse{5} Y
respondiendo el ángel, dijo á las mujeres: No temáis vosotras; porque yo
sé que buscáis á Jesús, que fué crucificado. \bibverse{6} No está aquí;
porque ha resucitado, como dijo. Venid, ved el lugar donde fué puesto el
Señor. \footnote{\textbf{28:6} Mat 12,40; Mat 16,21; Mat 17,23; Mat
  20,19} \bibverse{7} E id presto, decid á sus discípulos que ha
resucitado de los muertos: y he aquí va delante de vosotros á Galilea;
allí le veréis; he aquí, os lo he dicho. \footnote{\textbf{28:7} Mat
  26,32}

\bibverse{8} Entonces ellas, saliendo del sepulcro con temor y gran
gozo, fueron corriendo á dar las nuevas á sus discípulos. Y mientras
iban á dar las nuevas á sus discípulos, \bibverse{9} He aquí, Jesús les
sale al encuentro, diciendo: Salve. Y ellas se llegaron y abrazaron sus
pies, y le adoraron.

\bibverse{10} Entonces Jesús les dice: No temáis: id, dad las nuevas á
mis hermanos, para que vayan á Galilea, y allí me verán. \footnote{\textbf{28:10}
  Heb 2,11}

\hypertarget{la-falsa-afirmaciuxf3n-de-los-luxedderes-del-pueblo-del-cuerpo-robado-de-jesuxfas}{%
\subsection{La falsa afirmación de los líderes del pueblo del cuerpo
robado de
Jesús}\label{la-falsa-afirmaciuxf3n-de-los-luxedderes-del-pueblo-del-cuerpo-robado-de-jesuxfas}}

\bibverse{11} Y yendo ellas, he aquí unos de la guardia vinieron á la
ciudad, y dieron aviso á los príncipes de los sacerdotes de todas las
cosas que habían acontecido. \bibverse{12} Y juntados con los ancianos,
y habido consejo, dieron mucho dinero á los soldados, \bibverse{13}
Diciendo: Decid: Sus discípulos vinieron de noche, y le hurtaron,
durmiendo nosotros. \bibverse{14} Y si esto fuere oído del presidente,
nosotros le persuadiremos, y os haremos seguros. \bibverse{15} Y ellos,
tomando el dinero, hicieron como estaban instruídos: y este dicho fué
divulgado entre los Judíos hasta el día de hoy.

\hypertarget{jesuxfas-apareciuxf3-en-la-montauxf1a-de-galilea-su-uxfaltimo-mandato-a-los-once-discuxedpulos}{%
\subsection{Jesús apareció en la montaña de Galilea; su último mandato a
los once
discípulos}\label{jesuxfas-apareciuxf3-en-la-montauxf1a-de-galilea-su-uxfaltimo-mandato-a-los-once-discuxedpulos}}

\bibverse{16} Mas los once discípulos se fueron á Galilea, al monte
donde Jesús les había ordenado. \bibverse{17} Y como le vieron, le
adoraron: mas algunos dudaban. \bibverse{18} Y llegando Jesús, les
habló, diciendo: Toda potestad me es dada en el cielo y en la tierra.
\^{}\^{} \bibverse{19} Por tanto, id, y doctrinad á todos los Gentiles,
bautizándolos en el nombre del Padre, y del Hijo, y del Espíritu Santo:
\^{}\^{} \bibverse{20} Enseñándoles que guarden todas las cosas que os
he mandado: y he aquí, yo estoy con vosotros todos los días, hasta el
fin del mundo. Amén.
