\hypertarget{el-destino-de-noemuxed-en-la-tierra-de-los-moabitas}{%
\subsection{El destino de Noemí en la tierra de los
moabitas}\label{el-destino-de-noemuxed-en-la-tierra-de-los-moabitas}}

\hypertarget{section}{%
\section{1}\label{section}}

\bibverse{1} Y aconteció en los días que gobernaban los jueces, que hubo
hambre en la tierra. Y un varón de Beth-lehem de Judá, fué á peregrinar
en los campos de Moab, él y su mujer, y dos hijos suyos. \bibverse{2} El
nombre de aquel varón era Elimelech, y el de su mujer Noemi; y los
nombres de sus dos hijos eran, Mahalón y Chelión, Ephrateos de
Beth-lehem de Judá. Llegaron pues á los campos de Moab, y asentaron
allí. \bibverse{3} Y murió Elimelech, marido de Noemi, y quedó ella con
sus dos hijos; \bibverse{4} Los cuales tomaron para sí mujeres de Moab,
el nombre de la una Orpha, y el nombre de la otra Ruth; y habitaron allí
unos diez años. \bibverse{5} Y murieron también los dos, Mahalón y
Chelión, quedando así la mujer desamparada de sus dos hijos y de su
marido. \bibverse{6} Entonces se levantó con sus nueras, y volvióse de
los campos de Moab: porque oyó en el campo de Moab que Jehová había
visitado á su pueblo para darles pan.

\hypertarget{partida-de-noemuxed-y-sus-dos-nueras-para-regresar-a-beluxe9n-la-despedida-de-orpa-la-lealtad-de-ruth}{%
\subsection{Partida de Noemí y sus dos nueras para regresar a Belén; La
despedida de Orpa, la lealtad de
Ruth}\label{partida-de-noemuxed-y-sus-dos-nueras-para-regresar-a-beluxe9n-la-despedida-de-orpa-la-lealtad-de-ruth}}

\bibverse{7} Salió pues del lugar donde había estado, y con ella sus dos
nueras, y comenzaron á caminar para volverse á la tierra de Judá.
\bibverse{8} Y Noemi dijo á sus dos nueras: Andad, volveos cada una á la
casa de su madre: Jehová haga con vosotras misericordia, como la habéis
hecho con los muertos y conmigo. \bibverse{9} Déos Jehová que halléis
descanso, cada una en casa de su marido: besólas luego, y ellas lloraron
á voz en grito.

\bibverse{10} Y dijéronle: Ciertamente nosotras volveremos contigo á tu
pueblo.

\bibverse{11} Y Noemi respondió: Volveos, hijas mías: ¿para qué habéis
de ir conmigo? ¿tengo yo más hijos en el vientre, que puedan ser
vuestros maridos? \bibverse{12} Volveos, hijas mías, é idos; que yo ya
soy vieja para ser para varón. Y aunque dijese: Esperanza tengo; y esta
noche fuese con varón, y aun pariese hijos; \bibverse{13} ¿Habíais
vosotras de esperarlos hasta que fuesen grandes? ¿habías vosotras de
quedaros sin casar por amor de ellos? No, hijas mías; que mayor amargura
tengo yo que vosotras, pues la mano de Jehová ha salido contra mí.
\footnote{\textbf{1:13} Job 19,21}

\bibverse{14} Mas ellas alzando otra vez su voz, lloraron: y Orpha besó
á su suegra, mas Ruth se quedó con ella. \bibverse{15} Y Noemi dijo: He
aquí tu cuñada se ha vuelto á su pueblo y á sus dioses; vuélvete tú tras
ella.

\bibverse{16} Y Ruth respondió: No me ruegues que te deje, y me aparte
de ti: porque donde quiera que tú fueres, iré yo; y donde quiera que
vivieres, viviré. Tu pueblo será mi pueblo, y tu Dios mi Dios.
\bibverse{17} Donde tú murieres, moriré yo, y allí seré sepultada: así
me haga Jehová, y así me dé, que sólo la muerte hará separación entre mí
y ti.

\bibverse{18} Y viendo Noemi que estaba tan resuelta á ir con ella, dejó
de hablarle.

\hypertarget{llegada-y-recepciuxf3n-de-las-dos-mujeres-en-beluxe9n}{%
\subsection{Llegada y recepción de las dos mujeres en
Belén}\label{llegada-y-recepciuxf3n-de-las-dos-mujeres-en-beluxe9n}}

\bibverse{19} Anduvieron pues ellas dos hasta que llegaron á Beth-lehem:
y aconteció que entrando en Beth-lehem, toda la ciudad se conmovió por
razón de ellas, y decían: ¿No es ésta Noemi?

\bibverse{20} Y ella les respondía: No me llaméis Noemi, sino llamadme
Mara: porque en grande amargura me ha puesto el Todopoderoso.
\footnote{\textbf{1:20} Éxod 15,23}

\bibverse{21} Yo me fuí llena, mas vacía me ha vuelto Jehová. ¿Por qué
me llamaréis Noemi, ya que Jehová ha dado testimonio contra mí, y el
Todopoderoso me ha afligido? \bibverse{22} Así volvió Noemi y Ruth
Moabita su nuera con ella; volvió de los campos de Moab, y llegaron á
Beth-lehem en el principio de la siega de las cebadas.

\hypertarget{rut-viene-a-recoger-espigas-en-el-campo-del-booz-quien-pregunta-por-ella-y-la-recibe-amablemente}{%
\subsection{Rut viene a recoger espigas en el campo del booz, quien
pregunta por ella y la recibe
amablemente}\label{rut-viene-a-recoger-espigas-en-el-campo-del-booz-quien-pregunta-por-ella-y-la-recibe-amablemente}}

\hypertarget{section-1}{%
\section{2}\label{section-1}}

\bibverse{1} Y tenía Noemi un pariente de su marido, varón poderoso y de
hecho, de la familia de Elimelech, el cual se llamaba Booz. \bibverse{2}
Y Ruth la Moabita dijo á Noemi: Ruégote que me dejes ir al campo, y
cogeré espigas en pos de aquel á cuyos ojos hallare gracia. Y ella le
respondió: Ve, hija mía.

\bibverse{3} Fué pues, y llegando, espigó en el campo en pos de los
segadores: y aconteció por ventura, que la suerte del campo era de Booz,
el cual era de la parentela de Elimelech.

\bibverse{4} Y he aquí que Booz vino de Beth-lehem, y dijo á los
segadores: Jehová sea con vosotros. Y ellos respondieron: Jehová te
bendiga.

\bibverse{5} Y Booz dijo á su criado el sobrestante de los segadores:
¿Cúya es esta moza?

\bibverse{6} Y el criado, sobrestante de los segadores, respondió y
dijo: Es la moza de Moab, que volvió con Noemi de los campos de Moab;
\bibverse{7} Y ha dicho: Ruégote que me dejes coger y juntar tras los
segadores entre las gavillas: entró pues, y está desde por la mañana
hasta ahora, menos un poco que se detuvo en casa.

\bibverse{8} Entonces Booz dijo á Ruth: Oye, hija mía, no vayas á
espigar á otro campo, ni pases de aquí: y aquí estarás con mis mozas.
\bibverse{9} Mira bien al campo que segaren, y síguelas: porque yo he
mandado á los mozos que no te toquen. Y si tuvieres sed, ve á los vasos,
y bebe del agua que sacaren los mozos.

\bibverse{10} Ella entonces bajando su rostro inclinóse á tierra, y
díjole: ¿Por qué he hallado gracia en tus ojos para que tú me
reconozcas, siendo yo extranjera?

\bibverse{11} Y respondiendo Booz, díjole: Por cierto se me ha declarado
todo lo que has hecho con tu suegra después de la muerte de tu marido, y
que dejando á tu padre y á tu madre y la tierra donde naciste, has
venido á pueblo que no conociste antes. \bibverse{12} Jehová galardone
tu obra, y tu remuneración sea llena por Jehová Dios de Israel, que has
venido para cubrirte debajo de sus alas.

\bibverse{13} Y ella dijo: Señor mío, halle yo gracia delante de tus
ojos; porque me has consolado, y porque has hablado al corazón de tu
sierva, no siendo yo como una de tus criadas.

\hypertarget{rut-sigue-siendo-tratada-amablemente-por-booz-llega-a-casa-con-una-rica-cosecha-y-recibe-informaciuxf3n-sobre-booz-de-su-suegra}{%
\subsection{Rut sigue siendo tratada amablemente por Booz, llega a casa
con una rica cosecha y recibe información sobre Booz de su
suegra}\label{rut-sigue-siendo-tratada-amablemente-por-booz-llega-a-casa-con-una-rica-cosecha-y-recibe-informaciuxf3n-sobre-booz-de-su-suegra}}

\bibverse{14} Y Booz le dijo á la hora de comer: Allégate aquí, y come
del pan, y moja tu bocado en el vinagre. Y sentóse ella junto á los
segadores, y él le dió del potaje, y comió hasta que se hartó y le
sobró.

\bibverse{15} Levantóse luego para espigar. Y Booz mandó á sus criados,
diciendo: Coja también espigas entre las gavillas, y no la avergoncéis;
\bibverse{16} Antes echaréis á sabiendas de los manojos, y la dejaréis
que coja, y no la reprendáis. \footnote{\textbf{2:16} Lev 19,9}

\bibverse{17} Y espigó en el campo hasta la tarde, y desgranó lo que
había cogido, y fué como un epha de cebada. \bibverse{18} Y tomólo, y
vínose á la ciudad; y su suegra vió lo que había cogido. Sacó también
luego lo que le había sobrado después de harta, y dióselo.

\bibverse{19} Y díjole su suegra: ¿Dónde has espigado hoy? ¿y dónde has
trabajado? bendito sea el que te ha reconocido. Y ella declaró á su
suegra lo que le había acontecido con aquél, y dijo: El nombre del varón
con quien hoy he trabajado es Booz.

\bibverse{20} Y dijo Noemi á su nuera: Sea él bendito de Jehová, pues
que no ha rehusado á los vivos la benevolencia que tuvo para con los
finados. Díjole después Noemi: Nuestro pariente es aquel varón, y de
nuestros redentores es.

\bibverse{21} Y Ruth Moabita dijo: A más de esto me ha dicho: Júntate
con mis criados, hasta que hayan acabado toda mi siega.

\bibverse{22} Y Noemi respondió á Ruth su nuera: Mejor es, hija mía, que
salgas con sus criadas, que no que te encuentren en otro campo.
\bibverse{23} Estuvo pues junta con las mozas de Booz espigando, hasta
que la siega de las cebadas y la de los trigos fué acabada; mas con su
suegra habitó.

\hypertarget{siguiendo-el-consejo-de-noemuxed-rut-va-a-la-era-de-booz-y-se-acuesta-a-sus-pies}{%
\subsection{Siguiendo el consejo de Noemí, Rut va a la era de Booz y se
acuesta a sus
pies}\label{siguiendo-el-consejo-de-noemuxed-rut-va-a-la-era-de-booz-y-se-acuesta-a-sus-pies}}

\hypertarget{section-2}{%
\section{3}\label{section-2}}

\bibverse{1} Y díjole su suegra Noemi: Hija mía, ¿no te tengo de buscar
descanso, que te sea bueno? \bibverse{2} ¿No es Booz nuestro pariente,
con cuyas mozas tú has estado? He aquí que él avienta esta noche la
parva de las cebadas. \bibverse{3} Te lavarás pues, y te ungirás, y
vistiéndote tus vestidos, pasarás á la era; mas no te darás á conocer al
varón hasta que él haya acabado de comer y de beber. \bibverse{4} Y
cuando él se acostare, repara tú el lugar donde él se acostará, é irás,
y descubrirás los pies, y te acostarás allí; y él te dirá lo que hayas
de hacer.

\bibverse{5} Y le respondió: Haré todo lo que tú me mandares.
\bibverse{6} Descendió pues á la era, é hizo todo lo que su suegra le
había mandado. \bibverse{7} Y como Booz hubo comido y bebido, y su
corazón estuvo contento, retiróse á dormir á un lado del montón.
Entonces ella vino calladamente, y descubrió los pies, y acostóse.

\hypertarget{rut-habla-con-booz-recibe-la-confirmaciuxf3n-solicitada-y-regresa-a-noemuxed-con-un-regalo}{%
\subsection{Rut habla con Booz, recibe la confirmación solicitada y
regresa a Noemí con un
regalo}\label{rut-habla-con-booz-recibe-la-confirmaciuxf3n-solicitada-y-regresa-a-noemuxed-con-un-regalo}}

\bibverse{8} Y aconteció, que á la media noche se estremeció aquel
hombre, y palpó: y he aquí, la mujer que estaba acostada á sus pies.
\bibverse{9} Entonces él dijo: ¿Quién eres? Y ella respondió: Yo soy
Ruth tu sierva: extiende el borde de tu capa sobre tu sierva, por cuanto
eres pariente cercano. \footnote{\textbf{3:9} Deut 25,5; Ezeq 16,8}

\bibverse{10} Y él dijo: Bendita seas tú de Jehová, hija mía; que has
hecho mejor tu postrera gracia que la primera, no yendo tras los
mancebos, sean pobres ó ricos. \footnote{\textbf{3:10} Rut 2,11}
\bibverse{11} Ahora pues, no temas, hija mía: yo haré contigo lo que tú
dijeres, pues que toda la puerta de mi pueblo sabe que eres mujer
virtuosa. \bibverse{12} Y ahora, aunque es cierto que yo soy pariente
cercano, con todo eso hay pariente más cercano que yo. \bibverse{13}
Reposa esta noche, y cuando sea de día, si él te redimiere, bien,
redímate; mas si él no te quisiere redimir, yo te redimiré, vive Jehová.
Descansa pues hasta la mañana.

\bibverse{14} Y después que reposó á sus pies hasta la mañana,
levantóse, antes que nadie pudiese conocer á otro. Y él dijo: No se sepa
que haya venido mujer á la era. \bibverse{15} Después le dijo: Llega el
lienzo que traes sobre ti, y ten de él. Y teniéndolo ella, él midió seis
medidas de cebada, y púsoselas á cuestas: y vínose ella á la ciudad.

\bibverse{16} Así que vino á su suegra, ésta le dijo: ¿Qué pues, hija
mía? Y declaróle ella todo lo que con aquel varón le había acontecido.

\bibverse{17} Y dijo: Estas seis medidas de cebada me dió, diciéndome:
Porque no vayas vacía á tu suegra.

\bibverse{18} Entonces Noemi dijo: Reposa, hija mía, hasta que sepas
cómo cae la cosa: porque aquel hombre no parará hasta que hoy concluya
el negocio.

\hypertarget{la-negociaciuxf3n-puxfablica-entre-booz-y-el-solver}{%
\subsection{La negociación pública entre Booz y el
Solver}\label{la-negociaciuxf3n-puxfablica-entre-booz-y-el-solver}}

\hypertarget{section-3}{%
\section{4}\label{section-3}}

\bibverse{1} Y booz subió á la puerta y sentóse allí: y he aquí pasaba
aquel pariente del cual había Booz hablado, y díjole: Eh, fulano, ven
acá y siéntate. Y él vino, y sentóse. \bibverse{2} Entonces él tomó diez
varones de los ancianos de la ciudad, y dijo: Sentaos aquí. Y ellos se
sentaron. \bibverse{3} Luego dijo al pariente: Noemi, que ha vuelto del
campo de Moab, vende una parte de las tierras que tuvo nuestro hermano
Elimelech; \bibverse{4} Y yo decidí hacértelo saber, y decirte que la
tomes delante de los que están aquí sentados, y delante de los ancianos
de mi pueblo. Si hubieres de redimir, redime; y si no quisieres redimir,
decláramelo para que yo lo sepa: porque no hay otro que redima sino tú,
y yo después de ti. Y él respondió: Yo redimiré. \footnote{\textbf{4:4}
  Lev 25,25}

\bibverse{5} Entonces replicó Booz: El mismo día que tomares las tierras
de mano de Noemi, has de tomar también á Ruth Moabita, mujer del
difunto, para que suscites el nombre del muerto sobre su posesión.
\footnote{\textbf{4:5} Deut 25,5-6}

\bibverse{6} Y respondió el pariente: No puedo redimir por mi parte,
porque echaría á perder mi heredad: redime tú usando de mi derecho,
porque yo no podré redimir.

\bibverse{7} Había ya de largo tiempo esta costumbre en Israel en la
redención ó contrato, que para la confirmación de cualquier negocio, el
uno se quitaba el zapato y lo daba á su compañero: y este era el
testimonio en Israel. \footnote{\textbf{4:7} Deut 25,7-10} \bibverse{8}
Entonces el pariente dijo á Booz: Tómalo tú. Y descalzó su zapato.

\bibverse{9} Y Booz dijo á los ancianos y á todo el pueblo: Vosotros
sois hoy testigos de que tomo todas las cosas que fueron de Elimelech, y
todo lo que fué de Chelión y de Mahalón, de mano de Noemi; \bibverse{10}
Y que también tomo por mi mujer á Ruth Moabita, mujer de Mahalón, para
suscitar el nombre del difunto sobre su heredad, para que el nombre del
muerto no se borre de entre sus hermanos y de la puerta de su lugar.
Vosotros sois hoy testigos.

\bibverse{11} Y dijeron todos los del pueblo que estaban á la puerta con
los ancianos: Testigos somos. Jehová haga á la mujer que entra en tu
casa como á Rachêl y á Lea, las cuales dos edificaron la casa de Israel;
y tú seas ilustre en Ephrata, y tengas nombradía en Beth-lehem;
\bibverse{12} Y de la simiente que Jehová te diere de aquesta moza, sea
tu casa como la casa de Phares, al que parió Thamar á Judá.

\hypertarget{el-matrimonio-de-des-booz-con-rut-se-completuxf3-y-fue-bendecido-con-el-nacimiento-de-obed-uxedndice-de-guxe9nero-de-puxe9rez-a-david}{%
\subsection{El matrimonio de Des Booz con Rut se completó y fue
bendecido con el nacimiento de Obed; Índice de género de Pérez a
David}\label{el-matrimonio-de-des-booz-con-rut-se-completuxf3-y-fue-bendecido-con-el-nacimiento-de-obed-uxedndice-de-guxe9nero-de-puxe9rez-a-david}}

\bibverse{13} Booz pues tomó á Ruth, y ella fué su mujer; y luego que
entró á ella, Jehová le dió que concibiese y pariese un hijo.
\footnote{\textbf{4:13} Sal 127,3} \bibverse{14} Y las mujeres decían á
Noemi: Loado sea Jehová, que hizo que no te faltase hoy pariente, cuyo
nombre será nombrado en Israel. \bibverse{15} El cual será restaurador
de tu alma, y el que sustentará tu vejez; pues que tu nuera, la cual te
ama y te vale más que siete hijos, le ha parido. \bibverse{16} Y tomando
Noemi el hijo, púsolo en su regazo, y fuéle su ama. \bibverse{17} Y las
vecinas diciendo, A Noemi ha nacido un hijo, le pusieron nombre; y
llamáronle Obed. Este es padre de Isaí, padre de David.

\bibverse{18} Y estas son las generaciones de Phares: Phares engendró á
Hesrón; \^{}\^{} \bibverse{19} Y Hesrón engendró á Ram, y Ram engendró á
Aminadab; \^{}\^{} \bibverse{20} Y Aminadab engendró á Nahasón, y
Nahasón engendró á Salmón; \bibverse{21} Y Salmón engendró á Booz, y
Booz engendró á Obed; \bibverse{22} Y Obed engendró á Isaí, é Isaí
engendró á David.
