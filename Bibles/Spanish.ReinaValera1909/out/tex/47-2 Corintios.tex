\hypertarget{bendiciones}{%
\subsection{Bendiciones}\label{bendiciones}}

\hypertarget{section}{%
\section{1}\label{section}}

\bibverse{1} PABLO, apóstol de Jesucristo por la voluntad de Dios, y
Timoteo el hermano, á la iglesia de Dios que está en Corinto, juntamente
con todos los santos que están por toda la Acaya: \bibverse{2} Gracia y
paz á vosotros de Dios nuestro Padre, y del Señor Jesucristo.

\hypertarget{la-oraciuxf3n-de-acciuxf3n-de-gracias-del-apuxf3stol-por-el-consuelo-que-tanto-uxe9l-como-los-lectores-reciben-de-dios-en-el-sufrimiento}{%
\subsection{La oración de acción de gracias del apóstol por el consuelo
que tanto él como los lectores reciben de Dios en el
sufrimiento}\label{la-oraciuxf3n-de-acciuxf3n-de-gracias-del-apuxf3stol-por-el-consuelo-que-tanto-uxe9l-como-los-lectores-reciben-de-dios-en-el-sufrimiento}}

\bibverse{3} Bendito sea el Dios y Padre del Señor Jesucristo, el Padre
de misericordias, y el Dios de toda consolación, \footnote{\textbf{1:3}
  Rom 15,5} \bibverse{4} El cual nos consuela en todas nuestras
tribulaciones, para que podamos también nosotros consolar á los que
están en cualquiera angustia, con la consolación con que nosotros somos
consolados de Dios. \bibverse{5} Porque de la manera que abundan en
nosotros las aflicciones de Cristo, así abunda también por el mismo
Cristo nuestra consolación. \bibverse{6} Mas si somos atribulados, es
por vuestra consolación y salud; la cual es obrada en el sufrir las
mismas aflicciones que nosotros también padecemos: ó si somos
consolados, es por vuestra consolación y salud; \footnote{\textbf{1:6}
  2Cor 4,8-11; 2Cor 4,15} \bibverse{7} Y nuestra esperanza de vosotros
es firme; estando ciertos que como sois compañeros de las aflicciones,
así también lo sois de la consolación.

\hypertarget{mensaje-sobre-la-salvaciuxf3n-de-pablo-y-sus-colaboradores-del-peligro-de-muerte}{%
\subsection{Mensaje sobre la salvación de Pablo y sus colaboradores del
peligro de
muerte}\label{mensaje-sobre-la-salvaciuxf3n-de-pablo-y-sus-colaboradores-del-peligro-de-muerte}}

\bibverse{8} Porque hermanos, no queremos que ignoréis de nuestra
tribulación que nos fué hecha en Asia; que sobremanera fuimos cargados
sobre nuestras fuerzas de tal manera que estuviésemos en duda de la
vida. \bibverse{9} Mas nosotros tuvimos en nosotros mismos respuesta de
muerte, para que no confiemos en nosotros mismos, sino en Dios que
levanta los muertos: \bibverse{10} El cual nos libró, y libra de tanta
muerte; en el cual esperamos que aun nos librará; \bibverse{11}
Ayudándonos también vosotros con oración por nosotros, para que por la
merced hecha á nos por respeto de muchos, por muchos sean hechas gracias
por nosotros. \footnote{\textbf{1:11} Fil 1,19}

\hypertarget{el-modo-de-vida-honesto-del-apuxf3stol-y-su-veracidad-en-la-correspondencia}{%
\subsection{El modo de vida honesto del apóstol y su veracidad en la
correspondencia}\label{el-modo-de-vida-honesto-del-apuxf3stol-y-su-veracidad-en-la-correspondencia}}

\bibverse{12} Porque nuestra gloria es esta: el testimonio de nuestra
conciencia, que con simplicidad y sinceridad de Dios, no con sabiduría
carnal, mas con la gracia de Dios, hemos conversado en el mundo, y muy
más con vosotros. \footnote{\textbf{1:12} 2Cor 2,17; Heb 13,18; 1Cor
  1,17} \bibverse{13} Porque no os escribimos otras cosas de las que
leéis, ó también conocéis: y espero que aun hasta el fin las conoceréis:
\bibverse{14} Como también en parte habéis conocido que somos vuestra
gloria, así como también vosotros la nuestra, para el día del Señor
Jesús. \footnote{\textbf{1:14} 2Cor 5,12; Fil 2,16}

\hypertarget{el-relato-del-apuxf3stol-del-cambio-en-sus-planes-de-viaje-indicaciuxf3n-de-su-fiabilidad-como-apuxf3stol-de-cristo-y-de-dios-fiel}{%
\subsection{El relato del apóstol del cambio en sus planes de viaje;
Indicación de su fiabilidad como apóstol de Cristo y de Dios
fiel}\label{el-relato-del-apuxf3stol-del-cambio-en-sus-planes-de-viaje-indicaciuxf3n-de-su-fiabilidad-como-apuxf3stol-de-cristo-y-de-dios-fiel}}

\bibverse{15} Y con esta confianza quise primero ir á vosotros, para que
tuvieseis una segunda gracia; \bibverse{16} Y por vosotros pasar á
Macedonia, y de Macedonia venir otra vez á vosotros, y ser vuelto de
vosotros á Judea. \bibverse{17} Así que, pretendiendo esto, ¿usé quizá
de liviandad? ó lo que pienso hacer, ¿piénsolo según la carne, para que
haya en mí Sí y No? \bibverse{18} Antes, Dios fiel sabe que nuestra
palabra para con vosotros no es Sí y No.~\bibverse{19} Porque el Hijo de
Dios, Jesucristo, que por nosotros ha sido entre vosotros predicado, por
mí y Silvano y Timoteo, no ha sido Sí y No; mas ha sido Sí en él.
\footnote{\textbf{1:19} Hech 18,5} \bibverse{20} Porque todas las
promesas de Dios son en él Sí, y en él Amén, por nosotros á gloria de
Dios. \footnote{\textbf{1:20} Apoc 3,14}

\bibverse{21} Y el que nos confirma con vosotros en Cristo, y el que nos
ungió, es Dios; \footnote{\textbf{1:21} 1Jn 2,27} \bibverse{22} El cual
también nos ha sellado, y dado la prenda del Espíritu en nuestros
corazones. \footnote{\textbf{1:22} 2Cor 5,5; Rom 8,16; Efes 1,13}

\hypertarget{declaraciuxf3n-de-la-verdadera-razuxf3n-por-la-que-pablo-no-vino-a-corinto}{%
\subsection{Declaración de la verdadera razón por la que Pablo no vino a
Corinto}\label{declaraciuxf3n-de-la-verdadera-razuxf3n-por-la-que-pablo-no-vino-a-corinto}}

\bibverse{23} Mas yo llamo á Dios por testigo sobre mi alma, que por ser
indulgente con vosotros no he pasado todavía á Corinto. \bibverse{24} No
que nos enseñoreemos de vuestra fe, mas somos ayudadores de vuestro
gozo: porque por la fe estáis firmes. \footnote{\textbf{1:24} 1Pe 5,3;
  2Cor 4,5}

\hypertarget{section-1}{%
\section{2}\label{section-1}}

\bibverse{1} ESTO pues determiné para conmigo, no venir otra vez á
vosotros con tristeza. \footnote{\textbf{2:1} 1Cor 4,21; 2Cor 12,21}
\bibverse{2} Porque si yo os contristo, ¿quién será luego el que me
alegrará, sino aquel á quien yo contristare? \bibverse{3} Y esto mismo
os escribí, porque cuando llegare no tenga tristeza sobre tristeza de
los que me debiera gozar; confiando en vosotros todos que mi gozo es el
de todos vosotros. \bibverse{4} Porque por la mucha tribulación y
angustia del corazón os escribí con muchas lágrimas; no para que fueseis
contristados, mas para que supieseis cuánto más amor tengo para con
vosotros.

\hypertarget{eliminaciuxf3n-de-la-brecha-entre-pablo-y-los-corintios-recomendaciuxf3n-de-indulgencia-contra-el-malhechor-arrepentido}{%
\subsection{Eliminación de la brecha entre Pablo y los Corintios;
Recomendación de indulgencia contra el malhechor
arrepentido}\label{eliminaciuxf3n-de-la-brecha-entre-pablo-y-los-corintios-recomendaciuxf3n-de-indulgencia-contra-el-malhechor-arrepentido}}

\bibverse{5} Que si alguno me contristó, no me contristó á mí, sino en
parte, por no cargaros, á todos vosotros. \bibverse{6} Bástale al tal
esta reprensión hecha de muchos; \bibverse{7} Así que, al contrario,
vosotros más bien lo perdonéis y consoléis, porque no sea el tal
consumido de demasiada tristeza. \bibverse{8} Por lo cual os ruego que
confirméis el amor para con él. \bibverse{9} Porque también por este fin
os escribí, para tener experiencia de vosotros si sois obedientes en
todo. \bibverse{10} Y al que vosotros perdonareis, yo también: porque
también yo lo que he perdonado, si algo he perdonado, por vosotros lo he
hecho en persona de Cristo; \footnote{\textbf{2:10} Juan 20,23}
\bibverse{11} Porque no seamos engañados de Satanás: pues no ignoramos
sus maquinaciones. \footnote{\textbf{2:11} Luc 22,31; 1Pe 5,8}

\hypertarget{las-experiencias-del-apuxf3stol-en-troas-y-macedonia-su-alabanza-a-dios-por-el-efecto-victorioso-de-la-proclamaciuxf3n-de-la-salvaciuxf3n}{%
\subsection{Las experiencias del apóstol en Troas y Macedonia; Su
alabanza a Dios por el efecto victorioso de la proclamación de la
salvación}\label{las-experiencias-del-apuxf3stol-en-troas-y-macedonia-su-alabanza-a-dios-por-el-efecto-victorioso-de-la-proclamaciuxf3n-de-la-salvaciuxf3n}}

\bibverse{12} Cuando vine á Troas para el evangelio de Cristo, aunque me
fué abierta puerta en el Señor, \footnote{\textbf{2:12} Hech 14,27; 1Cor
  16,9} \bibverse{13} No tuve reposo en mi espíritu, por no haber
hallado á Tito mi hermano: así, despidiéndome de ellos, partí para
Macedonia. \footnote{\textbf{2:13} Hech 20,1; 2Cor 7,6}

\bibverse{14} Mas á Dios gracias, el cual hace que siempre triunfemos en
Cristo Jesús, y manifiesta el olor de su conocimiento por nosotros en
todo lugar. \bibverse{15} Porque para Dios somos buen olor de Cristo en
los que se salvan, y en los que se pierden: \footnote{\textbf{2:15} Éxod
  29,18; 1Cor 1,18} \bibverse{16} A éstos ciertamente olor de muerte
para muerte; y á aquéllos olor de vida para vida. Y para estas cosas
¿quién es suficiente? \footnote{\textbf{2:16} Luc 2,34; 2Cor 3,5}
\bibverse{17} Porque no somos como muchos, mercaderes falsos de la
palabra de Dios: antes con sinceridad, como de Dios, delante de Dios,
hablamos en Cristo. \footnote{\textbf{2:17} 2Cor 1,12; 2Cor 4,2; 1Pe
  4,11}

\hypertarget{la-iglesia-de-corinto-como-carta-de-recomendaciuxf3n-para-pablo-y-dios-como-base-segura-de-confianza-para-el-apuxf3stol}{%
\subsection{La iglesia de Corinto como carta de recomendación para Pablo
y Dios como base segura de confianza para el
apóstol}\label{la-iglesia-de-corinto-como-carta-de-recomendaciuxf3n-para-pablo-y-dios-como-base-segura-de-confianza-para-el-apuxf3stol}}

\hypertarget{section-2}{%
\section{3}\label{section-2}}

\bibverse{1} ¿COMENZAMOS otra vez á alabarnos á nosotros mismos? ¿ó
tenemos necesidad, como algunos, de letras de recomendación para
vosotros, ó de recomendación de vosotros? \footnote{\textbf{3:1} 2Cor
  5,12} \bibverse{2} Nuestras letras sois vosotros, escritas en nuestros
corazones, sabidas y leídas de todos los hombres; \footnote{\textbf{3:2}
  1Cor 9,2} \bibverse{3} Siendo manifiesto que sois letra de Cristo
administrada de nosotros, escrita no con tinta, mas con el Espíritu del
Dios vivo; no en tablas de piedra, sino en tablas de carne del corazón.
\footnote{\textbf{3:3} Éxod 24,12}

\bibverse{4} Y tal confianza tenemos por Cristo para con Dios:
\bibverse{5} No que seamos suficientes de nosotros mismos para pensar
algo como de nosotros mismos, sino que nuestra suficiencia es de Dios;
\footnote{\textbf{3:5} 2Cor 2,16}

\hypertarget{la-gloria-del-nuevo-pacto-y-el-ministerio-apostuxf3lico-sobre-el-antiguo-pacto-y-el-ministerio-de-moisuxe9s}{%
\subsection{La gloria del nuevo pacto y el ministerio apostólico sobre
el antiguo pacto y el ministerio de
Moisés}\label{la-gloria-del-nuevo-pacto-y-el-ministerio-apostuxf3lico-sobre-el-antiguo-pacto-y-el-ministerio-de-moisuxe9s}}

\bibverse{6} El cual asimismo nos hizo ministros suficientes de un nuevo
pacto: no de la letra, mas del espíritu; porque la letra mata, mas el
espíritu vivifica. \footnote{\textbf{3:6} Jer 31,31; 1Cor 11,25; Rom
  7,6; Juan 6,63}

\bibverse{7} Y si el ministerio de muerte en la letra grabado en
piedras, fué con gloria, tanto que los hijos de Israel no pudiesen poner
los ojos en la faz de Moisés á causa de la gloria de su rostro, la cual
había de perecer, \footnote{\textbf{3:7} Éxod 34,29-35} \bibverse{8}
¿Cómo no será más bien con gloria el ministerio del espíritu?
\footnote{\textbf{3:8} Gal 3,2; Gal 3,5} \bibverse{9} Porque si el
ministerio de condenación fué con gloria, mucho más abundará en gloria
el ministerio de justicia. \footnote{\textbf{3:9} Deut 27,26; Rom 1,17;
  Rom 3,21} \bibverse{10} Porque aun lo que fué glorioso, no es glorioso
en esta parte, en comparación de la excelente gloria. \bibverse{11}
Porque si lo que perece tuvo gloria, mucho más será en gloria lo que
permanece.

\hypertarget{la-diferencia-entre-los-dos-tipos-de-servicios-es-evidente-tanto-en-sus-servidores-como-en-sus-efectos}{%
\subsection{La diferencia entre los dos tipos de servicios es evidente
tanto en sus servidores como en sus
efectos}\label{la-diferencia-entre-los-dos-tipos-de-servicios-es-evidente-tanto-en-sus-servidores-como-en-sus-efectos}}

\bibverse{12} Así que, teniendo tal esperanza, hablamos con mucha
confianza; \bibverse{13} Y no como Moisés, que ponía un velo sobre su
faz, para que los hijos de Israel no pusiesen los ojos en el fin de lo
que había de ser abolido. \bibverse{14} Empero los sentidos de ellos se
embotaron; porque hasta el día de hoy les queda el mismo velo no
descubierto en la lección del antiguo testamento, el cual por Cristo es
quitado. \footnote{\textbf{3:14} Rom 11,25; Hech 28,27} \bibverse{15} Y
aun hasta el día de hoy, cuando Moisés es leído, el velo está puesto
sobre el corazón de ellos. \bibverse{16} Mas cuando se convirtieren al
Señor, el velo se quitará. \bibverse{17} Porque el Señor es el Espíritu;
y donde hay el Espíritu del Señor, allí hay libertad. \bibverse{18} Por
tanto, nosotros todos, mirando á cara descubierta como en un espejo la
gloria del Señor, somos transformados de gloria en gloria en la misma
semejanza, como por el Espíritu del Señor. \footnote{\textbf{3:18} 2Cor
  4,6}

\hypertarget{pablo-y-sus-seguidores-aparecen-como-verdaderos-mensajeros-de-cristo-con-valentuxeda-veracidad-e-iluminaciuxf3n-divina}{%
\subsection{Pablo y sus seguidores aparecen como verdaderos mensajeros
de Cristo con valentía, veracidad e iluminación
divina}\label{pablo-y-sus-seguidores-aparecen-como-verdaderos-mensajeros-de-cristo-con-valentuxeda-veracidad-e-iluminaciuxf3n-divina}}

\hypertarget{section-3}{%
\section{4}\label{section-3}}

\bibverse{1} POR lo cual teniendo nosotros esta administración según la
misericordia que hemos alcanzado, no desmayamos; \footnote{\textbf{4:1}
  2Cor 3,6; 1Cor 7,25} \bibverse{2} Antes quitamos los escondrijos de
vergüenza, no andando con astucia, ni adulterando la palabra de Dios,
sino por manifestación de la verdad encomendándonos á nosotros mismos á
toda conciencia humana delante de Dios. \footnote{\textbf{4:2} 2Cor
  2,17; 1Tes 2,5} \bibverse{3} Que si nuestro evangelio está aún
encubierto, entre los que se pierden está encubierto: \footnote{\textbf{4:3}
  1Cor 1,18} \bibverse{4} En los cuales el dios de este siglo cegó los
entendimientos de los incrédulos, para que no les resplandezca la lumbre
del evangelio de la gloria de Cristo, el cual es la imagen de Dios.
\footnote{\textbf{4:4} Heb 1,3} \bibverse{5} Porque no nos predicamos á
nosotros mismos, sino á Jesucristo, el Señor; y nosotros vuestros
siervos por Jesús. \footnote{\textbf{4:5} 2Cor 1,24} \bibverse{6} Porque
Dios, que mandó que de las tinieblas resplandeciese la luz, es el que
resplandeció en nuestros corazones, para iluminación del conocimiento de
la gloria de Dios en la faz de Jesucristo. \footnote{\textbf{4:6} Gén
  1,3; 2Cor 3,18}

\hypertarget{el-sufrimiento-externo-de-los-apuxf3stoles-ademuxe1s-de-su-confianza-en-la-fe}{%
\subsection{El sufrimiento externo de los apóstoles además de su
confianza en la
fe}\label{el-sufrimiento-externo-de-los-apuxf3stoles-ademuxe1s-de-su-confianza-en-la-fe}}

\bibverse{7} Tenemos empero este tesoro en vasos de barro, para que la
alteza del poder sea de Dios, y no de nosotros: \footnote{\textbf{4:7}
  1Cor 4,11-13; 2Cor 11,23-27} \bibverse{8} Estando atribulados en todo,
mas no angustiados; en apuros, mas no desesperamos; \bibverse{9}
Perseguidos, mas no desamparados; abatidos, mas no perecemos;
\bibverse{10} Llevando siempre por todas partes la muerte de Jesús en el
cuerpo, para que también la vida de Jesús sea manifestada en nuestros
cuerpos. \footnote{\textbf{4:10} 1Cor 15,31; Gal 6,17} \bibverse{11}
Porque nosotros que vivimos, siempre estamos entregados á muerte por
Jesús, para que también la vida de Jesús sea manifestada en nuestra
carne mortal. \footnote{\textbf{4:11} Rom 8,36} \bibverse{12} De manera
que la muerte obra en nosotros, y en vosotros la vida.

\bibverse{13} Empero teniendo el mismo espíritu de fe, conforme á lo que
está escrito: Creí, por lo cual también hablé: nosotros también creemos,
por lo cual también hablamos; \bibverse{14} Estando ciertos que el que
levantó al Señor Jesús, á nosotros también nos levantará por Jesús, y
nos pondrá con vosotros. \footnote{\textbf{4:14} 1Cor 6,14}
\bibverse{15} Porque todas estas cosas padecemos por vosotros, para que
abundando la gracia por muchos, en el hacimiento de gracias sobreabunde
á gloria de Dios. \footnote{\textbf{4:15} 2Cor 1,6; 2Cor 1,11}

\hypertarget{la-renovaciuxf3n-del-hombre-espiritual-tiene-lugar-en-la-muerte-del-hombre-exterior}{%
\subsection{La renovación del hombre espiritual tiene lugar en la muerte
del hombre
exterior}\label{la-renovaciuxf3n-del-hombre-espiritual-tiene-lugar-en-la-muerte-del-hombre-exterior}}

\bibverse{16} Por tanto, no desmayamos: antes aunque este nuestro hombre
exterior se va desgastando, el interior empero se renueva de día en día.
\footnote{\textbf{4:16} Efes 3,16} \bibverse{17} Porque lo que al
presente es momentáneo y leve de nuestra tribulación, nos obra un
sobremanera alto y eterno peso de gloria; \footnote{\textbf{4:17} Rom
  8,17-18; 1Pe 1,6} \bibverse{18} No mirando nosotros á las cosas que se
ven, sino á las que no se ven: porque las cosas que se ven son
temporales, mas las que no se ven son eternas. \footnote{\textbf{4:18}
  Heb 11,1}

\hypertarget{la-esperanza-y-el-anhelo-de-pablo-por-la-corporalidad-celestial-y-el-hogar-celestial}{%
\subsection{La esperanza y el anhelo de Pablo por la corporalidad
celestial y el hogar
celestial}\label{la-esperanza-y-el-anhelo-de-pablo-por-la-corporalidad-celestial-y-el-hogar-celestial}}

\hypertarget{section-4}{%
\section{5}\label{section-4}}

\bibverse{1} PORQUE sabemos, que si la casa terrestre de nuestra
habitación se deshiciere, tenemos de Dios un edificio, una casa no hecha
de manos, eterna en los cielos. \footnote{\textbf{5:1} Job 4,19; 2Pe
  1,14} \bibverse{2} Y por esto también gemimos, deseando ser
sobrevestidos de aquella nuestra habitación celestial; \bibverse{3}
Puesto que en verdad habremos sido hallados vestidos, y no desnudos.
\bibverse{4} Porque asimismo los que estamos en este tabernáculo,
gemimos agravados; porque no quisiéramos ser desnudados, sino
sobrevestidos, para que lo mortal sea absorbido por la vida. \footnote{\textbf{5:4}
  1Cor 15,51-53} \bibverse{5} Mas el que nos hizo para esto mismo, es
Dios; el cual nos ha dado la prenda del Espíritu. \footnote{\textbf{5:5}
  2Cor 1,22; Rom 8,16; Rom 8,23; Efes 1,13-14}

\bibverse{6} Así que vivimos confiados siempre, y sabiendo, que entre
tanto que estamos en el cuerpo, peregrinamos ausentes del Señor;
\footnote{\textbf{5:6} Heb 11,13} \bibverse{7} (Porque por fe andamos,
no por vista;) \footnote{\textbf{5:7} Rom 8,24; 1Pe 1,8} \bibverse{8}
Mas confiamos, y más quisiéramos partir del cuerpo, y estar presentes al
Señor. \footnote{\textbf{5:8} Fil 1,23} \bibverse{9} Por tanto
procuramos también, ó ausentes, ó presentes, serle agradables:
\footnote{\textbf{5:9} Sal 39,13} \bibverse{10} Porque es menester que
todos nosotros parezcamos ante el tribunal de Cristo, para que cada uno
reciba según lo que hubiere hecho por medio del cuerpo, ora sea bueno ó
malo. \footnote{\textbf{5:10} Juan 5,29; Hech 17,31; Rom 2,16; Rom
  14,10; Efes 6,8}

\hypertarget{comentarios-personales-especialmente-sobre-su-relaciuxf3n-con-la-comunidad}{%
\subsection{Comentarios personales, especialmente sobre su relación con
la
comunidad}\label{comentarios-personales-especialmente-sobre-su-relaciuxf3n-con-la-comunidad}}

\bibverse{11} Estando pues poseídos del temor del Señor, persuadimos á
los hombres, mas á Dios somos manifiestos; y espero que también en
vuestras conciencias somos manifiestos. \bibverse{12} No nos
encomendamos pues otra vez á vosotros, sino os damos ocasión de
gloriaros por nosotros, para que tengáis qué responder contra los que se
glorían en las apariencias, y no en el corazón. \bibverse{13} Porque si
loqueamos, es para Dios; y si estamos en seso, es para vosotros.

\hypertarget{referencia-al-contenido-peculiar-de-su-sermuxf3n-y-la-gloria-de-su-servicio-de-reconciliaciuxf3n}{%
\subsection{Referencia al contenido peculiar de su sermón y la gloria de
su servicio de
reconciliación}\label{referencia-al-contenido-peculiar-de-su-sermuxf3n-y-la-gloria-de-su-servicio-de-reconciliaciuxf3n}}

\bibverse{14} Porque el amor de Cristo nos constriñe, pensando esto: Que
si uno murió por todos, luego todos son muertos; \bibverse{15} Y por
todos murió, para que los que viven, ya no vivan para sí, mas para aquel
que murió y resucitó por ellos.

\bibverse{16} De manera que nosotros de aquí adelante á nadie conocemos
según la carne: y aun si á Cristo conocimos según la carne, empero ahora
ya no le conocemos. \bibverse{17} De modo que si alguno está en Cristo,
nueva criatura es: las cosas viejas pasaron; he aquí todas son hechas
nuevas. \footnote{\textbf{5:17} Rom 8,10; Gal 2,20; Gal 6,15; Apoc 21,5}
\bibverse{18} Y todo esto es de Dios, el cual nos reconcilió á sí por
Cristo; y nos dió el ministerio de la reconciliación. \footnote{\textbf{5:18}
  Rom 5,10} \bibverse{19} Porque ciertamente Dios estaba en Cristo
reconciliando el mundo á sí, no imputándole sus pecados, y puso en
nosotros la palabra de la reconciliación. \footnote{\textbf{5:19} Rom
  3,24-25; Col 1,19-20}

\bibverse{20} Así que, somos embajadores en nombre de Cristo, como si
Dios rogase por medio nuestro; os rogamos en nombre de Cristo:
Reconciliaos con Dios. \footnote{\textbf{5:20} Luc 10,16} \bibverse{21}
Al que no conoció pecado, hizo pecado por nosotros, para que nosotros
fuésemos hechos justicia de Dios en él. \footnote{\textbf{5:21} Is 53,6;
  Juan 8,46; Rom 1,17}

\hypertarget{pablo-como-apuxf3stol-es-ejemplar-por-su-abnegaciuxf3n-y-su-realizaciuxf3n-profesional-desinteresada-en-el-servicio-de-dios}{%
\subsection{Pablo, como apóstol, es ejemplar por su abnegación y su
realización profesional desinteresada en el servicio de
Dios}\label{pablo-como-apuxf3stol-es-ejemplar-por-su-abnegaciuxf3n-y-su-realizaciuxf3n-profesional-desinteresada-en-el-servicio-de-dios}}

\hypertarget{section-5}{%
\section{6}\label{section-5}}

\bibverse{1} Y ASÍ nosotros, como ayudadores juntamente con él, os
exhortamos también á que no recibáis en vano la gracia de Dios,
\footnote{\textbf{6:1} 2Cor 1,24} \bibverse{2} (Porque dice: En tiempo
aceptable te he oído, y en día de salud te he socorrido: he aquí ahora
el tiempo aceptable; he aquí ahora el día de salud:) \footnote{\textbf{6:2}
  Luc 4,19; Luc 4,21}

\bibverse{3} No dando á nadie ningún escándalo, porque el ministerio
nuestro no sea vituperado: \bibverse{4} Antes habiéndonos en todas cosas
como ministros de Dios, en mucha paciencia, en tribulaciones, en
necesidades, en angustias; \bibverse{5} En azotes, en cárceles, en
alborotos, en trabajos, en vigilias, en ayunos; \bibverse{6} En
castidad, en ciencia, en longanimidad, en bondad, en Espíritu Santo, en
amor no fingido; \footnote{\textbf{6:6} 1Tim 4,12} \bibverse{7} En
palabra de verdad, en potencia de Dios, en armas de justicia á diestro y
á siniestro; \footnote{\textbf{6:7} 2Cor 4,2; 1Cor 2,4; Efes 6,14-17}
\bibverse{8} Por honra y por deshonra, por infamia y por buena fama;
como engañadores, mas hombres de verdad; \bibverse{9} Como ignorados,
mas conocidos; como muriendo, mas he aquí vivimos; como castigados, mas
no muertos; \footnote{\textbf{6:9} 2Cor 4,10-11; Sal 118,18; Hech 14,19}
\bibverse{10} Como doloridos, mas siempre gozosos; como pobres, mas
enriqueciendo á muchos; como no teniendo nada, mas poseyéndolo todo.
\footnote{\textbf{6:10} Fil 4,12-13}

\hypertarget{peticiuxf3n-solemne-y-amorosa-a-los-corintios-para-la-restauraciuxf3n-completa-de-la-comuniuxf3n}{%
\subsection{Petición solemne y amorosa a los corintios para la
restauración completa de la
comunión}\label{peticiuxf3n-solemne-y-amorosa-a-los-corintios-para-la-restauraciuxf3n-completa-de-la-comuniuxf3n}}

\bibverse{11} Nuestra boca está abierta á vosotros, oh Corintios:
nuestro corazón es ensanchado. \bibverse{12} No estáis estrechos en
nosotros, mas estáis estrechos en vuestras propias entrañas.
\bibverse{13} Pues, para corresponder al propio modo (como á hijos
hablo), ensanchaos también vosotros. \footnote{\textbf{6:13} 1Cor 4,14}

\hypertarget{advertencia-contra-los-seres-paganos-y-demanda-de-perfecta-santificaciuxf3n}{%
\subsection{Advertencia contra los seres paganos y demanda de perfecta
santificación}\label{advertencia-contra-los-seres-paganos-y-demanda-de-perfecta-santificaciuxf3n}}

\bibverse{14} No os juntéis en yugo con los infieles: porque ¿qué
compañía tiene la justicia con la injusticia? ¿y qué comunión la luz con
las tinieblas? \footnote{\textbf{6:14} Efes 5,11} \bibverse{15} ¿Y qué
concordia Cristo con Belial? ¿ó qué parte el fiel con el infiel?
\bibverse{16} ¿Y qué concierto el templo de Dios con los ídolos? porque
vosotros sois el templo del Dios viviente, como Dios dijo: Habitaré y
andaré en ellos; y seré el Dios de ellos, y ellos serán mi pueblo.
\footnote{\textbf{6:16} 1Cor 3,16} \bibverse{17} Por lo cual salid de en
medio de ellos, y apartaos, dice el Señor, y no toquéis lo inmundo; y yo
os recibiré, \footnote{\textbf{6:17} Apoc 18,14} \bibverse{18} Y seré á
vosotros Padre, y vosotros me seréis á mí hijos é hijas, dice el Señor
Todopoderoso.

\hypertarget{section-6}{%
\section{7}\label{section-6}}

\bibverse{1} ASÍ que, amados, pues tenemos tales promesas, limpiémonos
de toda inmundicia de carne y de espíritu, perfeccionando la
santificación en temor de Dios.

\hypertarget{la-peticiuxf3n-del-apuxf3stol-de-amor-afirmaciuxf3n-de-amor-y-testimonio-de-confianza}{%
\subsection{La petición del apóstol de amor, afirmación de amor y
testimonio de
confianza}\label{la-peticiuxf3n-del-apuxf3stol-de-amor-afirmaciuxf3n-de-amor-y-testimonio-de-confianza}}

\bibverse{2} Admitidnos: á nadie hemos injuriado, á nadie hemos
corrompido, á nadie hemos engañado. \footnote{\textbf{7:2} 2Cor 12,17;
  Hech 20,33} \bibverse{3} No para condenaros lo digo; que ya he dicho
antes que estáis en nuestros corazones, para morir y para vivir
juntamente. \footnote{\textbf{7:3} 2Cor 6,11-13; Rom 6,8} \bibverse{4}
Mucha confianza tengo de vosotros, tengo de vosotros mucha gloria; lleno
estoy de consolación, sobreabundo de gozo en todas nuestras
tribulaciones.

\hypertarget{alegruxeda-del-apuxf3stol-por-la-llegada-y-el-mensaje-de-tito}{%
\subsection{Alegría del apóstol por la llegada y el mensaje de
Tito}\label{alegruxeda-del-apuxf3stol-por-la-llegada-y-el-mensaje-de-tito}}

\bibverse{5} Porque aun cuando vinimos á Macedonia, ningún reposo tuvo
nuestra carne; antes, en todo fuimos atribulados: de fuera, cuestiones;
de dentro, temores. \footnote{\textbf{7:5} Hech 20,1-2} \bibverse{6} Mas
Dios, que consuela á los humildes, nos consoló con la venida de Tito:
\footnote{\textbf{7:6} 2Cor 2,13; 2Cor 4,8} \bibverse{7} Y no sólo con
su venida, sino también con la consolación con que él fué consolado
acerca de vosotros, haciéndonos saber vuestro deseo grande, vuestro
lloro, vuestro celo por mí, para que así me gozase más.

\hypertarget{el-gozo-del-apuxf3stol-por-el-efecto-saludable-de-la-carta-penal-por-el-entendimiento-completamente-restaurado-y-por-el-informe-favorable-de-tito}{%
\subsection{El gozo del apóstol por el efecto saludable de la carta
penal, por el entendimiento completamente restaurado y por el informe
favorable de
Tito}\label{el-gozo-del-apuxf3stol-por-el-efecto-saludable-de-la-carta-penal-por-el-entendimiento-completamente-restaurado-y-por-el-informe-favorable-de-tito}}

\bibverse{8} Porque aunque os contristé por la carta, no me arrepiento,
bien que me arrepentí; porque veo que aquella carta, aunque por algún
tiempo os contristó, \footnote{\textbf{7:8} 2Cor 2,4} \bibverse{9} Ahora
me gozo, no porque hayáis sido contristados, sino porque fuisteis
contristados para arrepentimiento; porque habéis sido contristados según
Dios, para que ninguna pérdida padecieseis por nuestra parte.
\bibverse{10} Porque el dolor que es según Dios, obra arrepentimiento
saludable, de que no hay que arrepentirse; mas el dolor del siglo obra
muerte. \bibverse{11} Porque he aquí, esto mismo que según Dios fuisteis
contristados, cuánta solicitud ha obrado en vosotros, y aun defensa, y
aun enojo, y aun temor, y aun gran deseo, y aun celo, y aun vindicación.
En todo os habéis mostrado limpios en el negocio. \bibverse{12} Así que,
aunque os escribí, no fué por causa del que hizo la injuria, ni por
causa del que la padeció, mas para que os fuese manifiesta nuestra
solicitud que tenemos por vosotros delante de Dios. \bibverse{13} Por
tanto, tomamos consolación de vuestra consolación: empero mucho más nos
gozamos por el gozo de Tito, que haya sido recreado su espíritu de todos
vosotros.

\bibverse{14} Pues si algo me he gloriado para con él de vosotros, no he
sido avergonzado; antes, como todo lo que habíamos dicho de vosotros era
con verdad, así también nuestra gloria delante de Tito fué hallada
verdadera. \bibverse{15} Y sus entrañas son más abundantes para con
vosotros, cuando se acuerda de la obediencia de todos vosotros, de cómo
lo recibisteis con temor y temblor. \bibverse{16} Me gozo de que en todo
estoy confiado de vosotros.

\hypertarget{el-gratificante-ejemplar-uxe9xito-de-la-colecciuxf3n-con-las-comunidades-macedonias}{%
\subsection{El gratificante (ejemplar) éxito de la colección con las
comunidades
macedonias}\label{el-gratificante-ejemplar-uxe9xito-de-la-colecciuxf3n-con-las-comunidades-macedonias}}

\hypertarget{section-7}{%
\section{8}\label{section-7}}

\bibverse{1} ASIMISMO, hermanos, os hacemos saber la gracia de Dios que
ha sido dada á las iglesias de Macedonia: \footnote{\textbf{8:1} Rom
  15,26} \bibverse{2} Que en grande prueba de tribulación, la abundancia
de su gozo y su profunda pobreza abundaron en riquezas de su bondad.
\bibverse{3} Pues de su grado han dado conforme á sus fuerzas, yo
testifico, y aun sobre sus fuerzas; \bibverse{4} Pidiéndonos con muchos
ruegos, que aceptásemos la gracia y la comunicación del servicio para
los santos. \bibverse{5} Y no como lo esperábamos, mas aun á sí mismos
se dieron primeramente al Señor, y á nosotros por la voluntad de Dios.

\hypertarget{invitaciuxf3n-a-los-corintios-a-participar-activamente-en-la-colecta}{%
\subsection{Invitación a los corintios a participar activamente en la
colecta}\label{invitaciuxf3n-a-los-corintios-a-participar-activamente-en-la-colecta}}

\bibverse{6} De manera que exhortamos á Tito, que como comenzó antes,
así también acabe esta gracia entre vosotros también. \bibverse{7} Por
tanto, como en todo abundáis, en fe, y en palabra, y en ciencia, y en
toda solicitud, y en vuestro amor para con nosotros, que también
abundéis en esta gracia. \footnote{\textbf{8:7} 1Cor 1,5; 1Cor 16,1-2}

\bibverse{8} No hablo como quien manda, sino para poner á prueba, por la
eficacia de otros, la sinceridad también de la caridad vuestra.
\bibverse{9} Porque ya sabéis la gracia de nuestro Señor Jesucristo, que
por amor de vosotros se hizo pobre, siendo rico; para que vosotros con
su pobreza fueseis enriquecidos. \bibverse{10} Y en esto doy mi consejo;
porque esto os conviene á vosotros, que comenzasteis antes, no sólo á
hacerlo, mas aun á quererlo desde el año pasado. \bibverse{11} Ahora
pues, llevad también á cabo el hecho, para que como estuvisteis prontos
á querer, así también lo estéis en cumplir conforme á lo que tenéis.
\bibverse{12} Porque si primero hay la voluntad pronta, será acepta por
lo que tiene, no por lo que no tiene. \footnote{\textbf{8:12} Prov
  3,27-28; Mar 12,43} \bibverse{13} Porque no digo esto para que haya
para otros desahogo, y para vosotros apretura; \bibverse{14} Sino para
que en este tiempo, con igualdad, vuestra abundancia supla la falta de
ellos, para que también la abundancia de ellos supla vuestra falta,
porque haya igualdad; \bibverse{15} Como está escrito: El que recogió
mucho, no tuvo más; y el que poco, no tuvo menos.

\hypertarget{recomendaciuxf3n-de-tito-y-los-otros-dos-diputados-de-pablo}{%
\subsection{Recomendación de Tito y los otros dos diputados de
Pablo}\label{recomendaciuxf3n-de-tito-y-los-otros-dos-diputados-de-pablo}}

\bibverse{16} Empero gracias á Dios que dió la misma solicitud por
vosotros en el corazón de Tito. \bibverse{17} Pues á la verdad recibió
la exhortación; mas estando también muy solícito, de su voluntad partió
para vosotros. \bibverse{18} Y enviamos juntamente con él al hermano
cuya alabanza en el evangelio es por todas las iglesias; \footnote{\textbf{8:18}
  2Cor 12,18} \bibverse{19} Y no sólo esto, mas también fué ordenado por
las iglesias el compañero de nuestra peregrinación para llevar esta
gracia, que es administrada de nosotros para gloria del mismo Señor, y
para demostrar vuestro pronto ánimo: \footnote{\textbf{8:19} Gal 2,10}
\bibverse{20} Evitando que nadie nos vitupere en esta abundancia que
ministramos; \bibverse{21} Procurando las cosas honestas, no sólo
delante del Señor, mas aun delante de los hombres. \bibverse{22}
Enviamos también con ellos á nuestro hermano, al cual muchas veces hemos
experimentado diligente, mas ahora mucho más con la mucha confianza que
tiene en vosotros. \bibverse{23} Ora en orden á Tito, es mi compañero y
coadjutor para con vosotros; ó acerca de nuestros hermanos, los
mensajeros son de las iglesias, y la gloria de Cristo. \footnote{\textbf{8:23}
  2Cor 7,13; 2Cor 12,18} \bibverse{24} Mostrad pues, para con ellos á la
faz de las iglesias la prueba de vuestro amor, y de nuestra gloria
acerca de vosotros. \footnote{\textbf{8:24} 2Cor 7,14}

\hypertarget{lo-que-pablo-ha-elogiado-hasta-ahora-de-los-corintios-y-ahora-espera-y-quuxe9-razones-lo-han-determinado-a-enviar-a-los-hermanos-por-delante}{%
\subsection{Lo que Pablo ha elogiado hasta ahora de los corintios y
ahora espera y qué razones lo han determinado a enviar a los hermanos
por
delante}\label{lo-que-pablo-ha-elogiado-hasta-ahora-de-los-corintios-y-ahora-espera-y-quuxe9-razones-lo-han-determinado-a-enviar-a-los-hermanos-por-delante}}

\hypertarget{section-8}{%
\section{9}\label{section-8}}

\bibverse{1} PORQUE cuanto á la suministración para los santos, por
demás me es escribiros; \bibverse{2} Pues conozco vuestro pronto ánimo,
del cual me glorío yo entre los de Macedonia, que Acaya está apercibida
desde el año pasado; y vuestro ejemplo ha estimulado á muchos.
\footnote{\textbf{9:2} 2Cor 8,19} \bibverse{3} Mas he enviado los
hermanos, porque nuestra gloria de vosotros no sea vana en esta parte;
para que, como lo he dicho, estéis apercibidos; \bibverse{4} No sea que,
si vinieren conmigo Macedonios, y os hallaren desapercibidos, nos
avergoncemos nosotros, por no decir vosotros, de este firme gloriarnos.
\bibverse{5} Por tanto, tuve por cosa necesaria exhortar á los hermanos
que fuesen primero á vosotros, y apresten primero vuestra bendición
antes prometida, para que esté aparejada como de bendición, y no como de
mezquindad.

\hypertarget{otra-invitaciuxf3n-a-participar-activamente-en-la-colecciuxf3n-en-referencia-a-los-efectos-benuxe9ficos-de-la-obra-de-amor}{%
\subsection{Otra invitación a participar activamente en la colección en
referencia a los efectos benéficos de la obra de
amor}\label{otra-invitaciuxf3n-a-participar-activamente-en-la-colecciuxf3n-en-referencia-a-los-efectos-benuxe9ficos-de-la-obra-de-amor}}

\bibverse{6} Esto empero digo: El que siembra escasamente, también
segará escasamente; y el que siembra en bendiciones, en bendiciones
también segará. \bibverse{7} Cada uno dé como propuso en su corazón: no
con tristeza, ó por necesidad; porque Dios ama el dador alegre.
\footnote{\textbf{9:7} Rom 12,8} \bibverse{8} Y poderoso es Dios para
hacer que abunde en vosotros toda gracia; á fin de que, teniendo siempre
en todas las cosas todo lo que basta, abundéis para toda buena obra:
\bibverse{9} Como está escrito: Derramó, dió á los pobres; su justicia
permanece para siempre.

\bibverse{10} Y el que da simiente al que siembra, también dará pan para
comer, y multiplicará vuestra sementera, y aumentará los crecimientos de
los frutos de vuestra justicia; \bibverse{11} Para que estéis
enriquecidos en todo para toda bondad, la cual obra por nosotros
hacimiento de gracias á Dios. \bibverse{12} Porque la suministración de
este servicio, no solamente suple lo que á los santos falta, sino
también abunda en muchos hacimientos de gracias á Dios: \bibverse{13}
Que por la experiencia de esta suministración glorifican á Dios por la
obediencia que profesáis al evangelio de Cristo, y por la bondad de
contribuir para ellos y para todos; \bibverse{14} Asimismo por la
oración de ellos á favor vuestro, los cuales os quieren á causa de la
eminente gracia de Dios en vosotros. \bibverse{15} Gracias á Dios por su
don inefable.

\hypertarget{en-contraste-con-la-acusaciuxf3n-de-debilidad-de-caruxe1cter-y-cambio-carnal-pablo-seuxf1ala-el-poder-probado-y-comprobado-de-su-trabajo-a-sus-oponentes}{%
\subsection{En contraste con la acusación de debilidad de carácter y
cambio carnal, Pablo señala el poder probado y comprobado de su trabajo
a sus
oponentes}\label{en-contraste-con-la-acusaciuxf3n-de-debilidad-de-caruxe1cter-y-cambio-carnal-pablo-seuxf1ala-el-poder-probado-y-comprobado-de-su-trabajo-a-sus-oponentes}}

\hypertarget{section-9}{%
\section{10}\label{section-9}}

\bibverse{1} EMPERO yo Pablo, os ruego por la mansedumbre y modestia de
Cristo, yo que presente ciertamente soy bajo entre vosotros, mas ausente
soy confiado entre vosotros: \bibverse{2} Ruego pues, que cuando
estuviere presente, no tenga que ser atrevido con la confianza con que
estoy en ánimo de ser resuelto para con algunos, que nos tienen como si
anduviésemos según la carne. \footnote{\textbf{10:2} 2Cor 13,1-2; 1Cor
  4,21} \bibverse{3} Pues aunque andamos en la carne, no militamos según
la carne, \bibverse{4} (Porque las armas de nuestra milicia no son
carnales, sino poderosas en Dios para la destrucción de fortalezas;)
\bibverse{5} Destruyendo consejos, y toda altura que se levanta contra
la ciencia de Dios, y cautivando todo intento á la obediencia de Cristo;
\bibverse{6} Y estando prestos para castigar toda desobediencia, cuando
vuestra obediencia fuere cumplida.

\hypertarget{el-derecho-del-apuxf3stol-a-jactarse-en-su-oficio-y-defenderse-de-los-cargos-de-falta-de-valor-personal}{%
\subsection{El derecho del apóstol a jactarse en su oficio y defenderse
de los cargos de falta de valor
personal}\label{el-derecho-del-apuxf3stol-a-jactarse-en-su-oficio-y-defenderse-de-los-cargos-de-falta-de-valor-personal}}

\bibverse{7} Miráis las cosas según la apariencia. Si alguno está
confiado en sí mismo que es de Cristo, esto también piense por sí mismo,
que como él es de Cristo, así también nosotros somos de Cristo.
\bibverse{8} Porque aunque me gloríe aún un poco de nuestra potestad (la
cual el Señor nos dió para edificación y no para vuestra destrucción),
no me avergonzaré; \footnote{\textbf{10:8} 2Cor 13,10; 1Cor 5,4-5}
\bibverse{9} Porque no parezca como que os quiero espantar por cartas.
\bibverse{10} Porque á la verdad, dicen, las cartas son graves y
fuertes; mas la presencia corporal flaca, y la palabra menospreciable.
\bibverse{11} Esto piense el tal, que cuales somos en la palabra por
cartas estando ausentes, tales seremos también en hechos, estando
presentes.

\hypertarget{la-diferencia-entre-la-auto-fama-practicada-correctamente-por-pablo-y-la-presunciuxf3n-de-sus-oponentes}{%
\subsection{La diferencia entre la auto-fama practicada correctamente
por Pablo y la presunción de sus
oponentes}\label{la-diferencia-entre-la-auto-fama-practicada-correctamente-por-pablo-y-la-presunciuxf3n-de-sus-oponentes}}

\bibverse{12} Porque no osamos entremeternos ó compararnos con algunos
que se alaban á sí mismos: mas ellos, midiéndose á sí mismos por sí
mismos, y comparándose consigo mismos no son juiciosos. \footnote{\textbf{10:12}
  2Cor 3,1; 2Cor 5,12} \bibverse{13} Nosotros empero, no nos gloriaremos
fuera de nuestra medida, sino conforme á la medida de la regla, de la
medida que Dios nos repartió, para llegar aun hasta vosotros.
\footnote{\textbf{10:13} Rom 12,3; Rom 15,20; Gal 2,7} \bibverse{14}
Porque no nos extendemos sobre nuestra medida, como si no llegásemos
hasta vosotros: porque también hasta vosotros hemos llegado en el
evangelio de Cristo: \bibverse{15} No gloriándonos fuera de nuestra
medida en trabajos ajenos; mas teniendo esperanza del crecimiento de
vuestra fe, que seremos muy engrandecidos entre vosotros, conforme á
nuestra regla. \bibverse{16} Y que anunciaremos el evangelio en los
lugares más allá de vosotros, sin entrar en la medida de otro para
gloriarnos en lo que ya estaba aparejado. \bibverse{17} Mas el que se
gloría, gloríese en el Señor. \footnote{\textbf{10:17} 1Cor 1,31}
\bibverse{18} Porque no el que se alaba á sí mismo, el tal es aprobado;
mas aquel á quien Dios alaba. \footnote{\textbf{10:18} 1Cor 4,5}

\hypertarget{por-quuxe9-y-con-quuxe9-derecho-se-alaba-a-suxed-mismo-el-apuxf3stol}{%
\subsection{Por qué y con qué derecho se alaba a sí mismo el
apóstol}\label{por-quuxe9-y-con-quuxe9-derecho-se-alaba-a-suxed-mismo-el-apuxf3stol}}

\hypertarget{section-10}{%
\section{11}\label{section-10}}

\bibverse{1} OJALÁ toleraseis un poco mi locura; empero toleradme.
\bibverse{2} Pues que os celo con celo de Dios; porque os he desposado á
un marido, para presentaros como una virgen pura á Cristo. \footnote{\textbf{11:2}
  Efes 5,26-27} \bibverse{3} Mas temo que como la serpiente engañó á Eva
con su astucia, sean corrompidos así vuestros sentidos en alguna manera,
de la simplicidad que es en Cristo. \footnote{\textbf{11:3} Gén 3,4; Gén
  3,13} \bibverse{4} Porque si el que viene, predicare otro Jesús que el
que hemos predicado, ó recibiereis otro espíritu del que habéis
recibido, ú otro evangelio del que habéis aceptado, lo sufrierais bien.
\footnote{\textbf{11:4} Gal 1,8-9} \bibverse{5} Cierto pienso que en
nada he sido inferior á aquellos grandes apóstoles. \footnote{\textbf{11:5}
  2Cor 12,11; 1Cor 15,10; Gal 2,6; Gal 2,9} \bibverse{6} Porque aunque
soy basto en la palabra, empero no en la ciencia: mas en todo somos ya
del todo manifiestos á vosotros. \footnote{\textbf{11:6} 1Cor 2,1-2;
  1Cor 2,13; Efes 3,4}

\hypertarget{la-gloria-de-su-eficacia-desinteresada-gratuita-en-contraste-con-los-oponentes-que-trabajan-al-servicio-de-satanuxe1s}{%
\subsection{La gloria de su eficacia desinteresada (gratuita) en
contraste con los oponentes que trabajan al servicio de
Satanás}\label{la-gloria-de-su-eficacia-desinteresada-gratuita-en-contraste-con-los-oponentes-que-trabajan-al-servicio-de-satanuxe1s}}

\bibverse{7} ¿Pequé yo humillándome á mí mismo, para que vosotros
fueseis ensalzados, porque os he predicado el evangelio de Dios de
balde? \footnote{\textbf{11:7} 2Cor 12,13; 1Cor 9,12-18; Mat 10,8}
\bibverse{8} He despojado las otras iglesias, recibiendo salario para
ministraros á vosotros. \footnote{\textbf{11:8} Fil 4,10; Fil 4,15}
\bibverse{9} Y estando con vosotros y teniendo necesidad, á ninguno fuí
carga; porque lo que me faltaba, suplieron los hermanos que vinieron de
Macedonia: y en todo me guardé de seros gravoso, y me guardaré.
\bibverse{10} Es la verdad de Cristo en mí, que esta gloria no me será
cerrada en las partes de Acaya. \bibverse{11} ¿Por qué? ¿porque no os
amo? Dios lo sabe.

\bibverse{12} Mas lo que hago, haré aún, para cortar la ocasión de
aquellos que la desean, á fin de que en aquello que se glorían, sean
hallados semejantes á nosotros. \bibverse{13} Porque éstos son falsos
apóstoles, obreros fraudulentos, transfigurándose en apóstoles de
Cristo. \bibverse{14} Y no es maravilla, porque el mismo Satanás se
transfigura en ángel de luz. \bibverse{15} Así que, no es mucho si
también sus ministros se transfiguran como ministros de justicia; cuyo
fin será conforme á sus obras.

\hypertarget{otra-peticiuxf3n-del-apuxf3stol-por-su-tonta-fama-propia}{%
\subsection{Otra petición del apóstol por su tonta fama
propia}\label{otra-peticiuxf3n-del-apuxf3stol-por-su-tonta-fama-propia}}

\bibverse{16} Otra vez digo: Que nadie me estime ser loco; de otra
manera, recibidme como á loco, para que aun me gloríe yo un poquito.
\footnote{\textbf{11:16} 2Cor 12,6} \bibverse{17} Lo que hablo, no lo
hablo según el Señor, sino como en locura, con esta confianza de gloria.
\bibverse{18} Pues que muchos se glorían según la carne, también yo me
gloriaré. \bibverse{19} Porque de buena gana toleráis los necios, siendo
vosotros sabios: \bibverse{20} Porque toleráis si alguno os pone en
servidumbre, si alguno os devora, si alguno toma, si alguno se ensalza,
si alguno os hiere en la cara.

\hypertarget{el-apuxf3stol-se-jacta-de-su-ascendencia-de-su-oficio-de-la-plenitud-de-sus-sufrimientos-en-el-servicio-apostuxf3lico}{%
\subsection{El apóstol se jacta de su ascendencia, de su oficio, de la
plenitud de sus sufrimientos en el servicio
apostólico}\label{el-apuxf3stol-se-jacta-de-su-ascendencia-de-su-oficio-de-la-plenitud-de-sus-sufrimientos-en-el-servicio-apostuxf3lico}}

\bibverse{21} Dígolo cuanto á la afrenta, como si nosotros hubiésemos
sido flacos. Empero en lo que otro tuviere osadía (hablo con locura),
también yo tengo osadía. \bibverse{22} ¿Son Hebreos? yo también. ¿Son
Israelitas? yo también. ¿Son simiente de Abraham? también yo.
\footnote{\textbf{11:22} Fil 3,5} \bibverse{23} ¿Son ministros de
Cristo? (como poco sabio hablo) yo más: en trabajos más abundante; en
azotes sin medida; en cárceles más; en muertes, muchas veces.
\footnote{\textbf{11:23} 2Cor 6,4-5; 1Cor 15,10} \bibverse{24} De los
judíos cinco veces he recibido cuarenta azotes menos uno. \footnote{\textbf{11:24}
  Deut 25,3} \bibverse{25} Tres veces he sido azotado con varas; una vez
apedreado; tres veces he padecido naufragio; una noche y un día he
estado en lo profundo de la mar; \footnote{\textbf{11:25} Hech 16,22;
  Hech 14,19} \bibverse{26} En caminos muchas veces, peligros de ríos,
peligros de ladrones, peligros de los de mi nación, peligros de los
Gentiles, peligros en la ciudad, peligros en el desierto, peligros en la
mar, peligros entre falsos hermanos; \bibverse{27} En trabajo y fatiga,
en muchas vigilias, en hambre y sed, en muchos ayunos, en frío y en
desnudez; \footnote{\textbf{11:27} 2Cor 6,5}

\bibverse{28} Sin otras cosas además, lo que sobre mí se agolpa cada
día, la solicitud de todas las iglesias. \footnote{\textbf{11:28} Hech
  20,18-21; Hech 20,31} \bibverse{29} ¿Quién enferma, y yo no enfermo?
¿Quién se escandaliza, y yo no me quemo?

\bibverse{30} Si es menester gloriarse, me gloriaré yo de lo que es de
mi flaqueza. \footnote{\textbf{11:30} 2Cor 12,5} \bibverse{31} El Dios y
Padre del Señor nuestro Jesucristo, que es bendito por siglos, sabe que
no miento. \bibverse{32} En Damasco, el gobernador de la provincia del
rey Aretas guardaba la ciudad de los Damascenos para prenderme;
\bibverse{33} Y fuí descolgado del muro en un serón por una ventana, y
escapé de sus manos.

\hypertarget{el-apuxf3stol-se-jacta-de-las-muxe1s-altas-gracias-a-travuxe9s-de-revelaciones-celestiales-y-la-muxe1s-profunda-humillaciuxf3n-a-travuxe9s-del-sufrimiento-fuxedsico}{%
\subsection{El apóstol se jacta de las más altas gracias (a través de
revelaciones celestiales) y la más profunda humillación (a través del
sufrimiento
físico)}\label{el-apuxf3stol-se-jacta-de-las-muxe1s-altas-gracias-a-travuxe9s-de-revelaciones-celestiales-y-la-muxe1s-profunda-humillaciuxf3n-a-travuxe9s-del-sufrimiento-fuxedsico}}

\hypertarget{section-11}{%
\section{12}\label{section-11}}

\bibverse{1} CIERTO no me es conveniente gloriarme; mas vendré á las
visiones y á las revelaciones del Señor. \bibverse{2} Conozco á un
hombre en Cristo, que hace catorce años (si en el cuerpo, no lo sé; si
fuera del cuerpo, no lo sé: Dios lo sabe) fué arrebatado hasta el tercer
cielo. \bibverse{3} Y conozco tal hombre, (si en el cuerpo, ó fuera del
cuerpo, no lo sé: Dios lo sabe,) \bibverse{4} Que fué arrebatado al
paraíso, donde oyó palabras secretas que el hombre no puede decir.
\bibverse{5} De este tal me gloriaré, mas de mí mismo nada me gloriaré,
sino en mis flaquezas. \footnote{\textbf{12:5} 2Cor 11,30} \bibverse{6}
Por lo cual si quisiere gloriarme, no seré insensato: porque diré
verdad: empero lo dejo, porque nadie piense de mí más de lo que en mí
ve, ú oye de mí. \footnote{\textbf{12:6} 2Cor 10,8} \bibverse{7} Y
porque la grandeza de las revelaciones no me levante descomedidamente,
me es dado un aguijón en mi carne, un mensajero de Satanás que me
abofetee, para que no me enaltezca sobremanera. \bibverse{8} Por lo cual
tres veces he rogado al Señor, que se quite de mí. \bibverse{9} Y me ha
dicho: Bástate mi gracia; porque mi potencia en la flaqueza se
perfecciona. Por tanto, de buena gana me gloriaré más bien en mis
flaquezas, porque habite en mí la potencia de Cristo.

\bibverse{10} Por lo cual me gozo en las flaquezas, en afrentas, en
necesidades, en persecuciones, en angustias por Cristo; porque cuando
soy flaco, entonces soy poderoso. \footnote{\textbf{12:10} Fil 4,13}

\hypertarget{referencia-a-la-injusticia-de-los-corintios}{%
\subsection{Referencia a la injusticia de los
corintios}\label{referencia-a-la-injusticia-de-los-corintios}}

\bibverse{11} Heme hecho un necio en gloriarme: vosotros me
constreñisteis; pues yo había de ser alabado de vosotros: porque en nada
he sido menos que los sumos apóstoles, aunque soy nada. \footnote{\textbf{12:11}
  2Cor 11,5} \bibverse{12} Con todo esto, las señales de apóstol han
sido hechas entre vosotros en toda paciencia, en señales, y en
prodigios, y en maravillas. \footnote{\textbf{12:12} Rom 15,19; Heb 2,4}
\bibverse{13} Porque ¿qué hay en que habéis sido menos que las otras
iglesias, sino en que yo mismo no os he sido carga? Perdonadme esta
injuria. \footnote{\textbf{12:13} 2Cor 11,7-9}

\hypertarget{anuncio-de-la-inminente-llegada-del-apuxf3stol-rechazo-de-un-libelo}{%
\subsection{Anuncio de la inminente llegada del apóstol; Rechazo de un
libelo}\label{anuncio-de-la-inminente-llegada-del-apuxf3stol-rechazo-de-un-libelo}}

\bibverse{14} He aquí estoy aparejado para ir á vosotros la tercera vez,
y no os seré gravoso; porque no busco vuestras cosas, sino á vosotros:
porque no han de atesorar los hijos para los padres, sino los padres
para los hijos. \bibverse{15} Empero yo de muy buena gana despenderé y
seré despendido por vuestras almas, aunque amándoos más, sea amado
menos. \footnote{\textbf{12:15} Fil 2,17} \bibverse{16} Mas sea así, yo
no os he agravado: sino que, como soy astuto, os he tomado por engaño.
\bibverse{17} ¿Acaso os he engañado por alguno de los que he enviado á
vosotros? \bibverse{18} Rogué á Tito, y envié con él al hermano. ¿Os
engañó quizá Tito? ¿no hemos procedido con el mismo espíritu y por las
mismas pisadas?

\hypertarget{rectificaciuxf3n-de-una-opiniuxf3n-de-los-corintios-miedo-del-apuxf3stol-por-el-estatus-moral-de-la-comunidad}{%
\subsection{Rectificación de una opinión de los corintios; Miedo del
apóstol por el estatus moral de la
comunidad}\label{rectificaciuxf3n-de-una-opiniuxf3n-de-los-corintios-miedo-del-apuxf3stol-por-el-estatus-moral-de-la-comunidad}}

\bibverse{19} ¿Pensáis aún que nos excusamos con vosotros? Delante de
Dios en Cristo hablamos: mas todo, muy amados, por vuestra edificación.
\bibverse{20} Porque temo que cuando llegare, no os halle tales como
quiero, y yo sea hallado de vosotros cual no queréis; que haya entre
vosotros contiendas, envidias, iras, disensiones, detracciones,
murmuraciones, elaciones, bandos: \footnote{\textbf{12:20} 2Cor 10,2}
\bibverse{21} Que cuando volviere, me humille Dios entre vosotros, y
haya de llorar por muchos de los que antes habrán pecado, y no se han
arrepentido de la inmundicia y fornicación y deshonestidad que han
cometido. \footnote{\textbf{12:21} 2Cor 2,1; 2Cor 13,2}

\hypertarget{anuncio-de-juicio-imparcial-y-juicio-despiadado}{%
\subsection{Anuncio de juicio imparcial y juicio
despiadado}\label{anuncio-de-juicio-imparcial-y-juicio-despiadado}}

\hypertarget{section-12}{%
\section{13}\label{section-12}}

\bibverse{1} ESTA tercera vez voy á vosotros. En la boca de dos ó de
tres testigos consistirá todo negocio. \footnote{\textbf{13:1} 2Cor
  10,2; Mat 18,16} \bibverse{2} He dicho antes, y ahora digo otra vez
como presente, y ahora ausente lo escribo á los que antes pecaron, y á
todos los demás, que si voy otra vez, no perdonaré; \bibverse{3} Pues
buscáis una prueba de Cristo que habla en mí, el cual no es flaco para
con vosotros, antes es poderoso en vosotros. \bibverse{4} Porque aunque
fué crucificado por flaqueza, empero vive por potencia de Dios. Pues
también nosotros somos flacos con él, mas viviremos con él por la
potencia de Dios para con vosotros.

\bibverse{5} Examinaos á vosotros mismos si estáis en fe; probaos á
vosotros mismos. ¿No os conocéis á vosotros mismos, que Jesucristo está
en vosotros? si ya no sois reprobados. \bibverse{6} Mas espero que
conoceréis que nosotros no somos reprobados.

\bibverse{7} Y oramos á Dios que ninguna cosa mala hagáis; no para que
nosotros seamos hallados aprobados, mas para que vosotros hagáis lo que
es bueno, aunque nosotros seamos como reprobados. \bibverse{8} Porque
ninguna cosa podemos contra la verdad, sino por la verdad. \bibverse{9}
Por lo cual nos gozamos que seamos nosotros flacos, y que vosotros
estéis fuertes; y aun deseamos vuestra perfección. \bibverse{10} Por
tanto os escribo esto ausente, por no tratar presente con dureza,
conforme á la potestad que el Señor me ha dado para edificación, y no
para destrucción.

\hypertarget{advertencias-finales-saludos-y-bendiciones}{%
\subsection{Advertencias finales, saludos y
bendiciones}\label{advertencias-finales-saludos-y-bendiciones}}

\bibverse{11} Resta, hermanos, que tengáis gozo, seáis perfectos,
tengáis consolación, sintáis una misma cosa, tengáis paz; y el Dios de
paz y de caridad será con vosotros. \footnote{\textbf{13:11} Rom 15,33;
  Fil 4,4} \bibverse{12} Saludaos los unos á los otros con ósculo santo.
Todos los santos os saludan.

\bibverse{13} La gracia del Señor Jesucristo, y el amor de Dios, y la
participación del Espíritu Santo sea con vosotros todos. Amén. La
segunda Epístola á los Corintios fué enviada de Filipos de Macedonia con
Tito y Lucas. \bibverse{14}
