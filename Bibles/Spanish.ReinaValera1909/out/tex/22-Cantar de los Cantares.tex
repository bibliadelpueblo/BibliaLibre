\hypertarget{el-diuxe1logo-interno-de-sulammith-y-su-anhelo-de-amor}{%
\subsection{El diálogo interno de Sulammith y su anhelo de
amor}\label{el-diuxe1logo-interno-de-sulammith-y-su-anhelo-de-amor}}

\hypertarget{section}{%
\section{1}\label{section}}

\bibverse{1} Canción de canciones, la cual es de Salomón. \bibverse{2}
¡Oh si él me besara con ósculos de su boca! porque mejores son tus
amores que el vino. \bibverse{3} Por el olor de tus suaves ungüentos,
(ungüento derramado es tu nombre,) por eso las doncellas te amaron.
\bibverse{4} Llévame en pos de ti, correremos. Metióme el rey en sus
cámaras: nos gozaremos y alegraremos en ti; acordarémonos de tus amores
más que del vino: los rectos te aman.

\hypertarget{queja-de-belleza-de-niuxf1a-en-riesgo}{%
\subsection{Queja de belleza de niña en
riesgo}\label{queja-de-belleza-de-niuxf1a-en-riesgo}}

\bibverse{5} Morena soy, oh hijas de Jerusalem, mas codiciable; como las
cabañas de Cedar, como las tiendas de Salomón. \bibverse{6} No miréis en
que soy morena, porque el sol me miró. Los hijos de mi madre se airaron
contra mí, hiciéronme guarda de viñas; y mi viña, que era mía, no
guardé.

\hypertarget{tsolicitud-de-la-novia-para-una-cita}{%
\subsection{TSolicitud de la novia para una
cita}\label{tsolicitud-de-la-novia-para-una-cita}}

\bibverse{7} Hazme saber, ó tú á quien ama mi alma, dónde repastas,
dónde haces tener majada al medio día: porque, ¿por qué había yo de
estar como vagueando tras los rebaños de tus compañeros? \bibverse{8} Si
tú no lo sabes, oh hermosa entre las mujeres, sal, yéndote por las
huellas del rebaño, y apacienta tus cabritas junto á las cabañas de los
pastores.

\hypertarget{dulce-charla-de-amor}{%
\subsection{Dulce charla de amor}\label{dulce-charla-de-amor}}

\bibverse{9} A yegua de los carros de Faraón te he comparado, amiga mía.
\bibverse{10} Hermosas son tus mejillas entre los pendientes, tu cuello
entre los collares. \bibverse{11} Zarcillos de oro te haremos, con
clavos de plata. \bibverse{12} Mientras que el rey estaba en su
reclinatorio, mi nardo dió su olor. \bibverse{13} Mi amado es para mí un
manojito de mirra, que reposa entre mis pechos. \bibverse{14} Racimo de
copher en las viñas de Engadi es para mí mi amado. \bibverse{15} He aquí
que tú eres hermosa, amiga mía; he aquí que eres bella: tus ojos de
paloma. \footnote{\textbf{1:15} Cant 2,14; Cant 4,1; Cant 7,1-7; Cant
  6,4} \bibverse{16} He aquí que tú eres hermoso, amado mío, y suave:
nuestro lecho también florido. \footnote{\textbf{1:16} Cant 5,16}
\bibverse{17} Las vigas de nuestra casa son de cedro, y de ciprés los
artesonados.

\hypertarget{cantos-y-compromiso}{%
\subsection{Cantos y compromiso}\label{cantos-y-compromiso}}

\hypertarget{section-1}{%
\section{2}\label{section-1}}

\bibverse{1} Yo soy la rosa de Sarón, y el lirio de los valles.
\bibverse{2} Como el lirio entre las espinas, así es mi amiga entre las
doncellas. \bibverse{3} Como el manzano entre los árboles silvestres,
así es mi amado entre los mancebos: bajo la sombra del deseado me senté,
y su fruto fué dulce á mi paladar. \bibverse{4} Llevóme á la cámara del
vino, y su bandera sobre mí fué amor. \bibverse{5} Sustentadme con
frascos, corroboradme con manzanas; porque estoy enferma de amor.
\footnote{\textbf{2:5} Cant 5,8} \bibverse{6} Su izquierda esté debajo
de mi cabeza, y su derecha me abrace. \footnote{\textbf{2:6} Cant 8,3}
\bibverse{7} Yo os conjuro, oh doncellas de Jerusalem, por las gamas y
por las ciervas del campo, que no despertéis ni hagáis velar al amor,
hasta que quiera.

\footnote{\textbf{2:7} Cant 3,5; Cant 8,4}

\hypertarget{amor-primavera}{%
\subsection{Amor primavera}\label{amor-primavera}}

\bibverse{8} ¡La voz de mi amado! He aquí él viene saltando sobre los
montes, brincando sobre los collados. \bibverse{9} Mi amado es semejante
al gamo, ó al cabrito de los ciervos. Helo aquí, está tras nuestra
pared, mirando por las ventanas, mostrándose por las rejas.
\bibverse{10} Mi amado habló, y me dijo: Levántate, oh amiga mía,
hermosa mía, y vente. \bibverse{11} Porque he aquí ha pasado el
invierno, hase mudado, la lluvia se fué; \bibverse{12} Hanse mostrado
las flores en la tierra, el tiempo de la canción es venido, y en nuestro
país se ha oído la voz de la tórtola; \bibverse{13} La higuera ha echado
sus higos, y las vides en cierne dieron olor: levántate, oh amiga mía,
hermosa mía, y vente. \bibverse{14} Paloma mía, que estás en los
agujeros de la peña, en lo escondido de escarpados parajes, muéstrame tu
rostro, hazme oir tu voz; porque dulce es la voz tuya, y hermoso tu
aspecto.

\footnote{\textbf{2:14} Cant 4,7}

\hypertarget{dos-suspiros-de-amor-de-la-novia}{%
\subsection{Dos suspiros de amor de la
novia}\label{dos-suspiros-de-amor-de-la-novia}}

\bibverse{15} Cazadnos las zorras, las zorras pequeñas, que echan á
perder las viñas; pues que nuestras viñas están en cierne. \bibverse{16}
Mi amado es mío, y yo suya; él apacienta entre lirios. \bibverse{17}
Hasta que apunte el día, y huyan las sombras, tórnate, amado mío; sé
semejante al gamo, ó al cabrito de los ciervos, sobre los montes de
Bether. \footnote{\textbf{2:17} Cant 8,14}

\hypertarget{sueuxf1o-anhelante-de-la-novia}{%
\subsection{Sueño anhelante de la
novia}\label{sueuxf1o-anhelante-de-la-novia}}

\hypertarget{section-2}{%
\section{3}\label{section-2}}

\bibverse{1} Por las noches busqué en mi lecho al que ama mi alma:
busquélo, y no lo hallé. \footnote{\textbf{3:1} Cant 5,6} \bibverse{2}
Levantaréme ahora, y rodearé por la ciudad; por las calles y por las
plazas buscaré al que ama mi alma: busquélo, y no lo hallé. \bibverse{3}
Halláronme los guardas que rondan la ciudad, y díjeles: ¿Habéis visto al
que ama mi alma? \bibverse{4} Pasando de ellos un poco, hallé luego al
que mi alma ama: trabé de él, y no lo dejé, hasta que lo metí en casa de
mi madre, y en la cámara de la que me engendró. \footnote{\textbf{3:4}
  Cant 8,2} \bibverse{5} Yo os conjuro, oh doncellas de Jerusalem, por
las gamas y por las ciervas del campo, que no despertéis ni hagáis velar
al amor, hasta que quiera.

\hypertarget{la-procesiuxf3n-nupcial-del-novio}{%
\subsection{La procesión nupcial del
novio}\label{la-procesiuxf3n-nupcial-del-novio}}

\bibverse{6} ¿Quién es ésta que sube del desierto como columnita de
humo, sahumada de mirra y de incienso, y de todos polvos aromáticos?
\bibverse{7} He aquí es la litera de Salomón: sesenta valientes la
rodean, de los fuertes de Israel. \bibverse{8} Todos ellos tienen
espadas, diestros en la guerra; cada uno su espada sobre su muslo, por
los temores de la noche. \bibverse{9} El rey Salomón se hizo una carroza
de madera del Líbano. \bibverse{10} Sus columnas hizo de plata, su
respaldo de oro, su cielo de grana, su interior enlosado de amor, por
las doncellas de Jerusalem. \bibverse{11} Salid, oh doncellas de Sión, y
ved al rey Salomón con la corona con que le coronó su madre el día de su
desposorio, y el día del gozo de su corazón.

\hypertarget{descripciuxf3n-de-la-novia-por-el-novio}{%
\subsection{Descripción de la novia por el
novio}\label{descripciuxf3n-de-la-novia-por-el-novio}}

\hypertarget{section-3}{%
\section{4}\label{section-3}}

\bibverse{1} He aquí que tú eres hermosa, amiga mía; he aquí que tú eres
hermosa; tus ojos entre tus guedejas como de paloma; tus cabellos como
manada de cabras, que se muestran desde el monte de Galaad. \bibverse{2}
Tus dientes, como manadas de trasquiladas ovejas, que suben del
lavadero, todas con crías mellizas, y ninguna entre ellas estéril.
\footnote{\textbf{4:2} Cant 6,6} \bibverse{3} Tus labios, como un hilo
de grana, y tu habla hermosa; tus sienes, como cachos de granada á la
parte adentro de tus guedejas. \footnote{\textbf{4:3} Cant 6,7}
\bibverse{4} Tu cuello, como la torre de David, edificada para muestra;
mil escudos están colgados de ella, todos escudos de valientes.
\footnote{\textbf{4:4} Cant 7,5} \bibverse{5} Tus dos pechos, como dos
cabritos mellizos de gama, que son apacentados entre azucenas.
\footnote{\textbf{4:5} Cant 7,4} \bibverse{6} Hasta que apunte el día y
huyan las sombras, iréme al monte de la mirra, y al collado del
incienso. \footnote{\textbf{4:6} Cant 2,17} \bibverse{7} Toda tú eres
hermosa, amiga mía, y en ti no hay mancha.

\footnote{\textbf{4:7} Sal 45,14}

\hypertarget{la-boda}{%
\subsection{La boda}\label{la-boda}}

\bibverse{8} Conmigo del Líbano, oh esposa, conmigo ven del Líbano: mira
desde la cumbre de Amana, desde la cumbre de Senir y de Hermón, desde
las guaridas de los leones, desde los montes de los tigres. \bibverse{9}
Prendiste mi corazón, hermana, esposa mía; has preso mi corazón con uno
de tus ojos, con una gargantilla de tu cuello. \bibverse{10} ¡Cuán
hermosos son tus amores, hermana, esposa mía! ¡cuánto mejores que el
vino tus amores, y el olor de tus ungüentos que todas las especias
aromáticas! \bibverse{11} Como panal de miel destilan tus labios, oh
esposa; miel y leche hay debajo de tu lengua; y el olor de tus vestidos
como el olor del Líbano.

\hypertarget{comparaciuxf3n-de-la-esposa-de-la-novia-con-un-maravilloso-jarduxedn}{%
\subsection{Comparación de la esposa de la novia con un maravilloso
jardín}\label{comparaciuxf3n-de-la-esposa-de-la-novia-con-un-maravilloso-jarduxedn}}

\bibverse{12} Huerto cerrado eres, mi hermana, esposa mía; fuente
cerrada, fuente sellada. \bibverse{13} Tus renuevos paraíso de granados,
con frutos suaves, de cámphoras y nardos, \bibverse{14} Nardo y azafrán,
caña aromática y canela, con todos los árboles de incienso; mirra y
áloes, con todas las principales especias. \bibverse{15} Fuente de
huertos, pozo de aguas vivas, que corren del Líbano. \bibverse{16}
Levántate, Aquilón, y ven, Austro: sopla mi huerto, despréndanse sus
aromas. Venga mi amado á su huerto, y coma de su dulce fruta.

\hypertarget{el-joven-marido-toma-posesiuxf3n-de-su-jarduxedn-la-fiesta-de-bodas}{%
\subsection{El joven marido toma posesión de su jardín; la fiesta de
bodas}\label{el-joven-marido-toma-posesiuxf3n-de-su-jarduxedn-la-fiesta-de-bodas}}

\hypertarget{section-4}{%
\section{5}\label{section-4}}

\bibverse{1} Yo vine á mi huerto, oh hermana, esposa mía: cogido he mi
mirra y mis aromas; he comido mi panal y mi miel, mi vino y mi leche he
bebido. Comed, amigos; bebed, amados, y embriagaos.

\footnote{\textbf{5:1} Cant 6,2}

\hypertarget{besuch-des-bruxe4utigams}{%
\subsection{Besuch des Bräutigams}\label{besuch-des-bruxe4utigams}}

\bibverse{2} Yo dormía, pero mi corazón velaba: la voz de mi amado que
llamaba: Abreme, hermana mía, amiga mía, paloma mía, perfecta mía;
porque mi cabeza está llena de rocío, mis cabellos de las gotas de la
noche. \footnote{\textbf{5:2} Cant 6,9} \bibverse{3} Heme desnudado mi
ropa; ¿cómo la tengo de vestir? He lavado mis pies; ¿cómo los tengo de
ensuciar? \bibverse{4} Mi amado metió su mano por el agujero, y mis
entrañas se conmovieron dentro de mí. \bibverse{5} Yo me levanté para
abrir á mi amado, y mis manos gotearon mirra, y mis dedos mirra que
corría sobre las aldabas del candado. \bibverse{6} Abrí yo á mi amado;
mas mi amado se había ido, había ya pasado: y tras su hablar salió mi
alma: busquélo, y no lo hallé; llamélo, y no me respondió. \footnote{\textbf{5:6}
  Cant 3,1} \bibverse{7} Halláronme los guardas que rondan la ciudad:
hiriéronme, llagáronme, quitáronme mi manto de encima los guardas de los
muros.

\hypertarget{descripciuxf3n-del-novio-por-la-novia}{%
\subsection{Descripción del novio por la
novia}\label{descripciuxf3n-del-novio-por-la-novia}}

\bibverse{8} Yo os conjuro, oh doncellas de Jerusalem, si hallareis á mi
amado, que le hagáis saber como de amor estoy enferma. \bibverse{9} ¿Qué
es tu amado más que otro amado, oh la más hermosa de todas las mujeres?
¿qué es tu amado más que otro amado, que así nos conjuras? \bibverse{10}
Mi amado es blanco y rubio, señalado entre diez mil. \bibverse{11} Su
cabeza, como oro finísimo; sus cabellos crespos, negros como el cuervo.
\bibverse{12} Sus ojos, como palomas junto á los arroyos de las aguas,
que se lavan con leche, y á la perfección colocados. \footnote{\textbf{5:12}
  Cant 4,1} \bibverse{13} Sus mejillas, como una era de especias
aromáticas, como fragantes flores: sus labios, como lirios que destilan
mirra que trasciende. \footnote{\textbf{5:13} Sal 45,3} \bibverse{14}
Sus manos, como anillos de oro engastados de jacintos: su vientre, como
claro marfil cubierto de zafiros. \bibverse{15} Sus piernas, como
columnas de mármol fundadas sobre basas de fino oro: su aspecto como el
Líbano, escogido como los cedros. \bibverse{16} Su paladar, dulcísimo: y
todo él codiciable. Tal es mi amado, tal es mi amigo, oh doncellas de
Jerusalem.

\hypertarget{section-5}{%
\section{6}\label{section-5}}

\bibverse{1} ¿Dónde se ha ido tu amado, oh la más hermosa de todas las
mujeres? ¿Adónde se apartó tu amado, y le buscaremos contigo?
\bibverse{2} Mi amado descendió á su huerto, á las eras de los aromas,
para apacentar en los huertos, y para coger los lirios. \footnote{\textbf{6:2}
  Cant 4,6} \bibverse{3} Yo soy de mi amado, y mi amado es mío: él
apacienta entre los lirios.

\footnote{\textbf{6:3} Cant 2,16}

\hypertarget{descripciuxf3n-de-la-novia-por-el-novio-1}{%
\subsection{Descripción de la novia por el
novio}\label{descripciuxf3n-de-la-novia-por-el-novio-1}}

\bibverse{4} Hermosa eres tú, oh amiga mía, como Tirsa; de desear, como
Jerusalem; imponente como ejércitos en orden. \footnote{\textbf{6:4}
  Cant 1,15} \bibverse{5} Aparta tus ojos de delante de mí, porque ellos
me vencieron. Tu cabello es como manada de cabras, que se muestran en
Galaad. \footnote{\textbf{6:5} Cant 4,1} \bibverse{6} Tus dientes, como
manada de ovejas que suben del lavadero, todas con crías mellizas, y
estéril no hay entre ellas. \footnote{\textbf{6:6} Cant 4,2}
\bibverse{7} Como cachos de granada son tus sienes entre tus guedejas.
\footnote{\textbf{6:7} Cant 4,3} \bibverse{8} Sesenta son las reinas, y
ochenta las concubinas, y las doncellas sin cuento: \footnote{\textbf{6:8}
  Sal 45,15} \bibverse{9} Mas una es la paloma mía, la perfecta mía;
única es á su madre, escogida á la que la engendró. Viéronla las
doncellas, y llamáronla bienaventurada; las reinas y las concubinas, y
la alabaron.

\footnote{\textbf{6:9} Cant 5,2}

\hypertarget{la-procesiuxf3n-de-la-boda}{%
\subsection{La procesión de la boda}\label{la-procesiuxf3n-de-la-boda}}

\bibverse{10} ¿Quién es ésta que se muestra como el alba, hermosa como
la luna, esclarecida como el sol, imponente como ejércitos en orden?
\bibverse{11} Al huerto de los nogales descendí á ver los frutos del
valle, y para ver si brotaban las vides, si florecían los granados.
\bibverse{12} No lo supe: hame mi alma hecho como los carros de
Amminadab. \bibverse{13} Tórnate, tórnate, oh Sulamita; tórnate,
tórnate, y te miraremos. ¿Qué veréis en la Sulamita? Como la reunión de
dos campamentos.

\hypertarget{descripciuxf3n-del-baile-de-la-novia-alabado-sea-su-belleza}{%
\subsection{Descripción del baile de la novia; Alabado sea su
belleza}\label{descripciuxf3n-del-baile-de-la-novia-alabado-sea-su-belleza}}

\hypertarget{section-6}{%
\section{7}\label{section-6}}

\bibverse{1} ¡Cuán hermosos son tus pies en los calzados, oh hija de
príncipe! Los contornos de tus muslos son como joyas, obra de mano de
excelente maestro. \bibverse{2} Tu ombligo, como una taza redonda, que
no le falta bebida. Tu vientre, como montón de trigo, cercado de lirios.
\bibverse{3} Tus dos pechos, como dos cabritos mellizos de gama.
\bibverse{4} Tu cuello, como torre de marfil; tus ojos, como las
pesqueras de Hesbón junto á la puerta de Batrabbim; tu nariz, como la
torre del Líbano, que mira hacia Damasco. \footnote{\textbf{7:4} Cant
  4,5} \bibverse{5} Tu cabeza encima de ti, como el Carmelo; y el
cabello de tu cabeza, como la púrpura del rey ligada en los corredores.
\footnote{\textbf{7:5} Cant 4,4} \bibverse{6} ¡Qué hermosa eres, y cuán
suave, oh amor deleitoso!

\hypertarget{el-novio-alaba-a-la-novia}{%
\subsection{El novio alaba a la novia}\label{el-novio-alaba-a-la-novia}}

\bibverse{7} ¡Y tu estatura es semejante á la palma, y tus pechos á los
racimos! \footnote{\textbf{7:7} Cant 1,15; Cant 2,14} \bibverse{8} Yo
dije: Subiré á la palma, asiré sus ramos: y tus pechos serán ahora como
racimos de vid, y el olor de tu boca como de manzanas; \bibverse{9} Y tu
paladar como el buen vino, que se entra á mi amado suavemente, y hace
hablar los labios de los viejos.

\hypertarget{en-la-patria-de-la-esposa}{%
\subsection{En la patria de la esposa}\label{en-la-patria-de-la-esposa}}

\bibverse{10} Yo soy de mi amado, y conmigo tiene su contentamiento.
\bibverse{11} Ven, oh amado mío, salgamos al campo, moremos en las
aldeas. \bibverse{12} Levantémonos de mañana á las viñas; veamos si
brotan las vides, si se abre el cierne, si han florecido los granados;
allí te daré mis amores. \footnote{\textbf{7:12} Cant 2,10-13}
\bibverse{13} Las mandrágoras han dado olor, y á nuestras puertas hay
toda suerte de dulces frutas, nuevas y añejas, que para ti, oh amado
mío, he guardado. \footnote{\textbf{7:13} Cant 6,11}

\hypertarget{amante-y-hermano}{%
\subsection{Amante y hermano}\label{amante-y-hermano}}

\hypertarget{section-7}{%
\section{8}\label{section-7}}

\bibverse{1} ¡Oh quién te me diese como hermano que mamó los pechos de
mi madre; de modo que te halle yo fuera, y te bese, y no me
menosprecien! \bibverse{2} Yo te llevaría, te metiera en casa de mi
madre: tú me enseñarías, y yo te hiciera beber vino adobado del mosto de
mis granadas. \footnote{\textbf{8:2} Cant 3,4} \bibverse{3} Su izquierda
esté debajo de mi cabeza, y su derecha me abrace. \footnote{\textbf{8:3}
  Cant 2,6} \bibverse{4} Conjúroos, oh doncellas de Jerusalem, que no
despertéis, ni hagáis velar al amor, hasta que quiera.

\footnote{\textbf{8:4} Cant 2,7}

\hypertarget{en-el-destino-en-la-casa-del-esposo}{%
\subsection{En el destino en la casa del
esposo}\label{en-el-destino-en-la-casa-del-esposo}}

\bibverse{5} ¿Quién es ésta que sube del desierto, recostada sobre su
amado? Debajo de un manzano te desperté: allí tuvo tu madre dolores,
allí tuvo dolores la que te parió. \bibverse{6} Ponme como un sello
sobre tu corazón, como una marca sobre tu brazo: porque fuerte es como
la muerte el amor; duro como el sepulcro el celo: sus brasas, brasas de
fuego, fuerte llama. \bibverse{7} Las muchas aguas no podrán apagar el
amor, ni lo ahogarán los ríos. Si diese el hombre toda la hacienda de su
casa por este amor, de cierto lo menospreciaran.

\hypertarget{canciuxf3n-de-la-hermana-pequeuxf1a-que-frustruxf3-los-planes-de-los-codiciosos-hermanos}{%
\subsection{Canción de la hermana pequeña que frustró los planes de los
codiciosos
hermanos}\label{canciuxf3n-de-la-hermana-pequeuxf1a-que-frustruxf3-los-planes-de-los-codiciosos-hermanos}}

\bibverse{8} Tenemos una pequeña hermana, que no tiene pechos: ¿qué
haremos á nuestra hermana cuando de ella se hablare? \bibverse{9} Si
ella es muro, edificaremos sobre él un palacio de plata: y si fuere
puerta, la guarneceremos con tablas de cedro. \bibverse{10} Yo soy muro,
y mis pechos como torres, desde que fuí en sus ojos como la que halla
paz.

\hypertarget{canciuxf3n-de-los-dos-viuxf1edos}{%
\subsection{Canción de los dos
viñedos}\label{canciuxf3n-de-los-dos-viuxf1edos}}

\bibverse{11} Salomón tuvo una viña en Baal-hamón, la cual entregó á
guardas, cada uno de los cuales debía traer mil monedas de plata por su
fruto. \bibverse{12} Mi viña, que es mía, está delante de mí: las mil
serán tuyas, oh Salomón, y doscientas, de los que guardan su fruto.
\bibverse{13} Oh tú la que moras en los huertos, los compañeros escuchan
tu voz: házmela oir. \bibverse{14} Huye, amado mío; y sé semejante al
gamo, ó al cervatillo, sobre las montañas de los aromas.
