\hypertarget{la-soberanuxeda-uxfanica-del-hijo-de-dios-sobre-los-mensajeros-de-dios-del-antiguo-testamento}{%
\subsection{La soberanía única del Hijo de Dios sobre los mensajeros de
Dios del Antiguo
Testamento}\label{la-soberanuxeda-uxfanica-del-hijo-de-dios-sobre-los-mensajeros-de-dios-del-antiguo-testamento}}

\hypertarget{section}{%
\section{1}\label{section}}

\bibverse{1} Dios, habiendo hablado muchas veces y en muchas maneras en
otro tiempo á los padres por los profetas, \bibverse{2} En estos
postreros días nos ha hablado por el Hijo, al cual constituyó heredero
de todo, por el cual asimismo hizo el universo: \footnote{\textbf{1:2}
  Sal 2,8; Juan 1,3; Col 1,16} \bibverse{3} El cual siendo el resplandor
de su gloria, y la misma imagen de su sustancia, y sustentando todas las
cosas con la palabra de su potencia, habiendo hecho la purgación de
nuestros pecados por sí mismo, se sentó á la diestra de la Majestad en
las alturas, \footnote{\textbf{1:3} 2Cor 4,4; Col 1,15; Heb 9,14; Heb
  9,26; Mar 16,19} \bibverse{4} Hecho tanto más excelente que los
ángeles, cuanto alcanzó por herencia más excelente nombre que ellos.
\footnote{\textbf{1:4} Fil 2,9; 1Pe 3,22}

\hypertarget{evidencia-del-antiguo-testamento-de-la-exaltaciuxf3n-del-hijo-de-dios-sobre-los-uxe1ngeles}{%
\subsection{Evidencia del Antiguo Testamento de la exaltación del Hijo
de Dios sobre los
ángeles}\label{evidencia-del-antiguo-testamento-de-la-exaltaciuxf3n-del-hijo-de-dios-sobre-los-uxe1ngeles}}

\bibverse{5} Porque ¿á cuál de los ángeles dijo Dios jamás: Mi hijo eres
tú, hoy yo te he engendrado? y otra vez: Yo seré á él Padre, y él me
será á mí hijo?

\bibverse{6} Y otra vez, cuando introduce al Primogénito en la tierra,
dice: Y adórenle todos los ángeles de Dios. \bibverse{7} Y ciertamente
de los ángeles dice: El que hace á sus ángeles espíritus, y á sus
ministros llama de fuego.

\bibverse{8} Mas al hijo: Tu trono, oh Dios, por el siglo del siglo;
vara de equidad la vara de tu reino; \bibverse{9} Has amado la justicia,
y aborrecido la maldad; por lo cual te ungió Dios, el Dios tuyo, con
óleo de alegría más que á tus compañeros.

\bibverse{10} Y: Tú, oh Señor, en el principio fundaste la tierra; y los
cielos son obras de tus manos: \bibverse{11} Ellos perecerán, mas tú
eres permanente; y todos ellos se envejecerán como una vestidura;
\bibverse{12} Y como un vestido los envolverás, y serán mudados; empero
tú eres el mismo, y tus años no acabarán.

\bibverse{13} Pues, ¿á cuál de los ángeles dijo jamás: Siéntate á mi
diestra, hasta que ponga á tus enemigos por estrado de tus pies?

\bibverse{14} ¿No son todos espíritus administradores, enviados para
servicio á favor de los que serán herederos de salud? \footnote{\textbf{1:14}
  Sal 34,8; Sal 91,11-12}

\hypertarget{de-ahuxed-surge-la-obligaciuxf3n-de-que-obedezcamos-voluntariamente-las-palabras-que-nos-ha-dicho-este-hijo}{%
\subsection{De ahí surge la obligación de que obedezcamos
voluntariamente las palabras que nos ha dicho este
Hijo}\label{de-ahuxed-surge-la-obligaciuxf3n-de-que-obedezcamos-voluntariamente-las-palabras-que-nos-ha-dicho-este-hijo}}

\hypertarget{section-1}{%
\section{2}\label{section-1}}

\bibverse{1} Por tanto, es menester que con más diligencia atendamos á
las cosas que hemos oído, porque acaso no nos escurramos. \bibverse{2}
Porque si la palabra dicha por los ángeles fué firme, y toda rebelión y
desobediencia recibió justa paga de retribución, \bibverse{3} ¿Cómo
escaparemos nosotros, si tuviéremos en poco una salud tan grande? La
cual, habiendo comenzado á ser publicada por el Señor, ha sido
confirmada hasta nosotros por los que oyeron; \footnote{\textbf{2:3} Heb
  10,29; Heb 12,25} \bibverse{4} Testificando juntamente con ellos Dios,
con señales y milagros, y diversas maravillas, y repartimientos del
Espíritu Santo según su voluntad. \footnote{\textbf{2:4} Mar 16,20; 1Cor
  12,4-11; 2Cor 12,12; Hech 1,2-13; Hech 10,44-45}

\hypertarget{su-humillaciuxf3n-encarnaciuxf3n-y-sufrimiento-de-muerte-no-limita-su-sublimidad}{%
\subsection{Su humillación, encarnación y sufrimiento de muerte, no
limita su
sublimidad}\label{su-humillaciuxf3n-encarnaciuxf3n-y-sufrimiento-de-muerte-no-limita-su-sublimidad}}

\bibverse{5} Porque no sujetó á los ángeles el mundo venidero, del cual
hablamos. \bibverse{6} Testificó empero uno en cierto lugar, diciendo:
¿Qué es el hombre, que te acuerdas de él? ¿ó el hijo del hombre, que le
visitas? \bibverse{7} Tú le hiciste un poco menor que los ángeles,
coronástele de gloria y de honra, y pusístele sobre las obras de tus
manos; \bibverse{8} Todas las cosas sujetaste debajo de sus pies. Porque
en cuanto le sujetó todas las cosas, nada dejó que no sea sujeto á él;
mas aun no vemos que todas las cosas le sean sujetas.

\bibverse{9} Empero vemos coronado de gloria y de honra, por el
padecimiento de muerte, á aquel Jesús que es hecho un poco menor que los
ángeles, para que por gracia de Dios gustase la muerte por todos.
\footnote{\textbf{2:9} Fil 2,8-9}

\hypertarget{la-necesidad-de-la-humillaciuxf3n-especialmente-el-sufrimiento-de-la-muerte}{%
\subsection{La necesidad de la humillación, especialmente el sufrimiento
de la
muerte}\label{la-necesidad-de-la-humillaciuxf3n-especialmente-el-sufrimiento-de-la-muerte}}

\bibverse{10} Porque convenía que aquel por cuya causa son todas las
cosas, y por el cual todas las cosas subsisten, habiendo de llevar á la
gloria á muchos hijos, hiciese consumado por aflicciones al autor de la
salud de ellos. \footnote{\textbf{2:10} Heb 12,2} \bibverse{11} Porque
el que santifica y los que son santificados, de uno son todos: por lo
cual no se avergüenza de llamarlos hermanos, \footnote{\textbf{2:11} Mar
  3,34-35; Juan 17,19; Juan 20,17} \bibverse{12} Diciendo: Anunciaré á
mis hermanos tu nombre, en medio de la congregación te alabaré.

\bibverse{13} Y otra vez: Yo confiaré en él. Y otra vez: He aquí, yo y
los hijos que me dió Dios.

\hypertarget{las-beneficiosas-consecuencias-de-la-humillaciuxf3n}{%
\subsection{Las beneficiosas consecuencias de la
humillación}\label{las-beneficiosas-consecuencias-de-la-humillaciuxf3n}}

\bibverse{14} Así que, por cuanto los hijos participaron de carne y
sangre, él también participó de lo mismo, para destruir por la muerte al
que tenía el imperio de la muerte, es á saber, al diablo, \bibverse{15}
Y librar á los que por el temor de la muerte estaban por toda la vida
sujetos á servidumbre. \bibverse{16} Porque ciertamente no tomó á los
ángeles, sino á la simiente de Abraham tomó. \bibverse{17} Por lo cual,
debía ser en todo semejante á los hermanos, para venir á ser
misericordioso y fiel Pontífice en lo que es para con Dios, para expiar
los pecados del pueblo. \footnote{\textbf{2:17} Fil 2,7} \bibverse{18}
Porque en cuanto él mismo padeció siendo tentado, es poderoso para
socorrer á los que son tentados. \footnote{\textbf{2:18} Heb 4,15}

\hypertarget{el-hijo-de-dios-jesuxfas-en-su-majestad-sobre-el-ministro-de-dios-moisuxe9s}{%
\subsection{El Hijo de Dios Jesús en su majestad sobre el ministro de
Dios
Moisés}\label{el-hijo-de-dios-jesuxfas-en-su-majestad-sobre-el-ministro-de-dios-moisuxe9s}}

\hypertarget{section-2}{%
\section{3}\label{section-2}}

\bibverse{1} Por tanto, hermanos santos, participantes de la vocación
celestial, considerad al Apóstol y Pontífice de nuestra profesión,
Cristo Jesús; \footnote{\textbf{3:1} Heb 4,14} \bibverse{2} El cual es
fiel al que le constituyó, como también lo fué Moisés sobre toda su
casa. \footnote{\textbf{3:2} Núm 12,7} \bibverse{3} Porque de tanto
mayor gloria que Moisés éste es estimado digno, cuanto tiene mayor
dignidad que la casa el que la fabricó. \bibverse{4} Porque toda casa es
edificada de alguno: mas el que crió todas las cosas es Dios.
\bibverse{5} Y Moisés á la verdad fué fiel sobre toda su casa, como
siervo, para testificar lo que se había de decir; \bibverse{6} Mas
Cristo como hijo, sobre su casa; la cual casa somos nosotros, si hasta
el cabo retuviéremos firme la confianza y la gloria de la esperanza.
\footnote{\textbf{3:6} 1Pe 2,5; Efes 2,19}

\hypertarget{la-advertencia-del-salmista-contra-la-incredulidad-y-la-apostasuxeda}{%
\subsection{La advertencia del salmista contra la incredulidad y la
apostasía}\label{la-advertencia-del-salmista-contra-la-incredulidad-y-la-apostasuxeda}}

\bibverse{7} Por lo cual, como dice el Espíritu Santo: Si oyereis hoy su
voz, \footnote{\textbf{3:7} Heb 4,7} \bibverse{8} No endurezcáis
vuestros corazones como en la provocación, en el día de la tentación en
el desierto, \footnote{\textbf{3:8} Éxod 17,7; Núm 20,2-5} \bibverse{9}
Donde me tentaron vuestros padres; me probaron, y vieron mis obras
cuarenta años. \bibverse{10} A causa de lo cual me enemisté con esta
generación, y dije: Siempre divagan ellos de corazón, y no han conocido
mis caminos. \bibverse{11} Juré, pues, en mi ira: No entrarán en mi
reposo.

\bibverse{12} Mirad, hermanos, que en ninguno de vosotros haya corazón
malo de incredulidad para apartarse del Dios vivo: \bibverse{13} Antes
exhortaos los unos á los otros cada día, entre tanto que se dice Hoy;
porque ninguno de vosotros se endurezca con engaño de pecado:
\footnote{\textbf{3:13} 1Tes 5,11}

\hypertarget{el-ejemplo-de-advertencia-de-los-israelitas-en-el-desierto}{%
\subsection{El ejemplo de advertencia de los israelitas en el
desierto}\label{el-ejemplo-de-advertencia-de-los-israelitas-en-el-desierto}}

\bibverse{14} Porque participantes de Cristo somos hechos, con tal que
conservemos firme hasta el fin el principio de nuestra confianza;
\footnote{\textbf{3:14} Heb 6,11} \bibverse{15} Entre tanto que se dice:
Si oyereis hoy su voz, no endurezcáis vuestros corazones, como en la
provocación.

\bibverse{16} Porque algunos de los que habían salido de Egipto con
Moisés, habiendo oído, provocaron, aunque no todos. \bibverse{17} Mas
¿con cuáles estuvo enojado cuarenta años? ¿No fué con los que pecaron,
cuyos cuerpos cayeron en el desierto? \footnote{\textbf{3:17} Núm 14,29;
  1Cor 10,10}

\bibverse{18} ¿Y á quiénes juró que no entrarían en su reposo, sino á
aquellos que no obedecieron? \bibverse{19} Y vemos que no pudieron
entrar á causa de incredulidad.

\hypertarget{interpretaciuxf3n-de-la-promesa-del-salmo-sobre-el-resto-del-pueblo-de-dios}{%
\subsection{Interpretación de la promesa del salmo sobre el resto del
pueblo de
Dios}\label{interpretaciuxf3n-de-la-promesa-del-salmo-sobre-el-resto-del-pueblo-de-dios}}

\hypertarget{section-3}{%
\section{4}\label{section-3}}

\bibverse{1} Temamos, pues, que quedando aún la promesa de entrar en su
reposo, parezca alguno de vosotros haberse apartado. \bibverse{2} Porque
también á nosotros se nos ha evangelizado como á ellos; mas no les
aprovechó el oir la palabra á los que la oyeron sin mezclar fe.
\bibverse{3} Empero entramos en el reposo los que hemos creído, de la
manera que dijo: Como juré en mi ira, no entrarán en mi reposo: aun
acabadas las obras desde el principio del mundo. \bibverse{4} Porque en
un cierto lugar dijo así del séptimo día: Y reposó Dios de todas sus
obras en el séptimo día. \bibverse{5} Y otra vez aquí: No entrarán en mi
reposo.

\bibverse{6} Así que, pues que resta que algunos han de entrar en él, y
aquellos á quienes primero fué anunciado no entraron por causa de
desobediencia, \bibverse{7} Determina otra vez un cierto día, diciendo
por David: Hoy, después de tanto tiempo; como está dicho: Si oyereis su
voz hoy, no endurezcáis vuestros corazones.

\bibverse{8} Porque si Josué les hubiera dado el reposo, no hablaría
después de otro día. \footnote{\textbf{4:8} Deut 31,7; Jos 22,4}
\bibverse{9} Por tanto, queda un reposo para el pueblo de Dios.
\bibverse{10} Porque el que ha entrado en su reposo, también él ha
reposado de sus obras, como Dios de las suyas.

\hypertarget{exhortaciuxf3n-final-en-referencia-a-la-seriedad-y-el-poder-de-la-palabra-de-dios}{%
\subsection{Exhortación final en referencia a la seriedad y el poder de
la palabra de
Dios}\label{exhortaciuxf3n-final-en-referencia-a-la-seriedad-y-el-poder-de-la-palabra-de-dios}}

\bibverse{11} Procuremos pues de entrar en aquel reposo; que ninguno
caiga en semejante ejemplo de desobediencia. \footnote{\textbf{4:11} Heb
  3,16-19} \bibverse{12} Porque la palabra de Dios es viva y eficaz, y
más penetrante que toda espada de dos filos: y que alcanza hasta partir
el alma, y aun el espíritu, y las coyunturas y tuétanos, y discierne los
pensamientos y las intenciones del corazón. \footnote{\textbf{4:12} Jer
  23,29; Apoc 2,12} \bibverse{13} Y no hay cosa criada que no sea
manifiesta en su presencia; antes todas las cosas están desnudas y
abiertas á los ojos de aquel á quien tenemos que dar cuenta.

\hypertarget{jesuxfas-conoce-las-debilidades-humanas-por-experiencia-personal}{%
\subsection{Jesús conoce las debilidades humanas por experiencia
personal}\label{jesuxfas-conoce-las-debilidades-humanas-por-experiencia-personal}}

\bibverse{14} Por tanto, teniendo un gran Pontífice, que penetró los
cielos, Jesús el Hijo de Dios, retengamos nuestra profesión. \footnote{\textbf{4:14}
  Heb 3,1; Heb 9,11-12; Heb 10,23} \bibverse{15} Porque no tenemos un
Pontífice que no se pueda compadecer de nuestras flaquezas; mas tentado
en todo según nuestra semejanza, pero sin pecado. \footnote{\textbf{4:15}
  Heb 2,18; Juan 8,46} \bibverse{16} Lleguémonos pues confiadamente al
trono de la gracia, para alcanzar misericordia, y hallar gracia para el
oportuno socorro. \footnote{\textbf{4:16} Rom 3,25; Rom 5,2}

\hypertarget{con-cristo-se-encuentran-los-requisitos-necesarios-del-sumo-sacerdote-sugeridos-en-melquisedec}{%
\subsection{Con Cristo se encuentran los requisitos necesarios del sumo
sacerdote sugeridos en
Melquisedec}\label{con-cristo-se-encuentran-los-requisitos-necesarios-del-sumo-sacerdote-sugeridos-en-melquisedec}}

\hypertarget{section-4}{%
\section{5}\label{section-4}}

\bibverse{1} Porque todo pontífice, tomado de entre los hombres, es
constituído á favor de los hombres en lo que á Dios toca, para que
ofrezca presentes y sacrificios por los pecados: \bibverse{2} Que se
pueda compadecer de los ignorantes y extraviados, pues que él también
está rodeado de flaqueza; \bibverse{3} Y por causa de ella debe, como
por sí mismo, así también por el pueblo, ofrecer por los pecados.
\bibverse{4} Ni nadie toma para sí la honra, sino el que es llamado de
Dios, como Aarón. \footnote{\textbf{5:4} Éxod 28,1} \bibverse{5} Así
también Cristo no se glorificó á sí mismo haciéndose Pontífice, mas el
que le dijo: Tú eres mi Hijo, yo te he engendrado hoy;

\bibverse{6} Como también dice en otro lugar: Tú eres sacerdote
eternamente, según el orden de Melchîsedec.

\bibverse{7} El cual en los días de su carne, ofreciendo ruegos y
súplicas con gran clamor y lágrimas al que le podía librar de la muerte,
fué oído por su reverencial miedo. \footnote{\textbf{5:7} Mat 26,39-46}
\bibverse{8} Y aunque era Hijo, por lo que padeció aprendió la
obediencia; \footnote{\textbf{5:8} Fil 2,8} \bibverse{9} Y consumado,
vino á ser causa de eterna salud á todos los que le obedecen;
\bibverse{10} Nombrado de Dios pontífice según el orden de Melchîsedec.
\footnote{\textbf{5:10} Heb 7,-1}

\hypertarget{quejarse-de-la-inmadurez-la-indolencia-intelectual-y-el-atraso-de-los-lectores}{%
\subsection{Quejarse de la inmadurez, la indolencia intelectual y el
atraso de los
lectores}\label{quejarse-de-la-inmadurez-la-indolencia-intelectual-y-el-atraso-de-los-lectores}}

\bibverse{11} Del cual tenemos mucho que decir, y dificultoso de
declarar, por cuanto sois flacos para oir. \bibverse{12} Porque debiendo
ser ya maestros á causa del tiempo, tenéis necesidad de volver á ser
enseñados cuáles sean los primeros rudimentos de las palabras de Dios; y
habéis llegado á ser tales que tengáis necesidad de leche, y no de
manjar sólido. \bibverse{13} Que cualquiera que participa de la leche,
es inhábil para la palabra de la justicia, porque es niño; \footnote{\textbf{5:13}
  Efes 4,14}

\bibverse{14} Mas la vianda firme es para los perfectos, para los que
por la costumbre tienen los sentidos ejercitados en el discernimiento
del bien y del mal.

\hypertarget{es-una-cuestiuxf3n-de-progreso-la-recauxedda-es-peligrosa-y-puede-provocar-dauxf1os-incurables}{%
\subsection{Es una cuestión de progreso; La recaída es peligrosa y puede
provocar daños
incurables}\label{es-una-cuestiuxf3n-de-progreso-la-recauxedda-es-peligrosa-y-puede-provocar-dauxf1os-incurables}}

\hypertarget{section-5}{%
\section{6}\label{section-5}}

\bibverse{1} Por tanto, dejando la palabra del comienzo en la doctrina
de Cristo, vamos adelante á la perfección; no echando otra vez el
fundamento del arrepentimiento de obras muertas, y de la fe en Dios,
\bibverse{2} De la doctrina de bautismos, y de la imposición de manos, y
de la resurrección de los muertos, y del juicio eterno. \bibverse{3} Y
esto haremos á la verdad, si Dios lo permitiere. \bibverse{4} Porque es
imposible que los que una vez fueron iluminados y gustaron el don
celestial, y fueron hechos partícipes del Espíritu Santo, \bibverse{5} Y
asimismo gustaron la buena palabra de Dios, y las virtudes del siglo
venidero, \bibverse{6} Y recayeron, sean otra vez renovados para
arrepentimiento, crucificando de nuevo para sí mismos al Hijo de Dios, y
exponiéndole á vituperio. \bibverse{7} Porque la tierra que embebe el
agua que muchas veces vino sobre ella, y produce hierba provechosa á
aquellos de los cuales es labrada, recibe bendición de Dios:
\bibverse{8} Mas la que produce espinas y abrojos, es reprobada, y
cercana de maldición; cuyo fin será el ser abrasada.

\hypertarget{confiada-esperanza-de-superar-este-angustioso-estado-de-los-lectores-y-el-peligro-que-los-amenaza}{%
\subsection{Confiada esperanza de superar este angustioso estado de los
lectores y el peligro que los
amenaza}\label{confiada-esperanza-de-superar-este-angustioso-estado-de-los-lectores-y-el-peligro-que-los-amenaza}}

\bibverse{9} Pero de vosotros, oh amados, esperamos mejores cosas, y más
cercanas á salud, aunque hablamos así. \bibverse{10} Porque Dios no es
injusto para olvidar vuestra obra y el trabajo de amor que habéis
mostrado á su nombre, habiendo asistido y asistiendo aún á los santos.
\footnote{\textbf{6:10} Heb 10,32-34} \bibverse{11} Mas deseamos que
cada uno de vosotros muestre la misma solicitud hasta el cabo, para
cumplimiento de la esperanza: \footnote{\textbf{6:11} Heb 3,14; Fil 1,6}
\bibverse{12} Que no os hagáis perezosos, mas imitadores de aquellos que
por la fe y la paciencia heredan las promesas.

\hypertarget{el-fundamento-firme-de-la-esperanza-en-la-gloria-que-seguramente-se-espera-radica-en-las-confiables-promesas-de-dios}{%
\subsection{El fundamento firme de la esperanza en la gloria que
seguramente se espera radica en las confiables promesas de
Dios}\label{el-fundamento-firme-de-la-esperanza-en-la-gloria-que-seguramente-se-espera-radica-en-las-confiables-promesas-de-dios}}

\bibverse{13} Porque prometiendo Dios á Abraham, no pudiendo jurar por
otro mayor, juró por sí mismo, \bibverse{14} Diciendo: De cierto te
bendeciré bendiciendo, y multiplicando te multiplicaré. \bibverse{15} Y
así, esperando con largura de ánimo, alcanzó la promesa. \bibverse{16}
Porque los hombres ciertamente por el mayor que ellos juran: y el fin de
todas sus controversias es el juramento para confirmación. \footnote{\textbf{6:16}
  Éxod 22,10} \bibverse{17} Por lo cual, queriendo Dios mostrar más
abundantemente á los herederos de la promesa la inmutabilidad de su
consejo, interpuso juramento; \bibverse{18} Para que por dos cosas
inmutables, en las cuales es imposible que Dios mienta, tengamos un
fortísimo consuelo, los que nos acogemos á trabarnos de la esperanza
propuesta: \bibverse{19} La cual tenemos como segura y firme ancla del
alma, y que entra hasta dentro del velo; \bibverse{20} Donde entró por
nosotros como precursor Jesús, hecho Pontífice eternalmente según el
orden de Melchîsedec. \footnote{\textbf{6:20} Heb 5,6}

\hypertarget{jesuxfas-el-sumo-sacerdote-perfecto-para-siempre-seguxfan-el-orden-de-melquisedec}{%
\subsection{Jesús, el sumo sacerdote perfecto para siempre según el
orden de
Melquisedec}\label{jesuxfas-el-sumo-sacerdote-perfecto-para-siempre-seguxfan-el-orden-de-melquisedec}}

\hypertarget{section-6}{%
\section{7}\label{section-6}}

\bibverse{1} Porque este Melchîsedec, rey de Salem, sacerdote del Dios
Altísimo, el cual salió á recibir á Abraham que volvía de la derrota de
los reyes, y le bendijo, \footnote{\textbf{7:1} Gén 14,18-20}
\bibverse{2} Al cual asimismo dió Abraham los diezmos de todo,
primeramente él se interpreta Rey de justicia; y luego también Rey de
Salem, que es, Rey de paz; \bibverse{3} Sin padre, sin madre, sin
linaje; que ni tiene principio de días, ni fin de vida, mas hecho
semejante al Hijo de Dios, permanece sacerdote para siempre. \footnote{\textbf{7:3}
  Juan 7,27}

\hypertarget{melquisedec-es-muxe1s-digno-que-los-sacerdotes-levitas}{%
\subsection{Melquisedec es más digno que los sacerdotes
levitas}\label{melquisedec-es-muxe1s-digno-que-los-sacerdotes-levitas}}

\bibverse{4} Mirad pues cuán grande fué éste, al cual aun Abraham el
patriarca dió diezmos de los despojos. \bibverse{5} Y ciertamente los
que de los hijos de Leví toman el sacerdocio, tienen mandamiento de
tomar del pueblo los diezmos según la ley, es á saber, de sus hermanos
aunque también hayan salido de los lomos de Abraham. \bibverse{6} Mas
aquél cuya genealogía no es contada de ellos, tomó de Abraham los
diezmos, y bendijo al que tenía las promesas. \bibverse{7} Y sin
contradicción alguna, lo que es menos es bendecido de lo que es más.
\bibverse{8} Y aquí ciertamente los hombres mortales toman los diezmos:
mas allí, aquel del cual está dado testimonio que vive. \bibverse{9} Y,
por decirlo así, en Abraham fué diezmado también Leví, que recibe los
diezmos; \bibverse{10} Porque aun estaba en los lomos de su padre cuando
Melchîsedec le salió al encuentro.

\hypertarget{el-cambio-y-aboliciuxf3n-del-sacerdocio-provocado-por-el-sacerdocio-peculiar-de-jesuxfas}{%
\subsection{El cambio y abolición del sacerdocio provocado por el
sacerdocio peculiar de
Jesús}\label{el-cambio-y-aboliciuxf3n-del-sacerdocio-provocado-por-el-sacerdocio-peculiar-de-jesuxfas}}

\bibverse{11} Si pues la perfección era por el sacerdocio Levítico
(porque debajo de él recibió el pueblo la ley) ¿qué necesidad había aún
de que se levantase otro sacerdote según el orden de Melchîsedec, y que
no fuese llamado según el orden de Aarón? \bibverse{12} Pues mudado el
sacerdocio, necesario es que se haga también mudanza de la ley.
\bibverse{13} Porque aquel del cual esto se dice, de otra tribu es, de
la cual nadie asistió al altar. \bibverse{14} Porque notorio es que el
Señor nuestro nació de la tribu de Judá, sobre cuya tribu nada habló
Moisés tocante al sacerdocio. \footnote{\textbf{7:14} Gén 49,10; Is
  11,1; Mat 1,1-3} \bibverse{15} Y aun más manifiesto es, si á semejanza
de Melchîsedec se levanta otro sacerdote, \bibverse{16} El cual no es
hecho conforme á la ley del mandamiento carnal, sino según la virtud de
vida indisoluble; \bibverse{17} Pues se da testimonio de él: Tú eres
sacerdote para siempre, según el orden de Melchîsedec.

\hypertarget{la-razuxf3n-del-cambio-en-el-orden-de-los-sacerdotes-es-que-jesuxfas-deberuxeda-ser-el-garante-de-un-pacto-superior}{%
\subsection{La razón del cambio en el orden de los sacerdotes es que
Jesús debería ser el garante de un pacto
superior}\label{la-razuxf3n-del-cambio-en-el-orden-de-los-sacerdotes-es-que-jesuxfas-deberuxeda-ser-el-garante-de-un-pacto-superior}}

\bibverse{18} El mandamiento precedente, cierto se abroga por su
flaqueza é inutilidad; \bibverse{19} Porque nada perfeccionó la ley; mas
hízolo la introducción de mejor esperanza, por la cual nos acercamos á
Dios. \bibverse{20} Y por cuanto no fué sin juramento, \bibverse{21}
(Porque los otros cierto sin juramento fueron hechos sacerdotes; mas
éste, con juramento por el que le dijo: Juró el Señor, y no se
arrepentirá: Tú eres sacerdote eternamente según el orden de
Melchîsedec:)

\bibverse{22} Tanto de mejor testamento es hecho fiador Jesús.
\footnote{\textbf{7:22} Heb 8,6; Heb 12,24}

\bibverse{23} Y los otros cierto fueron muchos sacerdotes, en cuanto por
la muerte no podían permanecer. \bibverse{24} Mas éste, por cuanto
permanece para siempre, tiene un sacerdocio inmutable: \bibverse{25} Por
lo cual puede también salvar eternamente á los que por él se allegan á
Dios, viviendo siempre para interceder por ellos.

\hypertarget{jesuxfas-como-el-sumo-sacerdote-perfecto-y-eterno}{%
\subsection{Jesús como el sumo sacerdote perfecto y
eterno}\label{jesuxfas-como-el-sumo-sacerdote-perfecto-y-eterno}}

\bibverse{26} Porque tal pontífice nos convenía: santo, inocente,
limpio, apartado de los pecadores, y hecho más sublime que los cielos;
\bibverse{27} Que no tiene necesidad cada día, como los otros
sacerdotes, de ofrecer primero sacrificios por sus pecados, y luego por
los del pueblo: porque esto lo hizo una sola vez, ofreciéndose á sí
mismo. \footnote{\textbf{7:27} Lev 16,6; Lev 16,15}

\bibverse{28} Porque la ley constituye sacerdotes á hombres flacos; mas
la palabra del juramento, después de la ley, constituye al Hijo, hecho
perfecto para siempre.

\hypertarget{la-superioridad-del-ministerio-sumo-sacerdotal-celestial-de-jesuxfas-y-el-nuevo-pacto-del-que-uxe9l-es-mediador}{%
\subsection{La superioridad del ministerio sumo sacerdotal celestial de
Jesús y el nuevo pacto del que él es
mediador}\label{la-superioridad-del-ministerio-sumo-sacerdotal-celestial-de-jesuxfas-y-el-nuevo-pacto-del-que-uxe9l-es-mediador}}

\hypertarget{section-7}{%
\section{8}\label{section-7}}

\bibverse{1} Así que, la suma acerca de lo dicho es: Tenemos tal
pontífice que se asentó á la diestra del trono de la Majestad en los
cielos; \bibverse{2} Ministro del santuario, y de aquel verdadero
tabernáculo que el Señor asentó, y no hombre. \bibverse{3} Porque todo
pontífice es puesto para ofrecer presentes y sacrificios; por lo cual es
necesario que también éste tuviese algo que ofrecer. \bibverse{4} Así
que, si estuviese sobre la tierra, ni aun sería sacerdote, habiendo aún
los sacerdotes que ofrecen los presentes según la ley; \bibverse{5} Los
cuales sirven de bosquejo y sombra de las cosas celestiales, como fué
respondido á Moisés cuando había de acabar el tabernáculo: Mira, dice,
haz todas las cosas conforme al dechado que te ha sido mostrado en el
monte. \footnote{\textbf{8:5} Col 2,17} \bibverse{6} Mas ahora tanto
mejor ministerio es el suyo, cuanto es mediador de un mejor pacto, el
cual ha sido formado sobre mejores promesas. \footnote{\textbf{8:6} Heb
  7,22}

\bibverse{7} Porque si aquel primero fuera sin falta, cierto no se
hubiera procurado lugar de segundo. \bibverse{8} Porque reprendiéndolos
dice: He aquí vienen días, dice el Señor, y consumaré para con la casa
de Israel y para con la casa de Judá un nuevo pacto; \footnote{\textbf{8:8}
  Heb 10,16-17} \bibverse{9} No como el pacto que hice con sus padres el
día que los tomé por la mano para sacarlos de la tierra de Egipto:
porque ellos no permanecieron en mi pacto, y yo los menosprecié, dice el
Señor. \footnote{\textbf{8:9} Éxod 19,5-6} \bibverse{10} Por lo cual,
este es el pacto que ordenaré á la casa de Israel después de aquellos
días, dice el Señor: Daré mis leyes en el alma de ellos, y sobre el
corazón de ellos las escribiré; y seré á ellos por Dios, y ellos me
serán á mí por pueblo: \bibverse{11} Y ninguno enseñará á su prójimo, ni
ninguno á su hermano, diciendo: Conoce al Señor: porque todos me
conocerán, desde el menor de ellos hasta el mayor. \bibverse{12} Porque
seré propicio á sus injusticias, y de sus pecados y de sus iniquidades
no me acordaré más.

\bibverse{13} Diciendo, Nuevo pacto, dió por viejo al primero; y lo que
es dado por viejo y se envejece, cerca está de desvanecerse. \footnote{\textbf{8:13}
  Rom 10,4}

\hypertarget{la-imperfecciuxf3n-del-ministerio-sacerdotal-levuxedtico-y-la-perfecciuxf3n-o-superioridad-del-ministerio-sumo-sacerdotal-de-cristo}{%
\subsection{La imperfección del ministerio sacerdotal levítico y la
perfección (o superioridad) del ministerio sumo sacerdotal de
Cristo}\label{la-imperfecciuxf3n-del-ministerio-sacerdotal-levuxedtico-y-la-perfecciuxf3n-o-superioridad-del-ministerio-sumo-sacerdotal-de-cristo}}

\hypertarget{section-8}{%
\section{9}\label{section-8}}

\bibverse{1} Tenía empero también el primer pacto reglamentos del culto,
y santuario mundano. \bibverse{2} Porque el tabernáculo fué hecho: el
primero, en que estaban las lámparas, y la mesa, y los panes de la
proposición; lo que llaman el Santuario. \bibverse{3} Tras el segundo
velo estaba el tabernáculo, que llaman el Lugar Santísimo; \footnote{\textbf{9:3}
  Éxod 26,33} \bibverse{4} El cual tenía un incensario de oro, y el arca
del pacto cubierta de todas partes alrededor de oro; en la que estaba
una urna de oro que contenía el maná, y la vara de Aarón que reverdeció,
y las tablas del pacto; \footnote{\textbf{9:4} Éxod 16,33; Éxod
  25,10-22; Núm 17,23-25} \bibverse{5} Y sobre ella los querubines de
gloria que cubrían el propiciatorio; de las cuales cosas no se puede
ahora hablar en particular.

\bibverse{6} Y estas cosas así ordenadas, en el primer tabernáculo
siempre entraban los sacerdotes para hacer los oficios del culto;
\footnote{\textbf{9:6} Núm 18,3-4; Éxod 30,10; Lev 16,2; Lev 16,14-15}
\bibverse{7} Mas en el segundo, sólo el pontífice una vez en el año, no
sin sangre, la cual ofrece por sí mismo, y por los pecados de ignorancia
del pueblo: \bibverse{8} Dando en esto á entender el Espíritu Santo, que
aun no estaba descubierto el camino para el santuario, entre tanto que
el primer tabernáculo estuviese en pie. \bibverse{9} Lo cual era figura
de aquel tiempo presente, en el cual se ofrecían presentes y sacrificios
que no podían hacer perfecto, cuanto á la conciencia, al que servía con
ellos; \footnote{\textbf{9:9} Heb 7,19; Heb 10,1-2} \bibverse{10}
Consistiendo sólo en viandas y en bebidas, y en diversos lavamientos, y
ordenanzas acerca de la carne, impuestas hasta el tiempo de la
corrección. \footnote{\textbf{9:10} Lev 11,-1; Núm 19,-1}

\bibverse{11} Mas estando ya presente Cristo, pontífice de los bienes
que habían de venir, por el más amplio y más perfecto tabernáculo, no
hecho de manos, es á saber, no de esta creación; \bibverse{12} Y no por
sangre de machos cabríos ni de becerros, mas por su propia sangre, entró
una sola vez en el santuario, habiendo obtenido eterna redención.
\bibverse{13} Porque si la sangre de los toros y de los machos cabríos,
y la ceniza de la becerra, rociada á los inmundos, santifica para la
purificación de la carne, \footnote{\textbf{9:13} Núm 19,2; Núm 19,9;
  Núm 19,17} \bibverse{14} ¿Cuánto más la sangre de Cristo, el cual por
el Espíritu eterno se ofreció á sí mismo sin mancha á Dios, limpiará
vuestras conciencias de las obras de muerte para que sirváis al Dios
vivo? \footnote{\textbf{9:14} Heb 1,3; 1Pe 1,18-19; 1Jn 1,7; Apoc 1,5}

\hypertarget{cristo-como-mediador-de-un-nuevo-pacto-y-su-muerte-sacrificial-uxfanica-como-medio-eterno-de-su-servicio-celestial-como-sumo-sacerdote}{%
\subsection{Cristo como mediador de un nuevo pacto y su muerte
sacrificial única como medio eterno de su servicio celestial como sumo
sacerdote}\label{cristo-como-mediador-de-un-nuevo-pacto-y-su-muerte-sacrificial-uxfanica-como-medio-eterno-de-su-servicio-celestial-como-sumo-sacerdote}}

\bibverse{15} Así que, por eso es mediador del nuevo testamento, para
que interviniendo muerte para la remisión de las rebeliones que había
bajo del primer testamento, los que son llamados reciban la promesa de
la herencia eterna. \footnote{\textbf{9:15} Heb 12,24; 1Tim 2,5}
\bibverse{16} Porque donde hay testamento, necesario es que intervenga
muerte del testador. \bibverse{17} Porque el testamento con la muerte es
confirmado; de otra manera no es válido entre tanto que el testador
vive. \bibverse{18} De donde vino que ni aun el primero fué consagrado
sin sangre. \bibverse{19} Porque habiendo leído Moisés todos los
mandamientos de la ley á todo el pueblo, tomando la sangre de los
becerros y de los machos cabríos, con agua, y lana de grana, é hisopo,
roció al mismo libro, y también á todo el pueblo, \bibverse{20}
Diciendo: Esta es la sangre del testamento que Dios os ha mandado.

\bibverse{21} Y además de esto roció también con la sangre el
tabernáculo y todos los vasos del ministerio. \footnote{\textbf{9:21}
  Lev 8,15; Lev 8,19} \bibverse{22} Y casi todo es purificado según la
ley con sangre; y sin derramamiento de sangre no se hace remisión.
\footnote{\textbf{9:22} Lev 17,11}

\hypertarget{el-autosacrificio-uxfanico-y-sangriento-de-cristo-y-su-tremendo-significado-de-salvaciuxf3n-para-los-creyentes}{%
\subsection{El autosacrificio único y sangriento de Cristo y su tremendo
significado de salvación para los
creyentes}\label{el-autosacrificio-uxfanico-y-sangriento-de-cristo-y-su-tremendo-significado-de-salvaciuxf3n-para-los-creyentes}}

\bibverse{23} Fué, pues, necesario que las figuras de las cosas
celestiales fuesen purificadas con estas cosas; empero las mismas cosas
celestiales con mejores sacrificios que éstos. \bibverse{24} Porque no
entró Cristo en el santuario hecho de mano, figura del verdadero, sino
en el mismo cielo para presentarse ahora por nosotros en la presencia de
Dios. \footnote{\textbf{9:24} Heb 7,25; 1Jn 2,1} \bibverse{25} Y no para
ofrecerse muchas veces á sí mismo, como entra el pontífice en el
santuario cada año con sangre ajena; \bibverse{26} De otra manera fuera
necesario que hubiera padecido muchas veces desde el principio del
mundo: mas ahora una vez en la consumación de los siglos, para
deshacimiento del pecado se presentó por el sacrificio de sí mismo.
\bibverse{27} Y de la manera que está establecido á los hombres que
mueran una vez, y después el juicio; \footnote{\textbf{9:27} Gén 3,19}
\bibverse{28} Así también Cristo fué ofrecido una vez para agotar los
pecados de muchos; y la segunda vez, sin pecado, será visto de los que
le esperan para salud. \footnote{\textbf{9:28} Heb 10,10; Heb 10,12; Heb
  10,14}

\hypertarget{el-ejemplo-sombruxedo-y-la-insuficiencia-del-sacrificio-anual-de-reconciliaciuxf3n-del-sumo-sacerdote-levuxedtico-la-perfecciuxf3n-del-sacrificio-de-jesuxfas}{%
\subsection{El ejemplo sombrío y la insuficiencia del sacrificio anual
de reconciliación del sumo sacerdote levítico; la perfección del
sacrificio de
Jesús}\label{el-ejemplo-sombruxedo-y-la-insuficiencia-del-sacrificio-anual-de-reconciliaciuxf3n-del-sumo-sacerdote-levuxedtico-la-perfecciuxf3n-del-sacrificio-de-jesuxfas}}

\hypertarget{section-9}{%
\section{10}\label{section-9}}

\bibverse{1} Porque la ley, teniendo la sombra de los bienes venideros,
no la imagen misma de las cosas, nunca puede, por los mismos sacrificios
que ofrecen continuamente cada año, hacer perfectos á los que se
allegan. \footnote{\textbf{10:1} Heb 8,5} \bibverse{2} De otra manera
cesarían de ofrecerse; porque los que tributan este culto, limpios de
una vez, no tendrían más conciencia de pecado. \bibverse{3} Empero en
estos sacrificios cada año se hace conmemoración de los pecados.
\bibverse{4} Porque la sangre de los toros y de los machos cabríos no
puede quitar los pecados. \bibverse{5} Por lo cual, entrando en el
mundo, dice: Sacrificio y presente no quisiste; mas me apropiaste
cuerpo: \bibverse{6} Holocaustos y expiaciones por el pecado no te
agradaron. \bibverse{7} Entonces dije: Heme aquí (en la cabecera del
libro está escrito de mí) para que haga, oh Dios, tu voluntad.

\bibverse{8} Diciendo arriba: Sacrificio y presente, y holocaustos y
expiaciones por el pecado no quisiste, ni te agradaron, (las cuales
cosas se ofrecen según la ley,) \bibverse{9} Entonces dijo: Heme aquí
para que haga, oh Dios, tu voluntad. Quita lo primero, para establecer
lo postrero. \bibverse{10} En la cual voluntad somos santificados por la
ofrenda del cuerpo de Jesucristo hecha una sola vez. \footnote{\textbf{10:10}
  Juan 17,19}

\hypertarget{el-autosacrificio-uxfanico-y-perfectamente-vuxe1lido-de-jesuxfas-hace-que-todos-los-demuxe1s-sacrificios-por-el-pecado-sean-innecesarios-porque-hizo-que-los-creyentes-fueran-completamente-perfectos-ante-dios}{%
\subsection{El autosacrificio único y perfectamente válido de Jesús hace
que todos los demás sacrificios por el pecado sean innecesarios porque
hizo que los creyentes fueran completamente perfectos ante
Dios}\label{el-autosacrificio-uxfanico-y-perfectamente-vuxe1lido-de-jesuxfas-hace-que-todos-los-demuxe1s-sacrificios-por-el-pecado-sean-innecesarios-porque-hizo-que-los-creyentes-fueran-completamente-perfectos-ante-dios}}

\bibverse{11} Así que, todo sacerdote se presenta cada día ministrando y
ofreciendo muchas veces los mismos sacrificios, que nunca pueden quitar
los pecados: \footnote{\textbf{10:11} Éxod 29,38} \bibverse{12} Pero
éste, habiendo ofrecido por los pecados un solo sacrificio para siempre,
está sentado á la diestra de Dios, \bibverse{13} Esperando lo que resta,
hasta que sus enemigos sean puestos por estrado de sus pies. \footnote{\textbf{10:13}
  Sal 110,1} \bibverse{14} Porque con una sola ofrenda hizo perfectos
para siempre á los santificados.

\bibverse{15} Y atestíguanos lo mismo el Espíritu Santo; que después que
dijo: \bibverse{16} Y este es el pacto que haré con ellos después de
aquellos días, dice el Señor: Daré mis leyes en sus corazones, y en sus
almas las escribiré;

\bibverse{17} Añade: Y nunca más me acordaré de sus pecados é
iniquidades. \footnote{\textbf{10:17} Heb 8,12}

\bibverse{18} Pues donde hay remisión de éstos, no hay más ofrenda por
pecado.

\hypertarget{amonestaciuxf3n-general-para-perseverar-en-la-fe-la-esperanza-y-el-amor-en-comunidad-con-toda-la-comunidad}{%
\subsection{Amonestación general para perseverar en la fe, la esperanza
y el amor, en comunidad con toda la
comunidad}\label{amonestaciuxf3n-general-para-perseverar-en-la-fe-la-esperanza-y-el-amor-en-comunidad-con-toda-la-comunidad}}

\bibverse{19} Así que, hermanos, teniendo libertad para entrar en el
santuario por la sangre de Jesucristo, \bibverse{20} Por el camino que
él nos consagró nuevo y vivo, por el velo, esto es, por su carne;
\footnote{\textbf{10:20} Heb 9,8} \bibverse{21} Y teniendo un gran
sacerdote sobre la casa de Dios, \bibverse{22} Lleguémonos con corazón
verdadero, en plena certidumbre de fe, purificados los corazones de mala
conciencia, y lavados los cuerpos con agua limpia.

\bibverse{23} Mantengamos firme la profesión de nuestra fe sin fluctuar;
que fiel es el que prometió: \footnote{\textbf{10:23} Heb 4,14}

\bibverse{24} Y considerémonos los unos á los otros para provocarnos al
amor y á las buenas obras; \bibverse{25} No dejando nuestra
congregación, como algunos tienen por costumbre, mas exhortándonos; y
tanto más, cuanto veis que aquel día se acerca.

\hypertarget{advertencia-de-apostasuxeda-y-del-juicio-divino-que-golpearuxe1-a-los-que-se-burlan-de-la-gracia}{%
\subsection{Advertencia de apostasía y del juicio divino que golpeará a
los que se burlan de la
gracia}\label{advertencia-de-apostasuxeda-y-del-juicio-divino-que-golpearuxe1-a-los-que-se-burlan-de-la-gracia}}

\bibverse{26} Porque si pecáremos voluntariamente después de haber
recibido el conocimiento de la verdad, ya no queda sacrificio por el
pecado. \footnote{\textbf{10:26} Heb 6,4-8} \bibverse{27} Sino una
horrenda esperanza de juicio, y hervor de fuego que ha de devorar á los
adversarios. \bibverse{28} El que menospreciare la ley de Moisés, por el
testimonio de dos ó de tres testigos muere sin ninguna misericordia:
\bibverse{29} ¿Cuánto pensáis que será más digno de mayor castigo, el
que hollare al Hijo de Dios, y tuviere por inmunda la sangre del
testamento, en la cual fué santificado, é hiciere afrenta al Espíritu de
gracia? \footnote{\textbf{10:29} Heb 2,3; Heb 12,25} \bibverse{30}
Sabemos quién es el que dijo: Mía es la venganza, yo daré el pago, dice
el Señor. Y otra vez: El Señor juzgará su pueblo. \bibverse{31} Horrenda
cosa es caer en las manos del Dios vivo.

\hypertarget{recordatorio-para-ser-fiel-y-tener-confianza-en-la-esperanza-frente-al-sufrimiento-creciente-en-vista-de-la-recompensa-prometida}{%
\subsection{Recordatorio para ser fiel y tener confianza en la esperanza
frente al sufrimiento creciente en vista de la recompensa
prometida}\label{recordatorio-para-ser-fiel-y-tener-confianza-en-la-esperanza-frente-al-sufrimiento-creciente-en-vista-de-la-recompensa-prometida}}

\bibverse{32} Empero traed á la memoria los días pasados, en los cuales,
después de haber sido iluminados, sufristeis gran combate de
aflicciones: \footnote{\textbf{10:32} Heb 6,4} \bibverse{33} Por una
parte, ciertamente, con vituperios y tribulaciones fuisteis hechos
espectáculo; y por otra parte hechos compañeros de los que estaban en
tal estado. \footnote{\textbf{10:33} 1Cor 4,9} \bibverse{34} Porque de
mis prisiones también os resentisteis conmigo, y el robo de vuestros
bienes padecisteis con gozo, conociendo que tenéis en vosotros una mejor
sustancia en los cielos, y que permanece. \footnote{\textbf{10:34} Mat
  6,20; Mat 19,21; Mat 19,29} \bibverse{35} No perdáis pues vuestra
confianza, que tiene grande remuneración de galardón: \bibverse{36}
Porque la paciencia os es necesaria; para que, habiendo hecho la
voluntad de Dios, obtengáis la promesa. \bibverse{37} Porque aun un
poquito, y el que ha de venir vendrá, y no tardará. \bibverse{38} Ahora
el justo vivirá por fe; mas si se retirare, no agradará á mi alma.
\footnote{\textbf{10:38} Rom 1,17}

\bibverse{39} Pero nosotros no somos tales que nos retiremos para
perdición, sino fieles para ganancia del alma. \footnote{\textbf{10:39}
  1Tes 3,3}

\hypertarget{section-10}{%
\section{11}\label{section-10}}

\bibverse{1} Es pues la fe la sustancia de las cosas que se esperan, la
demostración de las cosas que no se ven. \footnote{\textbf{11:1} 2Cor
  5,7}

\hypertarget{modelos-del-antiguo-testamento-de-tal-fe}{%
\subsection{Modelos del Antiguo Testamento de tal
fe}\label{modelos-del-antiguo-testamento-de-tal-fe}}

\bibverse{2} Porque por ella alcanzaron testimonio los antiguos.
\bibverse{3} Por la fe entendemos haber sido compuestos los siglos por
la palabra de Dios, siendo hecho lo que se ve, de lo que no se veía.

\hypertarget{tres-ejemplos-de-huxe9roes-de-la-fe-de-la-uxe9poca-de-los-antepasados-de-abel-a-nouxe9}{%
\subsection{Tres ejemplos de héroes de la fe de la época de los
antepasados \hspace{0pt}\hspace{0pt}de Abel a
Noé}\label{tres-ejemplos-de-huxe9roes-de-la-fe-de-la-uxe9poca-de-los-antepasados-de-abel-a-nouxe9}}

\bibverse{4} Por la fe Abel ofreció á Dios mayor sacrificio que Caín,
por la cual alcanzó testimonio de que era justo, dando Dios testimonio á
sus presentes; y difunto, aun habla por ella. \footnote{\textbf{11:4}
  Gén 4,4}

\bibverse{5} Por la fe Enoc fué traspuesto para no ver muerte, y no fué
hallado, porque lo traspuso Dios. Y antes que fuese traspuesto, tuvo
testimonio de haber agradado á Dios. \footnote{\textbf{11:5} Gén 5,24}
\bibverse{6} Empero sin fe es imposible agradar á Dios; porque es
menester que el que á Dios se allega, crea que le hay, y que es
galardonador de los que le buscan.

\bibverse{7} Por la fe Noé, habiendo recibido respuesta de cosas que aun
no se veían, con temor aparejó el arca en que su casa se salvase: por la
cual fe condenó al mundo, y fué hecho heredero de la justicia que es por
la fe. \footnote{\textbf{11:7} Gén 6,8-9; Gén 6,13-22}

\hypertarget{ejemplos-de-la-uxe9poca-de-abraham-y-su-familia}{%
\subsection{Ejemplos de la época de Abraham y su
familia}\label{ejemplos-de-la-uxe9poca-de-abraham-y-su-familia}}

\bibverse{8} Por la fe Abraham, siendo llamado, obedeció para salir al
lugar que había de recibir por heredad; y salió sin saber dónde iba.
\footnote{\textbf{11:8} Gén 12,1-21} \bibverse{9} Por fe habitó en la
tierra prometida como en tierra ajena, morando en cabañas con Isaac y
Jacob, herederos juntamente de la misma promesa: \bibverse{10} Porque
esperaba ciudad con fundamentos, el artífice y hacedor de la cual es
Dios.

\bibverse{11} Por la fe también la misma Sara, siendo estéril, recibió
fuerza para concebir simiente; y parió aun fuera del tiempo de la edad,
porque creyó ser fiel el que lo había prometido. \bibverse{12} Por lo
cual también, de uno, y ése ya amortecido, salieron como las estrellas
del cielo en multitud, y como la arena innumerable que está á la orilla
de la mar.

\bibverse{13} Conforme á la fe murieron todos éstos sin haber recibido
las promesas, sino mirándolas de lejos, y creyéndolas, y saludándolas, y
confesando que eran peregrinos y advenedizos sobre la tierra.
\footnote{\textbf{11:13} Gén 23,4; Gén 47,9} \bibverse{14} Porque los
que esto dicen, claramente dan á entender que buscan una patria.
\bibverse{15} Que si se acordaran de aquella de donde salieron, cierto
tenían tiempo para volverse: \bibverse{16} Empero deseaban la mejor, es
á saber, la celestial; por lo cual Dios no se avergüenza de llamarse
Dios de ellos: porque les había aparejado ciudad.

\bibverse{17} Por fe ofreció Abraham á Isaac cuando fué probado, y
ofrecía al unigénito el que había recibido las promesas, \footnote{\textbf{11:17}
  Gén 22,-1; Sant 2,21} \bibverse{18} Habiéndole sido dicho: En Isaac te
será llamada simiente: \bibverse{19} Pensando que aun de los muertos es
Dios poderoso para levantar; de donde también le volvió á recibir por
figura.

\bibverse{20} Por fe bendijo Isaac á Jacob y á Esaú respecto á cosas que
habían de ser.

\bibverse{21} Por fe Jacob, muriéndose, bendijo á cada uno de los hijos
de José, y adoró estribando sobre la punta de su bordón.

\bibverse{22} Por fe José, muriéndose, se acordó de la partida de los
hijos de Israel; y dió mandamiento acerca de sus huesos.

\hypertarget{ejemplos-de-la-uxe9poca-de-moisuxe9s-y-josuuxe9}{%
\subsection{Ejemplos de la época de Moisés y
Josué}\label{ejemplos-de-la-uxe9poca-de-moisuxe9s-y-josuuxe9}}

\bibverse{23} Por fe Moisés, nacido, fué escondido de sus padres por
tres meses, porque le vieron hermoso niño; y no temieron el mandamiento
del rey. \footnote{\textbf{11:23} Éxod 2,-1; Éxod 12,1-12; Éxod 14,1-14}

\bibverse{24} Por fe Moisés, hecho ya grande, rehusó ser llamado hijo de
la hija de Faraón; \bibverse{25} Escogiendo antes ser afligido con el
pueblo de Dios, que gozar de comodidades temporales de pecado.
\bibverse{26} Teniendo por mayores riquezas el vituperio de Cristo que
los tesoros de los Egipcios; porque miraba á la remuneración.
\bibverse{27} Por fe dejó á Egipto, no temiendo la ira del rey; porque
se sostuvo como viendo al Invisible. \bibverse{28} Por fe celebró la
pascua y el derramamiento de la sangre, para que el que mataba los
primogénitos no los tocase.

\bibverse{29} Por fe pasaron el mar Bermejo como por tierra seca: lo
cual probando los Egipcios, fueron sumergidos.

\bibverse{30} Por fe cayeron los muros de Jericó con rodearlos siete
días.

\bibverse{31} Por fe Rahab la ramera no pereció juntamente con los
incrédulos, habiendo recibido á los espías con paz.

\hypertarget{ejemplos-de-huxe9roes-de-la-fe-de-la-historia-posterior-de-israel}{%
\subsection{Ejemplos de héroes de la fe de la historia posterior de
Israel}\label{ejemplos-de-huxe9roes-de-la-fe-de-la-historia-posterior-de-israel}}

\bibverse{32} ¿Y qué más digo? porque el tiempo me faltará contando de
Gedeón, de Barac, de Samsón, de Jephté, de David, de Samuel, y de los
profetas: \bibverse{33} Que por fe ganaron reinos, obraron justicia,
alcanzaron promesas, taparon las bocas de leones, \bibverse{34} Apagaron
fuegos impetuosos, evitaron filo de cuchillo, convalecieron de
enfermedades, fueron hechos fuertes en batallas, trastornaron campos de
extraños. \bibverse{35} Las mujeres recibieron sus muertos por
resurrección; unos fueron estirados, no aceptando el rescate, para ganar
mejor resurrección; \bibverse{36} Otros experimentaron vituperios y
azotes; y á más de esto prisiones y cárceles; \bibverse{37} Fueron
apedreados, aserrados, tentados, muertos á cuchillo; anduvieron de acá
para allá cubiertos de pieles de ovejas y de cabras, pobres,
angustiados, maltratados; \bibverse{38} De los cuales el mundo no era
digno; perdidos por los desiertos, por los montes, por las cuevas y por
las cavernas de la tierra.

\bibverse{39} Y todos éstos, aprobados por testimonio de la fe, no
recibieron la promesa; \bibverse{40} Proveyendo Dios alguna cosa mejor
para nosotros, para que no fuesen perfeccionados sin nosotros.

\hypertarget{exhortaciuxf3n-a-mantener-la-fidelidad-especialmente-con-respecto-al-ejemplo-de-jesuxfas}{%
\subsection{Exhortación a mantener la fidelidad, especialmente con
respecto al ejemplo de
Jesús}\label{exhortaciuxf3n-a-mantener-la-fidelidad-especialmente-con-respecto-al-ejemplo-de-jesuxfas}}

\hypertarget{section-11}{%
\section{12}\label{section-11}}

\bibverse{1} Por tanto nosotros también, teniendo en derredor nuestro
una tan grande nube de testigos, dejando todo el peso del pecado que nos
rodea, corramos con paciencia la carrera que nos es propuesta,
\bibverse{2} Puestos los ojos en al autor y consumador de la fe, en
Jesús; el cual, habiéndole sido propuesto gozo, sufrió la cruz,
menospreciando la vergüenza, y sentóse á la diestra del trono de Dios.
\footnote{\textbf{12:2} Heb 5,8-9; Fil 2,8; Fil 2,10}

\bibverse{3} Reducid pues á vuestro pensamiento á aquel que sufrió tal
contradicción de pecadores contra sí mismo, porque no os fatiguéis en
vuestros ánimos desmayando. \footnote{\textbf{12:3} Luc 2,34; Mat 26,67}

\hypertarget{recordatorio-para-permitir-que-los-desafuxedos-del-sufrimiento-sirvan-como-medio-para-promover-la-vida-de-fe}{%
\subsection{Recordatorio para permitir que los desafíos del sufrimiento
sirvan como medio para promover la vida de
fe}\label{recordatorio-para-permitir-que-los-desafuxedos-del-sufrimiento-sirvan-como-medio-para-promover-la-vida-de-fe}}

\bibverse{4} Que aun no habéis resistido hasta la sangre, combatiendo
contra el pecado: \bibverse{5} Y estáis ya olvidados de la exhortación
que como con hijos habla con vosotros, diciendo: Hijo mío, no
menosprecies el castigo del Señor, ni desmayes cuando eres de él
reprendido. \bibverse{6} Porque el Señor al que ama castiga, y azota á
cualquiera que recibe por hijo. \footnote{\textbf{12:6} Apoc 3,19}

\bibverse{7} Si sufrís el castigo, Dios se os presenta como á hijos;
porque ¿qué hijo es aquel á quien el padre no castiga? \bibverse{8} Mas
si estáis fuera del castigo, del cual todos han sido hechos
participantes, luego sois bastardos, y no hijos. \bibverse{9} Por otra
parte, tuvimos por castigadores á los padres de nuestra carne, y los
reverenciábamos, ¿por qué no obedeceremos mucho mejor al Padre de los
espíritus, y viviremos? \bibverse{10} Y aquéllos, á la verdad, por pocos
días nos castigaban como á ellos les parecía, mas éste para lo que nos
es provechoso, para que recibamos su santificación. \bibverse{11} Es
verdad que ningún castigo al presente parece ser causa de gozo, sino de
tristeza; mas después da fruto apacible de justicia á los que en él son
ejercitados.

\hypertarget{una-advertencia-a-la-comunidad-para-que-se-levante-y-cuide-a-los-miembros-duxe9biles-y-vulnerables}{%
\subsection{Una advertencia a la comunidad para que se levante y cuide a
los miembros débiles y
vulnerables}\label{una-advertencia-a-la-comunidad-para-que-se-levante-y-cuide-a-los-miembros-duxe9biles-y-vulnerables}}

\bibverse{12} Por lo cual alzad las manos caídas y las rodillas
paralizadas; \footnote{\textbf{12:12} Is 35,3} \bibverse{13} Y haced
derechos pasos á vuestros pies, porque lo que es cojo no salga fuera de
camino, antes sea sanado. \footnote{\textbf{12:13} Prov 4,26-27}

\bibverse{14} Seguid la paz con todos, y la santidad, sin la cual nadie
verá al Señor: \footnote{\textbf{12:14} Rom 12,18; 2Tim 2,22}
\bibverse{15} Mirando bien que ninguno se aparte de la gracia de Dios,
que ninguna raíz de amargura brotando os impida, y por ella muchos sean
contaminados; \footnote{\textbf{12:15} Deut 29,17} \bibverse{16} Que
ninguno sea fornicario, ó profano, como Esaú, que por una vianda vendió
su primogenitura. \footnote{\textbf{12:16} Gén 25,33-34} \bibverse{17}
Porque ya sabéis que aun después, deseando heredar la bendición, fué
reprobado (que no halló lugar de arrepentimiento), aunque la procuró con
lágrimas. \footnote{\textbf{12:17} Gén 27,30-40}

\hypertarget{otra-referencia-a-la-soberanuxeda-del-nuevo-pacto-y-la-inminente-decisiuxf3n-final}{%
\subsection{Otra referencia a la soberanía del nuevo pacto y la
inminente decisión
final}\label{otra-referencia-a-la-soberanuxeda-del-nuevo-pacto-y-la-inminente-decisiuxf3n-final}}

\bibverse{18} Porque no os habéis llegado al monte que se podía tocar, y
al fuego encendido, y al turbión, y á la oscuridad, y á la tempestad,
\footnote{\textbf{12:18} Éxod 19,12; Éxod 19,16; Éxod 19,18; Deut 4,11}
\bibverse{19} Y al sonido de la trompeta, y á la voz de las palabras, la
cual los que la oyeron rogaron que no se les hablase más; \footnote{\textbf{12:19}
  Éxod 20,19} \bibverse{20} Porque no podían tolerar lo que se mandaba:
Si bestia tocare al monte, será apedreada, ó pasada con dardo.
\bibverse{21} Y tan terrible cosa era lo que se veía, que Moisés dijo:
Estoy asombrado y temblando.

\bibverse{22} Mas os habéis llegado al monte de Sión, y á la ciudad del
Dios vivo, Jerusalem la celestial, y á la compañía de muchos millares de
ángeles, \footnote{\textbf{12:22} Gal 4,26; Efes 2,6; Fil 3,20; Apoc
  5,11; Apoc 21,2} \bibverse{23} Y á la congregación de los primogénitos
que están alistados en los cielos, y á Dios el Juez de todos, y á los
espíritus de los justos hechos perfectos, \footnote{\textbf{12:23} Luc
  10,20} \bibverse{24} Y á Jesús el Mediador del nuevo testamento, y á
la sangre del esparcimiento que habla mejor que la de Abel. \footnote{\textbf{12:24}
  Heb 9,15; Gén 4,10}

\hypertarget{la-gloria-del-fin-de-los-tiempos-aterradora-para-los-reacios-y-dichosa-para-los-obedientes}{%
\subsection{La gloria del fin de los tiempos, aterradora para los
reacios y dichosa para los
obedientes}\label{la-gloria-del-fin-de-los-tiempos-aterradora-para-los-reacios-y-dichosa-para-los-obedientes}}

\bibverse{25} Mirad que no desechéis al que habla. Porque si aquellos no
escaparon que desecharon al que hablaba en la tierra, mucho menos
nosotros, si desecháramos al que habla de los cielos. \footnote{\textbf{12:25}
  Heb 2,2; Heb 10,28-29} \bibverse{26} La voz del cual entonces conmovió
la tierra; mas ahora ha denunciado, diciendo: Aun una vez, y yo
conmoveré no solamente la tierra, mas aun el cielo. \bibverse{27} Y esta
palabra, Aun una vez, declara la mudanza de las cosas movibles, como de
cosas hechas, para que queden las cosas que son firmes. \bibverse{28}
Así que, tomando el reino inmóvil, retengamos la gracia por la cual
sirvamos á Dios agradándole con temor y reverencia; \bibverse{29} Porque
nuestro Dios es fuego consumidor. \footnote{\textbf{12:29} Heb 10,31;
  Deut 4,24}

\hypertarget{advertencias-individuales-por-el-amor-fraterno-la-pureza-moral-y-la-promociuxf3n-de-la-vida-comunitaria}{%
\subsection{Advertencias individuales por el amor fraterno, la pureza
moral y la promoción de la vida
comunitaria}\label{advertencias-individuales-por-el-amor-fraterno-la-pureza-moral-y-la-promociuxf3n-de-la-vida-comunitaria}}

\hypertarget{section-12}{%
\section{13}\label{section-12}}

\bibverse{1} Permanezca el amor fraternal. \footnote{\textbf{13:1} Juan
  13,34; 2Pe 1,7} \bibverse{2} No olvidéis la hospitalidad, porque por
ésta algunos, sin saberlo, hospedaron ángeles. \footnote{\textbf{13:2}
  Gén 18,3; Gén 19,2-3; Rom 12,13; 1Pe 4,9; 3Jn 1,5-8} \bibverse{3}
Acordaos de los presos, como presos juntamente con ellos; y de los
afligidos, como que también vosotros mismos sois del cuerpo. \footnote{\textbf{13:3}
  Mat 25,36} \bibverse{4} Honroso es en todos el matrimonio, y el lecho
sin mancilla; mas á los fornicarios y á los adúlteros juzgará Dios.

\bibverse{5} Sean las costumbres vuestras sin avaricia; contentos de lo
presente; porque él dijo: No te desampararé, ni te dejaré. \footnote{\textbf{13:5}
  1Tim 6,6} \bibverse{6} De tal manera que digamos confiadamente: El
Señor es mi ayudador; no temeré lo que me hará el hombre.

\hypertarget{amonestaciuxf3n-principal-de-ser-fieles-a-los-gobernantes-y-a-jesuxfas-el-que-permanece-en-la-eternidad-y-el-fin-del-servicio-del-sacrificio-por-el-pecado-juduxedo}{%
\subsection{Amonestación principal de ser fieles a los gobernantes y a
Jesús, el que permanece en la eternidad y el fin del servicio del
sacrificio por el pecado
judío}\label{amonestaciuxf3n-principal-de-ser-fieles-a-los-gobernantes-y-a-jesuxfas-el-que-permanece-en-la-eternidad-y-el-fin-del-servicio-del-sacrificio-por-el-pecado-juduxedo}}

\bibverse{7} Acordaos de vuestros pastores, que os hablaron la palabra
de Dios; la fe de los cuales imitad, considerando cuál haya sido el
éxito de su conducta. \bibverse{8} Jesucristo es el mismo ayer, y hoy, y
por los siglos. \bibverse{9} No seáis llevados de acá para allá por
doctrinas diversas y extrañas; porque buena cosa es afirmar el corazón
en la gracia, no en viandas, que nunca aprovecharon á los que anduvieron
en ellas. \footnote{\textbf{13:9} 2Cor 1,21; 1Tim 4,8; Rom 14,17; Efes
  4,14}

\bibverse{10} Tenemos un altar, del cual no tienen facultad de comer los
que sirven al tabernáculo. \bibverse{11} Porque los cuerpos de aquellos
animales, la sangre de los cuales es metida por el pecado en el
santuario por el pontífice, son quemados fuera del real. \bibverse{12}
Por lo cual también Jesús, para santificar al pueblo por su propia
sangre, padeció fuera de la puerta. \footnote{\textbf{13:12} Juan 19,17;
  Mat 21,39} \bibverse{13} Salgamos pues á él fuera del real, llevando
su vituperio. \footnote{\textbf{13:13} Heb 11,26; Heb 12,2}
\bibverse{14} Porque no tenemos aquí ciudad permanente, mas buscamos la
por venir. \footnote{\textbf{13:14} Heb 11,10; Heb 12,22} \bibverse{15}
Así que, ofrezcamos por medio de él á Dios siempre sacrificio de
alabanza, es á saber, fruto de labios que confiesen á su nombre.
\footnote{\textbf{13:15} Os 14,3; Sal 50,14; Sal 50,23}

\hypertarget{advertencias-individuales-repetidas-especialmente-con-respecto-al-comportamiento-contra-los-luxedderes-comunitarios}{%
\subsection{Advertencias individuales repetidas, especialmente con
respecto al comportamiento contra los líderes
comunitarios}\label{advertencias-individuales-repetidas-especialmente-con-respecto-al-comportamiento-contra-los-luxedderes-comunitarios}}

\bibverse{16} Y de hacer bien y de la comunicación no os olvidéis:
porque de tales sacrificios se agrada Dios.

\bibverse{17} Obedeced á vuestros pastores, y sujetaos á ellos; porque
ellos velan por vuestras almas, como aquellos que han de dar cuenta;
para que lo hagan con alegría, y no gimiendo; porque esto no os es útil.
\footnote{\textbf{13:17} 1Tes 5,12; Ezeq 3,17-19}

\bibverse{18} Orad por nosotros: porque confiamos que tenemos buena
conciencia, deseando conversar bien en todo. \footnote{\textbf{13:18}
  Rom 15,30; 2Cor 1,11-12} \bibverse{19} Y más os ruego que lo hagáis
así, para que yo os sea más presto restituído.

\hypertarget{clausura-de-la-carta-bendiciuxf3n-mensajes-personales-saludos}{%
\subsection{Clausura de la carta, bendición, mensajes personales,
saludos}\label{clausura-de-la-carta-bendiciuxf3n-mensajes-personales-saludos}}

\bibverse{20} Y el Dios de paz que sacó de los muertos á nuestro Señor
Jesucristo, el gran pastor de las ovejas, por la sangre del testamento
eterno, \footnote{\textbf{13:20} Juan 10,12; 1Pe 2,25} \bibverse{21} Os
haga aptos en toda obra buena para que hagáis su voluntad, haciendo él
en vosotros lo que es agradable delante de él por Jesucristo: al cual
sea gloria por los siglos de los siglos. Amén.

\bibverse{22} Empero os ruego, hermanos, que soportéis la palabra de
exhortación; porque os he escrito en breve. \bibverse{23} Sabed que
nuestro hermano Timoteo está suelto; con el cual, si viniere más presto,
os iré á ver.

\bibverse{24} Saludad á todos vuestros pastores, y á todos los santos.
Los de Italia os saludan.

\bibverse{25} La gracia sea con todos vosotros. Amén. Fué escrita á los
Hebreos desde Italia con Timoteo.
