\hypertarget{nehemuxedas-como-copero-del-rey-artajerjes-en-susa-su-dolor-por-la-desgracia-de-su-pauxeds}{%
\subsection{Nehemías como copero del rey Artajerjes en Susa; su dolor
por la desgracia de su
país}\label{nehemuxedas-como-copero-del-rey-artajerjes-en-susa-su-dolor-por-la-desgracia-de-su-pauxeds}}

\hypertarget{section}{%
\section{1}\label{section}}

\bibverse{1} Palabras de Nehemías, hijo de Hachâlías. Y acaeció en el
mes de Chisleu, en el año veinte, estando yo en Susán, capital del
reino,

\bibverse{2} Que vino Hanani, uno de mis hermanos, él y ciertos varones
de Judá, y preguntéles por los Judíos que habían escapado, que habían
quedado de la cautividad, y por Jerusalem. \bibverse{3} Y dijéronme: El
residuo, los que quedaron de la cautividad allí en la provincia, están
en gran mal y afrenta, y el muro de Jerusalem derribado, y sus puertas
quemadas á fuego.

\bibverse{4} Y fué que, como yo oí estas palabras, sentéme y lloré, y
enlutéme por algunos días, y ayuné y oré delante del Dios de los cielos.
\footnote{\textbf{1:4} Neh 9,1; Esd 9,3}

\hypertarget{la-penitencia-y-la-suxfaplica-de-nehemuxedas}{%
\subsection{La penitencia y la súplica de
Nehemías}\label{la-penitencia-y-la-suxfaplica-de-nehemuxedas}}

\bibverse{5} Y dije: Ruégote, oh Jehová, Dios de los cielos, fuerte,
grande, y terrible, que guarda el pacto y la misericordia á los que le
aman y guardan sus mandamientos; \footnote{\textbf{1:5} Neh 4,8; Dan 9,4}
\bibverse{6} Esté ahora atento tu oído, y tus ojos abiertos, para oir la
oración de tu siervo, que yo hago ahora delante de ti día y noche, por
los hijos de Israel tus siervos; y confieso los pecados de los hijos de
Israel que hemos contra ti cometido; sí, yo y la casa de mi padre hemos
pecado. \bibverse{7} En extremo nos hemos corrompido contra ti, y no
hemos guardado los mandamientos, y estatutos y juicios, que mandaste á
Moisés tu siervo.

\bibverse{8} Acuérdate ahora de la palabra que ordenaste á Moisés tu
siervo, diciendo: Vosotros prevaricaréis, y yo os esparciré por los
pueblos: \bibverse{9} Mas os volveréis á mí, y guardaréis mis
mandamientos, y los pondréis por obra. Si fuere vuestro lanzamiento
hasta el cabo de los cielos, de allí os juntaré; y traerlos he al lugar
que escogí para hacer habitar allí mi nombre. \footnote{\textbf{1:9}
  Deut 30,4}

\bibverse{10} Ellos pues son tus siervos y tu pueblo, los cuales
redimiste con tu gran fortaleza, y con tu mano fuerte. \bibverse{11}
Ruégote, oh Jehová, esté ahora atento tu oído á la oración de tu siervo,
y á la oración de tus siervos, quienes desean temer tu nombre: y ahora
concede hoy próspero suceso á tu siervo, y dale gracia delante de aquel
varón. Porque yo servía de copero al rey.

\hypertarget{nehemuxedas-recibe-permiso-y-autoridad-del-rey-persa-artajerjes-para-restaurar-jerusaluxe9n}{%
\subsection{Nehemías recibe permiso y autoridad del rey persa Artajerjes
para restaurar
Jerusalén}\label{nehemuxedas-recibe-permiso-y-autoridad-del-rey-persa-artajerjes-para-restaurar-jerusaluxe9n}}

\hypertarget{section-1}{%
\section{2}\label{section-1}}

\bibverse{1} Y fué en el mes de Nisán, en el año veinte del rey
Artajerjes, que estando ya el vino delante de él, tomé el vino, y dílo
al rey. Y como yo no había estado antes triste en su presencia,
\bibverse{2} Díjome el rey: ¿Por qué está triste tu rostro, pues no
estás enfermo? No es esto sino quebranto de corazón. Entonces temí en
gran manera.

\bibverse{3} Y dije al rey: El rey viva para siempre. ¿Cómo no estará
triste mi rostro, cuando la ciudad, casa de los sepulcros de mis padres,
está desierta, y sus puertas consumidas del fuego?

\bibverse{4} Y díjome el rey: ¿Qué cosa pides? Entonces oré al Dios de
los cielos,

\bibverse{5} Y dije al rey: Si al rey place, y si agrada tu siervo
delante de ti, que me envíes á Judá, á la ciudad de los sepulcros de mis
padres, y la reedificaré.

\bibverse{6} Entonces el rey me dijo, (y la reina estaba sentada junto á
él): ¿Hasta cuándo será tu viaje, y cuándo volverás? Y plugo al rey
enviarme, después que yo le señalé tiempo.

\bibverse{7} Además dije al rey: Si al rey place, dénseme cartas para
los gobernadores de la otra parte del río, que me franqueen el paso
hasta que llegue á Judá; \bibverse{8} Y carta para Asaph, guarda del
bosque del rey, á fin que me dé madera para enmaderar los portales del
palacio de la casa, y para el muro de la ciudad, y la casa donde
entraré. Y otorgómelo el rey, según la benéfica mano de Jehová sobre mí.

\bibverse{9} Y vine luego á los gobernadores de la otra parte del río, y
les dí las cartas del rey. Y el rey envió conmigo capitanes del ejército
y gente de á caballo. \bibverse{10} Y oyéndolo Sanballat Horonita, y
Tobías, el siervo Ammonita, disgustóles en extremo que viniese alguno
para procurar el bien de los hijos de Israel.

\hypertarget{el-recorrido-nocturno-de-nehemuxedas-por-las-murallas-de-la-ciudad-su-llamado-a-los-camaradas-nacionales-para-restaurar-el-muro}{%
\subsection{El recorrido nocturno de Nehemías por las murallas de la
ciudad; su llamado a los camaradas nacionales para restaurar el
muro}\label{el-recorrido-nocturno-de-nehemuxedas-por-las-murallas-de-la-ciudad-su-llamado-a-los-camaradas-nacionales-para-restaurar-el-muro}}

\bibverse{11} Llegué pues á Jerusalem, y estado que hube allí tres días,
\bibverse{12} Levantéme de noche, yo y unos pocos varones conmigo, y no
declaré á hombre alguno lo que Dios había puesto en mi corazón que
hiciese en Jerusalem; ni había bestia conmigo, excepto la cabalgadura en
que cabalgaba. \bibverse{13} Y salí de noche por la puerta del Valle
hacia la fuente del Dragón y á la puerta del Muladar; y consideré los
muros de Jerusalem que estaban derribados, y sus puertas que estaban
consumidas del fuego. \bibverse{14} Pasé luego á la puerta de la Fuente,
y al estanque del Rey; mas no había lugar por donde pasase la
cabalgadura en que iba. \bibverse{15} Y subí por el torrente de noche, y
consideré el muro, y regresando entré por la puerta del Valle, y
volvíme. \bibverse{16} Y no sabían los magistrados dónde yo había ido,
ni qué había hecho; ni hasta entonces lo había yo declarado á los Judíos
y sacerdotes, ni á los nobles y magistrados, ni á los demás que hacían
la obra.

\bibverse{17} Díjeles pues: Vosotros veis el mal en que estamos, que
Jerusalem está desierta, y sus puertas consumidas del fuego: venid, y
edifiquemos el muro de Jerusalem, y no seamos más en oprobio.

\hypertarget{compromiso-del-jefe-de-comunidad-el-riduxedculo-de-los-tres-oponentes-paganos-rechazados-por-nehemuxedas}{%
\subsection{Compromiso del jefe de comunidad; el ridículo de los tres
oponentes paganos rechazados por
Nehemías}\label{compromiso-del-jefe-de-comunidad-el-riduxedculo-de-los-tres-oponentes-paganos-rechazados-por-nehemuxedas}}

\bibverse{18} Entonces les declaré cómo la mano de mi Dios era buena
sobre mí, y asimismo las palabras del rey, que me había dicho. Y
dijeron: Levantémonos, y edifiquemos. Así esforzaron sus manos para
bien.

\bibverse{19} Mas habiéndolo oído Samballat Horonita, y Tobías el siervo
Ammonita, y Gesem el Arabe, escarnecieron de nosotros, y nos
despreciaron, diciendo: ¿Qué es esto que hacéis vosotros? ¿os rebeláis
contra el rey?

\bibverse{20} Y volvíles respuesta, y díjeles: El Dios de los cielos, él
nos prosperará, y nosotros sus siervos nos levantaremos y edificaremos:
que vosotros no tenéis parte, ni derecho, ni memoria en Jerusalem.
\footnote{\textbf{2:20} Efes 2,12}

\hypertarget{construcciuxf3n-pieza-a-pieza-del-muro-lista-de-los-involucrados-en-la-construcciuxf3n-del-muro}{%
\subsection{Construcción pieza a pieza del muro; Lista de los
involucrados en la construcción del
muro}\label{construcciuxf3n-pieza-a-pieza-del-muro-lista-de-los-involucrados-en-la-construcciuxf3n-del-muro}}

\hypertarget{section-2}{%
\section{3}\label{section-2}}

\bibverse{1} Y levantóse Eliasib el gran sacerdote con sus hermanos los
sacerdotes, y edificaron la puerta de las Ovejas. Ellos aparejaron y
levantaron sus puertas hasta la torre de Meah, aparejáronla hasta la
torre de Hananeel. \bibverse{2} Y junto á ella edificaron los varones de
Jericó: y luego edificó Zachûr hijo de Imri.

\bibverse{3} Y los hijos de Senaa edificaron la puerta del Pescado:
ellos la enmaderaron, y levantaron sus puertas, con sus cerraduras y sus
cerrojos. \bibverse{4} Y junto á ellos restauró Meremoth hijo de Urías,
hijo de Cos: y al lado de ellos, restauró Mesullam hijo de Berechîas,
hijo de Mesezabeel. Junto á ellos restauró Sadoc hijo de Baana.
\bibverse{5} E inmediato á ellos restauraron los Tecoitas; mas sus
grandes no prestaron su cerviz á la obra de su Señor.

\bibverse{6} Y la puerta Vieja restauraron Joiada hijo de Pasea, y
Mesullam hijo de Besodías: ellos la enmaderaron, y levantaron sus
puertas, con sus cerraduras y sus cerrojos. \bibverse{7} Junto á ellos
restauró Melatías Gabaonita, y Jadón Meronothita, varones de Gabaón y de
Mizpa, por la silla del gobernador de la otra parte del río.
\bibverse{8} Y junto á ellos restauró Uzziel hijo de Harhaía, de los
plateros; junto al cual restauró también Hananías, hijo de un perfumero.
Así dejaron reparado á Jerusalem hasta el muro ancho. \bibverse{9} Junto
á ellos restauró también Repaías hijo de Hur, príncipe de la mitad de la
región de Jerusalem. \bibverse{10} Asimismo restauró junto á ellos, y
frente á su casa, Jedaías hijo de Harumaph; y junto á él restauró Hattus
hijo de Hasbanías. \bibverse{11} Malchîas hijo de Harim y Hasub hijo de
Pahath-moab, restauraron la otra medida, y la torre de los Hornos.
\bibverse{12} Junto á ellos restauró Sallum hijo de Lohes, príncipe de
la mitad de la región de Jerusalem, él con sus hijas.

\bibverse{13} La puerta del Valle la restauró Hanún con los moradores de
Zanoa: ellos la reedificaron, y levantaron sus puertas, con sus
cerraduras y sus cerrojos, y mil codos en el muro hasta la puerta del
Muladar.

\bibverse{14} Y reedificó la puerta del Muladar, Malchîas hijo de
Rechâb, príncipe de la provincia de Beth-haccerem: él la reedificó, y
levantó sus puertas, sus cerraduras y sus cerrojos.

\bibverse{15} Y Sallum hijo de Chôl-hoce, príncipe de la región de
Mizpa, restauró la puerta de la Fuente: él la reedificó, y la enmaderó,
y levantó sus puertas, sus cerraduras y sus cerrojos, y el muro del
estanque de Selah hacia la huerta del rey, y hasta las gradas que
descienden de la ciudad de David. \bibverse{16} Después de él restauró
Nehemías hijo de Azbuc, príncipe de la mitad de la región de Beth-sur,
hasta delante de los sepulcros de David, y hasta el estanque labrado, y
hasta la casa de los Valientes. \bibverse{17} Tras él restauraron los
Levitas, Rehum hijo de Bani; junto á él restauró Asabías, príncipe de la
mitad de la región de Ceila en su región. \bibverse{18} Después de él
restauraron sus hermanos, Bavvai hijo de Henadad, príncipe de la mitad
de la región de Ceila. \bibverse{19} Y junto á él restauró Ezer hijo de
Jesuá, príncipe de Mizpa, la otra medida frente á la subida de la
armería de la esquina. \bibverse{20} Después de él se enfervorizó á
restaurar Baruch hijo de Zachâi la otra medida, desde la esquina hasta
la puerta de la casa de Eliasib gran sacerdote. \footnote{\textbf{3:20}
  Neh 3,1} \bibverse{21} Tras él restauró Meremoth hijo de Urías hijo de
Cos la otra medida, desde la entrada de la casa de Eliasib, hasta el
cabo de la casa de Eliasib. \footnote{\textbf{3:21} Esd 8,33}
\bibverse{22} Después de él restauraron los sacerdotes, los varones de
la campiña. \bibverse{23} Después de ellos restauraron Benjamín y Hasub,
frente á su casa: y después de estos restauró Azarías, hijo de Maasías
hijo de Ananías, cerca de su casa. \bibverse{24} Después de él restauró
Binnui hijo de Henadad la otra medida, desde la casa de Azarías hasta la
revuelta, y hasta la esquina. \bibverse{25} Paal hijo de Uzai, enfrente
de la esquina y la torre alta que sale de la casa del rey, que está en
el patio de la cárcel. Después de él, Pedaía hijo de Pharos. \footnote{\textbf{3:25}
  Jer 32,2; Jer 33,1} \bibverse{26} (Y los Nethineos estuvieron en Ophel
hasta enfrente de la puerta de las Aguas al oriente, y la torre que
sobresalía.) \bibverse{27} Después de él restauraron los Tecoitas la
otra medida, enfrente de la grande torre que sobresale, hasta el muro de
Ophel.

\bibverse{28} Desde la puerta de los Caballos restauraron los
sacerdotes, cada uno enfrente de su casa. \bibverse{29} Después de ellos
restauró Sadoc hijo de Immer, enfrente de su casa: y después de él
restauró Semaías hijo de Sechânías, guarda de la puerta oriental.
\bibverse{30} Tras él restauró Hananías hijo de Selemías, y Anún hijo
sexto de Salaph, la otra medida. Después de él restauró Mesullam, hijo
de Berechîas, enfrente de su cámara. \bibverse{31} Después de él
restauró Malchîas hijo del platero, hasta la casa de los Nethineos y de
los tratantes, enfrente de la puerta del Juicio, y hasta la sala de la
esquina. \bibverse{32} Y entre la sala de la esquina hasta la puerta de
las Ovejas, restauraron los plateros y los tratantes.

\hypertarget{continuaciuxf3n-de-la-construcciuxf3n-del-muro-a-pesar-del-riduxedculo-y-la-hostilidad-de-los-oponentes-paganos}{%
\subsection{Continuación de la construcción del muro a pesar del
ridículo y la hostilidad de los oponentes
paganos}\label{continuaciuxf3n-de-la-construcciuxf3n-del-muro-a-pesar-del-riduxedculo-y-la-hostilidad-de-los-oponentes-paganos}}

\hypertarget{section-3}{%
\section{4}\label{section-3}}

\bibverse{1} Y fué que como oyó Sanballat que nosotros edificábamos el
muro, encolerizóse y enojóse en gran manera, é hizo escarnio de los
Judíos. \bibverse{2} Y habló delante de sus hermanos y del ejército de
Samaria, y dijo: ¿Qué hacen estos débiles Judíos? ¿hanles de permitir?
¿han de sacrificar? ¿han de acabar en un día? ¿han de resucitar de los
montones del polvo las piedras que fueron quemadas?

\bibverse{3} Y estaba junto á él Tobías Ammonita, el cual dijo: Aun lo
que ellos edifican, si subiere una zorra derribará su muro de piedra.

\bibverse{4} Oye, oh Dios nuestro, que somos en menosprecio, y vuelve el
baldón de ellos sobre su cabeza, y dalos en presa en la tierra de su
cautiverio: \bibverse{5} Y no cubras su iniquidad, ni su pecado sea
raído delante de tu rostro; porque se airaron contra los que edificaban.

\bibverse{6} Edificamos pues el muro, y toda la muralla fué junta hasta
su mitad: y el pueblo tuvo ánimo para obrar.

\hypertarget{nuevos-ataques-de-los-oponentes-al-edificio-las-medidas-exitosas-de-nehemia-en-su-contra}{%
\subsection{Nuevos ataques de los oponentes al edificio; Las medidas
exitosas de Nehemia en su
contra}\label{nuevos-ataques-de-los-oponentes-al-edificio-las-medidas-exitosas-de-nehemia-en-su-contra}}

\bibverse{7} Mas acaeció que oyendo Sanballat y Tobías, y los Arabes, y
los Ammonitas, y los de Asdod, que los muros de Jerusalem eran
reparados, porque ya los portillos comenzaban á cerrarse,
encolerizáronse mucho; \bibverse{8} Y conspiraron todos á una para venir
á combatir á Jerusalem, y á hacerle daño. \bibverse{9} Entonces oramos á
nuestro Dios, y por causa de ellos pusimos guarda contra ellos de día y
de noche. \footnote{\textbf{4:9} Job 5,12}

\bibverse{10} Y dijo Judá: Las fuerzas de los acarreadores se han
enflaquecido, y el escombro es mucho, y no podemos edificar el muro.
\bibverse{11} Y nuestros enemigos dijeron: No sepan, ni vean, hasta que
entremos en medio de ellos, y los matemos, y hagamos cesar la obra.

\bibverse{12} Sucedió empero, que como vinieron los Judíos que habitaban
entre ellos, nos dieron aviso diez veces de todos los lugares de donde
volvían á nosotros.

\bibverse{13} Entonces puse por los bajos del lugar, detrás del muro, en
las alturas de los peñascos, puse el pueblo por familias con sus
espadas, con sus lanzas, y con sus arcos. \bibverse{14} Después miré, y
levantéme, y dije á los principales y á los magistrados, y al resto del
pueblo: No temáis delante de ellos: acordaos del Señor grande y
terrible, y pelead por vuestros hermanos, por vuestros hijos y por
vuestras hijas, por vuestras mujeres y por vuestras casas.

\bibverse{15} Y sucedió que como oyeron nuestros enemigos que lo
habíamos entendido, Dios disipó el consejo de ellos, y volvímonos todos
al muro, cada uno á su obra. \bibverse{16} Mas fué que desde aquel día
la mitad de los mancebos trabajaba en la obra, y la otra mitad de ellos
tenía lanzas y escudos, y arcos, y corazas; y los príncipes estaban tras
toda la casa de Judá. \bibverse{17} Los que edificaban en el muro, y los
que llevaban cargas y los que cargaban, con la una mano trabajaban en la
obra, y en la otra tenían la espada. \bibverse{18} Porque los que
edificaban, cada uno tenía su espada ceñida á sus lomos, y así
edificaban: y el que tocaba la trompeta estaba junto á mí. \bibverse{19}
Y dije á los principales, y á los magistrados y al resto del pueblo: La
obra es grande y larga, y nosotros estamos apartados en el muro, lejos
los unos de los otros: \bibverse{20} En el lugar donde oyereis la voz de
la trompeta, reuníos allí á nosotros: nuestro Dios peleará por nosotros.

\bibverse{21} Nosotros pues trabajábamos en la obra; y la mitad de ellos
tenían lanzas desde la subida del alba hasta salir las estrellas.
\bibverse{22} También dije entonces al pueblo: Cada uno con su criado se
quede dentro de Jerusalem, y hágannos de noche centinela, y de día á la
obra. \bibverse{23} Y ni yo, ni mis hermanos, ni mis mozos, ni la gente
de guardia que me seguía, desnudamos nuestro vestido: cada uno se
desnudaba solamente para lavarse.

\hypertarget{alivio-de-la-difuxedcil-situaciuxf3n-de-la-gente-comuxfan-mediante-el-alivio-de-la-deuda-el-gobierno-desinteresado-de-nehemuxedas}{%
\subsection{Alivio de la difícil situación de la gente común mediante el
alivio de la deuda; El gobierno desinteresado de
Nehemías}\label{alivio-de-la-difuxedcil-situaciuxf3n-de-la-gente-comuxfan-mediante-el-alivio-de-la-deuda-el-gobierno-desinteresado-de-nehemuxedas}}

\hypertarget{section-4}{%
\section{5}\label{section-4}}

\bibverse{1} Entonces fué grande el clamor del pueblo y de sus mujeres
contra los Judíos sus hermanos. \bibverse{2} Y había quien decía:
Nosotros, nuestros hijos y nuestras hijas, somos muchos: hemos por tanto
tomado grano para comer y vivir. \bibverse{3} Y había quienes decían:
Hemos empeñado nuestras tierras, y nuestras viñas, y nuestras casas,
para comprar grano en el hambre. \bibverse{4} Y había quienes decían:
Hemos tomado prestado dinero para el tributo del rey, sobre nuestras
tierras y viñas. \bibverse{5} Ahora bien, nuestra carne es como la carne
de nuestros hermanos, nuestros hijos como sus hijos: y he aquí que
nosotros sujetamos nuestros hijos y nuestras hijas á servidumbre, y hay
algunas de nuestras hijas sujetas: mas no hay facultad en nuestras manos
para rescatarlas, porque nuestras tierras y nuestras viñas son de otros.

\hypertarget{eliminaciuxf3n-de-los-males-mediante-las-resoluciones-de-la-asamblea-popular}{%
\subsection{Eliminación de los males mediante las resoluciones de la
asamblea
popular}\label{eliminaciuxf3n-de-los-males-mediante-las-resoluciones-de-la-asamblea-popular}}

\bibverse{6} Y enojéme en gran manera cuando oí su clamor y estas
palabras. \bibverse{7} Meditélo entonces para conmigo, y reprendí á los
principales y á los magistrados, y díjeles: ¿Tomáis cada uno usura de
vuestros hermanos? Y convoqué contra ellos una grande junta.
\bibverse{8} Y díjeles: Nosotros rescatamos á nuestros hermanos Judíos
que habían sido vendidos á las gentes, conforme á la facultad que había
en nosotros: ¿y vosotros aun vendéis á vuestros hermanos, y serán
vendidos á nosotros? Y callaron, que no tuvieron qué responder.
\bibverse{9} Y dije: No es bien lo que hacéis, ¿no andaréis en temor de
nuestro Dios, por no ser el oprobio de las gentes enemigas nuestras?
\bibverse{10} También yo, y mis hermanos, y mis criados, les hemos
prestado dinero y grano: relevémosles ahora de este gravamen.
\bibverse{11} Ruégoos que les devolváis hoy sus tierras, sus viñas, sus
olivares, y sus casas, y la centésima parte del dinero y grano, del vino
y del aceite que demandáis de ellos.

\bibverse{12} Y dijeron: Devolveremos, y nada les demandaremos; haremos
así como tú dices. Entonces convoqué los sacerdotes, y juramentélos que
harían conforme á esto.

\bibverse{13} Además sacudí mi vestido, y dije: Así sacuda Dios de su
casa y de su trabajo á todo hombre que no cumpliere esto, y así sea
sacudido y vacío. Y respondió toda la congregación: ¡Amén! Y alabaron á
Jehová. Y el pueblo hizo conforme á esto.

\hypertarget{el-altruismo-de-nehemuxedas-mientras-estaba-en-el-cargo}{%
\subsection{El altruismo de Nehemías mientras estaba en el
cargo}\label{el-altruismo-de-nehemuxedas-mientras-estaba-en-el-cargo}}

\bibverse{14} También desde el día que me mandó el rey que fuese
gobernador de ellos en la tierra de Judá, desde el año veinte del rey
Artajerjes hasta el año treinta y dos, doce años, ni yo ni mis hermanos
comimos el pan del gobernador. \bibverse{15} Mas los primeros
gobernadores que fueron antes de mí, cargaron al pueblo, y tomaron de
ellos por el pan y por el vino sobre cuarenta siclos de plata: á más de
esto, sus criados se enseñoreaban sobre el pueblo; pero yo no hice así,
á causa del temor de Dios. \bibverse{16} También en la obra de este muro
instauré mi parte, y no compramos heredad: y todos mis criados juntos
estaban allí á la obra. \bibverse{17} Además ciento y cincuenta hombres
de los Judíos y magistrados, y los que venían á nosotros de las gentes
que están en nuestros contornos, estaban á mi mesa. \bibverse{18} Y lo
que se aderezaba para cada día era un buey, seis ovejas escogidas, y
aves también se aparejaban para mí, y cada diez días vino en toda
abundancia: y con todo esto nunca requerí el pan del gobernador, porque
la servidumbre de este pueblo era grave. \bibverse{19} Acuérdate de mí
para bien, Dios mío, y de todo lo que hice á este pueblo. \footnote{\textbf{5:19}
  Neh 13,14; Neh 13,22; Neh 13,31}

\hypertarget{esquemas-y-asesinatos-de-sanbalat-y-sus-camaradas-su-rechazo-por-parte-de-nehemuxedas}{%
\subsection{Esquemas (y asesinatos) de Sanbalat y sus camaradas; su
rechazo por parte de
Nehemías}\label{esquemas-y-asesinatos-de-sanbalat-y-sus-camaradas-su-rechazo-por-parte-de-nehemuxedas}}

\hypertarget{section-5}{%
\section{6}\label{section-5}}

\bibverse{1} Y fué que habiendo oído Sanballat, y Tobías, y Gesem el
Arabe, y los demás nuestros enemigos, que había yo edificado el muro, y
que no quedaba en él portillo, (aunque hasta aquel tiempo no había
puesto en las puertas las hojas,) \bibverse{2} Sanballat y Gesem
enviaron á decirme: Ven, y compongámonos juntos en alguna de las aldeas
en el campo de Ono. Mas ellos habían pensado hacerme mal.

\bibverse{3} Y enviéles mensajeros, diciendo: Yo hago una grande obra, y
no puedo ir; porque cesaría la obra, dejándola yo para ir á vosotros.

\bibverse{4} Y enviaron á mí con el mismo asunto por cuatro veces, y yo
les respondí de la misma manera. \bibverse{5} Envió entonces Sanballat á
mí su criado, á decir lo mismo por quinta vez, con una carta abierta en
su mano, \bibverse{6} En la cual estaba escrito: Hase oído entre las
gentes, y Gasmu lo dice, que tú y los Judíos pensáis rebelaros; y que
por eso edificas tú el muro, con la mira, según estas palabras, de ser
tú su rey; \bibverse{7} Y que has puesto profetas que prediquen de ti en
Jerusalem, diciendo: ¡Rey en Judá! Y ahora serán oídas del rey las tales
palabras: ven por tanto, y consultemos juntos.

\bibverse{8} Entonces envié yo á decirle: No hay tal cosa como dices,
sino que de tu corazón tú lo inventas. \bibverse{9} Porque todos ellos
nos ponían miedo, diciendo: Debilitaránse las manos de ellos en la obra,
y no será hecha. Esfuerza pues mis manos, oh Dios.

\hypertarget{exponiendo-a-un-falso-profeta}{%
\subsection{Exponiendo a un falso
profeta}\label{exponiendo-a-un-falso-profeta}}

\bibverse{10} Vine luego en secreto á casa de Semaías hijo de Delaías,
hijo de Mehetabeel, porque él estaba encerrado; el cual me dijo:
Juntémonos en la casa de Dios dentro del templo, y cerremos las puertas
del templo, porque vienen para matarte; sí, esta noche vendrán á
matarte.

\bibverse{11} Entonces dije: ¿Un hombre como yo ha de huir? ¿y quién,
que como yo fuera, entraría al templo para salvar la vida? No entraré.
\bibverse{12} Y entendí que Dios no lo había enviado, sino que hablaba
aquella profecía contra mí, porque Tobías y Sanballat le habían
alquilado por salario. \bibverse{13} Porque sobornado fué para hacerme
temer así, y que pecase, y les sirviera de mal nombre con que fuera yo
infamado. \footnote{\textbf{6:13} Núm 18,7} \bibverse{14} Acuérdate,
Dios mío, de Tobías y de Sanballat, conforme á estas sus obras, y
también de Noadías profetisa, y de los otros profetas que hacían por
ponerme miedo. \footnote{\textbf{6:14} Neh 3,36-37}

\hypertarget{finalizaciuxf3n-de-la-construcciuxf3n-del-muro-correspondencia-sospechosa-entre-tobija-y-muchos-juduxedos-dedicados-a-uxe9l}{%
\subsection{Finalización de la construcción del muro; correspondencia
sospechosa entre Tobija y muchos judíos dedicados a
él}\label{finalizaciuxf3n-de-la-construcciuxf3n-del-muro-correspondencia-sospechosa-entre-tobija-y-muchos-juduxedos-dedicados-a-uxe9l}}

\bibverse{15} Acabóse pues el muro el veinticinco del mes de Elul, en
cincuenta y dos días. \bibverse{16} Y como lo oyeron todos nuestros
enemigos, temieron todas las gentes que estaban en nuestros alrededores,
y abatiéronse mucho sus ojos, y conocieron que por nuestro Dios había
sido hecha esta obra. \bibverse{17} Asimismo en aquellos días iban
muchas cartas de los principales de Judá á Tobías, y las de Tobías
venían á ellos. \bibverse{18} Porque muchos en Judá se habían conjurado
con él, porque era yerno de Sechânías hijo de Ara; y Johanán su hijo
había tomado la hija de Mesullam, hijo de Berechîas. \bibverse{19}
También contaban delante de mí sus buenas obras, y referíanle mis
palabras. Y enviaba Tobías cartas para atemorizarme.

\hypertarget{la-preocupaciuxf3n-de-nehemuxedas-por-la-seguridad-de-la-ciudad}{%
\subsection{La preocupación de Nehemías por la seguridad de la
ciudad}\label{la-preocupaciuxf3n-de-nehemuxedas-por-la-seguridad-de-la-ciudad}}

\hypertarget{section-6}{%
\section{7}\label{section-6}}

\bibverse{1} Y luego que el muro fué edificado, y asenté las puertas, y
fueron señalados porteros y cantores y Levitas, \bibverse{2} Mandé á mi
hermano Hanani, y á Hananías, príncipe del palacio de Jerusalem, (porque
era éste, como varón de verdad y temeroso de Dios, sobre muchos;)
\bibverse{3} Y díjeles: No se abran las puertas de Jerusalem hasta que
caliente el sol: y aun ellos presentes, cierren las puertas, y atrancad.
Y señalé guardas de los moradores de Jerusalem, cada cual en su guardia,
y cada uno delante de su casa.

\hypertarget{la-preocupaciuxf3n-de-nehemuxedas-por-aumentar-la-poblaciuxf3n-de-jerusaluxe9n-lista-de-los-israelitas-que-anteriormente-regresaron-del-cautiverio-con-zorobabel}{%
\subsection{La preocupación de Nehemías por aumentar la población de
Jerusalén; Lista de los israelitas que anteriormente regresaron del
cautiverio con
Zorobabel}\label{la-preocupaciuxf3n-de-nehemuxedas-por-aumentar-la-poblaciuxf3n-de-jerusaluxe9n-lista-de-los-israelitas-que-anteriormente-regresaron-del-cautiverio-con-zorobabel}}

\bibverse{4} Y la ciudad era espaciosa y grande, pero poco pueblo dentro
de ella, y no había casas reedificadas.

\bibverse{5} Y puso Dios en mi corazón que juntase los principales, y
los magistrados, y el pueblo, para que fuesen empadronados por el orden
de sus linajes: y hallé el libro de la genealogía de los que habían
subido antes, y encontré en él escrito:

\bibverse{6} Estos son los hijos de la provincia que subieron de la
cautividad, de la transmigración que hizo pasar Nabucodonosor rey de
Babilonia, y que volvieron á Jerusalem y á Judá cada uno á su ciudad;
\footnote{\textbf{7:6} Esd 2,-1} \bibverse{7} Los cuales vinieron con
Zorobabel, Jesuá, Nehemías, Azarías, Raamías, Nahamani, Mardochêo,
Bilsán, Misperet, Bigvai, Nehum, Baana. La cuenta de los varones del
pueblo de Israel:

\bibverse{8} Los hijos de Paros, dos mil ciento setenta y dos;
\bibverse{9} Los hijos de Sephatías, trescientos setenta y dos;
\bibverse{10} Los hijos de Ara, seiscientos cincuenta y dos;
\bibverse{11} Los hijos de Pahath-moab, de los hijos de Jesuá y de Joab,
dos mil ochocientos dieciocho; \bibverse{12} Los hijos de Elam, mil
doscientos cincuenta y cuatro; \bibverse{13} Los hijos de Zattu,
ochocientos cuarenta y cinco; \bibverse{14} Los hijos de Zachâi,
setecientos y sesenta; \bibverse{15} Los hijos de Binnui, seiscientos
cuarenta y ocho; \bibverse{16} Los hijos de Bebai, seiscientos
veintiocho; \bibverse{17} Los hijos de Azgad, dos mil seiscientos
veintidós; \bibverse{18} Los hijos de Adonicam, seiscientos sesenta y
siete; \bibverse{19} Los hijos de Bigvai, dos mil sesenta y siete;
\bibverse{20} Los hijos de Addin, seiscientos cincuenta y cinco;
\bibverse{21} Los hijos de Ater, de Ezechîas, noventa y ocho;
\bibverse{22} Los hijos de Hasum, trescientos veintiocho; \bibverse{23}
Los hijos de Besai, trescientos veinticuatro; \bibverse{24} Los hijos de
Hariph, ciento doce; \bibverse{25} Los hijos de Gabaón, noventa y cinco;
\bibverse{26} Los varones de Beth-lehem y de Netopha, ciento ochenta y
ocho; \bibverse{27} Los varones de Anathoth, ciento veintiocho;
\bibverse{28} Los varones de Beth-azmaveth, cuarenta y dos;
\bibverse{29} Los varones de Chîriath-jearim, Chephira y Beeroth,
setecientos cuarenta y tres; \bibverse{30} Los varones de Rama y de
Gebaa, seiscientos veintiuno; \bibverse{31} Los varones de Michmas,
ciento veintidós; \bibverse{32} Los varones de Beth-el y de Ai, ciento
veintitrés; \bibverse{33} Los varones de la otra Nebo, cincuenta y dos;
\bibverse{34} Los hijos de la otra Elam, mil doscientos cincuenta y
cuatro; \bibverse{35} Los hijos de Harim, trescientos y veinte;
\bibverse{36} Los hijos de Jericó, trescientos cuarenta y cinco;
\bibverse{37} Los hijos de Lod, de Hadid, y Ono, setecientos veintiuno;
\bibverse{38} Los hijos de Senaa, tres mil novecientos y treinta.
\bibverse{39} Sacerdotes: los hijos de Jedaías, de la casa de Jesuá,
novecientos setenta y tres; \bibverse{40} Los hijos de Immer, mil
cincuenta y dos; \bibverse{41} Los hijos de Pashur, mil doscientos
cuarenta y siete; \bibverse{42} Los hijos de Harim, mil diez y siete.
\bibverse{43} Levitas: los hijos de Jesuá, de Cadmiel, de los hijos de
Odevía, setenta y cuatro. \bibverse{44} Cantores: los hijos de Asaph,
ciento cuarenta y ocho. \bibverse{45} Porteros: los hijos de Sallum, los
hijos de Ater, los hijos de Talmón, los hijos de Accub, los hijos de
Hatita, los hijos de Sobai, ciento treinta y ocho.

\bibverse{46} Nethineos: los hijos de Siha, los hijos de Hasupha, los
hijos de Thabaoth, \bibverse{47} Los hijos de Chêros, los hijos de Siaa,
los hijos de Phadón, \bibverse{48} Los hijos de Lebana, los hijos de
Hagaba, los hijos de Salmai, \bibverse{49} Los hijos de Hanán, los hijos
de Giddel, los hijos de Gahar, \bibverse{50} Los hijos de Rehaía, los
hijos de Resín, los hijos de Necoda, \bibverse{51} Los hijos de Gazzam,
los hijos de Uzza, los hijos de Phasea, \bibverse{52} Los hijos de
Besai, los hijos de Meunim, los hijos de Nephisesim, \bibverse{53} Los
hijos de Bacbuc, los hijos de Hacupha, los hijos de Harhur,
\bibverse{54} Los hijos de Baslith, los hijos de Mehida, los hijos de
Harsa, \bibverse{55} Los hijos de Barcos, los hijos de Sísera, los hijos
de Tema, \bibverse{56} Los hijos de Nesía, los hijos de Hatipha.

\bibverse{57} Los hijos de los siervos de Salomón: los hijos de Sotai,
los hijos de Sophereth, los hijos de Perida, \bibverse{58} Los hijos de
Jahala, los hijos de Darcón, los hijos de Giddel, \bibverse{59} Los
hijos de Sephatías, los hijos de Hattil, los hijos de
Pochêreth-hassebaim, los hijos de Amón. \bibverse{60} Todos los
Nethineos, é hijos de los siervos de Salomón, trescientos noventa y dos.

\bibverse{61} Y estos son los que subieron de Telmelah, Tel-harsa,
Chêrub, Addón, é Immer, los cuales no pudieron mostrar la casa de sus
padres, ni su linaje, si eran de Israel: \bibverse{62} Los hijos de
Delaía, los hijos de Tobías, los hijos de Necoda, seiscientos cuarenta y
dos. \bibverse{63} Y de los sacerdotes: los hijos de Habaías, los hijos
de Cos, los hijos de Barzillai, el cual tomó mujer de las hijas de
Barzillai Galaadita, y se llamó del nombre de ellas.

\bibverse{64} Estos buscaron su registro de genealogías, y no se halló;
y fueron echados del sacerdocio. \bibverse{65} Y díjoles el Tirsatha que
no comiesen de las cosas más santas, hasta que hubiese sacerdote con
Urim y Thummim.

\bibverse{66} La congregación toda junta era de cuarenta y dos mil
trescientos y sesenta, \bibverse{67} Sin sus siervos y siervas, que eran
siete mil trescientos treinta y siete; y entre ellos había doscientos
cuarenta y cinco cantores y cantoras. \bibverse{68} Sus caballos,
setecientos treinta y seis; sus mulos, doscientos cuarenta y cinco;
\bibverse{69} Camellos, cuatrocientos treinta y cinco; asnos, seis mil
setecientos y veinte.

\bibverse{70} Y algunos de los príncipes de las familias dieron para la
obra. El Tirsatha dió para el tesoro mil dracmas de oro, cincuenta
tazones, y quinientas treinta vestiduras sacerdotales. \bibverse{71} Y
de los príncipes de las familias dieron para el tesoro de la obra,
veinte mil dracmas de oro, y dos mil y doscientas libras de plata.
\bibverse{72} Y lo que dió el resto del pueblo fué veinte mil dracmas de
oro, y dos mil libras de plata, y sesenta y siete vestiduras
sacerdotales.

\bibverse{73} Y habitaron los sacerdotes y los Levitas, y los porteros,
y los cantores, y los del pueblo, y los Nethineos, y todo Israel, en sus
ciudades. Y venido el mes séptimo, los hijos de Israel estaban en sus
ciudades.

\hypertarget{lectura-de-la-ley-por-esdras-y-celebraciuxf3n-de-la-fiesta-de-los-tabernuxe1culos}{%
\subsection{Lectura de la ley por Esdras y celebración de la Fiesta de
los
Tabernáculos}\label{lectura-de-la-ley-por-esdras-y-celebraciuxf3n-de-la-fiesta-de-los-tabernuxe1culos}}

\hypertarget{section-7}{%
\section{8}\label{section-7}}

\bibverse{1} Y juntóse todo el pueblo como un solo hombre en la plaza
que está delante de la puerta de las Aguas, y dijeron á Esdras el
escriba, que trajese el libro de la ley de Moisés, la cual mandó Jehová
á Israel. \footnote{\textbf{8:1} Esd 7,6} \bibverse{2} Y Esdras el
sacerdote, trajo la ley delante de la congregación, así de hombres como
de mujeres, y de todo entendido para escuchar, el primer día del mes
séptimo. \footnote{\textbf{8:2} Deut 31,10-13} \bibverse{3} Y leyó en el
libro delante de la plaza que está delante de la puerta de las Aguas,
desde el alba hasta el medio día, en presencia de hombres y mujeres y
entendidos; y los oídos de todo el pueblo estaban atentos al libro de la
ley. \bibverse{4} Y Esdras el escriba estaba sobre un púlpito de madera,
que habían hecho para ello; y junto á él estaban Mathithías, y Sema, y
Anías, y Urías, é Hilcías, y Maasías, á su mano derecha; y á su mano
izquierda, Pedaía, Misael, y Malchîas, y Hasum, y Hasbedana, Zachârías,
y Mesullam. \bibverse{5} Abrió pues Esdras el libro á ojos de todo el
pueblo, (porque estaba más alto que todo el pueblo); y como lo abrió,
todo el pueblo estuvo atento. \bibverse{6} Bendijo entonces Esdras á
Jehová, Dios grande. Y todo el pueblo respondió, ¡Amén! ¡Amén! alzando
sus manos; y humilláronse, y adoraron á Jehová inclinados á tierra.

\bibverse{7} Y Jesuá, y Bani, y Serebías, Jamín, Accub, Sabethai, Odías,
Maasías, Celita, Azarías, Jozabed, Hanán, Pelaía, Levitas, hacían
entender al pueblo la ley: y el pueblo estaba en su lugar. \bibverse{8}
Y leían en el libro de la ley de Dios claramente, y ponían el sentido,
de modo que entendiesen la lectura.

\hypertarget{la-invitaciuxf3n-de-nehemuxedas-y-esdras-a-las-personas-afligidas-a-celebrar-el-duxeda-con-alegruxeda-festiva}{%
\subsection{La invitación de Nehemías y Esdras a las personas afligidas
a celebrar el día con alegría
festiva}\label{la-invitaciuxf3n-de-nehemuxedas-y-esdras-a-las-personas-afligidas-a-celebrar-el-duxeda-con-alegruxeda-festiva}}

\bibverse{9} Y Nehemías el Tirsatha, y el sacerdote Esdras, escriba, y
los Levitas que hacían entender al pueblo, dijeron á todo el pueblo: Día
santo es á Jehová nuestro Dios; no os entristezcáis, ni lloréis: porque
todo el pueblo lloraba oyendo las palabras de la ley. \footnote{\textbf{8:9}
  Neh 5,14} \bibverse{10} Díjoles luego: Id, comed grosuras, y bebed
vino dulce, y enviad porciones á los que no tienen prevenido; porque día
santo es á nuestro Señor: y no os entristezcáis, porque el gozo de
Jehová es vuestra fortaleza.

\bibverse{11} Los Levitas pues, hacían callar á todo el pueblo,
diciendo: Callad, que es día santo, y no os entristezcáis.

\bibverse{12} Y todo el pueblo se fué á comer y á beber, y á enviar
porciones, y á gozar de grande alegría, porque habían entendido las
palabras que les habían enseñado.

\hypertarget{celebraciuxf3n-de-la-fiesta-de-los-tabernuxe1culos-con-lectura-constante-de-la-ley}{%
\subsection{Celebración de la Fiesta de los Tabernáculos con lectura
constante de la
ley}\label{celebraciuxf3n-de-la-fiesta-de-los-tabernuxe1culos-con-lectura-constante-de-la-ley}}

\bibverse{13} Y el día siguiente se juntaron los príncipes de las
familias de todo el pueblo, sacerdotes, y Levitas, á Esdras escriba,
para entender las palabras de la ley. \bibverse{14} Y hallaron escrito
en la ley que Jehová había mandado por mano de Moisés, que habitasen los
hijos de Israel en cabañas en la solemnidad del mes séptimo;
\bibverse{15} Y que hiciesen saber, y pasar pregón por todas sus
ciudades y por Jerusalem, diciendo: Salid al monte, y traed ramos de
oliva, y ramos de pino, y ramos de arrayán, y ramos de palmas, y ramos
de todo árbol espeso, para hacer cabañas como está escrito.

\bibverse{16} Salió pues el pueblo, y trajeron, é hiciéronse cabañas,
cada uno sobre su terrado, y en sus patios, y en los patios de la casa
de Dios, y en la plaza de la puerta de las Aguas, y en la plaza de la
puerta de Ephraim. \footnote{\textbf{8:16} Neh 8,1}

\bibverse{17} Y toda la congregación que volvió de la cautividad
hicieron cabañas, y en cabañas habitaron; porque desde los días de Josué
hijo de Nun hasta aquel día, no habían hecho así los hijos de Israel. Y
hubo alegría muy grande. \bibverse{18} Y leyó Esdras en el libro de la
ley de Dios cada día, desde el primer día hasta el postrero; é hicieron
la solemnidad por siete días, y al octavo día congregación, según el
rito.

\hypertarget{celebraciuxf3n-del-duxeda-de-la-penitencia-con-varias-horas-de-lectura-de-la-ley-y-varias-horas-de-confesiuxf3n}{%
\subsection{Celebración del día de la penitencia con varias horas de
lectura de la ley y varias horas de
confesión}\label{celebraciuxf3n-del-duxeda-de-la-penitencia-con-varias-horas-de-lectura-de-la-ley-y-varias-horas-de-confesiuxf3n}}

\hypertarget{section-8}{%
\section{9}\label{section-8}}

\bibverse{1} Y el día veinticuatro del mismo mes se juntaron los hijos
de Israel en ayuno, y con sacos, y tierra sobre sí. \bibverse{2} Y
habíase ya apartado la simiente de Israel de todos los extranjeros; y
estando en pie, confesaron sus pecados, y las iniquidades de sus padres.
\bibverse{3} Y puestos de pie en su lugar, leyeron en el libro de la ley
de Jehová su Dios la cuarta parte del día, y la cuarta parte confesaron
y adoraron á Jehová su Dios. \bibverse{4} Levantáronse luego sobre la
grada de los Levitas, Jesuá y Bani, Cadmiel, Sebanías, Bunni, Serebías,
Bani y Chênani, y clamaron en voz alta á Jehová su Dios.

\hypertarget{invitaciuxf3n-a-alabar-a-dios-referencia-a-los-maravillosos-actos-de-poder-y-gracia-de-dios-en-tiempos-prehistuxf3ricos-hasta-la-introducciuxf3n-de-su-pueblo-en-la-tierra-prometida}{%
\subsection{Invitación a alabar a Dios; Referencia a los maravillosos
actos de poder y gracia de Dios en tiempos prehistóricos hasta la
introducción de su pueblo en la tierra
prometida}\label{invitaciuxf3n-a-alabar-a-dios-referencia-a-los-maravillosos-actos-de-poder-y-gracia-de-dios-en-tiempos-prehistuxf3ricos-hasta-la-introducciuxf3n-de-su-pueblo-en-la-tierra-prometida}}

\bibverse{5} Y dijeron los Levitas, Jesuá y Cadmiel, Bani, Hosabnías,
Serebías, Odaías, Sebanías y Pethaía: Levantaos, bendecid á Jehová
vuestro Dios desde el siglo hasta el siglo: y bendigan el nombre tuyo,
glorioso y alto sobre toda bendición y alabanza. \bibverse{6} Tú, oh
Jehová, eres solo; tú hiciste los cielos, y los cielos de los cielos, y
toda su milicia, la tierra y todo lo que está en ella, los mares y todo
lo que hay en ellos; y tú vivificas todas estas cosas, y los ejércitos
de los cielos te adoran. \bibverse{7} Tú eres, oh Jehová, el Dios que
escogiste á Abram, y lo sacaste de Ur de los Caldeos, y pusístele el
nombre Abraham; \bibverse{8} Y hallaste fiel su corazón delante de ti, é
hiciste con él alianza para darle la tierra del Cananeo, del Hetheo, y
del Amorrheo, y del Pherezeo, y del Jebuseo, y del Gergeseo, para darla
á su simiente: y cumpliste tu palabra, porque eres justo. \footnote{\textbf{9:8}
  Gén 15,18-21}

\bibverse{9} Y miraste la aflicción de nuestros padres en Egipto, y
oíste el clamor de ellos en el mar Bermejo; \footnote{\textbf{9:9} Éxod
  3,7} \bibverse{10} Y diste señales y maravillas en Faraón, y en todos
sus siervos, y en todo el pueblo de su tierra; porque sabías que habían
hecho soberbiamente contra ellos; é hicístete nombre grande, como este
día. \bibverse{11} Y dividiste la mar delante de ellos, y pasaron por
medio de ella en seco; y á sus perseguidores echaste en los profundos,
como una piedra en grandes aguas. \footnote{\textbf{9:11} Éxod 14,21;
  Éxod 15,5; Éxod 15,10} \bibverse{12} Y con columna de nube los guiaste
de día, y con columna de fuego de noche, para alumbrarles el camino por
donde habían de ir. \footnote{\textbf{9:12} Éxod 13,21}

\bibverse{13} Y sobre el monte de Sinaí descendiste, y hablaste con
ellos desde el cielo, y dísteles juicios rectos, leyes verdaderas, y
estatutos y mandamientos buenos: \footnote{\textbf{9:13} Éxod 19,18;
  Éxod 20,1} \bibverse{14} Y notificásteles el sábado tuyo santo, y les
prescribiste, por mano de Moisés tu siervo, mandamientos y estatutos y
ley. \bibverse{15} Y dísteles pan del cielo en su hambre, y en su sed
les sacaste aguas de la piedra; y dijísteles que entrasen á poseer la
tierra, por la cual alzaste tu mano que se la habías de dar.

\bibverse{16} Mas ellos y nuestros padres hicieron soberbiamente, y
endurecieron su cerviz, y no escucharon tus mandamientos, \footnote{\textbf{9:16}
  Éxod 32,9} \bibverse{17} Y no quisieron oir, ni se acordaron de tus
maravillas que habías hecho con ellos; antes endurecieron su cerviz, y
en su rebelión pensaron poner caudillo para volverse á su servidumbre.
Tú empero, eres Dios de perdones, clemente y piadoso, tardo para la ira,
y de mucha misericordia, que no los dejaste. \footnote{\textbf{9:17} Núm
  14,4; Éxod 34,6} \bibverse{18} Además, cuando hicieron para sí becerro
de fundición, y dijeron: Este es tu Dios que te hizo subir de Egipto; y
cometieron grandes abominaciones; \footnote{\textbf{9:18} Éxod 32,4}
\bibverse{19} Tú, con todo, por tus muchas misericordias no los
abandonaste en el desierto: la columna de nube no se apartó de ellos de
día, para guiarlos por el camino, ni la columna de fuego de noche, para
alumbrarles el camino por el cual habían de ir. \bibverse{20} Y diste tu
espíritu bueno para enseñarlos, y no retiraste tu maná de su boca, y
agua les diste en su sed.

\bibverse{21} Y sustentástelos cuarenta años en el desierto; de ninguna
cosa tuvieron necesidad: sus vestidos no se envejecieron, ni se
hincharon sus pies. \footnote{\textbf{9:21} Deut 8,4} \bibverse{22} Y
dísteles reinos y pueblos, y los distribuiste por cantones: y poseyeron
la tierra de Sehón, y la tierra del rey de Hesbón, y la tierra de Og rey
de Basán. \footnote{\textbf{9:22} Núm 21,24; Núm 21,35} \bibverse{23} Y
multiplicaste sus hijos como las estrellas del cielo, y metístelos en la
tierra, de la cual habías dicho á sus padres que habían de entrar á
poseerla.

\bibverse{24} Y los hijos vinieron y poseyeron la tierra, y humillaste
delante de ellos á los moradores del país, á los Cananeos, los cuales
entregaste en su mano, y á sus reyes, y á los pueblos de la tierra, para
que hiciesen de ellos á su voluntad. \footnote{\textbf{9:24} Jos 12,-1}
\bibverse{25} Y tomaron ciudades fortalecidas, y tierra pingüe, y
heredaron casas llenas de todo bien, cisternas hechas, viñas y olivares,
y muchos árboles de comer; y comieron, y hartáronse, y engrosáronse, y
deleitáronse en tu grande bondad. \footnote{\textbf{9:25} Deut 6,10-11;
  Deut 32,15}

\hypertarget{en-posesiuxf3n-de-la-tierra-el-pueblo-con-desprecio-por-los-profetas-y-la-divina-paciencia-continuxfaa-con-su-conducta-pecaminosa-hasta-que-dios-los-entrega-en-manos-de-los-gentiles}{%
\subsection{En posesión de la tierra, el pueblo, con desprecio por los
profetas y la divina paciencia, continúa con su conducta pecaminosa
hasta que Dios los entrega en manos de los
gentiles}\label{en-posesiuxf3n-de-la-tierra-el-pueblo-con-desprecio-por-los-profetas-y-la-divina-paciencia-continuxfaa-con-su-conducta-pecaminosa-hasta-que-dios-los-entrega-en-manos-de-los-gentiles}}

\bibverse{26} Empero te irritaron, y rebeláronse contra ti, y echaron tu
ley tras sus espaldas, y mataron tus profetas que protestaban contra
ellos para convertirlos á ti; é hicieron grandes abominaciones.
\bibverse{27} Y entregástelos en mano de sus enemigos, los cuales los
afligieron: y en el tiempo de su tribulación clamaron á ti, y tú desde
los cielos los oíste; y según tus muchas miseraciones les dabas
salvadores, que los salvasen de mano de sus enemigos. \footnote{\textbf{9:27}
  Jue 3,9; Jue 3,15} \bibverse{28} Mas en teniendo reposo, se volvían á
hacer lo malo delante de ti; por lo cual los dejaste en mano de sus
enemigos, que se enseñorearon de ellos: pero convertidos clamaban otra
vez á ti, y tú desde los cielos los oías, y según tus miseraciones
muchas veces los libraste. \footnote{\textbf{9:28} Jue 2,18-21}
\bibverse{29} Y protestásteles que se volviesen á tu ley; mas ellos
hicieron soberbiamente, y no oyeron tus mandamientos, sino que pecaron
contra tus juicios, los cuales si el hombre hiciere, en ellos vivirá; y
dieron hombro renitente, y endurecieron su cerviz, y no escucharon.
\footnote{\textbf{9:29} Lev 18,5} \bibverse{30} Y alargaste sobre ellos
muchos años, y protestásteles con tu espíritu por mano de tus profetas,
mas no escucharon; por lo cual los entregaste en mano de los pueblos de
la tierra. \footnote{\textbf{9:30} Jer 7,25-26; Jer 44,4-6}

\bibverse{31} Empero por tus muchas misericordias no los consumiste, ni
los dejaste; porque eres Dios clemente y misericordioso. \footnote{\textbf{9:31}
  Lam 3,22}

\hypertarget{pida-nueva-gracia-y-lealtad-y-el-alivio-del-bien-merecido-sufrimiento-desde-el-dominio-asirio-hasta-el-presente}{%
\subsection{Pida nueva gracia y lealtad y el alivio del bien merecido
sufrimiento desde el dominio asirio hasta el
presente}\label{pida-nueva-gracia-y-lealtad-y-el-alivio-del-bien-merecido-sufrimiento-desde-el-dominio-asirio-hasta-el-presente}}

\bibverse{32} Ahora pues, Dios nuestro, Dios grande, fuerte, terrible,
que guardas el pacto y la misericordia, no sea tenido en poco delante de
ti todo el trabajo que nos ha alcanzado á nuestros reyes, á nuestros
príncipes, á nuestros sacerdotes, y á nuestros profetas, y á nuestros
padres, y á todo tu pueblo, desde los días de los reyes de Asiria hasta
este día. \footnote{\textbf{9:32} Neh 1,5} \bibverse{33} Tú empero eres
justo en todo lo que ha venido sobre nosotros; porque rectamente has
hecho, mas nosotros hemos hecho lo malo: \footnote{\textbf{9:33} Esd
  9,15; Dan 9,5; Dan 9,7} \bibverse{34} Y nuestros reyes, nuestros
príncipes, nuestros sacerdotes, y nuestros padres, no pusieron por obra
tu ley, ni atendieron á tus mandamientos y á tus testimonios, con que
les protestabas. \bibverse{35} Y ellos en su reino y en tu mucho bien
que les diste, y en la tierra espaciosa y pingüe que entregaste delante
de ellos, no te sirvieron, ni se convirtieron de sus malas obras.

\bibverse{36} He aquí que hoy somos siervos, henos aquí, siervos en la
tierra que diste á nuestros padres para que comiesen su fruto y su bien.
\bibverse{37} Y se multiplica su fruto para los reyes que has puesto
sobre nosotros por nuestros pecados, quienes se enseñorean sobre
nuestros cuerpos, y sobre nuestras bestias, conforme á su voluntad, y
estamos en grande angustia. \bibverse{38} A causa pues de todo eso
nosotros hacemos fiel alianza, y la escribimos, signada de nuestros
príncipes, de nuestros Levitas, y de nuestros sacerdotes.

\hypertarget{renovaciuxf3n-federal-mediante-contrato-escrito-y-firmado-de-los-jefes-especialmente-jefes-de-familia-del-pueblo}{%
\subsection{Renovación federal mediante contrato escrito y firmado de
los jefes (especialmente jefes de familia) del
pueblo}\label{renovaciuxf3n-federal-mediante-contrato-escrito-y-firmado-de-los-jefes-especialmente-jefes-de-familia-del-pueblo}}

\hypertarget{section-9}{%
\section{10}\label{section-9}}

\bibverse{1} Y los que firmaron fueron, Nehemías el Tirsatha, hijo de
Hachâlías, y Sedecías, \bibverse{2} Seraías, Azarías, Jeremías,
\bibverse{3} Pashur, Amarías, Malchías, \bibverse{4} Hattus, Sebanías,
Malluch, \bibverse{5} Harim, Meremoth, Obadías, \bibverse{6} Daniel,
Ginethón, Baruch, \bibverse{7} Mesullam, Abías, Miamín, \bibverse{8}
Maazías, Bilgai, Semeías: estos, sacerdotes. \bibverse{9} Y Levitas:
Jesuá hijo de Azanías, Binnui de los hijos de Henadad, Cadmiel;
\bibverse{10} Y sus hermanos Sebanías, Odaía, Celita, Pelaías, Hanán;
\bibverse{11} Michâ, Rehob, Hasabías, \bibverse{12} Zachû, Serebías,
Sebanías, \bibverse{13} Odaía, Bani, Beninu. \bibverse{14} Cabezas del
pueblo: Pharos, Pahath-moab, Elam, Zattu, Bani, \bibverse{15} Bunni,
Azgad, Bebai, \bibverse{16} Adonías, Bigvai, Adín, \bibverse{17} Ater,
Ezekías, Azur, \bibverse{18} Odaía, Hasum, Besai, \bibverse{19} Ariph,
Anathoth, Nebai, \bibverse{20} Magpías, Mesullam, Hezir, \bibverse{21}
Mesezabeel, Sadoc, Jadua, \bibverse{22} Pelatías, Hanán, Anaías,
\bibverse{23} Hoseas, Hananías, Asub, \bibverse{24} Lohes, Pilha, Sobec,
\bibverse{25} Rehum, Hasabna, Maaseías, \bibverse{26} Y Ahijas, Hanán,
Anan, \bibverse{27} Malluch, Harim, Baana.

\hypertarget{evitar-los-matrimonios-mixtos-y-no-tomar-el-suxe1bado}{%
\subsection{Evitar los matrimonios mixtos y no tomar el
sábado}\label{evitar-los-matrimonios-mixtos-y-no-tomar-el-suxe1bado}}

\bibverse{28} Y el resto del pueblo, los sacerdotes, Levitas, porteros,
y cantores, Nethineos, y todos los que se habían apartado de los pueblos
de las tierras á la ley de Dios, sus mujeres, sus hijos y sus hijas, y
todo el que tenía comprensión y discernimiento, \bibverse{29}
Adhiriéronse á sus hermanos, sus principales, y vinieron en la
protestación y en el juramento de que andarían en la ley de Dios, que
fué dada por mano de Moisés siervo de Dios, y que guardarían y
cumplirían todos los mandamientos de Jehová nuestro Señor, y sus juicios
y sus estatutos; \footnote{\textbf{10:29} Esd 2,70} \bibverse{30} Y que
no daríamos nuestras hijas á los pueblos de la tierra, ni tomaríamos sus
hijas para nuestros hijos. \bibverse{31} Asimismo, que si los pueblos de
la tierra trajesen á vender mercaderías y comestibles en día de sábado,
nada tomaríamos de ellos en sábado, ni en día santificado; y que
dejaríamos el año séptimo, con remisión de toda deuda.

\hypertarget{pago-oportuno-y-abundante-de-todos-los-deberes-y-obligaciones-relacionados-con-la-adoraciuxf3n-y-el-sacerdocio}{%
\subsection{Pago oportuno y abundante de todos los deberes y
obligaciones relacionados con la adoración y el
sacerdocio}\label{pago-oportuno-y-abundante-de-todos-los-deberes-y-obligaciones-relacionados-con-la-adoraciuxf3n-y-el-sacerdocio}}

\bibverse{32} Impusímonos además por ley el cargo de contribuir cada año
con la tercera parte de un siclo, para la obra de la casa de nuestro
Dios; \bibverse{33} Para el pan de la proposición, y para la ofrenda
continua, y para el holocausto continuo, de los sábados, y de las nuevas
lunas, y de las festividades, y para las santificaciones y sacrificios
por el pecado para expiar á Israel, y para toda la obra de la casa de
nuestro Dios. \bibverse{34} Echamos también las suertes, los sacerdotes,
los Levitas, y el pueblo, acerca de la ofrenda de la leña, para traerla
á la casa de nuestro Dios, según las casas de nuestros padres, en los
tiempos determinados cada un año, para quemar sobre el altar de Jehová
nuestro Dios, como está escrito en la ley. \bibverse{35} Y que cada año
traeríamos las primicias de nuestra tierra, y las primicias de todo
fruto de todo árbol, á la casa de Jehová: \footnote{\textbf{10:35} Lev
  6,5} \bibverse{36} Asimismo los primogénitos de nuestros hijos y de
nuestras bestias, como está escrito en la ley; y que traeríamos los
primogénitos de nuestras vacas y de nuestras ovejas á la casa de nuestro
Dios, á los sacerdotes que ministran en la casa de nuestro Dios:
\footnote{\textbf{10:36} Éxod 23,19} \bibverse{37} Que traeríamos
también las primicias de nuestras masas, y nuestras ofrendas, y del
fruto de todo árbol, del vino y del aceite, á los sacerdotes, á las
cámaras de la casa de nuestro Dios, y el diezmo de nuestra tierra á los
Levitas; y que los Levitas recibirían las décimas de nuestras labores en
todas las ciudades: \footnote{\textbf{10:37} Éxod 13,2} \bibverse{38} Y
que estaría el sacerdote hijo de Aarón con los Levitas, cuando los
Levitas recibirían el diezmo: y que los Levitas llevarían el diezmo del
diezmo á la casa de nuestro Dios, á las cámaras en la casa del tesoro.
\footnote{\textbf{10:38} Núm 18,21} \bibverse{39} Porque á las cámaras
han de llevar los hijos de Israel y los hijos de Leví la ofrenda del
grano, del vino, y del aceite; y allí estarán los vasos del santuario, y
los sacerdotes que ministran, y los porteros, y los cantores; y no
abandonaremos la casa de nuestro Dios. \footnote{\textbf{10:39} Núm
  18,26; Núm 18,28}

\hypertarget{una-duxe9cima-parte-de-la-poblaciuxf3n-rural-estuxe1-determinada-por-sorteo-a-mudarse-a-jerusaluxe9n}{%
\subsection{Una décima parte de la población rural está determinada por
sorteo a mudarse a
Jerusalén}\label{una-duxe9cima-parte-de-la-poblaciuxf3n-rural-estuxe1-determinada-por-sorteo-a-mudarse-a-jerusaluxe9n}}

\hypertarget{section-10}{%
\section{11}\label{section-10}}

\bibverse{1} Y habitaron los príncipes del pueblo en Jerusalem; mas el
resto del pueblo echó suertes para traer uno de diez que morase en
Jerusalem, ciudad santa, y las nueve partes en las otras ciudades.
\footnote{\textbf{11:1} Neh 7,5} \bibverse{2} Y bendijo el pueblo á
todos los varones que voluntariamente se ofrecieron á morar en
Jerusalem.

\hypertarget{listas-de-los-jefes-de-los-juduxedos-y-benjaminitas-que-vivuxedan-en-jerusaluxe9n-incluidos-sacerdotes-porteros-etc.}{%
\subsection{Listas de los jefes de los judíos y benjaminitas que vivían
en Jerusalén (incluidos sacerdotes, porteros,
etc.)}\label{listas-de-los-jefes-de-los-juduxedos-y-benjaminitas-que-vivuxedan-en-jerusaluxe9n-incluidos-sacerdotes-porteros-etc.}}

\bibverse{3} Y estos son los principales de la provincia que moraron en
Jerusalem; mas en las ciudades de Judá habitaron cada uno en su posesión
en sus ciudades, de Israel, de los sacerdotes, y Levitas, y Nethineos, y
de los hijos de los siervos de Salomón. \footnote{\textbf{11:3} Neh
  7,57; 1Cró 9,2-17} \bibverse{4} En Jerusalem pues habitaron de los
hijos de Judá, y de los hijos de Benjamín. De los hijos de Judá: Athaías
hijo de Uzzías, hijo de Zacarías, hijo de Amarías, hijo de Sephatías,
hijo de Mahalaleel, de los hijos de Phares; \bibverse{5} Y Maasías hijo
de Baruch, hijo de Colhoze, hijo de Hazaías, hijo de Adaías, hijo de
Joiarib, hijo de Zacarías, hijo de Siloni. \bibverse{6} Todos los hijos
de Phares que moraron en Jerusalem, fueron cuatrocientos setenta y ocho
hombres fuertes.

\bibverse{7} Y estos son los hijos de Benjamín: Salú hijo de Mesullam,
hijo de Joed, hijo de Pedaías, hijo de Colaías, hijo de Maaseías, hijo
de Ithiel, hijo de Jesaía. \bibverse{8} Y tras él, Gabbai, Sallai,
novecientos veinte y ocho. \bibverse{9} Y Joel hijo de Zichri, era
prefecto de ellos, y Jehudas hijo de Senua, el segundo en la ciudad.

\bibverse{10} De los sacerdotes: Jedaías hijo de Joiarib, Jachîn,
\bibverse{11} Seraías hijo de Hilcías, hijo de Mesullam, hijo de Sadoc,
hijo de Meraioth, hijo de Ahitub, príncipe de la casa de Dios,
\bibverse{12} Y sus hermanos los que hacían la obra de la casa,
ochocientos veintidós: y Adaías hijo de Jeroham, hijo de Pelalías, hijo
de Amsi, hijo de Zacarías, hijo de Pashur, hijo de Malachías,
\bibverse{13} Y sus hermanos, príncipes de familias, doscientos cuarenta
y dos: y Amasai hijo de Azarael, hijo de Azai, hijo de Mesillemoth, hijo
de Immer, \bibverse{14} Y sus hermanos, hombres de grande vigor, ciento
veintiocho: jefe de los cuales era Zabdiel, hijo de Gedolim.

\bibverse{15} Y de los Levitas: Semaías hijo de Hassub, hijo de Azricam,
hijo de Hasabías, hijo de Buni; \bibverse{16} Y Sabethai y Jozabad, de
los principales de los Levitas, sobrestantes de la obra exterior de la
casa de Dios; \bibverse{17} Y Mattanías hijo de Michâ, hijo de Zabdi,
hijo de Asaph, el principal, el que empezaba las alabanzas y acción de
gracias al tiempo de la oración; y Bacbucías el segundo de entre sus
hermanos; y Abda hijo de Samua, hijo de Galal, hijo de Jeduthún.
\bibverse{18} Todos los Levitas en la santa ciudad fueron doscientos
ochenta y cuatro.

\bibverse{19} Y los porteros, Accub, Talmón, y sus hermanos, guardas en
las puertas, ciento setenta y dos. \bibverse{20} Y el resto de Israel,
de los sacerdotes, de los Levitas, en todas las ciudades de Judá, cada
uno en su heredad. \bibverse{21} Y los Nethineos habitaban en Ophel; y
Siha y Gispa eran sobre los Nethineos.

\bibverse{22} Y el prepósito de los Levitas en Jerusalem era Uzzi hijo
de Bani, hijo de Hasabías, hijo de Mattanías, hijo de Michâ de los
cantores los hijos de Asaph, sobre la obra de la casa de Dios.
\bibverse{23} Porque había mandamiento del rey acerca de ellos, y
determinación acerca de los cantores para cada día. \bibverse{24} Y
Pethahías hijo de Mesezabel, de los hijos de Zerah hijo de Judá, estaba
á la mano del rey en todo negocio del pueblo.

\hypertarget{lista-de-lugares-que-luego-fueron-poblados-por-juduxedos-benjaminitas-y-levitas}{%
\subsection{Lista de lugares que luego fueron poblados por judíos,
benjaminitas y
levitas}\label{lista-de-lugares-que-luego-fueron-poblados-por-juduxedos-benjaminitas-y-levitas}}

\bibverse{25} Y tocante á las aldeas y sus tierras, algunos de los hijos
de Judá habitaron en Chîriat-arba y sus aldeas, y en Dibón y sus aldeas,
y en Jecabseel y sus aldeas; \bibverse{26} Y en Jesuá, Moladah, y en
Beth-pelet; \bibverse{27} Y en Hasar-sual, y en Beer-seba, y en sus
aldeas; \bibverse{28} Y en Siclag, y en Mechôna, y en sus aldeas;
\footnote{\textbf{11:28} Jos 15,31}

\bibverse{29} Y en En-rimmón, y en Soreah y en Jarmuth; \bibverse{30}
Zanoah, Adullam, y en sus aldeas; en Lachîs y sus tierras, Azeca y sus
aldeas. Y habitaron desde Beer-seba hasta el valle de Hinnom.
\bibverse{31} Y los hijos de Benjamín desde Geba habitaron en Michmas, y
Aía, y en Beth-el y sus aldeas; \bibverse{32} En Anathoth, Nob, Ananiah;
\bibverse{33} Hasor, Rama, Gitthaim; \bibverse{34} Hadid, Seboim,
Neballath; \bibverse{35} Lod, y Ono, valle de los artífices.
\bibverse{36} Y algunos de los Levitas, en los repartimientos de Judá y
de Benjamín.

\hypertarget{clases-de-sacerdotes-y-levitas-que-regresaron-con-zorobabel-y-jesuxfas}{%
\subsection{Clases de sacerdotes y levitas que regresaron con Zorobabel
y
Jesús}\label{clases-de-sacerdotes-y-levitas-que-regresaron-con-zorobabel-y-jesuxfas}}

\hypertarget{section-11}{%
\section{12}\label{section-11}}

\bibverse{1} Y estos son los sacerdotes y Levitas que subieron con
Zorobabel hijo de Sealthiel, y con Jesuá: Seraías, Jeremías, Esdras,
\footnote{\textbf{12:1} Esd 2,2} \bibverse{2} Amarías, Malluch, Hartus,
\bibverse{3} Sechânías, Rehum, Meremoth, \bibverse{4} Iddo, Ginetho,
Abías, \bibverse{5} Miamin, Maadías, Bilga, \bibverse{6} Semaías, y
Joiarib, Jedaías, \bibverse{7} Sallum, Amoc, Hilcías, Jedaías. Estos
eran los príncipes de los sacerdotes y sus hermanos en los días de
Jesuá.

\bibverse{8} Y los Levitas: Jesuá, Binnui, Cadmiel, Serebías, Judá, y
Mathanías, que con sus hermanos oficiaba en los himnos. \footnote{\textbf{12:8}
  Neh 11,17} \bibverse{9} Y Bacbucías y Unni, sus hermanos, cada cual en
su ministerio.

\hypertarget{la-luxednea-de-sumo-sacerdote}{%
\subsection{La línea de sumo
sacerdote}\label{la-luxednea-de-sumo-sacerdote}}

\bibverse{10} Y Jesuá engendró á Joiacim, y Joiacim engendró á Eliasib,
y Eliasib engendró á Joiada, \bibverse{11} Y Joiada engendró á Jonathán,
y Jonathán engendró á Jaddua.

\hypertarget{jefes-de-familia-sacerdotal-desde-la-uxe9poca-del-sumo-sacerdote-joiacim}{%
\subsection{Jefes de familia sacerdotal desde la época del sumo
sacerdote
Joiacim}\label{jefes-de-familia-sacerdotal-desde-la-uxe9poca-del-sumo-sacerdote-joiacim}}

\bibverse{12} Y en los días de Joiacim los sacerdotes cabezas de
familias fueron: de Seraías, Meraías; de Jeremías, Hananías;
\bibverse{13} De Esdras, Mesullam; de Amarías, Johanán; \bibverse{14} De
Melichâ, Jonathán; de Sebanías, Joseph; \bibverse{15} De Harim, Adna; de
Meraioth, Helcai; \bibverse{16} De Iddo, Zacarías; de Ginnethón,
Mesullam; \bibverse{17} De Abías, Zichri; de Miniamín, de Moadías,
Piltai; \bibverse{18} De Bilga, Sammua; de Semaías, Jonathán;
\bibverse{19} De Joiarib, Mathenai; de Jedaías, Uzzi; \bibverse{20} De
Sallai, Callai; de Amoc, Eber; \bibverse{21} De Hilcías, Hasabías; de
Jedaías, Nathanael.

\hypertarget{lista-de-los-levitas-hasta-la-uxe9poca-del-sumo-sacerdote-johanuxe1n}{%
\subsection{Lista de los levitas hasta la época del sumo sacerdote
Johanán}\label{lista-de-los-levitas-hasta-la-uxe9poca-del-sumo-sacerdote-johanuxe1n}}

\bibverse{22} Los Levitas en días de Eliasib, de Joiada, y de Johanán y
Jaddua, fueron escritos por cabezas de familias; también los sacerdotes,
hasta el reinado de Darío el Persa. \footnote{\textbf{12:22} Neh
  12,10-11} \bibverse{23} Los hijos de Leví, cabezas de familias, fueron
escritos en el libro de las crónicas hasta los días de Johanán, hijo de
Eliasib. \bibverse{24} Los cabezas de los Levitas: Hasabías, Serebías, y
Jesuá hijo de Cadmiel, y sus hermanos delante de ellos, para alabar y
para rendir gracias, conforme al estatuto de David varón de Dios,
guardando su turno. \bibverse{25} Mathanías, y Bacbucías, Obadías,
Mesullam, Talmón, Accub, guardas, eran porteros para la guardia á las
entradas de las puertas. \footnote{\textbf{12:25} Neh 11,17; Neh 11,19;
  2Cró 8,14; 1Cró 26,15; 1Cró 26,17} \bibverse{26} Estos fueron en los
días de Joiacim, hijo de Jesuá, hijo de Josadac, y en los días del
gobernador Nehemías, y del sacerdote Esdras, escriba. \footnote{\textbf{12:26}
  Neh 12,10; 1Cró 5,40-41; Neh 5,14; Esd 7,1-6}

\hypertarget{inauguraciuxf3n-de-la-muralla-de-la-ciudad}{%
\subsection{Inauguración de la muralla de la
ciudad}\label{inauguraciuxf3n-de-la-muralla-de-la-ciudad}}

\bibverse{27} Y á la dedicación del muro de Jerusalem buscaron á los
Levitas de todos sus lugares, para traerlos á Jerusalem, para hacer la
dedicación y la fiesta con alabanzas y con cánticos, con címbalos,
salterios y cítaras. \bibverse{28} Y fueron reunidos los hijos de los
cantores, así de la campiña alrededor de Jerusalem como de las aldeas de
Netophati; \bibverse{29} Y de la casa de Gilgal, y de los campos de
Geba, y de Azmaveth; porque los cantores se habían edificado aldeas
alrededor de Jerusalem. \bibverse{30} Y se purificaron los sacerdotes y
los Levitas; y purificaron al pueblo, y las puertas, y el muro.

\bibverse{31} Hice luego subir á los príncipes de Judá sobre el muro, y
puse dos coros grandes que fueron en procesión: el uno á la mano derecha
sobre el muro hacia la puerta del Muladar. \footnote{\textbf{12:31} Neh
  2,13; Neh 3,13} \bibverse{32} E iba tras de ellos Osaías, y la mitad
de los príncipes de Judá, \bibverse{33} Y Azarías, Esdras y Mesullam,
\bibverse{34} Judá y Benjamín, y Semaías, y Jeremías; \bibverse{35} Y de
los hijos de los sacerdotes iban con trompetas, Zacarías hijo de
Jonathán, hijo de Semaías, hijo de Mathanías, hijo de Michâías, hijo de
Zachûr, hijo de Asaph; \bibverse{36} Y sus hermanos Semaías, y Azarael,
Milalai, Gilalai, Maai, Nathanael, Judá y Hanani, con los instrumentos
músicos de David varón de Dios; y Esdras escriba, delante de ellos.
\bibverse{37} Y á la puerta de la Fuente, en derecho delante de ellos,
subieron por las gradas de la ciudad de David, por la subida del muro,
desde la casa de David hasta la puerta de las Aguas al oriente.

\bibverse{38} Y el segundo coro iba del lado opuesto, y yo en pos de él,
con la mitad del pueblo sobre el muro, desde la torre de los Hornos
hasta el muro ancho; \footnote{\textbf{12:38} Neh 3,11} \bibverse{39} Y
desde la puerta de Ephraim hasta la puerta vieja, y á la puerta del
Pescado, y la torre de Hananeel, y la torre de Hamath, hasta la puerta
de las Ovejas: y pararon en la puerta de la Cárcel. \bibverse{40}
Pararon luego los dos coros en la casa de Dios; y yo, y la mitad de los
magistrados conmigo; \bibverse{41} Y los sacerdotes, Eliacim, Maaseías,
Miniamin, Michâías, Elioenai, Zacarías, y Hananías, con trompetas;
\bibverse{42} Y Maaseías, y Semeías, y Eleazar, y Uzzi, y Johanán, y
Malchías, y Elam, y Ezer. Y los cantores cantaban alto, é Israhía era el
prefecto. \bibverse{43} Y sacrificaron aquel día grandes víctimas, é
hicieron alegrías; porque Dios los había recreado con grande
contentamiento: alegráronse también las mujeres y muchachos; y el
alborozo de Jerusalem fué oído de lejos.

\hypertarget{empleo-de-funcionarios-para-supervisar-los-ingresos-de-los-sacerdotes-y-levitas}{%
\subsection{Empleo de funcionarios para supervisar los ingresos de los
sacerdotes y
levitas}\label{empleo-de-funcionarios-para-supervisar-los-ingresos-de-los-sacerdotes-y-levitas}}

\bibverse{44} Y en aquel día fueron puestos varones sobre las cámaras de
los tesoros, de las ofrendas, de las primicias, y de los diezmos, para
juntar en ellas, de los campos de las ciudades, las porciones legales
para los sacerdotes y Levitas: porque era grande el gozo de Judá con
respecto á los sacerdotes y Levitas que asistían. \bibverse{45} Y habían
guardado la observancia de su Dios, y la observancia de la expiación,
como también los cantores y los porteros, conforme al estatuto de David
y de Salomón su hijo. \bibverse{46} Porque desde el tiempo de David y de
Asaph, ya de antiguo, había príncipes de cantores, y cántico y alabanza,
y acción de gracias á Dios. \footnote{\textbf{12:46} 1Cró 25,-1}
\bibverse{47} Y todo Israel en días de Zorobabel, y en días de Nehemías,
daba raciones á los cantores y á los porteros, cada cosa en su día:
consagraban asimismo sus porciones á los Levitas, y los Levitas
consagraban parte á los hijos de Aarón. \footnote{\textbf{12:47} Neh
  10,39}

\hypertarget{eliminaciuxf3n-de-los-componentes-paganos-especialmente-amonitas-y-moabitas-de-la-comunidad}{%
\subsection{Eliminación de los componentes paganos (especialmente
amonitas y moabitas) de la
comunidad}\label{eliminaciuxf3n-de-los-componentes-paganos-especialmente-amonitas-y-moabitas-de-la-comunidad}}

\hypertarget{section-12}{%
\section{13}\label{section-12}}

\bibverse{1} Aquel día se leyó en el libro de Moisés oyéndolo el pueblo,
y fué hallado en él escrito, que los Ammonitas y Moabitas no debían
entrar jamás en la congregación de Dios; \footnote{\textbf{13:1} Deut
  23,4-6} \bibverse{2} Por cuanto no salieron á recibir á los hijos de
Israel con pan y agua, antes alquilaron á Balaam contra ellos, para que
los maldijera: mas nuestro Dios volvió la maldición en bendición.
\footnote{\textbf{13:2} Núm 22,5-6} \bibverse{3} Y fué que, como oyeron
la ley, apartaron de Israel toda mistura.

\hypertarget{eliminaciuxf3n-de-la-celda-de-tobija-en-el-templo}{%
\subsection{Eliminación de la celda de Tobija en el
templo}\label{eliminaciuxf3n-de-la-celda-de-tobija-en-el-templo}}

\bibverse{4} Y antes de esto, Eliasib sacerdote, siendo superintendente
de la cámara de la casa de nuestro Dios, había emparentado con Tobías,
\bibverse{5} Y le había hecho una grande cámara, en la cual guardaban
antes las ofrendas, y el perfume, y los vasos, y el diezmo del grano, y
del vino y del aceite, que estaba mandado dar á los Levitas, á los
cantores, y á los porteros; y la ofrenda de los sacerdotes. \bibverse{6}
Mas á todo esto, yo no estaba en Jerusalem; porque el año treinta y dos
de Artajerjes rey de Babilonia, vine al rey; y al cabo de días fuí
enviado del rey. \bibverse{7} Y venido á Jerusalem, entendí el mal que
había hecho Eliasib en atención á Tobías, haciendo para él cámara en los
patios de la casa de Dios. \bibverse{8} Y dolióme en gran manera; y eché
todas las alhajas de la casa de Tobías fuera de la cámara; \bibverse{9}
Y dije que limpiasen las cámaras, é hice volver allí las alhajas de la
casa de Dios, las ofrendas y el perfume. \footnote{\textbf{13:9} Neh
  10,40}

\hypertarget{asegurar-la-correcta-entrega-de-los-tributos-a-los-levitas}{%
\subsection{Asegurar la correcta entrega de los tributos a los
levitas}\label{asegurar-la-correcta-entrega-de-los-tributos-a-los-levitas}}

\bibverse{10} Entendí asimismo que las partes de los Levitas no se les
habían dado; y que los Levitas y cantores que hacían el servicio se
habían huído cada uno á su heredad. \bibverse{11} Y reprendí á los
magistrados, y dije: ¿Por qué está la casa de Dios abandonada? Y
juntélos, y púselos en su lugar. \bibverse{12} Y todo Judá trajo el
diezmo del grano, del vino y del aceite, á los almacenes. \footnote{\textbf{13:12}
  Núm 18,21} \bibverse{13} Y puse por sobrestantes de ellos á Selemías
sacerdote, y á Sadoc escriba, y de los Levitas, á Pedaías; y á mano de
ellos Hanán hijo de Zaccur, hijo de Mathanías: pues que eran tenidos por
fieles, y de ellos era el repartir á sus hermanos.

\bibverse{14} Acuérdate de mí, oh Dios, en orden á esto, y no raigas mis
misericordias que hice en la casa de mi Dios, y en sus observancias.

\hypertarget{eliminar-la-profanaciuxf3n-del-suxe1bado-por-parte-de-empresarios-y-comerciantes}{%
\subsection{Eliminar la profanación del sábado por parte de empresarios
y
comerciantes}\label{eliminar-la-profanaciuxf3n-del-suxe1bado-por-parte-de-empresarios-y-comerciantes}}

\bibverse{15} En aquellos días ví en Judá algunos que pisaban en lagares
el sábado, y que acarreaban haces, y cargaban asnos con vino, y también
de uvas, de higos, y toda suerte de carga, y traían á Jerusalem en día
de sábado; y protestéles acerca del día que vendían el mantenimiento.
\footnote{\textbf{13:15} Neh 10,32; Jer 17,21-27} \bibverse{16} También
estaban en ella Tirios que traían pescado y toda mercadería, y vendían
en sábado á los hijos de Judá en Jerusalem. \bibverse{17} Y reprendí á
los señores de Judá, y díjeles: ¿Qué mala cosa es esta que vosotros
hacéis, profanando así el día del sábado? \bibverse{18} ¿No hicieron así
vuestros padres, y trajo nuestro Dios sobre nosotros todo este mal, y
sobre esta ciudad? ¿Y vosotros añadís ira sobre Israel profanando el
sábado?

\bibverse{19} Sucedió pues, que cuando iba oscureciendo á las puertas de
Jerusalem antes del sábado, dije que se cerrasen las puertas, y ordené
que no las abriesen hasta después del sábado; y puse á las puertas
algunos de mis criados, para que en día de sábado no entrasen carga.
\bibverse{20} Y quedáronse fuera de Jerusalem una y dos veces los
negociantes, y los que vendían toda especie de mercancía. \bibverse{21}
Y protestéles, y díjeles: ¿Por qué os quedáis vosotros delante del muro?
Si lo hacéis otra vez, os echaré mano. Desde entonces no vinieron en
sábado. \bibverse{22} Y dije á los Levitas que se purificasen, y
viniesen á guardar las puertas, para santificar el día del sábado.
También por esto acuérdate de mí, Dios mío, y perdóname según la
muchedumbre de tu misericordia.

\hypertarget{medidas-contra-los-matrimonios-mixtos-rechazo-del-hijo-de-un-sumo-sacerdote}{%
\subsection{Medidas contra los matrimonios mixtos; Rechazo del hijo de
un sumo
sacerdote}\label{medidas-contra-los-matrimonios-mixtos-rechazo-del-hijo-de-un-sumo-sacerdote}}

\bibverse{23} Ví asimismo en aquellos días Judíos que habían tomado
mujeres de Asdod, Ammonitas, y Moabitas: \bibverse{24} Y sus hijos la
mitad hablaban asdod, y conforme á la lengua de cada pueblo; que no
sabían hablar judaico. \bibverse{25} Y reñí con ellos, y maldíjelos, y
herí algunos de ellos, y arranquéles los cabellos, y juramentélos,
diciendo: No daréis vuestras hijas á sus hijos, y no tomaréis de sus
hijas para vuestros hijos, ó para vosotros. \footnote{\textbf{13:25}
  Deut 7,3} \bibverse{26} ¿No pecó por esto Salomón, rey de Israel? Bien
que en muchas gentes no hubo rey como él, que era amado de su Dios y
Dios lo había puesto por rey sobre todo Israel, aun á él hicieron pecar
las mujeres extranjeras. \footnote{\textbf{13:26} 1Re 11,3-8}
\bibverse{27} ¿Y obedeceremos á vosotros para cometer todo este mal tan
grande de prevaricar contra nuestro Dios, tomando mujeres extranjeras?

\bibverse{28} Y uno de los hijos de Joiada hijo de Eliasib el gran
sacerdote, era yerno de Sanballat Horonita: ahuyentélo por tanto de mí.
\footnote{\textbf{13:28} Neh 11,10; Neh 2,19} \bibverse{29} Acuérdate de
ellos, Dios mío, contra los que contaminan el sacerdocio, y el pacto del
sacerdocio y de los Levitas.

\hypertarget{fin-del-memorando}{%
\subsection{Fin del memorando}\label{fin-del-memorando}}

\bibverse{30} Limpiélos pues de todo extranjero, y puse á los sacerdotes
y Levitas por sus clases, á cada uno en su obra; \bibverse{31} Y para la
ofrenda de la leña en los tiempos señalados, y para las primicias.
Acuérdate de mí, Dios mío, para bien.
