\hypertarget{nombre-del-remitente-y-destinatario-de-la-carta-y-bendiciuxf3n-apostuxf3lica-a-la-congregaciuxf3n}{%
\subsection{Nombre del remitente y destinatario de la carta y bendición
apostólica a la
congregación}\label{nombre-del-remitente-y-destinatario-de-la-carta-y-bendiciuxf3n-apostuxf3lica-a-la-congregaciuxf3n}}

\hypertarget{section}{%
\section{1}\label{section}}

\bibverse{1} Pablo, siervo de Jesucristo, llamado á ser apóstol,
apartado para el evangelio de Dios, \footnote{\textbf{1:1} Hech 9,15;
  Hech 13,2; Gal 1,15} \bibverse{2} Que él había antes prometido por sus
profetas en las santas Escrituras, \footnote{\textbf{1:2} Rom 16,25-26;
  Tit 1,2; Luc 1,70} \bibverse{3} Acerca de su Hijo, (que fué hecho de
la simiente de David según la carne; \footnote{\textbf{1:3} 2Sam 7,12;
  Mat 22,42; Rom 9,5} \bibverse{4} El cual fué declarado Hijo de Dios
con potencia, según el espíritu de santidad, por la resurrección de los
muertos), de Jesucristo Señor nuestro, \footnote{\textbf{1:4} Hech
  13,33; Mat 28,18} \bibverse{5} Por el cual recibimos la gracia y el
apostolado, para la obediencia de la fe en todas las naciones en su
nombre, \footnote{\textbf{1:5} Rom 15,18; Gal 2,7; Gal 2,9; Hech
  26,16-18} \bibverse{6} Entre las cuales sois también vosotros,
llamados de Jesucristo: \bibverse{7} A todos los que estáis en Roma,
amados de Dios, llamados santos: Gracia y paz tengáis de Dios nuestro
Padre, y del Señor Jesucristo.

\footnote{\textbf{1:7} 1Cor 1,2; 2Cor 1,1-2; Efes 1,1; Núm 6,24-26}

\hypertarget{acciuxf3n-de-gracias-del-apuxf3stol-a-dios-por-el-estado-de-fe-de-la-comunidad-y-expresiuxf3n-del-deseo-de-poder-predicar-el-mensaje-de-salvaciuxf3n-tambiuxe9n-en-roma}{%
\subsection{Acción de gracias del Apóstol a Dios por el estado de fe de
la comunidad y expresión del deseo de poder predicar el mensaje de
salvación también en
Roma}\label{acciuxf3n-de-gracias-del-apuxf3stol-a-dios-por-el-estado-de-fe-de-la-comunidad-y-expresiuxf3n-del-deseo-de-poder-predicar-el-mensaje-de-salvaciuxf3n-tambiuxe9n-en-roma}}

\bibverse{8} Primeramente, doy gracias á mi Dios por Jesucristo acerca
de todos vosotros, de que vuestra fe es predicada en todo el mundo.
\footnote{\textbf{1:8} Rom 16,19} \bibverse{9} Porque testigo me es
Dios, al cual sirvo en mi espíritu en el evangelio de su Hijo, que sin
cesar me acuerdo de vosotros siempre en mis oraciones, \footnote{\textbf{1:9}
  Efes 1,16} \bibverse{10} Rogando, si al fin algún tiempo haya de
tener, por la voluntad de Dios, próspero viaje para ir á vosotros.
\footnote{\textbf{1:10} Rom 15,23; Rom 15,32; Hech 19,21} \bibverse{11}
Porque os deseo ver, para repartir con vosotros algún don espiritual,
para confirmaros; \footnote{\textbf{1:11} Rom 15,29} \bibverse{12} Es á
saber, para ser juntamente consolado con vosotros por la común fe
vuestra y juntamente mía. \footnote{\textbf{1:12} 2Pe 1,1}

\bibverse{13} Mas no quiero, hermanos, que ignoréis que muchas veces me
he propuesto ir á vosotros (empero hasta ahora he sido estorbado), para
tener también entre vosotros algún fruto, como entre los demás Gentiles.
\bibverse{14} A Griegos y á bárbaros, á sabios y á no sabios soy deudor.
\bibverse{15} Así que, cuanto á mí, presto estoy á anunciar el evangelio
también á vosotros que estáis en Roma.

\hypertarget{indicaciuxf3n-del-tema-la-justificaciuxf3n}{%
\subsection{Indicación del tema: La
justificación}\label{indicaciuxf3n-del-tema-la-justificaciuxf3n}}

\bibverse{16} Porque no me avergüenzo del evangelio: porque es potencia
de Dios para salud á todo aquel que cree; al Judío primeramente y
también al Griego. \footnote{\textbf{1:16} Sal 119,46; 1Cor 1,18; 1Cor
  1,24; 2Tim 1,8} \bibverse{17} Porque en él la justicia de Dios se
descubre de fe en fe; como está escrito: Mas el justo vivirá por la fe.

\footnote{\textbf{1:17} Rom 3,21-22}

\hypertarget{la-culpa-del-pecado-de-todo-paganismo}{%
\subsection{La culpa del pecado de todo
paganismo}\label{la-culpa-del-pecado-de-todo-paganismo}}

\bibverse{18} Porque manifiesta es la ira de Dios del cielo contra toda
impiedad é injusticia de los hombres, que detienen la verdad con
injusticia: \bibverse{19} Porque lo que de Dios se conoce, á ellos es
manifiesto; porque Dios se lo manifestó. \footnote{\textbf{1:19} Hech
  14,15-17; Hech 17,24-28} \bibverse{20} Porque las cosas invisibles de
él, su eterna potencia y divinidad, se echan de ver desde la creación
del mundo, siendo entendidas por las cosas que son hechas; de modo que
son inexcusables: \footnote{\textbf{1:20} Sal 19,2; Heb 11,3}
\bibverse{21} Porque habiendo conocido á Dios, no le glorificaron como á
Dios, ni dieron gracias; antes se desvanecieron en sus discursos, y el
necio corazón de ellos fué entenebrecido. \footnote{\textbf{1:21} Efes
  4,18}

\bibverse{22} Diciéndose ser sabios, se hicieron fatuos, \footnote{\textbf{1:22}
  Jer 10,14; 1Cor 1,20} \bibverse{23} Y trocaron la gloria del Dios
incorruptible en semejanza de imagen de hombre corruptible, y de aves, y
de animales de cuatro pies, y de serpientes.

\footnote{\textbf{1:23} Deut 4,15-19}

\hypertarget{el-juicio-divino-sobre-el-mundo-pagano-debido-a-su-ruina}{%
\subsection{El juicio divino sobre el mundo pagano debido a su
ruina}\label{el-juicio-divino-sobre-el-mundo-pagano-debido-a-su-ruina}}

\bibverse{24} Por lo cual también Dios los entregó á inmundicia, en las
concupiscencias de sus corazones, de suerte que contaminaron sus cuerpos
entre sí mismos: \footnote{\textbf{1:24} Hech 14,16} \bibverse{25} Los
cuales mudaron la verdad de Dios en mentira, honrando y sirviendo á las
criaturas antes que al Criador, el cual es bendito por los siglos. Amén.

\bibverse{26} Por esto Dios los entregó á afectos vergonzosos; pues aun
sus mujeres mudaron el natural uso en el uso que es contra naturaleza:
\bibverse{27} Y del mismo modo también los hombres, dejando el uso
natural de las mujeres, se encendieron en sus concupiscencias los unos
con los otros, cometiendo cosas nefandas hombres con hombres, y
recibiendo en sí mismos la recompensa que convino á su extravío.
\footnote{\textbf{1:27} Lev 18,22; Lev 20,13; 1Cor 6,9} \bibverse{28} Y
como á ellos no les pareció tener á Dios en su noticia, Dios los entregó
á una mente depravada, para hacer lo que no conviene, \bibverse{29}
Estando atestados de toda iniquidad, de fornicación, de malicia, de
avaricia, de maldad; llenos de envidia, de homicidios, de contiendas, de
engaños, de malignidades; \bibverse{30} Murmuradores, detractores,
aborrecedores de Dios, injuriosos, soberbios, altivos, inventores de
males, desobedientes á los padres, \bibverse{31} Necios, desleales, sin
afecto natural, implacables, sin misericordia: \bibverse{32} Que,
habiendo entendido el juicio de Dios que los que hacen tales cosas son
dignos de muerte, no sólo las hacen, mas aun consienten á los que las
hacen.

\hypertarget{el-juicio-de-la-ira-tambiuxe9n-estuxe1-ante-los-juduxedos-juzgar-a-los-demuxe1s-no-los-libera-del-juicio-de-dios}{%
\subsection{El juicio de la ira también está ante los judíos; juzgar a
los demás no los libera del juicio de
Dios}\label{el-juicio-de-la-ira-tambiuxe9n-estuxe1-ante-los-juduxedos-juzgar-a-los-demuxe1s-no-los-libera-del-juicio-de-dios}}

\hypertarget{section-1}{%
\section{2}\label{section-1}}

\bibverse{1} Por lo cual eres inexcusable, oh hombre, cualquiera que
juzgas: porque en lo que juzgas á otro, te condenas á ti mismo; porque
lo mismo haces, tú que juzgas. \bibverse{2} Mas sabemos que el juicio de
Dios es según verdad contra los que hacen tales cosas. \bibverse{3} ¿Y
piensas esto, oh hombre, que juzgas á los que hacen tales cosas, y haces
las mismas, que tú escaparás del juicio de Dios? \bibverse{4} ¿O
menosprecias las riquezas de su benignidad, y paciencia, y longanimidad,
ignorando que su benignidad te guía á arrepentimiento? \footnote{\textbf{2:4}
  2Pe 3,9; 2Pe 3,15} \bibverse{5} Mas por tu dureza, y por tu corazón no
arrepentido, atesoras para ti mismo ira para el día de la ira y de la
manifestación del justo juicio de Dios; \bibverse{6} El cual pagará á
cada uno conforme á sus obras: \bibverse{7} A los que perseverando en
bien hacer, buscan gloria y honra é inmortalidad, la vida eterna.
\bibverse{8} Mas á los que son contenciosos, y no obedecen á la verdad,
antes obedecen á la injusticia, enojo é ira; \footnote{\textbf{2:8} 2Tes
  1,8} \bibverse{9} Tribulación y angustia sobre toda persona humana que
obra lo malo, el Judío primeramente, y también el Griego:

\bibverse{10} Mas gloria y honra y paz á cualquiera que obra el bien, al
Judío primeramente, y también al Griego. \bibverse{11} Porque no hay
acepción de personas para con Dios.

\hypertarget{el-juicio-de-dios-es-el-mismo-para-los-juduxedos-que-para-los-gentiles-determinado-uxfanicamente-por-el-cumplimiento-de-la-ley}{%
\subsection{El juicio de Dios es el mismo para los judíos que para los
gentiles, determinado únicamente por el cumplimiento de la
ley}\label{el-juicio-de-dios-es-el-mismo-para-los-juduxedos-que-para-los-gentiles-determinado-uxfanicamente-por-el-cumplimiento-de-la-ley}}

\bibverse{12} Porque todos los que sin ley pecaron, sin ley también
perecerán; y todos los que en la ley pecaron, por la ley serán juzgados:
\bibverse{13} Porque no los oidores de la ley son justos para con Dios,
mas los hacedores de la ley serán justificados. \footnote{\textbf{2:13}
  Mat 7,21; Sant 1,22} \bibverse{14} Porque los Gentiles que no tienen
ley, naturalmente haciendo lo que es de la ley, los tales, aunque no
tengan ley, ellos son ley á sí mismos: \footnote{\textbf{2:14} Hech
  10,35} \bibverse{15} Mostrando la obra de la ley escrita en sus
corazones, dando testimonio juntamente sus conciencias, y acusándose y
también excusándose sus pensamientos unos con otros; \footnote{\textbf{2:15}
  Rom 1,32} \bibverse{16} En el día que juzgará el Señor lo encubierto
de los hombres, conforme á mi evangelio, por Jesucristo.

\footnote{\textbf{2:16} Luc 8,17}

\hypertarget{un-mejor-conocimiento-moral-y-la-capacidad-de-enseuxf1ar-no-hacen-que-los-juduxedos-sean-justos-ante-dios-su-fama-por-la-ley-es-nula-porque-la-transgrede}{%
\subsection{Un mejor conocimiento moral y la capacidad de enseñar no
hacen que los judíos sean justos ante Dios; su fama por la ley es nula
porque la
transgrede}\label{un-mejor-conocimiento-moral-y-la-capacidad-de-enseuxf1ar-no-hacen-que-los-juduxedos-sean-justos-ante-dios-su-fama-por-la-ley-es-nula-porque-la-transgrede}}

\bibverse{17} He aquí, tú tienes el sobrenombre de Judío, y estás
reposado en la ley, y te glorías en Dios, \bibverse{18} Y sabes su
voluntad, y apruebas lo mejor, instruído por la ley; \bibverse{19} Y
confías que eres guía de los ciegos, luz de los que están en tinieblas,
\footnote{\textbf{2:19} Mat 15,14} \bibverse{20} Enseñador de los que no
saben, maestro de niños, que tienes la forma de la ciencia y de la
verdad en la ley: \bibverse{21} Tú pues, que enseñas á otro, ¿no te
enseñas á ti mismo? ¿Tú, que predicas que no se ha de hurtar, hurtas?
\bibverse{22} ¿Tú, que dices que no se ha de adulterar, adulteras? ¿Tú,
que abominas los ídolos, cometes sacrilegio? \bibverse{23} ¿Tú, que te
jactas de la ley, con infracción de la ley deshonras á Dios?
\bibverse{24} Porque el nombre de Dios es blasfemado por causa de
vosotros entre los Gentiles, como está escrito.

\hypertarget{la-circuncisiuxf3n-no-tiene-valor-para-el-juduxedo-si-infringe-la-ley-la-circuncisiuxf3n-del-corazuxf3n-es-necesaria}{%
\subsection{La circuncisión no tiene valor para el judío si infringe la
ley; La circuncisión del ``corazón'' es
necesaria}\label{la-circuncisiuxf3n-no-tiene-valor-para-el-juduxedo-si-infringe-la-ley-la-circuncisiuxf3n-del-corazuxf3n-es-necesaria}}

\bibverse{25} Porque la circuncisión en verdad aprovecha, si guardares
la ley; mas si eres rebelde á la ley, tu circuncisión es hecha
incircuncisión. \footnote{\textbf{2:25} Jer 4,4} \bibverse{26} De manera
que, si el incircunciso guardare las justicias de la ley, ¿no será
tenida su incircuncisión por circuncisión? \footnote{\textbf{2:26} Gal
  5,6} \bibverse{27} Y lo que de su natural es incircunciso, guardando
perfectamente la ley, te juzgará á ti, que con la letra y con la
circuncisión eres rebelde á la ley. \bibverse{28} Porque no es Judío el
que lo es en manifiesto; ni la circuncisión es la que es en manifiesto
en la carne: \bibverse{29} Mas es Judío el que lo es en lo interior; y
la circuncisión es la del corazón, en espíritu, no en letra; la alabanza
del cual no es de los hombres, sino de Dios. \footnote{\textbf{2:29}
  Deut 30,6; Fil 3,3; Col 2,11}

\hypertarget{sin-embargo-la-posiciuxf3n-privilegiada-de-los-juduxedos-permanece-su-infidelidad-pone-la-fidelidad-de-dios-en-una-luz-muxe1s-brillante}{%
\subsection{Sin embargo, la posición privilegiada de los judíos
permanece; su infidelidad pone la fidelidad de Dios en una luz más
brillante}\label{sin-embargo-la-posiciuxf3n-privilegiada-de-los-juduxedos-permanece-su-infidelidad-pone-la-fidelidad-de-dios-en-una-luz-muxe1s-brillante}}

\hypertarget{section-2}{%
\section{3}\label{section-2}}

\bibverse{1} ¿Qué, pues, tiene más el Judío? ¿ó qué aprovecha la
circuncisión? \bibverse{2} Mucho en todas maneras. Lo primero
ciertamente, que la palabra de Dios les ha sido confiada. \bibverse{3}
¿Pues qué si algunos de ellos han sido incrédulos? ¿la incredulidad de
ellos habrá hecho vana la verdad de Dios? \footnote{\textbf{3:3} Rom
  9,6; Rom 11,29; 2Tim 2,13} \bibverse{4} En ninguna manera; antes bien
sea Dios verdadero, mas todo hombre mentiroso; como está escrito: Para
que seas justificado en tus dichos, y venzas cuando de ti se juzgare.
\footnote{\textbf{3:4} Sal 116,11}

\bibverse{5} Y si nuestra iniquidad encarece la justicia de Dios, ¿qué
diremos? ¿Será injusto Dios que da castigo? (hablo como hombre.)
\bibverse{6} En ninguna manera: de otra suerte ¿cómo juzgaría Dios el
mundo? \bibverse{7} Empero si la verdad de Dios por mi mentira creció á
gloria suya, ¿por qué aun así yo soy juzgado como pecador? \bibverse{8}
¿Y por qué no decir (como somos blasfemados, y como algunos dicen que
nosotros decimos): Hagamos males para que vengan bienes? la condenación
de los cuales es justa.

\footnote{\textbf{3:8} Rom 6,1}

\hypertarget{resultado-la-corrupciuxf3n-del-pecado-se-extiende-a-gentiles-y-juduxedos-y-es-confirmada-por-numerosas-escrituras}{%
\subsection{Resultado: la corrupción del pecado se extiende a gentiles y
judíos y es confirmada por numerosas
escrituras}\label{resultado-la-corrupciuxf3n-del-pecado-se-extiende-a-gentiles-y-juduxedos-y-es-confirmada-por-numerosas-escrituras}}

\bibverse{9} ¿Qué pues? ¿Somos mejores que ellos? En ninguna manera:
porque ya hemos acusado á Judíos y á Gentiles, que todos están debajo de
pecado. \footnote{\textbf{3:9} Rom 1,18-999} \bibverse{10} Como está
escrito: No hay justo, ni aun uno; \footnote{\textbf{3:10} Sal 14,1-3;
  Sal 53,2-4} \bibverse{11} No hay quien entienda, no hay quien busque á
Dios; \bibverse{12} Todos se apartaron, á una fueron hechos inútiles; no
hay quien haga lo bueno, no hay ni aun uno: \bibverse{13} Sepulcro
abierto es su garganta; con sus lenguas tratan engañosamente; veneno de
áspides está debajo de sus labios; \bibverse{14} Cuya boca está llena de
maledicencia y de amargura; \footnote{\textbf{3:14} Sal 10,7}
\bibverse{15} Sus pies son ligeros á derramar sangre; \footnote{\textbf{3:15}
  Is 59,7-8} \bibverse{16} Quebrantamiento y desventura hay en sus
caminos; \bibverse{17} Y camino de paz no conocieron: \footnote{\textbf{3:17}
  Luc 1,79} \bibverse{18} No hay temor de Dios delante de sus ojos.
\footnote{\textbf{3:18} Sal 36,2}

\bibverse{19} Empero sabemos que todo lo que la ley dice, á los que
están en la ley lo dice, para que toda boca se tape, y que todo el mundo
se sujete á Dios: \footnote{\textbf{3:19} Rom 2,12; Gal 3,22}
\bibverse{20} Porque por las obras de la ley ninguna carne se
justificará delante de él; porque por la ley es el conocimiento del
pecado.

\footnote{\textbf{3:20} Sal 143,2; Gal 2,16; Rom 7,7}

\hypertarget{la-justicia-de-dios-se-otorga-a-los-que-creen-en-jesuxfas}{%
\subsection{La justicia de Dios se otorga a los que creen en
Jesús}\label{la-justicia-de-dios-se-otorga-a-los-que-creen-en-jesuxfas}}

\bibverse{21} Mas ahora, sin la ley, la justicia de Dios se ha
manifestado, testificada por la ley y por los profetas: \footnote{\textbf{3:21}
  Rom 1,17; Hech 10,43} \bibverse{22} La justicia de Dios por la fe de
Jesucristo, para todos los que creen en él; porque no hay diferencia;
\footnote{\textbf{3:22} Fil 3,9} \bibverse{23} Por cuanto todos pecaron,
y están destituídos de la gloria de Dios; \footnote{\textbf{3:23} Rom
  5,2; Juan 5,44; Sal 84,12} \bibverse{24} Siendo justificados
gratuitamente por su gracia, por la redención que es en Cristo Jesús;
\footnote{\textbf{3:24} Rom 5,1; 2Cor 5,19; Efes 2,8} \bibverse{25} Al
cual Dios ha propuesto en propiciación por la fe en su sangre, para
manifestación de su justicia, atento á haber pasado por alto, en su
paciencia, los pecados pasados, \footnote{\textbf{3:25} Lev 16,12-15;
  Heb 4,16} \bibverse{26} Con la mira de manifestar su justicia en este
tiempo: para que él sea el justo, y el que justifica al que es de la fe
de Jesús.

\hypertarget{la-justicia-de-dios-por-la-fe-excluye-toda-fama-propia-y-se-aplica-tanto-a-los-gentiles-como-a-los-juduxedos}{%
\subsection{La justicia de Dios por la fe excluye toda fama propia y se
aplica tanto a los gentiles como a los
judíos}\label{la-justicia-de-dios-por-la-fe-excluye-toda-fama-propia-y-se-aplica-tanto-a-los-gentiles-como-a-los-juduxedos}}

\bibverse{27} ¿Dónde pues está la jactancia? Es excluída. ¿Por cuál ley?
¿de las obras? No; mas por la ley de la fe. \bibverse{28} Así que,
concluímos ser el hombre justificado por fe sin las obras de la ley.
\footnote{\textbf{3:28} Gal 2,16} \bibverse{29} ¿Es Dios solamente Dios
de los Judíos? ¿No es también Dios de los Gentiles? Cierto, también de
los Gentiles. \footnote{\textbf{3:29} Rom 10,12} \bibverse{30} Porque
uno es Dios, el cual justificará por la fe la circuncisión, y por medio
de la fe la incircuncisión. \footnote{\textbf{3:30} Rom 4,11-12}

\bibverse{31} ¿Luego deshacemos la ley por la fe? En ninguna manera;
antes establecemos la ley. \footnote{\textbf{3:31} Mat 5,17}

\hypertarget{evidencia-de-la-justicia-de-la-fe-en-abraham-y-mediante-un-testimonio-de-david}{%
\subsection{Evidencia de la justicia de la fe en Abraham y mediante un
testimonio de
David}\label{evidencia-de-la-justicia-de-la-fe-en-abraham-y-mediante-un-testimonio-de-david}}

\hypertarget{section-3}{%
\section{4}\label{section-3}}

\bibverse{1} ¿Qué, pues, diremos que halló Abraham nuestro padre según
la carne? \bibverse{2} Que si Abraham fué justificado por la obras,
tiene de qué gloriarse; mas no para con Dios. \bibverse{3} Porque ¿qué
dice la Escritura? Y creyó Abraham á Dios, y le fué atribuído á
justicia. \footnote{\textbf{4:3} Gal 3,6} \bibverse{4} Empero al que
obra, no se le cuenta el salario por merced, sino por deuda. \footnote{\textbf{4:4}
  Rom 11,6} \bibverse{5} Mas al que no obra, pero cree en aquél que
justifica al impío, la fe le es contada por justicia. \footnote{\textbf{4:5}
  Rom 3,26} \bibverse{6} Como también David dice ser bienaventurado el
hombre al cual Dios atribuye justicia sin obras, \bibverse{7} Diciendo:
Bienaventurados aquellos cuyas iniquidades son perdonadas, y cuyos
pecados son cubiertos. \bibverse{8} Bienaventurado el varón al cual el
Señor no imputó pecado.

\hypertarget{abraham-como-el-padre-de-todos-los-creyentes-incluidos-los-gentiles}{%
\subsection{Abraham como el padre de todos los creyentes, incluidos los
gentiles}\label{abraham-como-el-padre-de-todos-los-creyentes-incluidos-los-gentiles}}

\bibverse{9} ¿Es pues esta bienaventuranza solamente en la circuncisión,
ó también en la incircuncisión? porque decimos que á Abraham fué contada
la fe por justicia. \bibverse{10} ¿Cómo pues le fué contada? ¿en la
circuncisión, ó en la incircuncisión? No en la circuncisión, sino en la
incircuncisión. \bibverse{11} Y recibió la circuncisión por señal, por
sello de la justicia de la fe que tuvo en la incircuncisión: para que
fuese padre de todos los creyentes no circuncidados, para que también á
ellos les sea contado por justicia; \bibverse{12} Y padre de la
circuncisión, no solamente á los que son de la circuncisión, mas también
á los que siguen las pisadas de la fe que fué en nuestro padre Abraham
antes de ser circuncidado.

\footnote{\textbf{4:12} Mat 3,9}

\hypertarget{la-promesa-de-salvaciuxf3n-no-le-lleguxf3-a-abraham-por-la-ley-sino-por-la-fe}{%
\subsection{La promesa de salvación no le llegó a Abraham por la ley,
sino por la
fe}\label{la-promesa-de-salvaciuxf3n-no-le-lleguxf3-a-abraham-por-la-ley-sino-por-la-fe}}

\bibverse{13} Porque no por la ley fué dada la promesa á Abraham ó á su
simiente, que sería heredero del mundo, sino por la justicia de la fe.
\footnote{\textbf{4:13} Gén 22,17-18} \bibverse{14} Porque si los que
son de la ley son los herederos, vana es la fe, y anulada es la promesa.
\bibverse{15} Porque la ley obra ira; porque donde no hay ley, tampoco
hay transgresión. \footnote{\textbf{4:15} Rom 3,20; Rom 5,13; Rom 7,8;
  Rom 7,10}

\bibverse{16} Por tanto es por la fe, para que sea por gracia; para que
la promesa sea firme á toda simiente, no solamente al que es de la ley,
mas también al que es de la fe de Abraham, el cual es padre de todos
nosotros, \bibverse{17} (Como está escrito: Que por padre de muchas
gentes te he puesto) delante de Dios, al cual creyó; el cual da vida á
los muertos, y llama las cosas que no son, como las que son.

\hypertarget{la-fe-ejemplar-de-abraham}{%
\subsection{La fe ejemplar de Abraham}\label{la-fe-ejemplar-de-abraham}}

\bibverse{18} El creyó en esperanza contra esperanza, para venir á ser
padre de muchas gentes, conforme á lo que le había sido dicho: Así será
tu simiente. \bibverse{19} Y no se enflaqueció en la fe, ni consideró su
cuerpo ya muerto (siendo ya de casi cien años), ni la matriz muerta de
Sara; \footnote{\textbf{4:19} Gén 17,17} \bibverse{20} Tampoco en la
promesa de Dios dudó con desconfianza: antes fué esforzado en fe, dando
gloria á Dios, \footnote{\textbf{4:20} Heb 11,11} \bibverse{21}
Plenamente convencido de que todo lo que había prometido, era también
poderoso para hacerlo. \bibverse{22} Por lo cual también le fué
atribuído á justicia.

\hypertarget{tal-fe-tambiuxe9n-nos-trae-justicia-y-felicidad}{%
\subsection{Tal fe también nos trae justicia y
felicidad}\label{tal-fe-tambiuxe9n-nos-trae-justicia-y-felicidad}}

\bibverse{23} Y no solamente por él fué escrito que le haya sido
imputado; \bibverse{24} Sino también por nosotros, á quienes será
imputado, esto es, á los que creemos en el que levantó de los muertos á
Jesús Señor nuestro, \bibverse{25} El cual fué entregado por nuestros
delitos, y resucitado para nuestra justificación. \footnote{\textbf{4:25}
  Is 53,4-5; Rom 8,32; Rom 8,34}

\hypertarget{la-salvaciuxf3n-futura-estuxe1-garantizada-para-los-justificados-a-pesar-de-todas-las-tribulaciones-debido-al-amor-de-dios-demostrado-por-la-muerte-sacrificial-de-cristo}{%
\subsection{La salvación futura está garantizada para los justificados a
pesar de todas las tribulaciones debido al amor de Dios demostrado por
la muerte sacrificial de
Cristo}\label{la-salvaciuxf3n-futura-estuxe1-garantizada-para-los-justificados-a-pesar-de-todas-las-tribulaciones-debido-al-amor-de-dios-demostrado-por-la-muerte-sacrificial-de-cristo}}

\hypertarget{section-4}{%
\section{5}\label{section-4}}

\bibverse{1} Justificados pues por la fe, tenemos paz para con Dios por
medio de nuestro Señor Jesucristo: \footnote{\textbf{5:1} Rom 3,24; Rom
  3,28; Is 53,5} \bibverse{2} Por el cual también tenemos entrada por la
fe á esta gracia en la cual estamos firmes, y nos gloriamos en la
esperanza de la gloria de Dios. \footnote{\textbf{5:2} Juan 14,6; Efes
  3,12} \bibverse{3} Y no sólo esto, mas aun nos gloriamos en las
tribulaciones, sabiendo que la tribulación produce paciencia;
\footnote{\textbf{5:3} Sant 1,2; Sant 1,1-3} \bibverse{4} Y la
paciencia, prueba; y la prueba, esperanza; \footnote{\textbf{5:4} Sant
  1,12} \bibverse{5} Y la esperanza no avergüenza; porque el amor de
Dios está derramado en nuestros corazones por el Espíritu Santo que nos
es dado. \footnote{\textbf{5:5} Heb 6,18-19; Sal 22,6; Sal 25,3; Sal
  25,20}

\bibverse{6} Porque Cristo, cuando aun éramos flacos, á su tiempo murió
por los impíos. \bibverse{7} Ciertamente apenas muere alguno por un
justo: con todo podrá ser que alguno osara morir por el bueno.
\bibverse{8} Mas Dios encarece su caridad para con nosotros, porque
siendo aún pecadores, Cristo murió por nosotros. \footnote{\textbf{5:8}
  Juan 3,16; 1Jn 4,10}

\bibverse{9} Luego mucho más ahora, justificados en su sangre, por él
seremos salvos de la ira. \footnote{\textbf{5:9} Rom 1,18; Rom 2,5; Rom
  2,8} \bibverse{10} Porque si siendo enemigos, fuimos reconciliados con
Dios por la muerte de su Hijo, mucho más, estando reconciliados, seremos
salvos por su vida. \footnote{\textbf{5:10} Rom 8,7; Col 1,21; 2Cor 5,18}

\bibverse{11} Y no sólo esto, mas aun nos gloriamos en Dios por el Señor
nuestro Jesucristo, por el cual hemos ahora recibido la reconciliación.

\hypertarget{cristo-como-lo-opuesto-a-aduxe1n-la-gracia-que-trae-vida-inmortal-es-muxe1s-poderosa-que-el-pecado-mortal}{%
\subsection{Cristo como lo opuesto a Adán; la gracia que trae vida
inmortal es más poderosa que el pecado
mortal}\label{cristo-como-lo-opuesto-a-aduxe1n-la-gracia-que-trae-vida-inmortal-es-muxe1s-poderosa-que-el-pecado-mortal}}

\bibverse{12} De consiguiente, vino la reconciliación por uno, así como
el pecado entró en el mundo por un hombre, y por el pecado la muerte, y
la muerte así pasó á todos los hombres, pues que todos pecaron.
\bibverse{13} Porque hasta la ley, el pecado estaba en el mundo; pero no
se imputa pecado no habiendo ley. \footnote{\textbf{5:13} Rom 4,15}
\bibverse{14} No obstante, reinó la muerte desde Adam hasta Moisés, aun
en los que no pecaron á la manera de la rebelión de Adam; el cual es
figura del que había de venir.

\bibverse{15} Mas no como el delito, tal fué el don: porque si por el
delito de aquel uno murieron los muchos, mucho más abundó la gracia de
Dios á los muchos, y el don por la gracia de un hombre, Jesucristo.
\bibverse{16} Ni tampoco de la manera que por un pecado, así también el
don: porque el juicio á la verdad vino de un pecado para condenación,
mas la gracia vino de muchos delitos para justificación. \bibverse{17}
Porque, si por un delito reinó la muerte por uno, mucho más reinarán en
vida por un Jesucristo los que reciben la abundancia de la gracia, y del
don de la justicia.

\bibverse{18} Así que, de la manera que por un delito vino la culpa á
todos los hombres para condenación, así por una justicia vino la gracia
á todos los hombres para justificación de vida. \bibverse{19} Porque
como por la desobediencia de un hombre los muchos fueron constituídos
pecadores, así por la obediencia de uno los muchos serán constituídos
justos. \footnote{\textbf{5:19} Rom 3,26; Is 53,11} \bibverse{20} La ley
empero entró para que el pecado creciese; mas cuando el pecado creció,
sobrepujó la gracia; \footnote{\textbf{5:20} Rom 7,8; Rom 7,13; Gal 3,19}
\bibverse{21} Para que, de la manera que el pecado reinó para muerte,
así también la gracia reine por la justicia para vida eterna por
Jesucristo Señor nuestro. \footnote{\textbf{5:21} Rom 6,23}

\hypertarget{fuimos-crucificados-con-ellos-morimos-con-ellos-sepultados-con-ellos-y-resucitamos-con-cristo-jesuxfas}{%
\subsection{Fuimos crucificados con ellos, morimos con ellos, sepultados
con ellos y resucitamos con Cristo
Jesús}\label{fuimos-crucificados-con-ellos-morimos-con-ellos-sepultados-con-ellos-y-resucitamos-con-cristo-jesuxfas}}

\hypertarget{section-5}{%
\section{6}\label{section-5}}

\bibverse{1} ¿Pues qué diremos? Perseveraremos en pecado para que la
gracia crezca? \footnote{\textbf{6:1} Rom 3,5-8} \bibverse{2} En ninguna
manera. Porque los que somos muertos al pecado, ¿cómo viviremos aún en
él? \bibverse{3} ¿O no sabéis que todos los que somos bautizados en
Cristo Jesús, somos bautizados en su muerte? \footnote{\textbf{6:3} Gal
  3,27} \bibverse{4} Porque somos sepultados juntamente con él á muerte
por el bautismo; para que como Cristo resucitó de los muertos por la
gloria del Padre, así también nosotros andemos en novedad de vida.
\footnote{\textbf{6:4} Col 2,12; 1Pe 3,21}

\bibverse{5} Porque si fuimos plantados juntamente en él á la semejanza
de su muerte, así también lo seremos á la de su resurrección:
\bibverse{6} Sabiendo esto, que nuestro viejo hombre juntamente fué
crucificado con él, para que el cuerpo del pecado sea deshecho, á fin de
que no sirvamos más al pecado. \footnote{\textbf{6:6} Gal 5,24}
\bibverse{7} Porque el que es muerto, justificado es del pecado.

\hypertarget{viviendo-con-cristo-resucitado}{%
\subsection{Viviendo con Cristo
resucitado}\label{viviendo-con-cristo-resucitado}}

\bibverse{8} Y si morimos con Cristo, creemos que también viviremos con
él; \bibverse{9} Sabiendo que Cristo, habiendo resucitado de entre los
muertos, ya no muere: la muerte no se enseñoreará más de él.
\bibverse{10} Porque el haber muerto, al pecado murió una vez; mas el
vivir, á Dios vive. \bibverse{11} Así también vosotros, pensad que de
cierto estáis muertos al pecado, mas vivos á Dios en Cristo Jesús Señor
nuestro.

\footnote{\textbf{6:11} 2Cor 5,15; 1Pe 2,24}

\hypertarget{la-amonestaciuxf3n-del-apuxf3stol-a-los-fieles-de-permanecer-en-este-conocimiento-de-la-salvaciuxf3n-y-no-seguir-sirviendo-al-pecado}{%
\subsection{La amonestación del apóstol a los fieles de permanecer en
este conocimiento de la salvación y no seguir sirviendo al
pecado}\label{la-amonestaciuxf3n-del-apuxf3stol-a-los-fieles-de-permanecer-en-este-conocimiento-de-la-salvaciuxf3n-y-no-seguir-sirviendo-al-pecado}}

\bibverse{12} No reine, pues, el pecado en vuestro cuerpo mortal, para
que le obedezcáis en sus concupiscencias; \footnote{\textbf{6:12} Gén
  4,7} \bibverse{13} Ni tampoco presentéis vuestros miembros al pecado
por instrumentos de iniquidad; antes presentaos á Dios como vivos de los
muertos, y vuestros miembros á Dios por instrumentos de justicia.
\footnote{\textbf{6:13} Rom 12,1} \bibverse{14} Porque el pecado no se
enseñoreará de vosotros; pues no estáis bajo la ley, sino bajo la
gracia.

\footnote{\textbf{6:14} Rom 7,4-6; 1Jn 3,6}

\hypertarget{el-servicio-del-pecado-ha-dado-paso-a-la-justicia}{%
\subsection{El servicio del pecado ha dado paso a la
justicia}\label{el-servicio-del-pecado-ha-dado-paso-a-la-justicia}}

\bibverse{15} ¿Pues qué? ¿Pecaremos, porque no estamos bajo de la ley,
sino bajo de la gracia? En ninguna manera. \footnote{\textbf{6:15} Rom
  5,17; Rom 5,21} \bibverse{16} ¿No sabéis que á quien os prestáis
vosotros mismos por siervos para obedecerle, sois siervos de aquel á
quien obedecéis, ó del pecado para muerte, ó de la obediencia para
justicia? \footnote{\textbf{6:16} Juan 8,34} \bibverse{17} Empero
gracias á Dios, que aunque fuisteis siervos del pecado, habéis obedecido
de corazón á aquella forma de doctrina á la cual sois entregados;
\bibverse{18} Y libertados del pecado, sois hechos siervos de la
justicia. \footnote{\textbf{6:18} Juan 8,32}

\bibverse{19} Humana cosa digo, por la flaqueza de vuestra carne: que
como para iniquidad presentasteis vuestros miembros á servir á la
inmundicia y á la iniquidad, así ahora para santidad presentéis vuestros
miembros á servir á la justicia. \bibverse{20} Porque cuando fuisteis
siervos del pecado, erais libres acerca de la justicia. \bibverse{21}
¿Qué fruto, pues, teníais de aquellas cosas de las cuales ahora os
avergonzáis? porque el fin de ellas es muerte. \bibverse{22} Mas ahora,
librados del pecado, y hechos siervos á Dios, tenéis por vuestro fruto
la santificación, y por fin la vida eterna. \bibverse{23} Porque la paga
del pecado es muerte: mas la dádiva de Dios es vida eterna en Cristo
Jesús Señor nuestro. \footnote{\textbf{6:23} Rom 5,12; Sant 1,15}

\hypertarget{cuando-hemos-muerto-y-resucitado-con-cristo-estamos-leguxedtimamente-libres-de-la-ley-y-estamos-obligados-a-servir-al-cristo-resucitado-creyuxe9ndonos-muertos-al-pecado}{%
\subsection{Cuando hemos muerto y resucitado con Cristo, estamos
legítimamente libres de la ley y estamos obligados a servir al Cristo
resucitado creyéndonos muertos al
pecado}\label{cuando-hemos-muerto-y-resucitado-con-cristo-estamos-leguxedtimamente-libres-de-la-ley-y-estamos-obligados-a-servir-al-cristo-resucitado-creyuxe9ndonos-muertos-al-pecado}}

\hypertarget{section-6}{%
\section{7}\label{section-6}}

\bibverse{1} ¿Ignoráis, hermanos, (porque hablo con los que saben la
ley) que la ley se enseñorea del hombre entre tanto que vive?
\bibverse{2} Porque la mujer que está sujeta á marido, mientras el
marido vive está obligada á la ley; mas muerto el marido, libre es de la
ley del marido. \bibverse{3} Así que, viviendo el marido, se llamará
adúltera si fuere de otro varón; mas si su marido muriere, es libre de
la ley; de tal manera que no será adúltera si fuere de otro marido.
\bibverse{4} Así también vosotros, hermanos míos, estáis muertos á la
ley por el cuerpo de Cristo, para que seáis de otro, á saber, del que
resucitó de los muertos, á fin de que fructifiquemos á Dios.
\bibverse{5} Porque mientras estábamos en la carne, los afectos de los
pecados que eran por la ley, obraban en nuestros miembros fructificando
para muerte. \footnote{\textbf{7:5} Rom 6,21} \bibverse{6} Mas ahora
estamos libres de la ley, habiendo muerto á aquella en la cual estábamos
detenidos, para que sirvamos en novedad de espíritu, y no en vejez de
letra.

\footnote{\textbf{7:6} Rom 8,1-2; Rom 6,2; Rom 6,4}

\hypertarget{el-efecto-calamitoso-de-la-ley-que-familiariza-al-hombre-con-el-pecado-y-le-da-vida-al-pecado-en-la-carne}{%
\subsection{El efecto calamitoso de la ley, que familiariza al hombre
con el pecado y le da vida al pecado en la
carne}\label{el-efecto-calamitoso-de-la-ley-que-familiariza-al-hombre-con-el-pecado-y-le-da-vida-al-pecado-en-la-carne}}

\bibverse{7} ¿Qué pues diremos? ¿La ley es pecado? En ninguna manera.
Empero yo no conocí el pecado sino por la ley: porque tampoco conociera
la concupiscencia, si la ley no dijera: No codiciarás. \bibverse{8} Mas
el pecado, tomando ocasión, obró en mí por el mandamiento toda
concupiscencia: porque sin la ley el pecado está muerto. \footnote{\textbf{7:8}
  Rom 5,13; 1Cor 15,56} \bibverse{9} Así que, yo sin la ley vivía por
algún tiempo: mas venido el mandamiento, el pecado revivió, y yo morí.
\bibverse{10} Y hallé que el mandamiento, intimado para vida, para mí
era mortal: \bibverse{11} Porque el pecado, tomando ocasión, me engañó
por el mandamiento, y por él me mató. \footnote{\textbf{7:11} Heb 3,13}
\bibverse{12} De manera que la ley á la verdad es santa, y el
mandamiento santo, y justo, y bueno. \footnote{\textbf{7:12} 1Tim 1,8}

\bibverse{13} ¿Luego lo que es bueno, á mí me es hecho muerte? No; sino
que el pecado, para mostrarse pecado, por lo bueno me obró la muerte,
haciéndose pecado sobremanera pecante por el mandamiento.

\footnote{\textbf{7:13} Rom 5,20}

\hypertarget{la-impotencia-de-la-ley-y-de-la-buena-voluntad-ante-el-pecado-como-poder-en-la-carne}{%
\subsection{La impotencia de la ley y de la buena voluntad ante el
pecado como poder en la
carne}\label{la-impotencia-de-la-ley-y-de-la-buena-voluntad-ante-el-pecado-como-poder-en-la-carne}}

\bibverse{14} Porque sabemos que la ley es espiritual; mas yo soy
carnal, vendido á sujeción del pecado. \footnote{\textbf{7:14} Juan 3,6}
\bibverse{15} Porque lo que hago, no lo entiendo; ni lo que quiero,
hago; antes lo que aborrezco, aquello hago. \bibverse{16} Y si lo que no
quiero, esto hago, apruebo que la ley es buena. \bibverse{17} De manera
que ya no obro aquello, sino el pecado que mora en mí. \bibverse{18} Y
yo sé que en mí (es á saber, en mi carne) no mora el bien: porque tengo
el querer, mas efectuar el bien no lo alcanzo. \footnote{\textbf{7:18}
  Gén 6,5; Gén 8,21} \bibverse{19} Porque no hago el bien que quiero;
mas el mal que no quiero, éste hago. \bibverse{20} Y si hago lo que no
quiero, ya no lo obro yo, sino el pecado que mora en mí. \bibverse{21}
Así que, queriendo yo hacer el bien, hallo esta ley: Que el mal está en
mí. \bibverse{22} Porque según el hombre interior, me deleito en la ley
de Dios: \bibverse{23} Mas veo otra ley en mis miembros, que se rebela
contra la ley de mi espíritu, y que me lleva cautivo á la ley del pecado
que está en mis miembros. \bibverse{24} ¡Miserable hombre de mí! ¿quién
me librará del cuerpo de esta muerte? \bibverse{25} Gracias doy á Dios,
por Jesucristo Señor nuestro. Así que, yo mismo con la mente sirvo á la
ley de Dios, mas con la carne á la ley del pecado. \footnote{\textbf{7:25}
  1Cor 15,57}

\hypertarget{el-cristiano-estuxe1-bajo-la-ley-del-espuxedritu}{%
\subsection{El cristiano está bajo la ley del
Espíritu}\label{el-cristiano-estuxe1-bajo-la-ley-del-espuxedritu}}

\hypertarget{section-7}{%
\section{8}\label{section-7}}

\bibverse{1} Ahora pues, ninguna condenación hay para los que están en
Cristo Jesús, los que no andan conforme á la carne, mas conforme al
espíritu. \footnote{\textbf{8:1} Rom 8,33-34} \bibverse{2} Porque la ley
del Espíritu de vida en Cristo Jesús me ha librado de la ley del pecado
y de la muerte. \bibverse{3} Porque lo que era imposible á la ley, por
cuanto era débil por la carne, Dios enviando á su Hijo en semejanza de
carne de pecado, y á causa del pecado, condenó al pecado en la carne;
\footnote{\textbf{8:3} Hech 13,38; Hech 15,10; Heb 2,17} \bibverse{4}
Para que la justicia de la ley fuese cumplida en nosotros, que no
andamos conforme á la carne, mas conforme al espíritu.

\footnote{\textbf{8:4} Gal 5,16; Gal 5,25}

\hypertarget{el-contraste-entre-los-que-sirven-a-dios-en-el-espuxedritu-y-los-que-viven-por-los-instintos-de-la-carne}{%
\subsection{El contraste entre los que sirven a Dios en el Espíritu y
los que viven por los instintos de la
carne}\label{el-contraste-entre-los-que-sirven-a-dios-en-el-espuxedritu-y-los-que-viven-por-los-instintos-de-la-carne}}

\bibverse{5} Porque los que viven conforme á la carne, de las cosas que
son de la carne se ocupan; mas los que conforme al espíritu, de las
cosas del espíritu. \bibverse{6} Porque la intención de la carne es
muerte; mas la intención del espíritu, vida y paz: \footnote{\textbf{8:6}
  Rom 6,21; Gal 6,8} \bibverse{7} Por cuanto la intención de la carne es
enemistad contra Dios; porque no se sujeta á la ley de Dios, ni tampoco
puede. \footnote{\textbf{8:7} Sant 4,4} \bibverse{8} Así que, los que
están en la carne no pueden agradar á Dios.

\hypertarget{el-cristiano-como-morada-del-espuxedritu}{%
\subsection{El cristiano como morada del
Espíritu}\label{el-cristiano-como-morada-del-espuxedritu}}

\bibverse{9} Mas vosotros no estáis en la carne, sino en el espíritu, si
es que el Espíritu de Dios mora en vosotros. Y si alguno no tiene el
Espíritu de Cristo, el tal no es de él. \bibverse{10} Empero si Cristo
está en vosotros, el cuerpo á la verdad está muerto á causa del pecado;
mas el espíritu vive á causa de la justicia. \footnote{\textbf{8:10} Gal
  2,20} \bibverse{11} Y si el Espíritu de aquel que levantó de los
muertos á Jesús mora en vosotros, el que levantó á Cristo Jesús de los
muertos, vivificará también vuestros cuerpos mortales por su Espíritu
que mora en vosotros.

\hypertarget{la-posesiuxf3n-del-espuxedritu-garantiza-la-redenciuxf3n-fuxedsica-de-los-hijos-de-dios-si-soportan-los-sufrimientos-de-este-tiempo}{%
\subsection{La posesión del espíritu garantiza la redención física de
los hijos de Dios si soportan los sufrimientos de este
tiempo}\label{la-posesiuxf3n-del-espuxedritu-garantiza-la-redenciuxf3n-fuxedsica-de-los-hijos-de-dios-si-soportan-los-sufrimientos-de-este-tiempo}}

\bibverse{12} Así que, hermanos, deudores somos, no á la carne, para que
vivamos conforme á la carne: \bibverse{13} Porque si viviereis conforme
á la carne, moriréis; mas si por el espíritu mortificáis las obras de la
carne, viviréis. \footnote{\textbf{8:13} Rom 7,24; Gal 6,8; Efes 4,22-24}
\bibverse{14} Porque todos los que son guiados por el Espíritu de Dios,
los tales son hijos de Dios. \bibverse{15} Porque no habéis recibido el
espíritu de servidumbre para estar otra vez en temor; mas habéis
recibido el espíritu de adopción, por el cual clamamos, Abba, Padre.

\bibverse{16} Porque el mismo Espíritu da testimonio á nuestro espíritu
que somos hijos de Dios. \footnote{\textbf{8:16} 2Cor 1,22}
\bibverse{17} Y si hijos, también herederos; herederos de Dios, y
coherederos de Cristo; si empero padecemos juntamente con él, para que
juntamente con él seamos glorificados. \footnote{\textbf{8:17} Gal 4,7;
  Apoc 21,7}

\bibverse{18} Porque tengo por cierto que lo que en este tiempo se
padece, no es de comparar con la gloria venidera que en nosotros ha de
ser manifestada. \footnote{\textbf{8:18} 2Cor 4,17} \bibverse{19} Porque
el continuo anhelar de las criaturas espera la manifestación de los
hijos de Dios. \footnote{\textbf{8:19} Col 3,4; 1Jn 3,2} \bibverse{20}
Porque las criaturas sujetas fueron á vanidad, no de grado, mas por
causa del que las sujetó con esperanza, \footnote{\textbf{8:20} Gén
  3,17; Ecl 1,2} \bibverse{21} Que también las mismas criaturas serán
libradas de la servidumbre de corrupción en la libertad gloriosa de los
hijos de Dios. \footnote{\textbf{8:21} 2Pe 3,13} \bibverse{22} Porque
sabemos que todas las criaturas gimen á una, y á una están de parto
hasta ahora. \bibverse{23} Y no sólo ellas, mas también nosotros mismos,
que tenemos las primicias del Espíritu, nosotros también gemimos dentro
de nosotros mismos, esperando la adopción, es á saber, la redención de
nuestro cuerpo. \footnote{\textbf{8:23} 2Cor 5,2} \bibverse{24} Porque
en esperanza somos salvos; mas la esperanza que se ve, no es esperanza;
porque lo que alguno ve, ¿á qué esperarlo? \footnote{\textbf{8:24} 2Cor
  5,7} \bibverse{25} Empero si lo que no vemos esperamos, por paciencia
esperamos. \footnote{\textbf{8:25} Gal 5,5}

\bibverse{26} Y asimismo también el Espíritu ayuda nuestra flaqueza:
porque qué hemos de pedir como conviene, no lo sabemos; sino que el
mismo Espíritu pide por nosotros con gemidos indecibles. \bibverse{27}
Mas el que escudriña los corazones, sabe cuál es el intento del
Espíritu, porque conforme á la voluntad de Dios, demanda por los santos.

\hypertarget{el-comienzo-de-nuestra-comuniuxf3n-con-dios-obra-de-dios-garantiza-su-finalizaciuxf3n-final}{%
\subsection{El comienzo de nuestra comunión con Dios, obra de Dios,
garantiza su finalización
final}\label{el-comienzo-de-nuestra-comuniuxf3n-con-dios-obra-de-dios-garantiza-su-finalizaciuxf3n-final}}

\bibverse{28} Y sabemos que á los que á Dios aman, todas las cosas les
ayudan á bien, es á saber, á los que conforme al propósito son llamados.
\bibverse{29} Porque á los que antes conoció, también predestinó para
que fuesen hechos conformes á la imagen de su Hijo, para que él sea el
primogénito entre muchos hermanos; \footnote{\textbf{8:29} Col 1,18; Heb
  1,6} \bibverse{30} Y á los que predestinó, á éstos también llamó; y á
los que llamó, á éstos también justificó; y á los que justificó, á éstos
también glorificó.

\footnote{\textbf{8:30} Rom 3,26; 2Tes 2,13-14}

\hypertarget{por-tanto-nuestro-estado-de-salvaciuxf3n-estuxe1-divinamente-asegurado-contra-todos-los-poderes-y-nuestra-certeza-de-fe-y-seguridad-de-la-salvaciuxf3n-estuxe1-justificada}{%
\subsection{Por tanto, nuestro estado de salvación está divinamente
asegurado contra todos los poderes y nuestra certeza de fe y seguridad
de la salvación está
justificada}\label{por-tanto-nuestro-estado-de-salvaciuxf3n-estuxe1-divinamente-asegurado-contra-todos-los-poderes-y-nuestra-certeza-de-fe-y-seguridad-de-la-salvaciuxf3n-estuxe1-justificada}}

\bibverse{31} ¿Pues qué diremos á esto? Si Dios por nosotros, ¿quién
contra nosotros? \footnote{\textbf{8:31} Sal 118,6} \bibverse{32} El que
aun á su propio Hijo no perdonó, antes le entregó por todos nosotros,
¿cómo no nos dará también con él todas las cosas? \footnote{\textbf{8:32}
  Juan 3,16} \bibverse{33} ¿Quién acusará á los escogidos de Dios? Dios
es el que justifica. \bibverse{34} ¿Quién es el que condenará? Cristo es
el que murió; más aún, el que también resucitó, quien además está á la
diestra de Dios, el que también intercede por nosotros. \footnote{\textbf{8:34}
  1Jn 2,1; Heb 7,25}

\bibverse{35} ¿Quién nos apartará del amor de Cristo? tribulación? ó
angustia? ó persecución? ó hambre? ó desnudez? ó peligro? ó cuchillo?
\bibverse{36} Como está escrito: Por causa de ti somos muertos todo el
tiempo: somos estimados como ovejas de matadero. \bibverse{37} Antes, en
todas estas cosas hacemos más que vencer por medio de aquel que nos amó.
\footnote{\textbf{8:37} 1Jn 5,4} \bibverse{38} Por lo cual estoy cierto
que ni la muerte, ni la vida, ni ángeles, ni principados, ni potestades,
ni lo presente, ni lo por venir, \footnote{\textbf{8:38} Efes 6,12}
\bibverse{39} Ni lo alto, ni lo bajo, ni ninguna criatura nos podrá
apartar del amor de Dios, que es en Cristo Jesús Señor nuestro.

\hypertarget{introducciuxf3n-el-profundo-dolor-del-apuxf3stol-por-la-exclusiuxf3n-temporal-de-su-pueblo-de-la-salvaciuxf3n}{%
\subsection{Introducción: El profundo dolor del apóstol por la exclusión
temporal de su pueblo de la
salvación}\label{introducciuxf3n-el-profundo-dolor-del-apuxf3stol-por-la-exclusiuxf3n-temporal-de-su-pueblo-de-la-salvaciuxf3n}}

\hypertarget{section-8}{%
\section{9}\label{section-8}}

\bibverse{1} Verdad digo en Cristo, no miento, dándome testimonio mi
conciencia en el Espíritu Santo, \bibverse{2} Que tengo gran tristeza y
continuo dolor en mi corazón. \bibverse{3} Porque deseara yo mismo ser
apartado de Cristo por mis hermanos, los que son mis parientes según la
carne; \footnote{\textbf{9:3} Éxod 32,32} \bibverse{4} Que son
israelitas, de los cuales es la adopción, y la gloria, y el pacto, y la
data de la ley, y el culto, y las promesas; \footnote{\textbf{9:4} Éxod
  4,22; Deut 7,6; Gén 17,7; Éxod 20,-1; Éxod 40,34} \bibverse{5} Cuyos
son los padres, y de los cuales es Cristo según la carne, el cual es
Dios sobre todas las cosas, bendito por los siglos. Amén.

\footnote{\textbf{9:5} Mat 1,-1; Luc 3,23-34; Juan 1,1; Rom 1,3}

\hypertarget{las-promesas-de-dios-a-israel-son-inquebrantables-pero-no-se-aplican-a-todo-el-cuerpo-sino-solo-a-los-descendientes-espirituales-de-abraham}{%
\subsection{Las promesas de Dios a Israel son inquebrantables, pero no
se aplican a todo el cuerpo, sino solo a los descendientes espirituales
de
Abraham}\label{las-promesas-de-dios-a-israel-son-inquebrantables-pero-no-se-aplican-a-todo-el-cuerpo-sino-solo-a-los-descendientes-espirituales-de-abraham}}

\bibverse{6} No empero que la palabra de Dios haya faltado: porque no
todos los que son de Israel son Israelitas; \footnote{\textbf{9:6} Núm
  23,19; Rom 2,28} \bibverse{7} Ni por ser simiente de Abraham, son
todos hijos; mas: En Isaac te será llamada simiente. \bibverse{8} Quiere
decir: No los que son hijos de la carne, éstos son los hijos de Dios;
mas los que son hijos de la promesa, son contados en la generación.
\footnote{\textbf{9:8} Gal 4,23} \bibverse{9} Porque la palabra de la
promesa es esta: Como en este tiempo vendré, y tendrá Sara un hijo.
\bibverse{10} Y no sólo esto; mas también Rebeca concibiendo de uno, de
Isaac nuestro padre, \bibverse{11} (Porque no siendo aún nacidos, ni
habiendo hecho aún ni bien ni mal, para que el propósito de Dios
conforme á la elección, no por las obras sino por el que llama,
permaneciese;) \bibverse{12} Le fué dicho que el mayor serviría al
menor. \bibverse{13} Como está escrito: A Jacob amé, mas á Esaú
aborrecí.

\hypertarget{la-elecciuxf3n-para-la-salvaciuxf3n-es-obra-gratuita-de-la-gracia-de-dios-la-negaciuxf3n-de-la-salvaciuxf3n-y-la-gracia-no-permite-al-hombre-pelear-con-dios}{%
\subsection{La elección para la salvación es obra gratuita de la gracia
de Dios; la negación de la salvación y la gracia no permite al hombre
pelear con
Dios}\label{la-elecciuxf3n-para-la-salvaciuxf3n-es-obra-gratuita-de-la-gracia-de-dios-la-negaciuxf3n-de-la-salvaciuxf3n-y-la-gracia-no-permite-al-hombre-pelear-con-dios}}

\bibverse{14} ¿Pues qué diremos? ¿Que hay injusticia en Dios? En ninguna
manera. \bibverse{15} Mas á Moisés dice: Tendré misericordia del que
tendré misericordia, y me compadeceré del que me compadeceré.
\bibverse{16} Así que no es del que quiere, ni del que corre, sino de
Dios que tiene misericordia. \bibverse{17} Porque la Escritura dice de
Faraón: Que para esto mismo te he levantado, para mostrar en ti mi
potencia, y que mi nombre sea anunciado por toda la tierra.
\bibverse{18} De manera que del que quiere tiene misericordia; y al que
quiere, endurece. \footnote{\textbf{9:18} Éxod 4,21; 1Pe 2,8}

\bibverse{19} Me dirás pues: ¿Por qué, pues, se enoja? porque ¿quién
resistirá á su voluntad? \bibverse{20} Mas antes, oh hombre, ¿quién eres
tú, para que alterques con Dios? Dirá el vaso de barro al que le labró:
¿Por qué me has hecho tal? \bibverse{21} ¿O no tiene potestad el
alfarero para hacer de la misma masa un vaso para honra, y otro para
vergüenza? \bibverse{22} ¿Y qué, si Dios, queriendo mostrar la ira y
hacer notoria su potencia, soportó con mucha mansedumbre los vasos de
ira preparados para muerte, \footnote{\textbf{9:22} Rom 2,4; Prov 16,4}
\bibverse{23} Y para hacer notorias las riquezas de su gloria, mostrólas
para con los vasos de misericordia que él ha preparado para gloria;
\footnote{\textbf{9:23} Rom 8,29; Efes 1,3-12} \bibverse{24} Los cuales
también ha llamado, es á saber, á nosotros, no sólo de los Judíos, mas
también de los Gentiles? \bibverse{25} Como también en Oseas dice:
Llamaré al que no era mi pueblo, pueblo mío; y á la no amada, amada.
\bibverse{26} Y será, que en el lugar donde les fué dicho: Vosotros no
sois pueblo mío: allí serán llamados hijos del Dios viviente.

\bibverse{27} También Isaías clama tocante á Israel: Si fuere el número
de los hijos de Israel como la arena de la mar, las reliquias serán
salvas: \footnote{\textbf{9:27} Rom 11,5} \bibverse{28} Porque palabra
consumadora y abreviadora en justicia, porque palabra abreviada, hará el
Señor sobre la tierra.

\bibverse{29} Y como antes dijo Isaías: Si el Señor de los ejércitos no
nos hubiera dejado simiente, como Sodoma habríamos venido á ser, y á
Gomorra fuéramos semejantes.

\hypertarget{la-culpa-de-los-juduxedos-consistiuxf3-en-el-rechazo-de-la-justicia-de-la-fe-y-en-la-persecuciuxf3n-excesiva-de-la-justicia-de-las-obras}{%
\subsection{La culpa de los judíos consistió en el rechazo de la
justicia de la fe y en la persecución excesiva de la justicia de las
obras}\label{la-culpa-de-los-juduxedos-consistiuxf3-en-el-rechazo-de-la-justicia-de-la-fe-y-en-la-persecuciuxf3n-excesiva-de-la-justicia-de-las-obras}}

\bibverse{30} ¿Pues qué diremos? Que los Gentiles que no seguían
justicia, han alcanzado la justicia, es á saber, la justicia que es por
la fe;

\bibverse{31} Mas Israel que seguía la ley de justicia, no ha llegado á
la ley de justicia. \footnote{\textbf{9:31} Rom 10,2-3} \bibverse{32}
¿Por qué? Porque la seguían no por fe, mas como por las obras de la ley:
por lo cual tropezaron en la piedra de tropiezo, \bibverse{33} Como está
escrito: He aquí pongo en Sión piedra de tropiezo, y piedra de caída; y
aquel que creyere en ella, no será avergonzado.

\hypertarget{section-9}{%
\section{10}\label{section-9}}

\bibverse{1} Hermanos, ciertamente la voluntad de mi corazón y mi
oración á Dios sobre Israel, es para salud. \bibverse{2} Porque yo les
doy testimonio que tienen celo de Dios, mas no conforme á ciencia.
\bibverse{3} Porque ignorando la justicia de Dios, y procurando
establecer la suya propia, no se han sujetado á la justicia de Dios.

\hypertarget{la-falta-de-israel-es-auxfan-muxe1s-grave-ya-que-dios-no-ha-descuidado-nada-para-llevar-a-israel-a-la-justicia-de-la-fe-desde-la-uxe9poca-de-moisuxe9s}{%
\subsection{La falta de Israel es aún más grave ya que Dios no ha
descuidado nada para llevar a Israel a la justicia de la fe desde la
época de
Moisés}\label{la-falta-de-israel-es-auxfan-muxe1s-grave-ya-que-dios-no-ha-descuidado-nada-para-llevar-a-israel-a-la-justicia-de-la-fe-desde-la-uxe9poca-de-moisuxe9s}}

\bibverse{4} Porque el fin de la ley es Cristo, para justicia á todo
aquel que cree. \footnote{\textbf{10:4} Mat 5,17; Heb 8,13; Juan 3,18;
  Gal 3,24-25}

\bibverse{5} Porque Moisés describe la justicia que es por la ley: Que
el hombre que hiciere estas cosas, vivirá por ellas. \bibverse{6} Mas la
justicia que es por la fe dice así: No digas en tu corazón: ¿Quién
subirá al cielo? (esto es, para traer abajo á Cristo:) \bibverse{7} O,
¿quién descenderá al abismo? (esto es, para volver á traer á Cristo de
los muertos.) \bibverse{8} Mas ¿qué dice? Cercana está la palabra, en tu
boca y en tu corazón. Esta es la palabra de fe, la cual predicamos:
\bibverse{9} Que si confesares con tu boca al Señor Jesús, y creyeres en
tu corazón que Dios le levantó de los muertos, serás salvo.
\bibverse{10} Porque con el corazón se cree para justicia; mas con la
boca se hace confesión para salud. \bibverse{11} Porque la Escritura
dice: Todo aquel que en él creyere, no será avergonzado.

\bibverse{12} Porque no hay diferencia de Judío y de Griego: porque el
mismo que es Señor de todos, rico es para con todos los que le invocan:
\footnote{\textbf{10:12} Hech 10,34-35; Hech 15,9} \bibverse{13} Porque
todo aquel que invocare el nombre del Señor, será salvo. \bibverse{14}
¿Cómo, pues invocarán á aquel en el cual no han creído? ¿y cómo creerán
á aquel de quien no han oído? ¿y cómo oirán sin haber quien les
predique? \bibverse{15} ¿Y cómo predicarán si no fueren enviados? Como
está escrito: ¡Cuán hermosos son los pies de los que anuncian el
evangelio de la paz, de los que anuncian el evangelio de los bienes!

\hypertarget{la-inexcusabilidad-de-la-parte-incruxe9dula-de-israel-que-ha-rechazado-la-salvaciuxf3n-que-tambiuxe9n-le-fue-ofrecida}{%
\subsection{La inexcusabilidad de la parte incrédula de Israel, que ha
rechazado la salvación que también le fue
ofrecida}\label{la-inexcusabilidad-de-la-parte-incruxe9dula-de-israel-que-ha-rechazado-la-salvaciuxf3n-que-tambiuxe9n-le-fue-ofrecida}}

\bibverse{16} Mas no todos obedecen al evangelio; pues Isaías dice:
Señor, ¿quién ha creído á nuestro anuncio? \bibverse{17} Luego la fe es
por el oir; y el oir por la palabra de Dios. \bibverse{18} Mas digo: ¿No
han oído? Antes bien, por toda la tierra ha salido la fama de ellos, y
hasta los cabos de la redondez de la tierra las palabras de ellos.
\footnote{\textbf{10:18} Rom 15,19}

\bibverse{19} Mas digo: ¿No ha conocido esto Israel? Primeramente Moisés
dice: Yo os provocaré á celos con gente que no es mía; con gente
insensata os provocaré á ira.

\bibverse{20} E Isaías determinadamente dice: Fuí hallado de los que no
me buscaban; manifestéme á los que no preguntaban por mí.

\bibverse{21} Mas acerca de Israel dice: Todo el día extendí mis manos á
un pueblo rebelde y contradictor.

\hypertarget{la-mayor-parte-de-los-juduxedos-es-terca-y-rechazada-por-dios-pero-incluso-ahora-una-pequeuxf1a-parte-estuxe1-destinada-a-la-salvaciuxf3n-a-travuxe9s-de-la-gracia-de-dios}{%
\subsection{La mayor parte de los judíos es terca y rechazada por Dios,
pero incluso ahora una pequeña parte está destinada a la salvación a
través de la gracia de
Dios}\label{la-mayor-parte-de-los-juduxedos-es-terca-y-rechazada-por-dios-pero-incluso-ahora-una-pequeuxf1a-parte-estuxe1-destinada-a-la-salvaciuxf3n-a-travuxe9s-de-la-gracia-de-dios}}

\hypertarget{section-10}{%
\section{11}\label{section-10}}

\bibverse{1} Digo pues: ¿Ha desechado Dios á su pueblo? En ninguna
manera. Porque también yo soy Israelita, de la simiente de Abraham, de
la tribu de Benjamín. \footnote{\textbf{11:1} Sal 94,14; Jer 31,37; Fil
  3,5} \bibverse{2} No ha desechado Dios á su pueblo, al cual antes
conoció. ¿O no sabéis qué dice de Elías la Escritura? cómo hablando con
Dios contra Israel dice: \bibverse{3} Señor, á tus profetas han muerto,
y tus altares han derruído; y yo he quedado solo, y procuran matarme.
\bibverse{4} Mas ¿qué le dice la divina respuesta? He dejado para mí
siete mil hombres, que no han doblado la rodilla delante de Baal.
\bibverse{5} Así también, aun en este tiempo han quedado reliquias por
la elección de gracia. \bibverse{6} Y si por gracia, luego no por las
obras; de otra manera la gracia ya no es gracia. Y si por las obras, ya
no es gracia; de otra manera la obra ya no es obra.

\bibverse{7} ¿Qué pues? Lo que buscaba Israel aquello no ha alcanzado;
mas la elección lo ha alcanzado: y los demás fueron endurecidos;
\footnote{\textbf{11:7} Rom 9,31} \bibverse{8} Como está escrito: Dióles
Dios espíritu de remordimiento, ojos con que no vean, y oídos con que no
oigan, hasta el día de hoy. \footnote{\textbf{11:8} Deut 29,3}

\bibverse{9} Y David dice: Séales vuelta su mesa en lazo, y en red, y en
tropezadero, y en paga: \bibverse{10} Sus ojos sean obscurecidos para
que no vean, y agóbiales siempre el espinazo.

\hypertarget{el-propuxf3sito-divino-de-la-salvaciuxf3n-en-el-llamado-de-los-gentiles-era-vencer-la-incredulidad-de-los-juduxedos-estimuluxe1ndolos-a-emularlos-su-rechazo-no-es-definitivo}{%
\subsection{El propósito divino de la salvación en el llamado de los
gentiles era vencer la incredulidad de los judíos estimulándolos a
emularlos; su rechazo no es
definitivo}\label{el-propuxf3sito-divino-de-la-salvaciuxf3n-en-el-llamado-de-los-gentiles-era-vencer-la-incredulidad-de-los-juduxedos-estimuluxe1ndolos-a-emularlos-su-rechazo-no-es-definitivo}}

\bibverse{11} Digo pues: ¿Han tropezado para que cayesen? En ninguna
manera; mas por el tropiezo de ellos vino la salud á los Gentiles, para
que fuesen provocados á celos. \footnote{\textbf{11:11} Hech 13,46; Rom
  10,19; Deut 32,21} \bibverse{12} Y si la falta de ellos es la riqueza
del mundo, y el menoscabo de ellos la riqueza de los Gentiles, ¿cuánto
más el henchimiento de ellos?

\bibverse{13} Porque á vosotros hablo, Gentiles. Por cuanto pues, yo soy
apóstol de los Gentiles, mi ministerio honro, \bibverse{14} Por si en
alguna manera provocase á celos á mi carne, é hiciese salvos á algunos
de ellos. \bibverse{15} Porque si el extrañamiento de ellos es la
reconciliación del mundo, ¿qué será el recibimiento de ellos, sino vida
de los muertos?

\bibverse{16} Y si el primer fruto es santo, también lo es el todo, y si
la raíz es santa, también lo son las ramas. \bibverse{17} Que si algunas
de las ramas fueron quebradas, y tú, siendo acebuche, has sido ingerido
en lugar de ellas, y has sido hecho participante de la raíz y de la
grosura de la oliva; \footnote{\textbf{11:17} Efes 2,11-14}
\bibverse{18} No te jactes contra las ramas; y si te jactas, sabe que no
sustentas tú á la raíz, sino la raíz á ti. \footnote{\textbf{11:18} Juan
  4,22} \bibverse{19} Pues las ramas, dirás, fueron quebradas para que
yo fuese ingerido. \bibverse{20} Bien: por su incredulidad fueron
quebradas, mas tú por la fe estás en pie. No te ensoberbezcas, antes
teme, \footnote{\textbf{11:20} 1Cor 10,12} \bibverse{21} Que si Dios no
perdonó á las ramas naturales, á ti tampoco no perdone. \bibverse{22}
Mira, pues, la bondad y la severidad de Dios: la severidad ciertamente
en los que cayeron; mas la bondad para contigo, si permanecieres en la
bondad; pues de otra manera tú también serás cortado. \bibverse{23} Y
aun ellos, si no permanecieren en incredulidad, serán ingeridos; que
poderoso es Dios para volverlos á ingerir. \bibverse{24} Porque si tú
eres cortado del natural acebuche, y contra natura fuiste ingerido en la
buena oliva, ¿cuánto más éstos, que son las ramas naturales, serán
ingeridos en su oliva?

\hypertarget{todo-el-resto-del-pueblo-de-israel-eventualmente-llegaruxe1-a-la-fe-despuuxe9s-de-que-las-elecciones-gentiles-se-conviertan-y-todo-seruxe1-usado-para-la-justificaciuxf3n-y-glorificaciuxf3n-de-dios}{%
\subsection{Todo el resto del pueblo de Israel eventualmente llegará a
la fe después de que las elecciones gentiles se conviertan, y todo será
usado para la justificación y glorificación de
Dios}\label{todo-el-resto-del-pueblo-de-israel-eventualmente-llegaruxe1-a-la-fe-despuuxe9s-de-que-las-elecciones-gentiles-se-conviertan-y-todo-seruxe1-usado-para-la-justificaciuxf3n-y-glorificaciuxf3n-de-dios}}

\bibverse{25} Porque no quiero, hermanos, que ignoréis este misterio,
para que no seáis acerca de vosotros mismos arrogantes: que el
endurecimiento en parte ha acontecido en Israel, hasta que haya entrado
la plenitud de los Gentiles; \footnote{\textbf{11:25} Juan 10,16}
\bibverse{26} Y luego todo Israel será salvo; como está escrito: Vendrá
de Sión el Libertador, que quitará de Jacob la impiedad; \footnote{\textbf{11:26}
  Mat 23,39; Sal 14,7} \bibverse{27} Y este es mi pacto con ellos,
cuando quitare su pecados.

\bibverse{28} Así que, cuanto al evangelio, son enemigos por causa de
vosotros: mas cuanto á la elección, son muy amados por causa de los
padres. \bibverse{29} Porque sin arrepentimiento son las mercedes y la
vocación de Dios. \footnote{\textbf{11:29} Núm 23,19} \bibverse{30}
Porque como también vosotros en algún tiempo no creísteis á Dios, mas
ahora habéis alcanzado misericordia por la incredulidad de ellos;
\bibverse{31} Así también éstos ahora no han creído, para que, por la
misericordia para con vosotros, ellos también alcancen misericordia.
\bibverse{32} Porque Dios encerró á todos en incredulidad, para tener
misericordia de todos.

\bibverse{33} ¡Oh profundidad de las riquezas de la sabiduría y de la
ciencia de Dios! ¡Cuán incomprensibles son sus juicios, é inescrutables
sus caminos! \footnote{\textbf{11:33} Is 45,15; Is 55,8-9} \bibverse{34}
Porque ¿quién entendió la mente del Señor? ¿ó quién fué su consejero?
\footnote{\textbf{11:34} Jer 23,18; 1Cor 2,16} \bibverse{35} ¿O quién le
dió á él primero, para que le sea pagado?

\bibverse{36} Porque de él, y por él, y en él, son todas las cosas. A él
sea gloria por siglos. Amén.

\hypertarget{advertencia-general-como-entrada-santificaciuxf3n-de-la-vida-personal-a-travuxe9s-de-la-entrega-completa-a-dios}{%
\subsection{Advertencia general como entrada: santificación de la vida
personal a través de la entrega completa a
Dios}\label{advertencia-general-como-entrada-santificaciuxf3n-de-la-vida-personal-a-travuxe9s-de-la-entrega-completa-a-dios}}

\hypertarget{section-11}{%
\section{12}\label{section-11}}

\bibverse{1} Así que, hermanos, os ruego por las misericordias de Dios,
que presentéis vuestros cuerpos en sacrificio vivo, santo, agradable á
Dios, que es vuestro racional culto. \footnote{\textbf{12:1} Rom 6,13}
\bibverse{2} Y no os conforméis á este siglo; mas reformaos por la
renovación de vuestro entendimiento, para que experimentéis cuál sea la
buena voluntad de Dios, agradable y perfecta.

\footnote{\textbf{12:2} Efes 4,23; Efes 5,10; Efes 5,17}

\hypertarget{exhortaciuxf3n-a-la-humildad-del-individuo-y-al-uso-fiel-de-los-dones-recibidos-al-servicio-de-la-comunidad}{%
\subsection{Exhortación a la humildad del individuo y al uso fiel de los
dones recibidos al servicio de la
comunidad}\label{exhortaciuxf3n-a-la-humildad-del-individuo-y-al-uso-fiel-de-los-dones-recibidos-al-servicio-de-la-comunidad}}

\bibverse{3} Digo pues por la gracia que me es dada, á cada cual que
está entre vosotros, que no tenga más alto concepto de sí que el que
debe tener, sino que piense de sí con templanza, conforme á la medida de
fe que Dios repartió á cada uno. \footnote{\textbf{12:3} 1Cor 4,6; 1Cor
  12,11; Efes 4,7; Mat 20,26} \bibverse{4} Porque de la manera que en un
cuerpo tenemos muchos miembros, empero todos los miembros no tienen la
misma operación; \footnote{\textbf{12:4} 1Cor 12,12} \bibverse{5} Así
muchos somos un cuerpo en Cristo, mas todos miembros los unos de los
otros. \footnote{\textbf{12:5} 1Cor 12,27; Efes 4,4; Efes 4,25}
\bibverse{6} De manera que, teniendo diferentes dones según la gracia
que nos es dada, si el de profecía, úsese conforme á la medida de la fe;
\footnote{\textbf{12:6} 1Cor 4,7; 1Cor 12,4} \bibverse{7} O si
ministerio, en servir; ó el que enseña, en doctrina; \footnote{\textbf{12:7}
  1Pe 4,10-11} \bibverse{8} El que exhorta, en exhortar; el que reparte,
hágalo en simplicidad; el que preside, con solicitud; el que hace
misericordia, con alegría.

\footnote{\textbf{12:8} Mat 6,3; 2Cor 8,2; 2Cor 9,7}

\hypertarget{exhortaciuxf3n-a-amar-fraternalmente-y-a-ejercitar-sentimientos-cristianos-contra-amigos-y-enemigos}{%
\subsection{Exhortación a amar fraternalmente y a ejercitar sentimientos
cristianos contra amigos y
enemigos}\label{exhortaciuxf3n-a-amar-fraternalmente-y-a-ejercitar-sentimientos-cristianos-contra-amigos-y-enemigos}}

\bibverse{9} El amor sea sin fingimiento: aborreciendo lo malo,
llegándoos á lo bueno; \footnote{\textbf{12:9} 1Tim 1,5; Am 5,15}
\bibverse{10} Amándoos los unos á los otros con caridad fraternal;
previniéndoos con honra los unos á los otros; \footnote{\textbf{12:10}
  Juan 13,4-15; Fil 2,3} \bibverse{11} En el cuidado no perezosos;
ardientes en espíritu; sirviendo al Señor; \footnote{\textbf{12:11} Apoc
  3,15; Hech 18,25; Col 3,23} \bibverse{12} Gozosos en la esperanza;
sufridos en la tribulación; constantes en la oración; \footnote{\textbf{12:12}
  1Tes 5,17; Luc 18,1-8; Col 4,2} \bibverse{13} Comunicando á las
necesidades de los santos; siguiendo la hospitalidad. \footnote{\textbf{12:13}
  Heb 13,2; 3Jn 1,5-8}

\bibverse{14} Bendecid á los que os persiguen: bendecid, y no maldigáis.
\footnote{\textbf{12:14} Mat 5,44; 1Cor 4,12; Hech 7,59} \bibverse{15}
Gozaos con los que se gozan: llorad con los que lloran. \footnote{\textbf{12:15}
  Sal 35,13-14; 2Cor 11,29} \bibverse{16} Unánimes entre vosotros: no
altivos, mas acomodándoos á los humildes. No seáis sabios en vuestra
opinión. \footnote{\textbf{12:16} Rom 15,5; Fil 2,2} \bibverse{17} No
paguéis á nadie mal por mal; procurad lo bueno delante de todos los
hombres. \footnote{\textbf{12:17} Is 5,21; 1Tes 5,15; Prov 20,22; 2Cor
  8,21} \bibverse{18} Si se puede hacer, cuanto está en vosotros, tened
paz con todos los hombres. \footnote{\textbf{12:18} Mar 9,50; Heb 12,14}
\bibverse{19} No os venguéis vosotros mismos, amados míos; antes dad
lugar á la ira; porque escrito está: Mía es la venganza: yo pagaré, dice
el Señor. \footnote{\textbf{12:19} Lev 19,18; Mat 5,38-44} \bibverse{20}
Así que, si tu enemigo tuviere hambre, dale de comer; si tuviere sed,
dale de beber: que haciendo esto, ascuas de fuego amontonas sobre su
cabeza. \footnote{\textbf{12:20} 2Re 6,22}

\bibverse{21} No seas vencido de lo malo; mas vence con el bien el mal.

\hypertarget{exhortaciuxf3n-a-obedecer-a-las-autoridades-designadas-por-dios}{%
\subsection{Exhortación a obedecer a las autoridades designadas por
Dios}\label{exhortaciuxf3n-a-obedecer-a-las-autoridades-designadas-por-dios}}

\hypertarget{section-12}{%
\section{13}\label{section-12}}

\bibverse{1} Toda alma se someta á las potestades superiores; porque no
hay potestad sino de Dios; y las que son, de Dios son ordenadas.
\footnote{\textbf{13:1} Tit 3,1; Juan 19,11; Prov 8,15} \bibverse{2} Así
que, el que se opone á la potestad, á la ordenación de Dios resiste: y
los que resisten, ellos mismos ganan condenación para sí. \bibverse{3}
Porque los magistrados no son para temor al que bien hace, sino al malo.
¿Quieres pues no temer la potestad? haz lo bueno, y tendrás alabanza de
ella; \bibverse{4} Porque es ministro de Dios para tu bien. Mas si
hicieres lo malo, teme: porque no en vano lleva el cuchillo; porque es
ministro de Dios, vengador para castigo al que hace lo malo. \footnote{\textbf{13:4}
  2Cró 19,6-7} \bibverse{5} Por lo cual es necesario que le estéis
sujetos, no solamente por la ira, mas aun por la conciencia.
\bibverse{6} Porque por esto pagáis también los tributos; porque son
ministros de Dios que sirven á esto mismo.

\hypertarget{exhortaciones-al-cumplimiento-integral-de-los-deberes-especialmente-a-la-caridad-como-cumplimiento-de-la-ley}{%
\subsection{Exhortaciones al cumplimiento integral de los deberes,
especialmente a la caridad como cumplimiento de la
ley}\label{exhortaciones-al-cumplimiento-integral-de-los-deberes-especialmente-a-la-caridad-como-cumplimiento-de-la-ley}}

\bibverse{7} Pagad á todos lo que debéis: al que tributo, tributo; al
que pecho, pecho; al que temor, temor; al que honra, honra.

\bibverse{8} No debáis á nadie nada, sino amaros unos á otros; porque el
que ama al prójimo, cumplió la ley. \footnote{\textbf{13:8} Gal 5,14;
  1Tim 1,5} \bibverse{9} Porque: No adulterarás; no matarás; no
hurtarás; no dirás falso testimonio; no codiciarás; y si hay algún otro
mandamiento, en esta sentencia se comprende sumariamente: Amarás á tu
prójimo como á ti mismo. \bibverse{10} La caridad no hace mal al
prójimo: así que, el cumplimento de la ley es la caridad.

\hypertarget{el-fin-cercano-del-mundo-advierte-caminar-en-luz-y-santificar-la-vida-personal}{%
\subsection{El fin cercano del mundo advierte caminar en luz y
santificar la vida
personal}\label{el-fin-cercano-del-mundo-advierte-caminar-en-luz-y-santificar-la-vida-personal}}

\bibverse{11} Y esto, conociendo el tiempo, que es ya hora de
levantarnos del sueño; porque ahora nos está más cerca nuestra salud que
cuando creímos. \footnote{\textbf{13:11} Efes 5,14; 1Tes 5,6-8}
\bibverse{12} La noche ha pasado, y ha llegado el día: echemos, pues,
las obras de las tinieblas, y vistámonos las armas de luz. \footnote{\textbf{13:12}
  1Jn 2,8; Efes 5,11} \bibverse{13} Andemos como de día, honestamente:
no en glotonerías y borracheras, no en lechos y disoluciones, no en
pedencias y envidia: \footnote{\textbf{13:13} Luc 21,34; Efes 5,18}
\bibverse{14} Mas vestíos del Señor Jesucristo, y no hagáis caso de la
carne en sus deseos. \footnote{\textbf{13:14} Gal 3,27; 1Cor 9,27; Col
  2,23}

\hypertarget{juicio-sobre-el-tema-que-conmueve-a-la-comunidad-y-advierte-contra-la-condena-sin-amor-del-modo-de-vida-externo-del-pruxf3jimo}{%
\subsection{Juicio sobre el tema que conmueve a la comunidad y advierte
contra la condena sin amor del modo de vida externo del
prójimo}\label{juicio-sobre-el-tema-que-conmueve-a-la-comunidad-y-advierte-contra-la-condena-sin-amor-del-modo-de-vida-externo-del-pruxf3jimo}}

\hypertarget{section-13}{%
\section{14}\label{section-13}}

\bibverse{1} Recibid al flaco en la fe, pero no para contiendas de
disputas. \footnote{\textbf{14:1} Rom 15,1; 1Cor 8,9} \bibverse{2}
Porque uno cree que se ha de comer de todas cosas: otro que es débil,
come legumbres. \footnote{\textbf{14:2} Gén 1,29; Gén 9,3} \bibverse{3}
El que come, no menosprecie al que no come: y el que no come, no juzgue
al que come; porque Dios le ha levantado. \footnote{\textbf{14:3} Col
  2,16} \bibverse{4} ¿Tú quién eres que juzgas al siervo ajeno? para su
señor está en pie, ó cae: mas se afirmará; que poderoso es el Señor para
afirmarle. \footnote{\textbf{14:4} Mat 7,1; Sant 4,11; Sant 1,4-12}

\bibverse{5} Uno hace diferencia entre día y día; otro juzga iguales
todos los días. Cada uno esté asegurado en su ánimo. \footnote{\textbf{14:5}
  Gal 4,10} \bibverse{6} El que hace caso del día, hácelo para el Señor:
y el que no hace caso del día, no lo hace para el Señor. El que come,
come para el Señor, porque da gracias á Dios; y el que no come, no come
para el Señor, y da gracias á Dios. \bibverse{7} Porque ninguno de
nosotros vive para sí, y ninguno muere para sí. \bibverse{8} Que si
vivimos, para el Señor vivimos; y si morimos, para el Señor morimos. Así
que, ó que vivamos, ó que muramos, del Señor somos. \bibverse{9} Porque
Cristo para esto murió, y resucitó, y volvió á vivir, para ser Señor así
de los muertos como de los que viven.

\bibverse{10} Mas tú ¿por qué juzgas á tu hermano? ó tú también, ¿por
qué menosprecias á tu hermano? porque todos hemos de estar ante el
tribunal de Cristo. \footnote{\textbf{14:10} Mat 25,31-32; Hech 17,31;
  2Cor 5,10} \bibverse{11} Porque escrito está: Vivo yo, dice el Señor,
que á mí se doblará toda rodilla, y toda lengua confesará á Dios.
\footnote{\textbf{14:11} Fil 2,10-11}

\bibverse{12} De manera que, cada uno de nosotros dará á Dios razón de
sí.

\footnote{\textbf{14:12} Gal 6,5}

\hypertarget{exhortaciuxf3n-a-los-que-tienen-una-fe-fuerte-a-no-ofender-a-los-que-tienen-una-fe-duxe9bil-y-a-esforzarse-por-tener-una-conciencia-segura-en-todo-lo-que-hacen}{%
\subsection{Exhortación a los que tienen una fe fuerte a no ofender a
los que tienen una fe débil y a esforzarse por tener una conciencia
segura en todo lo que
hacen}\label{exhortaciuxf3n-a-los-que-tienen-una-fe-fuerte-a-no-ofender-a-los-que-tienen-una-fe-duxe9bil-y-a-esforzarse-por-tener-una-conciencia-segura-en-todo-lo-que-hacen}}

\bibverse{13} Así que, no juzguemos más los unos de los otros: antes
bien juzgad de no poner tropiezo ó escándalo al hermano. \footnote{\textbf{14:13}
  1Cor 10,33} \bibverse{14} Yo sé, y confío en el Señor Jesús, que de
suyo nada hay inmundo: mas á aquel que piensa alguna cosa ser inmunda,
para él es inmunda. \footnote{\textbf{14:14} Mat 15,11; Hech 10,15; Tit
  1,15} \bibverse{15} Empero si por causa de la comida tu hermano es
contristado, ya no andas conforme á la caridad. No arruines con tu
comida á aquél por el cual Cristo murió. \footnote{\textbf{14:15} 1Cor
  8,11-13} \bibverse{16} No sea pues blasfemado vuestro bien:
\bibverse{17} Que el reino de Dios no es comida ni bebida, sino justicia
y paz y gozo por el Espíritu Santo. \footnote{\textbf{14:17} 1Cor 8,8;
  Heb 13,9} \bibverse{18} Porque el que en esto sirve á Cristo, agrada á
Dios, y es acepto á los hombres. \bibverse{19} Así que, sigamos lo que
hace á la paz, y á la edificación de los unos á los otros. \bibverse{20}
No destruyas la obra de Dios por causa de la comida. Todas las cosas á
la verdad son limpias: mas malo es al hombre que come con escándalo.
\bibverse{21} Bueno es no comer carne, ni beber vino, ni nada en que tu
hermano tropiece, ó se ofenda, ó sea debilitado.

\bibverse{22} ¿Tienes tú fe? Tenla para contigo delante de Dios.
Bienaventurado el que no se condena á sí mismo con lo que aprueba.
\footnote{\textbf{14:22} Rom 14,2; 1Cor 10,25-27} \bibverse{23} Mas el
que hace diferencia, si comiere, es condenado, porque no comió por fe: y
todo lo que no es de fe, es pecado.

\hypertarget{exhortaciuxf3n-a-ser-pacientes-con-los-duxe9biles-y-a-la-unidad-de-los-cristianos-basada-en-el-ejemplo-de-cristo}{%
\subsection{Exhortación a ser pacientes con los débiles y a la unidad de
los cristianos basada en el ejemplo de
Cristo}\label{exhortaciuxf3n-a-ser-pacientes-con-los-duxe9biles-y-a-la-unidad-de-los-cristianos-basada-en-el-ejemplo-de-cristo}}

\hypertarget{section-14}{%
\section{15}\label{section-14}}

\bibverse{1} Así que, los que somos más firmes debemos sobrellevar las
flaquezas de los flacos, y no agradarnos á nosotros mismos. \bibverse{2}
Cada uno de nosotros agrade á su prójimo en bien, á edificación.
\footnote{\textbf{15:2} 1Cor 9,19; 1Cor 10,24; 1Cor 10,33} \bibverse{3}
Porque Cristo no se agradó á sí mismo; antes bien, como está escrito:
Los vituperios de los que te vituperan, cayeron sobre mí. \bibverse{4}
Porque las cosas que antes fueron escritas, para nuestra enseñanza
fueron escritas; para que por la paciencia, y por la consolación de las
Escrituras, tengamos esperanza. \bibverse{5} Mas el Dios de la paciencia
y de la consolación os dé que entre vosotros seáis unánimes según Cristo
Jesús; \footnote{\textbf{15:5} Fil 2,2} \bibverse{6} Para que concordes,
á una boca glorifiquéis al Dios y Padre de nuestro Señor Jesucristo.

\hypertarget{un-recordatorio-para-que-ambas-partes-de-la-comunidad-estuxe9n-unidas-y-tengan-una-fe-gozosa}{%
\subsection{Un recordatorio para que ambas partes de la comunidad estén
unidas y tengan una fe
gozosa}\label{un-recordatorio-para-que-ambas-partes-de-la-comunidad-estuxe9n-unidas-y-tengan-una-fe-gozosa}}

\bibverse{7} Por tanto, sobrellevaos los unos á los otros, como también
Cristo nos sobrellevó, para gloria de Dios. \bibverse{8} Digo, pues, que
Cristo Jesús fué hecho ministro de la circuncisión por la verdad de
Dios, para confirmar las promesas hechas á los padres, \bibverse{9} Y
para que los Gentiles glorifiquen á Dios por la misericordia; como está
escrito: Por tanto yo te confesaré entre los Gentiles, y cantaré á tu
nombre.

\bibverse{10} Y otra vez dice: Alegraos, Gentiles, con su pueblo.

\bibverse{11} Y otra vez: Alabad al Señor todos los Gentiles, y
magnificadle, todos los pueblos.

\bibverse{12} Y otra vez, dice Isaías: Estará la raíz de Jessé, y el que
se levantará á regir los Gentiles: los Gentiles esperarán en él.
\footnote{\textbf{15:12} Apoc 5,5}

\bibverse{13} Y el Dios de esperanza os llene de todo gozo y paz
creyendo, para que abundéis en esperanza por la virtud del Espíritu
Santo.

\hypertarget{revisiuxf3n-justificativa-del-apuxf3stol-de-la-carta-y-referencia-a-su-oficio-apostuxf3lico-para-los-gentiles}{%
\subsection{Revisión justificativa del apóstol de la carta y referencia
a su oficio apostólico para los
gentiles}\label{revisiuxf3n-justificativa-del-apuxf3stol-de-la-carta-y-referencia-a-su-oficio-apostuxf3lico-para-los-gentiles}}

\bibverse{14} Empero cierto estoy yo de vosotros, hermanos míos, que aun
vosotros mismos estáis llenos de bondad, llenos de todo conocimiento, de
tal manera que podáis amonestaros los unos á los otros. \bibverse{15}
Mas os he escrito, hermanos, en parte resueltamente, como amonestándoos
por la gracia que de Dios me es dada, \bibverse{16} Para ser ministro de
Jesucristo á los Gentiles, ministrando el evangelio de Dios, para que la
ofrenda de los Gentiles sea agradable, santificada por el Espíritu
Santo. \footnote{\textbf{15:16} Rom 11,13} \bibverse{17} Tengo, pues, de
qué gloriarme en Cristo Jesús en lo que mira á Dios. \bibverse{18}
Porque no osaría hablar alguna cosa que Cristo no haya hecho por mí para
la obediencia de los Gentiles, con la palabra y con las obras,
\bibverse{19} Con potencia de milagros y prodigios, en virtud del
Espíritu de Dios: de manera que desde Jerusalem, y por los alrededores
hasta Ilírico, he llenado todo del evangelio de Cristo. \footnote{\textbf{15:19}
  Mar 16,17; 2Cor 12,12} \bibverse{20} Y de esta manera me esforcé á
predicar el evangelio, no donde antes Cristo fuese nombrado, por no
edificar sobre ajeno fundamento: \footnote{\textbf{15:20} 2Cor 10,15-16}
\bibverse{21} Sino, como está escrito: A los que no fué anunciado de él,
verán: y los que no oyeron, entenderán.

\hypertarget{anuncio-de-los-pruxf3ximos-planes-de-viaje-del-apuxf3stol}{%
\subsection{Anuncio de los próximos planes de viaje del
apóstol}\label{anuncio-de-los-pruxf3ximos-planes-de-viaje-del-apuxf3stol}}

\bibverse{22} Por lo cual aun he sido impedido muchas veces de venir á
vosotros. \footnote{\textbf{15:22} Rom 1,13} \bibverse{23} Mas ahora no
teniendo más lugar en estas regiones, y deseando ir á vosotros muchos
años há, \footnote{\textbf{15:23} Rom 1,10-11} \bibverse{24} Cuando
partiere para España, iré á vosotros; porque espero que pasando os veré,
y que seré llevado de vosotros allá, si empero antes hubiere gozado de
vosotros. \bibverse{25} Mas ahora parto para Jerusalem á ministrar á los
santos. \footnote{\textbf{15:25} Hech 18,21; Hech 19,21; Hech 20,22;
  Hech 24,17} \bibverse{26} Porque Macedonia y Acaya tuvieron por bien
hacer una colecta para los pobres de los santos que están en Jerusalem.
\footnote{\textbf{15:26} 1Cor 16,1; 2Cor 8,1-4; 2Cor 8,9} \bibverse{27}
Porque les pareció bueno, y son deudores á ellos: porque si los Gentiles
han sido hechos participantes de sus bienes espirituales, deben también
ellos servirles en los carnales. \footnote{\textbf{15:27} 1Cor 9,11; Gal
  6,6} \bibverse{28} Así que, cuando hubiere concluído esto, y les
hubiere consignado este fruto, pasaré por vosotros á España.
\bibverse{29} Y sé que cuando llegue á vosotros, llegaré con abundancia
de la bendición del evangelio de Cristo.

\hypertarget{la-amonestaciuxf3n-del-apuxf3stol-a-la-iglesia-de-que-interceda-por-uxe9l}{%
\subsection{La amonestación del apóstol a la iglesia de que interceda
por
él}\label{la-amonestaciuxf3n-del-apuxf3stol-a-la-iglesia-de-que-interceda-por-uxe9l}}

\bibverse{30} Ruégoos empero, hermanos, por el Señor nuestro Jesucristo,
y por la caridad del Espíritu, que me ayudéis con oraciones por mí á
Dios, \bibverse{31} Que sea librado de los rebeldes que están en Judea,
y que la ofrenda de mi servicio á los santos en Jerusalem sea acepta;
\footnote{\textbf{15:31} 1Tes 2,15} \bibverse{32} Para que con gozo
llegue á vosotros por la voluntad de Dios, y que sea recreado juntamente
con vosotros. \bibverse{33} Y el Dios de paz sea con todos vosotros.
Amén.

\hypertarget{recomendaciuxf3n-de-phuxf6be-portador-de-la-carta-saludos-del-apuxf3stol-a-los-hermanos-en-roma}{%
\subsection{Recomendación de Phöbe, portador de la carta; Saludos del
Apóstol a los hermanos en
Roma}\label{recomendaciuxf3n-de-phuxf6be-portador-de-la-carta-saludos-del-apuxf3stol-a-los-hermanos-en-roma}}

\hypertarget{section-15}{%
\section{16}\label{section-15}}

\bibverse{1} Encomiéndoos empero á Febe nuestra hermana, la cual es
diaconisa de la iglesia que está en Cencreas: \bibverse{2} Que la
recibáis en el Señor, como es digno á los santos, y que la ayudéis en
cualquiera cosa en que os hubiere menester: porque ella ha ayudado á
muchos, y á mí mismo.

\bibverse{3} Saludad á Priscila y á Aquila, mis coadjutores en Cristo
Jesús; \bibverse{4} (Que pusieron sus cuellos por mi vida: á los cuales
no doy gracias yo solo, mas aun todas las iglesias de los Gentiles;)
\bibverse{5} Asimismo á la iglesia de su casa. Saludad á Epeneto, amado
mío, que es las primicias de Acaya en Cristo. \bibverse{6} Saludad á
María, la cual ha trabajado mucho con vosotros. \bibverse{7} Saludad á
Andrónico y á Junia, mis parientes, y mis compañeros en la cautividad,
los que son insignes entre los apóstoles; los cuales también fueron
antes de mí en Cristo. \bibverse{8} Saludad á Amplias, amado mío en el
Señor. \bibverse{9} Saludad á Urbano, nuestro ayudador en Cristo Jesús,
y á Stachîs, amado mío. \bibverse{10} Saludad á Apeles, probado en
Cristo. Saludad á los que son de Aristóbulo. \bibverse{11} Saludad á
Herodión, mi pariente. Saludad á los que son de la casa de Narciso, los
que están en el Señor. \bibverse{12} Saludad á Trifena y á Trifosa, las
cuales trabajan en el Señor. Saludad á Pérsida amada, la cual ha
trabajado mucho en el Señor. \bibverse{13} Saludad á Rufo, escogido en
el Señor, y á su madre y mía. \footnote{\textbf{16:13} Mar 15,21}
\bibverse{14} Saludad á Asíncrito, y á Flegonte, á Hermas, á Patrobas, á
Hermes, y á los hermanos que están con ellos. \bibverse{15} Saludad á
Filólogo y á Julia, á Nereo y á su hermana, y á Olimpas, y á todos los
santos que están con ellos. \bibverse{16} Saludaos los unos á los otros
con ósculo santo. Os saludan todas las iglesias de Cristo.

\hypertarget{advertencia-a-los-engauxf1adores-que-causan-divisiones-y-errores-en-la-iglesia}{%
\subsection{Advertencia a los engañadores que causan divisiones y
errores en la
iglesia}\label{advertencia-a-los-engauxf1adores-que-causan-divisiones-y-errores-en-la-iglesia}}

\bibverse{17} Y os ruego hermanos, que miréis los que causan disensiones
y escándalos contra la doctrina que vosotros habéis aprendido; y
apartaos de ellos. \footnote{\textbf{16:17} Mat 7,15; Tit 3,10; 2Tes 3,6}
\bibverse{18} Porque los tales no sirven al Señor nuestro Jesucristo,
sino á sus vientres; y con suaves palabras y bendiciones engañan los
corazones de los simples. \footnote{\textbf{16:18} Fil 3,19; Col 2,4}
\bibverse{19} Porque vuestra obediencia ha venido á ser notoria á todos;
así que me gozo de vosotros; mas quiero que seáis sabios en el bien, y
simples en el mal. \footnote{\textbf{16:19} Rom 1,8; 1Cor 14,20}
\bibverse{20} Y el Dios de paz quebrantará presto á Satanás debajo de
vuestros pies. La gracia del Señor nuestro Jesucristo sea con vosotros.

\hypertarget{saludos-de-los-amigos-de-pablo-a-roma-y-finalmente-alabanza-a-dios}{%
\subsection{Saludos de los amigos de Pablo a Roma y finalmente alabanza
a
Dios}\label{saludos-de-los-amigos-de-pablo-a-roma-y-finalmente-alabanza-a-dios}}

\bibverse{21} Os saludan Timoteo, mi coadjutor, y Lucio y Jasón y
Sosipater, mis parientes. \bibverse{22} Yo Tercio, que escribí la
epístola, os saludo en el Señor. \bibverse{23} Salúdaos Gayo, mi
huésped, y de toda la iglesia. Salúdaos Erasto, tesorero de la ciudad, y
el hermano Cuarto. \^{}\^{} \bibverse{24} La gracia del Señor nuestro
Jesucristo sea con todos vosotros. Amén. \bibverse{25} Y al que puede
confirmaros según mi evangelio y la predicación de Jesucristo, según la
revelación del misterio encubierto desde tiempos eternos, \bibverse{26}
Mas manifestado ahora, y por las Escrituras de los profetas, según el
mandamiento del Dios eterno, declarado á todas las gentes para que
obedezcan á la fe; \bibverse{27} Al solo Dios sabio, sea gloria por
Jesucristo para siempre. Amén. Fué escrita de Corinto á los Romanos,
enviada por medio de Febe, diaconisa de la iglesia de Cencreas.
