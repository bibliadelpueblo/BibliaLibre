\hypertarget{section}{%
\section{1}\label{section}}

\bibverse{1} Jacobo, siervo de Dios y del Señor Jesucristo, á las doce
tribus que están esparcidas, salud. \bibverse{2} Hermanos míos, tened
por sumo gozo cuando cayereis en diversas tentaciones; \bibverse{3}
Sabiendo que la prueba de vuestra fe obra paciencia. \bibverse{4} Mas
tenga la paciencia perfecta su obra, para que seáis perfectos y cabales,
sin faltar en alguna cosa. \bibverse{5} Y si alguno de vosotros tiene
falta de sabiduría, demándela á Dios, el cual da á todos abundantemente,
y no zahiere; y le será dada. \bibverse{6} Pero pida en fe, no dudando
nada: porque el que duda es semejante á la onda de la mar, que es movida
del viento, y echada de una parte á otra. \bibverse{7} No piense pues el
tal hombre que recibirá ninguna cosa del Señor. \bibverse{8} El hombre
de doblado ánimo es inconstante en todos sus caminos. \bibverse{9} El
hermano que es de baja suerte, gloríese en su alteza: \bibverse{10} Mas
el que es rico, en su bajeza; porque él se pasará como la flor de la
hierba. \bibverse{11} Porque salido el sol con ardor, la hierba se secó,
y su flor se cayó, y pereció su hermosa apariencia: así también se
marchitará el rico en todos sus caminos. \bibverse{12} Bienaventurado el
varón que sufre la tentación; porque cuando fuere probado, recibirá la
corona de vida, que Dios ha prometido á los que le aman. \bibverse{13}
Cuando alguno es tentado, no diga que es tentado de Dios: porque Dios no
puede ser tentado de los malos, ni él tienta á alguno: \bibverse{14}
Sino que cada uno es tentado, cuando de su propia concupiscencia es
atraído, y cebado. \bibverse{15} Y la concupiscencia, después que ha
concebido, pare el pecado: y el pecado, siendo cumplido, engendra
muerte. \bibverse{16} Amados hermanos míos, no erréis. \bibverse{17}
Toda buena dádiva y todo don perfecto es de lo alto, que desciende del
Padre de las luces, en el cual no hay mudanza, ni sombra de variación.
\bibverse{18} El, de su voluntad nos ha engendrado por la palabra de
verdad, para que seamos primicias de sus criaturas. \bibverse{19} Por
esto, mis amados hermanos, todo hombre sea pronto para oir, tardío para
hablar, tardío para airarse: \bibverse{20} Porque la ira del hombre no
obra la justicia de Dios. \bibverse{21} Por lo cual, dejando toda
inmundicia y superfluidad de malicia, recibid con mansedumbre la palabra
ingerida, la cual puede hacer salvas vuestras almas. \bibverse{22} Mas
sed hacedores de la palabra, y no tan solamente oidores, engañándoos á
vosotros mismos. \bibverse{23} Porque si alguno oye la palabra, y no la
pone por obra, este tal es semejante al hombre que considera en un
espejo su rostro natural. \bibverse{24} Porque él se consideró á sí
mismo, y se fué, y luego se olvidó qué tal era. \bibverse{25} Mas el que
hubiere mirado atentamente en la perfecta ley, que es la de la libertad,
y perseverado en ella, no siendo oidor olvidadizo, sino hacedor de la
obra, este tal será bienaventurado en su hecho. \bibverse{26} Si alguno
piensa ser religioso entre vosotros, y no refrena su lengua, sino
engañando su corazón, la religión del tal es vana. \bibverse{27} La
religión pura y sin mácula delante de Dios y Padre es esta: Visitar los
huérfanos y las viudas en sus tribulaciones, y guardarse sin mancha de
este mundo.

\hypertarget{section-1}{%
\section{2}\label{section-1}}

\bibverse{1} Hermanos míos, no tengáis la fe de nuestro Señor Jesucristo
glorioso en acepción de personas. \bibverse{2} Porque si en vuestra
congregación entra un hombre con anillo de oro, y de preciosa ropa, y
también entra un pobre con vestidura vil, \bibverse{3} Y tuviereis
respeto al que trae la vestidura preciosa, y le dijereis: Siéntate tú
aquí en buen lugar: y dijereis al pobre: Estáte tú allí en pie; ó
siéntate aquí debajo de mi estrado: \bibverse{4} ¿No juzgáis en vosotros
mismos, y venís á ser jueces de pensamientos malos? \bibverse{5}
Hermanos míos amados, oid: ¿No ha elegido Dios los pobres de este mundo,
ricos en fe, y herederos del reino que ha prometido á los que le aman?
\bibverse{6} Mas vosotros habéis afrentado al pobre. ¿No os oprimen los
ricos, y no son ellos los mismos que os arrastran á los juzgados?
\bibverse{7} ¿No blasfeman ellos el buen nombre que fué invocado sobre
vosotros? \bibverse{8} Si en verdad cumplís vosotros la ley real,
conforme á la Escritura: Amarás á tu prójimo como á ti mismo, bien
hacéis: \bibverse{9} Mas si hacéis acepción de personas, cometéis
pecado, y sois reconvenidos de la ley como transgresores. \bibverse{10}
Porque cualquiera que hubiere guardado toda la ley, y ofendiere en un
punto, es hecho culpado de todos. \bibverse{11} Porque el que dijo: No
cometerás adulterio, también ha dicho: No matarás. Ahora bien, si no
hubieres cometido adulterio, pero hubieres matado, ya eres hecho
transgresor de la ley. \bibverse{12} Así hablad, y así obrad, como los
que habéis de ser juzgados por la ley de libertad. \bibverse{13} Porque
juicio sin misericordia será hecho con aquel que no hiciere
misericordia: y la misericordia se gloría contra el juicio.
\bibverse{14} Hermanos míos, ¿qué aprovechará si alguno dice que tiene
fe, y no tiene obras? ¿Podrá la fe salvarle? \bibverse{15} Y si el
hermano ó la hermana están desnudos, y tienen necesidad del
mantenimiento de cada día, \bibverse{16} Y alguno de vosotros les dice:
Id en paz, calentaos y hartaos; pero no les diereis las cosas que son
necesarias para el cuerpo: ¿qué aprovechará? \bibverse{17} Así también
la fe, si no tuviere obras, es muerta en sí misma. \bibverse{18} Pero
alguno dirá: Tú tienes fe, y yo tengo obras: muéstrame tu fe sin tus
obras, y yo te mostraré mi fe por mis obras. \bibverse{19} Tú crees que
Dios es uno; bien haces: también los demonios creen, y tiemblan.
\bibverse{20} ¿Mas quieres saber, hombre vano, que la fe sin obras es
muerta? \bibverse{21} ¿No fué justificado por las obras Abraham nuestro
padre, cuando ofreció á su hijo Isaac sobre el altar? \bibverse{22} ¿No
ves que la fe obró con sus obras, y que la fe fué perfecta por las
obras? \bibverse{23} Y fué cumplida la Escritura que dice: Abraham creyó
á Dios, y le fué imputado á justicia, y fué llamado amigo de Dios.
\bibverse{24} Vosotros veis, pues, que el hombre es justificado por las
obras, y no solamente por la fe. \bibverse{25} Asimismo también Rahab la
ramera, ¿no fué justificada por obras, cuando recibió los mensajeros, y
los echó fuera por otro camino? \bibverse{26} Porque como el cuerpo sin
espíritu está muerto, así también la fe sin obras es muerta.

\hypertarget{section-2}{%
\section{3}\label{section-2}}

\bibverse{1} Hermanos míos, no os hagáis muchos maestros, sabiendo que
recibiremos mayor condenación. \bibverse{2} Porque todos ofendemos en
muchas cosas. Si alguno no ofende en palabra, éste es varón perfecto,
que también puede con freno gobernar todo el cuerpo. \bibverse{3} He
aquí nosotros ponemos frenos en las bocas de los caballos para que nos
obedezcan, y gobernamos todo su cuerpo. \bibverse{4} Mirad también las
naves: aunque tan grandes, y llevadas de impetuosos vientos, son
gobernadas con un muy pequeño timón por donde quisiere el que las
gobierna. \bibverse{5} Así también, la lengua es un miembro pequeño, y
se gloría de grandes cosas. He aquí, un pequeño fuego ¡cuán grande
bosque enciende! \bibverse{6} Y la lengua es un fuego, un mundo de
maldad. Así la lengua está puesta entre nuestros miembros, la cual
contamina todo el cuerpo, é inflama la rueda de la creación, y es
inflamada del infierno. \bibverse{7} Porque toda naturaleza de bestias,
y de aves, y de serpientes, y de seres de la mar, se doma y es domada de
la naturaleza humana: \bibverse{8} Pero ningún hombre puede domar la
lengua, que es un mal que no puede ser refrenado; llena de veneno
mortal. \bibverse{9} Con ella bendecimos al Dios y Padre, y con ella
maldecimos á los hombres, los cuales son hechos á la semejanza de Dios.
\bibverse{10} De una misma boca proceden bendición y maldición. Hermanos
míos, no conviene que estas cosas sean así hechas. \bibverse{11} ¿Echa
alguna fuente por una misma abertura agua dulce y amarga? \bibverse{12}
Hermanos míos, ¿puede la higuera producir aceitunas, ó la vid higos? Así
ninguna fuente puede hacer agua salada y dulce. \bibverse{13} ¿Quién es
sabio y avisado entre vosotros? muestre por buena conversación sus obras
en mansedumbre de sabiduría. \bibverse{14} Pero si tenéis envidia amarga
y contención en vuestros corazones, no os gloriéis, ni seáis mentirosos
contra la verdad: \bibverse{15} Que esta sabiduría no es la que
desciende de lo alto, sino terrena, animal, diabólica. \bibverse{16}
Porque donde hay envidia y contención, allí hay perturbación y toda obra
perversa. \bibverse{17} Mas la sabiduría que es de lo alto, primeramente
es pura, después pacífica, modesta, benigna, llena de misericordia y de
buenos frutos, no juzgadora, no fingida. \bibverse{18} Y el fruto de
justicia se siembra en paz para aquellos que hacen paz.

\hypertarget{section-3}{%
\section{4}\label{section-3}}

\bibverse{1} ¿DE dónde vienen las guerras y los pleitos entre vosotros?
¿No son de vuestras concupiscencias, las cuales combaten en vuestros
miembros? \bibverse{2} Codiciáis, y no tenéis; matáis y ardéis de
envidia, y no podéis alcanzar; combatís y guerreáis, y no tenéis lo que
deseáis, porque no pedís. \bibverse{3} Pedís, y no recibís, porque pedís
mal, para gastar en vuestros deleites. \bibverse{4} Adúlteros y
adúlteras, ¿no sabéis que la amistad del mundo es enemistad con Dios?
Cualquiera pues que quisiere ser amigo del mundo, se constituye enemigo
de Dios. \bibverse{5} ¿Pensáis que la Escritura dice sin causa: El
espíritu que mora en nosotros codicia para envidia? \bibverse{6} Mas él
da mayor gracia. Por esto dice: Dios resiste á los soberbios, y da
gracia á los humildes. \bibverse{7} Someteos pues á Dios; resistid al
diablo, y de vosotros huirá. \bibverse{8} Allegaos á Dios, y él se
allegará á vosotros. Pecadores, limpiad las manos; y vosotros de doblado
ánimo, purificad los corazones. \bibverse{9} Afligíos, y lamentad, y
llorad. Vuestra risa se convierta en lloro, y vuestro gozo en tristeza.
\bibverse{10} Humillaos delante del Señor, y él os ensalzará.
\bibverse{11} Hermanos, no murmuréis los unos de los otros. El que
murmura del hermano, y juzga á su hermano, este tal murmura de la ley, y
juzga á la ley; pero si tú juzgas á la ley, no eres guardador de la ley,
sino juez. \bibverse{12} Uno es el dador de la ley, que puede salvar y
perder: ¿quién eres tú que juzgas á otro? \bibverse{13} Ea ahora, los
que decís: Hoy y mañana iremos á tal ciudad, y estaremos allá un año, y
compraremos mercadería, y ganaremos: \bibverse{14} Y no sabéis lo que
será mañana. Porque ¿qué es vuestra vida? Ciertamente es un vapor que se
aparece por un poco de tiempo, y luego se desvanece. \bibverse{15} En
lugar de lo cual deberíais decir: Si el Señor quisiere, y si viviéremos,
haremos esto ó aquello. \bibverse{16} Mas ahora os jactáis en vuestras
soberbias. Toda jactancia semejante es mala. \bibverse{17} El pecado,
pues, está en aquel que sabe hacer lo bueno, y no lo hace.

\hypertarget{section-4}{%
\section{5}\label{section-4}}

\bibverse{1} Ea ya ahora, oh ricos, llorad aullando por vuestras
miserias que os vendrán. \bibverse{2} Vuestras riquezas están podridas:
vuestras ropas están comidas de polilla. \bibverse{3} Vuestro oro y
plata están corrompidos de orín; y su orín os será en testimonio, y
comerá del todo vuestras carnes como fuego. Os habéis allegado tesoro
para en los postreros días. \bibverse{4} He aquí, el jornal de los
obreros que han segado vuestras tierras, el cual por engaño no les ha
sido pagado de vosotros, clama; y los clamores de los que habían segado,
han entrado en los oídos del Señor de los ejércitos. \bibverse{5} Habéis
vivido en deleites sobre la tierra, y sido disolutos; habéis cebado
vuestros corazones como en el día de sacrificios. \bibverse{6} Habéis
condenado y muerto al justo; y él no os resiste. \bibverse{7} Pues,
hermanos, tened paciencia hasta la venida del Señor. Mirad cómo el
labrador espera el precioso fruto de la tierra, aguardando con
paciencia, hasta que reciba la lluvia temprana y tardía. \bibverse{8}
Tened también vosotros paciencia; confirmad vuestros corazones: porque
la venida del Señor se acerca. \bibverse{9} Hermanos, no os quejéis unos
contra otros, porque no seáis condenados; he aquí, el juez está delante
de la puerta. \bibverse{10} Hermanos míos, tomad por ejemplo de
aflicción y de paciencia, á los profetas que hablaron en nombre del
Señor. \bibverse{11} He aquí, tenemos por bienaventurados á los que
sufren. Habéis oído la paciencia de Job, y habéis visto el fin del
Señor, que el Señor es muy misericordioso y piadoso. \bibverse{12} Mas
sobre todo, hermanos míos, no juréis, ni por el cielo, ni por la tierra,
ni por otro cualquier juramento; sino vuestro sí sea sí, y vuestro no
sea no; porque no caigáis en condenación. \bibverse{13} ¿Está alguno
entre vosotros afligido? haga oración. ¿Está alguno alegre? cante
salmos. \bibverse{14} ¿Está alguno enfermo entre vosotros? llame á los
ancianos de la iglesia, y oren por él, ungiéndole con aceite en el
nombre del Señor. \bibverse{15} Y la oración de fe salvará al enfermo, y
el Señor lo levantará; y si estuviere en pecados, le serán perdonados.
\bibverse{16} Confesaos vuestras faltas unos á otros, y rogad los unos
por los otros, para que seáis sanos; la oración del justo, obrando
eficazmente, puede mucho. \bibverse{17} Elías era hombre sujeto á
semejantes pasiones que nosotros, y rogó con oración que no lloviese, y
no llovió sobre la tierra en tres años y seis meses. \bibverse{18} Y
otra vez oró, y el cielo dió lluvia, y la tierra produjo su fruto.
\bibverse{19} Hermanos, si alguno de entre vosotros ha errado de la
verdad, y alguno le convirtiere, \bibverse{20} Sepa que el que hubiere
hecho convertir al pecador del error de su camino, salvará un alma de
muerte, y cubrirá multitud de pecados.
