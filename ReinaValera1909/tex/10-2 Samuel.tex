\hypertarget{section}{%
\section{1}\label{section}}

\bibverse{1} Y aconteció después de la muerte de Saúl, que vuelto David
de la derrota de los Amalecitas, estuvo dos días en Siclag: \bibverse{2}
Y al tercer día acaeció, que vino uno del campo de Saúl, rotos sus
vestidos, y tierra sobre su cabeza: y llegando á David, postróse en
tierra, é hizo reverencia. \bibverse{3} Y preguntóle David: ¿De dónde
vienes? Y él respondió: Heme escapado del campo de Israel. \bibverse{4}
Y David le dijo: ¿Qué ha acontecido? ruégote que me lo digas. Y él
respondió: El pueblo huyó de la batalla, y también muchos del pueblo
cayeron y son muertos: también Saúl y Jonathán su hijo murieron.
\bibverse{5} Y dijo David á aquel mancebo que le daba las nuevas: ¿Cómo
sabes que Saúl es muerto, y Jonathán su hijo? \bibverse{6} Y el mancebo
que le daba las nuevas respondió: Casualmente vine al monte de Gilboa, y
hallé á Saúl que estaba recostado sobre su lanza, y venían tras él
carros y gente de á caballo. \bibverse{7} Y como él miró atrás, vióme y
llamóme; y yo dije: Heme aquí. \bibverse{8} Y él me dijo: ¿Quién eres
tú? Y yo le respondí: Soy Amalecita. \bibverse{9} Y él me volvió á
decir: Yo te ruego que te pongas sobre mí, y me mates, porque me toman
angustias, y toda mi alma está aún en mí. \bibverse{10} Yo entonces
púseme sobre él, y matélo, porque sabía que no podía vivir después de su
caída: y tomé la corona que tenía en su cabeza, y la ajorca que traía en
su brazo, y helas traído acá á mi señor. \bibverse{11} Entonces David
trabando de sus vestidos, rompiólos; y lo mismo hicieron los hombres que
estaban con él. \bibverse{12} Y lloraron y lamentaron, y ayunaron hasta
la tarde, por Saúl y por Jonathán su hijo, y por el pueblo de Jehová, y
por la casa de Israel: porque habían caído á cuchillo. \bibverse{13} Y
David dijo á aquel mancebo que le había traído las nuevas: ¿De dónde
eres tú? Y él respondió: Yo soy hijo de un extranjero, Amalecita.
\bibverse{14} Y díjole David: ¿Cómo no tuviste temor de extender tu mano
para matar al ungido de Jehová? \bibverse{15} Entonces llamó David uno
de los mancebos, y díjole: Llega, y mátalo. Y él lo hirió, y murió.
\bibverse{16} Y David le dijo: Tu sangre sea sobre tu cabeza, pues que
tu boca atestiguó contra ti, diciendo: Yo maté al ungido de Jehová.
\bibverse{17} Y endechó David á Saúl y á Jonathán su hijo con esta
endecha. \bibverse{18} (Dijo también que enseñasen al arco á los hijos
de Judá. He aquí que está escrito en el libro del derecho:)
\bibverse{19} ¡Perecido ha la gloria de Israel sobre tus montañas! ¡Cómo
han caído los valientes! \bibverse{20} No lo denunciéis en Gath, no deis
las nuevas en las plazas de Ascalón; porque no se alegren las hijas de
los Filisteos, porque no salten de gozo las hijas de los incircuncisos.
\bibverse{21} Montes de Gilboa, ni rocío ni lluvia caiga sobre vosotros,
ni seáis tierras de ofrendas; porque allí fué desechado el escudo de los
valientes, el escudo de Saúl, como si no hubiera sido ungido con aceite.
\bibverse{22} Sin sangre de muertos, sin grosura de valientes, el arco
de Jonathán nunca volvió, ni la espada de Saúl se tornó vacía.
\bibverse{23} Saúl y Jonathán, amados y queridos en su vida, en su
muerte tampoco fueron apartados: más ligeros que águilas, más fuertes
que leones. \bibverse{24} Hijas de Israel, llorad sobre Saúl, que os
vestía de escarlata en regocijos, que adornaba vuestras ropas con
ornamentos de oro. \bibverse{25} ¡Cómo han caído los valientes en medio
de la batalla! ¡Jonathán, muerto en tus alturas! \bibverse{26} Angustia
tengo por ti, hermano mío Jonathán, que me fuiste muy dulce: más
maravilloso me fué tu amor, que el amor de las mujeres. \bibverse{27}
¡Cómo han caído los valientes, y perecieron las armas de guerra!

\hypertarget{section-1}{%
\section{2}\label{section-1}}

\bibverse{1} Después de esto aconteció que David consultó á Jehová,
diciendo: ¿Subiré á alguna de las ciudades de Judá? Y Jehová le
respondió: Sube. Y David tornó á decir: ¿A dónde subiré? Y él le dijo: A
Hebrón. \bibverse{2} Y David subió allá, y con él sus dos mujeres,
Ahinoam Jezreelita y Abigail, la que fué mujer de Nabal del Carmelo.
\bibverse{3} Y llevó también David consigo los hombres que con él habían
estado, cada uno con su familia; los cuales moraron en las ciudades de
Hebrón. \bibverse{4} Y vinieron los varones de Judá, y ungieron allí á
David por rey sobre la casa de Judá. Y dieron aviso á David, diciendo:
Los de Jabes de Galaad son los que sepultaron á Saúl. \bibverse{5} Y
envió David mensajeros á los de Jabes de Galaad, diciéndoles: Benditos
seáis vosotros de Jehová, que habéis hecho esta misericordia con vuestro
señor Saúl en haberle dado sepultura. \bibverse{6} Ahora pues, Jehová
haga con vosotros misericordia y verdad; y yo también os haré bien por
esto que habéis hecho. \bibverse{7} Esfuércense pues ahora vuestras
manos, y sed valientes; pues que muerto Saúl vuestro señor, los de la
casa de Judá me han ungido por rey sobre ellos. \bibverse{8} Mas Abner
hijo de Ner, general de ejército de Saúl, tomó á Is-boseth hijo de Saúl,
é hízolo pasar al real: \bibverse{9} Y alzólo por rey sobre Galaad, y
sobre Gessuri, y sobre Jezreel, y sobre Ephraim, y sobre Benjamín, y
sobre todo Israel. \bibverse{10} De cuarenta años era Is-boseth hijo de
Saúl, cuando comenzó á reinar sobre Israel; y reinó dos años. Sola la
casa de Judá seguía á David. \bibverse{11} Y fué el número de los días
que David reinó en Hebrón sobre la casa de Judá, siete años y seis
meses. \bibverse{12} Y Abner hijo de Ner salió de Mahanaim á Gabaón con
los siervos de Is-boseth hijo de Saúl. \bibverse{13} Y Joab hijo de
Sarvia, y los siervos de David, salieron y encontráronlos junto al
estanque de Gabaón: y como se juntaron, paráronse los unos de la una
parte del estanque, y los otros de la otra. \bibverse{14} Y dijo Abner á
Joab: Levántense ahora los mancebos, y maniobren delante de nosotros. Y
Joab respondió: Levántense. \bibverse{15} Entonces se levantaron, y en
número de doce, pasaron de Benjamín de la parte de Is-boseth hijo de
Saúl; y doce de los siervos de David. \bibverse{16} Y cada uno echó mano
de la cabeza de su compañero, y metióle su espada por el costado,
cayendo así á una; por lo que fué llamado aquel lugar, Helcath-assurim,
el cual está en Gabaón. \bibverse{17} Y hubo aquel día una batalla muy
recia, y Abner y los hombres de Israel fueron vencidos de los siervos de
David. \bibverse{18} Y estaban allí los tres hijos de Sarvia: Joab, y
Abisai, y Asael. Este Asael era suelto de pies como un corzo del campo.
\bibverse{19} El cual Asael siguió á Abner, yendo tras de él sin
apartarse á diestra ni á siniestra. \bibverse{20} Y Abner miró atrás, y
dijo: ¿No eres tú Asael? Y él respondió: Sí. \bibverse{21} Entonces
Abner le dijo: Apártate á la derecha ó á la izquierda, y agárrate alguno
de los mancebos, y toma para ti sus despojos. Pero Asael no quiso
apartarse de en pos de él. \bibverse{22} Y Abner tornó á decir á Asael:
Apártate de en pos de mí, porque te heriré derribándote en tierra, y
después ¿cómo levantaré mi rostro á tu hermano Joab? \bibverse{23} Y no
queriendo él irse, hiriólo Abner con el regatón de la lanza por la
quinta costilla, y salióle la lanza por las espaldas, y cayó allí, y
murió en aquel mismo sitio. Y todos los que venían por aquel lugar donde
Asael había caído y estaba muerto, se paraban. \bibverse{24} Mas Joab y
Abisai siguieron á Abner; y púsoseles el sol cuando llegaron al collado
de Amma, que está delante de Gía, junto al camino del desierto de
Gabaón. \bibverse{25} Y juntáronse los hijos de Benjamín en un escuadrón
con Abner, y paráronse en la cumbre del collado. \bibverse{26} Y Abner
dió voces á Joab, diciendo: ¿Consumirá la espada perpetuamente? ¿no
sabes tú que al cabo se sigue amargura? ¿hasta cuándo no has de decir al
pueblo que se vuelvan de seguir á sus hermanos? \bibverse{27} Y Joab
respondió: Vive Dios que si no hubieras hablado, ya desde esta mañana el
pueblo hubiera dejado de seguir á sus hermanos. \bibverse{28} Entonces
Joab tocó el cuerno, y todo el pueblo se detuvo, y no siguió más á los
de Israel, ni peleó más. \bibverse{29} Y Abner y los suyos caminaron por
la campiña toda aquella noche, y pasando el Jordán cruzaron por todo
Bitrón, y llegaron á Mahanaim. \bibverse{30} Joab también volvió de
seguir á Abner, y juntando todo el pueblo, faltaron de los siervos de
David diecinueve hombres, y Asael. \bibverse{31} Mas los siervos de
David hirieron de los de Benjamín y de los de Abner, trescientos y
sesenta hombres, que murieron. Tomaron luego á Asael, y sepultáronlo en
el sepulcro de su padre en Beth-lehem. \bibverse{32} Y caminaron toda
aquella noche Joab y los suyos, y amanecióles en Hebrón.

\hypertarget{section-2}{%
\section{3}\label{section-2}}

\bibverse{1} Y hubo larga guerra entre la casa de Saúl y la casa de
David; mas David se iba fortificando, y la casa de Saúl iba en
disminución. \bibverse{2} Y nacieron hijos á David en Hebrón: su
primogénito fué Ammón, de Ahinoam Jezreelita; \bibverse{3} Su segundo
Chileab, de Abigail la mujer de Nabal, el del Carmelo; el tercero,
Absalóm, hijo de Maachâ, hija de Talmai rey de Gessur: \bibverse{4} El
cuarto, Adonías hijo de Haggith; el quinto, Saphatías hijo de Abital;
\bibverse{5} El sexto, Jetream, de Egla mujer de David. Estos nacieron á
David en Hebrón. \bibverse{6} Y como había guerra entre la casa de Saúl
y la de David, aconteció que Abner se esforzaba por la casa de Saúl.
\bibverse{7} Y había Saúl tenido una concubina que se llamaba Rispa,
hija de Aja. Y dijo Is-boseth á Abner: ¿Por qué has entrado á la
concubina de mi padre? \bibverse{8} Y enojóse Abner en gran manera por
las palabras de Is-boseth, y dijo: ¿Soy yo cabeza de perros respecto de
Judá? Yo he hecho hoy misericordia con la casa de Saúl tu padre, con sus
hermanos, y con sus amigos, y no te he entregado en las manos de David:
¿y tú me haces hoy cargo del pecado de esta mujer? \bibverse{9} Así haga
Dios á Abner y así le añada, si como ha jurado Jehová á David no hiciere
yo así con él, \bibverse{10} Trasladando el reino de la casa de Saúl, y
confirmando el trono de David sobre Israel y sobre Judá, desde Dan hasta
Beer-sebah. \bibverse{11} Y él no pudo responder palabra á Abner, porque
le temía. \bibverse{12} Y envió Abner mensajeros á David de su parte,
diciendo: ¿Cúya es la tierra? Y que le dijesen: Haz alianza conmigo, y
he aquí que mi mano será contigo para volver á ti á todo Israel.
\bibverse{13} Y David dijo: Bien; yo haré contigo alianza: mas una cosa
te pido, y es, que no me vengas á ver sin que primero traigas á Michâl
la hija de Saúl, cuando vinieres á verme. \bibverse{14} Después de esto
envió David mensajeros á Is-boseth hijo de Saúl, diciendo: Restitúyeme á
mi mujer Michâl, la cual yo desposé conmigo por cien prepucios de
Filisteos. \bibverse{15} Entonces Is-boseth envió, y quitóla á su marido
Paltiel, hijo de Lais. \bibverse{16} Y su marido fué con ella,
siguiéndola y llorando hasta Bahurim. Y díjole Abner: Anda, vuélvete.
Entonces él se volvió. \bibverse{17} Y habló Abner con los ancianos de
Israel, diciendo: Ayer y antes procurabais que David fuese rey sobre
vosotros; \bibverse{18} Ahora, pues, hacedlo; porque Jehová ha hablado á
David, diciendo: Por la mano de mi siervo David libraré á mi pueblo
Israel de mano de los Filisteos, y de mano de todos sus enemigos.
\bibverse{19} Y habló también Abner á los de Benjamín: y fué también
Abner á Hebrón á decir á David todo el parecer de los de Israel y de
toda la casa de Benjamín. \bibverse{20} Vino pues Abner á David en
Hebrón, y con él veinte hombres: y David hizo banquete á Abner y á los
que con él habían venido. \bibverse{21} Y dijo Abner á David: Yo me
levantaré é iré, y juntaré á mi señor el rey á todo Israel, para que
hagan contigo alianza, y tú reines como deseas. David despidió luego á
Abner, y él se fué en paz. \bibverse{22} Y he aquí los siervos de David
y Joab, que venían del campo, y traían consigo gran presa. Mas Abner no
estaba con David en Hebrón, que ya lo había él despedido, y él se había
ido en paz. \bibverse{23} Y luego que llegó Joab y todo el ejército que
con él estaba, fué dado aviso á Joab, diciendo: Abner hijo de Ner ha
venido al rey, y él le ha despedido, y se fué en paz. \bibverse{24}
Entonces Joab vino al rey, y díjole: ¿Qué has hecho? He aquí habíase
venido Abner á ti: ¿por qué pues lo dejaste que se fuése? \bibverse{25}
¿Sabes tú que Abner hijo de Ner ha venido para engañarte, y á saber tu
salida y tu entrada, y por entender todo lo que tú haces? \bibverse{26}
Y saliéndose Joab de con David, envió mensajeros tras Abner, los cuales
le volvieron desde el pozo de Sira, sin saberlo David. \bibverse{27} Y
como Abner volvió á Hebrón, apartólo Joab al medio de la puerta,
hablando con él blandamente, y allí le hirió por la quinta costilla, á
causa de la muerte de Asael su hermano, y murió. \bibverse{28} Cuando
David supo después esto, dijo: Limpio estoy yo y mi reino, por Jehová,
para siempre, de la sangre de Abner hijo de Ner. \bibverse{29} Caiga
sobre la cabeza de Joab, y sobre toda la casa de su padre; que nunca
falte de la casa de Joab quien padezca flujo, ni leproso, ni quien ande
con báculo, ni quien muera á cuchillo, ni quien tenga falta de pan.
\bibverse{30} Joab pues y Abisai su hermano mataron á Abner, porque él
había muerto á Asael, hermano de ellos en la batalla de Gabaón.
\bibverse{31} Entonces dijo David á Joab, y á todo el pueblo que con él
estaba: Romped vuestros vestidos, y ceñíos de sacos, y haced duelo
delante de Abner. Y el rey iba detrás del féretro. \bibverse{32} Y
sepultaron á Abner en Hebrón: y alzando el rey su voz, lloró junto al
sepulcro de Abner; y lloró también todo el pueblo. \bibverse{33} Y
endechando el rey al mismo Abner, decía: ¿Murió Abner como muere un
villano? \bibverse{34} Tus manos no estaban atadas, ni tus pies ligados
con grillos: caíste como los que caen delante de malos hombres. Y todo
el pueblo volvió á llorar sobre él. \bibverse{35} Y como todo el pueblo
viniese á dar de comer pan á David siendo aún de día, David juró,
diciendo: Así me haga Dios y así me añada, si antes que se ponga el sol
gustare yo pan, ú otra cualquier cosa. \bibverse{36} Súpolo así todo el
pueblo, y plugo en sus ojos; porque todo lo que el rey hacía parecía
bien en ojos de todo el pueblo. \bibverse{37} Y todo el pueblo y todo
Israel entendieron aquel día, que no había venido del rey que Abner hijo
de Ner muriese. \bibverse{38} Y el rey dijo á sus siervos: ¿No sabéis
que ha caído hoy en Israel un príncipe, y grande? \bibverse{39} Que yo
ahora aun soy tierno rey ungido; y estos hombres, los hijos de Sarvia,
muy duros me son: Jehová dé el pago al que mal hace, conforme á su
malicia.

\hypertarget{section-3}{%
\section{4}\label{section-3}}

\bibverse{1} Luego que oyó el hijo de Saúl que Abner había sido muerto
en Hebrón, las manos se le descoyuntaron, y fué atemorizado todo Israel.
\bibverse{2} Y tenía el hijo de Saúl dos varones, los cuales eran
capitanes de compañía, el nombre de uno era Baana, y el del otro Rechâb,
hijos de Rimmón Beerothita, de los hijos de Benjamín: (porque Beeroth
era contada con Benjamín; \bibverse{3} Estos Beerothitas se habían huído
á Gittaim, y habían sido peregrinos allí hasta entonces.) \bibverse{4} Y
Jonathán, hijo de Saúl, tenía un hijo lisiado de los pies de edad de
cinco años: que cuando la noticia de la muerte de Saúl y de Jonathán
vino de Jezreel, tomóle su ama y huyó; y como iba huyendo con celeridad,
cayó el niño y quedó cojo. Su nombre era Mephi-boseth. \bibverse{5} Los
hijos pues de Rimmón Beerothita, Rechâb y Baana, fueron y entraron en el
mayor calor del día en casa de Is-boseth, el cual estaba durmiendo en su
cámara la siesta. \bibverse{6} Entonces entraron ellos en medio de la
casa en hábito de mercaderes de grano, y le hirieron en la quinta
costilla. Escapáronse luego Rechâb y Baana su hermano; \bibverse{7} Pues
como entraron en la casa, estando él en su cama en su cámara de dormir,
lo hirieron y mataron, y cortáronle la cabeza, y habiéndola tomado,
caminaron toda la noche por el camino de la campiña. \bibverse{8} Y
trajeron la cabeza de Is-boseth á David en Hebrón, y dijeron al rey: He
aquí la cabeza de Is-boseth hijo de Saúl tu enemigo, que procuraba
matarte; y Jehová ha vengado hoy á mi señor el rey, de Saúl y de su
simiente. \bibverse{9} Y David respondió á Rechâb y á su hermano Baana,
hijos de Rimmón Beerothita, y díjoles: Vive Jehová que ha redimido mi
alma de toda angustia, \bibverse{10} Que cuando uno me dió nuevas,
diciendo, He aquí Saúl es muerto imaginándose que traía buenas nuevas,
yo lo prendí, y le maté en Siclag en pago de la nueva. \bibverse{11}
¿Cuánto más á los malos hombres que mataron á un hombre justo en su
casa, y sobre su cama? Ahora pues, ¿no tengo yo de demandar su sangre de
vuestras manos, y quitaros de la tierra? \bibverse{12} Entonces David
mandó á los mancebos, y ellos los mataron, y cortáronles las manos y los
pies, y colgáronlos sobre el estanque, en Hebrón. Luego tomaron la
cabeza de Is-boseth, y enterráronla en el sepulcro de Abner en Hebrón.

\hypertarget{section-4}{%
\section{5}\label{section-4}}

\bibverse{1} Y vinieron todas las tribus de Israel á David en Hebrón, y
hablaron, diciendo: He aquí nosotros somos tus huesos y tu carne.
\bibverse{2} Y aun ayer y antes, cuando Saúl reinaba sobre nosotros, tú
sacabas y volvías á Israel. Además Jehová te ha dicho: Tú apacentarás á
mi pueblo Israel, y tú serás sobre Israel príncipe. \bibverse{3}
Vinieron pues todos los ancianos de Israel al rey en Hebrón, y el rey
David hizo con ellos alianza en Hebrón delante de Jehová; y ungieron á
David por rey sobre Israel. \bibverse{4} Era David de treinta años
cuando comenzó á reinar, y reinó cuarenta años. \bibverse{5} En Hebrón
reinó sobre Judá siete años y seis meses: y en Jerusalem reinó treinta y
tres años sobre todo Israel y Judá. \bibverse{6} Entonces el rey y los
suyos fueron á Jerusalem al Jebuseo que habitaba en la tierra; el cual
habló á David, diciendo: Tú no entrarás acá, si no echares los ciegos y
los cojos; diciendo: No entrará acá David. \bibverse{7} Empero David
tomó la fortaleza de Sión, la cual es la ciudad de David. \bibverse{8} Y
dijo David aquel día: ¿Quién llegará hasta las canales, y herirá al
Jebuseo, y á los cojos y ciegos, á los cuales el alma de David aborrece?
Por esto se dijo: Ciego ni cojo no entrará en casa. \bibverse{9} Y David
moró en la fortaleza, y púsole por nombre la Ciudad de David: y edificó
alrededor, desde Millo para adentro. \bibverse{10} Y David iba creciendo
y aumentándose, y Jehová Dios de los ejércitos era con él. \bibverse{11}
E Hiram rey de Tiro envió también embajadores á David, y madera de
cedro, y carpinteros, y canteros para los muros, los cuales edificaron
la casa de David. \bibverse{12} Y entendió David que Jehová le había
confirmado por rey sobre Israel, y que había ensalzado su reino por amor
de su pueblo Israel. \bibverse{13} Y tomó David más concubinas y mujeres
de Jerusalem después que vino de Hebrón, y naciéronle más hijos é hijas.
\bibverse{14} Estos son los nombres de los que le nacieron en Jerusalem:
Sammua, y Sobab, y Nathán, y Salomón, \bibverse{15} E Ibhar, y Elisua, y
Nepheg, \bibverse{16} Y Japhia, y Elisama, y Eliada, y Eliphelet.
\bibverse{17} Y oyendo los Filisteos que habían ungido á David por rey
sobre Israel, todos los Filisteos subieron á buscar á David: lo cual
como David oyó, vino á la fortaleza. \bibverse{18} Y vinieron los
Filisteos, y extendiéronse por el valle de Raphaim. \bibverse{19}
Entonces consultó David á Jehová, diciendo: ¿Iré contra los Filisteos?
¿los entregarás en mis manos? Y Jehová respondió á David: Ve, porque
ciertamente entregaré los Filisteos en tus manos. \bibverse{20} Y vino
David á Baal-perasim, y allí los venció David, y dijo: Rompió Jehová mis
enemigos delante de mí, como quien rompe aguas. Y por esto llamó el
nombre de aquel lugar Baal-perasim. \bibverse{21} Y dejaron allí sus
ídolos, los cuales quemó David y los suyos. \bibverse{22} Y los
Filisteos tornaron á venir, y extendiéronse en el valle de Raphaim.
\bibverse{23} Y consultando David á Jehová, él le respondió: No subas;
mas rodéalos, y vendrás á ellos por delante de los morales:
\bibverse{24} Y cuando oyeres un estruendo que irá por las copas de los
morales, entonces te moverás; porque Jehová saldrá delante de ti á herir
el campo de los Filisteos. \bibverse{25} Y David lo hizo así, como
Jehová se lo había mandado; é hirió á los Filisteos desde Gabaa hasta
llegar á Gaza.

\hypertarget{section-5}{%
\section{6}\label{section-5}}

\bibverse{1} Y david tornó á juntar todos los escogidos de Israel,
treinta mil. \bibverse{2} Y levantóse David, y fué con todo el pueblo
que tenía consigo, de Baal de Judá, para hacer pasar de allí el arca de
Dios, sobre la cual era invocado el nombre de Jehová de los ejércitos,
que mora en ella entre los querubines. \bibverse{3} Y pusieron el arca
de Dios sobre un carro nuevo, y lleváronla de la casa de Abinadab, que
estaba en Gabaa: y Uzza y Ahio, hijos de Abinadab, guiaban el carro
nuevo. \bibverse{4} Y cuando lo llevaban de la casa de Abinadab que
estaba en Gabaa, con el arca de Dios, Ahio iba delante del arca.
\bibverse{5} Y David y toda la casa de Israel danzaban delante de Jehová
con toda suerte de instrumentos de madera de haya; con arpas, salterios,
adufes, flautas y címbalos. \bibverse{6} Y cuando llegaron á la era de
Nachôn, Uzza extendió la mano al arca de Dios, y túvola; porque los
bueyes daban sacudidas. \bibverse{7} Y el furor de Jehová se encendió
contra Uzza, é hiriólo allí Dios por aquella temeridad, y cayó allí
muerto junto al arca de Dios. \bibverse{8} Y entristecióse David por
haber herido Jehová á Uzza: y fué llamado aquel lugar Pérez-uzza, hasta
hoy. \bibverse{9} Y temiendo David á Jehová aquel día, dijo: ¿Cómo ha de
venir á mí el arca de Jehová? \bibverse{10} No quiso pues David traer á
sí el arca de Jehová á la ciudad de David; mas llevóla David á casa de
Obed-edom Getheo. \bibverse{11} Y estuvo el arca de Jehová en casa de
Obed-edom Getheo tres meses: y bendijo Jehová á Obed-edom y á toda su
casa. \bibverse{12} Y fué dado aviso al rey David, diciendo: Jehová ha
bendecido la casa de Obed-edom, y todo lo que tiene, á causa del arca de
Dios. Entonces David fué, y trajo el arca de Dios de casa de Obed-edom á
la ciudad de David con alegría. \bibverse{13} Y como los que llevaban el
arca de Dios habían andado seis pasos, sacrificaban un buey y un carnero
grueso. \bibverse{14} Y David saltaba con toda su fuerza delante de
Jehová; y tenía vestido David un ephod de lino. \bibverse{15} Así David
y toda la casa de Israel llevaban el arca de Jehová con júbilo y sonido
de trompeta. \bibverse{16} Y como el arca de Jehová llegó á la ciudad de
David, aconteció que Michâl hija de Saúl miró desde una ventana, y vió
al rey David que saltaba con toda su fuerza delante de Jehová: y
menosprecióle en su corazón. \bibverse{17} Metieron pues el arca de
Jehová, y pusiéronla en su lugar en medio de una tienda que David le
había tendido: y sacrificó David holocaustos y pacíficos delante de
Jehová. \bibverse{18} Y como David hubo acabado de ofrecer los
holocaustos y pacíficos, bendijo al pueblo en el nombre de Jehová de los
ejércitos. \bibverse{19} Y repartió á todo el pueblo, y á toda la
multitud de Israel, así á hombres como á mujeres, á cada uno una torta
de pan, y un pedazo de carne, y un frasco de vino. Y fuése todo el
pueblo, cada uno á su casa. \bibverse{20} Volvió luego David para
bendecir su casa: y saliendo Michâl á recibir á David, dijo: ¡Cuán
honrado ha sido hoy el rey de Israel, desnudándose hoy delante de las
criadas de sus siervos, como se desnudara un juglar! \bibverse{21}
Entonces David respondió á Michâl: Delante de Jehová, que me eligió más
bien que á tu padre y á toda su casa, mandándome que fuese príncipe
sobre el pueblo de Jehová, sobre Israel, danzaré delante de Jehová.
\bibverse{22} Y aun me haré más vil que esta vez, y seré bajo á mis
propios ojos; y delante de las criadas que dijiste, delante de ellas
seré honrado. \bibverse{23} Y Michâl hija de Saúl nunca tuvo hijos hasta
el día de su muerte.

\hypertarget{section-6}{%
\section{7}\label{section-6}}

\bibverse{1} Y aconteció que, estando ya el rey asentado en su casa,
después que Jehová le había dado reposo de todos sus enemigos en
derredor, \bibverse{2} Dijo el rey al profeta Nathán: Mira ahora, yo
moro en edificios de cedro, y el arca de Dios está entre cortinas.
\bibverse{3} Y Nathán dijo al rey: Anda, y haz todo lo que está en tu
corazón, que Jehová es contigo. \bibverse{4} Y aconteció aquella noche,
que fué palabra de Jehová á Nathán, diciendo: \bibverse{5} Ve y di á mi
siervo David: Así ha dicho Jehová: ¿Tú me has de edificar casa en que yo
more? \bibverse{6} Ciertamente no he habitado en casas desde el día que
saqué á los hijos de Israel de Egipto hasta hoy, sino que anduve en
tienda y en tabernáculo. \bibverse{7} Y en todo cuanto he andado con
todos los hijos de Israel, ¿he hablado palabra en alguna de las tribus
de Israel, á quien haya mandado que apaciente mi pueblo de Israel, para
decir: ¿Por qué no me habéis edificado casa de cedros? \bibverse{8}
Ahora pues, dirás así á mi siervo David: Así ha dicho Jehová de los
ejércitos: Yo te tomé de la majada, de detrás de las ovejas, para que
fueses príncipe sobre mi pueblo, sobre Israel; \bibverse{9} Y he sido
contigo en todo cuanto has andado, y delante de ti he talado todos tus
enemigos, y te he hecho nombre grande, como el nombre de los grandes que
son en la tierra. \bibverse{10} Además yo fijaré lugar á mi pueblo
Israel, yo lo plantaré, para que habite en su lugar, y nunca más sea
removido, ni los inicuos le aflijan más, como antes, \bibverse{11} Desde
el día que puse jueces sobre mi pueblo Israel; y yo te daré descanso de
todos tus enemigos. Asimismo Jehová te hace saber, que él te quiere
hacer casa. \bibverse{12} Y cuando tus días fueren cumplidos, y
durmieres con tus padres, yo estableceré tu simiente después de ti, la
cual procederá de tus entrañas, y aseguraré su reino. \bibverse{13} El
edificará casa á mi nombre, y yo afirmaré para siempre el trono de su
reino. \bibverse{14} Yo le seré á él padre, y él me será á mí hijo. Y si
él hiciere mal, yo le castigaré con vara de hombres, y con azotes de
hijos de hombres; \bibverse{15} Empero mi misericordia no se apartará de
él, como la aparté de Saúl, al cual quité de delante de ti.
\bibverse{16} Y será afirmada tu casa y tu reino para siempre delante de
tu rostro; y tu trono será estable eternalmente. \bibverse{17} Conforme
á todas estas palabras, y conforme á toda esta visión, así habló Nathán
á David. \bibverse{18} Y entró el rey David, y púsose delante de Jehová,
y dijo: Señor Jehová, ¿quién soy yo, y qué es mi casa, para que tú me
traigas hasta aquí? \bibverse{19} Y aun te ha parecido poco esto, Señor
Jehová, pues que también has hablado de la casa de tu siervo en lo por
venir. ¿Es ése el modo de obrar del hombre, Señor Jehová? \bibverse{20}
¿Y qué más puede añadir David hablando contigo? Tú pues conoces tu
siervo, Señor Jehová. \bibverse{21} Todas estas grandezas has obrado por
tu palabra y conforme á tu corazón, haciéndolas saber á tu siervo.
\bibverse{22} Por tanto tú te has engrandecido, Jehová Dios: por cuanto
no hay como tú, ni hay Dios fuera de ti, conforme á todo lo que hemos
oído con nuestros oídos. \bibverse{23} ¿Y quién como tu pueblo, como
Israel, en la tierra? una gente por amor de la cual Dios fuese á
redimírsela por pueblo, y le pusiese nombre, é hiciese por vosotros, oh
Israel, grandes y espantosas obras en tu tierra, por amor de tu pueblo,
oh Dios, que tú redimiste de Egipto, de las gentes y de sus dioses?
\bibverse{24} Porque tú te has confirmado á tu pueblo Israel por pueblo
tuyo para siempre: y tú, oh Jehová, fuiste á ellos por Dios.
\bibverse{25} Ahora pues, Jehová Dios, la palabra que has hablado sobre
tu siervo y sobre su casa, despiértala para siempre, y haz conforme á lo
que has dicho. \bibverse{26} Que sea engrandecido tu nombre para
siempre, y dígase: Jehová de los ejércitos es Dios sobre Israel; y que
la casa de tu siervo David sea firme delante de ti. \bibverse{27} Porque
tú, Jehová de los ejércitos, Dios de Israel, revelaste al oído de tu
siervo, diciendo: Yo te edificaré casa. Por esto tu siervo ha hallado en
su corazón para hacer delante de ti esta súplica. \bibverse{28} Ahora
pues, Jehová Dios, tú eres Dios, y tus palabras serán firmes, ya que has
dicho á tu siervo este bien. \bibverse{29} Tenlo pues ahora á bien, y
bendice la casa de tu siervo, para que perpetuamente permanezca delante
de ti: pues que tú, Jehová Dios, lo has dicho, y con tu bendición será
bendita la casa de tu siervo para siempre.

\hypertarget{section-7}{%
\section{8}\label{section-7}}

\bibverse{1} Después de esto aconteció, que David hirió á los Filisteos,
y los humilló: y tomó David á Methegamma de mano de los Filisteos.
\bibverse{2} Hirió también á los de Moab, y midiólos con cordel,
haciéndolos echar por tierra; y midió con dos cordeles para muerte, y un
cordel entero para vida; y fueron los Moabitas siervos debajo de
tributo. \bibverse{3} Asimismo hirió David á Hadad-ezer hijo de Rehob,
rey de Soba, yendo él á extender su término hasta el río de Eufrates.
\bibverse{4} Y tomó David de ellos mil y setecientos de á caballo, y
veinte mil hombres de á pie; y desjarretó David los caballos de todos
los carros, excepto cien carros de ellos que dejó. \bibverse{5} Y
vinieron los Siros de Damasco á dar ayuda á Hadad-ezer rey de Soba; y
David hirió de los Siros veinte y dos mil hombres. \bibverse{6} Puso
luego David guarnición en Siria la de Damasco, y fueron los Siros
siervos de David sujetos á tributo. Y Jehová guardó á David donde quiera
que fué. \bibverse{7} Y tomó David los escudos de oro que traían los
siervos de Hadad-ezer, y llevólos á Jerusalem. \bibverse{8} Asimismo de
Beta y de Beeroth, ciudades de Hadad-ezer, tomó el rey David gran copia
de metal. \bibverse{9} Entonces oyendo Toi, rey de Hamath, que David
había herido todo el ejército de Hadad-ezer, \bibverse{10} Envió Toi á
Joram su hijo al rey David, á saludarle pacíficamente y á bendecirle,
porque había peleado con Hadad-ezer y lo había vencido: porque Toi era
enemigo de Hadad-ezer. Y Joram llevaba en su mano vasos de plata, y
vasos de oro, y de metal; \bibverse{11} Los cuales el rey David dedicó á
Jehová, con la plata y el oro que tenía dedicado de todas las naciones
que había sometido: \bibverse{12} De los Siros, de los Moabitas, de los
Ammonitas, de los Filisteos, de los Amalecitas, y del despojo de
Hadad-ezer hijo de Rehob, rey de Soba. \bibverse{13} Y ganó David fama
cuando, volviendo de la rota de los Siros, hirió diez y ocho mil hombres
en el valle de la sal. \bibverse{14} Y puso guarnición en Edom, por toda
Edom puso guarnición; y todos los Idumeos fueron siervos de David. Y
Jehová guardó á David por donde quiera que fué. \bibverse{15} Y reinó
David sobre todo Israel; y hacía David derecho y justicia á todo su
pueblo. \bibverse{16} Y Joab hijo de Sarvia era general de su ejército;
y Josaphat hijo de Ahilud, canciller; \bibverse{17} Y Sadoc hijo de
Ahitud, y Ahimelech hijo de Abiathar, eran sacerdotes; y Seraía era
escriba; \bibverse{18} Y Benahía hijo de Joiada, era sobre los Ceretheos
y Peletheos; y los hijos de David eran los príncipes.

\hypertarget{section-8}{%
\section{9}\label{section-8}}

\bibverse{1} Y dijo David: ¿Ha quedado alguno de la casa de Saúl, á
quien haga yo misericordia por amor de Jonathán? \bibverse{2} Y había un
siervo de la casa de Saúl, que se llamaba Siba, al cual como llamaron
que viniese á David, el rey le dijo: ¿Eres tú Siba? Y él respondió: Tu
siervo. \bibverse{3} Y el rey dijo: ¿No ha quedado nadie de la casa de
Saúl, á quien haga yo misericordia de Dios? Y Siba respondió al rey: Aun
ha quedado un hijo de Jonathán, lisiado de los pies. \bibverse{4}
Entonces el rey le dijo: ¿Y ése dónde está? Y Siba respondió al rey: He
aquí, está en casa de Machîr hijo de Amiel, en Lodebar. \bibverse{5} Y
envió el rey David, y tomólo de casa de Machîr hijo de Amiel, de
Lodebar. \bibverse{6} Y venido Mephi-boseth, hijo de Jonathán hijo de
Saúl, á David, postróse sobre su rostro, é hizo reverencia. Y dijo
David: Mephi-boseth. Y él respondió: He aquí tu siervo. \bibverse{7} Y
díjole David: No tengas temor, porque yo á la verdad haré contigo
misericordia por amor de Jonathán tu padre, y te haré volver todas las
tierras de Saúl tu padre; y tú comerás siempre pan á mi mesa.
\bibverse{8} Y él inclinándose, dijo: ¿Quién es tu siervo, para que
mires á un perro muerto como yo? \bibverse{9} Entonces el rey llamó á
Siba, siervo de Saúl, y díjole: Todo lo que fué de Saúl y de toda su
casa, yo lo he dado al hijo de tu señor. \bibverse{10} Tú pues le
labrarás las tierras, tú con tus hijos, y tus siervos, y encerrarás los
frutos, para que el hijo de tu señor tenga con qué mantenerse; y
Mephi-boseth el hijo de tu señor comerá siempre pan á mi mesa. Y tenía
Siba quince hijos y veinte siervos. \bibverse{11} Y respondió Siba al
rey: Conforme á todo lo que ha mandado mi señor el rey á su siervo, así
lo hará tu siervo. Mephi-boseth, dijo el rey, comerá á mi mesa, como uno
de los hijos del rey. \bibverse{12} Y tenía Mephi-boseth un hijo
pequeño, que se llamaba Michâ. Y toda la familia de la casa de Siba eran
siervos de Mephi-boseth. \bibverse{13} Y moraba Mephi-boseth en
Jerusalem, porque comía siempre á la mesa del rey: y era cojo de ambos
pies.

\hypertarget{section-9}{%
\section{10}\label{section-9}}

\bibverse{1} Después de esto aconteció, que murió el rey de los hijos de
Ammón: y reinó en lugar suyo Hanún su hijo. \bibverse{2} Y dijo David:
Yo haré misericordia con Hanún hijo de Naas, como su padre la hizo
conmigo. Y envió David sus siervos á consolarlo por su padre. Mas
llegados los siervos de David á la tierra de los hijos de Ammón,
\bibverse{3} Los príncipes de los hijos de Ammón dijeron á Hanún su
señor: ¿Te parece que por honrar David á tu padre te ha enviado
consoladores? ¿no ha enviado David sus siervos á ti por reconocer é
inspeccionar la ciudad, para destruirla? \bibverse{4} Entonces Hanún
tomó los siervos de David, y rapóles la mitad de la barba, y cortóles
los vestidos por la mitad hasta las nalgas, y despachólos. \bibverse{5}
Lo cual como fué hecho saber á David, envió á encontrarles, porque ellos
estaban en extremo avergonzados; y el rey hizo decirles: Estaos en
Jericó hasta que os vuelva á nacer la barba, y entonces regresaréis.
\bibverse{6} Y viendo los hijos de Ammón que se habían hecho odiosos á
David, enviaron los hijos de Ammón y tomaron á sueldo á los Siros de la
casa de Rehob, y á los Siros de Soba, veinte mil hombres de á pie: y del
rey de Maaca mil hombres, y de Is-tob doce mil hombres. \bibverse{7} Lo
cual como oyó David, envió á Joab con todo el ejército de los valientes.
\bibverse{8} Y saliendo los hijos de Ammón, ordenaron sus escuadrones á
la entrada de la puerta: mas los Siros de Soba, y de Rehob, y de Is-tob,
y de Maaca, estaban de por sí en el campo. \bibverse{9} Viendo pues Joab
que había escuadrones delante y detrás de él, entresacó de todos los
escogidos de Israel, y púsose en orden contra los Siros. \bibverse{10}
Entregó luego lo que quedó del pueblo en mano de Abisai su hermano, y
púsolo en orden para encontrar á los Ammonitas. \bibverse{11} Y dijo: Si
los Siros me fueren superiores, tú me ayudarás; y si los hijos de Ammón
pudieren más que tú, yo te daré ayuda. \bibverse{12} Esfuérzate, y
esforcémonos por nuestro pueblo, y por las ciudades de nuestro Dios: y
haga Jehová lo que bien le pareciere. \bibverse{13} Y acercóse Joab, y
el pueblo que con él estaba, para pelear con los Siros; mas ellos
huyeron delante de él. \bibverse{14} Entonces los hijos de Ammón, viendo
que los Siros habían huído, huyeron también ellos delante de Abisai, y
entráronse en la ciudad. Y volvió Joab de los hijos de Ammón, y vínose á
Jerusalem. \bibverse{15} Mas viendo los Siros que habían caído delante
de Israel, tornáronse á juntar. \bibverse{16} Y envió Hadad-ezer, y sacó
los Siros que estaban de la otra parte del río, los cuales vinieron á
Helam, llevando por jefe á Sobach general del ejército de Hadad-ezer.
\bibverse{17} Y como fué dado aviso á David, juntó á todo Israel, y
pasando el Jordán vino á Helam: y los Siros se pusieron en orden contra
David, y pelearon con él. \bibverse{18} Mas los Siros huyeron delante de
Israel: é hirió David de los Siros la gente de setecientos carros, y
cuarenta mil hombres de á caballo: hirió también á Sobach general del
ejército, y murió allí. \bibverse{19} Viendo pues todos los reyes que
asistían á Hadad-ezer, como habían ellos sido derrotados delante de
Israel, hicieron paz con Israel, y sirviéronle; y de allí adelante
temieron los Siros de socorrer á los hijos de Ammón.

\hypertarget{section-10}{%
\section{11}\label{section-10}}

\bibverse{1} Y aconteció á la vuelta de un año, en el tiempo que salen
los reyes á la guerra, que David envió á Joab, y á sus siervos con él, y
á todo Israel; y destruyeron á los Ammonitas, y pusieron cerco á Rabba:
mas David se quedó en Jerusalem. \bibverse{2} Y acaeció que levantándose
David de su cama á la hora de la tarde, paseábase por el terrado de la
casa real, cuando vió desde el terrado una mujer que se estaba lavando,
la cual era muy hermosa. \bibverse{3} Y envió David á preguntar por
aquella mujer, y dijéronle: Aquella es Bath-sheba hija de Eliam, mujer
de Uría Hetheo. \bibverse{4} Y envió David mensajeros, y tomóla: y así
que hubo entrado á él, él durmió con ella. Purificóse luego ella de su
inmundicia, y se volvió á su casa. \bibverse{5} Y concibió la mujer, y
enviólo á hacer saber á David, diciendo: Yo estoy embarazada.
\bibverse{6} Entonces David envió á decir á Joab: Envíame á Uría Hetheo.
Y enviólo Joab á David. \bibverse{7} Y como Uría vino á él, preguntóle
David por la salud de Joab, y por la salud del pueblo, y asimismo de la
guerra. \bibverse{8} Después dijo David á Uría: Desciende á tu casa, y
lava tus pies. Y saliendo Uría de casa del rey, vino tras de él comida
real. \bibverse{9} Mas Uría durmió á la puerta de la casa del rey con
todos los siervos de su señor, y no descendió á su casa. \bibverse{10} E
hicieron saber esto á David, diciendo: Uría no ha descendido á su casa.
Y dijo David á Uría: ¿No has venido de camino? ¿por qué pues no
descendiste á tu casa? \bibverse{11} Y Uría respondió á David: El arca,
é Israel y Judá, están debajo de tiendas; y mi señor Joab, y los siervos
de mi señor sobre la haz del campo: ¿y había yo de entrar en mi casa
para comer y beber, y á dormir con mi mujer? Por vida tuya, y por vida
de tu alma, que yo no haré tal cosa. \bibverse{12} Y David dijo á Uría:
Estáte aquí aún hoy, y mañana te despacharé. Y quedóse Uría en Jerusalem
aquel día y el siguiente. \bibverse{13} Y David lo convidó, é hízole
comer y beber delante de sí, hasta embriagarlo. Y él salió á la tarde á
dormir en su cama con los siervos de su señor; mas no descendió á su
casa. \bibverse{14} Venida la mañana, escribió David á Joab una carta,
la cual envió por mano de Uría. \bibverse{15} Y escribió en la carta,
diciendo: Poned á Uría delante de la fuerza de la batalla, y
desamparadle, para que sea herido y muera. \bibverse{16} Así fué que
cuando Joab cercó la ciudad, puso á Uría en el lugar donde sabía que
estaban los hombres más valientes. \bibverse{17} Y saliendo luego los de
la ciudad, pelearon con Joab, y cayeron algunos del pueblo de los
siervos de David; y murió también Uría Hetheo. \bibverse{18} Entonces
envió Joab, é hizo saber á David todos los negocios de la guerra.
\bibverse{19} Y mandó al mensajero, diciendo: Cuando acabares de contar
al rey todos los negocios de la guerra, \bibverse{20} Si el rey
comenzare á enojarse, y te dijere: ¿Por qué os acercasteis á la ciudad
peleando? ¿no sabíais lo que suelen arrojar del muro? \bibverse{21}
¿Quién hirió á Abimelech hijo de Jerobaal? ¿no echó una mujer del muro
un pedazo de una rueda de molino, y murió en Thebes? ¿por qué os
llegasteis al muro?: entonces tú le dirás: También tu siervo Uría Hetheo
es muerto. \bibverse{22} Y fué el mensajero, y llegando, contó á David
todas las cosas á que Joab le había enviado. \bibverse{23} Y dijo el
mensajero á David: Prevalecieron contra nosotros los hombres, que
salieron á nosotros al campo, bien que nosotros les hicimos retroceder
hasta la entrada de la puerta; \bibverse{24} Pero los flecheros tiraron
contra tus siervos desde el muro, y murieron algunos de los siervos del
rey; y murió también tu siervo Uría Hetheo. \bibverse{25} Y David dijo
al mensajero: Dirás así á Joab: No tengas pesar de esto, que de igual y
semejante manera suele consumir la espada: esfuerza la batalla contra la
ciudad, hasta que la rindas. Y tú aliéntale. \bibverse{26} Y oyendo la
mujer de Uría que su marido Uría era muerto, hizo duelo por su marido.
\bibverse{27} Y pasado el luto, envió David y recogióla á su casa: y fué
ella su mujer, y parióle un hijo. Mas esto que David había hecho, fué
desagradable á los ojos de Jehová.

\hypertarget{section-11}{%
\section{12}\label{section-11}}

\bibverse{1} Y envió Jehová á Nathán á David, el cual viniendo á él,
díjole: Había dos hombres en una ciudad, el uno rico, y el otro pobre.
\bibverse{2} El rico tenía numerosas ovejas y vacas: \bibverse{3} Mas el
pobre no tenía más que una sola cordera, que él había comprado y criado,
y que había crecido con él y con sus hijos juntamente, comiendo de su
bocado, y bebiendo de su vaso, y durmiendo en su seno: y teníala como á
una hija. \bibverse{4} Y vino uno de camino al hombre rico; y él no
quiso tomar de sus ovejas y de sus vacas, para guisar al caminante que
le había venido, sino que tomó la oveja de aquel hombre pobre, y
aderezóla para aquél que le había venido. \bibverse{5} Entonces se
encendió el furor de David en gran manera contra aquel hombre, y dijo á
Nathán: Vive Jehová, que el que tal hizo es digno de muerte.
\bibverse{6} Y que él debe pagar la cordera con cuatro tantos, porque
hizo esta tal cosa, y no tuvo misericordia. \bibverse{7} Entonces dijo
Nathán á David: Tú eres aquel hombre. Así ha dicho Jehová, Dios de
Israel: Yo te ungí por rey sobre Israel, y te libré de la mano de Saúl;
\bibverse{8} Yo te dí la casa de tu señor, y las mujeres de tu señor en
tu seno: demás de esto te dí la casa de Israel y de Judá; y si esto es
poco, yo te añadiré tales y tales cosas. \bibverse{9} ¿Por qué pues
tuviste en poco la palabra de Jehová, haciendo lo malo delante de sus
ojos? A Uría Hetheo heriste á cuchillo, y tomaste por tu mujer á su
mujer, y á él mataste con el cuchillo de los hijos de Ammón.
\bibverse{10} Por lo cual ahora no se apartará jamás de tu casa la
espada; por cuanto me menospreciaste, y tomaste la mujer de Uría Hetheo
para que fuese tu mujer. \bibverse{11} Así ha dicho Jehová: He aquí yo
levantaré sobre ti el mal de tu misma casa, y tomaré tus mujeres delante
de tus ojos, y las daré á tu prójimo, el cual yacerá con tus mujeres á
la vista de este sol. \bibverse{12} Porque tú lo hiciste en secreto; mas
yo haré esto delante de todo Israel, y delante del sol. \bibverse{13}
Entonces dijo David á Nathán: Pequé contra Jehová. Y Nathán dijo á
David: También Jehová ha remitido tu pecado: no morirás. \bibverse{14}
Mas por cuanto con este negocio hiciste blasfemar á los enemigos de
Jehová, el hijo que te ha nacido morirá ciertamente. \bibverse{15} Y
Nathán se volvió á su casa. Y Jehová hirió al niño que la mujer de Uría
había parido á David, y enfermó gravemente. \bibverse{16} Entonces rogó
David á Dios por el niño; y ayunó David, recogióse, y pasó la noche
acostado en tierra. \bibverse{17} Y levantándose los ancianos de su casa
fueron á él para hacerlo levantar de tierra; mas él no quiso, ni comió
con ellos pan. \bibverse{18} Y al séptimo día murió el niño; pero sus
siervos no osaban hacerle saber que el niño era muerto, diciendo entre
sí: Cuando el niño aun vivía, le hablábamos, y no quería oir nuestra
voz: ¿pues cuánto más mal le hará, si le dijéremos que el niño es
muerto? \bibverse{19} Mas David viendo á sus siervos hablar entre sí,
entendió que el niño era muerto; por lo que dijo David á sus siervos:
¿Es muerto el niño? Y ellos respondieron: Muerto es. \bibverse{20}
Entonces David se levantó de tierra, y lavóse y ungióse, y mudó sus
ropas, y entró á la casa de Jehová, y adoró. Y después vino á su casa, y
demandó, y pusiéronle pan, y comió. \bibverse{21} Y dijéronle sus
siervos: ¿Qué es esto que has hecho? Por el niño, viviendo aún, ayunabas
y llorabas; y él muerto, levantástete y comiste pan. \bibverse{22} Y él
respondió: Viviendo aún el niño, yo ayunaba y lloraba, diciendo: ¿Quién
sabe si Dios tendrá compasión de mí, por manera que viva el niño?
\bibverse{23} Mas ahora que ya es muerto, ¿para qué tengo de ayunar?
¿podré yo hacerle volver? Yo voy á él, mas él no volverá á mí.
\bibverse{24} Y consoló David á Bath-sheba su mujer, y entrando á ella,
durmió con ella; y parió un hijo, y llamó su nombre Salomón, al cual amó
Jehová: \bibverse{25} Que envió por mano de Nathán profeta, y llamó su
nombre Jedidiah, á causa de Jehová. \bibverse{26} Y Joab peleaba contra
Rabba de los hijos de Ammón, y tomó la ciudad real. \bibverse{27}
Entonces envió Joab mensajeros á David, diciendo: Yo he peleado contra
Rabba, y he tomado la ciudad de las aguas. \bibverse{28} Junta pues
ahora el pueblo que queda, y asienta campo contra la ciudad, y tómala;
porque tomando yo la ciudad, no se llame de mi nombre. \bibverse{29} Y
juntando David todo el pueblo, fué contra Rabba, y combatióla, y tomóla.
\bibverse{30} Y tomó la corona de su rey de su cabeza, la cual pesaba un
talento de oro, y tenía piedras preciosas; y fué puesta sobre la cabeza
de David. Y trajo muy grande despojo de la ciudad. \bibverse{31} Sacó
además el pueblo que estaba en ella, y púsolo debajo de sierras, y de
trillos de hierro, y de hachas de hierro; é hízolos pasar por hornos de
ladrillos: y lo mismo hizo á todas las ciudades de los hijos de Ammón.
Volvióse luego David con todo el pueblo á Jerusalem.

\hypertarget{section-12}{%
\section{13}\label{section-12}}

\bibverse{1} Aconteció después de esto, que teniendo Absalom hijo de
David una hermana hermosa que se llamaba Thamar, enamoróse de ella Amnón
hijo de David. \bibverse{2} Y estaba Amnón angustiado hasta enfermar,
por Thamar su hermana: porque por ser ella virgen, parecía á Amnón que
sería cosa dificultosa hacerle algo. \bibverse{3} Y Amnón tenía un amigo
que se llamaba Jonadab, hijo de Simea, hermano de David: y era Jonadab
hombre muy astuto. \bibverse{4} Y éste le dijo: Hijo del rey, ¿por qué
de día en día vas así enflaqueciendo? ¿no me lo descubrirás á mí? Y
Amnón le respondió: Yo amo á Thamar la hermana de Absalom mi hermano.
\bibverse{5} Y Jonadab le dijo: Acuéstate en tu cama, y finge que estás
enfermo; y cuando tu padre viniere á visitarte, dile: Ruégote que venga
mi hermana Thamar, para que me conforte con alguna comida, y aderece
delante de mí alguna vianda, para que viendo yo, la coma de su mano.
\bibverse{6} Acostóse pues Amnón, y fingió que estaba enfermo, y vino el
rey á visitarle: y dijo Amnón al rey: Yo te ruego que venga mi hermana
Thamar, y haga delante de mí dos hojuelas, que coma yo de su mano.
\bibverse{7} Y David envió á Thamar á su casa, diciendo: Ve ahora á casa
de Amnón tu hermano, y hazle de comer. \bibverse{8} Y fué Thamar á casa
de su hermano Amnón, el cual estaba acostado; y tomó harina, y amasó é
hizo hojuelas delante de él, y aderezólas. \bibverse{9} Tomó luego la
sartén, y sacólas delante de él: mas él no quiso comer. Y dijo Amnón:
Echad fuera de aquí á todos. Y todos se salieron de allí. \bibverse{10}
Entonces Amnón dijo á Thamar: Trae la comida á la alcoba, para que yo
coma de tu mano. Y tomando Thamar las hojuelas que había aderezado,
llevólas á su hermano Amnón á la alcoba. \bibverse{11} Y como ella se
las puso delante para que comiese, él trabó de ella, diciéndole: Ven,
hermana mía acuéstate conmigo. \bibverse{12} Ella entonces le respondió:
No, hermano mío, no me hagas fuerza; porque no se ha de hacer así en
Israel. No hagas tal desacierto. \bibverse{13} Porque, ¿dónde iría yo
con mi deshonra? Y aun tú serías estimado como uno de los perversos en
Israel. Ruégote pues ahora que hables al rey, que no me negará á ti.
\bibverse{14} Mas él no la quiso oir; antes pudiendo más que ella la
forzó, y echóse con ella. \bibverse{15} Aborrecióla luego Amnón de tan
grande aborrecimiento, que el odio con que la aborreció fué mayor que el
amor con que la había amado. Y díjole Amnón: Levántate y vete.
\bibverse{16} Y ella le respondió: No es razón; mayor mal es éste de
echarme, que el que me has hecho. Mas él no la quiso oir: \bibverse{17}
Antes llamando su criado que le servía dijo: Echame ésta allá fuera, y
tras ella cierra la puerta. \bibverse{18} Y tenía ella sobre sí una ropa
de colores, traje que las hijas vírgenes de los reyes vestían. Echóla
pues fuera su criado, y cerró la puerta tras ella. \bibverse{19}
Entonces Thamar tomó ceniza, y esparcióla sobre su cabeza, y rasgó la
ropa de colores de que estaba vestida, y puestas sus manos sobre su
cabeza, fuése gritando. \bibverse{20} Y díjole su hermano Absalom: ¿Ha
estado contigo tu hermano Amnón? Pues calla ahora, hermana mía: tu
hermano es; no pongas tu corazón en este negocio. Y quedóse Thamar
desconsolada en casa de Absalom su hermano. \bibverse{21} Y luego que el
rey David oyó todo esto, fué muy enojado. \bibverse{22} Mas Absalom no
habló con Amnón ni malo ni bueno; bien que Absalom aborrecía á Amnón,
porque había forzado á Thamar su hermana. \bibverse{23} Y aconteció
pasados dos años, que Absalom tenía esquiladores en Bala-hasor, que está
junto á Ephraim; y convidó Absalom á todos los hijos del rey.
\bibverse{24} Y vino Absalom al rey, y díjole: He aquí, tu siervo tiene
ahora esquiladores: yo ruego que venga el rey y sus siervos con tu
siervo. \bibverse{25} Y respondió el rey á Absalom: No, hijo mío, no
vamos todos, porque no te hagamos costa. Y aunque porfió con él, no
quiso ir, mas bendíjolo. \bibverse{26} Entonces dijo Absalom: Si no,
ruégote que venga con nosotros Amnón mi hermano. Y el rey le respondió:
¿Para qué ha de ir contigo? \bibverse{27} Y como Absalom lo importunase,
dejó ir con él á Amnón y á todos los hijos del rey. \bibverse{28} Y
había Absalom dado orden á sus criados, diciendo: Ahora bien, mirad
cuándo el corazón de Amnón estará alegre del vino, y en diciéndoos yo,
Herid á Amnón, entonces matadle, y no temáis; que yo os lo he mandado.
Esforzaos pues, y sed valientes. \bibverse{29} Y los criados de Absalom
hicieron con Amnón como Absalom lo había mandado. Levantáronse luego
todos los hijos del rey, y subieron todos en sus mulos, y huyeron.
\bibverse{30} Y estando aún ellos en el camino, llegó á David el rumor
que decía: Absalom ha muerto á todos los hijos del rey, que ninguno de
ellos ha quedado. \bibverse{31} Entonces levantándose David, rasgó sus
vestidos, y echóse en tierra, y todos sus criados, rasgados sus
vestidos, estaban delante. \bibverse{32} Y Jonadab, hijo de Simea
hermano de David, habló y dijo: No diga mi señor que han muerto á todos
los jóvenes hijos del rey, que sólo Amnón es muerto: porque en boca de
Absalom estaba puesto desde el día que Amnón forzó á Thamar su hermana.
\bibverse{33} Por tanto, ahora no ponga mi señor el rey en su corazón
esa voz que dice: Todos los hijos del rey son muertos: porque sólo Amnón
es muerto. \bibverse{34} Absalom huyó luego. Entre tanto, alzando sus
ojos el mozo que estaba en atalaya, miró, y he aquí mucho pueblo que
venía á sus espaldas por el camino de hacia el monte. \bibverse{35} Y
dijo Jonadab al rey: He allí los hijos del rey que vienen: es así como
tu siervo ha dicho. \bibverse{36} Y como él acabó de hablar, he aquí los
hijos del rey que vinieron, y alzando su voz lloraron. Y también el
mismo rey y todos sus siervos lloraron con muy grandes lamentos.
\bibverse{37} Mas Absalom huyó, y fuése á Talmai hijo de Amiud, rey de
Gessur. Y David lloraba por su hijo todos los días. \bibverse{38} Y
después que Absalom huyó y se fué á Gessur, estuvo allá tres años.
\bibverse{39} Y el rey David deseó ver á Absalom: porque ya estaba
consolado acerca de Amnón que era muerto.

\hypertarget{section-13}{%
\section{14}\label{section-13}}

\bibverse{1} Y conociendo Joab hijo de Sarvia, que el corazón del rey
estaba por Absalom, \bibverse{2} Envió Joab á Tecoa, y tomó de allá una
mujer astuta, y díjole: Yo te ruego que te enlutes, y te vistas de ropas
de luto, y no te unjas con óleo, antes sé como mujer que ha mucho tiempo
que trae luto por algún muerto; \bibverse{3} Y entrando al rey, habla
con él de esta manera. Y puso Joab las palabras en su boca. \bibverse{4}
Entró pues aquella mujer de Tecoa al rey, y postrándose en tierra sobre
su rostro hizo reverencia, y dijo: Oh rey, salva. \bibverse{5} Y el rey
le dijo: ¿Qué tienes? Y ella respondió: Yo á la verdad soy una mujer
viuda y mi marido es muerto. \bibverse{6} Y tu sierva tenía dos hijos y
los dos riñeron en el campo; y no habiendo quien los despartiese, hirió
el uno al otro, y matólo. \bibverse{7} Y he aquí toda la parentela se ha
levantado contra tu sierva, diciendo: Entrega al que mató á su hermano,
para que le hagamos morir por la vida de su hermano á quien él mató, y
quitemos también el heredero. Así apagarán el ascua que me ha quedado,
no dejando á mi marido nombre ni reliquia sobre la tierra. \bibverse{8}
Entonces el rey dijo á la mujer: Vete á tu casa, que yo mandaré acerca
de ti. \bibverse{9} Y la mujer de Tecoa dijo al rey: Rey señor mío, la
maldad sea sobre mí y sobre la casa de mi padre; mas el rey y su trono
sin culpa. \bibverse{10} Y el rey dijo: Al que hablare contra ti, tráelo
á mí, que no te tocará más. \bibverse{11} Dijo ella entonces: Ruégote,
oh rey, que te acuerdes de Jehová tu Dios, que no dejes á los cercanos
de la sangre aumentar el daño con destruir á mi hijo. Y él respondió:
Vive Jehová, que no caerá ni un cabello de la cabeza de tu hijo en
tierra. \bibverse{12} Y la mujer dijo: Ruégote que hable tu criada una
palabra á mi señor el rey. Y él dijo: Habla. \bibverse{13} Entonces la
mujer dijo: ¿Por qué pues piensas tú otro tanto contra el pueblo de
Dios? que hablando el rey esta palabra, es como culpado, por cuanto el
rey no hace volver á su fugitivo. \bibverse{14} Porque de cierto
morimos, y somos como aguas derramadas por tierra, que no pueden volver
á recogerse: ni Dios quita la vida, sino que arbitra medio para que su
desviado no sea de él excluído. \bibverse{15} Y que yo he venido ahora
para decir esto al rey mi señor, es porque el pueblo me ha puesto miedo.
Mas tu sierva dijo: Hablaré ahora al rey: quizá él hará lo que su sierva
diga. \bibverse{16} Pues el rey oirá, para librar á su sierva de mano
del hombre que me quiere raer á mí, y á mi hijo juntamente, de la
heredad de Dios. \bibverse{17} Tu sierva pues dice: Que sea ahora la
respuesta de mi señor el rey para descanso; pues que mi señor el rey es
como un ángel de Dios para escuchar lo bueno y lo malo. Así Jehová tu
Dios sea contigo. \bibverse{18} Entonces él respondió, y dijo á la
mujer: Yo te ruego que no me encubras nada de lo que yo te preguntare. Y
la mujer dijo: Hable mi señor el rey. \bibverse{19} Y el rey dijo: ¿No
ha sido la mano de Joab contigo en todas estas cosas? Y la mujer
respondió y dijo: Vive tu alma, rey señor mío, que no hay que apartarse
á derecha ni á izquierda de todo lo que mi señor el rey ha hablado:
porque tu siervo Joab, él me mandó, y él puso en boca de tu sierva todas
estas palabras; \bibverse{20} Y que trocara la forma de las palabras,
Joab tu siervo lo ha hecho: mas mi señor es sabio, conforme á la
sabiduría de un ángel de Dios, para conocer lo que hay en la tierra.
\bibverse{21} Entonces el rey dijo á Joab: He aquí yo hago esto: ve, y
haz volver al mozo Absalom. \bibverse{22} Y Joab se postró en tierra
sobre su rostro, é hizo reverencia, y después que bendijo al rey, dijo:
Hoy ha entendido tu siervo que he hallado gracia en tus ojos, rey señor
mío; pues que ha hecho el rey lo que su siervo ha dicho. \bibverse{23}
Levantóse luego Joab, y fué á Gessur, y volvió á Absalom á Jerusalem.
\bibverse{24} Mas el rey dijo: Váyase á su casa, y no vea mi rostro. Y
volvióse Absalom á su casa, y no vió el rostro del rey. \bibverse{25} Y
no había en todo Israel hombre tan hermoso como Absalom, de alabar en
gran manera: desde la planta de su pie hasta la mollera no había en él
defecto. \bibverse{26} Y cuando se cortaba el cabello, (lo cual hacía al
fin de cada año, pues le causaba molestia, y por eso se lo cortaba),
pesaba el cabello de su cabeza doscientos siclos de peso real.
\bibverse{27} Y Naciéronle á Absalom tres hijos, y una hija que se llamó
Thamar, la cual era hermosa de ver. \bibverse{28} Y estuvo Absalom por
espacio de dos años en Jerusalem, y no vió la cara del rey.
\bibverse{29} Y mandó Absalom por Joab, para enviarlo al rey; mas no
quiso venir á él; ni aunque envió por segunda vez, quiso él venir.
\bibverse{30} Entonces dijo á sus siervos: Bien sabéis las tierras de
Joab junto á mi lugar, donde tiene sus cebadas; id, y pegadles fuego; y
los siervos de Absalom pegaron fuego á las tierras. \bibverse{31}
Levantóse por tanto Joab, y vino á Absalom á su casa, y díjole: ¿Por qué
han puesto fuego tus siervos á mis tierras? \bibverse{32} Y Absalom
respondió á Joab: He aquí, yo he enviado por ti, diciendo que vinieses
acá, á fin de enviarte yo al rey á que le dijeses: ¿Para qué vine de
Gessur? mejor me fuera estar aún allá. Vea yo ahora la cara del rey; y
si hay en mí pecado, máteme. \bibverse{33} Vino pues Joab al rey, é
hízoselo saber. Entonces llamó á Absalom, el cual vino al rey, é inclinó
su rostro á tierra delante del rey: y el rey besó á Absalom.

\hypertarget{section-14}{%
\section{15}\label{section-14}}

\bibverse{1} Aconteció después de esto, que Absalom se hizo de carros y
caballos, y cincuenta hombres que corriesen delante de él. \bibverse{2}
Y levantábase Absalom de mañana, y poníase á un lado del camino de la
puerta; y á cualquiera que tenía pleito y venía al rey á juicio, Absalom
le llamaba á sí, y decíale: ¿De qué ciudad eres? Y él respondía: Tu
siervo es de una de las tribus de Israel. \bibverse{3} Entonces Absalom
le decía: Mira, tus palabras son buenas y justas: mas no tienes quien te
oiga por el rey. \bibverse{4} Y decía Absalom: ¡Quién me pusiera por
juez en la tierra, para que viniesen á mí todos los que tienen pleito ó
negocio, que yo les haría justicia! \bibverse{5} Y acontecía que, cuando
alguno se llegaba para inclinarse á él, él extendía la mano, y lo
tomaba, y lo besaba. \bibverse{6} Y de esta manera hacía con todo Israel
que venía al rey á juicio: y así robaba Absalom el corazón de los de
Israel. \bibverse{7} Y al cabo de cuarenta años aconteció que Absalom
dijo al rey: Yo te ruego me permitas que vaya á Hebrón, á pagar mi voto
que he prometido á Jehová: \bibverse{8} Porque tu siervo hizo voto
cuando estaba en Gessur en Siria, diciendo: Si Jehová me volviere á
Jerusalem, yo serviré á Jehová. \bibverse{9} Y el rey le dijo: Ve en
paz. Y él se levantó, y se fué á Hebrón. \bibverse{10} Empero envió
Absalom espías por todas las tribus de Israel, diciendo: Cuando oyereis
el sonido de la trompeta, diréis: Absalom reina en Hebrón. \bibverse{11}
Y fueron con Absalom doscientos hombres de Jerusalem por él convidados,
los cuales iban en su sencillez, sin saber nada. \bibverse{12} También
envió Absalom por Achitophel Gilonita, del consejo de David, á Gilo su
ciudad, mientras hacía sus sacrificios. Y la conjuración vino á ser
grande, pues se iba aumentando el pueblo con Absalom. \bibverse{13} Y
vino el aviso á David, diciendo: El corazón de todo Israel va tras
Absalom. \bibverse{14} Entonces David dijo á todos sus siervos que
estaban con él en Jerusalem: Levantaos y huyamos, porque no podremos
escapar delante de Absalom; daos priesa á partir, no sea que
apresurándose él nos alcance, y arroje el mal sobre nosotros, y hiera la
ciudad á filo de espada. \bibverse{15} Y los siervos del rey dijeron al
rey: He aquí, tus siervos están prestos á todo lo que nuestro señor el
rey eligiere. \bibverse{16} El rey entonces salió, con toda su familia
en pos de él. Y dejó el rey diez mujeres concubinas para que guardasen
la casa. \bibverse{17} Salió pues el rey con todo el pueblo que le
seguía, y paráronse en un lugar distante. \bibverse{18} Y todos sus
siervos pasaban á su lado, con todos los Ceretheos y Peletheos; y todos
los Getheos, seiscientos hombres que habían venido á pie desde Gath,
iban delante del rey. \bibverse{19} Y dijo el rey á Ittai Getheo: ¿Para
qué vienes tú también con nosotros? vuélvete y quédate con el rey;
porque tú eres extranjero, y desterrado también de tu lugar.
\bibverse{20} ¿Ayer viniste, y téngote de hacer hoy que mudes lugar para
ir con nosotros? Yo voy como voy: tú vuélvete, y haz volver á tus
hermanos; en ti haya misericordia y verdad. \bibverse{21} Y respondió
Ittai al rey, diciendo: Vive Dios, y vive mi señor el rey, que, ó para
muerte ó para vida, donde mi señor el rey estuviere, allí estará también
tu siervo. \bibverse{22} Entonces David dijo á Ittai: Ven pues, y pasa.
Y pasó Ittai Getheo, y todos sus hombres, y toda su familia.
\bibverse{23} Y todo el país lloró en alta voz; pasó luego toda la gente
el torrente de Cedrón; asimismo pasó el rey, y todo el pueblo pasó, al
camino que va al desierto. \bibverse{24} Y he aquí, también iba Sadoc, y
con él todos los Levitas que llevaban el arca del pacto de Dios; y
asentaron el arca del pacto de Dios. Y subió Abiathar después que hubo
acabado de salir de la ciudad todo el pueblo. \bibverse{25} Pero dijo el
rey á Sadoc: Vuelve el arca de Dios á la ciudad; que si yo hallare
gracia en los ojos de Jehová, él me volverá, y me hará ver á ella y á su
tabernáculo: \bibverse{26} Y si dijere: No me agradas: aquí estoy, haga
de mí lo que bien le pareciere. \bibverse{27} Dijo aún el rey á Sadoc
sacerdote: ¿No eres tú el vidente? Vuélvete en paz á la ciudad; y con
vosotros vuestros dos hijos, tu hijo Ahimaas, y Jonathán hijo de
Abiathar. \bibverse{28} Mirad, yo me detendré en los campos del
desierto, hasta que venga respuesta de vosotros que me dé aviso.
\bibverse{29} Entonces Sadoc y Abiathar volvieron el arca de Dios á
Jerusalem; y estuviéronse allá. \bibverse{30} Y David subió la cuesta de
las olivas; y subióla llorando, llevando la cabeza cubierta, y los pies
descalzos. También todo el pueblo que tenía consigo cubrió cada uno su
cabeza, y subieron llorando así como subían. \bibverse{31} Y dieron
aviso á David, diciendo: Achitophel está entre los que conspiraron con
Absalom. Entonces dijo David: Entontece ahora, oh Jehová, el consejo de
Achitophel. \bibverse{32} Y como David llegó á la cumbre del monte para
adorar allí á Dios, he aquí Husai Arachîta que le salió al encuentro,
trayendo rota su ropa, y tierra sobre su cabeza. \bibverse{33} Y díjole
David: Si pasares conmigo, serme has de carga; \bibverse{34} Mas si
volvieres á la ciudad, y dijeres á Absalom: Rey, yo seré tu siervo; como
hasta aquí he sido siervo de tu padre, así seré ahora siervo tuyo,
entonces tú me disiparás el consejo de Achitophel. \bibverse{35} ¿No
estarán allí contigo Sadoc y Abiathar sacerdotes? Por tanto, todo lo que
oyeres en la casa del rey, darás aviso de ello á Sadoc y á Abiathar
sacerdotes. \bibverse{36} Y he aquí que están con ellos sus dos hijos,
Ahimaas el de Sadoc, y Jonathán el de Abiathar: por mano de ellos me
enviaréis aviso de todo lo que oyereis. \bibverse{37} Así se vino Husai
amigo de David á la ciudad; y Absalom entró en Jerusalem.

\hypertarget{section-15}{%
\section{16}\label{section-15}}

\bibverse{1} Y como David pasó un poco de la cumbre del monte, he aquí
Siba, el criado de Mephi-boseth, que lo salía á recibir con un par de
asnos enalbardados, y sobre ellos doscientos panes, y cien hilos de
pasas, y cien panes de higos secos, y un cuero de vino. \bibverse{2} Y
dijo el rey á Siba: ¿Qué es esto? Y Siba respondió: Los asnos son para
la familia del rey, en que suban; los panes y la pasa para los criados,
que coman; y el vino, para que beban los que se cansaren en el desierto.
\bibverse{3} Y dijo el rey: ¿Dónde está el hijo de tu señor? Y Siba
respondió al rey: He aquí él se ha quedado en Jerusalem, porque ha
dicho: Hoy me devolverá la casa de Israel el reino de mi padre.
\bibverse{4} Entonces el rey dijo á Siba: He aquí, sea tuyo todo lo que
tiene Mephi-boseth. Y respondió Siba inclinándose: Rey señor mío, halle
yo gracia delante de ti. \bibverse{5} Y vino el rey David hasta Bahurim:
y he aquí, salía uno de la familia de la casa de Saúl, el cual se
llamaba Semei, hijo de Gera; y salía maldiciendo, \bibverse{6} Y echando
piedras contra David, y contra todos los siervos del rey David: y todo
el pueblo, y todos los hombres valientes estaban á su diestra y á su
siniestra. \bibverse{7} Y decía Semei, maldiciéndole: Sal, sal, varón de
sangres, y hombre de Belial: \bibverse{8} Jehová te ha dado el pago de
toda la sangre de la casa de Saúl, en lugar del cual tú has reinado: mas
Jehová ha entregado el reino en mano de tu hijo Absalom; y hete aquí
sorprendido en tu maldad, porque eres varón de sangres. \bibverse{9}
Entonces Abisai hijo de Sarvia dijo al rey: ¿Por qué maldice este perro
muerto á mi señor el rey? Yo te ruego que me dejes pasar, y quitaréle la
cabeza. \bibverse{10} Y el rey respondió: ¿Qué tengo yo con vosotros,
hijos de Sarvia? El maldice así, porque Jehová le ha dicho que maldiga á
David: ¿quién pues le dirá: Por qué lo haces así? \bibverse{11} Y dijo
David á Abisai y á todos sus siervos: He aquí, mi hijo que ha salido de
mis entrañas, acecha á mi vida: ¿cuánto más ahora un hijo de Benjamín?
Dejadle que maldiga, que Jehová se lo ha dicho. \bibverse{12} Quizá
mirará Jehová á mi aflicción, y me dará Jehová bien por sus maldiciones
de hoy. \bibverse{13} Y como David y los suyos iban por el camino, Semei
iba por el lado del monte delante de él, andando y maldiciendo, y
arrojando piedras delante de él, y esparciendo polvo. \bibverse{14} Y el
rey y todo el pueblo que con él estaba, llegaron fatigados, y
descansaron allí. \bibverse{15} Y Absalom y todo el pueblo, los varones
de Israel, entraron en Jerusalem, y con él Achitophel. \bibverse{16} Y
acaeció luego, que como Husai Arachîta amigo de David hubo llegado á
Absalom, díjole Husai: Viva el rey, viva el rey. \bibverse{17} Y Absalom
dijo á Husai: ¿Este es tu agradecimiento para con tu amigo? ¿por qué no
fuiste con tu amigo? \bibverse{18} Y Husai respondió á Absalom: No:
antes al que eligiere Jehová y este pueblo y todos los varones de
Israel, de aquél seré yo, y con aquél quedaré. \bibverse{19} ¿Y á quién
había yo de servir? ¿no es á su hijo? Como he servido delante de tu
padre, así seré delante de ti. \bibverse{20} Entonces dijo Absalom á
Achitophel: Consultad qué haremos. \bibverse{21} Y Achitophel dijo á
Absalom: Entra á las concubinas de tu padre, que él dejó para guardar la
casa; y todo el pueblo de Israel oirá que te has hecho aborrecible á tu
padre, y así se esforzarán las manos de todos los que están contigo.
\bibverse{22} Entonces pusieron una tienda á Absalom sobre el terrado, y
entró Absalom á las concubinas de su padre, en ojos de todo Israel.
\bibverse{23} Y el consejo que daba Achitophel en aquellos días, era
como si consultaran la palabra de Dios. Tal era el consejo de
Achitophel, así con David como con Absalom.

\hypertarget{section-16}{%
\section{17}\label{section-16}}

\bibverse{1} Entonces Achitophel dijo á Absalom: Yo escogeré ahora doce
mil hombres, y me levantaré, y seguiré á David esta noche; \bibverse{2}
Y daré sobre él cuando él estará cansado y flaco de manos: lo
atemorizaré, y todo el pueblo que está con él huirá, y heriré al rey
solo. \bibverse{3} Así tornaré á todo el pueblo á ti: y cuando ellos
hubieren vuelto, (pues aquel hombre es el que tú quieres) todo el pueblo
estará en paz. \bibverse{4} Esta razón pareció bien á Absalom y á todos
los ancianos de Israel. \bibverse{5} Y dijo Absalom: Llama también ahora
á Husai Arachîta, para que asimismo oigamos lo que él dirá. \bibverse{6}
Y como Husai vino á Absalom, hablóle Absalom, diciendo: Así ha dicho
Achitophel; ¿seguiremos su consejo, ó no? Di tú. \bibverse{7} Entonces
Husai dijo á Absalom: El consejo que ha dado esta vez Achitophel no es
bueno. \bibverse{8} Y añadió Husai: Tú sabes que tu padre y los suyos
son hombres valientes, y que están con amargura de ánimo, como la osa en
el campo cuando le han quitado los hijos. Además, tu padre es hombre de
guerra, y no tendrá la noche con el pueblo. \bibverse{9} He aquí él
estará ahora escondido en alguna cueva, ó en otro lugar: y si al
principio cayeren algunos de los tuyos, oirálo quien lo oyere, y dirá:
El pueblo que sigue á Absalom ha sido derrotado. \bibverse{10} Así aun
el hombre valiente, cuyo corazón sea como corazón de león, sin duda
desmayará: porque todo Israel sabe que tu padre es hombre valiente, y
que los que están con él son esforzados. \bibverse{11} Aconsejo pues que
todo Israel se junte á ti, desde Dan hasta Beer-seba, en multitud como
la arena que está á la orilla de la mar, y que tú en persona vayas á la
batalla. \bibverse{12} Entonces le acometeremos en cualquier lugar que
pudiere hallarse, y daremos sobre él como cuando el rocío cae sobre la
tierra, y ni uno dejaremos de él, y de todos los que con él están.
\bibverse{13} Y si se recogiere en alguna ciudad, todos los de Israel
traerán sogas á aquella ciudad, y la arrastraremos hasta el arroyo, que
nunca más parezca piedra de ella. \bibverse{14} Entonces Absalom y todos
los de Israel dijeron: El consejo de Husai Arachîta es mejor que el
consejo de Achitophel. Porque había Jehová ordenado que el acertado
consejo de Achitophel se frustrara, para que Jehová hiciese venir el mal
sobre Absalom. \bibverse{15} Dijo luego Husai á Sadoc y á Abiathar
sacerdotes: Así y así aconsejó Achitophel á Absalom y á los ancianos de
Israel: y de esta manera aconsejé yo. \bibverse{16} Por tanto enviad
inmediatamente, y dad aviso á David, diciendo: No quedes esta noche en
los campos del desierto, sino pasa luego el Jordán, porque el rey no sea
consumido, y todo el pueblo que con él está. \bibverse{17} Y Jonathán y
Ahimaas estaban junto á la fuente de Rogel, porque no podían ellos
mostrarse viniendo á la ciudad; fué por tanto una criada, y dióles el
aviso: y ellos fueron, y noticiáronlo al rey David. \bibverse{18} Empero
fueron vistos por un mozo, el cual dió cuenta á Absalom: sin embargo los
dos se dieron priesa á caminar, y llegaron á casa de un hombre en
Bahurim, que tenía un pozo en su patio, dentro del cual se metieron.
\bibverse{19} Y tomando la mujer de la casa una manta, extendióla sobre
la boca del pozo, y tendió sobre ella el grano trillado; y no se penetró
el negocio. \bibverse{20} Llegando luego los criados de Absalom á la
casa á la mujer, dijéronle: ¿Dónde están Ahimaas y Jonathán? Y la mujer
les respondió: Ya han pasado el vado de las aguas. Y como ellos los
buscaron y no los hallaron, volviéronse á Jerusalem. \bibverse{21} Y
después que ellos se hubieron ido, estotros salieron del pozo, y
fuéronse, y dieron aviso al rey David; y dijéronle: Levantaos y daos
priesa á pasar las aguas, porque Achitophel ha dado tal consejo contra
vosotros. \bibverse{22} Entonces David se levantó, y todo el pueblo que
con él estaba, y pasaron el Jordán antes que amaneciese; ni siquiera
faltó uno que no pasase el Jordán. \bibverse{23} Y Achitophel, viendo
que no se había puesto por obra su consejo, enalbardó su asno, y
levantóse, y fuése á su casa en su ciudad; y después de disponer acerca
de su casa, ahorcóse y murió, y fué sepultado en el sepulcro de su
padre. \bibverse{24} Y David llegó á Mahanaim, y Absalom pasó el Jordán
con toda la gente de Israel. \bibverse{25} Y Absalom constituyó á Amasa
sobre el ejército en lugar de Joab, el cual Amasa fué hijo de un varón
de Israel llamado Itra, el cual había entrado á Abigail hija de Naas,
hermana de Sarvia, madre de Joab. \bibverse{26} Y asentó campo Israel
con Absalom en tierra de Galaad. \bibverse{27} Y luego que David llegó á
Mahanaim, Sobi hijo de Naas de Rabba de los hijos de Ammon, y Machîr
hijo de Ammiel de Lodebar, y Barzillai Galaadita de Rogelim,
\bibverse{28} Trajeron á David y al pueblo que estaba con él, camas, y
tazas, y vasijas de barro, y trigo, y cebada, y harina, y grano tostado,
habas, lentejas, y garbanzos tostados, \bibverse{29} Miel, manteca,
ovejas, y quesos de vacas, para que comiesen; porque dijeron: Aquel
pueblo está hambriento, y cansado, y tendrá sed en el desierto.

\hypertarget{section-17}{%
\section{18}\label{section-17}}

\bibverse{1} David pues revistó el pueblo que tenía consigo, y puso
sobre ellos tribunos y centuriones. \bibverse{2} Y consignó la tercera
parte del pueblo al mando de Joab, y otra tercera al mando de Abisai,
hijo de Sarvia, hermano de Joab, y la otra tercera parte al mando de
Ittai Getheo. Y dijo el rey al pueblo: Yo también saldré con vosotros.
\bibverse{3} Mas el pueblo dijo: No saldrás; porque si nosotros
huyéremos, no harán caso de nosotros; y aunque la mitad de nosotros
muera, no harán caso de nosotros: mas tú ahora vales tanto como diez mil
de nosotros. Será pues mejor que tú nos des ayuda desde la ciudad.
\bibverse{4} Entonces el rey les dijo: Yo haré lo que bien os pareciere.
Y púsose el rey á la entrada de la puerta, mientras salía todo el pueblo
de ciento en ciento y de mil en mil. \bibverse{5} Y el rey mandó á Joab
y á Abisai y á Ittai, diciendo: Tratad benignamente por amor de mí al
mozo Absalom. Y todo el pueblo oyó cuando dió el rey orden acerca de
Absalom á todos los capitanes. \bibverse{6} Salió pues el pueblo al
campo contra Israel, y dióse la batalla en el bosque de Ephraim;
\bibverse{7} Y allí cayó el pueblo de Israel delante de los siervos de
David, é hízose allí en aquel día una gran matanza de veinte mil
hombres; \bibverse{8} Y derramándose allí el ejército por la haz de toda
la tierra, fueron más los que consumió el bosque de los del pueblo, que
los que consumió el cuchillo aquel día. \bibverse{9} Y encontróse
Absalom con los siervos de David: é iba Absalom sobre un mulo, y el mulo
se entró debajo de un espeso y grande alcornoque, y asiósele la cabeza
al alcornoque, y quedó entre el cielo y la tierra; pues el mulo en que
iba pasó delante. \bibverse{10} Y viéndolo uno, avisó á Joab, diciendo:
He aquí que he visto á Absalom colgado de un alcornoque. \bibverse{11} Y
Joab respondió al hombre que le daba la nueva: Y viéndolo tú, ¿por qué
no le heriste luego allí echándole á tierra? y sobre mí, que te hubiera
dado diez siclos de plata, y un talabarte. \bibverse{12} Y el hombre
dijo á Joab: Aunque me importara en mis manos mil siclos de plata, no
extendiera yo mi mano contra el hijo del rey; porque nosotros lo oímos
cuando el rey te mandó á ti y á Abisai y á Ittai, diciendo: Mirad que
ninguno toque en el joven Absalom. \bibverse{13} Por otra parte, habría
yo hecho traición contra mi vida (pues que al rey nada se le esconde), y
tú mismo estarías en contra. \bibverse{14} Y respondió Joab: No es razón
que yo te ruegue. Y tomando tres dardos en sus manos, hincólos en el
corazón de Absalom, que aun estaba vivo en medio del alcornoque.
\bibverse{15} Cercándolo luego diez mancebos escuderos de Joab, hirieron
á Absalom, y acabáronle. \bibverse{16} Entonces Joab tocó la corneta, y
el pueblo se volvió de seguir á Israel, porque Joab detuvo al pueblo.
\bibverse{17} Tomando después á Absalom, echáronle en un gran hoyo en el
bosque, y levantaron sobre él un muy grande montón de piedras; y todo
Israel huyó, cada uno á sus estancias. \bibverse{18} Y había Absalom en
su vida tomado y levantádose una columna, la cual está en el valle del
rey; porque había dicho: Yo no tengo hijo que conserve la memoria de mi
nombre. Y llamó aquella columna de su nombre: y así se llamó el Lugar de
Absalom, hasta hoy. \bibverse{19} Entonces Ahimaas hijo de Sadoc dijo:
¿Correré ahora, y daré las nuevas al rey de como Jehová ha defendido su
causa de la mano de sus enemigos? \bibverse{20} Y respondió Joab: Hoy no
llevarás las nuevas: las llevarás otro día: no darás hoy la nueva,
porque el hijo del rey es muerto. \bibverse{21} Y Joab dijo á Cusi: Ve
tú, y di al rey lo que has visto. Y Cusi hizo reverencia á Joab, y
corrió. \bibverse{22} Entonces Ahimaas hijo de Sadoc tornó á decir á
Joab: Sea lo que fuere, yo correré ahora tras Cusi. Y Joab dijo: Hijo
mío, ¿para qué has tú de correr, pues que no hallarás premio por las
nuevas? \bibverse{23} Mas él respondió: Sea lo que fuere, yo correré.
Entonces le dijo: Corre. Corrió pues Ahimaas por el camino de la
llanura, y pasó delante de Cusi. \bibverse{24} Estaba David á la sazón
sentado entre las dos puertas; y el atalaya había ido al terrado de
sobre la puerta en el muro, y alzando sus ojos, miró, y vió á uno que
corría solo. \bibverse{25} El atalaya dió luego voces, é hízolo saber al
rey. Y el rey dijo: Si es solo, buenas nuevas trae. En tanto que él
venía acercándose, \bibverse{26} Vió el atalaya otro que corría; y dió
voces el atalaya al portero, diciendo: He aquí otro hombre que corre
solo. Y el rey dijo: Éste también es mensajero. \bibverse{27} Y el
atalaya volvió á decir: Paréceme el correr del primero como el correr de
Ahimaas hijo de Sadoc. Y respondió el rey: Ese es hombre de bien, y
viene con buena nueva. \bibverse{28} Entonces Ahimaas dijo en alta voz
al rey: Paz. E inclinóse á tierra delante del rey, y dijo: Bendito sea
Jehová Dios tuyo, que ha entregado á los hombres que habían levantado
sus manos contra mi señor el rey. \bibverse{29} Y el rey dijo: ¿El mozo
Absalom tiene paz? Y Ahimaas respondió: Vi yo un grande alboroto cuando
envió Joab al siervo del rey y á mí tu siervo; mas no sé qué era.
\bibverse{30} Y el rey dijo: Pasa, y ponte allí. Y él pasó, y paróse.
\bibverse{31} Y luego vino Cusi, y dijo: Reciba nueva mi señor el rey,
que hoy Jehová ha defendido tu causa de la mano de todos los que se
habían levantado contra ti. \bibverse{32} El rey entonces dijo á Cusi:
¿El mozo Absalom tiene paz? Y Cusi respondió: Como aquel mozo sean los
enemigos de mi señor el rey, y todos los que se levantan contra ti para
mal. \bibverse{33} Entonces el rey se turbó, y subióse á la sala de la
puerta, y lloró; y yendo, decía así: ¡Hijo mío Absalom, hijo mío, hijo
mío Absalom! ¡Quién me diera que muriera yo en lugar de ti, Absalom,
hijo mío, hijo mío!

\hypertarget{section-18}{%
\section{19}\label{section-18}}

\bibverse{1} Y dieron aviso á Joab: He aquí el rey llora, y hace duelo
por Absalom. \bibverse{2} Y volvióse aquel día la victoria en luto para
todo el pueblo; porque oyó decir el pueblo aquel día que el rey tenía
dolor por su hijo. \bibverse{3} Entróse el pueblo aquel día en la ciudad
escondidamente, como suele entrar á escondidas el pueblo avergonzado que
ha huído de la batalla. \bibverse{4} Mas el rey, cubierto el rostro,
clamaba en alta voz: ¡Hijo mío Absalom, Absalom, hijo mío, hijo mío!
\bibverse{5} Y entrando Joab en casa al rey, díjole: Hoy has avergonzado
el rostro de todos tus siervos, que han hoy librado tu vida, y la vida
de tus hijos y de tus hijas, y la vida de tus mujeres, y la vida de tus
concubinas, \bibverse{6} Amando á los que te aborrecen, y aborreciendo á
los que te aman: porque hoy has declarado que nada te importan tus
príncipes y siervos; pues hoy echo de ver que si Absalom viviera, bien
que nosotros todos estuviéramos hoy muertos, entonces te contentaras.
\bibverse{7} Levántate pues ahora, y sal fuera, y halaga á tus siervos:
porque juro por Jehová, que si no sales, ni aun uno quede contigo esta
noche; y de esto te pesará más que de todos los males que te han
sobrevenido desde tu mocedad hasta ahora. \bibverse{8} Entonces se
levantó el rey, y sentóse á la puerta; y fué declarado á todo el pueblo,
diciendo: He aquí el rey está sentado á la puerta. Y vino todo el pueblo
delante del rey; mas Israel había huído, cada uno á sus estancias.
\bibverse{9} Y todo el pueblo porfiaba en todas las tribus de Israel,
diciendo: El rey nos ha librado de mano de nuestros enemigos, y él nos
ha salvado de mano de los Filisteos; y ahora había huído de la tierra
por miedo de Absalom. \bibverse{10} Y Absalom, á quien habíamos ungido
sobre nosotros, es muerto en la batalla. ¿Por qué pues os estáis ahora
quedos en orden á hacer volver al rey? \bibverse{11} Y el rey David
envió á Sadoc y á Abiathar sacerdotes, diciendo: Hablad á los ancianos
de Judá y decidles: ¿Por qué seréis vosotros los postreros en volver el
rey á su casa, ya que la palabra de todo Israel ha venido al rey de
volverle á su casa? \bibverse{12} Vosotros sois mis hermanos; mis huesos
y mi carne sois: ¿por qué pues seréis vosotros los postreros en volver
al rey? \bibverse{13} Asimismo diréis á Amasa: ¿No eres tú también hueso
mío y carne mía? Así me haga Dios, y así me añada, si no fueres general
del ejército delante de mí para siempre, en lugar de Joab. \bibverse{14}
Así inclinó el corazón de todos los varones de Judá, como el de un solo
hombre, para que enviasen á decir al rey: Vuelve tú, y todos tus
siervos. \bibverse{15} Volvió pues el rey, y vino hasta el Jordán. Y
Judá vino á Gilgal, á recibir al rey y pasarlo el Jordán. \bibverse{16}
Y Semei hijo de Gera, hijo de Benjamín, que era de Bahurim, dióse priesa
á venir con los hombres de Judá á recibir al rey David; \bibverse{17} Y
con él venían mil hombres de Benjamín; asimismo Siba criado de la casa
de Saúl, con sus quince hijos y sus veinte siervos, los cuales pasaron
el Jordán delante del rey. \bibverse{18} Atravesó después la barca para
pasar la familia del rey, y para hacer lo que le pluguiera. Entonces
Semei hijo de Gera se postró delante del rey cuando él había pasado el
Jordán. \bibverse{19} Y dijo al rey: No me impute mi señor iniquidad, ni
tengas memoria de los males que tu siervo hizo el día que mi señor el
rey salió de Jerusalem, para guardarlos el rey en su corazón;
\bibverse{20} Porque yo tu siervo conozco haber pecado, y he venido hoy
el primero de toda la casa de José, para descender á recibir á mi señor
el rey. \bibverse{21} Y Abisai hijo de Sarvia respondió y dijo: ¿No ha
de morir por esto Semei, que maldijo al ungido de Jehová? \bibverse{22}
David entonces dijo: ¿Qué tenéis vosotros conmigo, hijos de Sarvia, que
me habéis de ser hoy adversarios? ¿ha de morir hoy alguno en Israel? ¿no
conozco yo que hoy soy rey sobre Israel? \bibverse{23} Y dijo el rey á
Semei: No morirás. Y el rey se lo juró. \bibverse{24} También
Mephi-boseth hijo de Saúl descendió á recibir al rey: no había lavado
sus pies, ni había cortado su barba, ni tampoco había lavado sus
vestidos, desde el día que el rey salió hasta el día que vino en paz.
\bibverse{25} Y luego que vino él á Jerusalem á recibir al rey, el rey
le dijo: Mephi-boseth, ¿por qué no fuiste conmigo? \bibverse{26} Y él
dijo: Rey señor mío, mi siervo me ha engañado; pues había tu siervo
dicho: Enalbardaré un asno, y subiré en él, é iré al rey; porque tu
siervo es cojo. \bibverse{27} Empero él revolvió á tu siervo delante de
mi señor el rey; mas mi señor el rey es como un ángel de Dios: haz pues
lo que bien te pareciere. \bibverse{28} Porque toda la casa de mi padre
era digna de muerte delante de mi señor el rey, y tú pusiste á tu siervo
entre los convidados de tu mesa. ¿Qué derecho pues tengo aún para
quejarme más contra el rey? \bibverse{29} Y el rey le dijo: ¿Para qué
hablas más palabras? Yo he determinado que tú y Siba partáis las
tierras. \bibverse{30} Y Mephi-boseth dijo al rey: Y aun tómelas él
todas, pues que mi señor el rey ha vuelto en paz á su casa.
\bibverse{31} También Barzillai Galaadita descendió de Rogelim, y pasó
el Jordán con el rey, para acompañarle de la otra parte del Jordán.
\bibverse{32} Y era Barzillai muy viejo, de ochenta años, el cual había
dado provisión al rey cuando estaba en Mahanaim, porque era hombre muy
rico. \bibverse{33} Y el rey dijo á Barzillai: Pasa conmigo, y yo te
daré de comer conmigo en Jerusalem. \bibverse{34} Mas Barzillai dijo al
rey: ¿Cuántos son los días del tiempo de mi vida, para que yo suba con
el rey á Jerusalem? \bibverse{35} Yo soy hoy día de edad de ochenta
años, que ya no haré diferencia entre lo bueno y lo malo: ¿tomará gusto
ahora tu siervo en lo que comiere ó bebiere? ¿oiré más la voz de los
cantores y de las cantoras? ¿para qué, pues, sería aún tu siervo molesto
á mi señor el rey? \bibverse{36} Pasará tu siervo un poco el Jordán con
el rey: ¿por qué me ha de dar el rey tan grande recompensa?
\bibverse{37} Yo te ruego que dejes volver á tu siervo, y que muera en
mi ciudad, junto al sepulcro de mi padre y de mi madre. He aquí tu
siervo Chimham; que pase él con mi señor el rey, y hazle lo que bien te
pareciere. \bibverse{38} Y el rey dijo: Pues pase conmigo Chimham, y yo
haré con él como bien te parezca: y todo lo que tú pidieres de mí, yo lo
haré. \bibverse{39} Y todo el pueblo pasó el Jordán: y luego que el rey
hubo también pasado, el rey besó á Barzillai, y bendíjolo; y él se
volvió á su casa. \bibverse{40} El rey entonces pasó á Gilgal, y con él
pasó Chimham; y todo el pueblo de Judá, con la mitad del pueblo de
Israel, pasaron al rey. \bibverse{41} Y he aquí todos los varones de
Israel vinieron al rey, y le dijeron: ¿Por qué los hombres de Judá,
nuestros hermanos, te han llevado, y han hecho pasar el Jordán al rey y
á su familia, y á todos los varones de David con él? \bibverse{42} Y
todos los varones de Judá respondieron á todos los de Israel: Porque el
rey es nuestro pariente. Mas ¿por qué os enojáis vosotros de eso? ¿hemos
nosotros comido algo del rey? ¿hemos recibido de él algún don?
\bibverse{43} Entonces respondieron los varones de Israel, y dijeron á
los de Judá: Nosotros tenemos en el rey diez partes, y en el mismo David
más que vosotros: ¿por qué pues nos habéis tenido en poco? ¿no hablamos
nosotros primero en volver á nuestro rey? Y el razonamiento de los
varones de Judá fué más fuerte que el de los varones de Israel.

\hypertarget{section-19}{%
\section{20}\label{section-19}}

\bibverse{1} Y acaeció estar allí un hombre perverso que se llamaba
Seba, hijo de Bichri, hombre de Benjamín, el cual tocó la corneta, y
dijo: No tenemos nosotros parte en David, ni heredad en el hijo de Isaí:
Israel, ¡cada uno á sus estancias! \bibverse{2} Así se fueron de en pos
de David todos los hombres de Israel, y seguían á Seba hijo de Bichri:
mas los de Judá fueron adheridos á su rey, desde el Jordán hasta
Jerusalem. \bibverse{3} Y luego que llegó David á su casa en Jerusalem,
tomó el rey las diez mujeres concubinas que había dejado para guardar la
casa, y púsolas en una casa en guarda, y dióles de comer: pero nunca más
entró á ellas, sino que quedaron encerradas hasta que murieron en viudez
de por vida. \bibverse{4} Después dijo el rey á Amasa: Júntame los
varones de Judá para dentro de tres días, y hállate tú aquí presente.
\bibverse{5} Fué pues Amasa á juntar á Judá; pero detúvose más del
tiempo que le había sido señalado. \bibverse{6} Y dijo David á Abisai:
Seba hijo de Bichri nos hará ahora más mal que Absalom: toma pues tú los
siervos de tu señor, y ve tras él, no sea que halle las ciudades
fortificadas, y se nos vaya de delante. \bibverse{7} Entonces salieron
en pos de él los hombres de Joab, y los Ceretheos y Peletheos, y todos
los valientes: salieron de Jerusalem para ir tras Seba hijo de Bichri.
\bibverse{8} Y estando ellos cerca de la grande peña que está en Gabaón,
salióles Amasa al encuentro. Ahora bien, la vestidura que Joab tenía
sobrepuesta estábale ceñida, y sobre ella el cinto de una daga pegada á
sus lomos en su vaina, de la que así como él avanzó, cayóse aquélla.
\bibverse{9} Entonces Joab dijo á Amasa: ¿Tienes paz, hermano mío? Y
tomó Joab con la diestra la barba de Amasa, para besarlo. \bibverse{10}
Y como Amasa no se cuidó de la daga que Joab en la mano tenía, hirióle
éste con ella en la quinta costilla, y derramó sus entrañas por tierra,
y cayó muerto sin darle segundo golpe. Después Joab y su hermano Abisai
fueron en seguimiento de Seba hijo de Bichri. \bibverse{11} Y uno de los
criados de Joab se paró junto á él, diciendo: Cualquiera que amare á
Joab y á David vaya en pos de Joab. \bibverse{12} Y Amasa se había
revolcado en la sangre en mitad del camino: y viendo aquel hombre que
todo el pueblo se paraba, apartó á Amasa del camino al campo, y echó
sobre él una vestidura, porque veía que todos los que venían se paraban
junto á él. \bibverse{13} Luego, pues, que fué apartado del camino,
pasaron todos los que seguían á Joab, para ir tras Seba hijo de Bichri.
\bibverse{14} Y él pasó por todas las tribus de Israel hasta Abel y
Beth-maachâ y todo Barim: y juntáronse, y siguiéronlo también.
\bibverse{15} Y vinieron y cercáronlo en Abel de Beth-maachâ, y pusieron
baluarte contra la ciudad; y puesto que fué al muro, todo el pueblo que
estaba con Joab trabajaba por derribar la muralla. \bibverse{16}
Entonces una mujer sabia dió voces en la ciudad, diciendo: Oid, oid;
ruégoos que digáis á Joab se llegue á acá, para que yo hable con él.
\bibverse{17} Y como él se acercó á ella, dijo la mujer: ¿Eres tú Joab?
Y él respondió: Yo soy. Y ella le dijo: Oye las palabras de tu sierva. Y
él respondió: Oigo. \bibverse{18} Entonces tornó ella á hablar,
diciendo: Antiguamente solían hablar, diciendo: Quien preguntare,
pregunte en Abel: y así concluían. \bibverse{19} Yo soy de las pacíficas
y fieles de Israel: y tú procuras destruir una ciudad que es madre de
Israel: ¿por qué destruyes la heredad de Jehová? \bibverse{20} Y Joab
respondió, diciendo: Nunca tal, nunca tal me acontezca, que yo destruya
ni deshaga. \bibverse{21} La cosa no es así: mas un hombre del monte de
Ephraim, que se llama Seba hijo de Bichri, ha levantado su mano contra
el rey David: entregad á ése solamente, y me iré de la ciudad. Y la
mujer dijo á Joab: He aquí su cabeza te será echada desde el muro.
\bibverse{22} La mujer fué luego á todo el pueblo con su sabiduría; y
ellos cortaron la cabeza á Seba hijo de Bichri, y echáronla á Joab. Y él
tocó la corneta, y esparciéronse de la ciudad, cada uno á su estancia. Y
Joab se volvió al rey á Jerusalem. \bibverse{23} Así quedó Joab sobre
todo el ejército de Israel; y Benaía hijo de Joiada sobre los Ceretheos
y Peletheos; \bibverse{24} Y Adoram sobre los tributos; y Josaphat hijo
de Ahillud, el canciller; \bibverse{25} Y Seba, escriba; y Sadoc y
Abiathar, sacerdotes; é Ira Jaireo fué un jefe principal cerca de David.
\bibverse{26}

\hypertarget{section-20}{%
\section{21}\label{section-20}}

\bibverse{1} Y en los días de David hubo hambre por tres años
consecutivos. Y David consultó á Jehová, y Jehová le dijo: Es por Saúl,
y por aquella casa de sangre; porque mató á los Gabaonitas. \bibverse{2}
Entonces el rey llamó á los Gabaonitas, y hablóles. (Los Gabaonitas no
eran de los hijos de Israel, sino del residuo de los Amorrheos, á los
cuales los hijos de Israel habían hecho juramento: mas Saúl había
procurado matarlos con motivo de celo por los hijos de Israel y de
Judá.) \bibverse{3} Dijo pues David á los Gabaonitas: ¿Qué os haré, y
con qué expiaré para que bendigáis á la heredad de Jehová? \bibverse{4}
Y los Gabaonitas le respondieron: No tenemos nosotros querella sobre
plata ni sobre oro con Saúl y con su casa: ni queremos que muera hombre
de Israel. Y él les dijo: Lo que vosotros dijereis os haré. \bibverse{5}
Y ellos respondieron al rey: De aquel hombre que nos destruyó, y que
maquinó contra nosotros, para extirparnos sin dejar nada de nosotros en
todo el término de Israel; \bibverse{6} Dénsenos siete varones de sus
hijos, para que los ahorquemos á Jehová en Gabaa de Saúl, el escogido de
Jehová. Y el rey dijo: Yo los daré. \bibverse{7} Y perdonó el rey á
Mephi-boseth, hijo de Jonathán, hijo de Saúl, por el juramento de Jehová
que hubo entre ellos, entre David y Jonathán hijo de Saúl. \bibverse{8}
Mas tomó el rey dos hijos de Rispa hija de Aja, los cuales ella había
parido á Saúl, á saber, á Armoni y á Mephi-boseth; y cinco hijos de
Michâl hija de Saúl, los cuales ella había parido á Adriel, hijo de
Barzillai Molathita; \bibverse{9} Y entrególos en manos de los
Gabaonitas, y ellos los ahorcaron en el monte delante de Jehová: y
murieron juntos aquellos siete, los cuales fueron muertos en el tiempo
de la siega, en los primeros días, en el principio de la siega de las
cebadas. \bibverse{10} Tomando luego Rispa hija de Aja un saco,
tendióselo sobre un peñasco, desde el principio de la siega hasta que
llovió sobre ellos agua del cielo; y no dejó á ninguna ave del cielo
asentarse sobre ellos de día, ni bestias del campo de noche.
\bibverse{11} Y fué dicho á David lo que hacía Rispa hija de Aja,
concubina de Saúl. \bibverse{12} Entonces David fué, y tomó los huesos
de Saúl y los huesos de Jonathán su hijo, de los hombres de Jabes de
Galaad, que los habían hurtado de la plaza de Beth-san, donde los habían
colgado los Filisteos, cuando deshicieron los Filisteos á Saúl en
Gilboa: \bibverse{13} E hizo llevar de allí los huesos de Saúl y los
huesos de Jonathán su hijo; y juntaron también los huesos de los
ahorcados. \bibverse{14} Y sepultaron los huesos de Saúl y los de su
hijo Jonathán en tierra de Benjamín, en Sela, en el sepulcro de Cis su
padre; é hicieron todo lo que el rey había mandado. Después se aplacó
Dios con la tierra. \bibverse{15} Y como los Filisteos tornaron á hacer
guerra á Israel, descendió David y sus siervos con él, y pelearon con
los Filisteos: y David se cansó. \bibverse{16} En esto Isbi-benob, el
cual era de los hijos del gigante, y el peso de cuya lanza era de
trescientos siclos de metal, y tenía él ceñida una nueva espada, trató
de herir á David: \bibverse{17} Mas Abisai hijo de Sarvia le socorrió, é
hirió al Filisteo, y matólo. Entonces los hombres de David le juraron,
diciendo: Nunca más de aquí adelante saldrás con nosotros á batalla,
porque no apagues la lámpara de Israel. \bibverse{18} Otra segunda
guerra hubo después en Gob contra los Filisteos: entonces Sibechâi
Husathita hirió á Saph, que era de los hijos del gigante. \bibverse{19}
Otra guerra hubo en Gob contra los Filisteos, en la cual Elhanan, hijo
de Jaare-oregim de Beth-lehem, hirió á Goliath Getheo, el asta de cuya
lanza era como un enjullo de telar. \bibverse{20} Después hubo otra
guerra en Gath, donde hubo un hombre de grande altura, el cual tenía
doce dedos en las manos, y otros doce en los pies, veinticuatro en
todos: y también era de los hijos del gigante. \bibverse{21} Este
desafió á Israel, y matólo Jonathán, hijo de Sima hermano de David.
\bibverse{22} Estos cuatro le habían nacido al gigante en Gath, los
cuales cayeron por la mano de David, y por la mano de sus siervos.

\hypertarget{section-21}{%
\section{22}\label{section-21}}

\bibverse{1} Y habló David á Jehová las palabras de este cántico, el día
que Jehová le había librado de la mano de todos sus enemigos, y de la
mano de Saúl. \bibverse{2} Y dijo: Jehová es mi roca, y mi fortaleza, y
mi libertador; \bibverse{3} Dios de mi roca, en él confiaré: mi escudo,
y el cuerno de mi salud, mi fortaleza, y mi refugio; mi salvador, que me
librarás de violencia. \bibverse{4} Invocaré á Jehová, digno de ser
loado, y seré salvo de mis enemigos. \bibverse{5} Cuando me cercaron
ondas de muerte, y arroyos de iniquidad me asombraron, \bibverse{6} Me
rodearon los dolores del infierno, y me tomaron descuidado lazos de
muerte. \bibverse{7} Tuve angustia, invoqué á Jehová, y clamé á mi Dios:
y él oyó mi voz desde su templo; llegó mi clamor á sus oídos.
\bibverse{8} La tierra se removió, y tembló; los fundamentos de los
cielos fueron movidos, y se estremecieron, porque él se airó.
\bibverse{9} Subió humo de sus narices, y de su boca fuego consumidor,
por el cual se encendieron carbones. \bibverse{10} Y abajó los cielos, y
descendió: una oscuridad debajo de sus pies. \bibverse{11} Subió sobre
el querubín, y voló: aparecióse sobre las alas del viento. \bibverse{12}
Puso tinieblas alrededor de sí á modo de pabellones; aguas negras y
espesas nubes. \bibverse{13} Del resplandor de su presencia se
encendieron ascuas ardientes. \bibverse{14} Jehová tronó desde los
cielos, y el Altísimo dió su voz; \bibverse{15} Arrojó saetas, y
desbaratólos; relampagueó, y consumiólos. \bibverse{16} Entonces
aparecieron los manantiales de la mar, y los fundamentos del mundo
fueron descubiertos, á la reprensión de Jehová, al resoplido del aliento
de su nariz. \bibverse{17} Extendió su mano de lo alto, y arrebatóme, y
sacóme de copiosas aguas. \bibverse{18} Libróme de fuertes enemigos, de
aquellos que me aborrecían, los cuales eran más fuertes que yo.
\bibverse{19} Asaltáronme en el día de mi calamidad; mas Jehová fué mi
sostén. \bibverse{20} Sacóme á anchura; libróme, porque puso su voluntad
en mí. \bibverse{21} Remuneróme Jehová conforme á mi justicia; y
conforme á la limpieza de mis manos, me dió la paga. \bibverse{22}
Porque yo guardé los caminos de Jehová, y no me aparté impíamente de mi
Dios. \bibverse{23} Porque delante de mí tengo todas sus ordenanzas; y
atento á sus fueros, no me retiraré de ellos. \bibverse{24} Y fuí
íntegro para con él, y guardéme de mi iniquidad. \bibverse{25}
Remuneróme por tanto Jehová conforme á mi justicia, y conforme á mi
limpieza delante de sus ojos. \bibverse{26} Con el bueno eres benigno, y
con el íntegro te muestras íntegro; \bibverse{27} Limpio eres para con
el limpio, mas con el perverso eres rígido. \bibverse{28} Y tú salvas al
pueblo humilde; mas tus ojos sobre los altivos, para abatirlos.
\bibverse{29} Porque tú eres mi lámpara, oh Jehová: Jehová da luz á mis
tinieblas. \bibverse{30} Porque en ti romperé ejércitos, y con mi Dios
saltaré las murallas. \bibverse{31} Dios, perfecto su camino: la palabra
de Jehová purificada, escudo es de todos los que en él esperan.
\bibverse{32} Porque ¿qué Dios hay sino Jehová? ¿ó quién es fuerte sino
nuestro Dios? \bibverse{33} Dios es el que con virtud me corrobora, y el
que despeja mi camino; \bibverse{34} El que hace mis pies como de
ciervas, y el que me asienta en mis alturas; \bibverse{35} El que enseña
mis manos para la pelea, y da que con mis brazos quiebre el arco de
acero. \bibverse{36} Tú me diste asimismo el escudo de tu salud, y tu
benignidad me ha acrecentado. \bibverse{37} Tú ensanchaste mis pasos
debajo de mí, para que no titubeasen mis rodillas. \bibverse{38}
Perseguiré á mis enemigos, y quebrantarélos; y no me volveré hasta que
los acabe. \bibverse{39} Los consumiré, y los heriré, y no se
levantarán; y caerán debajo de mis pies. \bibverse{40} Ceñísteme de
fortaleza para la batalla, y postraste debajo de mí los que contra mí se
levantaron. \bibverse{41} Tú me diste la cerviz de mis enemigos, de mis
aborrecedores, y que yo los destruyese. \bibverse{42} Miraron, y no hubo
quien los librase; á Jehová, mas no les respondió. \bibverse{43} Yo los
desmenuzaré como polvo de la tierra; hollarélos como á lodo de las
plazas, y los disiparé. \bibverse{44} Tú me libraste de contiendas de
pueblos: tú me guardaste para que fuese cabeza de gentes: pueblos que no
conocía, me sirvieron. \bibverse{45} Los extraños titubeaban á mí: en
oyendo, me obedecían. \bibverse{46} Los extraños desfallecían, y
temblaban en sus escondrijos. \bibverse{47} Viva Jehová, y sea bendita
mi roca; sea ensalzado el Dios, la roca de mi salvamento: \bibverse{48}
El Dios que me ha vengado, y sujeta los pueblos debajo de mí;
\bibverse{49} Y que me saca de entre mis enemigos: tú me sacaste en alto
de entre los que se levantaron contra mí: librásteme del varón de
iniquidades. \bibverse{50} Por tanto yo te confesaré entre las gentes,
oh Jehová, y cantaré á tu nombre. \bibverse{51} El que engrandece las
saludes de su rey, y hace misericordia á su ungido, á David, y á su
simiente, para siempre.

\hypertarget{section-22}{%
\section{23}\label{section-22}}

\bibverse{1} Estas son las postreras palabras de David. Dijo David hijo
de Isaí, dijo aquel varón que fué levantado alto, el ungido del Dios de
Jacob, el suave en cánticos de Israel: \bibverse{2} El espíritu de
Jehová ha hablado por mí, y su palabra ha sido en mi lengua.
\bibverse{3} El Dios de Israel ha dicho, hablóme el Fuerte de Israel: El
señoreador de los hombres será justo, señoreador en temor de Dios.
\bibverse{4} Será como la luz de la mañana cuando sale el sol, de la
mañana sin nubes; cuando la hierba de la tierra brota por medio del
resplandor después de la lluvia. \bibverse{5} No así mi casa para con
Dios: sin embargo él ha hecho conmigo pacto perpetuo, ordenado en todas
las cosas, y será guardado; bien que toda esta mi salud, y todo mi deseo
no lo haga él florecer todavía. \bibverse{6} Mas los de Belial serán
todos ellos como espinas arrancadas, las cuales nadie toma con la mano;
\bibverse{7} Sino que el que quiere tocar en ellas, ármase de hierro y
de asta de lanza, y son quemadas en su lugar. \bibverse{8} Estos son los
nombres de los valientes que tuvo David: El Tachmonita, que se sentaba
en cátedra, principal de los capitanes: era éste Adino el Eznita, que
mató en una ocasión sobre ochocientos hombres. \bibverse{9} Después de
éste, Eleazar, hijo de Dodo de Ahohi, fué de los tres valientes que
estaban con David, cuando desafiaron á los Filisteos que se habían
juntado allí á la batalla, y subieron los de Israel. \bibverse{10} Este,
levantándose, hirió á los Filisteos hasta que su mano se cansó, y
quedósele contraída á la espada. Aquel día Jehová hizo gran salud: y
volvióse el pueblo en pos de él solamente á tomar el despojo.
\bibverse{11} Después de éste fué Samma, hijo de Age Araita: que
habiéndose juntado los Filisteos en una aldea, había allí una suerte de
tierra llena de lentejas, y el pueblo había huído delante de los
Filisteos: \bibverse{12} El entonces se paró en medio de la suerte de
tierra, y defendióla, é hirió á los Filisteos; y Jehová hizo una gran
salud. \bibverse{13} Y tres de los treinta principales descendieron y
vinieron en tiempo de la siega á David á la cueva de Adullam: y el campo
de los Filisteos estaba en el valle de Raphaim. \bibverse{14} David
entonces estaba en la fortaleza, y la guarnición de los Filisteos estaba
en Beth-lehem. \bibverse{15} Y David tuvo deseo, y dijo: ¡Quién me diera
á beber del agua de la cisterna de Beth-lehem, que está á la puerta!
\bibverse{16} Entonces los tres valientes rompieron por el campo de los
Filisteos, y sacaron agua de la cisterna de Beth-lehem, que estaba á la
puerta; y tomaron, y trajéronla á David: mas él no la quiso beber, sino
derramóla á Jehová, diciendo: \bibverse{17} Lejos sea de mí, oh Jehová,
que yo haga esto. ¿He de beber yo la sangre de los varones que fueron
con peligro de su vida? Y no quiso beberla. Los tres valientes hicieron
esto. \bibverse{18} Y Abisai hermano de Joab, hijo de Sarvia, fué el
principal de los tres; el cual alzó su lanza contra trescientos, que
mató; y tuvo nombre entre los tres. \bibverse{19} El era el más
aventajado de los tres, y el primero de ellos; mas no llegó á los tres
primeros. \bibverse{20} Después, Benaía hijo de Joiada, hijo de un varón
esforzado, grande en hechos, de Cabseel. Este hirió dos leones de Moab:
y él mismo descendió, é hirió un león en medio de un foso en el tiempo
de la nieve: \bibverse{21} También hirió él á un Egipcio, hombre de
grande estatura: y tenía el Egipcio una lanza en su mano; mas descendió
á él con un palo, y arrebató al Egipcio la lanza de la mano, y matólo
con su propia lanza. \bibverse{22} Esto hizo Benaía hijo de Joiada, y
tuvo nombre entre los tres valientes. \bibverse{23} De los treinta fué
el más aventajado; pero no llegó á los tres primeros. Y púsolo David en
su consejo. \bibverse{24} Asael hermano de Joab fué de los treinta;
Elhaanan hijo de Dodo de Beth-lehem; \bibverse{25} Samma de Harodi,
Elica de Harodi; \bibverse{26} Heles de Palti, Hira, hijo de Jecces, de
Tecoa; \bibverse{27} Abiezer de Anathoth, Mebunnai de Husa;
\bibverse{28} Selmo de Hahoh, Maharai de Netophath; \bibverse{29} Helec
hijo de Baana de Netophath, Ittai hijo de Ribai de Gabaa de los hijos de
Benjamín; \bibverse{30} Benaía Pirathonita, Hiddai del arroyo de Gaas;
\bibverse{31} Abi-albon de Arbath, Asmaveth de Barhum; \bibverse{32}
Elihaba de Saalbón, Jonathán de los hijos de Jassén; \bibverse{33} Samma
de Arar, Ahiam hijo de Sarar de Arar. \bibverse{34} Elipheleth hijo de
Asbai hijo de Maachâti; Eliam hijo de Achîtophel de Gelón; \bibverse{35}
Hesrai del Carmelo, Pharai de Arbi; \bibverse{36} Igheal hijo de Nathán
de Soba, Bani de Gadi; \bibverse{37} Selec de Ammón, Naharai de Beeroth,
escudero de Joab hijo de Sarvia; \bibverse{38} Ira de Ithri, Gareb de
Ithri; \bibverse{39} Uría Hetheo. Entre todos treinta y siete.

\hypertarget{section-23}{%
\section{24}\label{section-23}}

\bibverse{1} Y volvió el furor de Jehová á encenderse contra Israel, é
incitó á David contra ellos á que dijese: Ve, cuenta á Israel y á Judá.
\bibverse{2} Y dijo el rey á Joab, general del ejército que tenía
consigo: Rodea todas las tribus de Israel, desde Dan hasta Beer-seba, y
contad el pueblo, para que yo sepa el número de la gente. \bibverse{3} Y
Joab respondió al rey: Añada Jehová tu Dios al pueblo cien veces tanto
como son, y que lo vea mi señor el rey; mas ¿para qué quiere esto mi
señor el rey? \bibverse{4} Empero la palabra del rey pudo más que Joab,
y que los capitanes del ejército. Salió pues Joab, con los capitanes del
ejército, de delante del rey, para contar el pueblo de Israel.
\bibverse{5} Y pasando el Jordán asentaron en Aroer, á la mano derecha
de la ciudad que está en medio de la arroyada de Gad y junto á Jazer.
\bibverse{6} Después vinieron á Galaad, y á la tierra baja de Absi: y de
allí vinieron á Dan-jaán y alrededor de Sidón. \bibverse{7} Y vinieron
luego á la fortaleza de Tiro, y á todas las ciudades de los Heveos y de
los Cananeos; y salieron al mediodía de Judá, á Beer-seba. \bibverse{8}
Y después que hubieron andado toda la tierra, volvieron á Jerusalem al
cabo de nueve meses y veinte días. \bibverse{9} Y Joab dió la cuenta del
número del pueblo al rey; y fueron los de Israel ochocientos mil hombres
fuertes que sacaban espada; y de los de Judá quinientos mil hombres.
\bibverse{10} Y después que David hubo contado el pueblo, punzóle su
corazón; y dijo David á Jehová: Yo he pecado gravemente por haber hecho
esto; mas ahora, oh Jehová, ruégote que quites el pecado de tu siervo,
porque yo he obrado muy neciamente. \bibverse{11} Y por la mañana,
cuando David se hubo levantado, fué palabra de Jehová á Gad profeta,
vidente de David, diciendo: \bibverse{12} Ve, y di á David: Así ha dicho
Jehová: Tres cosas te ofrezco: tú te escogerás una de ellas, la cual yo
haga. \bibverse{13} Vino pues Gad á David, é intimóle, y díjole:
¿Quieres que te vengan siete años de hambre en tu tierra? ¿ó que huyas
tres meses delante de tus enemigos, y que ellos te persigan? ¿ó que tres
días haya pestilencia en tu tierra? Piensa ahora, y mira qué responderé
al que me ha enviado. \bibverse{14} Entonces David dijo á Gad: En grande
angustia estoy: ruego que caiga en la mano de Jehová, porque sus
miseraciones son muchas, y que no caiga yo en manos de hombres.
\bibverse{15} Y envió Jehová pestilencia á Israel desde la mañana hasta
el tiempo señalado: y murieron del pueblo, desde Dan hasta Beer-seba,
setenta mil hombres. \bibverse{16} Y como el ángel extendió su mano
sobre Jerusalem para destruirla, Jehová se arrepintió de aquel mal, y
dijo al ángel que destruía el pueblo: Basta ahora; detén tu mano.
Entonces el ángel de Jehová estaba junto á la era de Arauna Jebuseo.
\bibverse{17} Y David dijo á Jehová, cuando vió al ángel que hería al
pueblo: Yo pequé, yo hice la maldad: ¿qué hicieron estas ovejas? Ruégote
que tu mano se torne contra mí, y contra la casa de mi padre.
\bibverse{18} Y Gad vino á David aquel día, y díjole: Sube, y haz un
altar á Jehová en la era de Arauna Jebuseo. \bibverse{19} Y subió David,
conforme al dicho de Gad, que Jehová le había mandado. \bibverse{20} Y
mirando Arauna, vió al rey y á sus siervos que pasaban á él. Saliendo
entonces Arauna, inclinóse delante del rey hacia tierra. \bibverse{21} Y
Arauna dijo: ¿Por qué viene mi señor el rey á su siervo? Y David
respondió: Para comprar de ti la era, para edificar altar á Jehová, á
fin de que la mortandad cese del pueblo. \bibverse{22} Y Arauna dijo á
David: Tome y sacrifique mi señor el rey lo que bien le pareciere; he
aquí bueyes para el holocausto; y trillos y otros pertrechos de bueyes
para leña: \bibverse{23} Todo lo da como un rey Arauna al rey. Luego
dijo Arauna al rey: Jehová tu Dios te sea propicio. \bibverse{24} Y el
rey dijo á Arauna: No, sino por precio te lo compraré; porque no
ofreceré á Jehová mi Dios holocaustos por nada. Entonces David compró la
era y los bueyes por cincuenta siclos de plata. \bibverse{25} Y edificó
allí David un altar á Jehová, y sacrificó holocaustos y pacíficos; y
Jehová se aplacó con la tierra, y cesó la plaga de Israel.
