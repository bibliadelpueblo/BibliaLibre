\hypertarget{section}{%
\section{1}\label{section}}

\bibverse{1} Paulus, ein Knecht Jesu Christi, berufen zum Apostel,
ausgesondertx-morph=``strongMorph:TG5772'', zu predigen das Evangelium
Gottes, \bibverse{2} welches er zuvor verheißen
hatx-morph=``strongMorph:TG5662'' durch seine Propheten in der heiligen
Schrift, \bibverse{3} von seinem Sohn, der geboren
istx-morph=``strongMorph:TG5637'' von dem Samen Davids nach dem Fleisch
\bibverse{4} und kräftig erwiesenx-morph=``strongMorph:TG5685'' als ein
Sohn Gottes nach dem Geist, der da heiligt, seit der Zeit, da er
auferstanden ist von den Toten, Jesus Christus, unser HERR, \bibverse{5}
durch welchen wir haben empfangenx-morph=``strongMorph:TG5627'' Gnade
und Apostelamt, unter allen Heiden den Gehorsam des Glaubens
aufzurichten unter seinem Namen, \bibverse{6} unter welchen ihr auch
seidx-morph=``strongMorph:TG5748'', die da berufen sind von Jesu
Christo, \bibverse{7} allen, die zu Rom
sindx-morph=``strongMorph:TG5752'', den Liebsten Gottes und berufenen
Heiligen: Gnade sei mit euch und Friede von Gott, unserm Vater, und dem
HERRN Jesus Christus! \bibverse{8} Aufs erste
dankex-morph=``strongMorph:TG5719'' ich meinem Gott durch Jesum Christum
euer aller halben, daß man von eurem Glauben in aller Welt
sagtx-morph=``strongMorph:TG5743''. \bibverse{9} Denn Gott
istx-morph=``strongMorph:TG5748'' mein Zeuge, welchem ich
dienex-morph=``strongMorph:TG5719'' in meinem Geist am Evangelium von
seinem Sohn, daß ich ohne Unterlaß euer
gedenkex-morph=``strongMorph:TG5731'' \bibverse{10} und allezeit in
meinem Gebet flehex-morph=``strongMorph:TG5740'', ob sich's einmal
zutragen wolltex-morph=``strongMorph:TG5701'', daß ich zu euch
kämex-morph=``strongMorph:TG5629'' durch Gottes Willen. \bibverse{11}
Denn mich verlangtx-morph=``strongMorph:TG5719'', euch zu
sehenx-morph=``strongMorph:TG5629'', auf daß ich euch
mitteilex-morph=``strongMorph:TG5632'' etwas geistlicher Gabe, euch zu
stärkenx-morph=``strongMorph:TG5683''; \bibverse{12} das
istx-morph=``strongMorph:TG5748'', daß ich samt euch getröstet
würdex-morph=``strongMorph:TG5683'' durch euren und meinen Glauben, den
wir untereinander haben. \bibverse{13} Ich
willx-morph=``strongMorph:TG5719'' euch aber nicht
verhaltenx-morph=``strongMorph:TG5721'', liebe Brüder, daß ich mir oft
habe vorgesetztx-morph=``strongMorph:TG5639'', zu euch zu
kommenx-morph=``strongMorph:TG5629'' (bin aber
verhindertx-morph=``strongMorph:TG5681''
bisherx-morph=``strongMorph:TG5773''), daß ich auch unter euch Frucht
schafftex-morph=``strongMorph:TG5632'' gleichwie unter andern Heiden.
\bibverse{14} Ich binx-morph=``strongMorph:TG5748'' ein Schuldner der
Griechen und der Ungriechen, der Weisen und der Unweisen. \bibverse{15}
Darum, soviel an mir ist, bin ich geneigt, auch euch zu Rom das
Evangelium zu predigenx-morph=``strongMorph:TG5670''. \bibverse{16} Denn
ich schäme michx-morph=``strongMorph:TG5736'' des Evangeliums von
Christo nicht; denn es istx-morph=``strongMorph:TG5748'' eine Kraft
Gottes, die da selig macht alle, die daran
glaubenx-morph=``strongMorph:TG5723'', die Juden vornehmlich und auch
die Griechen. \bibverse{17} Sintemal darin offenbart
wirdx-morph=``strongMorph:TG5743'' die Gerechtigkeit, die vor Gott gilt,
welche kommt aus Glauben in Glauben; wie denn geschrieben
stehtx-morph=``strongMorph:TG5769'': ``Der Gerechte wird seines Glaubens
lebenx-morph=''strongMorph:TG5695``.'' \bibverse{18} Denn Gottes Zorn
vom Himmel wird offenbartx-morph=``strongMorph:TG5743'' über alles
gottlose Wesen und Ungerechtigkeit der Menschen, die die Wahrheit in
Ungerechtigkeit aufhaltenx-morph=``strongMorph:TG5723''. \bibverse{19}
Denn was man von Gott weiß, istx-morph=``strongMorph:TG5748'' ihnen
offenbar; denn Gott hat es ihnen
offenbartx-morph=``strongMorph:TG5656'', \bibverse{20} damit daß Gottes
unsichtbares Wesen, das ist seine ewige Kraft und Gottheit, wird
ersehenx-morph=``strongMorph:TG5746'', so man des
wahrnimmtx-morph=``strongMorph:TG5743'', an den Werken, nämlich an der
Schöpfung der Welt; also daß sie keine Entschuldigung
habenx-morph=``strongMorph:TG5750'', \bibverse{21} dieweil sie
wußtenx-morph=``strongMorph:TG5631'', daß ein Gott ist, und haben ihn
nicht gepriesenx-morph=``strongMorph:TG5656'' als einen Gott noch ihm
gedanktx-morph=``strongMorph:TG5656'', sondern sind in ihrem Dichten
eitel gewordenx-morph=``strongMorph:TG5681'', und ihr unverständiges
Herz ist verfinstertx-morph=``strongMorph:TG5681''. \bibverse{22} Da sie
sich fürx-morph=``strongMorph:TG5750'' Weise
hieltenx-morph=``strongMorph:TG5723'', sind sie zu Narren
gewordenx-morph=``strongMorph:TG5681'' \bibverse{23} und haben
verwandeltx-morph=``strongMorph:TG5656'' die Herrlichkeit des
unvergänglichen Gottes in ein Bild gleich dem vergänglichen Menschen und
der Vögel und der vierfüßigen und der kriechenden Tiere. \bibverse{24}
Darum hat sie auch Gott dahingegebenx-morph=``strongMorph:TG5656'' in
ihrer Herzen Gelüste, in Unreinigkeit, zu
schändenx-morph=``strongMorph:TG5729'' ihre eigenen Leiber an sich
selbst, \bibverse{25} sie, die Gottes Wahrheit haben
verwandeltx-morph=``strongMorph:TG5656'' in die Lüge und haben
geehrtx-morph=``strongMorph:TG5662'' und
gedientx-morph=``strongMorph:TG5656'' dem Geschöpfe mehr denn dem
Schöpferx-morph=``strongMorph:TG5660'', der da gelobt
istx-morph=``strongMorph:TG5748'' in Ewigkeit. Amen. \bibverse{26} Darum
hat sie auch Gott dahingegebenx-morph=``strongMorph:TG5656'' in
schändliche Lüste: denn ihre Weiber haben
verwandeltx-morph=``strongMorph:TG5656'' den natürlichen Brauch in den
unnatürlichen; \bibverse{27} desgleichen auch die Männer haben
verlassenx-morph=``strongMorph:TG5631'' den natürlichen Brauch des
Weibes und sind aneinander erhitztx-morph=``strongMorph:TG5681'' in
ihren Lüsten und haben Mann mit Mann Schande
getriebenx-morph=``strongMorph:TG5740'' und den Lohn ihres Irrtums (wie
es denn sein solltex-morph=``strongMorph:TG5713'') an sich selbst
empfangenx-morph=``strongMorph:TG5723''. \bibverse{28} Und gleichwie sie
nicht geachtet habenx-morph=``strongMorph:TG5656'', daß sie Gott
erkennetenx-morph=``strongMorph:TG5721'', hat sie Gott auch
dahingegebenx-morph=``strongMorph:TG5656'' in verkehrten Sinn, zu
tunx-morph=``strongMorph:TG5721'', was nicht
taugtx-morph=``strongMorph:TG5723'', \bibverse{29}
vollx-morph=``strongMorph:TG5772'' alles Ungerechten, Hurerei,
Schalkheit, Geizes, Bosheit, voll Neides, Mordes, Haders, List, giftig,
Ohrenbläser, \bibverse{30} Verleumder, Gottesverächter, Frevler,
hoffärtig, ruhmredig, Schädliche, den Eltern ungehorsam, \bibverse{31}
Unvernünftige, Treulose, Lieblose, unversöhnlich, unbarmherzig.
\bibverse{32} Sie wissenx-morph=``strongMorph:TG5631'' Gottes
Gerechtigkeit, daß, die solches tunx-morph=``strongMorph:TG5723'', des
Todes würdig sindx-morph=``strongMorph:TG5748'', und
tunx-morph=``strongMorph:TG5719'' es nicht allein, sondern haben auch
Gefallenx-morph=``strongMorph:TG5719'' an denen, die es
tunx-morph=``strongMorph:TG5723''.

\hypertarget{section-1}{%
\section{2}\label{section-1}}

\bibverse{1} Darum, o Mensch, kannstx-morph=``strongMorph:TG5748'' du
dich nicht entschuldigen, wer du auch bist, der da
richtetx-morph=``strongMorph:TG5723''. Denn worin du einen andern
richtestx-morph=``strongMorph:TG5719'',
verdammstx-morph=``strongMorph:TG5719'' du dich selbst; sintemal du eben
dasselbe tustx-morph=``strongMorph:TG5719'', was du
richtestx-morph=``strongMorph:TG5723''. \bibverse{2} Denn wir
wissenx-morph=``strongMorph:TG5758'', daß Gottes Urteil
istx-morph=``strongMorph:TG5748'' recht über die, so solches
tunx-morph=``strongMorph:TG5723''. \bibverse{3} Denkst
dux-morph=``strongMorph:TG5736'' aber, o Mensch, der du
richtestx-morph=``strongMorph:TG5723'' die, die solches
tunx-morph=``strongMorph:TG5723'', und
tustx-morph=``strongMorph:TG5723'' auch dasselbe, daß du dem Urteil
Gottes entrinnen werdestx-morph=``strongMorph:TG5695''? \bibverse{4}
Oder verachtest dux-morph=``strongMorph:TG5719'' den Reichtum seiner
Güte, Geduld und Langmütigkeit? Weißtx-morph=``strongMorph:TG5723'' du
nicht, daß dich Gottes Güte zur Buße
leitetx-morph=``strongMorph:TG5719''? \bibverse{5} Du aber nach deinem
verstockten und unbußfertigen Herzen
häufestx-morph=``strongMorph:TG5719'' dir selbst den Zorn auf den Tag
des Zornes und der Offenbarung des gerechten Gerichtes Gottes,
\bibverse{6} welcher geben wirdx-morph=``strongMorph:TG5692'' einem
jeglichen nach seinen Werken: \bibverse{7} Preis und Ehre und
unvergängliches Wesen denen, die mit Geduld in guten Werken
trachtenx-morph=``strongMorph:TG5723'' nach dem ewigen Leben;
\bibverse{8} aber denen, die da zänkisch sind und der Wahrheit nicht
gehorchenx-morph=``strongMorph:TG5723'',
gehorchenx-morph=``strongMorph:TG5734'' aber der Ungerechtigkeit,
Ungnade, und Zorn; \bibverse{9} Trübsal und Angst über alle Seelen der
Menschen, die da Böses tunx-morph=``strongMorph:TG5740'', vornehmlich
der Juden und auch der Griechen; \bibverse{10} Preis aber und Ehre und
Friede allen denen, die da Gutes tunx-morph=``strongMorph:TG5740'',
vornehmlich den Juden und auch den Griechen. \bibverse{11} Denn es
istx-morph=``strongMorph:TG5748'' kein Ansehen der Person vor Gott.
\bibverse{12} Welche ohne Gesetz gesündigt
habenx-morph=``strongMorph:TG5627'', die werden auch ohne Gesetz
verloren werdenx-morph=``strongMorph:TG5698''; und welche unter dem
Gesetz gesündigt habenx-morph=``strongMorph:TG5627'', die werden durchs
Gesetz verurteilt werdenx-morph=``strongMorph:TG5701'' \bibverse{13}
(sintemal vor Gott nicht, die das Gesetz hören, gerecht sind, sondern
die das Gesetz tun, werden gerecht seinx-morph=``strongMorph:TG5701''.
\bibverse{14} Denn so die Heiden, die das Gesetz nicht
habenx-morph=``strongMorph:TG5723'', doch von Natur
tunx-morph=``strongMorph:TG5725'' des Gesetzes Werk,
sindx-morph=``strongMorph:TG5748'' dieselben, dieweil sie das Gesetz
nicht habenx-morph=``strongMorph:TG5723'', sich selbst ein Gesetz,
\bibverse{15} als die da beweisenx-morph=``strongMorph:TG5731'', des
Gesetzes Werk sei geschrieben in ihren Herzen, sintemal ihr Gewissen
ihnen zeugtx-morph=``strongMorph:TG5723'', dazu auch die Gedanken, die
sich untereinander verklagenx-morph=``strongMorph:TG5723'' oder
entschuldigenx-morph=``strongMorph:TG5740''), \bibverse{16} auf den Tag,
da Gott das Verborgene der Menschen durch Jesus Christus richten
wirdx-morph=``strongMorph:TG5692''\textbar x-morph=``strongMorph:TG5719''
laut meines Evangeliums. \bibverse{17} Siehe aber zu: du
heißestx-morph=``strongMorph:TG5743'' ein Jude und
verlässestx-morph=``strongMorph:TG5736'' dich aufs Gesetz und rühmest
dichx-morph=``strongMorph:TG5736'' Gottes \bibverse{18} und
weißtx-morph=``strongMorph:TG5719'' seinen Willen; und weil du aus dem
Gesetz unterrichtet bistx-morph=``strongMorph:TG5746'',
prüfestx-morph=``strongMorph:TG5719'' du, was das Beste zu tun
seix-morph=``strongMorph:TG5723'', \bibverse{19} und
vermissestx-morph=``strongMorph:TG5754'' dich, zu
seinx-morph=``strongMorph:TG5750'' ein Leiter der Blinden, ein Licht
derer, die in Finsternis sind, \bibverse{20} ein Züchtiger der
Törichten, ein Lehrer der Einfältigen,
hastx-morph=``strongMorph:TG5723'' die Form, was zu wissen und recht
ist, im Gesetz. \bibverse{21} Nun lehrstx-morph=``strongMorph:TG5723''
du andere, und lehrst dichx-morph=``strongMorph:TG5719'' selber nicht;
du predigstx-morph=``strongMorph:TG5723'', man solle nicht
stehlenx-morph=``strongMorph:TG5721'', und du
stiehlstx-morph=``strongMorph:TG5719''; \bibverse{22} du
sprichstx-morph=``strongMorph:TG5723'' man solle nicht
ehebrechenx-morph=``strongMorph:TG5721'', und du brichst die
Ehex-morph=``strongMorph:TG5719''; dir
greueltx-morph=``strongMorph:TG5740'' vor den Götzen, und du raubest
Gott, was sein istx-morph=``strongMorph:TG5719''; \bibverse{23} du
rühmst dichx-morph=``strongMorph:TG5736'' des Gesetzes, und
schändestx-morph=``strongMorph:TG5719'' Gott durch Übertretung des
Gesetzes; \bibverse{24} denn ``eurethalben wird Gottes Name
gelästertx-morph=''strongMorph:TG5743" unter den Heiden'', wie
geschrieben stehtx-morph=``strongMorph:TG5769''. \bibverse{25} Die
Beschneidung istx-morph=``strongMorph:TG5719'' wohl nütz, wenn du das
Gesetz hältstx-morph=``strongMorph:TG5725'';
hältstx-morph=``strongMorph:TG5753'' du das Gesetz aber nicht, so bist
du aus einem Beschnittenen schon ein Unbeschnittener
gewordenx-morph=``strongMorph:TG5754''. \bibverse{26} So nun der
Unbeschnittene das Gesetz hältx-morph=``strongMorph:TG5725'', meinst du
nicht, daß da der Unbeschnittene werde für einen Beschnittenen
gerechnetx-morph=``strongMorph:TG5701''? \bibverse{27} Und wird also,
der von Natur unbeschnitten ist und das Gesetz
vollbringtx-morph=``strongMorph:TG5723'', dich
richtenx-morph=``strongMorph:TG5692'', der du unter dem Buchstaben und
der Beschneidung bist und das Gesetz übertrittst. \bibverse{28} Denn das
istx-morph=``strongMorph:TG5748'' nicht ein Jude, der auswendig ein Jude
ist, auch ist das nicht eine Beschneidung, die auswendig am Fleisch
geschieht; \bibverse{29} sondern das ist ein Jude, der's inwendig
verborgen ist, und die Beschneidung des Herzens ist eine Beschneidung,
die im Geist und nicht im Buchstaben geschieht. Eines solchen Lob ist
nicht aus Menschen, sondern aus Gott.

\hypertarget{section-2}{%
\section{3}\label{section-2}}

\bibverse{1} Was haben denn die Juden für Vorteil, oder was nützt die
Beschneidung? \bibverse{2} Fürwahr sehr viel. Zum ersten: ihnen ist
vertrautx-morph=``strongMorph:TG5681'', was Gott geredet hat.
\bibverse{3} Daß aber etliche nicht daran
glaubenx-morph=``strongMorph:TG5656'', was liegt daran? Sollte ihr
Unglaube Gottes Glauben aufhebenx-morph=``strongMorph:TG5692''?
\bibverse{4} Das sei fernex-morph=``strongMorph:TG5636''! Es bleibe
vielmehr also, daß Gott seix-morph=``strongMorph:TG5737'' wahrhaftig und
alle Menschen Lügner; wie geschrieben
stehtx-morph=``strongMorph:TG5769'': ``Auf daß du
gerechtx-morph=''strongMorph:TG5686" seist in deinen Worten und
überwindestx-morph=``strongMorph:TG5661'', wenn du gerichtet
wirstx-morph=``strongMorph:TG5745''.'' \bibverse{5} Ist's aber also, daß
unsere Ungerechtigkeit Gottes Gerechtigkeit
preistx-morph=``strongMorph:TG5719'', was wollen wir
sagenx-morph=``strongMorph:TG5692''? Ist denn Gott auch ungerecht, wenn
er darüber zürntx-morph=``strongMorph:TG5723''? (Ich
redex-morph=``strongMorph:TG5719'' also auf Menschenweise.) \bibverse{6}
Das sei fernex-morph=``strongMorph:TG5636''! Wie könnte sonst Gott die
Welt
richtenx-morph=``strongMorph:TG5692''\textbar x-morph=``strongMorph:TG5719''?
\bibverse{7} Denn so die Wahrheit Gottes durch meine Lüge herrlicher
wirdx-morph=``strongMorph:TG5656'' zu seinem Preis, warum sollte ich
denn noch als Sünder gerichtet werdenx-morph=``strongMorph:TG5743''
\bibverse{8} und nicht vielmehr also tun, wie wir gelästert
werdenx-morph=``strongMorph:TG5743'' und wie etliche
sprechenx-morph=``strongMorph:TG5748'', daß wir
sagenx-morph=``strongMorph:TG5721'': ``Lasset uns Übles
tunx-morph=''strongMorph:TG5661``, auf das Gutes daraus
kommex-morph=''strongMorph:TG5632"''? welcher Verdammnis
istx-morph=``strongMorph:TG5748'' ganz recht. \bibverse{9} Was sagen wir
denn nun? Haben wir einen Vorteilx-morph=``strongMorph:TG5736''? Gar
keinen. Denn wir haben droben bewiesenx-morph=``strongMorph:TG5662'',
daß beide, Juden und Griechen, alle unter der Sünde
sindx-morph=``strongMorph:TG5750'', \bibverse{10} wie denn geschrieben
stehtx-morph=``strongMorph:TG5769'': ``Da
istx-morph=''strongMorph:TG5748" nicht, der gerecht sei, auch nicht
einer. \bibverse{11} Da istx-morph=``strongMorph:TG5748'' nicht, der
verständig seix-morph=``strongMorph:TG5723''; da
istx-morph=``strongMorph:TG5748'' nicht, der nach Gott
fragex-morph=``strongMorph:TG5723''. \bibverse{12} Sie sind alle
abgewichenx-morph=``strongMorph:TG5656'' und allesamt untüchtig
gewordenx-morph=``strongMorph:TG5681''. Da
istx-morph=``strongMorph:TG5748'' nicht, der Gutes
tuex-morph=``strongMorph:TG5723'', auch nicht
einerx-morph=``strongMorph:TG5748''. \bibverse{13} Ihr Schlund ist ein
offenesx-morph=``strongMorph:TG5772'' Grab; mit ihren Zungen handeln sie
trüglichx-morph=``strongMorph:TG5707''. Otterngift ist unter den Lippen;
\bibverse{14} ihr Mund ist vollx-morph=``strongMorph:TG5719'' Fluchens
und Bitterkeit. \bibverse{15} Ihre Füße sind eilend, Blut zu
vergießenx-morph=``strongMorph:TG5658''; \bibverse{16} auf ihren Wegen
ist eitel Schaden und Herzeleid, \bibverse{17} und den Weg des Friedens
wissenx-morph=``strongMorph:TG5627'' sie nicht. \bibverse{18} Es
istx-morph=``strongMorph:TG5748'' keine Furcht Gottes vor ihren Augen.''
\bibverse{19} Wir wissenx-morph=``strongMorph:TG5758'' aber, daß, was
das Gesetz sagtx-morph=``strongMorph:TG5719'', das
sagtx-morph=``strongMorph:TG5719'' es denen, die unter dem Gesetz sind,
auf daß aller Mund verstopft werdex-morph=``strongMorph:TG5652'' und
alle Welt Gott schuldig seix-morph=``strongMorph:TG5638''; \bibverse{20}
darum daß kein Fleisch durch des Gesetzes Werke vor ihm gerecht sein
kannx-morph=``strongMorph:TG5701''; denn durch das Gesetz kommt
Erkenntnis der Sünde. \bibverse{21} Nun aber ist ohne Zutun des Gesetzes
die Gerechtigkeit, die vor Gott gilt,
offenbartx-morph=``strongMorph:TG5769'' und
bezeugtx-morph=``strongMorph:TG5746'' durch das Gesetz und die
Propheten. \bibverse{22} Ich sage aber von solcher Gerechtigkeit vor
Gott, die da kommt durch den Glauben an Jesum Christum zu allen und auf
alle, die da glaubenx-morph=``strongMorph:TG5723''. \bibverse{23} Denn
es istx-morph=``strongMorph:TG5748'' hier kein Unterschied: sie
sindx-morph=``strongMorph:TG5627'' allzumal Sünder und
mangelnx-morph=``strongMorph:TG5743'' des Ruhmes, den sie bei Gott haben
sollten, \bibverse{24} und werdenx-morph=``strongMorph:TG5746'' ohne
Verdienst gerecht aus seiner Gnade durch die Erlösung, so durch Jesum
Christum geschehen ist, \bibverse{25} welchen Gott hat
vorgestelltx-morph=``strongMorph:TG5639'' zu einem Gnadenstuhl durch den
Glauben in seinem Blut, damit er die Gerechtigkeit, die vor ihm gilt,
darbiete in dem, daß er Sünde vergibt, welche bisher geblieben
warx-morph=``strongMorph:TG5761'' unter göttlicher Geduld; \bibverse{26}
auf daß er zu diesen Zeiten darböte die Gerechtigkeit, die vor ihm gilt;
auf daß er allein gerecht seix-morph=``strongMorph:TG5750'' und
gerechtx-morph=``strongMorph:TG5723'' mache den, der da ist des Glaubens
an Jesum. \bibverse{27} Wo bleibt nun der Ruhm? Er ist
ausgeschlossenx-morph=``strongMorph:TG5681''. Durch das Gesetz? Durch
der Werke Gesetz? Nicht also, sondern durch des Glaubens Gesetz.
\bibverse{28} So haltenx-morph=``strongMorph:TG5736'' wir nun dafür, daß
der Mensch gerecht werdex-morph=``strongMorph:TG5745'' ohne des Gesetzes
Werke, allein durch den Glauben. \bibverse{29} Oder ist Gott allein der
Juden Gott? Ist er nicht auch der Heiden Gott? Ja freilich, auch der
Heiden Gott. \bibverse{30} Sintemal es ist ein einiger Gott, der da
gerecht machtx-morph=``strongMorph:TG5692'' die Beschnittenen aus dem
Glauben und die Unbeschnittenen durch den Glauben. \bibverse{31} Wie?
Hebenx-morph=``strongMorph:TG5719'' wir denn das Gesetz auf durch den
Glauben? Das sei fernex-morph=``strongMorph:TG5636''! sondern wir
richtenx-morph=``strongMorph:TG5719'' das Gesetz auf.

\hypertarget{section-3}{%
\section{4}\label{section-3}}

\bibverse{1} Was sagenx-morph=``strongMorph:TG5692'' wir denn von unserm
Vater Abraham, daß er gefunden habex-morph=``strongMorph:TG5760'' nach
dem Fleisch? \bibverse{2} Das sagen wir:
Istx-morph=``strongMorph:TG5681'' Abraham durch die Werke gerecht, so
hatx-morph=``strongMorph:TG5719'' er wohl Ruhm, aber nicht vor Gott.
\bibverse{3} Was sagtx-morph=``strongMorph:TG5719'' denn die Schrift?
``Abraham hat Gott geglaubtx-morph=''strongMorph:TG5656``, und das ist
ihm zur Gerechtigkeit gerechnetx-morph=''strongMorph:TG5681``.''
\bibverse{4} Dem aber, der mit Werken
umgehtx-morph=``strongMorph:TG5740'', wird der Lohn nicht aus Gnade
zugerechnetx-morph=``strongMorph:TG5736'', sondern aus Pflicht.
\bibverse{5} Dem aber, der nicht mit Werken
umgehtx-morph=``strongMorph:TG5740'',
glaubtx-morph=``strongMorph:TG5723'' aber an den, der die Gottlosen
gerecht machtx-morph=``strongMorph:TG5723'', dem wird sein Glaube
gerechnetx-morph=``strongMorph:TG5736'' zur Gerechtigkeit. \bibverse{6}
Nach welcher Weise auch David sagtx-morph=``strongMorph:TG5719'', daß
die Seligkeit sei allein des Menschen, welchem Gott
zurechnetx-morph=``strongMorph:TG5736'' die Gerechtigkeit ohne Zutun der
Werke, da er spricht: \bibverse{7} ``Selig sind die, welchen ihre
Ungerechtigkeiten vergeben sindx-morph=''strongMorph:TG5681" und welchen
ihre Sünden bedeckt sindx-morph=``strongMorph:TG5681''! \bibverse{8}
Selig ist der Mann, welchem Gott die Sünde nicht
zurechnetx-morph=``strongMorph:TG5667''!'' \bibverse{9} Nun diese
Seligkeit, geht sie über die Beschnittenen oder auch über die
Unbeschnittenen? Wir müssen ja sagenx-morph=``strongMorph:TG5719'', daß
Abraham sei sein Glaube zur Gerechtigkeit
gerechnetx-morph=``strongMorph:TG5681''. \bibverse{10} Wie ist er ihm
denn zugerechnetx-morph=``strongMorph:TG5681''? Als er beschnitten oder
als er unbeschnitten warx-morph=``strongMorph:TG5752''? Nicht, als er
beschnitten, sondern als er unbeschnitten war. \bibverse{11} Das Zeichen
der Beschneidung empfingx-morph=``strongMorph:TG5627'' er zum Siegel der
Gerechtigkeit des Glaubens, welchen er hatte, als er noch nicht
beschnitten war, auf daß er würdex-morph=``strongMorph:TG5750'' ein
Vater aller, die da glaubenx-morph=``strongMorph:TG5723'' und nicht
beschnitten sind, daß ihnen solches auch gerechnet
werdex-morph=``strongMorph:TG5683'' zur Gerechtigkeit; \bibverse{12} und
würde auch ein Vater der Beschneidung, derer, die nicht allein
beschnitten sind, sondern auch wandelnx-morph=``strongMorph:TG5723'' in
den Fußtapfen des Glaubens, welcher war in unserm Vater Abraham, als er
noch unbeschnitten war. \bibverse{13} Denn die Verheißung, daß er sollte
seinx-morph=``strongMorph:TG5750'' der Welt Erbe, ist nicht geschehen
Abraham oder seinem Samen durchs Gesetz, sondern durch die Gerechtigkeit
des Glaubens. \bibverse{14} Denn wo die vom Gesetz Erben sind, so
istx-morph=``strongMorph:TG5769'' der Glaube nichts, und die Verheißung
ist abgetanx-morph=``strongMorph:TG5769''. \bibverse{15} Sintemal das
Gesetz nur Zorn anrichtetx-morph=``strongMorph:TG5736''; denn wo das
Gesetz nicht istx-morph=``strongMorph:TG5748'', da ist auch keine
Übertretung. \bibverse{16} Derhalben muß die Gerechtigkeit durch den
Glauben kommen, auf daß sie sei aus Gnaden und die Verheißung fest
bleibex-morph=``strongMorph:TG5750'' allem Samen, nicht dem allein, der
unter dem Gesetz ist, sondern auch dem, der des Glaubens Abrahams ist,
welcher istx-morph=``strongMorph:TG5748'' unser aller Vater
\bibverse{17} (wie geschrieben stehtx-morph=``strongMorph:TG5769'':
``Ich habe dich gesetztx-morph=''strongMorph:TG5758" zum Vater vieler
Völker'') vor Gott, dem er geglaubt hatx-morph=``strongMorph:TG5656'',
der da lebendig machtx-morph=``strongMorph:TG5723'' die Toten und
ruftx-morph=``strongMorph:TG5723'' dem, was nicht
istx-morph=``strongMorph:TG5752'', daß es
seix-morph=``strongMorph:TG5752''. \bibverse{18} Und er hat
geglaubtx-morph=``strongMorph:TG5656'' auf Hoffnung, da nichts zu hoffen
war, auf daß er würdex-morph=``strongMorph:TG5635'' ein Vater vieler
Völker, wie denn zu ihm gesagt istx-morph=``strongMorph:TG5772'': ``Also
soll dein Same seinx-morph=''strongMorph:TG5704``.'' \bibverse{19} Und
er ward nicht schwachx-morph=``strongMorph:TG5660'' im Glauben,
sahx-morph=``strongMorph:TG5656'' auch nicht an seinem eigenen Leib,
welcher schon erstorbenx-morph=``strongMorph:TG5772'' war (weil er schon
fast hundertjährig warx-morph=``strongMorph:TG5723''), auch nicht den
erstorbenen Leib der Sara; \bibverse{20} denn er
zweifeltex-morph=``strongMorph:TG5681'' nicht an der Verheißung Gottes
durch Unglauben, sondern ward starkx-morph=``strongMorph:TG5681'' im
Glauben und gabx-morph=``strongMorph:TG5631'' Gott die Ehre
\bibverse{21} und wußte aufs
allergewissestex-morph=``strongMorph:TG5685'', daß, was Gott
verheißtx-morph=``strongMorph:TG5766'', das
kannx-morph=``strongMorph:TG5748'' er auch
tunx-morph=``strongMorph:TG5658''. \bibverse{22} Darum ist's ihm auch
zur Gerechtigkeit gerechnetx-morph=``strongMorph:TG5681''. \bibverse{23}
Das ist aber nicht geschriebenx-morph=``strongMorph:TG5648'' allein um
seinetwillen, daß es ihm zugerechnet istx-morph=``strongMorph:TG5681'',
\bibverse{24} sondern auch um unsertwillen, welchen es zugerechnet
werdenx-morph=``strongMorph:TG5745'' sollx-morph=``strongMorph:TG5719'',
so wir glaubenx-morph=``strongMorph:TG5723'' an den, der unsern HERRN
Jesus auferweckt hatx-morph=``strongMorph:TG5660'' von den Toten,
\bibverse{25} welcher ist um unsrer Sünden willen
dahingegebenx-morph=``strongMorph:TG5681'' und um unsrer Gerechtigkeit
willen auferwecktx-morph=``strongMorph:TG5681''.

\hypertarget{section-4}{%
\section{5}\label{section-4}}

\bibverse{1} Nun wir denn sind gerecht
gewordenx-morph=``strongMorph:TG5685'' durch den Glauben, so
habenx-morph=``strongMorph:TG5719'' wir Frieden mit Gott durch unsern
HERRN Jesus Christus, \bibverse{2} durch welchen wir auch den Zugang
habenx-morph=``strongMorph:TG5758'' im Glauben zu dieser Gnade, darin
wir stehenx-morph=``strongMorph:TG5758'', und
rühmenx-morph=``strongMorph:TG5736'' uns der Hoffnung der zukünftigen
Herrlichkeit, die Gott geben soll. \bibverse{3} Nicht allein aber das,
sondern wir rühmenx-morph=``strongMorph:TG5736'' uns auch der Trübsale,
dieweil wir wissenx-morph=``strongMorph:TG5761'', daß Trübsal Geduld
bringtx-morph=``strongMorph:TG5736''; \bibverse{4} Geduld aber bringt
Erfahrung; Erfahrung aber bringt Hoffnung; \bibverse{5} Hoffnung aber
läßt nicht zu Schanden werdenx-morph=``strongMorph:TG5719''. Denn die
Liebe Gottes ist ausgegossenx-morph=``strongMorph:TG5769'' in unser Herz
durch den heiligen Geist, welcher uns gegeben
istx-morph=``strongMorph:TG5685''. \bibverse{6} Denn auch Christus, da
wir noch schwach warenx-morph=``strongMorph:TG5752'' nach der Zeit, ist
für uns Gottlose gestorbenx-morph=``strongMorph:TG5627''. \bibverse{7}
Nun stirbtx-morph=``strongMorph:TG5695'' kaum jemand um eines Gerechten
willen; um des Guten willen dürftex-morph=``strongMorph:TG5719''
vielleicht jemand sterbenx-morph=``strongMorph:TG5629''. \bibverse{8}
Darum preisetx-morph=``strongMorph:TG5719'' Gott seine Liebe gegen uns,
daß Christus für uns gestorben istx-morph=``strongMorph:TG5627'', da wir
noch Sünder warenx-morph=``strongMorph:TG5752''. \bibverse{9} So werden
wir ja viel mehr durch ihn bewahrt werdenx-morph=``strongMorph:TG5701''
vor dem Zorn, nachdem wir durch sein Blut gerecht geworden
sindx-morph=``strongMorph:TG5685''. \bibverse{10} Denn so wir Gott
versöhnt sindx-morph=``strongMorph:TG5648'' durch den Tod seines Sohnes,
da wir noch Feinde warenx-morph=``strongMorph:TG5752'', viel mehr werden
wir selig werdenx-morph=``strongMorph:TG5701'' durch sein Leben, so wir
nun versöhnt sindx-morph=``strongMorph:TG5651''. \bibverse{11} Nicht
allein aber das, sondern wir rühmenx-morph=``strongMorph:TG5740'' uns
auch Gottes durch unsern HERRN Jesus Christus, durch welchen wir nun die
Versöhnung empfangen habenx-morph=``strongMorph:TG5627''. \bibverse{12}
Derhalben, wie durch einen Menschen die Sünde ist
gekommenx-morph=``strongMorph:TG5627'' in die Welt und der Tod durch die
Sünde, und ist also der Tod zu allen Menschen
durchgedrungenx-morph=``strongMorph:TG5627'', dieweil sie alle gesündigt
habenx-morph=``strongMorph:TG5627''; \bibverse{13} denn die Sünde
warx-morph=``strongMorph:TG5713'' wohl in der Welt bis auf das Gesetz;
aber wo kein Gesetz istx-morph=``strongMorph:TG5752'', da
achtetx-morph=``strongMorph:TG5743'' man der Sünde nicht. \bibverse{14}
Doch herrschtex-morph=``strongMorph:TG5656'' der Tod von Adam an bis auf
Moses auch über die, die nicht gesündigt
habenx-morph=``strongMorph:TG5660'' mit gleicher Übertretung wie Adam,
welcher istx-morph=``strongMorph:TG5748'' ein Bild des, der zukünftig
warx-morph=``strongMorph:TG5723''. \bibverse{15} Aber nicht verhält
sich's mit der Gabe wie mit der Sünde. Denn so an eines Sünde viele
gestorben sindx-morph=``strongMorph:TG5627'', so ist viel mehr Gottes
Gnade und Gabe vielen reichlich
widerfahrenx-morph=``strongMorph:TG5656'' durch die Gnade des einen
Menschen Jesus Christus. \bibverse{16} Und nicht ist die Gabe allein
über eine Sünde, wie durch des einen
Sündersx-morph=``strongMorph:TG5660'' eine Sünde alles Verderben. Denn
das Urteil ist gekommen aus einer Sünde zur Verdammnis; die Gabe aber
hilft auch aus vielen Sünden zur Gerechtigkeit. \bibverse{17} Denn so um
des einen Sünde willen der Tod geherrscht
hatx-morph=``strongMorph:TG5656'' durch den einen, viel mehr werden die,
so da empfangenx-morph=``strongMorph:TG5723'' die Fülle der Gnade und
der Gabe zur Gerechtigkeit, herrschenx-morph=``strongMorph:TG5692'' im
Leben durch einen, Jesum Christum. \bibverse{18} Wie nun durch eines
Sünde die Verdammnis über alle Menschen gekommen ist, so ist auch durch
eines Gerechtigkeit die Rechtfertigung des Lebens über alle Menschen
gekommen. \bibverse{19} Denn gleichwie durch eines Menschen Ungehorsam
viele Sünder geworden sindx-morph=``strongMorph:TG5681'', also auch
durch eines Gehorsam werdenx-morph=``strongMorph:TG5701'' viele
Gerechte. \bibverse{20} Das Gesetz aber ist neben
eingekommenx-morph=``strongMorph:TG5627'', auf daß die Sünde mächtiger
würdex-morph=``strongMorph:TG5661''. Wo aber die Sünde mächtig geworden
istx-morph=``strongMorph:TG5656'', da ist doch die Gnade viel mächtiger
gewordenx-morph=``strongMorph:TG5656'', \bibverse{21} auf daß, gleichwie
die Sünde geherrscht hatx-morph=``strongMorph:TG5656'' zum Tode, also
auch herrschex-morph=``strongMorph:TG5661'' die Gnade durch die
Gerechtigkeit zum ewigen Leben durch Jesum Christum, unsern HERRN.

\hypertarget{section-5}{%
\section{6}\label{section-5}}

\bibverse{1} Was wollen wir hierzu sagenx-morph=``strongMorph:TG5692''?
Sollen wir denn in der Sünde beharrenx-morph=``strongMorph:TG5692'', auf
daß die Gnade desto mächtiger werdex-morph=``strongMorph:TG5661''?
\bibverse{2} Das sei fernex-morph=``strongMorph:TG5636''! Wie sollten
wir in der Sünde wollen lebenx-morph=``strongMorph:TG5692'', der wir
abgestorben sindx-morph=``strongMorph:TG5627''? \bibverse{3} Wisset ihr
nichtx-morph=``strongMorph:TG5719'', daß alle, die wir in Jesus Christus
getauft sindx-morph=``strongMorph:TG5681'', die sind in seinen Tod
getauftx-morph=``strongMorph:TG5681''? \bibverse{4} So sind wir ja mit
ihm begrabenx-morph=``strongMorph:TG5648'' durch die Taufe in den Tod,
auf daß, gleichwie Christus ist auferwecktx-morph=``strongMorph:TG5681''
von den Toten durch die Herrlichkeit des Vaters, also sollen auch wir in
einem neuen Leben wandelnx-morph=``strongMorph:TG5661''. \bibverse{5} So
wir aber samt ihm gepflanzt werdenx-morph=``strongMorph:TG5754'' zu
gleichem Tode, so werden wir auch seiner Auferstehung gleich
seinx-morph=``strongMorph:TG5704'', \bibverse{6} dieweil wir
wissenx-morph=``strongMorph:TG5723'', daß unser alter Mensch samt ihm
gekreuzigt istx-morph=``strongMorph:TG5681'', auf daß der sündliche Leib
aufhörex-morph=``strongMorph:TG5686'', daß wir hinfort der Sünde nicht
mehr dienenx-morph=``strongMorph:TG5721''. \bibverse{7} Denn wer
gestorben istx-morph=``strongMorph:TG5631'', der ist
gerechtfertigtx-morph=``strongMorph:TG5769'' von der Sünde. \bibverse{8}
Sind wir aber mit Christo gestorbenx-morph=``strongMorph:TG5627'', so
glaubenx-morph=``strongMorph:TG5719'' wir, daß wir auch mit ihm leben
werdenx-morph=``strongMorph:TG5692'', \bibverse{9} und
wissenx-morph=``strongMorph:TG5761'', daß Christus, von den Toten
auferwecktx-morph=``strongMorph:TG5685'', hinfort nicht
stirbtx-morph=``strongMorph:TG5719''; der Tod wird hinfort nicht mehr
über ihn herrschenx-morph=``strongMorph:TG5719''. \bibverse{10} Denn was
er gestorben istx-morph=``strongMorph:TG5627'', das ist er der Sünde
gestorbenx-morph=``strongMorph:TG5627'' zu einem Mal; was er aber
lebtx-morph=``strongMorph:TG5719'', das
lebtx-morph=``strongMorph:TG5719'' er Gott. \bibverse{11} Also auch ihr,
haltetx-morph=``strongMorph:TG5737'' euch dafür, daß ihr der Sünde
gestorben seidx-morph=``strongMorph:TG5750'' und
lebtx-morph=``strongMorph:TG5723'' Gott in Christo Jesus, unserm HERRN.
\bibverse{12} So lasset nun die Sünde nicht
herrschenx-morph=``strongMorph:TG5720'' in eurem sterblichen Leibe, ihr
Gehorsam zu leistenx-morph=``strongMorph:TG5721'' in seinen Lüsten.
\bibverse{13} Auch begebetx-morph=``strongMorph:TG5720'' nicht der Sünde
eure Glieder zu Waffen der Ungerechtigkeit, sondern
begebetx-morph=``strongMorph:TG5657'' euch selbst Gott, als die da aus
den Toten lebendig sindx-morph=``strongMorph:TG5723'', und eure Glieder
Gott zu Waffen der Gerechtigkeit. \bibverse{14} Denn die Sünde wird
nicht herrschen können überx-morph=``strongMorph:TG5692'' euch, sintemal
ihr nicht unter dem Gesetz seidx-morph=``strongMorph:TG5748'', sondern
unter der Gnade. \bibverse{15} Wie nun? Sollen wir
sündigenx-morph=``strongMorph:TG5692'', dieweil wir nicht unter dem
Gesetz, sondern unter der Gnade sindx-morph=``strongMorph:TG5748''? Das
sei fernex-morph=``strongMorph:TG5636''! \bibverse{16} Wisset
ihrx-morph=``strongMorph:TG5758'' nicht: welchem ihr euch
begebetx-morph=``strongMorph:TG5719'' zu Knechten in Gehorsam, des
Knechte seidx-morph=``strongMorph:TG5748'' ihr, dem ihr gehorsam
seidx-morph=``strongMorph:TG5719'', es sei der Sünde zum Tode oder dem
Gehorsam zur Gerechtigkeit? \bibverse{17} Gott sei aber gedankt, daß ihr
Knechte der Sünde gewesen seidx-morph=``strongMorph:TG5713'', aber nun
gehorsam gewordenx-morph=``strongMorph:TG5656'' von Herzen dem Vorbilde
der Lehre, welchem ihr ergeben seidx-morph=``strongMorph:TG5681''.
\bibverse{18} Denn nun ihr frei geworden
seidx-morph=``strongMorph:TG5685'' von der Sünde, seid ihr Knechte der
Gerechtigkeit gewordenx-morph=``strongMorph:TG5681''. \bibverse{19} Ich
muß menschlich davon redenx-morph=``strongMorph:TG5719'' um der
Schwachheit willen eures Fleisches. Gleichwie ihr eure Glieder begeben
habetx-morph=``strongMorph:TG5656'' zum Dienst der Unreinigkeit und von
einer Ungerechtigkeit zur andern, also
begebetx-morph=``strongMorph:TG5657'' auch nun eure Glieder zum Dienst
der Gerechtigkeit, daß sie heilig werden. \bibverse{20} Denn da ihr der
Sünde Knechte wartx-morph=``strongMorph:TG5713'', da
wartx-morph=``strongMorph:TG5713'' ihr frei von der Gerechtigkeit.
\bibverse{21} Was hattet ihrx-morph=``strongMorph:TG5707'' nun zu der
Zeit für Frucht? Welcher ihr euch jetzt
schämetx-morph=``strongMorph:TG5736''; denn ihr Ende ist der Tod.
\bibverse{22} Nun ihr aber seidx-morph=``strongMorph:TG5685'' von der
Sünde frei und Gottes Knechte gewordenx-morph=``strongMorph:TG5685'',
habtx-morph=``strongMorph:TG5719'' ihr eure Frucht, daß ihr heilig
werdet, das Ende aber ist das ewige Leben. \bibverse{23} Denn der Tod
ist der Sünde Sold; aber die Gabe Gottes ist das ewige Leben in Christo
Jesu, unserm HERRN.

\hypertarget{section-6}{%
\section{7}\label{section-6}}

\bibverse{1} Wisset ihr nichtx-morph=``strongMorph:TG5719'', liebe
Brüder (denn ich redex-morph=``strongMorph:TG5719'' mit solchen, die das
Gesetz wissenx-morph=``strongMorph:TG5723''), daß das Gesetz herrscht
überx-morph=``strongMorph:TG5719'' den Menschen solange er
lebtx-morph=``strongMorph:TG5719''? \bibverse{2} Denn ein Weib, das
unter dem Manne ist, ist an ihn gebundenx-morph=``strongMorph:TG5769''
durch das Gesetz, solange der Mann lebtx-morph=``strongMorph:TG5723'';
so aber der Mann stirbtx-morph=``strongMorph:TG5632'', so ist sie
losx-morph=``strongMorph:TG5769'' vom Gesetz, das den Mann betrifft.
\bibverse{3} Wo sie nun eines andern Mannes
wirdx-morph=``strongMorph:TG5638'', solange der Mann
lebtx-morph=``strongMorph:TG5723'', wird sie eine Ehebrecherin
geheißenx-morph=``strongMorph:TG5692''; so aber der Mann
stirbtx-morph=``strongMorph:TG5632'', istx-morph=``strongMorph:TG5748''
sie frei vom Gesetz, daß sie nicht eine Ehebrecherin
istx-morph=``strongMorph:TG5750'', wo sie eines andern Mannes
wirdx-morph=``strongMorph:TG5637''. \bibverse{4} Also seid auch ihr,
meine Brüder, getötetx-morph=``strongMorph:TG5681'' dem Gesetz durch den
Leib Christi, daß ihr eines andern seidx-morph=``strongMorph:TG5635'',
nämlich des, der von den Toten auferweckt
istx-morph=``strongMorph:TG5685'', auf daß wir Gott Frucht
bringenx-morph=``strongMorph:TG5661''. \bibverse{5} Denn da wir im
Fleisch warenx-morph=``strongMorph:TG5713'', da waren die sündigen
Lüste, welche durchs Gesetz sich erregtenx-morph=``strongMorph:TG5710'',
kräftig in unsern Gliedern, dem Tode
Fruchtx-morph=``strongMorph:TG5658'' zu bringen. \bibverse{6} Nun aber
sindx-morph=``strongMorph:TG5681'' wir vom Gesetz los und ihm
abgestorbenx-morph=``strongMorph:TG5631''\textbar x-morph=``strongMorph:TG5625''x-morph=``strongMorph:TG5631'',
das uns gefangenhieltx-morph=``strongMorph:TG5712'', also daß wir dienen
sollenx-morph=``strongMorph:TG5721'' im neuen Wesen des Geistes und
nicht im alten Wesen des Buchstabens. \bibverse{7} Was wollen wir denn
nun sagenx-morph=``strongMorph:TG5692''? Ist das Gesetz Sünde? Das sei
fernex-morph=``strongMorph:TG5636''! Aber die Sünde
erkanntex-morph=``strongMorph:TG5627'' ich nicht, außer durchs Gesetz.
Denn ich wußtex-morph=``strongMorph:TG5715'' nichts von der Lust, wo das
Gesetz nicht hätte gesagtx-morph=``strongMorph:TG5707'': ``Laß dich
nicht gelüstenx-morph=''strongMorph:TG5692``!'' \bibverse{8} Da
nahmx-morph=``strongMorph:TG5631'' aber die Sünde Ursache am Gebot und
erregtex-morph=``strongMorph:TG5662'' in mir allerlei Lust; denn ohne
das Gesetz war die Sünde tot. \bibverse{9} Ich aber
lebtex-morph=``strongMorph:TG5707'' weiland ohne Gesetz; da aber das
Gebot kamx-morph=``strongMorph:TG5631'', ward die Sünde wieder
lebendigx-morph=``strongMorph:TG5656'', \bibverse{10} ich aber
starbx-morph=``strongMorph:TG5627''; und es fand
sichx-morph=``strongMorph:TG5681'', daß das Gebot mir zum Tode
gereichte, das mir doch zum Leben gegeben war. \bibverse{11} Denn die
Sünde nahmx-morph=``strongMorph:TG5631'' Ursache am Gebot und
betrogx-morph=``strongMorph:TG5656'' mich und
tötetex-morph=``strongMorph:TG5656'' mich durch dasselbe Gebot.
\bibverse{12} Das Gesetz ist ja heilig, und das Gebot ist heilig, recht
und gut. \bibverse{13} Ist denn, das da gut ist, mir zum Tod
gewordenx-morph=``strongMorph:TG5754''? Das sei
fernex-morph=``strongMorph:TG5636''! Aber die Sünde, auf daß sie
erscheinex-morph=``strongMorph:TG5652'', wie sie Sünde ist, hat sie mir
durch das Gute den Tod gewirktx-morph=``strongMorph:TG5740'', auf daß
die Sünde würdex-morph=``strongMorph:TG5638'' überaus sündig durchs
Gebot. \bibverse{14} Denn wir wissenx-morph=``strongMorph:TG5758'', daß
das Gesetz geistlich istx-morph=``strongMorph:TG5748''; ich
binx-morph=``strongMorph:TG5748'' aber fleischlich, unter die Sünde
verkauftx-morph=``strongMorph:TG5772''. \bibverse{15} Denn ich
weißx-morph=``strongMorph:TG5719'' nicht, was ich
tuex-morph=``strongMorph:TG5736''. Denn ich
tuex-morph=``strongMorph:TG5719'' nicht, was ich
willx-morph=``strongMorph:TG5719''; sondern, was ich
hassex-morph=``strongMorph:TG5719'', das tue
ichx-morph=``strongMorph:TG5719''. \bibverse{16} So ich aber das
tuex-morph=``strongMorph:TG5719'', was ich nicht
willx-morph=``strongMorph:TG5719'', so
gebex-morph=``strongMorph:TG5748'' ich zu, daß das Gesetz gut sei.
\bibverse{17} So tuex-morph=``strongMorph:TG5736'' ich nun dasselbe
nicht, sondern die Sünde, die in mir
wohntx-morph=``strongMorph:TG5723''. \bibverse{18} Denn ich
weißx-morph=``strongMorph:TG5758'', daß in mir, das
istx-morph=``strongMorph:TG5748'' in meinem Fleische,
wohntx-morph=``strongMorph:TG5719'' nichts Gutes.
Wollenx-morph=``strongMorph:TG5721'' habex-morph=``strongMorph:TG5736''
ich wohl, aber vollbringenx-morph=``strongMorph:TG5738'' das Gute
findex-morph=``strongMorph:TG5719'' ich nicht. \bibverse{19} Denn das
Gute, das ich willx-morph=``strongMorph:TG5719'', das
tuex-morph=``strongMorph:TG5719'' ich nicht; sondern das Böse, das ich
nicht willx-morph=``strongMorph:TG5719'', das
tuex-morph=``strongMorph:TG5719'' ich. \bibverse{20} So ich aber
tuex-morph=``strongMorph:TG5719'', was ich nicht
willx-morph=``strongMorph:TG5719'', so tuex-morph=``strongMorph:TG5736''
ich dasselbe nicht; sondern die Sünde, die in mir
wohntx-morph=``strongMorph:TG5723''. \bibverse{21} So
findex-morph=``strongMorph:TG5719'' ich mir nun ein Gesetz, der ich
willx-morph=``strongMorph:TG5723'' das Gute
tunx-morph=``strongMorph:TG5721'', daß mir das Böse
anhangtx-morph=``strongMorph:TG5736''. \bibverse{22} Denn ich habe
Lustx-morph=``strongMorph:TG5736'' an Gottes Gesetz nach dem inwendigen
Menschen. \bibverse{23} Ich sehex-morph=``strongMorph:TG5719'' aber ein
ander Gesetz in meinen Gliedern, das da
widerstreitetx-morph=``strongMorph:TG5740'' dem Gesetz in meinem Gemüte
und nimmt mich gefangenx-morph=``strongMorph:TG5723'' in der Sünde
Gesetz, welches istx-morph=``strongMorph:TG5752'' in meinen Gliedern.
\bibverse{24} Ich elender Mensch! wer wird mich
erlösenx-morph=``strongMorph:TG5695'' von dem Leibe dieses Todes?
\bibverse{25} Ich dankex-morph=``strongMorph:TG5719'' Gott durch Jesum
Christum, unserm HERRN. So dienex-morph=``strongMorph:TG5719'' ich nun
mit dem Gemüte dem Gesetz Gottes, aber mit dem Fleische dem Gesetz der
Sünde.

\hypertarget{section-7}{%
\section{8}\label{section-7}}

\bibverse{1} So ist nun nichts Verdammliches an denen, die in Christo
Jesu sind, die nicht nach dem Fleisch
wandelnx-morph=``strongMorph:TG5723'', sondern nach dem Geist.
\bibverse{2} Denn das Gesetz des Geistes, der da lebendig macht in
Christo Jesu, hat mich frei gemachtx-morph=``strongMorph:TG5656'' von
dem Gesetz der Sünde und des Todes. \bibverse{3} Denn was dem Gesetz
unmöglich war (sintemal es durch das Fleisch geschwächt
wardx-morph=``strongMorph:TG5707''), das tat Gott und
sandtex-morph=``strongMorph:TG5660'' seinen Sohn in der Gestalt des
sündlichen Fleisches und der Sünde halben und
verdammtex-morph=``strongMorph:TG5656'' die Sünde im Fleisch,
\bibverse{4} auf daß die Gerechtigkeit, vom Gesetz erfordert, in uns
erfüllt würdex-morph=``strongMorph:TG5686'', die wir nun nicht nach dem
Fleische wandelnx-morph=``strongMorph:TG5723'', sondern nach dem Geist.
\bibverse{5} Denn die da fleischlich sindx-morph=``strongMorph:TG5752'',
die sindx-morph=``strongMorph:TG5719'' fleischlich gesinnt; die aber
geistlich sind, die sind geistlich gesinnt. \bibverse{6} Aber
fleischlich gesinnt sein ist der Tod, und geistlich gesinnt sein ist
Leben und Friede. \bibverse{7} Denn fleischlich gesinnt sein ist wie
eine Feindschaft wider Gott, sintemal das Fleisch dem Gesetz Gottes
nicht untertan istx-morph=``strongMorph:TG5743''; denn es
vermag'sx-morph=``strongMorph:TG5736'' auch nicht. \bibverse{8} Die aber
fleischlich sindx-morph=``strongMorph:TG5752'',
könnenx-morph=``strongMorph:TG5736'' Gott nicht
gefallenx-morph=``strongMorph:TG5658''. \bibverse{9} Ihr aber
seidx-morph=``strongMorph:TG5748'' nicht fleischlich, sondern geistlich,
so anders Gottes Geist in euch wohntx-morph=``strongMorph:TG5719''. Wer
aber Christi Geist nicht hatx-morph=``strongMorph:TG5719'', der
istx-morph=``strongMorph:TG5748'' nicht sein. \bibverse{10} So nun aber
Christus in euch ist, so ist der Leib zwar tot um der Sünde willen, der
Geist aber ist Leben um der Gerechtigkeit willen. \bibverse{11} So nun
der Geist des, der Jesum von den Toten auferweckt
hatx-morph=``strongMorph:TG5660'', in euch
wohntx-morph=``strongMorph:TG5719'', so wird auch derselbe, der Christum
von den Toten auferweckt hatx-morph=``strongMorph:TG5660'', eure
sterblichen Leiber lebendig machenx-morph=``strongMorph:TG5692'' um
deswillen, daß sein Geist in euch
wohntx-morph=``strongMorph:TG5723''\textbar x-morph=``strongMorph:TG5625''x-morph=``strongMorph:TG5723''.
\bibverse{12} So sindx-morph=``strongMorph:TG5748'' wir nun, liebe
Brüder, Schuldner nicht dem Fleisch, daß wir nach dem Fleisch
lebenx-morph=``strongMorph:TG5721''. \bibverse{13} Denn wo ihr nach dem
Fleisch lebetx-morph=``strongMorph:TG5719'', so
werdetx-morph=``strongMorph:TG5719'' ihr sterben
müssenx-morph=``strongMorph:TG5721''; wo ihr aber durch den Geist des
Fleisches Geschäfte tötetx-morph=``strongMorph:TG5719'', so werdet ihr
lebenx-morph=``strongMorph:TG5695''. \bibverse{14} Denn welche der Geist
Gottes treibtx-morph=``strongMorph:TG5743'', die
sindx-morph=``strongMorph:TG5748'' Gottes Kinder. \bibverse{15} Denn ihr
habt nicht einen knechtischen Geist
empfangenx-morph=``strongMorph:TG5627'', daß ihr euch abermals fürchten
müßtet; sondern ihr habt einen kindlichen Geist
empfangenx-morph=``strongMorph:TG5627'', durch welchen wir
rufenx-morph=``strongMorph:TG5719'': Abba, lieber Vater! \bibverse{16}
Derselbe Geist gibt Zeugnisx-morph=``strongMorph:TG5719'' unserem Geist,
daß wir Kinder Gottes sindx-morph=``strongMorph:TG5748''. \bibverse{17}
Sind wir denn Kinder, so sind wir auch Erben, nämlich Gottes Erben und
Miterben Christi, so wir anders mit
leidenx-morph=``strongMorph:TG5719'', auf daß wir auch mit zur
Herrlichkeitx-morph=``strongMorph:TG5686'' erhoben werden. \bibverse{18}
Denn ich haltex-morph=``strongMorph:TG5736'' es dafür, daß dieser Zeit
Leiden der Herrlichkeit nicht wert sei, die an uns
sollx-morph=``strongMorph:TG5723'' offenbart
werdenx-morph=``strongMorph:TG5683''. \bibverse{19} Denn das ängstliche
Harren der Kreatur wartetx-morph=``strongMorph:TG5736'' auf die
Offenbarung der Kinder Gottes. \bibverse{20} Sintemal die Kreatur
unterworfen istx-morph=``strongMorph:TG5648'' der Eitelkeit ohne ihren
Willen, sondern um deswillen, der sie unterworfen
hatx-morph=``strongMorph:TG5660'', auf Hoffnung. \bibverse{21} Denn auch
die Kreatur wird frei werdenx-morph=``strongMorph:TG5701'' vom Dienst
des vergänglichen Wesens zu der herrlichen Freiheit der Kinder Gottes.
\bibverse{22} Denn wir wissenx-morph=``strongMorph:TG5758'', daß alle
Kreatur sehnt sichx-morph=``strongMorph:TG5719'' mit uns und
ängstetx-morph=``strongMorph:TG5719'' sich noch immerdar. \bibverse{23}
Nicht allein aber sie, sondern auch wir selbst, die wir
habenx-morph=``strongMorph:TG5723'' des Geistes Erstlinge,
sehnenx-morph=``strongMorph:TG5719'' uns auch bei uns selbst nach der
Kindschaft und wartenx-morph=``strongMorph:TG5740'' auf unsers Leibes
Erlösung. \bibverse{24} Denn wir sind wohl
seligx-morph=``strongMorph:TG5681'', doch in der Hoffnung. Die Hoffnung
aber, die man siehtx-morph=``strongMorph:TG5746'',
istx-morph=``strongMorph:TG5748'' nicht Hoffnung; denn wie kann man des
hoffenx-morph=``strongMorph:TG5719'', das man
siehtx-morph=``strongMorph:TG5719''? \bibverse{25} So wir aber des
hoffenx-morph=``strongMorph:TG5719'', das wir nicht
sehenx-morph=``strongMorph:TG5719'', so
wartenx-morph=``strongMorph:TG5736'' wir sein durch Geduld.
\bibverse{26} Desgleichen auch der Geist
hilftx-morph=``strongMorph:TG5736'' unsrer Schwachheit auf. Denn wir
wissenx-morph=``strongMorph:TG5758'' nicht, was wir beten
sollenx-morph=``strongMorph:TG5667'', wie sich's
gebührtx-morph=``strongMorph:TG5748''; sondern der Geist selbst
vertrittx-morph=``strongMorph:TG5719'' uns aufs beste mit
unaussprechlichem Seufzen. \bibverse{27} Der aber die Herzen
erforschtx-morph=``strongMorph:TG5723'', der
weißx-morph=``strongMorph:TG5758'', was des Geistes Sinn sei; denn er
vertrittx-morph=``strongMorph:TG5719'' die Heiligen nach dem, das Gott
gefällt. \bibverse{28} Wir wissenx-morph=``strongMorph:TG5758'' aber,
daß denen, die Gott liebenx-morph=``strongMorph:TG5723'', alle Dinge zum
Besten dienenx-morph=``strongMorph:TG5719'', denen, die nach dem Vorsatz
berufen sindx-morph=``strongMorph:TG5752''. \bibverse{29} Denn welche er
zuvor ersehen hatx-morph=``strongMorph:TG5656'', die hat er auch
verordnetx-morph=``strongMorph:TG5656'', daß sie gleich sein sollten dem
Ebenbilde seines Sohnes, auf daß derselbe der Erstgeborene
seix-morph=``strongMorph:TG5750'' unter vielen Brüdern. \bibverse{30}
Welche er aber verordnet hatx-morph=``strongMorph:TG5656'', die hat er
auch berufenx-morph=``strongMorph:TG5656''; welche er aber berufen
hatx-morph=``strongMorph:TG5656'', die hat er auch gerecht
gemachtx-morph=``strongMorph:TG5656'', welche er aber hat gerecht
gemachtx-morph=``strongMorph:TG5656'', die hat er auch herrlich
gemachtx-morph=``strongMorph:TG5656''. \bibverse{31} Was wollen wir nun
hierzu sagenx-morph=``strongMorph:TG5692''? Ist Gott für uns, wer mag
wider uns sein? \bibverse{32} welcher auch seines eigenen Sohnes nicht
hat verschontx-morph=``strongMorph:TG5662'', sondern hat ihn für uns
alle dahingegebenx-morph=``strongMorph:TG5656''; wie sollte er uns mit
ihm nicht alles schenkenx-morph=``strongMorph:TG5695''? \bibverse{33}
Wer will die Auserwählten Gottes
beschuldigenx-morph=``strongMorph:TG5692''? Gott ist hier, der da
gerecht machtx-morph=``strongMorph:TG5723''. \bibverse{34} Wer will
verdammenx-morph=``strongMorph:TG5723''\textbar x-morph=``strongMorph:TG5694''?
Christus ist hier, der gestorben istx-morph=``strongMorph:TG5631'', ja
vielmehr, der auch auferweckt istx-morph=``strongMorph:TG5685'', welcher
istx-morph=``strongMorph:TG5748'' zur Rechten Gottes und
vertrittx-morph=``strongMorph:TG5719'' uns. \bibverse{35} Wer will uns
scheidenx-morph=``strongMorph:TG5692'' von der Liebe Gottes? Trübsal
oder Angst oder Verfolgung oder Hunger oder Blöße oder Fährlichkeit oder
Schwert? \bibverse{36} wie geschrieben
stehtx-morph=``strongMorph:TG5769'': ``Um deinetwillen werden wir
getötetx-morph=''strongMorph:TG5743" den ganzen Tag; wir sind
geachtetx-morph=``strongMorph:TG5681'' wie Schlachtschafe.''
\bibverse{37} Aber in dem allem überwinden wir
weitx-morph=``strongMorph:TG5719'' um deswillen, der uns geliebt
hatx-morph=``strongMorph:TG5660''. \bibverse{38} Denn ich bin
gewißx-morph=``strongMorph:TG5769'', daß weder Tod noch Leben, weder
Engel noch Fürstentümer noch Gewalten, weder
Gegenwärtigesx-morph=``strongMorph:TG5761'' noch
Zukünftigesx-morph=``strongMorph:TG5723'', \bibverse{39} weder Hohes
noch Tiefes noch keine andere Kreatur magx-morph=``strongMorph:TG5695''
uns scheidenx-morph=``strongMorph:TG5658'' von der Liebe Gottes, die in
Christo Jesu ist, unserm HERRN.

\hypertarget{section-8}{%
\section{9}\label{section-8}}

\bibverse{1} Ich sagex-morph=``strongMorph:TG5719'' die Wahrheit in
Christus und lügex-morph=``strongMorph:TG5736'' nicht, wie mir Zeugnis
gibtx-morph=``strongMorph:TG5723'' mein Gewissen in dem Heiligen Geist,
\bibverse{2} daß ich große Traurigkeit und Schmerzen ohne Unterlaß in
meinem Herzen habex-morph=``strongMorph:TG5748''. \bibverse{3} Ich habe
gewünschtx-morph=``strongMorph:TG5711'', verbannt zu
seinx-morph=``strongMorph:TG5750'' von Christo für meine Brüder, die
meine Gefreundeten sind nach dem Fleisch; \bibverse{4} die da
sindx-morph=``strongMorph:TG5748'' von Israel, welchen gehört die
Kindschaft und die Herrlichkeit und der Bund und das Gesetz und der
Gottesdienst und die Verheißungen; \bibverse{5} welcher auch sind die
Väter, und aus welchen Christus herkommt nach dem Fleisch,
derx-morph=``strongMorph:TG5752'' da ist Gott über alles, gelobt in
Ewigkeit. Amen. \bibverse{6} Aber nicht sage ich solches, als ob Gottes
Wort darum aus seix-morph=``strongMorph:TG5758''. Denn es sind nicht
alle Israeliter, die von Israel sind; \bibverse{7} auch nicht alle, die
Abrahams Same sindx-morph=``strongMorph:TG5748'', sind darum auch
Kinder. Sondern ``in Isaak soll dir der Same genannt
seinx-morph=''strongMorph:TG5701"''. \bibverse{8}
Dasx-morph=``strongMorph:TG5748'' ist: nicht sind das Gottes Kinder, die
nach dem Fleisch Kinder sind; sondern die Kinder der Verheißung werden
für Samen gerechnetx-morph=``strongMorph:TG5736''. \bibverse{9} Denn
dies ist ein Wort der Verheißung, da er spricht: ``Um diese Zeit will
ich kommenx-morph=''strongMorph:TG5695``, und Sara soll einen Sohn
habenx-morph=''strongMorph:TG5704``.'' \bibverse{10} Nicht allein aber
ist's mit dem also, sondern auch, da Rebekka von dem
einenx-morph=``strongMorph:TG5723'', unserm Vater Isaak, schwanger ward:
\bibverse{11} ehe die Kinder geboren warenx-morph=``strongMorph:TG5685''
und weder Gutes noch Böses getan hattenx-morph=``strongMorph:TG5660'',
auf daß der Vorsatz Gottes bestündex-morph=``strongMorph:TG5725'' nach
der Wahl, \bibverse{12} nicht aus Verdienst der Werke, sondern aus Gnade
des Berufersx-morph=``strongMorph:TG5723'', ward zu ihr
gesagtx-morph=``strongMorph:TG5681'': ``Der Ältere soll dienstbar
werdenx-morph=''strongMorph:TG5692" dem Jüngeren'', \bibverse{13} wie
denn geschrieben stehtx-morph=``strongMorph:TG5769'': ``Jakob habe ich
geliebtx-morph=''strongMorph:TG5656``, aber Esau habe ich
gehaßtx-morph=''strongMorph:TG5656``.'' \bibverse{14} Was wollen wir
denn hier sagenx-morph=``strongMorph:TG5692''? Ist denn Gott ungerecht?
Das sei fernex-morph=``strongMorph:TG5636''! \bibverse{15} Denn er
sprichtx-morph=``strongMorph:TG5719'' zu Mose: ``Welchem ich gnädig
binx-morph=''strongMorph:TG5725``, dem bin ich
gnädigx-morph=''strongMorph:TG5692``; und welches ich mich
erbarmex-morph=''strongMorph:TG5725``, des erbarme ich
michx-morph=''strongMorph:TG5692``.'' \bibverse{16} So liegt es nun
nicht an jemandes Wollenx-morph=``strongMorph:TG5723'' oder
Laufenx-morph=``strongMorph:TG5723'', sondern an Gottes
Erbarmenx-morph=``strongMorph:TG5723''. \bibverse{17} Denn die Schrift
sagtx-morph=``strongMorph:TG5719'' zum Pharao: ``Ebendarum habe ich dich
erwecktx-morph=''strongMorph:TG5656``, daß ich an dir meine Macht
erzeigex-morph=''strongMorph:TG5672``, auf daß mein Name verkündigt
werdex-morph=''strongMorph:TG5652" in allen Landen.'' \bibverse{18} So
erbarmt er sichx-morph=``strongMorph:TG5719'' nun, welches er
willx-morph=``strongMorph:TG5719'', und
verstocktx-morph=``strongMorph:TG5719'', welchen er
willx-morph=``strongMorph:TG5719''. \bibverse{19} So
sagstx-morph=``strongMorph:TG5692'' du zu mir: Was
beschuldigtx-morph=``strongMorph:TG5736'' er uns denn? Wer kann seinem
Willen widerstehenx-morph=``strongMorph:TG5758''? \bibverse{20} Ja,
lieber Mensch, wer bistx-morph=``strongMorph:TG5748'' du denn, daß du
mit Gott rechten willstx-morph=``strongMorph:TG5740''?
Sprichtx-morph=``strongMorph:TG5692'' auch ein Werk zu seinem
Meisterx-morph=``strongMorph:TG5660'': Warum
machstx-morph=``strongMorph:TG5656'' du mich also? \bibverse{21}
Hatx-morph=``strongMorph:TG5719'' nicht ein Töpfer Macht, aus einem
Klumpen zu machenx-morph=``strongMorph:TG5658'' ein Gefäß zu Ehren und
das andere zu Unehren? \bibverse{22} Derhalben, da Gott
wolltex-morph=``strongMorph:TG5723'' Zorn
erzeigenx-morph=``strongMorph:TG5670'' und
kundtunx-morph=``strongMorph:TG5658'' seine Macht, hat er mit großer
Geduld getragenx-morph=``strongMorph:TG5656'' die Gefäße des Zorns, die
da zugerichtet sindx-morph=``strongMorph:TG5772'' zur Verdammnis;
\bibverse{23} auf daß er kundtätex-morph=``strongMorph:TG5661'' den
Reichtum seiner Herrlichkeit an den Gefäßen der Barmherzigkeit, die er
bereitet hatx-morph=``strongMorph:TG5656'' zur Herrlichkeit,
\bibverse{24} welche er berufen hatx-morph=``strongMorph:TG5656'',
nämlich uns, nicht allein aus den Juden sondern auch aus den Heiden.
\bibverse{25} Wie er denn auch durch Hosea
sprichtx-morph=``strongMorph:TG5719'': ``Ich will das mein Volk
heißenx-morph=''strongMorph:TG5692``, daß nicht mein Volk war, und meine
Liebex-morph=''strongMorph:TG5772``, die nicht meine Liebe
warx-morph=''strongMorph:TG5772``.'' \bibverse{26} ``Und soll
geschehenx-morph=''strongMorph:TG5704``: An dem Ort, da zu ihnen gesagt
wardx-morph=''strongMorph:TG5681``: `Ihr seid nicht mein Volk', sollen
sie Kinder des lebendigenx-morph=''strongMorph:TG5723" Gottes genannt
werdenx-morph=``strongMorph:TG5701''.'' \bibverse{27} Jesaja aber
schreitx-morph=``strongMorph:TG5719'' für Israel: ``Wenn die Zahl der
Kinder Israel würde seinx-morph=''strongMorph:TG5753" wie der Sand am
Meer, so wird doch nur der Überrest selig
werdenx-morph=``strongMorph:TG5701''; \bibverse{28} denn es wird ein
Verderben und Steuernx-morph=``strongMorph:TG5723''
geschehenx-morph=``strongMorph:TG5723'' zur Gerechtigkeit, und der HERR
wird das Steuernx-morph=``strongMorph:TG5772''
tunx-morph=``strongMorph:TG5692'' auf Erden.'' \bibverse{29} Und wie
Jesaja zuvorsagtex-morph=``strongMorph:TG5758'': ``Wenn uns nicht der
HERR Zebaoth hätte lassen Samen übrig
bleibenx-morph=''strongMorph:TG5627``, so
wärenx-morph=''strongMorph:TG5675" wir wie Sodom und Gomorra.''
\bibverse{30} Was wollen wir nun hier
sagenx-morph=``strongMorph:TG5692''? Das wollen wir sagen: Die Heiden,
die nicht haben nach der Gerechtigkeit
getrachtetx-morph=``strongMorph:TG5723'', haben Gerechtigkeit
erlangtx-morph=``strongMorph:TG5627''; ich sage aber von der
Gerechtigkeit, die aus dem Glauben kommt. \bibverse{31} Israel aber hat
dem Gesetz der Gerechtigkeit
nachgetrachtetx-morph=``strongMorph:TG5723'', und hat das Gesetz der
Gerechtigkeit nicht erreichtx-morph=``strongMorph:TG5656''.
\bibverse{32} Warum das? Darum daß sie es nicht aus dem Glauben, sondern
aus den Werken des Gesetzes suchen. Denn sie haben sich
gestoßenx-morph=``strongMorph:TG5656'' an den Stein des Anlaufens,
\bibverse{33} wie geschrieben stehtx-morph=``strongMorph:TG5769'':
``Siehex-morph=''strongMorph:TG5628" da, ich
legex-morph=``strongMorph:TG5719'' in Zion einen Stein des Anlaufens und
einen Fels des Ärgernisses; und wer an ihn
glaubtx-morph=``strongMorph:TG5723'', der soll nicht zu Schanden
werdenx-morph=``strongMorph:TG5701''.''

\hypertarget{section-9}{%
\section{10}\label{section-9}}

\bibverse{1} Liebe Brüder, meines Herzens Wunsch
istx-morph=``strongMorph:TG5748'', und ich flehe auch zu Gott für
Israel, daß sie selig werden. \bibverse{2} Denn ich
gebex-morph=``strongMorph:TG5719'' ihnen das Zeugnis, daß sie
eifernx-morph=``strongMorph:TG5719'' um Gott, aber mit Unverstand.
\bibverse{3} Denn sie erkennenx-morph=``strongMorph:TG5723'' die
Gerechtigkeit nicht, die vor Gott gilt, und
trachtenx-morph=``strongMorph:TG5723'', ihre eigene Gerechtigkeit
aufzurichtenx-morph=``strongMorph:TG5658'', und sind also der
Gerechtigkeit, die vor Gott gilt, nicht
untertanx-morph=``strongMorph:TG5648''. \bibverse{4} Denn Christus ist
des Gesetzes Ende; wer an den glaubtx-morph=``strongMorph:TG5723'', der
ist gerecht. \bibverse{5} Mose schreibtx-morph=``strongMorph:TG5719''
wohl von der Gerechtigkeit, die aus dem Gesetz kommt: ``Welcher Mensch
dies tutx-morph=''strongMorph:TG5660``, der wird dadurch
lebenx-morph=''strongMorph:TG5695``.'' \bibverse{6} Aber die
Gerechtigkeit aus dem Glauben sprichtx-morph=``strongMorph:TG5719''
also: ``Sprichx-morph=''strongMorph:TG5632" nicht in deinem Herzen: Wer
will hinauf gen Himmel fahrenx-morph=``strongMorph:TG5695''?'' (Das
istx-morph=``strongMorph:TG5748'' nichts anderes denn Christum
herabholenx-morph=``strongMorph:TG5629''.) \bibverse{7} Oder: ``Wer will
hinab in die Tiefe fahrenx-morph=''strongMorph:TG5695``?'' (Das
istx-morph=``strongMorph:TG5748'' nichts anderes denn Christum von den
Toten holenx-morph=``strongMorph:TG5629''.) \bibverse{8} Aber was sagt
siex-morph=``strongMorph:TG5719''? ``Das Wort
istx-morph=''strongMorph:TG5748" dir nahe, in deinem Munde und in deinem
Herzen.'' Dies istx-morph=``strongMorph:TG5748'' das Wort vom Glauben,
das wir predigenx-morph=``strongMorph:TG5719''. \bibverse{9} Denn so du
mit deinem Munde bekennstx-morph=``strongMorph:TG5661'' Jesum, daß er
der HERR sei, und glaubstx-morph=``strongMorph:TG5661'' in deinem
Herzen, daß ihn Gott von den Toten auferweckt
hatx-morph=``strongMorph:TG5656'', so wirst du
seligx-morph=``strongMorph:TG5701''. \bibverse{10} Denn so man von
Herzen glaubtx-morph=``strongMorph:TG5743'', so wird man gerecht; und so
man mit dem Munde bekenntx-morph=``strongMorph:TG5743'', so wird man
selig. \bibverse{11} Denn die Schrift
sprichtx-morph=``strongMorph:TG5719'': ``Wer an ihn
glaubtx-morph=''strongMorph:TG5723``, wird nicht zu Schanden
werdenx-morph=''strongMorph:TG5701``.'' \bibverse{12} Es
istx-morph=``strongMorph:TG5748'' hier kein Unterschied unter Juden und
Griechen; es ist aller zumal ein HERR,
reichx-morph=``strongMorph:TG5723'' über alle, die ihn
anrufenx-morph=``strongMorph:TG5734''. \bibverse{13} Denn ``wer den
Namen des HERRN wird anrufenx-morph=''strongMorph:TG5672``, soll selig
werdenx-morph=''strongMorph:TG5701``.'' \bibverse{14} Wie sollen sie
aber den anrufenx-morph=``strongMorph:TG5698'', an den sie nicht
glaubenx-morph=``strongMorph:TG5656''? Wie sollen sie aber an den
glaubenx-morph=``strongMorph:TG5692'', von dem sie nichts gehört
habenx-morph=``strongMorph:TG5656''? wie sollen sie aber
hörenx-morph=``strongMorph:TG5692'' ohne
Predigerx-morph=``strongMorph:TG5723''? \bibverse{15} Wie sollen sie
aber predigenx-morph=``strongMorph:TG5692'', wo sie nicht gesandt
werdenx-morph=``strongMorph:TG5652''? Wie denn geschrieben
stehtx-morph=``strongMorph:TG5769'': ``Wie lieblich sich die Füße derer,
die den Frieden verkündigenx-morph=''strongMorph:TG5734``, die das Gute
verkündigenx-morph=''strongMorph:TG5734``!'' \bibverse{16} Aber sie sind
nicht alle dem Evangelium gehorsamx-morph=``strongMorph:TG5656''. Denn
Jesaja sagtx-morph=``strongMorph:TG5719'': ``HERR, wer
glaubtx-morph=''strongMorph:TG5656" unserm Predigen?'' \bibverse{17} So
kommt der Glaube aus der Predigt, das Predigen aber aus dem Wort Gottes.
\bibverse{18} Ich sagex-morph=``strongMorph:TG5719'' aber: Haben sie es
nicht gehörtx-morph=``strongMorph:TG5656''? Wohl, es ist ja in alle
Lande ausgegangenx-morph=``strongMorph:TG5627'' ihr Schall und in alle
Welt ihre Worte. \bibverse{19} Ich sagex-morph=``strongMorph:TG5719''
aber: Hat es Israel nicht erkanntx-morph=``strongMorph:TG5627''? Aufs
erste sprichtx-morph=``strongMorph:TG5719'' Mose: ``Ich will euch eifern
machenx-morph=''strongMorph:TG5692" über dem, das nicht ein Volk ist;
und über ein unverständiges Volk will ich euch
erzürnenx-morph=``strongMorph:TG5692''.'' \bibverse{20} Jesaja aber darf
wohlx-morph=``strongMorph:TG5719'' so
sagenx-morph=``strongMorph:TG5719'': ``Ich bin
gefundenx-morph=''strongMorph:TG5681" von denen, die mich nicht gesucht
habenx-morph=``strongMorph:TG5723'', und
binx-morph=``strongMorph:TG5633'' erschienen denen, die nicht nach mir
gefragt habenx-morph=``strongMorph:TG5723''.'' \bibverse{21} Zu Israel
aber sprichtx-morph=``strongMorph:TG5719'' er: ``Den ganzen Tag habe ich
meine Hände ausgestrecktx-morph=''strongMorph:TG5656" zu dem Volk, das
sich nicht sagen läßtx-morph=``strongMorph:TG5723'' und
widersprichtx-morph=``strongMorph:TG5723''.''

\hypertarget{section-10}{%
\section{11}\label{section-10}}

\bibverse{1} So sagex-morph=``strongMorph:TG5719'' ich nun: Hat denn
Gott sein Volk verstoßenx-morph=``strongMorph:TG5662''? Das sei
fernex-morph=``strongMorph:TG5636''! Denn ich
binx-morph=``strongMorph:TG5748'' auch ein Israeliter von dem Samen
Abrahams, aus dem Geschlecht Benjamin. \bibverse{2} Gott hat sein Volk
nicht verstoßenx-morph=``strongMorph:TG5662'', welches er zuvor ersehen
hatx-morph=``strongMorph:TG5656''. Oder wisset
ihrx-morph=``strongMorph:TG5758'' nicht, was die Schrift
sagtx-morph=``strongMorph:TG5719'' von Elia, wie er
trittx-morph=``strongMorph:TG5719'' vor Gott wider Israel und
sprichtx-morph=``strongMorph:TG5723'': \bibverse{3} ``HERR, sie haben
deine Propheten getötetx-morph=''strongMorph:TG5656" und deine Altäre
zerbrochenx-morph=``strongMorph:TG5656''; und ich bin allein
übriggebliebenx-morph=``strongMorph:TG5681'', und sie
stehenx-morph=``strongMorph:TG5719'' mir nach meinem Leben''?
\bibverse{4} Aber was sagtx-morph=``strongMorph:TG5719'' die göttliche
Antwort? ``Ich habe mir lassen übrig
bleibenx-morph=''strongMorph:TG5627" siebentausend Mann, die nicht haben
ihre Kniee gebeugtx-morph=``strongMorph:TG5656'' vor dem Baal.''
\bibverse{5} Also gehet es auch jetzt zu dieser Zeit mit diesen, die
übriggeblieben sindx-morph=``strongMorph:TG5754'' nach der Wahl der
Gnade. \bibverse{6} Ist's aber aus Gnaden, so ist's nicht aus Verdienst
der Werke; sonst würde Gnade nicht Gnade
seinx-morph=``strongMorph:TG5736''. Ist's aber aus Verdienst der Werke,
so istx-morph=``strongMorph:TG5748'' die Gnade nichts; sonst
wärex-morph=``strongMorph:TG5748'' Verdienst nicht Verdienst.
\bibverse{7} Wie denn nun? Was Israel
suchtx-morph=``strongMorph:TG5719'', das
erlangtex-morph=``strongMorph:TG5627'' es nicht; die Auserwählten aber
erlangtenx-morph=``strongMorph:TG5627'' es. Die andern sind
verstocktx-morph=``strongMorph:TG5681'', \bibverse{8} wie geschrieben
stehtx-morph=``strongMorph:TG5769'': ``Gott hat ihnen
gegebenx-morph=''strongMorph:TG5656" eine Geist des Schlafs, Augen, daß
sie nicht sehenx-morph=``strongMorph:TG5721'', und Ohren, daß sie nicht
hörenx-morph=``strongMorph:TG5721'', bis auf den heutigen Tag.''
\bibverse{9} Und David sprichtx-morph=``strongMorph:TG5719'': ``Laß
ihren Tisch zu einem Strick werdenx-morph=''strongMorph:TG5676" und zu
einer Berückung und zum Ärgernis und ihnen zur Vergeltung. \bibverse{10}
Verblendex-morph=``strongMorph:TG5682'' ihre Augen, daß sie nicht
sehenx-morph=``strongMorph:TG5721'', und
beugex-morph=``strongMorph:TG5657'' ihren Rücken allezeit.''
\bibverse{11} So sagex-morph=``strongMorph:TG5719'' ich nun: Sind sie
darum angelaufenx-morph=``strongMorph:TG5656'', daß sie fallen
solltenx-morph=``strongMorph:TG5632''? Das sei
fernex-morph=``strongMorph:TG5636''! Sondern aus ihrem Fall ist den
Heiden das Heil widerfahren, auf daß sie denen nacheifern
solltenx-morph=``strongMorph:TG5658''. \bibverse{12} Denn so ihr Fall
der Welt Reichtum ist, und ihr Schade ist der Heiden Reichtum, wie viel
mehr, wenn ihre Zahl voll würde? \bibverse{13} Mit euch Heiden
redex-morph=``strongMorph:TG5719'' ich; denn dieweil ich der Heiden
Apostel binx-morph=``strongMorph:TG5748'', will ich mein Amt
preisenx-morph=``strongMorph:TG5719'', \bibverse{14} ob ich möchte die,
so mein Fleisch sind, zu eifern reizenx-morph=``strongMorph:TG5661'' und
ihrer etliche selig machenx-morph=``strongMorph:TG5661''. \bibverse{15}
Denn so ihre Verwerfung der Welt Versöhnung ist, was wird ihre Annahme
anders sein als Leben von den Toten? \bibverse{16} Ist der Anbruch
heilig, so ist auch der Teig heilig; und so die Wurzel heilig ist, so
sind auch die Zweige heilig. \bibverse{17} Ob aber nun etliche von den
Zweigen ausgebrochen sindx-morph=``strongMorph:TG5681'' und du, da du
ein wilder Ölbaum warstx-morph=``strongMorph:TG5752'', bist unter sie
gepfropftx-morph=``strongMorph:TG5681'' und teilhaftig
gewordenx-morph=``strongMorph:TG5633'' der Wurzel und des Safts im
Ölbaum, \bibverse{18} so rühmex-morph=``strongMorph:TG5737'' dich nicht
wider die Zweige. Rühmstx-morph=``strongMorph:TG5736'' du dich aber
wider sie, so sollst du wissen, daß du die Wurzel nicht
trägstx-morph=``strongMorph:TG5719'', sondern die Wurzel trägt dich.
\bibverse{19} So sprichstx-morph=``strongMorph:TG5692'' du: Die Zweige
sind ausgebrochenx-morph=``strongMorph:TG5681'', das ich hineingepfropft
würdex-morph=``strongMorph:TG5686''. \bibverse{20} Ist wohl geredet! Sie
sind ausgebrochenx-morph=``strongMorph:TG5681'' um ihres Unglaubens
willen; du stehstx-morph=``strongMorph:TG5758'' aber durch den Glauben.
Seix-morph=``strongMorph:TG5720'' nicht stolz, sondern
fürchtex-morph=``strongMorph:TG5732''\textbar x-morph=``strongMorph:TG5737''
dich. \bibverse{21} Hat Gott die natürlichen Zweige nicht
verschontx-morph=``strongMorph:TG5662'', daß er vielleicht dich auch
nicht verschonex-morph=``strongMorph:TG5667''. \bibverse{22} Darum
schaux-morph=``strongMorph:TG5657'' die Güte und den Ernst Gottes: den
Ernst an denen, die gefallen sindx-morph=``strongMorph:TG5631'', die
Güte aber an dir, sofern du an der Güte
bleibstx-morph=``strongMorph:TG5661''; sonst wirst du auch abgehauen
werdenx-morph=``strongMorph:TG5691''. \bibverse{23} Und jene, so nicht
bleibenx-morph=``strongMorph:TG5661'' in dem Unglauben, werden
eingepfropft werdenx-morph=``strongMorph:TG5701''; Gott
kannx-morph=``strongMorph:TG5748'' sie wohl wieder
einpfropfenx-morph=``strongMorph:TG5658''. \bibverse{24} Denn so du aus
dem Ölbaum, der von Natur aus wild warx-morph=``strongMorph:TG5648'',
bist abgehauenx-morph=``strongMorph:TG5681'' und wider die Natur in den
guten Ölbaum gepropft, wie viel mehr werden die natürlichen eingepropft
inx-morph=``strongMorph:TG5701'' ihren eigenen Ölbaum. \bibverse{25} Ich
willx-morph=``strongMorph:TG5719'' euch nicht
verhaltenx-morph=``strongMorph:TG5721'', liebe Brüder, dieses Geheimnis
(auf daß ihr nicht stolz seidx-morph=``strongMorph:TG5753''): Blindheit
ist Israel zum Teil widerfahrenx-morph=``strongMorph:TG5754'', so lange,
bis die Fülle der Heiden eingegangen seix-morph=``strongMorph:TG5632''
\bibverse{26} und also das ganze Israel selig
werdex-morph=``strongMorph:TG5701'', wie geschrieben
stehtx-morph=``strongMorph:TG5769'': ``Es wird
kommenx-morph=''strongMorph:TG5692" aus Zion, der da
erlösex-morph=``strongMorph:TG5740'' und
abwendex-morph=``strongMorph:TG5692'' das gottlose Wesen von Jakob.
\bibverse{27} Und dies ist mein Testament mit ihnen, wenn ich ihre
Sünden werde wegnehmenx-morph=``strongMorph:TG5643''.'' \bibverse{28}
Nach dem Evangelium sind sie zwar Feinde um euretwillen; aber nach der
Wahl sind sie Geliebte um der Väter willen. \bibverse{29} Gottes Gaben
und Berufung können ihn nicht gereuen. \bibverse{30} Denn gleicherweise
wie auch ihr weiland nicht habt geglaubtx-morph=``strongMorph:TG5656''
an Gott, nun aber Barmherzigkeit überkommen
habtx-morph=``strongMorph:TG5681'' durch ihren Unglauben, \bibverse{31}
also haben auch jene jetzt nicht wollen
glaubenx-morph=``strongMorph:TG5656'' an die Barmherzigkeit, die euch
widerfahren ist, auf daß sie auch Barmherzigkeit
überkommenx-morph=``strongMorph:TG5686''. \bibverse{32} Denn Gott hat
alle beschlossenx-morph=``strongMorph:TG5656'' unter den Unglauben, auf
daß er sich aller erbarmex-morph=``strongMorph:TG5661''. \bibverse{33} O
welch eine Tiefe des Reichtums, beides, der Weisheit und Erkenntnis
Gottes! Wie gar unbegreiflich sind sein Gerichte und unerforschlich
seine Wege! \bibverse{34} Denn wer hat des HERRN Sinn
erkanntx-morph=``strongMorph:TG5627'', oder wer ist sein Ratgeber
gewesenx-morph=``strongMorph:TG5633''? \bibverse{35} Oder wer hat ihm
etwas zuvor gegebenx-morph=``strongMorph:TG5656'', daß ihm werde
wiedervergoltenx-morph=``strongMorph:TG5701''? \bibverse{36} Denn von
ihm und durch ihn und zu ihm sind alle Dinge. Ihm sei Ehre in Ewigkeit!
Amen.

\hypertarget{section-11}{%
\section{12}\label{section-11}}

\bibverse{1} Ich ermahnex-morph=``strongMorph:TG5719'' euch nun, liebe
Brüder, durch die Barmherzigkeit Gottes, daß ihr eure Leiber
begebetx-morph=``strongMorph:TG5658'' zum Opfer, das da
lebendigx-morph=``strongMorph:TG5723'', heilig und Gott wohlgefällig
sei, welches sei euer vernünftiger Gottesdienst. \bibverse{2} Und
stelletx-morph=``strongMorph:TG5728'' euch nicht dieser Welt gleich,
sondern verändertx-morph=``strongMorph:TG5744'' euch durch die
Erneuerung eures Sinnes, auf daß ihr prüfen
mögetx-morph=``strongMorph:TG5721'', welches da sei der gute,
wohlgefällige und vollkommene Gotteswille. \bibverse{3}
Dennx-morph=``strongMorph:TG5719'' ich sage euch durch die Gnade, die
mir gegeben istx-morph=``strongMorph:TG5685'',
jedermannx-morph=``strongMorph:TG5752'' unter euch, daß niemand weiter
von sich haltex-morph=``strongMorph:TG5721'', als sich's
gebührtx-morph=``strongMorph:TG5748'' zu
haltenx-morph=``strongMorph:TG5721'', sondern daß er von sich
mäßigx-morph=``strongMorph:TG5721'' haltex-morph=``strongMorph:TG5721'',
ein jeglicher, nach dem Gott ausgeteilt
hatx-morph=``strongMorph:TG5656'' das Maß des Glaubens. \bibverse{4}
Denn gleicherweise als wir in einem Leibe viele Glieder
habenx-morph=``strongMorph:TG5719'', aber alle Glieder nicht einerlei
Geschäft habenx-morph=``strongMorph:TG5719'', \bibverse{5} also
sindx-morph=``strongMorph:TG5748'' wir viele ein Leib in Christus, aber
untereinander ist einer des andern Glied, \bibverse{6} und
habenx-morph=``strongMorph:TG5723'' mancherlei Gaben nach der Gnade, die
uns gegeben istx-morph=``strongMorph:TG5685''. \bibverse{7} Hat jemand
Weissagung, so sei sie dem Glauben gemäß. Hat jemand ein Amt, so warte
er des Amts. Lehrtx-morph=``strongMorph:TG5723'' jemand, so warte er der
Lehre. \bibverse{8} Ermahntx-morph=``strongMorph:TG5723'' jemand, so
warte er des Ermahnens. Gibtx-morph=``strongMorph:TG5723'' jemand, so
gebe er einfältig. Regiertx-morph=``strongMorph:TG5734'' jemand, so sei
er sorgfältig. Übt jemand Barmherzigkeitx-morph=``strongMorph:TG5723'',
so tue er's mit Lust. \bibverse{9} Die Liebe sei nicht falsch.
Hassetx-morph=``strongMorph:TG5723'' das Arge,
hangetx-morph=``strongMorph:TG5746'' dem Guten an. \bibverse{10} Die
brüderliche Liebe untereinander sei herzlich. Einer
kommex-morph=``strongMorph:TG5740'' dem andern mit Ehrerbietung zuvor.
\bibverse{11} Seid nicht träge in dem, was ihr tun sollt. Seid
brünstigx-morph=``strongMorph:TG5723'' im Geiste.
Schicketx-morph=``strongMorph:TG5723'' euch in die Zeit. \bibverse{12}
Seid fröhlichx-morph=``strongMorph:TG5723'' in Hoffnung,
geduldigx-morph=``strongMorph:TG5723'' in Trübsal, haltet
anx-morph=``strongMorph:TG5723'' am Gebet. \bibverse{13}
Nehmetx-morph=``strongMorph:TG5723'' euch der Notdurft der Heiligen an.
Herbergetx-morph=``strongMorph:TG5723'' gern. \bibverse{14}
Segnetx-morph=``strongMorph:TG5720'', die euch
verfolgenx-morph=``strongMorph:TG5723'';
segnetx-morph=``strongMorph:TG5720'' und
fluchetx-morph=``strongMorph:TG5737'' nicht. \bibverse{15}
Freutx-morph=``strongMorph:TG5721'' euch mit den
Fröhlichenx-morph=``strongMorph:TG5723'' und
weintx-morph=``strongMorph:TG5721'' mit den
Weinendenx-morph=``strongMorph:TG5723''. \bibverse{16} Habt einerlei
Sinnx-morph=``strongMorph:TG5723'' untereinander.
Trachtetx-morph=``strongMorph:TG5723'' nicht nach hohen Dingen, sondern
haltetx-morph=``strongMorph:TG5734'' euch herunter zu den Niedrigen.
\bibverse{17} Haltetx-morph=``strongMorph:TG5737'' euch nicht selbst für
klug. Vergeltetx-morph=``strongMorph:TG5723'' niemand Böses mit Bösem.
Fleißigtx-morph=``strongMorph:TG5734'' euch der Ehrbarkeit gegen
jedermann. \bibverse{18} Ist es möglich, soviel an euch ist, so habt mit
allen Menschen Friedenx-morph=``strongMorph:TG5723''. \bibverse{19}
Rächetx-morph=``strongMorph:TG5723'' euch selber nicht, meine Liebsten,
sondern gebetx-morph=``strongMorph:TG5628'' Raum dem Zorn Gottes; denn
es steht geschriebenx-morph=``strongMorph:TG5769'': ``Die Rache ist
mein; ich will vergeltenx-morph=''strongMorph:TG5692``,
sprichtx-morph=''strongMorph:TG5719" der HERR.'' \bibverse{20} So nun
deinen Feind hungertx-morph=``strongMorph:TG5725'', so
speisex-morph=``strongMorph:TG5720'' ihn;
dürstetx-morph=``strongMorph:TG5725'' ihn, so
tränkex-morph=``strongMorph:TG5720'' ihn. Wenn du das
tustx-morph=``strongMorph:TG5723'', so wirst du feurige Kohlen auf sein
Haupt sammelnx-morph=``strongMorph:TG5692''. \bibverse{21} Laß dich
nicht das Böse überwindenx-morph=``strongMorph:TG5744'', sondern
überwindex-morph=``strongMorph:TG5720'' das Böse mit Gutem.

\hypertarget{section-12}{%
\section{13}\label{section-12}}

\bibverse{1} Jedermann sei untertanx-morph=``strongMorph:TG5732'' der
Obrigkeit, die Gewalt über ihn hatx-morph=``strongMorph:TG5723''. Denn
es istx-morph=``strongMorph:TG5748'' keine Obrigkeit ohne von Gott; wo
aber Obrigkeit istx-morph=``strongMorph:TG5752'', die
istx-morph=``strongMorph:TG5748'' von Gott
verordnetx-morph=``strongMorph:TG5772''. \bibverse{2} Wer sich nun der
Obrigkeit widersetztx-morph=``strongMorph:TG5734'', der
widerstrebtx-morph=``strongMorph:TG5758'' Gottes Ordnung; die aber
widerstrebenx-morph=``strongMorph:TG5761'', werden über sich ein Urteil
empfangenx-morph=``strongMorph:TG5695''. \bibverse{3} Denn die
Gewaltigen sindx-morph=``strongMorph:TG5748'' nicht den guten Werken,
sondern den bösen zu fürchten. Willst dux-morph=``strongMorph:TG5719''
dich aber nicht fürchtenx-morph=``strongMorph:TG5738'' vor der
Obrigkeit, so tuex-morph=``strongMorph:TG5720'' Gutes, so wirst du Lob
von ihr habenx-morph=``strongMorph:TG5692''. \bibverse{4} Denn sie
istx-morph=``strongMorph:TG5748'' Gottes Dienerin dir zu gut.
Tustx-morph=``strongMorph:TG5725'' du aber Böses, so fürchte
dichx-morph=``strongMorph:TG5732''\textbar x-morph=``strongMorph:TG5737'';
denn sie trägtx-morph=``strongMorph:TG5719'' das Schwert nicht umsonst;
sie istx-morph=``strongMorph:TG5748'' Gottes Dienerin, eine Rächerin zur
Strafe über den, der Böses tutx-morph=``strongMorph:TG5723''.
\bibverse{5} Darum ist's not, untertan zu
seinx-morph=``strongMorph:TG5733'', nicht allein um der Strafe willen,
sondern auch um des Gewissens willen. \bibverse{6} Derhalben müßt ihr
auch Schoß gebenx-morph=``strongMorph:TG5719''; denn sie
sindx-morph=``strongMorph:TG5748'' Gottes Diener, die solchen Schutz
handhabenx-morph=``strongMorph:TG5723''. \bibverse{7} So
gebetx-morph=``strongMorph:TG5628'' nun jedermann, was ihr schuldig
seid: Schoß, dem der Schoß gebührt; Zoll, dem der Zoll gebührt; Furcht,
dem die Furcht gebührt; Ehre, dem die Ehre gebührt. \bibverse{8}
Seidx-morph=``strongMorph:TG5720'' niemand nichts schuldig, als daß ihr
euch untereinander liebtx-morph=``strongMorph:TG5721''; denn wer den
andern liebtx-morph=``strongMorph:TG5723'', der hat das Gesetz
erfülltx-morph=``strongMorph:TG5758''. \bibverse{9} Denn was da gesagt
ist: ``Du sollst nicht ehebrechenx-morph=''strongMorph:TG5692``; du
sollst nicht tötenx-morph=''strongMorph:TG5692``; du sollst nicht
stehlenx-morph=''strongMorph:TG5692``; du sollst nicht falsch Zeugnis
gebenx-morph=''strongMorph:TG5692``; dich soll nichts
gelüstenx-morph=''strongMorph:TG5692"'', und so ein anderes Gebot mehr
ist, das wird in diesen Worten
zusammengefaßtx-morph=``strongMorph:TG5743'': ``Du sollst deinen
Nächsten liebenx-morph=''strongMorph:TG5692" wie dich selbst.''
\bibverse{10} Denn Liebe tutx-morph=``strongMorph:TG5736'' dem Nächsten
nichts Böses. So ist nun die Liebe des Gesetzes Erfüllung. \bibverse{11}
Und weil wir solches wissenx-morph=``strongMorph:TG5761'', nämlich die
Zeit, daß die Stunde da ist, aufzustehenx-morph=``strongMorph:TG5683''
vom Schlaf (sintemal unser Heil jetzt näher ist, denn da wir gläubig
wurdenx-morph=``strongMorph:TG5656''; \bibverse{12} die Nacht ist
vorgerücktx-morph=``strongMorph:TG5656'', der Tag aber nahe
herbeigekommenx-morph=``strongMorph:TG5758''): so lasset uns
ablegenx-morph=``strongMorph:TG5643'' die Werke der Finsternis und
anlegenx-morph=``strongMorph:TG5672'' die Waffen des Lichtes.
\bibverse{13} Lasset uns ehrbar wandelnx-morph=``strongMorph:TG5661''
als am Tage, nicht in Fressen und Saufen, nicht in Kammern und Unzucht,
nicht in Hader und Neid; \bibverse{14} sondern ziehet
anx-morph=``strongMorph:TG5669'' den HERRN Jesus Christus und
wartetx-morph=``strongMorph:TG5732'' des Leibes, doch also, daß er nicht
geil werde.

\hypertarget{section-13}{%
\section{14}\label{section-13}}

\bibverse{1} Den Schwachenx-morph=``strongMorph:TG5723'' im Glauben
nehmet aufx-morph=``strongMorph:TG5732'' und verwirrt die Gewissen
nicht. \bibverse{2} Einer glaubtx-morph=``strongMorph:TG5719'' er möge
allerlei essenx-morph=``strongMorph:TG5629''; welcher aber schwach
istx-morph=``strongMorph:TG5723'', der ißtx-morph=``strongMorph:TG5719''
Kraut. \bibverse{3} Welcher ißtx-morph=``strongMorph:TG5723'', der
verachtex-morph=``strongMorph:TG5720'' den nicht, der da nicht
ißtx-morph=``strongMorph:TG5723''; und welcher nicht
ißtx-morph=``strongMorph:TG5723'', der
richtex-morph=``strongMorph:TG5720'' den nicht, der da
ißtx-morph=``strongMorph:TG5723''; denn Gott hat ihn
aufgenommenx-morph=``strongMorph:TG5639''. \bibverse{4} Wer
bistx-morph=``strongMorph:TG5748'' du, daß du einen fremden Knecht
richtestx-morph=``strongMorph:TG5723''? Er
stehtx-morph=``strongMorph:TG5719'' oder
fälltx-morph=``strongMorph:TG5719'' seinem HERRN. Er mag aber wohl
aufgerichtet werdenx-morph=``strongMorph:TG5701''; denn Gott
kannx-morph=``strongMorph:TG5748'' ihn wohl
aufrichtenx-morph=``strongMorph:TG5658''. \bibverse{5} Einer
hältx-morph=``strongMorph:TG5719'' einen Tag vor dem andern; der andere
aber hältx-morph=``strongMorph:TG5719'' alle Tage gleich. Ein jeglicher
seix-morph=``strongMorph:TG5744'' in seiner Meinung gewiß. \bibverse{6}
Welcher auf die Tage hältx-morph=``strongMorph:TG5723'', der
tut'sx-morph=``strongMorph:TG5719'' dem HERRN; und welcher nichts darauf
hältx-morph=``strongMorph:TG5723'', der
tut'sx-morph=``strongMorph:TG5719'' auch dem HERRN. Welcher
ißtx-morph=``strongMorph:TG5723'', der ißtx-morph=``strongMorph:TG5719''
dem HERRN, denn er danktx-morph=``strongMorph:TG5719'' Gott; welcher
nicht ißtx-morph=``strongMorph:TG5723'', der
ißtx-morph=``strongMorph:TG5719'' dem HERRN nicht und
danktx-morph=``strongMorph:TG5719'' Gott. \bibverse{7} Denn unser keiner
lebtx-morph=``strongMorph:TG5719'' sich selber, und keiner
stirbtx-morph=``strongMorph:TG5719'' sich selber. \bibverse{8}
Lebenx-morph=``strongMorph:TG5725'' wir, so
lebenx-morph=``strongMorph:TG5719'' wir dem HERRN;
sterbenx-morph=``strongMorph:TG5725'' wir, so
sterbenx-morph=``strongMorph:TG5719'' wir dem HERRN. Darum, wir
lebenx-morph=``strongMorph:TG5725'' oder
sterbenx-morph=``strongMorph:TG5725'', so
sindx-morph=``strongMorph:TG5748'' wir des HERRN. \bibverse{9} Denn dazu
ist Christus auch gestorbenx-morph=``strongMorph:TG5627'' und
auferstandenx-morph=``strongMorph:TG5627'' und wieder lebendig
gewordenx-morph=``strongMorph:TG5656'', daß er über Tote und
Lebendigex-morph=``strongMorph:TG5723'' HERR
seix-morph=``strongMorph:TG5661''. \bibverse{10} Du aber, was
richtestx-morph=``strongMorph:TG5719'' du deinen Bruder? Oder, du
anderer, was verachtestx-morph=``strongMorph:TG5719'' du deinen Bruder?
Wir werden alle vor den Richtstuhl Christi dargestellt
werdenx-morph=``strongMorph:TG5695''; \bibverse{11} denn es steht
geschriebenx-morph=``strongMorph:TG5769'': ``So wahr ich
lebex-morph=''strongMorph:TG5719``, sprichtx-morph=''strongMorph:TG5719"
der HERR, mir sollen alle Kniee gebeugt
werdenx-morph=``strongMorph:TG5692'', und alle Zungen sollen Gott
bekennenx-morph=``strongMorph:TG5698''.'' \bibverse{12} So wird nun ein
jeglicher für sich selbst Gott Rechenschaft
gebenx-morph=``strongMorph:TG5692''. \bibverse{13} Darum lasset uns
nicht mehr einer den andern richtenx-morph=``strongMorph:TG5725'';
sondern das richtetx-morph=``strongMorph:TG5657'' vielmehr, daß niemand
seinem Bruder einen Anstoß oder Ärgernis
darstellex-morph=``strongMorph:TG5721''. \bibverse{14} Ich
weißx-morph=``strongMorph:TG5758'' und bin
gewißx-morph=``strongMorph:TG5769'' in dem HERRN Jesus, daß nichts
gemein ist an sich selbst; nur dem, der es
rechnetx-morph=``strongMorph:TG5740'' fürx-morph=``strongMorph:TG5750''
gemein, dem ist's gemein. \bibverse{15} So aber dein Bruder um deiner
Speise willen betrübt wirdx-morph=``strongMorph:TG5743'', so
wandelstx-morph=``strongMorph:TG5719'' du schon nicht nach der Liebe.
Verderbex-morph=``strongMorph:TG5720'' den nicht mit deiner Speise, um
welches willen Christus gestorben istx-morph=``strongMorph:TG5627''.
\bibverse{16} Darum schaffet, daß euer Schatz nicht verlästert
werdex-morph=``strongMorph:TG5744''. \bibverse{17} Denn das Reich Gottes
istx-morph=``strongMorph:TG5748'' nicht Essen und Trinken, sondern
Gerechtigkeit und Friede und Freude in dem heiligen Geiste.
\bibverse{18} Wer darin Christo dientx-morph=``strongMorph:TG5723'', der
ist Gott gefällig und den Menschen wert. \bibverse{19} Darum laßt uns
dem nachstrebenx-morph=``strongMorph:TG5725'', was zum Frieden dient und
was zur Besserung untereinander dient. \bibverse{20}
Verstörex-morph=``strongMorph:TG5720'' nicht um der Speise willen Gottes
Werk. Es ist zwar alles rein; aber es ist nicht gut dem, der es
ißtx-morph=``strongMorph:TG5723'' mit einem Anstoß seines Gewissens.
\bibverse{21} Es ist besser, du essestx-morph=``strongMorph:TG5629''
kein Fleisch und trinkestx-morph=``strongMorph:TG5629'' keinen Wein und
tust nichts, daran sich dein Bruder stößtx-morph=``strongMorph:TG5719''
oder ärgertx-morph=``strongMorph:TG5743'' oder schwach
wirdx-morph=``strongMorph:TG5719''. \bibverse{22}
Hastx-morph=``strongMorph:TG5719'' du den Glauben, so
habex-morph=``strongMorph:TG5720'' ihn bei dir selbst vor Gott. Selig
ist, der sich selbst kein Gewissen machtx-morph=``strongMorph:TG5723''
in dem, was er annimmtx-morph=``strongMorph:TG5719''. \bibverse{23} Wer
aber darüber zweifeltx-morph=``strongMorph:TG5734'', und
ißtx-morph=``strongMorph:TG5632'' doch, der ist
verdammtx-morph=``strongMorph:TG5769''; denn es geht nicht aus dem
Glauben. Was aber nicht aus dem Glauben
gehtx-morph=``strongMorph:TG5748'', das ist Sünde.

\hypertarget{section-14}{%
\section{15}\label{section-14}}

\bibverse{1} Wir aber, die wir stark sind,
sollenx-morph=``strongMorph:TG5719'' der Schwachen Gebrechlichkeit
tragenx-morph=``strongMorph:TG5721'' und nicht gefallen an uns selber
habenx-morph=``strongMorph:TG5721''. \bibverse{2} Es stelle sich ein
jeglicher unter uns also, daß er seinem Nächsten
gefallex-morph=``strongMorph:TG5720'' zum Guten, zur Besserung.
\bibverse{3} Denn auch Christus hattex-morph=``strongMorph:TG5656''
nicht an sich selber Gefallen, sondern wie geschrieben
stehtx-morph=``strongMorph:TG5769'': ``Die Schmähungen derer, die dich
schmähenx-morph=''strongMorph:TG5723``, sind auf mich
gefallenx-morph=''strongMorph:TG5627``.'' \bibverse{4} Was aber zuvor
geschrieben istx-morph=``strongMorph:TG5648'', das ist uns zur Lehre
geschriebenx-morph=``strongMorph:TG5648'', auf daß wir durch Geduld und
Trost der Schrift Hoffnung habenx-morph=``strongMorph:TG5725''.
\bibverse{5} Der Gott aber der Geduld und des Trostes
gebex-morph=``strongMorph:TG5630'' euch, daß ihr einerlei
gesinntx-morph=``strongMorph:TG5721'' seid untereinander nach Jesu
Christo, \bibverse{6} auf daß ihr einmütig mit einem Munde
lobetx-morph=``strongMorph:TG5725'' Gott und den Vater unseres HERRN
Jesu Christi. \bibverse{7} Darum nehmetx-morph=``strongMorph:TG5732''
euch untereinander auf, gleichwie euch Christus hat
aufgenommenx-morph=``strongMorph:TG5639'' zu Gottes Lobe. \bibverse{8}
Ich sagex-morph=``strongMorph:TG5719'' aber, daß Jesus Christus
seix-morph=``strongMorph:TG5771'' ein Diener gewesen der Juden um der
Wahrhaftigkeit willen Gottes, zu
bestätigenx-morph=``strongMorph:TG5658'' die Verheißungen, den Vätern
geschehen; \bibverse{9} daß die Heiden aber Gott
lobenx-morph=``strongMorph:TG5658'' um der Barmherzigkeit willen, wie
geschrieben stehtx-morph=``strongMorph:TG5769'': ``Darum will ich dich
lobenx-morph=''strongMorph:TG5698" unter den Heiden und deinem Namen
singenx-morph=``strongMorph:TG5692''.'' \bibverse{10} Und abermals
spricht erx-morph=``strongMorph:TG5719'': ``Freut
euchx-morph=''strongMorph:TG5682``, ihr Heiden, mit seinem Volk!''
\bibverse{11} Und abermals: ``Lobtx-morph=''strongMorph:TG5720" den
HERRN, alle Heiden, und preisetx-morph=``strongMorph:TG5657'' ihn, alle
Völker!'' \bibverse{12} Und abermals
sprichtx-morph=``strongMorph:TG5719'' Jesaja: ``Es wird
seinx-morph=''strongMorph:TG5704" die Wurzel Jesse's, und der
auferstehen wirdx-morph=``strongMorph:TG5734'', zu herrschen
überx-morph=``strongMorph:TG5721'' die Heiden; auf den werden die Heiden
hoffenx-morph=``strongMorph:TG5692''.'' \bibverse{13} Der Gott aber der
Hoffnung erfüllex-morph=``strongMorph:TG5659'' euch mit aller Freude und
Frieden im Glaubenx-morph=``strongMorph:TG5721'', daß ihr völlige
Hoffnung habetx-morph=``strongMorph:TG5721'' durch die Kraft des
heiligen Geistes. \bibverse{14} Ich weißx-morph=``strongMorph:TG5769''
aber gar wohl von euch, liebe Brüder, daß ihr selber voll Gütigkeit
seidx-morph=``strongMorph:TG5748'',
erfülltx-morph=``strongMorph:TG5772'' mit Erkenntnis, daß ihr euch
untereinander könnetx-morph=``strongMorph:TG5740''
ermahnenx-morph=``strongMorph:TG5721''. \bibverse{15} Ich habe es aber
dennoch gewagt und euch etwas wollen
schreibenx-morph=``strongMorph:TG5656'', liebe Brüder, euch zu
erinnernx-morph=``strongMorph:TG5723'', um der Gnade willen, die mir von
Gott gegeben istx-morph=``strongMorph:TG5685'', \bibverse{16} daß ich
soll seinx-morph=``strongMorph:TG5750'' ein Diener Christi unter den
Heiden, priesterlichx-morph=``strongMorph:TG5723'' zu warten des
Evangeliums Gottes, auf daß die Heiden ein Opfer
werden,x-morph=``strongMorph:TG5638'' Gott angenehm,
geheiligtx-morph=``strongMorph:TG5772'' durch den heiligen Geist.
\bibverse{17} Darum kannx-morph=``strongMorph:TG5719'' ich mich rühmen
in Jesus Christo, daß ich Gott diene. \bibverse{18} Denn ich wollte
nicht wagenx-morph=``strongMorph:TG5692'', etwas zu
redenx-morph=``strongMorph:TG5721'', wo dasselbe Christus nicht durch
mich wirktex-morph=``strongMorph:TG5662'', die Heiden zum Gehorsam zu
bringen durch Wort und Werk, \bibverse{19} durch Kraft der Zeichen und
Wunder und durch Kraft des Geistes Gottes, also daß ich von Jerusalem an
und umher bis Illyrien alles mit dem Evangelium Christi erfüllt
habex-morph=``strongMorph:TG5760'' \bibverse{20} und mich sonderlich
geflissenx-morph=``strongMorph:TG5740'', das Evangelium zu
predigenx-morph=``strongMorph:TG5733'', wo Christi
Namex-morph=``strongMorph:TG5681'' nicht bekannt war, auf daß ich nicht
auf einen fremden Grund bautex-morph=``strongMorph:TG5725'',
\bibverse{21} sondern wie geschrieben
stehtx-morph=``strongMorph:TG5769'': ``Welchen ist nicht von ihm
verkündigtx-morph=''strongMorph:TG5648``, die sollen's
sehenx-morph=''strongMorph:TG5695``, und welche nicht gehört
habenx-morph=''strongMorph:TG5754``, sollen's
verstehenx-morph=''strongMorph:TG5704``.'' \bibverse{22} Das ist auch
die Ursache, warum ich vielmal verhindert
wordenx-morph=``strongMorph:TG5712'', zu euch zu
kommenx-morph=``strongMorph:TG5629''. \bibverse{23} Nun ich aber nicht
mehr Raum habex-morph=``strongMorph:TG5723'' in diesen Ländern,
habex-morph=``strongMorph:TG5723'' aber Verlangen, zu euch zu
kommenx-morph=``strongMorph:TG5629'', von vielen Jahren her,
\bibverse{24} so will ich zu euch kommenx-morph=``strongMorph:TG5695'',
wenn ich reisen werdex-morph=``strongMorph:TG5741'' nach Spanien. Denn
ich hoffex-morph=``strongMorph:TG5719'', daß ich da
durchreisenx-morph=``strongMorph:TG5740'' und euch
sehenx-morph=``strongMorph:TG5664'' werde und von euch dorthin geleitet
werden mögex-morph=``strongMorph:TG5683'', so doch, daß ich zuvor mich
ein wenig an euch ergötzex-morph=``strongMorph:TG5686''. \bibverse{25}
Nun aber fahre ich hinx-morph=``strongMorph:TG5736'' gen Jerusalem den
Heiligen zu Dienstx-morph=``strongMorph:TG5723''. \bibverse{26} Denn
diex-morph=``strongMorph:TG5656'' aus Mazedonien und Achaja haben willig
eine gemeinsame Steuer zusammengelegtx-morph=``strongMorph:TG5670'' den
armen Heiligen zu Jerusalem. \bibverse{27} Sie haben's willig
getanx-morph=``strongMorph:TG5656'', und
sindx-morph=``strongMorph:TG5748'' auch ihre Schuldner. Denn so die
Heiden sind ihrer geistlichen Güter teilhaftig
gewordenx-morph=``strongMorph:TG5656'', ist's
billigx-morph=``strongMorph:TG5719'', daß sie ihnen auch in leiblichen
Gütern Dienst beweisenx-morph=``strongMorph:TG5658''. \bibverse{28} Wenn
ich nun solches ausgerichtetx-morph=``strongMorph:TG5660'' und ihnen
diese Frucht versiegelt habex-morph=``strongMorph:TG5671'', will ich
durch euch nach Spanien ziehenx-morph=``strongMorph:TG5695''.
\bibverse{29} Ich weißx-morph=``strongMorph:TG5758'' aber, wenn ich zu
euch kommex-morph=``strongMorph:TG5740'', daß ich mit vollem Segen des
Evangeliums Christi kommen werdex-morph=``strongMorph:TG5695''.
\bibverse{30} Ich ermahnex-morph=``strongMorph:TG5719'' euch aber, liebe
Brüder, durch unsern HERRN Jesus Christus und durch die Liebe des
Geistes, daß ihr helfet kämpfenx-morph=``strongMorph:TG5664'' mit Beten
für mich zu Gott, \bibverse{31} auf daß ich errettet
werdex-morph=``strongMorph:TG5686'' von den
Ungläubigenx-morph=``strongMorph:TG5723'' in Judäa, und daß mein Dienst,
den ich für Jerusalem tue, angenehm werdex-morph=``strongMorph:TG5638''
den Heiligen, \bibverse{32} auf daß ich mit Freuden zu euch
kommex-morph=``strongMorph:TG5632'' durch den Willen Gottes und mich mit
euch erquickex-morph=``strongMorph:TG5667''. \bibverse{33} Der Gott aber
des Friedens sei mit euch allen! Amen.

\hypertarget{section-15}{%
\section{16}\label{section-15}}

\bibverse{1} Ich befehlex-morph=``strongMorph:TG5719'' euch aber unsere
Schwester Phöbe, welche istx-morph=``strongMorph:TG5752'' im Dienste der
Gemeinde zu Kenchreä, \bibverse{2} daß ihr sie
aufnehmetx-morph=``strongMorph:TG5667'' in dem HERRN, wie sich's ziemt
den Heiligen, und tutx-morph=``strongMorph:TG5632'' ihr Beistand in
allem Geschäfte, darin sie euer bedarfx-morph=``strongMorph:TG5725'';
denn sie hat auch vielen Beistand getanx-morph=``strongMorph:TG5675'',
auch mir selbst. \bibverse{3} Grüßtx-morph=``strongMorph:TG5663'' die
Priscilla und den Aquila, meine Gehilfen in Christo Jesu, \bibverse{4}
welche haben für mein Leben ihren Hals
dargegebenx-morph=``strongMorph:TG5656'', welchen nicht allein ich
dankex-morph=``strongMorph:TG5719'', sondern alle Gemeinden unter den
Heiden. \bibverse{5} Auch grüßet die Gemeinde in ihrem Hause.
Grüßetx-morph=``strongMorph:TG5663'' Epänetus, meinen Lieben, welcher
istx-morph=``strongMorph:TG5748'' der Erstling unter denen aus Achaja in
Christo. \bibverse{6} Grüßetx-morph=``strongMorph:TG5663'' Maria, welche
viel Mühe und Arbeit mit uns gehabt hatx-morph=``strongMorph:TG5656''.
\bibverse{7} Grüßetx-morph=``strongMorph:TG5663'' den Andronikus und den
Junias, meine Gefreundeten und meine Mitgefangenen, welche
sindx-morph=``strongMorph:TG5748'' berühmte Apostel und vor mir
gewesenx-morph=``strongMorph:TG5754'' in Christo. \bibverse{8}
Grüßetx-morph=``strongMorph:TG5663'' Amplias, meinen Lieben in dem
HERRN. \bibverse{9} Grüßetx-morph=``strongMorph:TG5663'' Urban, unsern
Gehilfen in Christo, und Stachys, meinen Lieben. \bibverse{10}
Grüßetx-morph=``strongMorph:TG5663'' Apelles, den Bewährten in Christo.
Grüßetx-morph=``strongMorph:TG5663'', die da sind von des Aristobulus
Gesinde. \bibverse{11} Grüßetx-morph=``strongMorph:TG5663'' Herodian,
meinen Gefreundeten. Grüßetx-morph=``strongMorph:TG5663'', die da
sindx-morph=``strongMorph:TG5752'' von des Narzissus Gesinde in dem
HERRN. \bibverse{12} Grüßetx-morph=``strongMorph:TG5663'' die Tryphäna
und die Tryphosa, welche in dem HERRN gearbeitet
habenx-morph=``strongMorph:TG5723''.
Grüßetx-morph=``strongMorph:TG5663'' die Persis, meine Liebe, welch in
dem HERRN viel gearbeitet hatx-morph=``strongMorph:TG5656''.
\bibverse{13} Grüßetx-morph=``strongMorph:TG5663'' Rufus, den
Auserwählten in dem HERRN, und seine und meine Mutter. \bibverse{14}
Grüßetx-morph=``strongMorph:TG5663'' Asynkritus, Phlegon, Hermas,
Patrobas, Hermes und die Brüder bei ihnen. \bibverse{15}
Grüßetx-morph=``strongMorph:TG5663'' Philologus und die Julia, Nereus
und seine Schwester und Olympas und alle Heiligen bei ihnen.
\bibverse{16} Grüßetx-morph=``strongMorph:TG5663'' euch untereinander
mit dem heiligen Kuß. Es grüßenx-morph=``strongMorph:TG5736'' euch die
Gemeinden Christi. \bibverse{17} Ich
ermahnex-morph=``strongMorph:TG5719'' euch aber, liebe Brüder, daß ihr
achtetx-morph=``strongMorph:TG5721'' auf die, die da Zertrennung und
Ärgernis anrichtenx-morph=``strongMorph:TG5723'' neben der Lehre, die
ihr gelernt habtx-morph=``strongMorph:TG5627'', und
weichetx-morph=``strongMorph:TG5657'' von ihnen. \bibverse{18} Denn
solche dienenx-morph=``strongMorph:TG5719'' nicht dem HERRN Jesus
Christus, sondern ihrem Bauche; und durch süße Worte und prächtige Reden
verführenx-morph=``strongMorph:TG5719'' sie unschuldige Herzen.
\bibverse{19} Denn euer Gehorsam ist bei jedermann kund
gewordenx-morph=``strongMorph:TG5633''. Derhalben
freuex-morph=``strongMorph:TG5719'' ich mich über euch; ich will aber,
daß ihr weise seidx-morph=``strongMorph:TG5719''
zumx-morph=``strongMorph:TG5750'' Guten, aber einfältig zum Bösen.
\bibverse{20} Aber der Gott des Friedens
zertretex-morph=``strongMorph:TG5692'' den Satan unter eure Füße in
kurzem. Die Gnade unsers HERRN Jesu Christi sei mit euch! \bibverse{21}
Es grüßenx-morph=``strongMorph:TG5736'' euch Timotheus, mein Gehilfe,
und Luzius und Jason und Sosipater, meine Gefreundeten. \bibverse{22}
Ich, Tertius, grüßex-morph=``strongMorph:TG5736'' euch, der ich diesen
Brief geschrieben habex-morph=``strongMorph:TG5660'', in dem HERRN.
\bibverse{23} Es grüßtx-morph=``strongMorph:TG5736'' euch Gajus, mein
und der ganzen Gemeinde Wirt. Es grüßtx-morph=``strongMorph:TG5736''
euch Erastus, der Stadt Rentmeister, und Quartus, der Bruder.
\bibverse{24} Die Gnade unsers HERRN Jesus Christus sei mit euch allen!
Amen. \bibverse{25} Dem aber, der euch
stärkenx-morph=``strongMorph:TG5658'' kannx-morph=``strongMorph:TG5740''
laut meines Evangeliums und der Predigt von Jesu Christo, durch welche
das Geheimnis offenbart ist, das von der Welt her verschwiegen gewesen
istx-morph=``strongMorph:TG5772'', \bibverse{26} nun aber
offenbartx-morph=``strongMorph:TG5685'', auch
kundgemachtx-morph=``strongMorph:TG5685'' durch der Propheten Schriften
nach Befehl des ewigen Gottes, den Gehorsam des Glaubens aufzurichten
unter allen Heiden: \bibverse{27} demselben Gott, der allein weise ist,
sei Ehre durch Jesum Christum in Ewigkeit! Amen.
