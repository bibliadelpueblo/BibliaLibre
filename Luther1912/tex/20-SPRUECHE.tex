\hypertarget{section}{%
\section{1}\label{section}}

\bibverse{1} Dies sind die Sprüche Salomos, des Königs in Israel, des
Sohnes Davids, \bibverse{2} zu lernenx-morph=``strongMorph:TH8800''
Weisheit und Zucht,x-morph=``strongMorph:TH8687'' Verstand \bibverse{3}
`03947'\textbar x-morph=``strongMorph:TH8800''
Klugheitx-morph=``strongMorph:TH8687'', Gerechtigkeit, Recht und
Schlecht; \bibverse{4} daß die Unverständigen
klugx-morph=``strongMorph:TH8800'' und die Jünglinge vernünftig und
vorsichtig werden. \bibverse{5} Wer weise ist der
hörtx-morph=``strongMorph:TH8799'' zu und
bessertx-morph=``strongMorph:TH8686'' sich; wer
verständigx-morph=``strongMorph:TH8737'' ist, der
läßtx-morph=``strongMorph:TH8799'' sich raten, \bibverse{6} daß er
verstehex-morph=``strongMorph:TH8687'' die Sprüche und ihre Deutung, die
Lehre der Weisen und ihre Beispiele. \bibverse{7} Des HERRN Furcht ist
Anfang der Erkenntnis. Die Ruchlosen
verachtenx-morph=``strongMorph:TH8804'' Weisheit und Zucht. \bibverse{8}
Mein Kind, gehorchex-morph=``strongMorph:TH8798'' der Zucht deines
Vaters und verlaßx-morph=``strongMorph:TH8799'' nicht das Gebot deiner
Mutter. \bibverse{9} Denn solches ist ein schöner Schmuck deinem Haupt
und eine Kette an deinem Hals. \bibverse{10} Mein Kind, wenn dich die
bösen Buben lockenx-morph=``strongMorph:TH8762'', so
folgex-morph=``strongMorph:TH8799'' nicht. \bibverse{11} Wenn sie
sagenx-morph=``strongMorph:TH8799'': ``Gehex-morph=''strongMorph:TH8798"
mit uns! wir wollen auf Blut lauernx-morph=``strongMorph:TH8799'' und
den Unschuldigen ohne Ursache nachstellenx-morph=``strongMorph:TH8799'';
\bibverse{12} wir wollen sie lebendig
verschlingenx-morph=``strongMorph:TH8799'' wie die Hölle und die Frommen
wie die, so hinunter in die Grube fahrenx-morph=``strongMorph:TH8802'';
\bibverse{13} wir wollen großes Gut
findenx-morph=``strongMorph:TH8799''; wir wollen unsre Häuser mit Raub
füllenx-morph=``strongMorph:TH8762''; \bibverse{14} wage es mit uns! es
soll unserx-morph=``strongMorph:TH8686'' aller ein Beutel sein'':
\bibverse{15} mein Kind, wandlex-morph=``strongMorph:TH8799'' den Weg
nicht mit ihnen; wehrex-morph=``strongMorph:TH8798'' deinem Fuß vor
ihrem Pfad. \bibverse{16} Denn ihr Füße
laufenx-morph=``strongMorph:TH8799'' zum Bösen und
eilenx-morph=``strongMorph:TH8762'', Blut zu
vergießenx-morph=``strongMorph:TH8800''. \bibverse{17} Denn es ist
vergeblich, das Netz auswerfenx-morph=``strongMorph:TH8794'' vor den
Augen der Vögel. \bibverse{18} Sie aber
lauernx-morph=``strongMorph:TH8799'' auf ihr eigen Blut und stellen sich
selbstx-morph=``strongMorph:TH8799'' nach dem Leben. \bibverse{19} Also
geht es allen, die nach Gewinn geizenx-morph=``strongMorph:TH8802'', daß
ihr Geiz ihnen das Leben nimmtx-morph=``strongMorph:TH8799''.
\bibverse{20} Die Weisheit klagtx-morph=``strongMorph:TH8799'' draußen
und läßt sich hörenx-morph=``strongMorph:TH8799'' auf den Gassen;
\bibverse{21} sie ruftx-morph=``strongMorph:TH8799'' in dem Eingang des
Tores, vorn unter dem Volk; sie redetx-morph=``strongMorph:TH8799'' ihre
Worte in der Stadt: \bibverse{22} Wie lange wollt ihr Unverständigen
unverständig seinx-morph=``strongMorph:TH8799'' und die
Spötterx-morph=``strongMorph:TH8801'' Lustx-morph=``strongMorph:TH8804''
zu Spötterei und die Ruchlosen die Lehre
hassenx-morph=``strongMorph:TH8799''? \bibverse{23} Kehret
euchx-morph=``strongMorph:TH8799'' zu meiner Strafe. Siehe, ich will
euch heraussagenx-morph=``strongMorph:TH8686'' meinen Geist und euch
meine Worte kundtunx-morph=``strongMorph:TH8686''. \bibverse{24} Weil
ich denn rufex-morph=``strongMorph:TH8804'', und ihr weigert
euchx-morph=``strongMorph:TH8762'', ich recke meine Hand
ausx-morph=``strongMorph:TH8804'', und niemand achtet
daraufx-morph=``strongMorph:TH8688'', \bibverse{25} und laßt
fahrenx-morph=``strongMorph:TH8799'' allen meinen Rat und wollt meine
Strafe nichtx-morph=``strongMorph:TH8804'': \bibverse{26} so will ich
auch lachenx-morph=``strongMorph:TH8799'' in eurem Unglück und eurer
spottenx-morph=``strongMorph:TH8799'', wenn da
kommtx-morph=``strongMorph:TH8800'', was ihr fürchtet, \bibverse{27}
wenn über euch kommtx-morph=``strongMorph:TH8800'' wie ein
Sturmx-morph=``strongMorph:TH8675'', was ihr fürchtet, und euer Unglück
alsx-morph=``strongMorph:TH8799'' ein Wetter, wenn über euch Angst und
Not kommtx-morph=``strongMorph:TH8800''. \bibverse{28} Dann werden sie
nach mir rufenx-morph=``strongMorph:TH8799'', aber ich werde nicht
antwortenx-morph=``strongMorph:TH8799''; sie werden mich
suchenx-morph=``strongMorph:TH8762'', und nicht
findenx-morph=``strongMorph:TH8799''. \bibverse{29} Darum, daß sie
haßtenx-morph=``strongMorph:TH8804'' die Lehre und wollten des HERRN
Furcht nicht habenx-morph=``strongMorph:TH8804'', \bibverse{30} wollten
meinen Rat nichtx-morph=``strongMorph:TH8804'' und
lästertenx-morph=``strongMorph:TH8804'' alle meine Strafe: \bibverse{31}
so sollen sie essenx-morph=``strongMorph:TH8799'' von den Früchten ihres
Wesens und ihres Rats satt werdenx-morph=``strongMorph:TH8799''.
\bibverse{32} Was die Unverständigen gelüstet, tötet
siex-morph=``strongMorph:TH8799'', und der Ruchlosen Glück bringt sie
umx-morph=``strongMorph:TH8762''. \bibverse{33} Wer aber mir
gehorchtx-morph=``strongMorph:TH8802'', wird sicher
bleibenx-morph=``strongMorph:TH8799'' und genug
habenx-morph=``strongMorph:TH8768'' und kein Unglück fürchten.

\hypertarget{section-1}{%
\section{2}\label{section-1}}

\bibverse{1} Mein Kind, so du willst meine Rede
annehmenx-morph=``strongMorph:TH8799'' und meine Gebote bei dir
behaltenx-morph=``strongMorph:TH8799'', \bibverse{2} daß dein Ohr auf
Weisheit achthatx-morph=``strongMorph:TH8687'' und du dein Herz mit
Fleiß dazu neigestx-morph=``strongMorph:TH8686''; \bibverse{3} ja, so du
mit Fleiß darnach rufestx-morph=``strongMorph:TH8799'' und darum
betestx-morph=``strongMorph:TH8799''; \bibverse{4} so du sie
suchestx-morph=``strongMorph:TH8762'' wie Silber und nach ihr
froschestx-morph=``strongMorph:TH8799'' wie nach Schätzen: \bibverse{5}
alsdann wirst du die Furcht des HERRN
verstehenx-morph=``strongMorph:TH8799'' und Gottes Erkenntnis
findenx-morph=``strongMorph:TH8799''. \bibverse{6} Denn der HERR
gibtx-morph=``strongMorph:TH8799'' Weisheit, und aus seinem Munde kommt
Erkenntnis und Verstand. \bibverse{7} Er
läßt'sx-morph=``strongMorph:TH8799''\textbar x-morph=``strongMorph:TH8675''x-morph=``strongMorph:TH8804''
den Aufrichtigen gelingen und beschirmt die
Frommenx-morph=``strongMorph:TH8802'' \bibverse{8} und behütet die, so
recht tunx-morph=``strongMorph:TH8800'', und
bewahrtx-morph=``strongMorph:TH8799'' den Weg seiner Heiligen.
\bibverse{9} Alsdann wirst du verstehenx-morph=``strongMorph:TH8799''
Gerechtigkeit und Recht und Frömmigkeit und allen guten Weg.
\bibverse{10} Denn Weisheit wird in dein Herz
eingehenx-morph=``strongMorph:TH8799'', daß du
gernex-morph=``strongMorph:TH8799'' lernst; \bibverse{11} guter Rat wird
dich bewahrenx-morph=``strongMorph:TH8799'', und Verstand wird dich
behütenx-morph=``strongMorph:TH8799'', \bibverse{12} daß du nicht
geratestx-morph=``strongMorph:TH8687'' auf den Weg der Bösen noch unter
die verkehrten Schwätzerx-morph=``strongMorph:TH8764'', \bibverse{13}
die da verlassenx-morph=``strongMorph:TH8802'' die rechte Bahn und
gehenx-morph=``strongMorph:TH8800'' finstere Wege, \bibverse{14} die
sich freuen, Böses zu tunx-morph=``strongMorph:TH8800'', und sind
fröhlichx-morph=``strongMorph:TH8799'' in ihrem bösen, verkehrten Wesen,
\bibverse{15} welche ihren Weg verkehren und
folgenx-morph=``strongMorph:TH8737'' ihrem Abwege; \bibverse{16} daß du
nicht geratestx-morph=``strongMorph:TH8687'' an eines andern Weib, an
eine Fremdex-morph=``strongMorph:TH8801'', die glatte Worte
gibtx-morph=``strongMorph:TH8689'' \bibverse{17} und
verläßtx-morph=``strongMorph:TH8802'' den Freund ihrer Jugend und
vergißtx-morph=``strongMorph:TH8804'' den Bund ihres Gottes
\bibverse{18} (denn ihr Haus neigtx-morph=``strongMorph:TH8804'' sich
zum Tod und ihre Gänge zu den Verlorenen; \bibverse{19} alle, die zu ihr
eingehenx-morph=``strongMorph:TH8802'', kommen nicht
wiederx-morph=``strongMorph:TH8799'' und ergreifen den Weg des Lebens
nichtx-morph=``strongMorph:TH8686''); \bibverse{20} auf daß du
wandelstx-morph=``strongMorph:TH8799'' auf gutem Wege und
bleibstx-morph=``strongMorph:TH8799'' auf der rechten Bahn.
\bibverse{21} Denn die Gerechten werden im Lande
wohnenx-morph=``strongMorph:TH8799'', und die Frommen werden darin
bleibenx-morph=``strongMorph:TH8735''; \bibverse{22} aber die Gottlosen
werden aus dem Lande ausgerottetx-morph=``strongMorph:TH8735'', und die
Verächterx-morph=``strongMorph:TH8802'' werden daraus
vertilgtx-morph=``strongMorph:TH8799''.

\hypertarget{section-2}{%
\section{3}\label{section-2}}

\bibverse{1} Mein Kind, vergißx-morph=``strongMorph:TH8799'' meines
Gesetzes nicht, und dein Herz behaltex-morph=``strongMorph:TH8799''
meine Gebote. \bibverse{2} Denn sie werden dir langes Leben und gute
Jahre und Frieden bringenx-morph=``strongMorph:TH8686''; \bibverse{3}
Gnade und Treue werden dich nicht lassenx-morph=``strongMorph:TH8799''.
Hängex-morph=``strongMorph:TH8798'' sie an deinen Hals und
schreibex-morph=``strongMorph:TH8798'' sie auf die Tafel deines Herzens,
\bibverse{4} so wirstx-morph=``strongMorph:TH8798'' du Gunst und
Klugheit finden, die Gott und Menschen gefällt. \bibverse{5}
Verlaßx-morph=``strongMorph:TH8798'' dich auf den HERRN von ganzem
Herzen und verlaß dichx-morph=``strongMorph:TH8735'' nicht auf deinen
Verstand; \bibverse{6} sondern gedenkex-morph=``strongMorph:TH8798'' an
ihn in allen deinen Wegen, so wird er dich recht
führenx-morph=``strongMorph:TH8762''. \bibverse{7} Dünke dich nicht,
weise zu sein, sondern fürchtex-morph=``strongMorph:TH8798'' den HERRN
und weichex-morph=``strongMorph:TH8798'' vom Bösen. \bibverse{8} Das
wird deinem Leibe gesund sein und deine Gebeine erquicken. \bibverse{9}
Ehrex-morph=``strongMorph:TH8761'' den HERRN von deinem Gut und von den
Erstlingen all deines Einkommens, \bibverse{10} so werden deine Scheunen
voll werdenx-morph=``strongMorph:TH8735'' und deine Kelter mit Most
übergehenx-morph=``strongMorph:TH8799''. \bibverse{11} Mein Kind,
verwirfx-morph=``strongMorph:TH8799'' die Zucht des HERRN nicht und sei
nicht ungeduldigx-morph=``strongMorph:TH8799'' über seine Strafe.
\bibverse{12} Denn welchen der HERR liebtx-morph=``strongMorph:TH8686'',
den straft erx-morph=``strongMorph:TH8799'', und hat doch
Wohlgefallenx-morph=``strongMorph:TH8799'' an ihm wie ein Vater am Sohn.
\bibverse{13} Wohl dem Menschen, der Weisheit
findetx-morph=``strongMorph:TH8804'', und dem Menschen, der Verstand
bekommtx-morph=``strongMorph:TH8686''! \bibverse{14} Denn es ist besser,
sie zu erwerben, als Silber; denn ihr Ertrag ist besser als Gold.
\bibverse{15} Sie ist edler denn Perlen; und alles, was du wünschen
magst, ist ihr nicht zu vergleichenx-morph=``strongMorph:TH8799''.
\bibverse{16} Langes Leben ist zu ihrer rechten Hand; zu ihrer Linken
ist Reichtum und Ehre. \bibverse{17} Ihre Wege sind liebliche Wege, und
alle ihre Steige sind Friede. \bibverse{18} Sie ist ein Baum des Lebens
allen, die sie ergreifenx-morph=``strongMorph:TH8688''; und
seligx-morph=``strongMorph:TH8794'' sind, die sie
haltenx-morph=``strongMorph:TH8802''. \bibverse{19} Denn der HERR hat
die Erde durch Weisheit gegründetx-morph=``strongMorph:TH8804'' und
durch seinen Rat die Himmel bereitetx-morph=``strongMorph:TH8790''.
\bibverse{20} Durch seine Weisheit sind die Tiefen
zerteiltx-morph=``strongMorph:TH8738'' und die Wolken mit Tau triefend
gemachtx-morph=``strongMorph:TH8799''. \bibverse{21} Mein Kind, laß sie
nicht von deinen Augen weichenx-morph=``strongMorph:TH8799'', so
wirstx-morph=``strongMorph:TH8798'' du glückselig und klug werden.
\bibverse{22} Das wird deiner Seele Leben sein und ein Schmuck deinem
Halse. \bibverse{23} Dann wirst du sicher
wandelnx-morph=``strongMorph:TH8799'' auf deinem Wege, daß dein Fuß sich
nicht stoßen wirdx-morph=``strongMorph:TH8799''. \bibverse{24} Legst du
dichx-morph=``strongMorph:TH8799'', so wirst du dich nicht
fürchtenx-morph=``strongMorph:TH8799'',
sondernx-morph=``strongMorph:TH8804'' süßx-morph=``strongMorph:TH8804''
schlafen, \bibverse{25} daß du dich nicht
fürchtenx-morph=``strongMorph:TH8799'' darfst vor plötzlichem Schrecken
noch vor dem Sturm der Gottlosen, wenn er
kommtx-morph=``strongMorph:TH8799''. \bibverse{26} Denn der HERR ist
dein Trotz; der behütetx-morph=``strongMorph:TH8804'' deinen Fuß, daß er
nicht gefangen werde. \bibverse{27}
Weigerex-morph=``strongMorph:TH8799'' dich nicht, dem Dürftigen Gutes zu
tun, so deine Hand von Gott hat, solches zu
tunx-morph=``strongMorph:TH8800''. \bibverse{28}
Sprichx-morph=``strongMorph:TH8799'' nicht zu deinem Nächsten: ``Geh
hinx-morph=''strongMorph:TH8798" und komm
wiederx-morph=``strongMorph:TH8798''; morgen will ich dir
gebenx-morph=``strongMorph:TH8799'''', so du es wohl hast. \bibverse{29}
Trachtex-morph=``strongMorph:TH8799'' nicht Böses wider deinen Nächsten,
der auf Treue bei dir wohntx-morph=``strongMorph:TH8802''. \bibverse{30}
Haderex-morph=``strongMorph:TH8799'' nicht mit jemand ohne Ursache, so
er dir kein Leid getan hatx-morph=``strongMorph:TH8804''. \bibverse{31}
Eifere nichtx-morph=``strongMorph:TH8762'' einem Frevler nach und
erwählex-morph=``strongMorph:TH8799'' seiner Wege keinen; \bibverse{32}
denn der HERR hat Greuel an dem
Abtrünnigenx-morph=``strongMorph:TH8737'', und sein Geheimnis ist bei
den Frommen. \bibverse{33} Im Hause des Gottlosen ist der Fluch des
HERRN; aber das Haus der Gerechten wird
gesegnetx-morph=``strongMorph:TH8762''. \bibverse{34} Er wird der
Spötterx-morph=``strongMorph:TH8801''
spottenx-morph=``strongMorph:TH8686''; aber den
Elendenx-morph=``strongMorph:TH8675'' wird er Gnade
gebenx-morph=``strongMorph:TH8799''. \bibverse{35} Die Weisen werden
Ehre erbenx-morph=``strongMorph:TH8799''; aber wenn die Narren
hochkommenx-morph=``strongMorph:TH8688'', werden sie doch zu Schanden.

\hypertarget{section-3}{%
\section{4}\label{section-3}}

\bibverse{1} Höretx-morph=``strongMorph:TH8798'', meine Kinder, die
Zucht eures Vaters; merket aufx-morph=``strongMorph:TH8685'', daß ihr
lernt und klug werdetx-morph=``strongMorph:TH8800''! \bibverse{2} Denn
ich gebex-morph=``strongMorph:TH8804'' euch eine gute Lehre;
verlaßtx-morph=``strongMorph:TH8799'' mein Gesetz nicht. \bibverse{3}
Denn ich war meines Vaters Sohn, ein zarter und ein einziger vor meiner
Mutter. \bibverse{4} Und er lehrtex-morph=``strongMorph:TH8686'' mich
und sprachx-morph=``strongMorph:TH8799'': Laß dein Herz meine Worte
aufnehmenx-morph=``strongMorph:TH8799'';
haltex-morph=``strongMorph:TH8798'' meine Gebote, so wirst du
lebenx-morph=``strongMorph:TH8798''. \bibverse{5}
Nimmx-morph=``strongMorph:TH8798'' an Weisheit,
nimmx-morph=``strongMorph:TH8798'' an Verstand;
vergißx-morph=``strongMorph:TH8799'' nicht und
weichex-morph=``strongMorph:TH8799'' nicht von der Rede meines Mundes.
\bibverse{6} Verlaßx-morph=``strongMorph:TH8799'' sie nicht, so wird sie
dich bewahrenx-morph=``strongMorph:TH8799''; liebe
siex-morph=``strongMorph:TH8798'', so wird sie dich
behütenx-morph=``strongMorph:TH8799''. \bibverse{7} Denn der Weisheit
Anfang ist, wenn man sie gerne hörtx-morph=``strongMorph:TH8798'' und
die Klugheit lieber hat als alle Güter. \bibverse{8} Achte sie
hochx-morph=``strongMorph:TH8769'', so wird sie dich
erhöhenx-morph=``strongMorph:TH8787'', und wird dich zu Ehren
bringenx-morph=``strongMorph:TH8762'', wo du sie
herzestx-morph=``strongMorph:TH8762''. \bibverse{9} Sie
wirdx-morph=``strongMorph:TH8799'' dein Haupt schön schmücken und wird
dich zierenx-morph=``strongMorph:TH8762'' mit einer prächtigen Krone.
\bibverse{10} So hörex-morph=``strongMorph:TH8798'', mein Kind, und nimm
anx-morph=``strongMorph:TH8798'' meine Rede, so werden deiner Jahre viel
werdenx-morph=``strongMorph:TH8799''. \bibverse{11} Ich will dich den
Weg der Weisheit führenx-morph=``strongMorph:TH8689''; ich will dich auf
rechter Bahn leitenx-morph=``strongMorph:TH8689'', \bibverse{12} daß,
wenn du gehstx-morph=``strongMorph:TH8800'', dein Gang dir nicht sauer
werdex-morph=``strongMorph:TH8799'', und wenn du
läufstx-morph=``strongMorph:TH8799'', daß du nicht
anstoßestx-morph=``strongMorph:TH8735''. \bibverse{13}
Fassex-morph=``strongMorph:TH8685'' die Zucht, laß nicht
davonx-morph=``strongMorph:TH8686'';
bewahrex-morph=``strongMorph:TH8798'' sie, denn sie ist dein Leben.
\bibverse{14} Kommx-morph=``strongMorph:TH8799'' nicht auf der Gottlosen
Pfad und trittx-morph=``strongMorph:TH8762'' nicht auf den Weg der
Bösen. \bibverse{15} Laß ihn fahrenx-morph=``strongMorph:TH8798'' und
gehe nicht darinx-morph=``strongMorph:TH8799''; weiche von
ihmx-morph=``strongMorph:TH8798'' und gehe
vorüberx-morph=``strongMorph:TH8798''. \bibverse{16} Denn sie
schlafenx-morph=``strongMorph:TH8799'' nicht, sie haben denn Übel
getanx-morph=``strongMorph:TH8686''; und ruhen
nichtx-morph=``strongMorph:TH8738'', sie haben den
Schadenx-morph=``strongMorph:TH8686''\textbar x-morph=``strongMorph:TH8675''
getanx-morph=``strongMorph:TH8799''. \bibverse{17} Denn sie
nährenx-morph=``strongMorph:TH8804'' sich von gottlosem Brot und
trinkenx-morph=``strongMorph:TH8799'' vom Wein des Frevels.
\bibverse{18} Aber der Gerechten Pfad glänzt wie das Licht, das
immerx-morph=``strongMorph:TH8802'' heller
leuchtetx-morph=``strongMorph:TH8804'' bis auf den
vollenx-morph=``strongMorph:TH8737'' Tag. \bibverse{19} Der Gottlosen
Weg aber ist wie Dunkel; sie wissenx-morph=``strongMorph:TH8804'' nicht,
wo sie fallen werdenx-morph=``strongMorph:TH8735''. \bibverse{20} Mein
Sohn, merke aufx-morph=``strongMorph:TH8685'' meine Worte und
neigex-morph=``strongMorph:TH8685'' dein Ohr zu meiner Rede.
\bibverse{21} Laß sie nicht von deinen Augen
fahrenx-morph=``strongMorph:TH8686'',
behaltex-morph=``strongMorph:TH8798'' sie in deinem Herzen.
\bibverse{22} Denn sie sind das Leben denen, die sie
findenx-morph=``strongMorph:TH8802'', und gesund ihrem ganzen Leibe.
\bibverse{23} Behütex-morph=``strongMorph:TH8798'' dein Herz mit allem
Fleiß; denn daraus geht das Leben. \bibverse{24} Tue von
dirx-morph=``strongMorph:TH8685'' den verkehrten Mund und laß das
Lästermaul fernex-morph=``strongMorph:TH8685'' von dir sein.
\bibverse{25} Laß deine Augen stracks vor sich
sehenx-morph=``strongMorph:TH8686'' und deine Augenlider richtig vor dir
hin blickenx-morph=``strongMorph:TH8686''. \bibverse{26} Laß deinen Fuß
gleich vor sich gehenx-morph=``strongMorph:TH8761'', so gehst du
gewißx-morph=``strongMorph:TH8735''. \bibverse{27}
Wankex-morph=``strongMorph:TH8799'' weder zur Rechten noch zur Linken;
wendex-morph=``strongMorph:TH8685'' deinen Fuß vom Bösen.

\hypertarget{section-4}{%
\section{5}\label{section-4}}

\bibverse{1} Mein Kind, merke aufx-morph=``strongMorph:TH8685'' meine
Weisheit; neigex-morph=``strongMorph:TH8685'' dein Ohr zu meiner Lehre,
\bibverse{2} daß du bewahrestx-morph=``strongMorph:TH8800'' guten Rat
und dein Mund wisse Unterschied zu haltenx-morph=``strongMorph:TH8799''.
\bibverse{3} Denn die Lippen der Hurex-morph=``strongMorph:TH8801''
sindx-morph=``strongMorph:TH8799'' süß wie Honigseim, und ihre Kehle ist
glätter als Öl, \bibverse{4} aber hernach bitter wie Wermut und scharf
wie ein zweischneidiges Schwert. \bibverse{5} Ihre Füße
laufenx-morph=``strongMorph:TH8802'' zum Tod hinunter; ihre Gänge
führenx-morph=``strongMorph:TH8799'' ins Grab. \bibverse{6} Sie geht
nicht stracksx-morph=``strongMorph:TH8762'' auf dem Wege des Lebens;
unstetx-morph=``strongMorph:TH8804'' sind ihre Tritte, daß sie nicht
weiß, wo sie gehtx-morph=``strongMorph:TH8799''. \bibverse{7} So
gehorchetx-morph=``strongMorph:TH8798'' mir nun, meine Kinder, und
weichetx-morph=``strongMorph:TH8799'' nicht von der Rede meines Mundes.
\bibverse{8} Laß deine Wege fernex-morph=``strongMorph:TH8685'' von ihr
sein, und nahex-morph=``strongMorph:TH8799'' nicht zur Tür ihres Hauses,
\bibverse{9} daß du nicht den Fremden
gebestx-morph=``strongMorph:TH8799'' deine Ehre und deine Jahre dem
Grausamen; \bibverse{10} daß sich nicht
Fremdex-morph=``strongMorph:TH8801'' von deinem Vermögen
sättigenx-morph=``strongMorph:TH8799'' und deine Arbeit nicht sei in
eines andern Haus, \bibverse{11} und müssest hernach
seufzenx-morph=``strongMorph:TH8804'', wenn du Leib und Gut verzehrt
hastx-morph=``strongMorph:TH8800'', \bibverse{12} und
sprechenx-morph=``strongMorph:TH8804'': ``Ach, wie habe ich die Zucht
gehaßtx-morph=''strongMorph:TH8804" und wie hat mein Herz die Strafe
verschmähtx-morph=``strongMorph:TH8804''! \bibverse{13} wie habe ich
nicht gehorchtx-morph=``strongMorph:TH8804'' der Stimme meiner
Lehrerx-morph=``strongMorph:TH8688'' und mein Ohr nicht
geneigtx-morph=``strongMorph:TH8689'' zu denen, die mich
lehrtenx-morph=``strongMorph:TH8764''! \bibverse{14} Ich bin schier in
alles Unglück gekommen vor allen Leuten und allem Volk.'' \bibverse{15}
Trinkx-morph=``strongMorph:TH8798'' Wasser aus deiner Grube und
Flüssex-morph=``strongMorph:TH8802'' aus deinem Brunnen. \bibverse{16}
Laß deine Brunnen herausfließenx-morph=``strongMorph:TH8799'' und die
Wasserbäche auf die Gassen. \bibverse{17} Habe du aber sie allein, und
kein Fremderx-morph=``strongMorph:TH8801'' mit dir. \bibverse{18} Dein
Born sei gesegnetx-morph=``strongMorph:TH8803'', und freue
dichx-morph=``strongMorph:TH8798'' des Weibes deiner Jugend.
\bibverse{19} Sie ist lieblich wie die Hinde und holdselig wie ein Reh.
Laß dich ihre Liebe allezeit sättigenx-morph=``strongMorph:TH8762'' und
ergötzex-morph=``strongMorph:TH8799'' dich allewege in ihrer Liebe.
\bibverse{20} Mein Kind, warum willst du dich an der
Fremdenx-morph=``strongMorph:TH8801''
ergötzenx-morph=``strongMorph:TH8799'' und
herzestx-morph=``strongMorph:TH8762'' eine andere? \bibverse{21} Denn
jedermanns Wege sind offen vor dem HERRN, und er
mißtx-morph=``strongMorph:TH8764'' alle ihre Gänge. \bibverse{22} Die
Missetat des Gottlosen wird ihn fangenx-morph=``strongMorph:TH8799'',
und er wird mit dem Strick seiner Sünde gehalten
werdenx-morph=``strongMorph:TH8735''. \bibverse{23} Er wird
sterbenx-morph=``strongMorph:TH8799'', darum daß er sich nicht will
ziehen lassen; und um seiner großen Torheit willen wird's ihm nicht wohl
gehenx-morph=``strongMorph:TH8799''.

\hypertarget{section-5}{%
\section{6}\label{section-5}}

\bibverse{1} Mein Kind, wirst du Bürgex-morph=``strongMorph:TH8804'' für
deinen Nächsten und hast deine Hand bei einem
Fremdenx-morph=``strongMorph:TH8801''
verhaftetx-morph=``strongMorph:TH8804'', \bibverse{2} so bist du
verknüpftx-morph=``strongMorph:TH8738'' durch die Rede deines Mundes und
gefangenx-morph=``strongMorph:TH8738'' mit den Reden deines Mundes.
\bibverse{3} So tuex-morph=``strongMorph:TH8798'' doch, mein Kind, also
und errettex-morph=``strongMorph:TH8734'' dich, denn du bist deinem
Nächsten in die Hände gekommenx-morph=``strongMorph:TH8804'':
eilex-morph=``strongMorph:TH8798'', drängex-morph=``strongMorph:TH8690''
und treibex-morph=``strongMorph:TH8798'' deinen Nächsten. \bibverse{4}
Laßx-morph=``strongMorph:TH8799'' deine Augen nicht schlafen, noch
deinen Augenlider schlummern. \bibverse{5}
Errettex-morph=``strongMorph:TH8734'' dich wie ein Reh von der Hand und
wie eine Vogel aus der Hand des Voglers. \bibverse{6}
Gehex-morph=``strongMorph:TH8798'' hin zur Ameise, du Fauler;
siehex-morph=``strongMorph:TH8798'' ihre Weise an und
lernex-morph=``strongMorph:TH8798''! \bibverse{7} Ob sie wohl keinen
Fürsten noch Hauptmannx-morph=``strongMorph:TH8802'' noch
Herrnx-morph=``strongMorph:TH8802'' hat, \bibverse{8}
bereitetx-morph=``strongMorph:TH8686'' sie doch ihr Brot im Sommer und
sammeltx-morph=``strongMorph:TH8804'' ihre Speise in der Ernte.
\bibverse{9} Wie lange liegstx-morph=``strongMorph:TH8799'' du, Fauler?
Wann willst du aufstehenx-morph=``strongMorph:TH8799'' von deinem
Schlaf? \bibverse{10} Ja, schlafe noch ein wenig, schlummere ein wenig,
schlage die Hände ineinander ein wenig, daß du
schlafestx-morph=``strongMorph:TH8800'', \bibverse{11} so wird dich die
Armut übereilenx-morph=``strongMorph:TH8804'' wie ein
Fußgängerx-morph=``strongMorph:TH8764'' und der Mangel wie ein
gewappneter Mann. \bibverse{12} Ein heilloser Mensch, ein schädlicher
Mann gehtx-morph=``strongMorph:TH8802'' mit verstelltem Munde,
\bibverse{13} winktx-morph=``strongMorph:TH8802'' mit Augen,
deutetx-morph=``strongMorph:TH8802'' mit Füßen,
zeigtx-morph=``strongMorph:TH8688'' mit Fingern, \bibverse{14}
trachtetx-morph=``strongMorph:TH8802'' allezeit Böses und Verkehrtes in
seinem Herzen und richtetx-morph=``strongMorph:TH8762''
Haderx-morph=``strongMorph:TH8675'' an. \bibverse{15} Darum wird ihm
plötzlich sein Verderben kommenx-morph=``strongMorph:TH8799'', und er
wird schnell zerbrochen werdenx-morph=``strongMorph:TH8735'', da keine
Hilfe dasein wird. \bibverse{16} Diese sechs Stücke
haßtx-morph=``strongMorph:TH8804'' der HERR, und am siebenten hat er
einen Greuel: \bibverse{17} hohex-morph=``strongMorph:TH8802'' Augen,
falsche Zunge, Hände, die unschuldig Blut
vergießenx-morph=``strongMorph:TH8802'', \bibverse{18} Herz, das mit
böser Tücke umgehtx-morph=``strongMorph:TH8802'', Füße, die behend
sindx-morph=``strongMorph:TH8764'', Schaden zu
tunx-morph=``strongMorph:TH8800'', \bibverse{19} falscher Zeuge, der
frech Lügen redetx-morph=``strongMorph:TH8686'' und wer Hader zwischen
Brüdern anrichtetx-morph=``strongMorph:TH8764''. \bibverse{20} Mein
Kind, bewahrex-morph=``strongMorph:TH8798'' die Gebote deines Vaters und
laß nicht fahrenx-morph=``strongMorph:TH8799'' das Gesetz deiner Mutter.
\bibverse{21} Bindex-morph=``strongMorph:TH8798'' sie zusammen auf dein
Herz allewege und hängex-morph=``strongMorph:TH8798'' sie an deinen
Hals, \bibverse{22} wenn du gehstx-morph=``strongMorph:TH8692'', daß sie
dich geleitenx-morph=``strongMorph:TH8686''; wenn du dich
legstx-morph=``strongMorph:TH8800'', daß sie dich
bewahrenx-morph=``strongMorph:TH8799''; wenn du
aufwachstx-morph=``strongMorph:TH8689'', daß sie zu dir
sprechenx-morph=``strongMorph:TH8799''. \bibverse{23} Denn das Gebot ist
eine Leuchte und das Gesetz ein Licht, und die Strafe der Zucht ist ein
Weg des Lebens, \bibverse{24} auf daß du
bewahrtx-morph=``strongMorph:TH8800'' werdest vor dem bösen Weibe, vor
der glatten Zunge der Fremden. \bibverse{25} Laß dich ihre Schöne nicht
gelüstenx-morph=``strongMorph:TH8799'' in deinem Herzen und verfange
dich nichtx-morph=``strongMorph:TH8799'' an ihren Augenlidern.
\bibverse{26} Denn eine Hurex-morph=``strongMorph:TH8802'' bringt einen
ums Brot; aber eines andern Weib fängtx-morph=``strongMorph:TH8799'' das
edle Leben. \bibverse{27} Kann auch jemand ein Feuer im Busen
behaltenx-morph=``strongMorph:TH8799'', daß seine Kleider nicht
brennenx-morph=``strongMorph:TH8735''? \bibverse{28} Wie sollte jemand
auf Kohlen gehenx-morph=``strongMorph:TH8762'', daß seine Füße nicht
verbrannt würdenx-morph=``strongMorph:TH8735''? \bibverse{29} Also
gehet's dem, der zu seines Nächsten Weib
gehtx-morph=``strongMorph:TH8802''; es bleibt keiner
ungestraftx-morph=``strongMorph:TH8735'', der sie
berührtx-morph=``strongMorph:TH8802''. \bibverse{30} Es ist einem Diebe
nicht so große Schmachx-morph=``strongMorph:TH8799'', ob er
stiehltx-morph=``strongMorph:TH8799'', seine Seele zu
sättigenx-morph=``strongMorph:TH8763'', weil ihn
hungertx-morph=``strongMorph:TH8799''; \bibverse{31} und ob er ergriffen
wirdx-morph=``strongMorph:TH8738'', gibt er's siebenfältig
wiederx-morph=``strongMorph:TH8762'' und
legtx-morph=``strongMorph:TH8799'' dar alles Gut in seinem Hause.
\bibverse{32} Aber wer mit einem Weibe die Ehe
brichtx-morph=``strongMorph:TH8802'', der ist ein Narr;
derx-morph=``strongMorph:TH8799'' bringt sein Leben ins
Verderbenx-morph=``strongMorph:TH8688''. \bibverse{33} Dazu
trifftx-morph=``strongMorph:TH8799'' ihn Plage und Schande, und seine
Schande wird nicht ausgetilgtx-morph=``strongMorph:TH8735''.
\bibverse{34} Denn der Grimm des Mannes eifert, und schont
nichtx-morph=``strongMorph:TH8799'' zur Zeit der Rache \bibverse{35} und
sieht keine Person anx-morph=``strongMorph:TH8799'', die da versöhne,
und nimmt's nicht anx-morph=``strongMorph:TH8799'', ob du
vielx-morph=``strongMorph:TH8686'' schenken wolltest.

\hypertarget{section-6}{%
\section{7}\label{section-6}}

\bibverse{1} Mein Kind, behaltex-morph=``strongMorph:TH8798'' meine Rede
und verbirgx-morph=``strongMorph:TH8799'' meine Gebote bei dir.
\bibverse{2} Behaltex-morph=``strongMorph:TH8798'' meine Gebote, so
wirst du lebenx-morph=``strongMorph:TH8798'', und mein Gesetz wie deinen
Augapfel. \bibverse{3} Bindex-morph=``strongMorph:TH8798'' sie an deine
Finger; schreibex-morph=``strongMorph:TH8798'' sie auf die Tafel deines
Herzens. \bibverse{4} Sprichx-morph=``strongMorph:TH8798'' zur Weisheit:
``Du bist meine Schwester'', und nennex-morph=``strongMorph:TH8799'' die
Klugheit deine Freundin, \bibverse{5} daß du behütet
werdestx-morph=``strongMorph:TH8800'' vor dem
fremdenx-morph=``strongMorph:TH8801'' Weibe, vor einer andern, die
glatte Worte gibtx-morph=``strongMorph:TH8689''. \bibverse{6} Denn am
Fenster meines Hauses gucktex-morph=``strongMorph:TH8738'' ich durchs
Gitter \bibverse{7} und sahx-morph=``strongMorph:TH8799'' unter den
Unverständigen und ward gewahrx-morph=``strongMorph:TH8799'' unter den
Kindern eines törichten Jünglings, \bibverse{8} der
gingx-morph=``strongMorph:TH8802'' auf der Gasse an einer Ecke und
tratx-morph=``strongMorph:TH8799'' daher auf dem Wege bei ihrem Hause,
\bibverse{9} in der Dämmerung, am Abend des Tages, da es Nacht ward und
dunkel war. \bibverse{10} Und siehe, da
begegnetex-morph=``strongMorph:TH8800'' ihm ein Weib im
Hurenschmuckx-morph=``strongMorph:TH8802'',
listigx-morph=``strongMorph:TH8803'', \bibverse{11}
wildx-morph=``strongMorph:TH8802'' und
unbändigx-morph=``strongMorph:TH8802'', daß ihr Füße in ihrem Hause
nicht bleibenx-morph=``strongMorph:TH8799'' können. \bibverse{12} Jetzt
ist sie draußen, jetzt auf der Gasse, und
lauertx-morph=``strongMorph:TH8799'' an allen Ecken. \bibverse{13} Und
erwischte ihnx-morph=``strongMorph:TH8689'' und
küßtex-morph=``strongMorph:TH8804'' ihn
unverschämtx-morph=``strongMorph:TH8689'' und
sprachx-morph=``strongMorph:TH8799'' zu ihm: \bibverse{14} Ich habe
Dankopfer für mich heute bezahltx-morph=``strongMorph:TH8765'' für meine
Gelübde. \bibverse{15} Darum bin
herausgegangenx-morph=``strongMorph:TH8804'', dir zu
begegnenx-morph=``strongMorph:TH8800'', dein Angesicht zu
suchenx-morph=``strongMorph:TH8763'', und habe dich
gefundenx-morph=``strongMorph:TH8799''. \bibverse{16} Ich habe mein Bett
schön geschmücktx-morph=``strongMorph:TH8804'' mit bunten Teppichen aus
Ägypten. \bibverse{17} Ich habe mein Lager mit Myrrhe, Aloe und Zimt
besprengtx-morph=``strongMorph:TH8804''. \bibverse{18}
Kommx-morph=``strongMorph:TH8798'', laßx-morph=``strongMorph:TH8799''
und buhlen bis an den Morgen und laß und der Liebe
pflegenx-morph=``strongMorph:TH8691''. \bibverse{19} Denn der Mann ist
nicht daheim; er ist einen fernen Weg
gezogenx-morph=``strongMorph:TH8804''. \bibverse{20} Er hat den Geldsack
mit sich genommenx-morph=``strongMorph:TH8804''; er wird erst aufs Fest
wieder heimkommenx-morph=``strongMorph:TH8799''. \bibverse{21} Sie
überredetex-morph=``strongMorph:TH8689'' ihn mit vielen Worten und
gewann ihnx-morph=``strongMorph:TH8686'' mit ihrem glatten Munde.
\bibverse{22} Er folgtx-morph=``strongMorph:TH8802'' ihr alsbald nach,
wie ein Ochse zur Fleischbank geführt
wirdx-morph=``strongMorph:TH8799'', und wie zur Fessel, womit man die
Narren züchtigt, \bibverse{23} bis sie ihm mit dem Pfeil die Leber
spaltetx-morph=``strongMorph:TH8762''; wie ein Vogel zum Strick
eiltx-morph=``strongMorph:TH8763'' und weiß
nichtx-morph=``strongMorph:TH8804'', daß es ihm sein Leben gilt.
\bibverse{24} So gehorchetx-morph=``strongMorph:TH8798'' mir nun, meine
Kinder, und merket aufx-morph=``strongMorph:TH8685'' die Rede meines
Mundes. \bibverse{25} Laß dein Herz nicht
weichenx-morph=``strongMorph:TH8799'' auf ihren Weg und laß dich nicht
verführenx-morph=``strongMorph:TH8799'' auf ihrer Bahn. \bibverse{26}
Denn sie hat viele verwundet und gefälltx-morph=``strongMorph:TH8689'',
und sind allerlei Mächtige von ihr
erwürgtx-morph=``strongMorph:TH8803''. \bibverse{27} Ihr Haus sind Wege
zum Grab, da man hinunterfährtx-morph=``strongMorph:TH8802'' in des
Todes Kammern.

\hypertarget{section-7}{%
\section{8}\label{section-7}}

\bibverse{1} Ruftx-morph=``strongMorph:TH8799'' nicht die Weisheit, und
die Klugheit läßt sich hörenx-morph=``strongMorph:TH8799''? \bibverse{2}
Öffentlich am Wege und an der Straße steht
siex-morph=``strongMorph:TH8738''. \bibverse{3} An den Toren bei der
Stadt, da man zur Tür eingeht, schreit
siex-morph=``strongMorph:TH8799'': \bibverse{4} O ihr Männer, ich
schreiex-morph=``strongMorph:TH8799'' zu euch und rufe den Leuten.
\bibverse{5} Merktx-morph=``strongMorph:TH8685'', ihr Unverständigen,
auf Klugheit und, ihr Toren, nehmtx-morph=``strongMorph:TH8685'' es zu
Herzen! \bibverse{6} Höretx-morph=``strongMorph:TH8798'', denn ich will
redenx-morph=``strongMorph:TH8762'', was fürstlich ist, und lehren, was
recht ist. \bibverse{7} Denn mein Mund soll die Wahrheit
redenx-morph=``strongMorph:TH8799'', und meine Lippen sollen hassen, was
gottlos ist. \bibverse{8} Alle Reden meines Mundes sind gerecht; es ist
nichts Verkehrtesx-morph=``strongMorph:TH8737'' noch falsches darin.
\bibverse{9} Sie sind alle gerade denen, die sie
verstehenx-morph=``strongMorph:TH8688'', und richtig denen, die es
annehmen wollenx-morph=``strongMorph:TH8802''. \bibverse{10}
Nehmetx-morph=``strongMorph:TH8798'' an meine Zucht lieber denn Silber,
und die Lehre achtet höher dennx-morph=``strongMorph:TH8737'' köstliches
Gold. \bibverse{11} Denn Weisheit ist besser als Perlen; und alles, was
man wünschen mag, kann ihr nicht gleichenx-morph=``strongMorph:TH8799''.
\bibverse{12} Ich, Weisheit, wohnex-morph=``strongMorph:TH8804'' bei der
Klugheit und weiß guten Rat zu gebenx-morph=``strongMorph:TH8799''.
\bibverse{13} Die Furcht des HERRN haßtx-morph=``strongMorph:TH8800''
das Arge, die Hoffart, den Hochmut und bösen Weg; und ich bin
feindx-morph=``strongMorph:TH8804'' dem verkehrten Mund. \bibverse{14}
Mein ist beides, Rat und Tat; ich habe Verstand und Macht. \bibverse{15}
Durch mich regierenx-morph=``strongMorph:TH8799'' die Könige und
setzenx-morph=``strongMorph:TH8779'' die
Ratsherrenx-morph=``strongMorph:TH8802'' das Recht. \bibverse{16} Durch
mich herrschenx-morph=``strongMorph:TH8799'' die Fürsten und alle
Regentenx-morph=``strongMorph:TH8802'' auf Erden. \bibverse{17} Ich
liebex-morph=``strongMorph:TH8799'', die mich
liebenx-morph=``strongMorph:TH8802''; und die mich frühe
suchenx-morph=``strongMorph:TH8764'', finden
michx-morph=``strongMorph:TH8799''. \bibverse{18} Reichtum und Ehre ist
bei mir, währendes Gut und Gerechtigkeit. \bibverse{19} Meine Frucht ist
besser denn Gold und feines Gold und mein Ertrag besser denn
auserlesenesx-morph=``strongMorph:TH8737'' Silber. \bibverse{20} Ich
wandlex-morph=``strongMorph:TH8762'' auf dem rechten Wege, auf der
Straße des Rechts, \bibverse{21} daß ich wohl
versorgex-morph=``strongMorph:TH8687'', die mich
liebenx-morph=``strongMorph:TH8802'', und ihre Schätze
vollmachex-morph=``strongMorph:TH8762''. \bibverse{22} Der HERR hat mich
gehabtx-morph=``strongMorph:TH8804'' im Anfang seiner Wege; ehe er etwas
schuf, war ich da. \bibverse{23} Ich bin
eingesetztx-morph=``strongMorph:TH8738'' von Ewigkeit, von Anfang, vor
der Erde. \bibverse{24} Da die Tiefen noch nicht waren, da war ich schon
geborenx-morph=``strongMorph:TH8797'', da die Brunnen noch nicht mit
Wasser quollenx-morph=``strongMorph:TH8737''. \bibverse{25} Ehe denn die
Berge eingesenkt warenx-morph=``strongMorph:TH8717'', vor den Hügeln war
ich geborenx-morph=``strongMorph:TH8797'', \bibverse{26} da er die Erde
noch nicht gemacht hattex-morph=``strongMorph:TH8804'' und was darauf
ist, noch die Berge des Erdbodens. \bibverse{27} Da er die Himmel
bereitetex-morph=``strongMorph:TH8687'', war ich daselbst, da er die
Tiefe mit seinem Ziel faßtex-morph=``strongMorph:TH8800''. \bibverse{28}
Da er die Wolken droben festetex-morph=``strongMorph:TH8763'', da er
festigtex-morph=``strongMorph:TH8800'' die Brunnen der Tiefe,
\bibverse{29} da er dem Meer das Ziel
setztex-morph=``strongMorph:TH8800'' und den Wassern, daß sie nicht
überschreitenx-morph=``strongMorph:TH8799'' seinen Befehl, da er den
Grund der Erde legtex-morph=``strongMorph:TH8800'': \bibverse{30} da war
ich der Werkmeister bei ihm und hatte meine Lust täglich und
spieltex-morph=``strongMorph:TH8764'' vor ihm allezeit \bibverse{31} und
spieltex-morph=``strongMorph:TH8764'' auf seinem Erdboden, und meine
Lust ist bei den Menschenkindern. \bibverse{32} So
gehorchetx-morph=``strongMorph:TH8798'' mir nun, meine Kinder. Wohl
denen, die meine Wege haltenx-morph=``strongMorph:TH8799''!
\bibverse{33} Höretx-morph=``strongMorph:TH8798'' die Zucht und werdet
weisex-morph=``strongMorph:TH8798'' und lasset sie nicht
fahrenx-morph=``strongMorph:TH8799''. \bibverse{34} Wohl dem Menschen,
der mir gehorchtx-morph=``strongMorph:TH8802'', daß er
wachex-morph=``strongMorph:TH8800'' an meiner Tür täglich, daß er
wartex-morph=``strongMorph:TH8800'' an den Pfosten meiner Tür.
\bibverse{35} Wer mich findetx-morph=``strongMorph:TH8802'', der
findetx-morph=``strongMorph:TH8804''\textbar x-morph=``strongMorph:TH8675''x-morph=``strongMorph:TH8802''
das Leben und wird Wohlgefallen vom HERRN
erlangenx-morph=``strongMorph:TH8686''. \bibverse{36} Wer aber an mir
sündigtx-morph=``strongMorph:TH8802'', der
verletztx-morph=``strongMorph:TH8802'' seine Seele. Alle, die mich
hassenx-morph=``strongMorph:TH8764'',
liebenx-morph=``strongMorph:TH8804'' den Tod.

\hypertarget{section-8}{%
\section{9}\label{section-8}}

\bibverse{1} Die Weisheit bautex-morph=``strongMorph:TH8804'' ihr Haus
und hiebx-morph=``strongMorph:TH8804'' sieben Säulen, \bibverse{2}
schlachtetex-morph=``strongMorph:TH8804'' ihr Vieh und trug ihren Wein
aufx-morph=``strongMorph:TH8804'' und
bereitetex-morph=``strongMorph:TH8804'' ihren Tisch \bibverse{3} und
sandte ihre Dirnen ausx-morph=``strongMorph:TH8804'', zu
rufenx-morph=``strongMorph:TH8799'' oben auf den Höhen der Stadt:
\bibverse{4} ``Wer verständig ist, der mache
sichx-morph=''strongMorph:TH8799" hierher!'', und zum Narren sprach
siex-morph=``strongMorph:TH8804'': \bibverse{5}
``Kommetx-morph=''strongMorph:TH8798``,
zehretx-morph=''strongMorph:TH8798" von meinem Brot und
trinketx-morph=``strongMorph:TH8798'' den Wein, den ich
schenkex-morph=``strongMorph:TH8804''; \bibverse{6}
verlaßtx-morph=``strongMorph:TH8798'' das unverständige Wesen, so werdet
ihr lebenx-morph=``strongMorph:TH8798'', und gehet
aufx-morph=``strongMorph:TH8798'' dem Wege der Klugheit.'' \bibverse{7}
Wer den Spötterx-morph=``strongMorph:TH8801''
züchtigtx-morph=``strongMorph:TH8802'', der muß Schande auf sich
nehmenx-morph=``strongMorph:TH8802''; und wer den Gottlosen
straftx-morph=``strongMorph:TH8688'', der muß gehöhnt werden.
\bibverse{8} Strafex-morph=``strongMorph:TH8686'' den
Spötterx-morph=``strongMorph:TH8801'' nicht, er haßt
dichx-morph=``strongMorph:TH8799''; strafex-morph=``strongMorph:TH8685''
den Weisen, der wird dich liebenx-morph=``strongMorph:TH8799''.
\bibverse{9} Gibx-morph=``strongMorph:TH8798'' dem Weisen, so wird er
noch weiser werdenx-morph=``strongMorph:TH8799'';
lehrex-morph=``strongMorph:TH8685'' den Gerechten, so wird er in der
Lehre zunehmenx-morph=``strongMorph:TH8686''. \bibverse{10} Der Weisheit
Anfang ist des HERRN Furcht, und den Heiligen erkennen ist Verstand.
\bibverse{11} Denn durch mich werden deiner Tage viel
werdenx-morph=``strongMorph:TH8799'' und werden dir der Jahre des Lebens
mehr werdenx-morph=``strongMorph:TH8686''. \bibverse{12} Bist du
weisex-morph=``strongMorph:TH8804'', so bist du dir
weisex-morph=``strongMorph:TH8804''; bist du ein
Spötterx-morph=``strongMorph:TH8804'', so wirst du es allein
tragenx-morph=``strongMorph:TH8799''. \bibverse{13} Es ist aber ein
törichtes, wildesx-morph=``strongMorph:TH8802'' Weib, voll Schwätzens,
und weißx-morph=``strongMorph:TH8804'' nichts; \bibverse{14} die
sitztx-morph=``strongMorph:TH8804'' in der Tür ihres Hauses auf dem
Stuhl, oben in der Stadt, \bibverse{15} zu
ladenx-morph=``strongMorph:TH8800'' alle, die
vorübergehenx-morph=``strongMorph:TH8802'' und richtig auf ihrem Wege
wandelnx-morph=``strongMorph:TH8764'': \bibverse{16} ``Wer unverständig
ist, der mache sichx-morph=''strongMorph:TH8799" hierher!'', und zum
Narren spricht siex-morph=``strongMorph:TH8804'': \bibverse{17} ``Die
gestohlenenx-morph=''strongMorph:TH8803" Wasser sind
süßx-morph=``strongMorph:TH8799'', und das verborgene Brot schmeckt
wohlx-morph=``strongMorph:TH8799''.'' \bibverse{18} Er
weißx-morph=``strongMorph:TH8804'' aber nicht, daß daselbst Tote sind
und ihre Gästex-morph=``strongMorph:TH8803'' in der tiefen Grube.

\hypertarget{section-9}{%
\section{10}\label{section-9}}

\bibverse{1} Dies sind die Sprüche Salomos. Ein weiser Sohn ist seines
Vaters Freudex-morph=``strongMorph:TH8762''; aber ein törichter Sohn ist
seiner Mutter Grämen. \bibverse{2} Unrecht Gut hilft
nichtx-morph=``strongMorph:TH8686''; aber Gerechtigkeit
errettetx-morph=``strongMorph:TH8686'' vor dem Tode. \bibverse{3} Der
HERR läßt die Seele des Gerechten nicht Hunger
leidenx-morph=``strongMorph:TH8686''; er stößt aber
wegx-morph=``strongMorph:TH8799'' der Gottlosen Begierde. \bibverse{4}
Lässige Hand machtx-morph=``strongMorph:TH8802''
armx-morph=``strongMorph:TH8802''; aber der Fleißigen Hand macht
reichx-morph=``strongMorph:TH8686''. \bibverse{5} Wer im Sommer
sammeltx-morph=``strongMorph:TH8802'', der ist
klugx-morph=``strongMorph:TH8688''; wer aber in der Ernte
schläftx-morph=``strongMorph:TH8737'', wird zu
Schandenx-morph=``strongMorph:TH8688''. \bibverse{6} Den Segen hat das
Haupt des Gerechten; aber den Mund der Gottlosen wird ihr Frevel
überfallenx-morph=``strongMorph:TH8762''. \bibverse{7} Das Gedächtnis
der Gerechten bleibt im Segen; aber der Gottlosen Name wird
verwesenx-morph=``strongMorph:TH8799''. \bibverse{8} Wer weise von
Herzen ist nimmtx-morph=``strongMorph:TH8799'' die Gebote an; wer aber
ein Narrenmaul hat, wird geschlagenx-morph=``strongMorph:TH8735''.
\bibverse{9} Wer unschuldig lebtx-morph=``strongMorph:TH8799'', der
lebtx-morph=``strongMorph:TH8802'' sicher; wer aber
verkehrtx-morph=``strongMorph:TH8764'' ist auf seinen Wegen, wird
offenbar werdenx-morph=``strongMorph:TH8735''. \bibverse{10} Wer mit
Augen winktx-morph=``strongMorph:TH8802'', wird Mühsal
anrichtenx-morph=``strongMorph:TH8799''; und der ein Narrenmaul hat,
wird geschlagenx-morph=``strongMorph:TH8735''. \bibverse{11} Des
Gerechten Mund ist ein Brunnen des Lebens; aber den Mund der Gottlosen
wird ihr Frevel überfallenx-morph=``strongMorph:TH8762''. \bibverse{12}
Haß erregtx-morph=``strongMorph:TH8787'' Hader; aber Liebe
decktx-morph=``strongMorph:TH8762'' zu alle Übertretungen. \bibverse{13}
In den Lippen des Verständigenx-morph=``strongMorph:TH8737''
findetx-morph=``strongMorph:TH8735'' man Weisheit; aber auf den Rücken
der Narren gehört eine Rute. \bibverse{14} Die Weisen
bewahrenx-morph=``strongMorph:TH8799'' die Lehre; aber der Narren Mund
ist nahe dem Schrecken. \bibverse{15} Das Gut des Reichen ist seine
feste Stadt; aber die Armen macht die Armut blöde. \bibverse{16} Der
Gerechte braucht sein Gut zum Leben; aber der Gottlose braucht sein
Einkommen zur Sünde. \bibverse{17} Die Zucht
haltenx-morph=``strongMorph:TH8802'' ist der Weg zum Leben; wer aber der
Zurechtweisung nicht achtetx-morph=``strongMorph:TH8802'', der bleibt in
der Irrex-morph=``strongMorph:TH8688''. \bibverse{18} Falsche Mäuler
bergenx-morph=``strongMorph:TH8764'' Haß; und wer
verleumdetx-morph=``strongMorph:TH8688'', der ist ein Narr.
\bibverse{19} Wo viel Worte sind, da geht's
ohnex-morph=``strongMorph:TH8799'' Sünde nicht ab; wer aber seine Lippen
hältx-morph=``strongMorph:TH8802'', ist
klugx-morph=``strongMorph:TH8688''. \bibverse{20} Des Gerechten Zunge
ist köstlichesx-morph=``strongMorph:TH8737'' Silber; aber der Gottlosen
Herz ist wie nichts. \bibverse{21} Des Gerechten Lippen
weidenx-morph=``strongMorph:TH8799'' viele; aber die Narren werden an
ihrer Torheit sterbenx-morph=``strongMorph:TH8799''. \bibverse{22} Der
Segen des HERRN macht reichx-morph=``strongMorph:TH8686''
ohnex-morph=``strongMorph:TH8686'' Mühe. \bibverse{23} Ein Narr
treibtx-morph=``strongMorph:TH8800'' Mutwillen und hat dazu noch seinen
Spott; aber der Mann ist weise, der aufmerkt. \bibverse{24} Was der
Gottlose fürchtet, das wird ihm begegnenx-morph=``strongMorph:TH8799'';
und was die Gerechten begehren, wird ihnen
gegebenx-morph=``strongMorph:TH8799''. \bibverse{25} Der Gottlose ist
wie ein Wetter, das vorübergehtx-morph=``strongMorph:TH8800'' und nicht
mehr ist; der Gerechte aber besteht ewiglich. \bibverse{26} Wie der
Essig den Zähnen und der Rauch den Augen tut, so tut der Faule denen,
die ihn sendenx-morph=``strongMorph:TH8802''. \bibverse{27} Die Furcht
des HERRN mehrtx-morph=``strongMorph:TH8686'' die Tage; aber die Jahre
der Gottlosen werden verkürztx-morph=``strongMorph:TH8799''.
\bibverse{28} Das Warten der Gerechten wird Freude werden; aber der
Gottlosen Hoffnung wird verloren seinx-morph=``strongMorph:TH8799''.
\bibverse{29} Der Weg des HERRN ist des Frommen Trotz; aber die
Übeltäterx-morph=``strongMorph:TH8802'' sind blöde. \bibverse{30} Der
Gerechte wird nimmermehr umgestoßenx-morph=``strongMorph:TH8735''; aber
die Gottlosen werden nicht im Lande
bleibenx-morph=``strongMorph:TH8799''. \bibverse{31} Der Mund des
Gerechten bringtx-morph=``strongMorph:TH8799'' Weisheit; aber die Zunge
der Verkehrten wird ausgerottetx-morph=``strongMorph:TH8735''.
\bibverse{32} Die Lippen der Gerechten
lehrenx-morph=``strongMorph:TH8799'' heilsame Dinge; aber der Gottlosen
Mund ist verkehrt.

\hypertarget{section-10}{%
\section{11}\label{section-10}}

\bibverse{1} Falsche Waage ist dem HERRN ein Greuel; aber völliges
Gewicht ist sein Wohlgefallen. \bibverse{2} Wo Stolz ist,
dax-morph=``strongMorph:TH8804'' ist auchx-morph=``strongMorph:TH8799''
Schmach; aber Weisheit ist bei den
Demütigenx-morph=``strongMorph:TH8803''. \bibverse{3} Unschuld wird die
Frommen leitenx-morph=``strongMorph:TH8686''; aber die Bosheit wird die
Verächterx-morph=``strongMorph:TH8802''
verstörenx-morph=``strongMorph:TH8799''\textbar x-morph=``strongMorph:TH8675''x-morph=``strongMorph:TH8804''.
\bibverse{4} Gut hilft nichtx-morph=``strongMorph:TH8686'' am Tage des
Zorns; aber Gerechtigkeit errettetx-morph=``strongMorph:TH8686'' vom
Tod. \bibverse{5} Die Gerechtigkeit des Frommen macht seinen Weg
ebenx-morph=``strongMorph:TH8762''; aber der Gottlose wird
fallenx-morph=``strongMorph:TH8799'' durch sein gottloses Wesen.
\bibverse{6} Die Gerechtigkeit der Frommen wird sie
errettenx-morph=``strongMorph:TH8686''; aber die
Verächterx-morph=``strongMorph:TH8802'' werden
gefangenx-morph=``strongMorph:TH8735'' in ihrer Bosheit. \bibverse{7}
Wenn der gottlose Mensch stirbt, ist seine Hoffnung
verlorenx-morph=``strongMorph:TH8799'' und das Harren des Ungerechten
wird zunichtex-morph=``strongMorph:TH8804''. \bibverse{8} Der Gerechte
wird aus seiner Not erlöstx-morph=``strongMorph:TH8738'', und der
Gottlose kommtx-morph=``strongMorph:TH8799'' an seine Statt.
\bibverse{9} Durch den Mund des Heuchlers wird sein Nächster
verderbtx-morph=``strongMorph:TH8686''; aber die Gerechten merken's und
werden erlöstx-morph=``strongMorph:TH8735''. \bibverse{10} Eine Stadt
freut sichx-morph=``strongMorph:TH8799'', wenn's den Gerechten wohl
geht; und wenn die Gottlosen umkommenx-morph=``strongMorph:TH8800'',
wird man froh. \bibverse{11} Durch den Segen der Frommen wird eine Stadt
erhobenx-morph=``strongMorph:TH8799''; aber durch den Mund der Gottlosen
wird sie zerbrochenx-morph=``strongMorph:TH8735''. \bibverse{12} Wer
seinen Nächsten schändetx-morph=``strongMorph:TH8802'', ist ein Narr;
aber ein verständiger Mann schweigt stillx-morph=``strongMorph:TH8686''.
\bibverse{13} Ein Verleumder verrätx-morph=``strongMorph:TH8802'', was
er heimlich weißx-morph=``strongMorph:TH8764''; aber wer eines
getreuenx-morph=``strongMorph:TH8738'' Herzens ist,
verbirgtx-morph=``strongMorph:TH8764'' es. \bibverse{14} Wo nicht Rat
ist, da geht das Volk unterx-morph=``strongMorph:TH8799''; wo aber viel
Ratgeberx-morph=``strongMorph:TH8802'' sind, da geht es wohl zu.
\bibverse{15} Wer für einen andernx-morph=``strongMorph:TH8801''
Bürgex-morph=``strongMorph:TH8804'' wird, der wird Schaden
habenx-morph=``strongMorph:TH8735''; wer aber sich vor
Gelobenx-morph=``strongMorph:TH8802''
hütetx-morph=``strongMorph:TH8802'', ist
sicherx-morph=``strongMorph:TH8802''. \bibverse{16} Ein holdselig Weib
erlangtx-morph=``strongMorph:TH8799'' Ehre; aber die Tyrannen
erlangenx-morph=``strongMorph:TH8799'' Reichtum. \bibverse{17} Ein
barmherziger Mann tut sich selber Gutesx-morph=``strongMorph:TH8802'';
aber ein unbarmherziger betrübtx-morph=``strongMorph:TH8802'' auch sein
eigen Fleisch. \bibverse{18} Der Gottlosen
Arbeitx-morph=``strongMorph:TH8802'' wird fehlschlagen; aber wer
Gerechtigkeit sätx-morph=``strongMorph:TH8802'', das ist gewisses Gut.
\bibverse{19} Gerechtigkeit fördert zum Leben; aber dem Übel
nachjagenx-morph=``strongMorph:TH8764'' fördert zum Tod. \bibverse{20}
Der HERR hat Greuel an den verkehrten Herzen, und Wohlgefallen an den
Frommen. \bibverse{21} Den Bösen hilft
nichtsx-morph=``strongMorph:TH8735'', wenn sie auch alle Hände
zusammentäten; aber der Gerechten Same wird errettet
werdenx-morph=``strongMorph:TH8738''. \bibverse{22} Ein schönes Weib
ohnex-morph=``strongMorph:TH8802'' Zucht ist wie eine Sau mit einem
goldenen Haarband. \bibverse{23} Der Gerechten Wunsch muß doch wohl
geraten, und der Gottlosen Hoffen wird Unglück. \bibverse{24} Einer
teilt ausx-morph=``strongMorph:TH8764'' und hat immer
mehrx-morph=``strongMorph:TH8737''; ein anderer
kargtx-morph=``strongMorph:TH8802'', da er nicht soll, und wird doch
ärmer. \bibverse{25} Die Seele, die da reichlich segnet, wird
gelabtx-morph=``strongMorph:TH8792''; wer reichlich
tränktx-morph=``strongMorph:TH8688'', der wird auch getränkt
werdenx-morph=``strongMorph:TH8686''. \bibverse{26} Wer Korn
innehältx-morph=``strongMorph:TH8802'', dem
fluchenx-morph=``strongMorph:TH8799'' die Leute; aber Segen kommt über
den, der es verkauftx-morph=``strongMorph:TH8688''. \bibverse{27} Wer da
Gutes suchtx-morph=``strongMorph:TH8802'', dem
widerfährtx-morph=``strongMorph:TH8762'' Gutes; wer aber nach Unglück
ringtx-morph=``strongMorph:TH8802'', dem wird's
begegnenx-morph=``strongMorph:TH8799''. \bibverse{28} Wer sich auf
seinen Reichtum verläßtx-morph=``strongMorph:TH8802'', der wird
untergehenx-morph=``strongMorph:TH8799''; aber die Gerechten werden
grünenx-morph=``strongMorph:TH8799'' wie ein Blatt. \bibverse{29} Wer
sein eigen Haus betrübtx-morph=``strongMorph:TH8802'', der wird Wind zum
Erbteil habenx-morph=``strongMorph:TH8799''; und ein Narr muß ein Knecht
des Weisen sein. \bibverse{30} Die Frucht des Gerechten ist ein Baum des
Lebens, und ein Weiser gewinntx-morph=``strongMorph:TH8802'' die Herzen.
\bibverse{31} So der Gerechte auf Erden leiden
mußx-morph=``strongMorph:TH8792'', wie viel mehr der Gottlose und der
Sünderx-morph=``strongMorph:TH8802''!

\hypertarget{section-11}{%
\section{12}\label{section-11}}

\bibverse{1} Wer sich gernx-morph=``strongMorph:TH8802'' läßt strafen,
derx-morph=``strongMorph:TH8802'' wird klug werden; wer aber ungestraft
sein willx-morph=``strongMorph:TH8802'', der bleibt ein Narr.
\bibverse{2} Wer fromm ist, der bekommtx-morph=``strongMorph:TH8686''
Trost vom HERRN; aber ein Ruchloser verdammt sich
selbstx-morph=``strongMorph:TH8686''. \bibverse{3} Ein gottlos Wesen
fördert den Menschen nichtx-morph=``strongMorph:TH8735''; aber die
Wurzel der Gerechten wird bleibenx-morph=``strongMorph:TH8735''.
\bibverse{4} Ein tugendsam Weib ist eine Krone ihres Mannes; aber eine
bösex-morph=``strongMorph:TH8688'' ist wie Eiter in seinem Gebein.
\bibverse{5} Die Gedanken der Gerechten sind redlich; aber die Anschläge
der Gottlosen sind Trügerei. \bibverse{6} Der Gottlosen Reden
richtenx-morph=``strongMorph:TH8800'' Blutvergießen an; aber der Frommen
Mund errettetx-morph=``strongMorph:TH8686''. \bibverse{7} Die Gottlosen
werden umgestürztx-morph=``strongMorph:TH8800'' und nicht mehr sein;
aber das Haus der Gerechten bleibt stehenx-morph=``strongMorph:TH8799''.
\bibverse{8} Eines weisen Mannes Rat wird
gelobtx-morph=``strongMorph:TH8792''; aber die da
tückischx-morph=``strongMorph:TH8737'' sind, werden zu Schanden.
\bibverse{9} Wer geringx-morph=``strongMorph:TH8737'' ist und wartet des
Seinen, der ist besser, denn der groß sein
willx-morph=``strongMorph:TH8693'', und des Brotes mangelt.
\bibverse{10} Der Gerechte erbarmtx-morph=``strongMorph:TH8802'' sich
seines Viehs; aber das Herz der Gottlosen ist unbarmherzig.
\bibverse{11} Wer seinen Acker bautx-morph=``strongMorph:TH8802'', der
wird Brot die Fülle habenx-morph=``strongMorph:TH8799''; wer aber
unnötigen Sachen nachgehtx-morph=``strongMorph:TH8764'', der ist ein
Narr. \bibverse{12} Des Gottlosen Lustx-morph=``strongMorph:TH8804''
ist, Schaden zu tun; aber die Wurzel der Gerechten wird Frucht
bringenx-morph=``strongMorph:TH8799''. \bibverse{13} Der Böse wird
gefangen in seinen eigenen falschen Worten; aber der Gerechte
entgehtx-morph=``strongMorph:TH8799'' der Angst. \bibverse{14} Viel
Gutes kommtx-morph=``strongMorph:TH8799'' dem Mann durch die Frucht des
Mundes; und dem Menschen wird vergolten, nach dem seine Hände verdient
habenx-morph=``strongMorph:TH8686''\textbar x-morph=``strongMorph:TH8675''x-morph=``strongMorph:TH8799''.
\bibverse{15} Dem Narren gefällt seine Weise wohl; aber wer auf Rat
hörtx-morph=``strongMorph:TH8802'', der ist weise. \bibverse{16} Ein
Narr zeigtx-morph=``strongMorph:TH8735'' seinen Zorn alsbald; aber wer
die Schmach birgtx-morph=``strongMorph:TH8802'', ist klug. \bibverse{17}
Wer wahrhaftig istx-morph=``strongMorph:TH8686'', der sagt
freix-morph=``strongMorph:TH8686'', was recht ist; aber ein falscher
Zeuge betrügt. \bibverse{18} Wer unvorsichtig
herausfährtx-morph=``strongMorph:TH8802'', sticht wie ein Schwert; aber
die Zunge der Weisen ist heilsam. \bibverse{19} Wahrhaftiger Mund
bestehtx-morph=``strongMorph:TH8735'' ewiglich; aber die falsche Zunge
besteht nicht langex-morph=``strongMorph:TH8686''. \bibverse{20} Die, so
Böses ratenx-morph=``strongMorph:TH8802'', betrügen; aber die zum
Frieden ratenx-morph=``strongMorph:TH8802'', schaffen Freude.
\bibverse{21} Es wird dem Gerechten kein Leid
geschehenx-morph=``strongMorph:TH8792''; aber die Gottlosen werden voll
Unglück seinx-morph=``strongMorph:TH8804''. \bibverse{22} Falsche Mäuler
sind dem HERRN ein Greuel; die aber treulich
handelnx-morph=``strongMorph:TH8802'', gefallen ihm wohl. \bibverse{23}
Ein verständiger Mann trägt nicht Klugheit zur
Schaux-morph=``strongMorph:TH8802''; aber das Herz der Narren ruft seine
Narrheit ausx-morph=``strongMorph:TH8799''. \bibverse{24} Fleißige Hand
wird herrschenx-morph=``strongMorph:TH8799''; die aber lässig ist, wird
müssen zinsen. \bibverse{25} Sorge im Herzen
kränktx-morph=``strongMorph:TH8686'', aber ein freundliches Wort
erfreutx-morph=``strongMorph:TH8762''. \bibverse{26} Der Gerechte hat's
besserx-morph=``strongMorph:TH8686'' denn sein Nächster; aber der
Gottlosen Weg verführt siex-morph=``strongMorph:TH8686''. \bibverse{27}
Einem Lässigen gerätx-morph=``strongMorph:TH8799'' sein Handel nicht;
aber ein fleißiger Mensch wird reich. \bibverse{28} Auf dem Wege der
Gerechtigkeit ist Leben, und auf ihrem gebahnten Pfad ist kein Tod.

\hypertarget{section-12}{%
\section{13}\label{section-12}}

\bibverse{1} Ein weiser Sohn läßt sich vom Vater züchtigen; aber ein
Spötterx-morph=``strongMorph:TH8801''
gehorchtx-morph=``strongMorph:TH8804'' der Strafe nicht. \bibverse{2}
Die Frucht des Mundes genießtx-morph=``strongMorph:TH8799'' man; aber
die Verächterx-morph=``strongMorph:TH8802'' denken nur zu freveln.
\bibverse{3} Wer seinen Mund bewahrtx-morph=``strongMorph:TH8802'', der
bewahrtx-morph=``strongMorph:TH8802'' sein Leben; wer aber mit seinem
Maul herausfährtx-morph=``strongMorph:TH8802'', der kommt in Schrecken.
\bibverse{4} Der Faule begehrtx-morph=``strongMorph:TH8693'' und
kriegt's doch nicht; aber die Fleißigen kriegen
genugx-morph=``strongMorph:TH8792''. \bibverse{5} Der Gerechte ist der
Lüge feindx-morph=``strongMorph:TH8799''; aber der Gottlose
schändetx-morph=``strongMorph:TH8686'' und schmäht sich
selbstx-morph=``strongMorph:TH8686''. \bibverse{6} Die Gerechtigkeit
behütetx-morph=``strongMorph:TH8799'' den Unschuldigen; aber das
gottlose Wesen bringt zu Fallx-morph=``strongMorph:TH8762'' den Sünder.
\bibverse{7} Mancher ist armx-morph=``strongMorph:TH8711'' bei großem
Gut, und mancher ist reichx-morph=``strongMorph:TH8693'' bei seiner
Armut. \bibverse{8} Mit Reichtum kann einer sein Leben erretten; aber
ein Armerx-morph=``strongMorph:TH8802''
hörtx-morph=``strongMorph:TH8804'' kein Schelten. \bibverse{9} Das Licht
der Gerechten brennt fröhlichx-morph=``strongMorph:TH8799''; aber die
Leuchte der Gottlosen wird auslöschenx-morph=``strongMorph:TH8799''.
\bibverse{10} Unter den Stolzen istx-morph=``strongMorph:TH8799'' immer
Hader; aber Weisheit ist bei denen, die sich raten
lassenx-morph=``strongMorph:TH8737''. \bibverse{11} Reichtum wird
wenigx-morph=``strongMorph:TH8799'', wo man's vergeudet; was man aber
zusammenhältx-morph=``strongMorph:TH8802'', das wird
großx-morph=``strongMorph:TH8686''. \bibverse{12} Die Hoffnung, die sich
verziehtx-morph=``strongMorph:TH8794'',
ängstetx-morph=``strongMorph:TH8688'' das Herz; wenn's aber
kommtx-morph=``strongMorph:TH8802'', was man begehrt, das ist wie ein
Baum des Lebens. \bibverse{13} Wer das Wort
verachtetx-morph=``strongMorph:TH8802'', der verderbt sich
selbstx-morph=``strongMorph:TH8735''; wer aber das Gebot fürchtet, dem
wird's vergoltenx-morph=``strongMorph:TH8792''. \bibverse{14} Die Lehre
des Weisen ist eine Quelle des Lebens, zu
meidenx-morph=``strongMorph:TH8800'' die Stricke des Todes.
\bibverse{15} Feine Klugheit schafftx-morph=``strongMorph:TH8799''
Gunst; aber der Verächterx-morph=``strongMorph:TH8802'' Weg bringt Wehe.
\bibverse{16} Ein Kluger tutx-morph=``strongMorph:TH8799'' alles mit
Vernunft; ein Narr aber breitet Narrheit
ausx-morph=``strongMorph:TH8799''. \bibverse{17} Ein gottloser Bote
bringtx-morph=``strongMorph:TH8799'' Unglück; aber ein treuer Bote ist
heilsam. \bibverse{18} Wer Zucht läßt
fahrenx-morph=``strongMorph:TH8802'', der hat Armut und Schande; wer
sich gerne strafen läßt, wirdx-morph=``strongMorph:TH8802'' zu ehren
kommenx-morph=``strongMorph:TH8792''. \bibverse{19}
Wenn'sx-morph=``strongMorph:TH8738'' kommt, was man begehrt, das tut dem
Herzen wohlx-morph=``strongMorph:TH8799''; aber das Böse
meidenx-morph=``strongMorph:TH8800'' ist den Toren ein Greuel.
\bibverse{20} Wer mit den Weisen umgehtx-morph=``strongMorph:TH8802'',
der wird weisex-morph=``strongMorph:TH8799''; wer aber der Narren
Gesellex-morph=``strongMorph:TH8802'' ist, der wird Unglück
habenx-morph=``strongMorph:TH8735''. \bibverse{21} Unglück
verfolgtx-morph=``strongMorph:TH8762'' die Sünder; aber den Gerechten
wird Gutes vergoltenx-morph=``strongMorph:TH8762''. \bibverse{22} Der
Gute wird vererbenx-morph=``strongMorph:TH8686'' auf Kindeskind; aber
des Sündersx-morph=``strongMorph:TH8802'' Gut wird für den Gerechten
gespartx-morph=``strongMorph:TH8803''. \bibverse{23} Es ist viel Speise
in den Furchen der
Armenx-morph=``strongMorph:TH8676''x-morph=``strongMorph:TH8802''; aber
die Unrecht tun, verderbenx-morph=``strongMorph:TH8737''. \bibverse{24}
Wer seine Rute schontx-morph=``strongMorph:TH8802'', der
haßtx-morph=``strongMorph:TH8802'' seinen Sohn; wer ihn aber
liebhatx-morph=``strongMorph:TH8802'', der züchtigt ihn
baldx-morph=``strongMorph:TH8765''. \bibverse{25} Der Gerechte
ißtx-morph=``strongMorph:TH8802'', daß sein Seele satt wird; der
Gottlosen Bauch aber hat nimmer genugx-morph=``strongMorph:TH8799''.

\hypertarget{section-13}{%
\section{14}\label{section-13}}

\bibverse{1} Durch weise Weiber wird das Haus
erbautx-morph=``strongMorph:TH8804''; eine Närrin aber
zerbricht'sx-morph=``strongMorph:TH8799'' mit ihrem Tun. \bibverse{2}
Wer den HERRN fürchtet, der wandeltx-morph=``strongMorph:TH8802'' auf
rechter Bahn; wer ihn aber verachtetx-morph=``strongMorph:TH8802'', der
geht auf Abwegenx-morph=``strongMorph:TH8737''. \bibverse{3} Narren
reden tyrannisch; aber die Weisen bewahrenx-morph=``strongMorph:TH8799''
ihren Mund. \bibverse{4} Wo nicht Ochsen sind, da ist die Krippe rein;
aber wo der Ochse geschäftig ist, da ist viel Einkommen. \bibverse{5}
Ein treuer Zeuge lügt nicht; aber ein Falscher Zeuge
redetx-morph=``strongMorph:TH8686'' frech
Lügenx-morph=``strongMorph:TH8762''. \bibverse{6} Der
Spötterx-morph=``strongMorph:TH8801''
suchtx-morph=``strongMorph:TH8765'' Weisheit, und findet sie nicht; aber
dem Verständigenx-morph=``strongMorph:TH8737'' ist die Erkenntnis
leichtx-morph=``strongMorph:TH8738''. \bibverse{7}
Gehex-morph=``strongMorph:TH8798'' von dem Narren; denn du
lernstx-morph=``strongMorph:TH8804'' nichts von ihm. \bibverse{8} Das
ist des Klugen Weisheit, daß er auf seinen Weg
merktx-morph=``strongMorph:TH8687''; aber der Narren Torheit ist eitel
Trug. \bibverse{9} Die Narren treiben das
Gespöttx-morph=``strongMorph:TH8686'' mit der Sünde; aber die Frommen
haben Lust an den Frommen. \bibverse{10} Das Herz
kenntx-morph=``strongMorph:TH8802'' sein eigen Leid, und in seine Freude
kann sich kein Fremderx-morph=``strongMorph:TH8801''
mengenx-morph=``strongMorph:TH8691''. \bibverse{11} Das Haus der
Gottlosen wird vertilgtx-morph=``strongMorph:TH8735''; aber die Hütte
der Frommen wird grünenx-morph=``strongMorph:TH8686''. \bibverse{12} Es
gefällt manchem ein Weg wohl; aber endlich bringt er ihn zum Tode.
\bibverse{13} Auch beim Lachen kann das Herz
trauernx-morph=``strongMorph:TH8799'', und nach der Freude kommt Leid.
\bibverse{14} Einem losenx-morph=``strongMorph:TH8803'' Menschen wird's
gehen wie er handeltx-morph=``strongMorph:TH8799''; aber ein Frommer
wird über ihn sein. \bibverse{15} Ein Unverständiger
glaubtx-morph=``strongMorph:TH8686'' alles; aber ein Kluger merkt
aufx-morph=``strongMorph:TH8799'' seinen Gang. \bibverse{16} Ein Weiser
fürchtet sich und meidetx-morph=``strongMorph:TH8802'' das Arge; ein
Narr aber fährt trotzigx-morph=``strongMorph:TH8693''
hindurchx-morph=``strongMorph:TH8802''. \bibverse{17} Ein Ungeduldiger
handelt törichtx-morph=``strongMorph:TH8799''; aber ein Bedächtiger haßt
esx-morph=``strongMorph:TH8735''. \bibverse{18} Die Unverständigen
erbenx-morph=``strongMorph:TH8804'' Narrheit; aber es ist der Klugen
Kronex-morph=``strongMorph:TH8686'', vorsichtig handeln. \bibverse{19}
Die Bösen müssen sich bückenx-morph=``strongMorph:TH8804'' vor dem Guten
und die Gottlosen in den Toren des Gerechten. \bibverse{20} Einen
Armenx-morph=``strongMorph:TH8802'' hassenx-morph=``strongMorph:TH8735''
auch seine Nächsten; aber die Reichen haben viele
Freundex-morph=``strongMorph:TH8802''. \bibverse{21} Der
Sünderx-morph=``strongMorph:TH8802''
verachtetx-morph=``strongMorph:TH8802'' seinen Nächsten; aber wohl dem,
der sich der Elendenx-morph=``strongMorph:TH8675''
erbarmtx-morph=``strongMorph:TH8781''! \bibverse{22} Die mit bösen
Ränken umgehenx-morph=``strongMorph:TH8802'', werden
fehlgehenx-morph=``strongMorph:TH8799''; die aber Gutes denken, denen
wird Treue und Güte widerfahrenx-morph=``strongMorph:TH8802''.
\bibverse{23} Wo man arbeitet, da ist genug; wo man aber mit Worten
umgeht, da ist Mangel. \bibverse{24} Den Weisen ist ihr Reichtum eine
Krone; aber die Torheit der Narren bleibt Torheit. \bibverse{25} Ein
treuer Zeuge errettetx-morph=``strongMorph:TH8688'' das Leben; aber ein
falscher Zeuge betrügtx-morph=``strongMorph:TH8686''. \bibverse{26} Wer
den HERRN fürchtet, der hat eine sichere Festung, und seine Kinder
werden auch beschirmt. \bibverse{27} Die Furcht des HERRN ist eine
Quelle des Lebens, daß man meidex-morph=``strongMorph:TH8800'' die
Stricke des Todes. \bibverse{28} Wo ein König viel Volks hat, das ist
seine Herrlichkeit; wo aber wenig Volks ist, das macht einen Herrn
blöde. \bibverse{29} Wer geduldig ist, der ist weise; wer aber
ungeduldig ist, der offenbartx-morph=``strongMorph:TH8688'' seine
Torheit. \bibverse{30} Ein gütiges Herz ist des Leibes Leben; aber Neid
ist Eiter in den Gebeinen. \bibverse{31} Wer dem Geringen Gewalt
tutx-morph=``strongMorph:TH8802'', der
lästertx-morph=``strongMorph:TH8765'' desselben
Schöpferx-morph=``strongMorph:TH8802''; aber wer sich des Armen
erbarmtx-morph=``strongMorph:TH8802'', der
ehrtx-morph=``strongMorph:TH8764'' Gott. \bibverse{32} Der Gottlose
besteht nichtx-morph=``strongMorph:TH8735'' in seinem Unglück; aber der
Gerechte ist auch in seinem Tod getrostx-morph=``strongMorph:TH8802''.
\bibverse{33} Im Herzen des Verständigen
ruhtx-morph=``strongMorph:TH8799''
Weisheitx-morph=``strongMorph:TH8737'', und wird
offenbarx-morph=``strongMorph:TH8735'' unter den Narren. \bibverse{34}
Gerechtigkeit erhöhetx-morph=``strongMorph:TH8787'' ein Volk; aber die
Sünde ist der Leute Verderben. \bibverse{35} Ein
klugerx-morph=``strongMorph:TH8688'' Knecht gefällt dem König wohl; aber
einem schändlichen Knecht ist er feindx-morph=``strongMorph:TH8688''.

\hypertarget{section-14}{%
\section{15}\label{section-14}}

\bibverse{1} Eine linde Antwort stilltx-morph=``strongMorph:TH8686'' den
Zorn; aber ein hartes Wort richtet Grimm
anx-morph=``strongMorph:TH8686''. \bibverse{2} Der Weisen Zunge macht
die Lehre lieblichx-morph=``strongMorph:TH8686''; der Narren Mund speit
eitelx-morph=``strongMorph:TH8686'' Narrheit. \bibverse{3} Die Augen des
HERRN schauen an allen Orten beidex-morph=``strongMorph:TH8802'', die
Bösen und die Frommen. \bibverse{4} Ein heilsame Zunge ist ein Baum des
Lebens; aber eine lügenhafte macht Herzeleid. \bibverse{5} Der Narr
lästertx-morph=``strongMorph:TH8799'' die Zucht seines Vaters; wer aber
Strafe annimmtx-morph=``strongMorph:TH8802'', der wird klug
werdenx-morph=``strongMorph:TH8686''. \bibverse{6} In des Gerechten Haus
ist Guts genug; aber in dem Einkommen des Gottlosen ist
Verderbenx-morph=``strongMorph:TH8737''. \bibverse{7} Der Weisen Mund
streutx-morph=``strongMorph:TH8762'' guten Rat; aber der Narren Herz ist
nicht richtig. \bibverse{8} Der Gottlosen Opfer ist dem HERRN ein
Greuel; aber das Gebet der Frommen ist ihm angenehm. \bibverse{9} Der
Gottlosen Weg ist dem HERRN ein Greuel; wer aber der Gerechtigkeit
nachjagtx-morph=``strongMorph:TH8764'', den liebt
erx-morph=``strongMorph:TH8799''. \bibverse{10} Den Weg
verlassenx-morph=``strongMorph:TH8802'' bringt böse Züchtigung, und wer
Strafe haßtx-morph=``strongMorph:TH8802'', der muß
sterbenx-morph=``strongMorph:TH8799''. \bibverse{11} Hölle und Abgrund
ist vor dem HERRN; wie viel mehr der Menschen Herzen! \bibverse{12} Der
Spötterx-morph=``strongMorph:TH8801'' liebt den
nichtx-morph=``strongMorph:TH8799'', der ihn
straftx-morph=``strongMorph:TH8687'', und geht
nichtx-morph=``strongMorph:TH8799'' zu den Weisen. \bibverse{13} Ein
fröhlich Herz macht ein fröhlichx-morph=``strongMorph:TH8686''
Angesicht; aber wenn das Herz bekümmert ist, so fällt auch der Mut.
\bibverse{14} Ein klugesx-morph=``strongMorph:TH8737'' Herz handelt
bedächtigx-morph=``strongMorph:TH8762''; aber der Narren
Mundx-morph=``strongMorph:TH8675'' geht mit Torheit
umx-morph=``strongMorph:TH8799''. \bibverse{15} Ein Betrübter hat nimmer
einen guten Tag; aber ein guter Mut ist ein täglich Wohlleben.
\bibverse{16} Es ist besser ein wenig mit der Furcht des HERRN denn
großer Schatz, darin Unruhe ist. \bibverse{17} Es ist besser ein Gericht
Kraut mit Liebe, denn ein gemästeterx-morph=``strongMorph:TH8803'' Ochse
mit Haß. \bibverse{18} Ein zorniger Mann richtet Hader
anx-morph=``strongMorph:TH8762''; ein Geduldiger aber
stilltx-morph=``strongMorph:TH8686'' den Zank. \bibverse{19} Der Weg des
Faulen ist dornig; aber der Weg des Frommen ist wohl
gebahntx-morph=``strongMorph:TH8803''. \bibverse{20} Ein weiser Sohn
erfreutx-morph=``strongMorph:TH8762'' den Vater, und ein törichter
Mensch ist seiner Mutter Schandex-morph=``strongMorph:TH8802''.
\bibverse{21} Dem Toren ist die Torheit eine Freude; aber ein
verständiger Mann bleibt auf dem rechtenx-morph=``strongMorph:TH8762''
Wegex-morph=``strongMorph:TH8800''. \bibverse{22} Die Anschläge werden
zunichtex-morph=``strongMorph:TH8687'', wo nicht Rat ist; wo aber viel
Ratgeberx-morph=``strongMorph:TH8802'' sind, bestehen
siex-morph=``strongMorph:TH8799''. \bibverse{23} Es ist einem Manne eine
Freude, wenn er richtig antwortet; und ein Wort zu seiner Zeit ist sehr
lieblich. \bibverse{24} Der Weg des Lebens geht überwärts für den
Klugenx-morph=``strongMorph:TH8688'', auf daß er
meidex-morph=``strongMorph:TH8800'' die Hölle unterwärts. \bibverse{25}
Der HERR wird das Haus des Hoffärtigen
zerbrechenx-morph=``strongMorph:TH8799'' und die Grenze der Witwe
bestätigenx-morph=``strongMorph:TH8686''. \bibverse{26} Die Anschläge
des Argen sind dem HERRN ein Greuel; aber freundlich reden die Reinen.
\bibverse{27} Der Geizigex-morph=``strongMorph:TH8802''
verstörtx-morph=``strongMorph:TH8802'' sein eigen Haus; wer aber
Geschenke haßtx-morph=``strongMorph:TH8802'', der wird
lebenx-morph=``strongMorph:TH8799''. \bibverse{28} Das Herz des
Gerechten ersinntx-morph=``strongMorph:TH8799'', was zu
antwortenx-morph=``strongMorph:TH8800'' ist; aber der Mund der Gottlosen
schäumtx-morph=``strongMorph:TH8686'' Böses. \bibverse{29} Der HERR ist
fern von den Gottlosen; aber der Gerechten Gebet erhört
erx-morph=``strongMorph:TH8799''. \bibverse{30} Freundlicher Anblick
erfreutx-morph=``strongMorph:TH8762'' das Herz; eine gute Botschaft
labtx-morph=``strongMorph:TH8762'' das Gebein. \bibverse{31} Das Ohr,
das da hörtx-morph=``strongMorph:TH8802'' die Strafe des Lebens, wird
unter den Weisen wohnenx-morph=``strongMorph:TH8799''. \bibverse{32} Wer
sich nicht ziehen läßtx-morph=``strongMorph:TH8802'', der macht sich
selbst zunichtex-morph=``strongMorph:TH8802''; wer aber auf Strafe
hörtx-morph=``strongMorph:TH8802'', der
wirdx-morph=``strongMorph:TH8802'' klug. \bibverse{33} Die Furcht des
HERRN ist Zucht und Weisheit; und ehe man zu Ehren kommt, muß man zuvor
leiden.

\hypertarget{section-15}{%
\section{16}\label{section-15}}

\bibverse{1} Der Mensch setzt sich's wohl vor im Herzen; aber vom HERRN
kommt, was die Zunge reden soll. \bibverse{2} Einem jeglichen dünken
seine Wege rein; aber der HERR wägtx-morph=``strongMorph:TH8802'' die
Geister. \bibverse{3} Befiehlx-morph=``strongMorph:TH8798'' dem HERRN
deine Werke, so werden deine Anschläge
fortgehenx-morph=``strongMorph:TH8735''. \bibverse{4} Der HERR macht
alles zu bestimmtem Zielx-morph=``strongMorph:TH8804'', auch den
Gottlosen für den bösen Tag. \bibverse{5} Ein stolzes Herz ist dem HERRN
ein Greuel und wird nicht ungestraft
bleibenx-morph=``strongMorph:TH8735'', wenn sie gleich alle aneinander
hängen. \bibverse{6} Durch Güte und Treue wird Missetat
versöhntx-morph=``strongMorph:TH8792'', und durch die Furcht des HERRN
meidetx-morph=``strongMorph:TH8800'' man das Böse. \bibverse{7} Wenn
jemands Wege dem HERRN wohl gefallenx-morph=``strongMorph:TH8800'', so
macht er auch seine Feindex-morph=``strongMorph:TH8802'' mit ihm
zufriedenx-morph=``strongMorph:TH8686''. \bibverse{8} Es ist besser ein
wenig mit Gerechtigkeit denn viel Einkommen mit Unrecht. \bibverse{9}
Des Menschen Herz erdenktx-morph=``strongMorph:TH8762'' sich seinen Weg;
aber der HERR allein gibtx-morph=``strongMorph:TH8686'', daß er
fortgehe. \bibverse{10} Weissagung ist in dem Munde des Königs; sein
Mund fehlt nichtx-morph=``strongMorph:TH8799'' im Gericht. \bibverse{11}
Rechte Waage und Gewicht ist vom HERRN; und alle Pfunde im Sack sind
seine Werke. \bibverse{12} Den Königen ist Unrecht
tunx-morph=``strongMorph:TH8800'' ein Greuel; denn durch Gerechtigkeit
wird der Thron befestigtx-morph=``strongMorph:TH8735''. \bibverse{13}
Recht raten gefällt den Königen; und wer aufrichtig
redetx-morph=``strongMorph:TH8802'', wird
geliebtx-morph=``strongMorph:TH8799''. \bibverse{14} Des Königs Grimm
ist ein Bote des Todes; aber ein weiser Mann wird ihn
versöhnenx-morph=``strongMorph:TH8762''. \bibverse{15} Wenn des Königs
Angesicht freundlich ist, das ist Leben, und seine Gnade ist wie ein
Spätregen. \bibverse{16} Nimm anx-morph=``strongMorph:TH8800'' die
Weisheit, denn sie ist besser als Gold;
undx-morph=``strongMorph:TH8737'' Verstand
habenx-morph=``strongMorph:TH8800'' ist edler als Silber. \bibverse{17}
Der Frommen Weg meidetx-morph=``strongMorph:TH8800'' das Arge; und wer
seinen Weg bewahrtx-morph=``strongMorph:TH8802'', der
erhältx-morph=``strongMorph:TH8802'' sein Leben. \bibverse{18} Wer zu
Grunde gehen soll, der wird zuvor stolz; und Hochmut kommt vor dem Fall.
\bibverse{19} Es ist besser niedrigen Gemüts sein mit den
Elendenx-morph=``strongMorph:TH8675'', denn Raub
austeilenx-morph=``strongMorph:TH8763'' mit den Hoffärtigen.
\bibverse{20} Wer eine Sache klüglichx-morph=``strongMorph:TH8688''
führt, der findetx-morph=``strongMorph:TH8799'' Glück; und wohl dem, der
sich auf den HERRN verläßtx-morph=``strongMorph:TH8802''! \bibverse{21}
Ein Verständiger wird gerühmtx-morph=``strongMorph:TH8735'' für einen
weisenx-morph=``strongMorph:TH8737'' Mann, und liebliche Reden lehren
wohlx-morph=``strongMorph:TH8686''. \bibverse{22} Klugheit ist wie ein
Brunnen des Lebens dem, der sie hat; aber die Zucht der Narren ist
Narrheit. \bibverse{23} Ein weises Herz redet
klugx-morph=``strongMorph:TH8686'' und lehrt
wohlx-morph=``strongMorph:TH8686''. \bibverse{24} Die Reden des
Freundlichen sind Honigseim, trösten die Seele und erfrischen die
Gebeine. \bibverse{25} Manchem gefällt ein Weg wohl; aber zuletzt bringt
er ihn zum Tode. \bibverse{26} Mancher
kommtx-morph=``strongMorph:TH8804'' zu großem
Unglückx-morph=``strongMorph:TH8804'' durch sein eigen Maul.
\bibverse{27} Ein loser Mensch gräbtx-morph=``strongMorph:TH8802'' nach
Unglück, und in seinem Maul brennt Feuer. \bibverse{28} Ein verkehrter
Mensch richtet Hader anx-morph=``strongMorph:TH8762'', und ein
Verleumder machtx-morph=``strongMorph:TH8688'' Freunde uneins.
\bibverse{29} Ein Frevler locktx-morph=``strongMorph:TH8762'' seinen
Nächsten und führtx-morph=``strongMorph:TH8689'' ihn auf keinen guten
Weg. \bibverse{30} Wer mit den Augen
winktx-morph=``strongMorph:TH8802'', denktx-morph=``strongMorph:TH8800''
nichts Gutes; und wer mit den Lippen
andeutetx-morph=``strongMorph:TH8802'',
vollbringtx-morph=``strongMorph:TH8765'' Böses. \bibverse{31} Graue
Haare sind eine Krone der Ehren, die auf dem Wege der Gerechtigkeit
gefundenx-morph=``strongMorph:TH8735'' wird. \bibverse{32} Ein
Geduldiger ist besser denn ein Starker, und der seines Mutes Herr
istx-morph=``strongMorph:TH8802'', denn der Städte
gewinntx-morph=``strongMorph:TH8802''. \bibverse{33} Das Los wird
geworfenx-morph=``strongMorph:TH8714'' in den Schoß; aber es fällt, wie
der HERR will.

\hypertarget{section-16}{%
\section{17}\label{section-16}}

\bibverse{1} Es ist ein trockner Bissen, daran man sich genügen läßt,
besser denn ein Haus voll Geschlachtetes mit Hader. \bibverse{2} Ein
klugerx-morph=``strongMorph:TH8688'' Knecht wird
herrschenx-morph=``strongMorph:TH8799'' über
unfleißigex-morph=``strongMorph:TH8688'' Erben und wird unter den
Brüdern das Erbe austeilenx-morph=``strongMorph:TH8799''. \bibverse{3}
Wie das Feuer Silber und der Ofen Gold, also
prüftx-morph=``strongMorph:TH8802'' der HERR die Herzen. \bibverse{4}
Ein Böserx-morph=``strongMorph:TH8688'' achtet
aufx-morph=``strongMorph:TH8688'' böse Mäuler, und ein Falscher
gehorchtx-morph=``strongMorph:TH8688'' den schädlichen Zungen.
\bibverse{5} Wer des Dürftigenx-morph=``strongMorph:TH8802''
spottetx-morph=``strongMorph:TH8802'', der
höhntx-morph=``strongMorph:TH8765'' desselben
Schöpferx-morph=``strongMorph:TH8802''; und wer sich über eines andern
Unglück freut, der wird nicht ungestraft
bleibenx-morph=``strongMorph:TH8735''. \bibverse{6} Der Alten Krone sind
Kindeskinder, und der Kinder Ehre sind ihre Väter. \bibverse{7} Es steht
einem Narren nicht wohl an, von hohen Dingen reden, viel weniger einem
Fürsten, daß er gern lügt. \bibverse{8} Wer zu schenken hat, dem ist's
ein Edelstein; wo er sich hin kehrtx-morph=``strongMorph:TH8799'', ist
er klug geachtetx-morph=``strongMorph:TH8686''. \bibverse{9} Wer Sünde
zudecktx-morph=``strongMorph:TH8764'', der
machtx-morph=``strongMorph:TH8764'' Freundschaft; wer aber die Sache
aufrührtx-morph=``strongMorph:TH8802'', der
machtx-morph=``strongMorph:TH8688'' Freunde uneins. \bibverse{10}
Schelten bringt mehrx-morph=``strongMorph:TH8799'' ein an dem
Verständigenx-morph=``strongMorph:TH8688'' denn hundert
Schlägex-morph=``strongMorph:TH8687'' an dem Narren. \bibverse{11} Ein
bitterer Mensch trachtetx-morph=``strongMorph:TH8762'', eitel Schaden zu
tun; aber es wird ein grimmiger Engel über ihn
kommenx-morph=``strongMorph:TH8792''. \bibverse{12} Es ist besser, einem
Bären begegnenx-morph=``strongMorph:TH8800'', dem die Jungen geraubt
sind, denn einem Narren in seiner Narrheit. \bibverse{13} Wer Gutes mit
Bösem vergiltx-morph=``strongMorph:TH8688'', von dessen Haus wird Böses
nicht
lassenx-morph=``strongMorph:TH8799''\textbar x-morph=``strongMorph:TH8675''x-morph=``strongMorph:TH8686''.
\bibverse{14} Wer Hader anfängt, ist gleich dem, der dem Wasser den Damm
aufreißtx-morph=``strongMorph:TH8802''. Laß du
vomx-morph=``strongMorph:TH8800'' Hader, ehe du drein gemengt
wirstx-morph=``strongMorph:TH8694''. \bibverse{15} Wer den Gottlosen
gerechtsprichtx-morph=``strongMorph:TH8688'' und den Gerechten
verdammtx-morph=``strongMorph:TH8688'', die sind beide dem HERRN ein
Greuel. \bibverse{16} Was soll dem Narren Geld in der Hand, Weisheit zu
kaufenx-morph=``strongMorph:TH8800'', so er doch ein Narr ist?
\bibverse{17} Ein Freund liebtx-morph=``strongMorph:TH8802'' allezeit,
und als ein Bruder wird er in Not
erfundenx-morph=``strongMorph:TH8735''. \bibverse{18} Es ist ein Narr,
der in die Hand gelobtx-morph=``strongMorph:TH8802'' und Bürge
wirdx-morph=``strongMorph:TH8802'' für seinen Nächsten. \bibverse{19}
Wer Zank liebtx-morph=``strongMorph:TH8802'', der
liebtx-morph=``strongMorph:TH8802'' Sünde; und wer seine Türe hoch
machtx-morph=``strongMorph:TH8688'', ringtx-morph=``strongMorph:TH8764''
nach Einsturz. \bibverse{20} Ein verkehrtes Herz findet
nichtsx-morph=``strongMorph:TH8799'' Gutes; und der
verkehrterx-morph=``strongMorph:TH8738'' Zunge ist, wird in Unglück
fallenx-morph=``strongMorph:TH8799''. \bibverse{21} Wer einen Narren
zeugtx-morph=``strongMorph:TH8802'', der hat Grämen; und eines Narren
Vater hat keine Freudex-morph=``strongMorph:TH8799''. \bibverse{22} Ein
fröhlich Herz macht das Leben lustigx-morph=``strongMorph:TH8686''; aber
ein betrübter Mut vertrocknetx-morph=``strongMorph:TH8762'' das Gebein.
\bibverse{23} Der Gottlose nimmtx-morph=``strongMorph:TH8799'' heimlich
gern Geschenke, zu beugenx-morph=``strongMorph:TH8687'' den Weg des
Rechts. \bibverse{24} Ein Verständigerx-morph=``strongMorph:TH8688''
gebärdet sich weise; ein Narr wirft die Augen hin und her. \bibverse{25}
Ein törichter Sohn ist seines Vaters Trauern und Betrübnis der Mutter,
die ihn geboren hatx-morph=``strongMorph:TH8802''. \bibverse{26} Es ist
nicht gut, daß man den Gerechten schindetx-morph=``strongMorph:TH8800'',
noch den Edlen zu schlagenx-morph=``strongMorph:TH8687'', der recht
handelt. \bibverse{27} Ein Vernünftigerx-morph=``strongMorph:TH8802''
mäßigtx-morph=``strongMorph:TH8802'' seine Rede; und ein verständiger
Mann ist kaltesx-morph=``strongMorph:TH8675'' Muts. \bibverse{28} Ein
Narr, wenn er schwiegex-morph=``strongMorph:TH8688'', wurde auch für
weise gerechnetx-morph=``strongMorph:TH8735'', und für
verständigx-morph=``strongMorph:TH8737'', wenn er das Maul
hieltex-morph=``strongMorph:TH8801''.

\hypertarget{section-17}{%
\section{18}\label{section-17}}

\bibverse{1} Wer sich absondertx-morph=``strongMorph:TH8737'', der
suchtx-morph=``strongMorph:TH8762'', was ihn gelüstet, und setzt
sichx-morph=``strongMorph:TH8691'' wider alles, was gut ist.
\bibverse{2} Ein Narr hat nicht Lustx-morph=``strongMorph:TH8799'' am
Verstand, sondern kundzutun, was in seinem Herzen
stecktx-morph=``strongMorph:TH8692''. \bibverse{3} Wo der Gottlose hin
kommtx-morph=``strongMorph:TH8800'', da
kommtx-morph=``strongMorph:TH8804'' Verachtung und Schmach mit Hohn.
\bibverse{4} Die Worte in eines Mannes Munde sind wie tiefe Wasser, und
die Quelle der Weisheit ist ein vollerx-morph=``strongMorph:TH8802''
Strom. \bibverse{5} Es ist nicht gut, die Person des Gottlosen
achtenx-morph=``strongMorph:TH8800'', zu
beugenx-morph=``strongMorph:TH8687'' den Gerechten im Gericht.
\bibverse{6} Die Lippen des Narren bringenx-morph=``strongMorph:TH8799''
Zank, und sein Mund ringtx-morph=``strongMorph:TH8799'' nach Schlägen.
\bibverse{7} Der Mund des Narren schadet ihm selbst, und seine Lippen
fangen seine eigene Seele. \bibverse{8} Die Worte des Verleumders sind
Schlägex-morph=``strongMorph:TH8693'' und
gehenx-morph=``strongMorph:TH8804'' einem durchs Herz. \bibverse{9} Wer
lässig istx-morph=``strongMorph:TH8693'' in seiner Arbeit, der ist ein
Bruder des, der das Seine umbringtx-morph=``strongMorph:TH8688''.
\bibverse{10} Der Name des HERRN ist ein festes Schloß; der Gerechte
läuft dahinx-morph=``strongMorph:TH8799'' und wird
beschirmtx-morph=``strongMorph:TH8738''. \bibverse{11} Das Gut des
Reichen ist ihm eine feste Stadt und wie
hohex-morph=``strongMorph:TH8737'' Mauern in seinem Dünkel.
\bibverse{12} Wenn einer zu Grunde gehen soll, wird sein Herz zuvor
stolzx-morph=``strongMorph:TH8799''; und ehe man zu Ehren kommt, muß man
zuvor leiden. \bibverse{13} Wer antwortetx-morph=``strongMorph:TH8688''
ehe er hörtx-morph=``strongMorph:TH8799'', dem ist's Narrheit und
Schande. \bibverse{14} Wer ein fröhlich Herz hat, der weiß sich in
seinem Leiden zu haltenx-morph=``strongMorph:TH8770''; wenn aber der Mut
liegt, wer kann's tragenx-morph=``strongMorph:TH8799''? \bibverse{15}
Ein verständigesx-morph=``strongMorph:TH8737'' Herz weiß sich vernünftig
zu haltenx-morph=``strongMorph:TH8799''; und die Weisen hören gern, wie
man vernünftig handeltx-morph=``strongMorph:TH8762''. \bibverse{16} Das
Geschenk des Menschen macht ihm Raumx-morph=``strongMorph:TH8686'' und
bringtx-morph=``strongMorph:TH8686'' ihn vor die großen Herren.
\bibverse{17} Ein jeglicher ist zuerst in seiner Sache gerecht;
kommtx-morph=``strongMorph:TH8804''\textbar x-morph=``strongMorph:TH8675''
aber sein Nächster hinzux-morph=``strongMorph:TH8799'', so findet
sich'sx-morph=``strongMorph:TH8804''. \bibverse{18} Das Los
stilltx-morph=``strongMorph:TH8686'' den Hader und
scheidetx-morph=``strongMorph:TH8686'' zwischen den Mächtigen.
\bibverse{19} Ein verletzter Bruder hält
härterx-morph=``strongMorph:TH8737'' den eine feste Stadt, und Zank hält
härterx-morph=``strongMorph:TH8675'' denn Riegel am Palast.
\bibverse{20} Einem Mann wird vergoltenx-morph=``strongMorph:TH8799'',
darnach sein Mund geredet hat, und er wird
gesättigtx-morph=``strongMorph:TH8799'' von der Frucht seiner Lippen.
\bibverse{21} Tod und Leben steht in der Zunge Gewalt; wer sie
liebtx-morph=``strongMorph:TH8802'', der wird von ihrer Frucht
essenx-morph=``strongMorph:TH8799''. \bibverse{22} Wer eine Ehefrau
findetx-morph=``strongMorph:TH8804'', der findet
etwasx-morph=``strongMorph:TH8804'' Gutes und kann guter Dinge
seinx-morph=``strongMorph:TH8686'' im HERRN. \bibverse{23} Ein
Armerx-morph=``strongMorph:TH8802'' redetx-morph=``strongMorph:TH8762''
mit Flehen, ein Reicher antwortetx-morph=``strongMorph:TH8799'' stolz.
\bibverse{24} Ein treuerx-morph=``strongMorph:TH8710'' Freund
liebtx-morph=``strongMorph:TH8802'' mehr uns steht fester bei denn ein
Bruder.

\hypertarget{section-18}{%
\section{19}\label{section-18}}

\bibverse{1} Ein Armerx-morph=``strongMorph:TH8802'', der in seiner
Frömmigkeit wandeltx-morph=``strongMorph:TH8802'', ist besser denn ein
Verkehrter mit seinen Lippen, der doch ein Narr ist. \bibverse{2} Wo man
nicht mit Vernunft handelt, da geht's nicht wohl zu; und wer
schnellx-morph=``strongMorph:TH8801'' ist mit Füßen, der tut sich
Schadenx-morph=``strongMorph:TH8802''. \bibverse{3} Die Torheit eines
Menschen verleitetx-morph=``strongMorph:TH8762'' seinen Weg, und doch
tobtx-morph=``strongMorph:TH8799'' sein Herz wider den HERRN.
\bibverse{4} Gut machtx-morph=``strongMorph:TH8686'' viele Freunde; aber
der Arme wird von seinen Freunden
verlassenx-morph=``strongMorph:TH8735''. \bibverse{5} Ein falscher Zeuge
bleibt nicht ungestraftx-morph=``strongMorph:TH8735''; und wer Lügen
frech redetx-morph=``strongMorph:TH8686'', wird nicht
entrinnenx-morph=``strongMorph:TH8735''. \bibverse{6} Viele
schmeichelnx-morph=``strongMorph:TH8762'' der Person des Fürsten; und
alle sind Freunde des, der Geschenke gibt. \bibverse{7} Den
Armenx-morph=``strongMorph:TH8802'' hassenx-morph=``strongMorph:TH8804''
alle seine Brüder; wie viel mehrx-morph=``strongMorph:TH8804'' halten
sich seine Freunde von ihm fern! Und wer sich auf Worte
verläßtx-morph=``strongMorph:TH8764'', dem wird nichts. \bibverse{8} Wer
klug wirdx-morph=``strongMorph:TH8802'',
liebtx-morph=``strongMorph:TH8802'' sein Leben;
undx-morph=``strongMorph:TH8802'' der Verständige
findetx-morph=``strongMorph:TH8800'' Gutes. \bibverse{9} Ein falscher
Zeuge bleibt nicht ungestraftx-morph=``strongMorph:TH8735''; und wer
frech Lügen redetx-morph=``strongMorph:TH8686'', wird
umkommenx-morph=``strongMorph:TH8799''. \bibverse{10} Dem Narren steht
nicht wohl an, gute Tage haben, viel weniger einem Knecht, zu
herrschenx-morph=``strongMorph:TH8800'' über Fürsten. \bibverse{11} Wer
geduldig istx-morph=``strongMorph:TH8689'', der ist ein kluger Mensch,
und ist ihm eine Ehre, daß er Untugend überhören
kannx-morph=``strongMorph:TH8800''. \bibverse{12} Die Ungnade des Königs
ist wie das Brüllen eines jungen Löwen; aber seine Gnade ist wie der Tau
auf dem Grase. \bibverse{13} Ein törichter Sohn ist seines Vaters
Herzeleid, und ein zänkisches Weib ein
stetigesx-morph=``strongMorph:TH8802'' Triefen. \bibverse{14} Haus und
Güter vererben die Eltern; aber ein
vernünftigesx-morph=``strongMorph:TH8688'' Weib kommt vom HERRN.
\bibverse{15} Faulheit bringtx-morph=``strongMorph:TH8686'' Schlafen,
und eine lässige Seele wird Hunger leidenx-morph=``strongMorph:TH8799''.
\bibverse{16} Wer das Gebot bewahrtx-morph=``strongMorph:TH8802'', der
bewahrtx-morph=``strongMorph:TH8802'' sein Leben; wer aber seines Weges
nicht achtetx-morph=``strongMorph:TH8802'', wird
sterbenx-morph=``strongMorph:TH8799''. \bibverse{17} Wer sich des Armen
erbarmtx-morph=``strongMorph:TH8802'', der
leihetx-morph=``strongMorph:TH8688'' dem HERRN; der wird ihm wieder
Gutes vergeltenx-morph=``strongMorph:TH8762''. \bibverse{18}
Züchtigex-morph=``strongMorph:TH8761'' deinen Sohn, solange Hoffnung da
ist; aber laßx-morph=``strongMorph:TH8799'' deine Seele nicht bewegt
werden, ihn zu tötenx-morph=``strongMorph:TH8687''. \bibverse{19} Großer
Grimm muß Schaden leidenx-morph=``strongMorph:TH8802''; denn willst du
ihm steuernx-morph=``strongMorph:TH8686'', so wird er noch
größerx-morph=``strongMorph:TH8686''. \bibverse{20}
Gehorchex-morph=``strongMorph:TH8798'' dem Rat, und
nimmx-morph=``strongMorph:TH8761'' Zucht an, daß du hernach weise
seiestx-morph=``strongMorph:TH8799''. \bibverse{21} Es sind viel
Anschläge in eines Mannes Herzen; aber der Rat des HERRN
bestehtx-morph=``strongMorph:TH8799''. \bibverse{22} Ein Mensch hat Lust
an seiner Wohltat; und ein Armerx-morph=``strongMorph:TH8802'' ist
besser denn ein Lügner. \bibverse{23} Die Furcht des HERRN fördert zum
Leben, und wird satt bleibenx-morph=``strongMorph:TH8799'', daß kein
Übel sie heimsuchen wirdx-morph=``strongMorph:TH8735''. \bibverse{24}
Der Faule verbirgtx-morph=``strongMorph:TH8804'' seine Hand im Topf und
bringtx-morph=``strongMorph:TH8686'' sie nicht wieder zum Munde.
\bibverse{25} Schlägtx-morph=``strongMorph:TH8686'' man den
Spötterx-morph=``strongMorph:TH8801'', so wird der Unverständige
klugx-morph=``strongMorph:TH8686''; straftx-morph=``strongMorph:TH8689''
man einen Verständigenx-morph=``strongMorph:TH8737'', so wird er
vernünftigx-morph=``strongMorph:TH8799''. \bibverse{26} Wer Vater
verstörtx-morph=``strongMorph:TH8764'' und Mutter
verjagtx-morph=``strongMorph:TH8686'', der ist ein
schändlichesx-morph=``strongMorph:TH8688'' und
verfluchtesx-morph=``strongMorph:TH8688'' Kind. \bibverse{27} Laß
abx-morph=``strongMorph:TH8798'', mein Sohn, zu
hörenx-morph=``strongMorph:TH8800'' die Zucht, und doch
abzuirrenx-morph=``strongMorph:TH8800'' von vernünftiger Lehre.
\bibverse{28} Ein loser Zeuge spottetx-morph=``strongMorph:TH8686'' des
Rechts, und der Gottlosen Mund verschlingtx-morph=``strongMorph:TH8762''
das Unrecht. \bibverse{29} Den Spötternx-morph=``strongMorph:TH8801''
sind Strafen bereitetx-morph=``strongMorph:TH8738'', und Schläge auf der
Narren Rücken.

\hypertarget{section-19}{%
\section{20}\label{section-19}}

\bibverse{1} Der Wein macht lose Leutex-morph=``strongMorph:TH8801'',
und starkes Getränk macht wildx-morph=``strongMorph:TH8802''; wer dazu
Lust hatx-morph=``strongMorph:TH8802'', wird nimmer
weisex-morph=``strongMorph:TH8799''. \bibverse{2} Das Schrecken des
Königs ist wie das Brüllen eines jungen Löwen; wer ihn
erzürntx-morph=``strongMorph:TH8693'', der
sündigtx-morph=``strongMorph:TH8802'' wider sein Leben. \bibverse{3} Es
ist dem Mann eine Ehre, vom Hader bleiben; aber die gern
Hadernx-morph=``strongMorph:TH8691'', sind allzumal Narren. \bibverse{4}
Um der Kälte willen will der Faule nicht
pflügenx-morph=``strongMorph:TH8799''; so muß er in der Ernte
bettelnx-morph=``strongMorph:TH8804''\textbar x-morph=``strongMorph:TH8675''x-morph=``strongMorph:TH8799''
und nichts kriegen. \bibverse{5} Der Rat im Herzen eines Mannes ist wie
tiefe Wasser; aber ein Verständiger kann's merken, was er
meintx-morph=``strongMorph:TH8799''. \bibverse{6} Viele Menschen werden
fromm gerühmtx-morph=``strongMorph:TH8799''; aber wer will
findenx-morph=``strongMorph:TH8799'' einen, der rechtschaffen fromm sei?
\bibverse{7} Ein Gerechter, der in seiner Frömmigkeit
wandeltx-morph=``strongMorph:TH8693'', des Kindern wird's wohl gehen
nach ihm. \bibverse{8} Ein König, der auf seinem Stuhl
sitztx-morph=``strongMorph:TH8802'', zu richten,
zerstreutx-morph=``strongMorph:TH8764'' alles Arge mit seinen Augen.
\bibverse{9} Wer kann sagenx-morph=``strongMorph:TH8799'': Ich bin
reinx-morph=``strongMorph:TH8765'' in meinem Herzen und
lauterx-morph=``strongMorph:TH8804'' von meiner Sünde? \bibverse{10}
Mancherlei Gewicht und Maß ist beides Greuel dem HERRN. \bibverse{11}
Auch einen Knaben kenntx-morph=``strongMorph:TH8691'' man an seinem
Wesen, ob er fromm und redlich werden will. \bibverse{12} Ein
hörendx-morph=``strongMorph:TH8802'' Ohr und
sehendx-morph=``strongMorph:TH8802'' Auge, die
machtx-morph=``strongMorph:TH8804'' beide der HERR. \bibverse{13} Liebe
den Schlaf nichtx-morph=``strongMorph:TH8799'', daß du nicht arm
werdestx-morph=``strongMorph:TH8735''; laß deine Augen wacker
seinx-morph=``strongMorph:TH8798'', so wirst du Brot genug
habenx-morph=``strongMorph:TH8798''. \bibverse{14} ``Böse, böse!''
sprichtx-morph=``strongMorph:TH8799'' manx-morph=``strongMorph:TH8802'',
wenn man's hat; aber wenn's weg istx-morph=``strongMorph:TH8801'', so
rühmt man es dannx-morph=``strongMorph:TH8691''. \bibverse{15} Es gibt
Gold und viele Perlen; aber ein vernünftiger Mund ist ein edles Kleinod.
\bibverse{16} Nimmx-morph=``strongMorph:TH8798'' dem sein Kleid, der für
einen andernx-morph=``strongMorph:TH8801''
Bürgex-morph=``strongMorph:TH8804'' wird, und
pfändex-morph=``strongMorph:TH8798'' ihn um des Fremden willen.
\bibverse{17} Das gestohlene Brot schmeckt dem Manne wohl; aber hernach
wird ihm der Mund voll Kieselsteine
werdenx-morph=``strongMorph:TH8735''. \bibverse{18} Anschläge bestehen,
wenn man sie mit Rat führtx-morph=``strongMorph:TH8735''; und Krieg soll
man mit Vernunft führenx-morph=``strongMorph:TH8798''. \bibverse{19} Sei
unverworrenx-morph=``strongMorph:TH8802'' mit dem, der Heimlichkeit
offenbartx-morph=``strongMorph:TH8802'', und mit dem
Verleumderx-morph=``strongMorph:TH8691'' und mit dem
falschenx-morph=``strongMorph:TH8802'' Maul. \bibverse{20} Wer seinem
Vater und seiner Mutter fluchtx-morph=``strongMorph:TH8764'', des
Leuchte wird verlöschenx-morph=``strongMorph:TH8799''
mittenx-morph=``strongMorph:TH8676'' in der Finsternis. \bibverse{21}
Das Erbe, darnach man zuerst sehr eilt
wirdx-morph=``strongMorph:TH8794''\textbar x-morph=``strongMorph:TH8675''x-morph=``strongMorph:TH8794''
zuletzt nicht gesegnet seinx-morph=``strongMorph:TH8792''. \bibverse{22}
Sprichx-morph=``strongMorph:TH8799'' nicht: Ich will Böses
vergeltenx-morph=``strongMorph:TH8762''!
Harrex-morph=``strongMorph:TH8761'' des HERRN, der wird dir
helfenx-morph=``strongMorph:TH8686''. \bibverse{23} Mancherlei Gewicht
ist ein Greuel dem HERRN, und eine falsche Waage ist nicht gut.
\bibverse{24} Jedermanns Gänge kommen vom HERRN. Welcher Mensch
verstehtx-morph=``strongMorph:TH8799'' seinen Weg? \bibverse{25} Es ist
dem Menschen ein Strick, sich mit Heiligem
übereilenx-morph=``strongMorph:TH8804'' und erst nach den Geloben
überlegenx-morph=``strongMorph:TH8763''. \bibverse{26} Ein weiser König
zerstreutx-morph=``strongMorph:TH8764'' die Gottlosen und
bringtx-morph=``strongMorph:TH8686'' das Rad über sie. \bibverse{27}
Eine Leuchte des HERRN ist des Menschen Geist; die geht
durchx-morph=``strongMorph:TH8802'' alle Kammern des Leibes.
\bibverse{28} Fromm und wahrhaftig sein
behütetx-morph=``strongMorph:TH8799'' den König, und sein Thron
bestehtx-morph=``strongMorph:TH8804'' durch Frömmigkeit. \bibverse{29}
Der Jünglinge Stärke ist ihr Preis; und graues Haar ist der Alten
Schmuck. \bibverse{30} Man muß dem Bösen wehren mit harter Strafe und
mit ernsten Schlägen, die man fühlt.

\hypertarget{section-20}{%
\section{21}\label{section-20}}

\bibverse{1} Des Königs Herz ist in der Hand des HERRN wie Wasserbäche,
und er neigtx-morph=``strongMorph:TH8686'' es wohin er
willx-morph=``strongMorph:TH8799''. \bibverse{2} Einen jeglichen dünkt
sein Weg recht; aber der HERR wägtx-morph=``strongMorph:TH8802'' die
Herzen. \bibverse{3} Wohl und recht tunx-morph=``strongMorph:TH8800''
ist dem HERRN lieberx-morph=``strongMorph:TH8737'' denn Opfer.
\bibverse{4} Hoffärtige Augen und stolzer Mut, die Leuchte der
Gottlosen, ist Sünde. \bibverse{5} Die Anschläge eines Emsigen bringen
Überfluß; wer aber allzu rasch istx-morph=``strongMorph:TH8801'', dem
wird's mangeln. \bibverse{6} Wer Schätze sammelt mit Lügen,
derx-morph=``strongMorph:TH8737'' wird fehlgehen und ist unter denen,
die den Tod suchenx-morph=``strongMorph:TH8764''. \bibverse{7} Der
Gottlosen Rauben wird sie erschreckenx-morph=``strongMorph:TH8799'';
denn sie wollten nichtx-morph=``strongMorph:TH8765''
tunx-morph=``strongMorph:TH8800'', was recht war. \bibverse{8} Wer mit
Schuld beladen ist, geht krumme Wege; wer aber rein ist, des Werk ist
recht. \bibverse{9} Es ist besser wohnenx-morph=``strongMorph:TH8800''
im Winkel auf dem Dach, denn bei einem
zänkischenx-morph=``strongMorph:TH8675'' Weibe in einem Haus beisammen.
\bibverse{10} Die Seele des Gottlosen
wünschtx-morph=``strongMorph:TH8765'' Arges und gönnt seinem Nächsten
nichtsx-morph=``strongMorph:TH8717''. \bibverse{11} Wenn der
Spötterx-morph=``strongMorph:TH8801''
gestraftx-morph=``strongMorph:TH8800'' wird, so werden die
Unvernünftigen Weisex-morph=``strongMorph:TH8799''; und wenn man einen
Weisen unterrichtetx-morph=``strongMorph:TH8687'', so
wirdx-morph=``strongMorph:TH8799'' er vernünftig. \bibverse{12} Der
Gerechte hält sich weislichx-morph=``strongMorph:TH8688'' gegen des
Gottlosen Haus; aber die Gottlosen denken nur Schaden zu
tunx-morph=``strongMorph:TH8764''. \bibverse{13} Wer seine Ohren
verstopftx-morph=``strongMorph:TH8801'' vor dem Schreien des Armen, der
wird auch rufenx-morph=``strongMorph:TH8799'', und nicht erhört
werdenx-morph=``strongMorph:TH8735''. \bibverse{14} Eine heimliche Gabe
stilltx-morph=``strongMorph:TH8799'' den Zorn, und ein Geschenk im Schoß
den heftigen Grimm. \bibverse{15} Es ist dem Gerechten eine Freude, zu
tunx-morph=``strongMorph:TH8800'', was recht ist, aber eine Furcht den
Übeltäternx-morph=``strongMorph:TH8802''. \bibverse{16} Ein Mensch, der
vom Wege der Klugheitx-morph=``strongMorph:TH8687''
irrtx-morph=``strongMorph:TH8802'', wird
bleibenx-morph=``strongMorph:TH8799'' in der Toten Gemeinde.
\bibverse{17} Wer gern in Freuden lebtx-morph=``strongMorph:TH8802'',
dem wird's mangeln; und wer Wein und Öl
liebtx-morph=``strongMorph:TH8802'', wird nicht
reichx-morph=``strongMorph:TH8686''. \bibverse{18} Der Gottlose muß für
den Gerechten gegeben werden und der
Verächterx-morph=``strongMorph:TH8802'' für die Frommen. \bibverse{19}
Es ist besser, wohnenx-morph=``strongMorph:TH8800'' im wüsten Lande denn
bei einem zänkischenx-morph=``strongMorph:TH8675'' und zornigen Weibe.
\bibverse{20} Im Hause des Weisen ist ein
lieblicherx-morph=``strongMorph:TH8737'' Schatz und Öl; aber ein Narr
verschlemmtx-morph=``strongMorph:TH8762'' es. \bibverse{21} Wer der
Gerechtigkeit und Güte nachjagtx-morph=``strongMorph:TH8802'', der
findetx-morph=``strongMorph:TH8799'' Leben, Gerechtigkeit und Ehre.
\bibverse{22} Ein Weiser gewinntx-morph=``strongMorph:TH8804'' die Stadt
der Starken und stürztx-morph=``strongMorph:TH8686'' ihre Macht, darauf
sie sich verläßt. \bibverse{23} Wer seinen Mund und seine Zunge
bewahrtx-morph=``strongMorph:TH8802'', der
bewahrtx-morph=``strongMorph:TH8802'' seine Seele vor Angst.
\bibverse{24} Der stolz und vermessen ist, heißt ein
Spötterx-morph=``strongMorph:TH8801'', der im Zorn Stolz
beweistx-morph=``strongMorph:TH8802''. \bibverse{25} Der Faule
stirbtx-morph=``strongMorph:TH8686'' über seinem Wünschen; denn seine
Hände wollen nichtsx-morph=``strongMorph:TH8765''
tunx-morph=``strongMorph:TH8800''. \bibverse{26} Er
wünschtx-morph=``strongMorph:TH8694'' den ganzen Tag; aber der Gerechte
gibtx-morph=``strongMorph:TH8799'', und versagt
nichtx-morph=``strongMorph:TH8799''. \bibverse{27} Der Gottlosen Opfer
ist ein Greuel; denn es wird in Sünden
geopfertx-morph=``strongMorph:TH8686''. \bibverse{28} Ein lügenhafter
Zeuge wird umkommenx-morph=``strongMorph:TH8799''; aber wer sich sagen
läßtx-morph=``strongMorph:TH8802'', den läßt man auch allezeit wiederum
redenx-morph=``strongMorph:TH8762''. \bibverse{29} Der Gottlose fährt
mit dem Kopf hindurchx-morph=``strongMorph:TH8689''; aber wer fromm ist,
des Weg wird
bestehenx-morph=``strongMorph:TH8799''\textbar x-morph=``strongMorph:TH8675''x-morph=``strongMorph:TH8686''.
\bibverse{30} Es hilft keine Weisheit, kein Verstand, kein Rat wider den
HERRN. \bibverse{31} Rosse werden zum Streittage
bereitetx-morph=``strongMorph:TH8716''; aber der Sieg kommt vom HERRN.

\hypertarget{section-21}{%
\section{22}\label{section-21}}

\bibverse{1} Ein guter Ruf ist köstlicherx-morph=``strongMorph:TH8737''
denn großer Reichtum, und Gunst besser denn Silber und Gold.
\bibverse{2} Reiche und Armex-morph=``strongMorph:TH8802'' müssen
untereinander seinx-morph=``strongMorph:TH8738''; der HERR hat sie alle
gemachtx-morph=``strongMorph:TH8802''. \bibverse{3} Der Kluge
siehtx-morph=``strongMorph:TH8804'' das Unglück und
verbirgtx-morph=``strongMorph:TH8738''\textbar x-morph=``strongMorph:TH8675''
sichx-morph=``strongMorph:TH8799''; die Unverständigen gehen
hindurchx-morph=``strongMorph:TH8804'' und werden
beschädigtx-morph=``strongMorph:TH8738''. \bibverse{4} Wo man leidet in
des HERRN Furcht, da ist Reichtum, Ehre und Leben. \bibverse{5} Stachel
und Stricke sind auf dem Wege des Verkehrten; wer sich aber davon
fernhältx-morph=``strongMorph:TH8799'',
bewahrtx-morph=``strongMorph:TH8802'' sein Leben. \bibverse{6}
`02596'\textbar x-morph=``strongMorph:TH8798'' Wie man einen Knaben
gewöhnt, so läßt er nicht davonx-morph=``strongMorph:TH8799'', wenn er
alt wirdx-morph=``strongMorph:TH8686''. \bibverse{7} Der Reiche
herrschtx-morph=``strongMorph:TH8799'' über die
Armenx-morph=``strongMorph:TH8802''; und wer
borgtx-morph=``strongMorph:TH8801'', ist des
Leihersx-morph=``strongMorph:TH8688'' Knecht. \bibverse{8} Wer Unrecht
sätx-morph=``strongMorph:TH8802'', der wird Mühsal
erntenx-morph=``strongMorph:TH8799'' und wird durch die Rute seiner
Bosheit umkommenx-morph=``strongMorph:TH8799''. \bibverse{9} Ein gütiges
Auge wird gesegnetx-morph=``strongMorph:TH8792''; denn er
gibtx-morph=``strongMorph:TH8804'' von seinem Brot den Armen.
\bibverse{10} Treibe den Spötterx-morph=``strongMorph:TH8801''
ausx-morph=``strongMorph:TH8763'', so geht der Zank
wegx-morph=``strongMorph:TH8799'', so hört
aufx-morph=``strongMorph:TH8799'' Hader und Schmähung. \bibverse{11} Wer
ein treuesx-morph=``strongMorph:TH8675'' Herz und
lieblichex-morph=``strongMorph:TH8802'' Rede hat, des Freund ist der
König. \bibverse{12} Die Augen des HERRN
behütenx-morph=``strongMorph:TH8804'' guten Rat; aber die Worte des
Verächtersx-morph=``strongMorph:TH8802'' verkehrt
erx-morph=``strongMorph:TH8762''. \bibverse{13} Der Faule
sprichtx-morph=``strongMorph:TH8804'': Es ist ein Löwe draußen, ich
möchte erwürgt werdenx-morph=``strongMorph:TH8735'' auf der Gasse.
\bibverse{14} Der Hurenx-morph=``strongMorph:TH8801'' Mund ist eine
Tiefe Grube; wem der HERR ungnädig istx-morph=``strongMorph:TH8803'',
der fällt hineinx-morph=``strongMorph:TH8799''. \bibverse{15} Torheit
stecktx-morph=``strongMorph:TH8803'' dem Knaben im Herzen; aber die Rute
der Zucht wird sie fern von ihm treibenx-morph=``strongMorph:TH8686''.
\bibverse{16} Wer dem Armen Unrecht tutx-morph=``strongMorph:TH8802'',
daß seines Guts viel werdex-morph=``strongMorph:TH8687'', der wird auch
einem Reichen gebenx-morph=``strongMorph:TH8802'', und Mangel haben.
\bibverse{17} Neigex-morph=``strongMorph:TH8685'' deine Ohren und
hörex-morph=``strongMorph:TH8798'' die Worte der Weisen und
nimmx-morph=``strongMorph:TH8799'' zu Herzen meine Lehre. \bibverse{18}
Denn es wird dir sanft tun, wo du sie wirst im Sinne
behaltenx-morph=``strongMorph:TH8799'' und sie werden miteinander durch
deinen Mund wohl geratenx-morph=``strongMorph:TH8735''. \bibverse{19}
Daß deine Hoffnung sei auf den HERRN,
erinnerex-morph=``strongMorph:TH8689'' ich dich an solches heute dir
zugut. \bibverse{20} Habe ich dir's nicht mannigfaltig
vorgeschriebenx-morph=``strongMorph:TH8804'' mit Rat und Lehren,
\bibverse{21} daß ich dir zeigtex-morph=``strongMorph:TH8687'' einen
gewissen Grund der Wahrheit, daß du recht antworten
könntestx-morph=``strongMorph:TH8687'' denen, die dich
sendenx-morph=``strongMorph:TH8802''? \bibverse{22}
Beraubex-morph=``strongMorph:TH8799'' den Armen nicht, ob er wohl arm
ist, und unterdrückex-morph=``strongMorph:TH8762'' den Elenden nicht im
Tor. \bibverse{23} Denn der HERR wird ihre Sache
führenx-morph=``strongMorph:TH8799'' und wird
ihrex-morph=``strongMorph:TH8802'' Untertreter
untertretenx-morph=``strongMorph:TH8804''. \bibverse{24} Geselle
dichx-morph=``strongMorph:TH8691'' nicht zum Zornigen und halte dich
nichtx-morph=``strongMorph:TH8799'' zu einem grimmigen Mann;
\bibverse{25} du möchtest seinen Weg
lernenx-morph=``strongMorph:TH8799'' und an deiner Seele Schaden
nehmenx-morph=``strongMorph:TH8804''. \bibverse{26} Sei nicht bei denen,
die ihre Hand verhaftenx-morph=``strongMorph:TH8802'' und für Schuld
Bürge werdenx-morph=``strongMorph:TH8802''; \bibverse{27} denn wo du es
nicht hast, zu bezahlenx-morph=``strongMorph:TH8763'', so wird man dir
dein Bett unter dir wegnehmenx-morph=``strongMorph:TH8799''.
\bibverse{28} Verrücke nichtx-morph=``strongMorph:TH8686'' die vorigen
Grenzen, die deine Väter gemacht habenx-morph=``strongMorph:TH8804''.
\bibverse{29} Siehstx-morph=``strongMorph:TH8804'' du einen Mann behend
in seinem Geschäft, der wird vor den Königen
stehenx-morph=``strongMorph:TH8691'' und wird nicht
stehenx-morph=``strongMorph:TH8691'' vor den Unedlen.

\hypertarget{section-22}{%
\section{23}\label{section-22}}

\bibverse{1} Wenn du sitzestx-morph=``strongMorph:TH8799'' und
issestx-morph=``strongMorph:TH8800'' mit einem
Herrnx-morph=``strongMorph:TH8802'', sox-morph=``strongMorph:TH8800''
merkex-morph=``strongMorph:TH8799'', wen du vor dir hast, \bibverse{2}
und setzex-morph=``strongMorph:TH8804'' ein Messer an deine Kehle, wenn
du gierig bist. \bibverse{3} Wünsche dir
nichtsx-morph=``strongMorph:TH8691'' von seinen feinen Speisen; denn es
ist falsches Brot. \bibverse{4} Bemühe
dichx-morph=``strongMorph:TH8799'' nicht reich zu
werdenx-morph=``strongMorph:TH8687'' und laß
abx-morph=``strongMorph:TH8798'' von deinen Fündlein. \bibverse{5} Laß
dein Augen nicht
fliegenx-morph=``strongMorph:TH8686''\textbar x-morph=``strongMorph:TH8675''x-morph=``strongMorph:TH8799''
nach dem, was du nicht haben kannst; denn
dasselbex-morph=``strongMorph:TH8800''
machtx-morph=``strongMorph:TH8799'' sich Flügel wie ein Adler und
fliegtx-morph=``strongMorph:TH8799''\textbar x-morph=``strongMorph:TH8675''x-morph=``strongMorph:TH8687''
gen Himmel. \bibverse{6} Ißx-morph=``strongMorph:TH8799'' nicht Brot bei
einem Neidischen und wünschex-morph=``strongMorph:TH8691'' dir von
seinen feinen Speisen nichts. \bibverse{7} Denn wie ein Gespenst
istx-morph=``strongMorph:TH8804'' er inwendig; er
sprichtx-morph=``strongMorph:TH8799'': Ißx-morph=``strongMorph:TH8798''
und trinkx-morph=``strongMorph:TH8798''! und sein Herz ist doch nicht
mit dir. \bibverse{8} Deine Bissen die du gegessen
hattestx-morph=``strongMorph:TH8804'', mußt du
ausspeienx-morph=``strongMorph:TH8686'', und mußt deine freundlichen
Worte verloren habenx-morph=``strongMorph:TH8765''. \bibverse{9}
Redex-morph=``strongMorph:TH8762'' nicht vor des Narren Ohren; denn er
verachtetx-morph=``strongMorph:TH8799'' die Klugheit deiner Rede.
\bibverse{10} Verrücke nichtx-morph=``strongMorph:TH8686'' die vorigen
Grenzen und gehex-morph=``strongMorph:TH8799'' nicht auf der Waisen
Acker. \bibverse{11} Denn ihr Erlöserx-morph=``strongMorph:TH8802'' ist
mächtig; der wird ihre Sache wider dich
ausführenx-morph=``strongMorph:TH8799''. \bibverse{12}
Gibx-morph=``strongMorph:TH8685'' dein Herz zur Zucht und deine Ohren zu
vernünftiger Rede. \bibverse{13} Laß nicht
abx-morph=``strongMorph:TH8799'' den Knaben zu züchtigen; denn wenn du
ihn mit der Rute haustx-morph=``strongMorph:TH8686'', so wird man ihn
nicht tötenx-morph=``strongMorph:TH8799''. \bibverse{14} Du
haustx-morph=``strongMorph:TH8686'' ihn mit der Rute; aber du
errettestx-morph=``strongMorph:TH8686'' seine Seele vom Tode.
\bibverse{15} Mein Sohn, wenn dein Herz weise
istx-morph=``strongMorph:TH8804'', so freut
sichx-morph=``strongMorph:TH8799'' auch mein Herz; \bibverse{16} und
meine Nieren sind frohx-morph=``strongMorph:TH8799'', wenn deine Lippen
redenx-morph=``strongMorph:TH8763'', was recht ist. \bibverse{17} Dein
Herz folge nichtx-morph=``strongMorph:TH8762'' den Sündern, sondern sei
täglich in der Furcht des HERRN. \bibverse{18} Denn es wird dir hernach
gut sein, und dein Warten wird nicht
trügenx-morph=``strongMorph:TH8735''. \bibverse{19}
Hörex-morph=``strongMorph:TH8798'', mein Sohn, und sei
weisex-morph=``strongMorph:TH8798'' und
richtex-morph=``strongMorph:TH8761'' dein Herz in den Weg. \bibverse{20}
Sei nicht unter den Säufernx-morph=``strongMorph:TH8802'' und
Schlemmernx-morph=``strongMorph:TH8802''; \bibverse{21} denn die
Säuferx-morph=``strongMorph:TH8802'' und
Schlemmerx-morph=``strongMorph:TH8802''
verarmenx-morph=``strongMorph:TH8735'', und ein Schläfer muß zerrissene
Kleider tragenx-morph=``strongMorph:TH8686''. \bibverse{22}
Gehorchex-morph=``strongMorph:TH8798'' deinem Vater, der dich gezeugt
hatx-morph=``strongMorph:TH8804'', und
verachtex-morph=``strongMorph:TH8799'' deine Mutter nicht, wenn sie alt
wirdx-morph=``strongMorph:TH8804''. \bibverse{23}
Kaufex-morph=``strongMorph:TH8798'' Wahrheit, und
verkaufex-morph=``strongMorph:TH8799'' sie nicht, Weisheit, Zucht und
Verstand. \bibverse{24} Der Vater eines Gerechten
freutx-morph=``strongMorph:TH8799'' sich; und wer einen Weisen
gezeugtx-morph=``strongMorph:TH8802'' hat, ist fröhlich
darüberx-morph=``strongMorph:TH8799''. \bibverse{25} Laß sich deinen
Vater und deine Mutter freuenx-morph=``strongMorph:TH8799'', und
fröhlich seinx-morph=``strongMorph:TH8799'', die dich geboren
hatx-morph=``strongMorph:TH8802''. \bibverse{26}
Gibx-morph=``strongMorph:TH8798'' mir, mein Sohn, dein Herz, und laß
deinen Augen meine Wege wohl
gefallenx-morph=``strongMorph:TH8799''x-morph=``strongMorph:TH8799''.
\bibverse{27} Denn eine Hurex-morph=``strongMorph:TH8802'' ist eine
tiefe Grube, und eine Ehebrecherin ist ein enger Brunnen. \bibverse{28}
Auch lauert siex-morph=``strongMorph:TH8799'' wie ein Räuber, und die
Frechenx-morph=``strongMorph:TH8802'' unter den Menschen
sammeltx-morph=``strongMorph:TH8686'' sie zu sich. \bibverse{29} Wo ist
Weh? wo ist Leid? wo ist Zankx-morph=``strongMorph:TH8675''? wo ist
Klagen? wo sind Wunden ohne Ursache? wo sind trübe Augen? \bibverse{30}
Wo man beim Wein liegtx-morph=``strongMorph:TH8764'' und
kommtx-morph=``strongMorph:TH8802'', auszusaufen, was eingeschenkt
istx-morph=``strongMorph:TH8800''. \bibverse{31}
Siehex-morph=``strongMorph:TH8799'' den Wein nicht an, daß er so rot
istx-morph=``strongMorph:TH8691'' und im
Glasex-morph=``strongMorph:TH8675'' so schön
stehtx-morph=``strongMorph:TH8799''. Er geht glatt
einx-morph=``strongMorph:TH8691''; \bibverse{32} aber danach
beißtx-morph=``strongMorph:TH8799'' er wie eine Schlange und
stichtx-morph=``strongMorph:TH8686'' wie eine Otter. \bibverse{33} So
werden deine Augen nach andern Weibernx-morph=``strongMorph:TH8801''
sehenx-morph=``strongMorph:TH8799'', und dein Herz wird verkehrte Dinge
redenx-morph=``strongMorph:TH8762'', \bibverse{34} und wirst sein wie
einer, der mitten im Meer schläftx-morph=``strongMorph:TH8802'', und wie
einer schläftx-morph=``strongMorph:TH8802'' oben auf dem Mastbaum.
\bibverse{35} ``Sie schlagen michx-morph=''strongMorph:TH8689``, aber es
tut mir nicht wehx-morph=''strongMorph:TH8804``; sie klopfen
michx-morph=''strongMorph:TH8804``, aber ich fühle es
nichtx-morph=''strongMorph:TH8804``. Wann will ich
aufwachenx-morph=''strongMorph:TH8686``, daß ich's
mehrx-morph=''strongMorph:TH8686"
treibex-morph=``strongMorph:TH8762''?''

\hypertarget{section-23}{%
\section{24}\label{section-23}}

\bibverse{1} Folge nichtx-morph=``strongMorph:TH8762'' bösen Leuten und
wünsche nichtx-morph=``strongMorph:TH8691'', bei ihnen zu sein;
\bibverse{2} denn ihr Herz trachtex-morph=``strongMorph:TH8799'' nach
Schaden, und ihre Lippen ratenx-morph=``strongMorph:TH8762'' zu Unglück.
\bibverse{3} Durch Weisheit wird ein Haus
gebautx-morph=``strongMorph:TH8735'' und durch Verstand
erhaltenx-morph=``strongMorph:TH8709''. \bibverse{4} Durch ordentliches
Haushalten werden die Kammern vollx-morph=``strongMorph:TH8735'' aller
köstlichen, lieblichen Reichtümer. \bibverse{5} Ein weiser Mann ist
stark, und ein vernünftiger Mann ist
mächtigx-morph=``strongMorph:TH8764'' von Kräften. \bibverse{6} Denn mit
Rat muß man Krieg führenx-morph=``strongMorph:TH8799''; und wo viele
Ratgeberx-morph=``strongMorph:TH8802'' sind, da ist der Sieg.
\bibverse{7} Weisheit ist dem Narren zu
hochx-morph=``strongMorph:TH8802''; er darf seinen Mund im Tor nicht
auftunx-morph=``strongMorph:TH8799''. \bibverse{8} Wer sich
vornimmtx-morph=``strongMorph:TH8764'', Böses zu
tunx-morph=``strongMorph:TH8687'', den
heißtx-morph=``strongMorph:TH8799'' man billig einen Erzbösewicht.
\bibverse{9} Des Narren Tücke ist Sünde, und der
Spötterx-morph=``strongMorph:TH8801'' ist ein Greuel vor den Leuten.
\bibverse{10} Der ist nicht starkx-morph=``strongMorph:TH8694'', der in
der Not nicht fest ist. \bibverse{11}
Errettex-morph=``strongMorph:TH8685'' die, so man töten
willx-morph=``strongMorph:TH8803''; und
entziehx-morph=``strongMorph:TH8799'' dich nicht von denen,
diex-morph=``strongMorph:TH8801'' man würgen will. \bibverse{12}
Sprichst dux-morph=``strongMorph:TH8799'': ``Siehe, wir
verstehen'sx-morph=''strongMorph:TH8804" nicht!'' meinst du nicht, der
die Herzen wägtx-morph=``strongMorph:TH8802'', merkt
esx-morph=``strongMorph:TH8799'', und der auf deine Seele
achthatx-morph=``strongMorph:TH8802'', kennt
esx-morph=``strongMorph:TH8799'' und
vergiltx-morph=``strongMorph:TH8689'' dem Menschen nach seinem Werk?
\bibverse{13} Ißx-morph=``strongMorph:TH8798'', mein Sohn, Honig, denn
er ist gut, und Honigseim ist süß in deinem Halse. \bibverse{14}
Alsox-morph=``strongMorph:TH8798'' lerne die Weisheit für deine Seele.
Wo du sie findestx-morph=``strongMorph:TH8804'', so wird's hernach wohl
gehen, und deine Hoffnung wird nicht umsonst
seinx-morph=``strongMorph:TH8735''. \bibverse{15}
Laurex-morph=``strongMorph:TH8799'' nicht als Gottloser auf das Haus des
Gerechten; verstörex-morph=``strongMorph:TH8762'' seine Ruhe nicht.
\bibverse{16} Denn ein Gerechter fälltx-morph=``strongMorph:TH8799''
siebenmal und steht wieder aufx-morph=``strongMorph:TH8804''; aber die
Gottlosen versinkenx-morph=``strongMorph:TH8735'' im Unglück.
\bibverse{17} Freue dichx-morph=``strongMorph:TH8799'' des
Fallesx-morph=``strongMorph:TH8800'' deines
Feindesx-morph=``strongMorph:TH8802'' nicht, und dein Herz sei nicht
frohx-morph=``strongMorph:TH8799'' über seinem
Unglückx-morph=``strongMorph:TH8736''; \bibverse{18} der HERR möchte es
sehenx-morph=``strongMorph:TH8799'', und es möchte ihm
übelx-morph=``strongMorph:TH8804'' gefallen und er seine Zorn von ihm
wendenx-morph=``strongMorph:TH8689''. \bibverse{19} Erzürne
dichx-morph=``strongMorph:TH8691'' nicht über die
Bösenx-morph=``strongMorph:TH8688'' und eifere
nichtx-morph=``strongMorph:TH8762'' über die Gottlosen. \bibverse{20}
Denn der Böse hat nichts zu hoffen, und die Leuchte der Gottlosen wird
verlöschenx-morph=``strongMorph:TH8799''. \bibverse{21} Mein Kind,
fürchtex-morph=``strongMorph:TH8798'' den HERRN und den König und menge
dichx-morph=``strongMorph:TH8691'' nicht unter die
Aufrührerx-morph=``strongMorph:TH8802''. \bibverse{22} Denn ihr
Verderben wird plötzlich entstehenx-morph=``strongMorph:TH8799''; und
wer weißx-morph=``strongMorph:TH8802'', wann beider Unglück kommt?
\bibverse{23} Dies sind auch Worte von Weisen. Die Person ansehen im
Gericht istx-morph=``strongMorph:TH8687'' nicht gut. \bibverse{24} Wer
zum Gottlosen sprichtx-morph=``strongMorph:TH8802'': ``Du bist fromm'',
dem fluchenx-morph=``strongMorph:TH8799'' die Leute, und das Volk haßt
ihnx-morph=``strongMorph:TH8799''. \bibverse{25} Welche aber
strafenx-morph=``strongMorph:TH8688'', die gefallen
wohlx-morph=``strongMorph:TH8799'', und kommt ein reicher Segen auf
siex-morph=``strongMorph:TH8799''. \bibverse{26} Eine richtige
Antwortx-morph=``strongMorph:TH8688'' ist wie ein lieblicher
Kußx-morph=``strongMorph:TH8799''. \bibverse{27}
Richtex-morph=``strongMorph:TH8685'' draußen dein Geschäft aus und
bearbeitex-morph=``strongMorph:TH8761'' deinen Acker; darnach
bauex-morph=``strongMorph:TH8804'' dein Haus. \bibverse{28} Sei nicht
Zeuge ohne Ursache wider deinen Nächsten und betrüge
nichtx-morph=``strongMorph:TH8765'' mit deinem Munde. \bibverse{29}
Sprichx-morph=``strongMorph:TH8799'' nicht: ``Wie man mir
tutx-morph=''strongMorph:TH8799``, so will ich wieder
tunx-morph=''strongMorph:TH8804" und einem jeglichen sein Werk
vergeltenx-morph=``strongMorph:TH8686''.'' \bibverse{30} Ich
gingx-morph=``strongMorph:TH8804'' am Acker des Faulen vorüber und am
Weinberg des Narren; \bibverse{31} und siehe, da waren
eitelx-morph=``strongMorph:TH8804'' Nesseln darauf, und er stand
vollx-morph=``strongMorph:TH8795'' Disteln, und die Mauer war
eingefallenx-morph=``strongMorph:TH8738''. \bibverse{32} Da ich das
sahx-morph=``strongMorph:TH8799'', nahm
ich'sx-morph=``strongMorph:TH8799'' zu Herzen und
schautex-morph=``strongMorph:TH8804'' und lernte
daranx-morph=``strongMorph:TH8804''. \bibverse{33} Du willst ein wenig
schlafen und ein wenig schlummern und ein wenig deine Hände zusammentun,
daß du ruhestx-morph=``strongMorph:TH8800'': \bibverse{34} aber es wird
dir deine Armut kommenx-morph=``strongMorph:TH8804'' wie ein
Wandererx-morph=``strongMorph:TH8693'' und dein Mangel wie ein
gewappneter Mann.

\hypertarget{section-24}{%
\section{25}\label{section-24}}

\bibverse{1} Dies sind auch Sprüche Salomos, die hinzugesetzt
habenx-morph=``strongMorph:TH8689'' die Männer Hiskias, des Königs in
Juda. \bibverse{2} Es ist Gottes Ehre, eine Sache
verbergenx-morph=``strongMorph:TH8687''; aber der Könige Ehre ist's,
eine Sache zu erforschenx-morph=``strongMorph:TH8800''. \bibverse{3} Der
Himmel ist hoch und die Erde tief; aber der Könige Herz ist
unerforschlich. \bibverse{4} Man tuex-morph=``strongMorph:TH8800'' den
Schaum vom Silber, so wird ein reinesx-morph=``strongMorph:TH8802''
Gefäß darausx-morph=``strongMorph:TH8799''. \bibverse{5} Man tue den
Gottlosen hinwegx-morph=``strongMorph:TH8800'' vor dem König, so wird
sein Thron mit Gerechtigkeit befestigtx-morph=``strongMorph:TH8735''.
\bibverse{6} Prangex-morph=``strongMorph:TH8691'' nicht vor dem König
und trittx-morph=``strongMorph:TH8799'' nicht an den Ort der Großen.
\bibverse{7} Denn es ist dir besser, daß man zu dir
sagex-morph=``strongMorph:TH8800'': Tritt hier
heraufx-morph=``strongMorph:TH8798''! als daß du vor dem Fürsten
erniedrigtx-morph=``strongMorph:TH8687'' wirst, daß es deine Augen sehen
müssenx-morph=``strongMorph:TH8804''. \bibverse{8} Fahre nicht bald
herausx-morph=``strongMorph:TH8799'', zu
zankenx-morph=``strongMorph:TH8800''; denn was willst du hernach
machenx-morph=``strongMorph:TH8799'', wenn dich dein Nächster beschämt
hatx-morph=``strongMorph:TH8687''? \bibverse{9}
Führex-morph=``strongMorph:TH8798'' deine Sache mit deinem Nächsten, und
offenbarex-morph=``strongMorph:TH8762'' nicht eines andern Heimlichkeit,
\bibverse{10} auf daß nicht übel von dir
sprechex-morph=``strongMorph:TH8762'', der es
hörtx-morph=``strongMorph:TH8802'', und dein böses Gerücht nimmer
ablassex-morph=``strongMorph:TH8799''. \bibverse{11} Ein Wort
geredetx-morph=``strongMorph:TH8803'' zu seiner
Zeitx-morph=``strongMorph:TH8675'', ist wie goldene Äpfel auf silbernen
Schalen. \bibverse{12} Wer einem Weisen
gehorchtx-morph=``strongMorph:TH8802'', der ihn
straftx-morph=``strongMorph:TH8688'', das ist wie ein goldenes Stirnband
und goldenes Halsband. \bibverse{13} Wie die Kühle des Schnees zur Zeit
der Ernte, so ist ein treuerx-morph=``strongMorph:TH8737'' Bote dem, der
ihn gesandt hatx-morph=``strongMorph:TH8802'', und
labtx-morph=``strongMorph:TH8686'' seines Herrn Seele. \bibverse{14} Wer
viel versprichtx-morph=``strongMorph:TH8693'' und hält nicht, der ist
wie Wolken und Wind ohne Regen. \bibverse{15} Durch Geduld wird ein
Fürst versöhntx-morph=``strongMorph:TH8792'', und eine linde Zunge
brichtx-morph=``strongMorph:TH8799'' die Härtigkeit. \bibverse{16}
Findestx-morph=``strongMorph:TH8804'' du Honig, so
ißx-morph=``strongMorph:TH8798'' davon, so viel dir genug ist, daß du
nicht zu satt wirstx-morph=``strongMorph:TH8799'' und speiest ihn
ausx-morph=``strongMorph:TH8689''. \bibverse{17}
Entziehx-morph=``strongMorph:TH8685'' deinen Fuß vom Hause deines
Nächsten; er möchte dein überdrüssigx-morph=``strongMorph:TH8799'' und
dir gram werdenx-morph=``strongMorph:TH8804''. \bibverse{18} Wer wider
seinen Nächsten falsch Zeugnis redetx-morph=``strongMorph:TH8802'', der
ist ein Spieß, Schwert und scharferx-morph=``strongMorph:TH8802'' Pfeil.
\bibverse{19} Die Hoffnung auf einen
Treulosenx-morph=``strongMorph:TH8802'' zur Zeit der Not ist wie ein
fauler Zahn und gleitender Fuß. \bibverse{20} Wer einem betrübten Herzen
Lieder singtx-morph=``strongMorph:TH8802'', das ist, wie wenn einer das
Kleid ablegtx-morph=``strongMorph:TH8688'' am kalten Tage, und wie Essig
auf der Kreide. \bibverse{21} Hungert deinen
Feindx-morph=``strongMorph:TH8802'', so
speisex-morph=``strongMorph:TH8685'' ihn mit Brot; dürstet ihn, so
tränkex-morph=``strongMorph:TH8685'' ihn mit Wasser. \bibverse{22} Denn
du wirst feurige Kohlen auf sein Haupt
häufenx-morph=``strongMorph:TH8802'', und der HERR wird dir's
vergeltenx-morph=``strongMorph:TH8762''. \bibverse{23} Der Nordwind
bringtx-morph=``strongMorph:TH8787'' Ungewitter, und die heimliche Zunge
macht sauresx-morph=``strongMorph:TH8737'' Angesicht. \bibverse{24} Es
ist besser, im Winkel auf dem Dach sitzenx-morph=``strongMorph:TH8800''
denn bei einem zänkischenx-morph=``strongMorph:TH8675'' Weibe in einem
Haus beisammen. \bibverse{25} Eine gute Botschaft aus fernen Landen ist
wie kalt Wasser einer durstigen Seele. \bibverse{26} Ein Gerechter, der
vor einem Gottlosen fälltx-morph=``strongMorph:TH8801'', ist wie ein
getrübterx-morph=``strongMorph:TH8737'' Brunnen und eine
verderbtex-morph=``strongMorph:TH8716'' Quelle. \bibverse{27} Wer
zuvielx-morph=``strongMorph:TH8687'' Honig
ißtx-morph=``strongMorph:TH8800'', das ist nicht gut; und wer schwere
Dinge erforscht, dem wird's zu schwer. \bibverse{28} Ein Mann, der
seinen Geist nicht halten kann, ist wie eine
offenex-morph=``strongMorph:TH8803'' Stadt ohne Mauern.

\hypertarget{section-25}{%
\section{26}\label{section-25}}

\bibverse{1} Wie der Schnee im Sommer und Regen in der Ernte, also reimt
sich dem Narren die Ehre nicht. \bibverse{2} Wie ein Vogel
dahinfährtx-morph=``strongMorph:TH8800'' und eine Schwalbe
fliegtx-morph=``strongMorph:TH8800'', also ein unverdienter Fluch trifft
nichtx-morph=``strongMorph:TH8799''. \bibverse{3} Dem Roß eine Geißel
und dem Esel einen Zaum und dem Narren eine Rute auf den Rücken!
\bibverse{4} Antwortex-morph=``strongMorph:TH8799'' dem Narren nicht
nach seiner Narrheit, daß du ihm nicht auch gleich
werdestx-morph=``strongMorph:TH8799''. \bibverse{5}
Antwortex-morph=``strongMorph:TH8798'' aber dem Narren nach seiner
Narrheit, daß er sich nicht weise lasse dünken. \bibverse{6} Wer eine
Sache durch einen törichten Boten
ausrichtetx-morph=``strongMorph:TH8802'', der ist
wiex-morph=``strongMorph:TH8764'' ein Lahmer an den Füßen und
nimmtx-morph=``strongMorph:TH8802'' Schaden. \bibverse{7} Wie einem
Krüppel das Tanzenx-morph=``strongMorph:TH8804'', also steht den Narren
an, von Weisheit zu reden. \bibverse{8} Wer einem Narren Ehre
antutx-morph=``strongMorph:TH8802'', das ist, als wenn einer einen edlen
Stein auf den Rabenstein
würfex-morph=``strongMorph:TH8675''x-morph=``strongMorph:TH8800''.
\bibverse{9} Ein Spruch in eines Narren Mund ist
wiex-morph=``strongMorph:TH8804'' ein Dornzweig, der in eines Trunkenen
Hand sticht. \bibverse{10} Ein guter Meister macht ein Ding
rechtx-morph=``strongMorph:TH8789''; aber wer einen Stümper
dingtx-morph=``strongMorph:TH8802'', dem wird's
verderbtx-morph=``strongMorph:TH8802''. \bibverse{11} Wie ein Hund sein
Gespeites wieder frißtx-morph=``strongMorph:TH8804'', also
istx-morph=``strongMorph:TH8802'' der Narr, der seine Narrheit wieder
treibt. \bibverse{12} Wenn du einen
siehstx-morph=``strongMorph:TH8804'', der sich weise dünkt, da ist an
einem Narren mehr Hoffnung denn an ihm. \bibverse{13} Der Faule
sprichtx-morph=``strongMorph:TH8804'': Es ist ein junger Löwe auf dem
Wege und ein Löwe auf den Gassen. \bibverse{14} Ein Fauler wendet
sichx-morph=``strongMorph:TH8735'' im Bette wie die Tür in der Angel.
\bibverse{15} Der Faule verbirgtx-morph=``strongMorph:TH8804'' seine
Hand in dem Topf, und wird ihm sauerx-morph=``strongMorph:TH8738'', daß
er sie zum Munde bringex-morph=``strongMorph:TH8687''. \bibverse{16} Ein
Fauler dünkt sich weiser denn sieben, die da Sitten
lehrenx-morph=``strongMorph:TH8688''. \bibverse{17} Wer
vorgehtx-morph=``strongMorph:TH8802'' und sich
mengtx-morph=``strongMorph:TH8693'' in fremden Hader, der ist wie einer,
der den Hund bei den Ohren zwacktx-morph=``strongMorph:TH8688''.
\bibverse{18} Wie ein Unsinnigerx-morph=``strongMorph:TH8700'' mit
Geschoß und Pfeilen schießtx-morph=``strongMorph:TH8802'' und tötet,
\bibverse{19} also tut ein falscherx-morph=``strongMorph:TH8765'' Mensch
mit seinem Nächsten und sprichtx-morph=``strongMorph:TH8804'' danach:
Ich habe gescherztx-morph=``strongMorph:TH8764''. \bibverse{20} Wenn
nimmer Holz da ist, so verlischtx-morph=``strongMorph:TH8799'' das
Feuer; und wenn der Verleumder weg ist, so hört der Hader
aufx-morph=``strongMorph:TH8799''. \bibverse{21} Wie die Kohlen eine
Glut und Holz ein Feuer, also facht ein
zänkischerx-morph=``strongMorph:TH8675'' Mann Hader
anx-morph=``strongMorph:TH8771''. \bibverse{22} Die Worte des
Verleumders sind wie Schlägex-morph=``strongMorph:TH8693'', und sie
gehenx-morph=``strongMorph:TH8804'' durchs Herz. \bibverse{23}
Brünstigex-morph=``strongMorph:TH8801'' Lippen und ein böses Herz ist
wie eine Scherbe, mit Silberschaum
überzogenx-morph=``strongMorph:TH8794''. \bibverse{24} Der
Feindx-morph=``strongMorph:TH8802'' verstellt
sichx-morph=``strongMorph:TH8735'' mit seiner Rede, und im Herzen
istx-morph=``strongMorph:TH8799'' er falsch. \bibverse{25} Wenn er seine
Stimme holdselig machtx-morph=``strongMorph:TH8762'', so
glaubex-morph=``strongMorph:TH8686'' ihm nicht; denn es sind sieben
Greuel in seinem Herzen. \bibverse{26} Wer den Haß heimlich
hältx-morph=``strongMorph:TH8691'', Schaden zu tun, des Bosheit wird vor
der Gemeinde offenbar werdenx-morph=``strongMorph:TH8735''.
\bibverse{27} Wer eine Grube machtx-morph=``strongMorph:TH8802'', der
wird hineinfallenx-morph=``strongMorph:TH8799''; und wer einen Stein
wälztx-morph=``strongMorph:TH8802'', auf den wird er
zurückkommenx-morph=``strongMorph:TH8799''. \bibverse{28} Eine falsche
Zunge haßtx-morph=``strongMorph:TH8799'' den, der sie straft; und ein
Heuchelmaul richtetx-morph=``strongMorph:TH8799'' Verderben an.

\hypertarget{section-26}{%
\section{27}\label{section-26}}

\bibverse{1} Rühme dichx-morph=``strongMorph:TH8691'' nicht des
morgenden Tages; denn du weißtx-morph=``strongMorph:TH8799'' nicht, was
heute sich begeben magx-morph=``strongMorph:TH8799''. \bibverse{2} Laß
dich einen andernx-morph=``strongMorph:TH8801''
lobenx-morph=``strongMorph:TH8762'', und nicht deinen Mund, einen
Fremden, und nicht deine eigenen Lippen. \bibverse{3} Stein ist schwer
und Sand ist Last; aber des Narren Zorn ist schwerer denn die beiden.
\bibverse{4} Zorn ist ein wütig Ding, und Grimm ist ungestüm; aber wer
kann vor dem Neid bestehenx-morph=``strongMorph:TH8799''? \bibverse{5}
Offenex-morph=``strongMorph:TH8794'' Strafe ist besser denn
heimlichex-morph=``strongMorph:TH8794'' Liebe. \bibverse{6} Die Schläge
des Liebhabersx-morph=``strongMorph:TH8802'' meinen's recht
gutx-morph=``strongMorph:TH8737''; aber die Küsse des
Hassersx-morph=``strongMorph:TH8802'' sind gar zu
reichlichx-morph=``strongMorph:TH8737''. \bibverse{7} Eine satte Seele
zertrittx-morph=``strongMorph:TH8799'' wohl Honigseim; aber einer
hungrigen Seele ist alles Bittere süß. \bibverse{8} Wie ein Vogel, der
aus seinem Nest weichtx-morph=``strongMorph:TH8802'', also ist, wer von
seiner Stätte weichtx-morph=``strongMorph:TH8802''. \bibverse{9} Das
Herz erfreutx-morph=``strongMorph:TH8762'' sich an Salbe und Räuchwerk;
aber ein Freund ist lieblich um Rats willen der Seele. \bibverse{10}
Deinen Freund und deines Vaters Freund
verlaßx-morph=``strongMorph:TH8799'' nicht, und
gehex-morph=``strongMorph:TH8799'' nicht ins Haus deines Bruders, wenn
dir's übel geht; denn dein Nachbar in der Nähe ist besser als dein
Bruder in der Ferne. \bibverse{11} Sei
weisex-morph=``strongMorph:TH8798'', mein Sohn, so
freutx-morph=``strongMorph:TH8761'' sich mein Herz, so will ich
antwortenx-morph=``strongMorph:TH8686'' dem, der mich
schmähtx-morph=``strongMorph:TH8802''. \bibverse{12} Ein Kluger
siehtx-morph=``strongMorph:TH8804'' das Unglück und verbirgt
sichx-morph=``strongMorph:TH8738''; aber die Unverständigen gehen
hindurchx-morph=``strongMorph:TH8804'' und leiden
Schadenx-morph=``strongMorph:TH8738''. \bibverse{13}
Nimmx-morph=``strongMorph:TH8798'' dem sein Kleid, der für einen
andernx-morph=``strongMorph:TH8801'' Bürge
wirdx-morph=``strongMorph:TH8804'', und
pfändex-morph=``strongMorph:TH8798'' ihn um der Fremden willen.
\bibverse{14} Wenn einer seinen Nächsten des Morgens
frühx-morph=``strongMorph:TH8687'' mit lauter Stimme
segnetx-morph=``strongMorph:TH8764'', das wird ihm für einen Fluch
gerechnetx-morph=``strongMorph:TH8735''. \bibverse{15} Ein
zänkischesx-morph=``strongMorph:TH8675'' Weib und stetiges Triefen,
wenn's sehrx-morph=``strongMorph:TH8802'' regnet, werden wohl
miteinander verglichenx-morph=``strongMorph:TH8739''. \bibverse{16} Wer
sie aufhältx-morph=``strongMorph:TH8802'', der
hältx-morph=``strongMorph:TH8804'' den Wind und will das Öl mit der Hand
fassenx-morph=``strongMorph:TH8799''. \bibverse{17} Ein Messer
wetztx-morph=``strongMorph:TH8799'' das andere
undx-morph=``strongMorph:TH8686'' ein Mann den andern. \bibverse{18} Wer
seinen Feigenbaum bewahrtx-morph=``strongMorph:TH8802'', der
ißtx-morph=``strongMorph:TH8799'' Früchte davon; und wer seinen Herrn
bewahrtx-morph=``strongMorph:TH8802'', wird
geehrtx-morph=``strongMorph:TH8792''. \bibverse{19} Wie das Spiegelbild
im Wasser ist gegenüber dem Angesicht, also ist eines Menschen Herz
gegenüber dem andern. \bibverse{20} Hölle und
Abgrundx-morph=``strongMorph:TH8675'' werden nimmer
vollx-morph=``strongMorph:TH8799'', und der Menschen Augen sind auch
unersättlichx-morph=``strongMorph:TH8799''. \bibverse{21} Ein Mann wird
durch den Mund des, der ihn lobt, bewährt wie Silber im Tiegel und das
Gold im Ofen. \bibverse{22} Wenn du den Narren im Mörser
zerstießestx-morph=``strongMorph:TH8799'' mit dem Stämpel wie Grütze, so
ließe doch seine Narrheit nicht von ihmx-morph=``strongMorph:TH8799''.
\bibverse{23} Auf deine Schafe habex-morph=``strongMorph:TH8799''
achtx-morph=``strongMorph:TH8800'' und
nimmx-morph=``strongMorph:TH8798'' dich deiner Herden an. \bibverse{24}
Denn Gut währt nicht ewiglich, und die Krone währt nicht für und für.
\bibverse{25} Das Heu ist weggeführtx-morph=``strongMorph:TH8804'', und
wiederum ist Gras dax-morph=``strongMorph:TH8738'' und wird Kraut auf
den Bergen gesammeltx-morph=``strongMorph:TH8738''. \bibverse{26} Die
Lämmer kleiden dich und die Böcke geben dir das Geld, einen Acker zu
kaufen. \bibverse{27} Du hast Ziegenmilch genug zu deiner Speise, zur
Speise deines Hauses und zur Nahrung deiner Dirnen.

\hypertarget{section-27}{%
\section{28}\label{section-27}}

\bibverse{1} Der Gottlose fliehtx-morph=``strongMorph:TH8804'', und
niemand jagtx-morph=``strongMorph:TH8802'' ihn; der Gerechte aber ist
getrostx-morph=``strongMorph:TH8799'' wie ein junger Löwe. \bibverse{2}
Um des Landes Sünde willen werden viel Änderungen der Fürstentümer; aber
um der Leute willen, die verständigx-morph=``strongMorph:TH8688'' und
vernünftigx-morph=``strongMorph:TH8802'' sind, bleiben sie
langex-morph=``strongMorph:TH8686''. \bibverse{3} Ein
armerx-morph=``strongMorph:TH8802'' Mann, der die Geringen
bedrücktx-morph=``strongMorph:TH8802'', ist wie ein
Meltaux-morph=``strongMorph:TH8802'', der die Frucht verdirbt.
\bibverse{4} Die das Gesetz verlassenx-morph=``strongMorph:TH8802'',
lobenx-morph=``strongMorph:TH8762'' den Gottlosen; die es aber
bewahrenx-morph=``strongMorph:TH8802'', sind
unwilligx-morph=``strongMorph:TH8691'' auf sie. \bibverse{5} Böse Leute
merkenx-morph=``strongMorph:TH8799'' nicht aufs Recht; die aber nach dem
HERRN fragenx-morph=``strongMorph:TH8764'',
merkenx-morph=``strongMorph:TH8799'' auf alles. \bibverse{6} Es ist
besser ein Armerx-morph=``strongMorph:TH8802'', der in seiner
Frömmigkeit gehtx-morph=``strongMorph:TH8802'', denn ein Reicher, der in
verkehrten Wegen geht. \bibverse{7} Wer das Gesetz
bewahrtx-morph=``strongMorph:TH8802'', ist ein
verständigesx-morph=``strongMorph:TH8688'' Kind; wer aber der
Schlemmerx-morph=``strongMorph:TH8802''
Gesellex-morph=``strongMorph:TH8802'' ist,
schändetx-morph=``strongMorph:TH8686'' seinen Vater. \bibverse{8} Wer
sein Gut mehrtx-morph=``strongMorph:TH8688'' mit Wucher und Zins, der
sammeltx-morph=``strongMorph:TH8762'' es für den, der sich der Armen
erbarmtx-morph=``strongMorph:TH8802''. \bibverse{9} Wer sein Ohr
abwendetx-morph=``strongMorph:TH8688'', das Gesetz zu
hörenx-morph=``strongMorph:TH8800'', des Gebet ist ein Greuel.
\bibverse{10} Wer die Frommen verführtx-morph=``strongMorph:TH8688'' auf
bösem Wege, der wird in seine Grube
fallenx-morph=``strongMorph:TH8799''; aber die Frommen werden Gutes
ererbenx-morph=``strongMorph:TH8799''. \bibverse{11} Ein Reicher dünkt
sich, weise zu sein; aber ein verständigerx-morph=``strongMorph:TH8688''
Armer durchschautx-morph=``strongMorph:TH8799'' ihn. \bibverse{12} Wenn
die Gerechten Oberhand habenx-morph=``strongMorph:TH8800'', so geht's
sehr fein zu; wenn aber Gottlose
aufkommenx-morph=``strongMorph:TH8800'', wendet
sich'sx-morph=``strongMorph:TH8792'' unter den Leuten. \bibverse{13} Wer
seine Missetat leugnetx-morph=``strongMorph:TH8764'', dem wird's nicht
gelingenx-morph=``strongMorph:TH8686''; wer sie aber
bekenntx-morph=``strongMorph:TH8688'' und
läßtx-morph=``strongMorph:TH8802'', der wird Barmherzigkeit
erlangenx-morph=``strongMorph:TH8792''. \bibverse{14} Wohl dem, der sich
allewege fürchtetx-morph=``strongMorph:TH8764''; wer aber sein Herz
verhärtetx-morph=``strongMorph:TH8688'', wird in Unglück
fallenx-morph=``strongMorph:TH8799''. \bibverse{15} Ein Gottloser, der
über ein armes Volk regiertx-morph=``strongMorph:TH8802'', das ist ein
brüllenderx-morph=``strongMorph:TH8802'' Löwe und
gierigerx-morph=``strongMorph:TH8802'' Bär. \bibverse{16} Wenn ein Fürst
ohne Verstand ist, so geschieht viel Unrecht; wer aber den Geiz
haßtx-morph=``strongMorph:TH8802'', der wird
langex-morph=``strongMorph:TH8686'' leben. \bibverse{17} Ein Mensch, der
am Blut einer Seele schuldig istx-morph=``strongMorph:TH8803'', der wird
flüchtig seinx-morph=``strongMorph:TH8799'' bis zur Grube, und
niemandx-morph=``strongMorph:TH8799'' halte ihn auf. \bibverse{18} Wer
fromm einhergehtx-morph=``strongMorph:TH8802'', dem wird
geholfenx-morph=``strongMorph:TH8735''; wer aber verkehrtes Weges
istx-morph=``strongMorph:TH8737'', wird auf einmal
fallenx-morph=``strongMorph:TH8799''. \bibverse{19} Wer seinen Acker
bautx-morph=``strongMorph:TH8802'', wird Brot genug
habenx-morph=``strongMorph:TH8799''; wer aber dem Müßiggang
nachgehtx-morph=``strongMorph:TH8764'', wird Armut genug haben.
\bibverse{20} Ein treuer Mann wird viel gesegnet; wer aber
eiltx-morph=``strongMorph:TH8801'', reich zu
werdenx-morph=``strongMorph:TH8687'', wird nicht unschuldig
bleibenx-morph=``strongMorph:TH8735''. \bibverse{21} Person
ansehenx-morph=``strongMorph:TH8687'' ist nicht gut; und mancher tut
übelx-morph=``strongMorph:TH8799'' auch wohl um ein Stück Brot.
\bibverse{22} Wer eiltx-morph=``strongMorph:TH8737'' zum Reichtum und
ist neidisch, der weißx-morph=``strongMorph:TH8799'' nicht, daß Mangel
ihm begegnen wirdx-morph=``strongMorph:TH8799''. \bibverse{23} Wer einen
Menschen straftx-morph=``strongMorph:TH8688'', wird hernach Gunst
findenx-morph=``strongMorph:TH8799'', mehr denn der da
heucheltx-morph=``strongMorph:TH8688''. \bibverse{24} Wer seinem Vater
oder seiner Mutter etwas nimmtx-morph=``strongMorph:TH8802'' und
sprichtx-morph=``strongMorph:TH8802'', es sei nicht Sünde, der ist des
Verderbersx-morph=``strongMorph:TH8688'' Geselle. \bibverse{25} Ein
Stolzer erwecktx-morph=``strongMorph:TH8762'' Zank; wer aber auf den
HERRN sich verläßtx-morph=``strongMorph:TH8802'', wird
gelabtx-morph=``strongMorph:TH8792''. \bibverse{26} Wer sich auf sein
Herz verläßtx-morph=``strongMorph:TH8802'', ist ein Narr; wer aber mit
Weisheit gehtx-morph=``strongMorph:TH8802'', wird
entrinnenx-morph=``strongMorph:TH8735''. \bibverse{27} Wer dem
Armenx-morph=``strongMorph:TH8802'' gibtx-morph=``strongMorph:TH8802'',
dem wird nichts mangeln; wer aber seine Augen
abwendetx-morph=``strongMorph:TH8688'', der wird viel verflucht.
\bibverse{28} Wenn die Gottlosen
aufkommenx-morph=``strongMorph:TH8800'', so verbergen
sichx-morph=``strongMorph:TH8735'' die Leute; wenn sie aber
umkommenx-morph=``strongMorph:TH8800'', werden der Gerechten
vielx-morph=``strongMorph:TH8799''.

\hypertarget{section-28}{%
\section{29}\label{section-28}}

\bibverse{1} Wer wider die Strafe
halsstarrigx-morph=``strongMorph:TH8688'' ist, der wird plötzlich
verderbenx-morph=``strongMorph:TH8735'' ohne alle Hilfe. \bibverse{2}
Wenn der Gerechten viel sindx-morph=``strongMorph:TH8800'', freut
sichx-morph=``strongMorph:TH8799'' das Volk; wenn aber der Gottlose
herrschtx-morph=``strongMorph:TH8800'',
seufztx-morph=``strongMorph:TH8735'' das Volk. \bibverse{3} Wer Weisheit
liebtx-morph=``strongMorph:TH8802'',
erfreutx-morph=``strongMorph:TH8762'' seinen Vater; wer aber
mitx-morph=``strongMorph:TH8802'' Hurenx-morph=``strongMorph:TH8802''
umgeht, kommt umx-morph=``strongMorph:TH8762'' sein Gut. \bibverse{4}
Ein König richtet das Land aufx-morph=``strongMorph:TH8686'' durchs
Recht; ein geiziger aber verderbtx-morph=``strongMorph:TH8799'' es.
\bibverse{5} Wer mit seinem Nächsten
heucheltx-morph=``strongMorph:TH8688'', der
breitetx-morph=``strongMorph:TH8802'' ein Netz aus für seine Tritte.
\bibverse{6} Wenn ein Böser sündigt, verstrickt er sich selbst; aber ein
Gerechter freut sichx-morph=``strongMorph:TH8799'' und hat Wonne.
\bibverse{7} Der Gerechte erkenntx-morph=``strongMorph:TH8802'' die
Sache der Armen; der Gottlose achtetx-morph=``strongMorph:TH8799'' keine
Vernunft. \bibverse{8} Die Spötter bringen
frechx-morph=``strongMorph:TH8686'' eine Stadt in Aufruhr; aber die
Weisen stillenx-morph=``strongMorph:TH8686'' den Zorn. \bibverse{9} Wenn
ein Weiser mit einem Narren zu rechten
kommtx-morph=``strongMorph:TH8737'', er
zürnex-morph=``strongMorph:TH8804'' oder
lachex-morph=``strongMorph:TH8804'', so hat er nicht Ruhe. \bibverse{10}
Die Blutgierigen hassenx-morph=``strongMorph:TH8799'' den Frommen; aber
die Gerechten suchenx-morph=``strongMorph:TH8762'' sein Heil.
\bibverse{11} Ein Narr schüttet seinen Geist ganz
ausx-morph=``strongMorph:TH8686''; aber ein Weiser
hältx-morph=``strongMorph:TH8762'' an sich. \bibverse{12} Ein
Herrx-morph=``strongMorph:TH8802'', der zu Lügen Lust
hatx-morph=``strongMorph:TH8688'', des
Dienerx-morph=``strongMorph:TH8764'' sind alle gottlos. \bibverse{13}
Armex-morph=``strongMorph:TH8802'' und Reiche begegnen
einanderx-morph=``strongMorph:TH8738'': beider Augen
erleuchtetx-morph=``strongMorph:TH8688'' der HERR. \bibverse{14} Ein
König, der die Armen treulich richtetx-morph=``strongMorph:TH8802'', des
Thron wird ewig bestehenx-morph=``strongMorph:TH8735''. \bibverse{15}
Rute und Strafe gibtx-morph=``strongMorph:TH8799'' Weisheit; aber ein
Knabe, sich selbst überlassenx-morph=``strongMorph:TH8794'', macht
seiner Mutter Schandex-morph=``strongMorph:TH8688''. \bibverse{16} Wo
vielex-morph=``strongMorph:TH8800'' Gottlose sind, da
sindx-morph=``strongMorph:TH8799'' viel Sünden; aber die Gerechten
werden ihren Fall erlebenx-morph=``strongMorph:TH8799''. \bibverse{17}
Züchtigex-morph=``strongMorph:TH8761'' deinen Sohn, so wird er dich
ergötzenx-morph=``strongMorph:TH8686'' und wird deiner Seele sanft
tunx-morph=``strongMorph:TH8799''. \bibverse{18} Wo keine Weissagung
ist, wird das Volk wild und wüstx-morph=``strongMorph:TH8735''; wohl
aber dem, der das Gesetz handhabtx-morph=``strongMorph:TH8802''!
\bibverse{19} Ein Knecht läßt sich mit Worten nicht
züchtigenx-morph=``strongMorph:TH8735''; denn ob er sie gleich
verstehtx-morph=``strongMorph:TH8799'', nimmt er sich's doch nicht an.
\bibverse{20} Siehstx-morph=``strongMorph:TH8804'' du einen, der
schnellx-morph=``strongMorph:TH8801'' ist zu reden, da ist am Narren
mehr Hoffnung denn an ihm. \bibverse{21} Wenn ein Knecht von Jugend auf
zärtlich gehaltenx-morph=``strongMorph:TH8764'' wird, so will er darnach
ein Junker sein. \bibverse{22} Ein zorniger Mann richtet Hader
anx-morph=``strongMorph:TH8762'', und ein Grimmiger tut viel Sünde.
\bibverse{23} Die Hoffart des Menschen wird ihn
stürzenx-morph=``strongMorph:TH8686''; aber der Demütige wird Ehre
empfangenx-morph=``strongMorph:TH8799''. \bibverse{24} Wer mit Dieben
teilhatx-morph=``strongMorph:TH8802'', den Fluch aussprechen
hörtx-morph=``strongMorph:TH8799'', und sagt's nicht
anx-morph=``strongMorph:TH8686'', der haßtx-morph=``strongMorph:TH8802''
sein Leben. \bibverse{25} Vor Menschen sich scheuen
bringtx-morph=``strongMorph:TH8799'' zu Fall; wer sich aber auf den
HERRN verläßtx-morph=``strongMorph:TH8802'', wird
beschütztx-morph=``strongMorph:TH8792''. \bibverse{26} Viele
suchenx-morph=``strongMorph:TH8764'' das Angesicht eines
Fürstenx-morph=``strongMorph:TH8802''; aber eines jeglichen Gericht
kommt vom HERRN. \bibverse{27} Ein ungerechter Mann ist dem Gerechten
ein Greuel; und wer rechtes Weges ist, der ist des Gottlosen Greuel.

\hypertarget{section-29}{%
\section{30}\label{section-29}}

\bibverse{1} Dies sind die Worte Agurs, des Sohnes Jakes. Lehre und
Redex-morph=``strongMorph:TH8803'' des Mannes: Ich habe mich gemüht, o
Gott; ich habe mich gemüht, o Gott, und ablassen müssen. \bibverse{2}
Denn ich bin der allernärrischste, und Menschenverstand ist nicht bei
mir; \bibverse{3} ich habe Weisheit nicht
gelerntx-morph=``strongMorph:TH8804'', daß ich den Heiligen
erkennetex-morph=``strongMorph:TH8799''. \bibverse{4} Wer fährt
hinaufx-morph=``strongMorph:TH8804'' gen Himmel und
herabx-morph=``strongMorph:TH8799''? Wer
faßtx-morph=``strongMorph:TH8804'' den Wind in seine Hände? Wer
bindetx-morph=``strongMorph:TH8804'' die Wasser in ein Kleid? Wer hat
alle Enden der Welt gestelltx-morph=``strongMorph:TH8689''? Wie heißt
er? Und wie heißt sein Sohn? Weißt du dasx-morph=``strongMorph:TH8799''?
\bibverse{5} Alle Worte Gottes sind
durchläutertx-morph=``strongMorph:TH8803''; er ist ein Schild denen, die
auf ihn trauenx-morph=``strongMorph:TH8802''. \bibverse{6} Tue nichts
zux-morph=``strongMorph:TH8686'' seinen Worten, daß er dich nicht
strafex-morph=``strongMorph:TH8686'' und werdest lügenhaft
erfundenx-morph=``strongMorph:TH8738''. \bibverse{7} Zweierlei bitte ich
von dirx-morph=``strongMorph:TH8804''; das wollest du mir nicht
weigernx-morph=``strongMorph:TH8799'', ehe ich denn
sterbex-morph=``strongMorph:TH8799'': \bibverse{8} Abgötterei und Lüge
laß fernex-morph=``strongMorph:TH8685'' von mir sein; Armut und Reichtum
gibx-morph=``strongMorph:TH8799'' mir
nichtx-morph=``strongMorph:TH8685'', laß mich aber mein bescheiden Teil
Speise dahinnehmen. \bibverse{9} Ich möchte sonst, wo ich zu satt
würdex-morph=``strongMorph:TH8799'',
verleugnenx-morph=``strongMorph:TH8765'' und
sagenx-morph=``strongMorph:TH8804'': Wer ist der HERR? Oder wo ich zu
arm würdex-morph=``strongMorph:TH8735'', möchte ich
stehlenx-morph=``strongMorph:TH8804'' und mich an dem Namen meines
Gottes vergreifenx-morph=``strongMorph:TH8804''. \bibverse{10}
Verleumdex-morph=``strongMorph:TH8686'' den Knecht nicht bei seinem
Herrn, daß er dir nicht fluchex-morph=``strongMorph:TH8762'' und du die
Schuld tragen müssestx-morph=``strongMorph:TH8804''. \bibverse{11} Es
ist eine Art, die ihrem Vater fluchtx-morph=``strongMorph:TH8762'' und
ihre Mutter nicht segnetx-morph=``strongMorph:TH8762''; \bibverse{12}
eine Art, die sich rein dünkt, und ist doch von ihrem Kot nicht
gewaschenx-morph=``strongMorph:TH8795''; \bibverse{13} eine Art, die
ihre Augen hoch trägtx-morph=``strongMorph:TH8804'' und ihre Augenlider
emporhältx-morph=``strongMorph:TH8735''; \bibverse{14} eine Art, die
Schwerter für Zähne hat und Messer für Backenzähne und
verzehrtx-morph=``strongMorph:TH8800'' die Elenden im Lande und die
Armen unter den Leuten. \bibverse{15} Blutegel hat zwei Töchter: Bring
herx-morph=``strongMorph:TH8798'', bring
herx-morph=``strongMorph:TH8798''! Drei Dinge sind nicht zu
sättigenx-morph=``strongMorph:TH8799'', und das vierte
sprichtx-morph=``strongMorph:TH8804'' nicht: Es ist genug: \bibverse{16}
die Hölle, der Frauen verschlossenen Mutter, die Erde wird nicht des
Wassers sattx-morph=``strongMorph:TH8804'', und das Feuer
sprichtx-morph=``strongMorph:TH8804'' nicht: Es ist genug. \bibverse{17}
Ein Auge, das den Vater verspottetx-morph=``strongMorph:TH8799'', und
verachtetx-morph=``strongMorph:TH8799'' der Mutter zu gehorchen, das
müssen die Raben am Bach aushackenx-morph=``strongMorph:TH8799'' und die
jungen Adler fressenx-morph=``strongMorph:TH8799''. \bibverse{18} Drei
sind mir zu wunderbarx-morph=``strongMorph:TH8738'', und das vierte
verstehe ich nichtx-morph=``strongMorph:TH8804'': \bibverse{19} des
Adlers Weg am Himmel, der Schlange Weg auf einem Felsen, des Schiffes
Weg mitten im Meer und eines Mannes Weg an einer Jungfrau. \bibverse{20}
Also ist auch der Weg der Ehebrecherinx-morph=``strongMorph:TH8764'';
die verschlingtx-morph=``strongMorph:TH8804'' und
wischtx-morph=``strongMorph:TH8804'' ihr Maul und
sprichtx-morph=``strongMorph:TH8804'': Ich habe kein Böses
getanx-morph=``strongMorph:TH8804''. \bibverse{21} Ein Land wird durch
dreierlei unruhigx-morph=``strongMorph:TH8804'', und das vierte
kannx-morph=``strongMorph:TH8799'' es nicht
ertragenx-morph=``strongMorph:TH8800'': \bibverse{22} ein Knecht, wenn
er König wirdx-morph=``strongMorph:TH8799''; ein Narr, wenn er zu satt
istx-morph=``strongMorph:TH8799''; \bibverse{23} eine
Verschmähtex-morph=``strongMorph:TH8803'', wenn sie geehelicht
wirdx-morph=``strongMorph:TH8735''; und eine Magd, wenn sie ihrer Frau
Erbinx-morph=``strongMorph:TH8799'' wird. \bibverse{24} Vier sind klein
auf Erden und klüger denn die Weisenx-morph=``strongMorph:TH8794'':
\bibverse{25} die Ameisen, ein schwaches Volk; dennoch schaffen
siex-morph=``strongMorph:TH8686'' im Sommer ihre Speise, \bibverse{26}
Kaninchen, ein schwaches Volk; dennoch legt
esx-morph=``strongMorph:TH8799'' sein Haus in den Felsen, \bibverse{27}
Heuschrecken, haben keinen König; dennoch ziehen
siex-morph=``strongMorph:TH8799'' aus ganzx-morph=``strongMorph:TH8802''
in Haufen, \bibverse{28} die Spinne, wirktx-morph=``strongMorph:TH8762''
mit ihren Händen und ist in der Könige Schlössern. \bibverse{29}
Dreierlei haben einen feinenx-morph=``strongMorph:TH8688'' Gang, und das
vierte gehtx-morph=``strongMorph:TH8800''
wohlx-morph=``strongMorph:TH8688'': \bibverse{30} der Löwe, mächtig
unter den Tieren und kehrt nicht um vor
jemandx-morph=``strongMorph:TH8799''; \bibverse{31} ein Windhund von
guten Lenden, und ein Widder, und ein König, wider den sich niemand
legen darf. \bibverse{32} Bist du ein Narr
gewesenx-morph=``strongMorph:TH8804'' und zu hoch
gefahrenx-morph=``strongMorph:TH8692'' und hast Böses
vorgehabtx-morph=``strongMorph:TH8804'', so lege die Hand aufs Maul.
\bibverse{33} Wenn man Milch stößt, so macht man Butter
darausx-morph=``strongMorph:TH8686''; und wer die Nase hart schneuzt,
zwingt Blut herausx-morph=``strongMorph:TH8686''; und wer den Zorn
reizt, zwingt Hader herausx-morph=``strongMorph:TH8686''.

\hypertarget{section-30}{%
\section{31}\label{section-30}}

\bibverse{1} Dies sind die Worte des Königs Lamuel, die Lehre, die ihn
seine Mutter lehrtex-morph=``strongMorph:TH8765''. \bibverse{2} Ach mein
Auserwählter, ach du Sohn meines Leibes, ach mein gewünschter Sohn,
\bibverse{3} laßx-morph=``strongMorph:TH8799'' nicht den Weibern deine
Kraft und gehe die Wege nicht, darin sich die Könige
verderbenx-morph=``strongMorph:TH8687''! \bibverse{4} O, nicht den
Königen, Lamuel, nicht den Königen ziemt es, Wein zu
trinkenx-morph=``strongMorph:TH8800'',
nochx-morph=``strongMorph:TH8675'' den
Fürstenx-morph=``strongMorph:TH8802'' starkes Getränk! \bibverse{5} Sie
möchten trinkenx-morph=``strongMorph:TH8799'' und der
Rechtex-morph=``strongMorph:TH8794''
vergessenx-morph=``strongMorph:TH8799'' und
verändernx-morph=``strongMorph:TH8762'' die Sache aller elenden Leute.
\bibverse{6} Gebtx-morph=``strongMorph:TH8798'' starkes Getränk denen,
die am Umkommen sindx-morph=``strongMorph:TH8802'', und den Wein den
betrübten Seelen, \bibverse{7} daß sie
trinkenx-morph=``strongMorph:TH8799'' und ihres Elends
vergessenx-morph=``strongMorph:TH8799'' und ihres Unglücks nicht mehr
gedenkenx-morph=``strongMorph:TH8799''. \bibverse{8} Tue deinen Mund
aufx-morph=``strongMorph:TH8798'' für die Stummen und für die Sache
aller, die verlassen sind. \bibverse{9} Tue deinen Mund
aufx-morph=``strongMorph:TH8798'' und
richtex-morph=``strongMorph:TH8798'' recht und
rächex-morph=``strongMorph:TH8798'' den Elenden und Armen. \bibverse{10}
Wem ein tugendsam Weib beschert istx-morph=``strongMorph:TH8799'', die
ist viel edler denn die köstlichsten Perlen. \bibverse{11} Ihres Mannes
Herz darf sich auf sie verlassenx-morph=``strongMorph:TH8804'', und
Nahrung wird ihm nicht mangelnx-morph=``strongMorph:TH8799''.
\bibverse{12} Sie tutx-morph=``strongMorph:TH8804'' ihm Liebes und kein
Leides ihr Leben lang. \bibverse{13} Sie
gehtx-morph=``strongMorph:TH8804'' mit Wolle und Flachs um und
arbeitetx-morph=``strongMorph:TH8799'' gern mit ihren Händen.
\bibverse{14} Sie ist wie ein
Kaufmannsschiffx-morph=``strongMorph:TH8802'', das seine Nahrung von
ferne bringtx-morph=``strongMorph:TH8686''. \bibverse{15} Sie steht vor
Tages aufx-morph=``strongMorph:TH8799'' und
gibtx-morph=``strongMorph:TH8799'' Speise ihrem Hause und Essen ihren
Dirnen. \bibverse{16} Sie denkt nachx-morph=``strongMorph:TH8804'' einem
Acker und kauftx-morph=``strongMorph:TH8799'' ihn und
pflanztx-morph=``strongMorph:TH8804'' einen Weinberg von den Früchten
ihrer Hände. \bibverse{17} Sie gürtetx-morph=``strongMorph:TH8804'' ihre
Lenden mit Kraft und stärktx-morph=``strongMorph:TH8762'' ihre Arme.
\bibverse{18} Sie merktx-morph=``strongMorph:TH8804'', wie ihr Handel
Frommen bringt; ihre Leuchte verlischt des Nachts
nichtx-morph=``strongMorph:TH8799''. \bibverse{19} Sie streckt ihre Hand
nachx-morph=``strongMorph:TH8765'' dem Rocken, und ihre Finger
fassenx-morph=``strongMorph:TH8804'' die Spindel. \bibverse{20} Sie
breitetx-morph=``strongMorph:TH8804'' ihre Hände aus zu dem Armen und
reichtx-morph=``strongMorph:TH8765'' ihre Hand dem Dürftigen.
\bibverse{21} Sie fürchtetx-morph=``strongMorph:TH8799'' für ihr Haus
nicht den Schnee; denn ihr ganzes Haus hat zwiefache
Kleiderx-morph=``strongMorph:TH8803''. \bibverse{22} Sie
machtx-morph=``strongMorph:TH8804'' sich selbst Decken; feine Leinwand
und Purpur ist ihr Kleid. \bibverse{23} Ihr Mann ist
bekanntx-morph=``strongMorph:TH8737'' in den Toren, wenn er
sitztx-morph=``strongMorph:TH8800'' bei den Ältesten des Landes.
\bibverse{24} Sie machtx-morph=``strongMorph:TH8804'' einen Rock und
verkauft ihnx-morph=``strongMorph:TH8799''; einen Gürtel
gibtx-morph=``strongMorph:TH8804'' sie dem Krämer. \bibverse{25} Kraft
und Schöne sind ihr Gewand, und sie lachtx-morph=``strongMorph:TH8799''
des kommenden Tages. \bibverse{26} Sie tut ihren Mund
aufx-morph=``strongMorph:TH8804'' mit Weisheit, und auf ihrer Zunge ist
holdselige Lehre. \bibverse{27} Sie
schautx-morph=``strongMorph:TH8802'', wie es in ihrem Hause
zugehtx-morph=``strongMorph:TH8675'', und
ißtx-morph=``strongMorph:TH8799'' ihr Brot nicht mit Faulheit.
\bibverse{28} Ihre Söhne stehen aufx-morph=``strongMorph:TH8804'' und
preisen sie seligx-morph=``strongMorph:TH8762''; ihr Mann
lobtx-morph=``strongMorph:TH8762'' sie: \bibverse{29} ``Viele Töchter
haltenx-morph=''strongMorph:TH8804" sich tugendsam; du aber übertriffst
sie allex-morph=``strongMorph:TH8804''.'' \bibverse{30} Lieblich und
schön sein ist nichts; ein Weib, das den HERRN fürchtet, soll man
lobenx-morph=``strongMorph:TH8691''. \bibverse{31} Sie wird gerühmt
werdenx-morph=``strongMorph:TH8798'' von den Früchten ihrer Hände, und
ihre Werke werden sie lobenx-morph=``strongMorph:TH8762'' in den Toren.
