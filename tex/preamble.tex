\usepackage[hyperfootnotes=false]{hyperref}
\usepackage{pdfpages}
\usepackage{fontspec}
\usepackage{fancyhdr}
\usepackage[compact, bottomtitles]{titlesec}
\usepackage{alphalph}
\usepackage[para, perpage]{footmisc}
\usepackage{footnpag}
\usepackage{graphicx}
\usepackage{svg}
\usepackage{ragged2e}
\usepackage{titletoc}
\usepackage{ifthen}
\usepackage[toc]{multitoc}
\usepackage{etoolbox}
\usepackage{alphalph}
\usepackage{paracol}
\usepackage{comment}
\usepackage{titlesec}
\usepackage{xcolor}
%\usepackage{bidi}
\usepackage{multicol}

\extrafloats{100}
\maxdeadcycles=200
\renewcommand{\thefootnote}{\alphalph{\value{footnote}}}
\renewcommand*{\multicolumntoc}{2}
\patchcmd{\makefootnoteparagraph}
   {\columnwidth}{\textwidth}
   {\typeout{Success}}{\typeout{patch failed}}

\setmainfont{Bookerly}
            [UprightFont = *,
             BoldFont = * Bold,
             ItalicFont = * Italic,
             Ligatures=TeX]

\makeatletter             
\newlength\fake@f
\newlength\fake@c
\def\fakesc#1{%
  \begingroup%
  \xdef\fake@name{\csname\curr@fontshape/\f@size\endcsname}%
  \fontsize{\fontdimen8\fake@name}{\baselineskip}\selectfont%
  \uppercase{#1}%
  \endgroup%
}
\makeatother
\newcommand\fauxsc[1]{\fauxschelper#1 \relax\relax}
\def\fauxschelper#1 #2\relax{%
  \fauxschelphelp#1\relax\relax%
  \if\relax#2\relax\else\ \fauxschelper#2\relax\fi%
}
\def\Hscale{.83}\def\Vscale{.72}\def\Cscale{1.00}
\def\fauxschelphelp#1#2\relax{%
  \ifnum`#1=\lccode`#1\relax\scalebox{\Hscale}[\Vscale]{\char\uccode`#1}\else%
    \scalebox{\Cscale}[1]{#1}\fi%
  \ifx\relax#2\relax\else\fauxschelphelp#2\relax\fi}

\renewcommand{\textsc}[1]{{\fauxsc{#1}}}
\renewcommand{\emph}[1]{{\textit{#1}}}

\pagestyle{fancy}
\renewcommand{\sectionmark}[1]{\markright{\thesection~- ~#1}}
\renewcommand{\chaptermark}[1]{\markboth{\chaptername~\thechapter~-~ #1}{}}
\pagestyle{empty}

\titleformat{\chapter}[display]
  {\normalfont\bfseries}{}{0pt}{\Huge}

\fancyhf{}

\fancypagestyle{bible}{%
    \fancyhead{} %Clean headers
    \fancyhead[L]{\leftmark \ \thesection}
    \fancyhead[R]{\thepage}
    \fancyhead[RO]{\leftmark \ \thesection}
    \fancyhead[LO]{\thepage}
}

\fancypagestyle{plain}{%
  \fancyhead{}
  \fancyhead[R, RO]{\thepage}
}

\renewcommand{\chaptermark}[1]{\markboth{\thechapter. {\slshape{##1}}}{}}
\renewcommand{\headrulewidth}{0pt}
% \renewcommand*{\thefootnote}{\alph{footnote}}
\titleformat{\chapter}[hang]{\normalfont\bfseries}{}{0pt}{\Huge}
\titleformat{\section}[wrap]{\bfseries\huge}{}{0ex}{}[]
\titleformat{\subsection}[hang]{\bfseries}{}{0ex}{}[]
\newcommand{\bibleverse}[1]{{\bfseries{#1}}}
\renewcommand{\thesection}{\arabic{section}}
\renewcommand{\chaptermark}[1]{\markboth{\MakeUppercase{#1}}{}}
\setcounter{secnumdepth}{1}
\setcounter{tocdepth}{0}
\global\let\endtitlepage\relax
\makeatletter

\titlecontents{chapter}% <section-type>
[0pt]% <left>
{\bfseries\small}% <above-code>
{\small\thecontentslabel \quad}%<numbered-entry-format>
{}% <numberless-entry-format>
{\small\mdseries\titlerule*[0.75em]{.}\bfseries\contentspage}

\defaultfontfeatures{Scale=MatchLowercase} %so that different fonts have same xheight
\newfontfamily\hebrewfont[Script=Hebrew]{Frank Ruehl CLM}
\newfontfamily\greekfont[Script=Greek]{Cardo}
\newcommand{\hebrew}[1]{{\hebrewfont{#1}}}
\newcommand{\greek}[1]{{\greekfont{#1}}}

\newcommand{\biblebeginparacol}{\begin{paracol}{2}}
\newcommand{\bibleendparacol}{\end{paracol}}